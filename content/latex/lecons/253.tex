\documentclass[12pt, a4paper]{report}

% LuaLaTeX :

\RequirePackage{iftex}
\RequireLuaTeX

% Packages :

\usepackage[french]{babel}
%\usepackage[utf8]{inputenc}
%\usepackage[T1]{fontenc}
\usepackage[pdfencoding=auto, pdfauthor={Hugo Delaunay}, pdfsubject={Mathématiques}, pdfcreator={agreg.skyost.eu}]{hyperref}
\usepackage{amsmath}
\usepackage{amsthm}
%\usepackage{amssymb}
\usepackage{stmaryrd}
\usepackage{tikz}
\usepackage{tkz-euclide}
\usepackage{fontspec}
\defaultfontfeatures[Erewhon]{FontFace = {bx}{n}{Erewhon-Bold.otf}}
\usepackage{fourier-otf}
\usepackage[nobottomtitles*]{titlesec}
\usepackage{fancyhdr}
\usepackage{listings}
\usepackage{catchfilebetweentags}
\usepackage[french, capitalise, noabbrev]{cleveref}
\usepackage[fit, breakall]{truncate}
\usepackage[top=2.5cm, right=2cm, bottom=2.5cm, left=2cm]{geometry}
\usepackage{enumitem}
\usepackage{tocloft}
\usepackage{microtype}
%\usepackage{mdframed}
%\usepackage{thmtools}
\usepackage{xcolor}
\usepackage{tabularx}
\usepackage{xltabular}
\usepackage{aligned-overset}
\usepackage[subpreambles=true]{standalone}
\usepackage{environ}
\usepackage[normalem]{ulem}
\usepackage{etoolbox}
\usepackage{setspace}
\usepackage[bibstyle=reading, citestyle=draft]{biblatex}
\usepackage{xpatch}
\usepackage[many, breakable]{tcolorbox}
\usepackage[backgroundcolor=white, bordercolor=white, textsize=scriptsize]{todonotes}
\usepackage{luacode}
\usepackage{float}
\usepackage{needspace}
\everymath{\displaystyle}

% Police :

\setmathfont{Erewhon Math}

% Tikz :

\usetikzlibrary{calc}
\usetikzlibrary{3d}

% Longueurs :

\setlength{\parindent}{0pt}
\setlength{\headheight}{15pt}
\setlength{\fboxsep}{0pt}
\titlespacing*{\chapter}{0pt}{-20pt}{10pt}
\setlength{\marginparwidth}{1.5cm}
\setstretch{1.1}

% Métadonnées :

\author{agreg.skyost.eu}
\date{\today}

% Titres :

\setcounter{secnumdepth}{3}

\renewcommand{\thechapter}{\Roman{chapter}}
\renewcommand{\thesubsection}{\Roman{subsection}}
\renewcommand{\thesubsubsection}{\arabic{subsubsection}}
\renewcommand{\theparagraph}{\alph{paragraph}}

\titleformat{\chapter}{\huge\bfseries}{\thechapter}{20pt}{\huge\bfseries}
\titleformat*{\section}{\LARGE\bfseries}
\titleformat{\subsection}{\Large\bfseries}{\thesubsection \, - \,}{0pt}{\Large\bfseries}
\titleformat{\subsubsection}{\large\bfseries}{\thesubsubsection. \,}{0pt}{\large\bfseries}
\titleformat{\paragraph}{\bfseries}{\theparagraph. \,}{0pt}{\bfseries}

\setcounter{secnumdepth}{4}

% Table des matières :

\renewcommand{\cftsecleader}{\cftdotfill{\cftdotsep}}
\addtolength{\cftsecnumwidth}{10pt}

% Redéfinition des commandes :

\renewcommand*\thesection{\arabic{section}}
\renewcommand{\ker}{\mathrm{Ker}}

% Nouvelles commandes :

\newcommand{\website}{https://github.com/imbodj/SenCoursDeMaths}

\newcommand{\tr}[1]{\mathstrut ^t #1}
\newcommand{\im}{\mathrm{Im}}
\newcommand{\rang}{\operatorname{rang}}
\newcommand{\trace}{\operatorname{trace}}
\newcommand{\id}{\operatorname{id}}
\newcommand{\stab}{\operatorname{Stab}}
\newcommand{\paren}[1]{\left(#1\right)}
\newcommand{\croch}[1]{\left[ #1 \right]}
\newcommand{\Grdcroch}[1]{\Bigl[ #1 \Bigr]}
\newcommand{\grdcroch}[1]{\bigl[ #1 \bigr]}
\newcommand{\abs}[1]{\left\lvert #1 \right\rvert}
\newcommand{\limi}[3]{\lim_{#1\to #2}#3}
\newcommand{\pinf}{+\infty}
\newcommand{\minf}{-\infty}
%%%%%%%%%%%%%% ENSEMBLES %%%%%%%%%%%%%%%%%
\newcommand{\ensemblenombre}[1]{\mathbb{#1}}
\newcommand{\Nn}{\ensemblenombre{N}}
\newcommand{\Zz}{\ensemblenombre{Z}}
\newcommand{\Qq}{\ensemblenombre{Q}}
\newcommand{\Qqp}{\Qq^+}
\newcommand{\Rr}{\ensemblenombre{R}}
\newcommand{\Cc}{\ensemblenombre{C}}
\newcommand{\Nne}{\Nn^*}
\newcommand{\Zze}{\Zz^*}
\newcommand{\Zzn}{\Zz^-}
\newcommand{\Qqe}{\Qq^*}
\newcommand{\Rre}{\Rr^*}
\newcommand{\Rrp}{\Rr_+}
\newcommand{\Rrm}{\Rr_-}
\newcommand{\Rrep}{\Rr_+^*}
\newcommand{\Rrem}{\Rr_-^*}
\newcommand{\Cce}{\Cc^*}
%%%%%%%%%%%%%%  INTERVALLES %%%%%%%%%%%%%%%%%
\newcommand{\intff}[2]{\left[#1\;,\; #2\right]  }
\newcommand{\intof}[2]{\left]#1 \;, \;#2\right]  }
\newcommand{\intfo}[2]{\left[#1 \;,\; #2\right[  }
\newcommand{\intoo}[2]{\left]#1 \;,\; #2\right[  }

\providecommand{\newpar}{\\[\medskipamount]}

\newcommand{\annexessection}{%
  \newpage%
  \subsection*{Annexes}%
}

\providecommand{\lesson}[3]{%
  \title{#3}%
  \hypersetup{pdftitle={#2 : #3}}%
  \setcounter{section}{\numexpr #2 - 1}%
  \section{#3}%
  \fancyhead[R]{\truncate{0.73\textwidth}{#2 : #3}}%
}

\providecommand{\development}[3]{%
  \title{#3}%
  \hypersetup{pdftitle={#3}}%
  \section*{#3}%
  \fancyhead[R]{\truncate{0.73\textwidth}{#3}}%
}

\providecommand{\sheet}[3]{\development{#1}{#2}{#3}}

\providecommand{\ranking}[1]{%
  \title{Terminale #1}%
  \hypersetup{pdftitle={Terminale #1}}%
  \section*{Terminale #1}%
  \fancyhead[R]{\truncate{0.73\textwidth}{Terminale #1}}%
}

\providecommand{\summary}[1]{%
  \textit{#1}%
  \par%
  \medskip%
}

\tikzset{notestyleraw/.append style={inner sep=0pt, rounded corners=0pt, align=center}}

%\newcommand{\booklink}[1]{\website/bibliographie\##1}
\newcounter{reference}
\newcommand{\previousreference}{}
\providecommand{\reference}[2][]{%
  \needspace{20pt}%
  \notblank{#1}{
    \needspace{20pt}%
    \renewcommand{\previousreference}{#1}%
    \stepcounter{reference}%
    \label{reference-\previousreference-\thereference}%
  }{}%
  \todo[noline]{%
    \protect\vspace{20pt}%
    \protect\par%
    \protect\notblank{#1}{\cite{[\previousreference]}\\}{}%
    \protect\hyperref[reference-\previousreference-\thereference]{p. #2}%
  }%
}

\definecolor{devcolor}{HTML}{00695c}
\providecommand{\dev}[1]{%
  \reversemarginpar%
  \todo[noline]{
    \protect\vspace{20pt}%
    \protect\par%
    \bfseries\color{devcolor}\href{\website/developpements/#1}{[DEV]}
  }%
  \normalmarginpar%
}

% En-têtes :

\pagestyle{fancy}
\fancyhead[L]{\truncate{0.23\textwidth}{\thepage}}
\fancyfoot[C]{\scriptsize \href{\website}{\texttt{https://github.com/imbodj/SenCoursDeMaths}}}

% Couleurs :

\definecolor{property}{HTML}{ffeb3b}
\definecolor{proposition}{HTML}{ffc107}
\definecolor{lemma}{HTML}{ff9800}
\definecolor{theorem}{HTML}{f44336}
\definecolor{corollary}{HTML}{e91e63}
\definecolor{definition}{HTML}{673ab7}
\definecolor{notation}{HTML}{9c27b0}
\definecolor{example}{HTML}{00bcd4}
\definecolor{cexample}{HTML}{795548}
\definecolor{application}{HTML}{009688}
\definecolor{remark}{HTML}{3f51b5}
\definecolor{algorithm}{HTML}{607d8b}
%\definecolor{proof}{HTML}{e1f5fe}
\definecolor{exercice}{HTML}{e1f5fe}

% Théorèmes :

\theoremstyle{definition}
\newtheorem{theorem}{Théorème}

\newtheorem{property}[theorem]{Propriété}
\newtheorem{proposition}[theorem]{Proposition}
\newtheorem{lemma}[theorem]{Activité d'introduction}
\newtheorem{corollary}[theorem]{Conséquence}

\newtheorem{definition}[theorem]{Définition}
\newtheorem{notation}[theorem]{Notation}

\newtheorem{example}[theorem]{Exemple}
\newtheorem{cexample}[theorem]{Contre-exemple}
\newtheorem{application}[theorem]{Application}

\newtheorem{algorithm}[theorem]{Algorithme}
\newtheorem{exercice}[theorem]{Exercice}

\theoremstyle{remark}
\newtheorem{remark}[theorem]{Remarque}

\counterwithin*{theorem}{section}

\newcommand{\applystyletotheorem}[1]{
  \tcolorboxenvironment{#1}{
    enhanced,
    breakable,
    colback=#1!8!white,
    %right=0pt,
    %top=8pt,
    %bottom=8pt,
    boxrule=0pt,
    frame hidden,
    sharp corners,
    enhanced,borderline west={4pt}{0pt}{#1},
    %interior hidden,
    sharp corners,
    after=\par,
  }
}

\applystyletotheorem{property}
\applystyletotheorem{proposition}
\applystyletotheorem{lemma}
\applystyletotheorem{theorem}
\applystyletotheorem{corollary}
\applystyletotheorem{definition}
\applystyletotheorem{notation}
\applystyletotheorem{example}
\applystyletotheorem{cexample}
\applystyletotheorem{application}
\applystyletotheorem{remark}
%\applystyletotheorem{proof}
\applystyletotheorem{algorithm}
\applystyletotheorem{exercice}

% Environnements :

\NewEnviron{whitetabularx}[1]{%
  \renewcommand{\arraystretch}{2.5}
  \colorbox{white}{%
    \begin{tabularx}{\textwidth}{#1}%
      \BODY%
    \end{tabularx}%
  }%
}

% Maths :

\DeclareFontEncoding{FMS}{}{}
\DeclareFontSubstitution{FMS}{futm}{m}{n}
\DeclareFontEncoding{FMX}{}{}
\DeclareFontSubstitution{FMX}{futm}{m}{n}
\DeclareSymbolFont{fouriersymbols}{FMS}{futm}{m}{n}
\DeclareSymbolFont{fourierlargesymbols}{FMX}{futm}{m}{n}
\DeclareMathDelimiter{\VERT}{\mathord}{fouriersymbols}{152}{fourierlargesymbols}{147}

% Code :

\definecolor{greencode}{rgb}{0,0.6,0}
\definecolor{graycode}{rgb}{0.5,0.5,0.5}
\definecolor{mauvecode}{rgb}{0.58,0,0.82}
\definecolor{bluecode}{HTML}{1976d2}
\lstset{
  basicstyle=\footnotesize\ttfamily,
  breakatwhitespace=false,
  breaklines=true,
  %captionpos=b,
  commentstyle=\color{greencode},
  deletekeywords={...},
  escapeinside={\%*}{*)},
  extendedchars=true,
  frame=none,
  keepspaces=true,
  keywordstyle=\color{bluecode},
  language=Python,
  otherkeywords={*,...},
  numbers=left,
  numbersep=5pt,
  numberstyle=\tiny\color{graycode},
  rulecolor=\color{black},
  showspaces=false,
  showstringspaces=false,
  showtabs=false,
  stepnumber=2,
  stringstyle=\color{mauvecode},
  tabsize=2,
  %texcl=true,
  xleftmargin=10pt,
  %title=\lstname
}

\newcommand{\codedirectory}{}
\newcommand{\inputalgorithm}[1]{%
  \begin{algorithm}%
    \strut%
    \lstinputlisting{\codedirectory#1}%
  \end{algorithm}%
}




\begin{document}
  %<*content>
  \lesson{analysis}{253}{Utilisation de la notion de convexité en analyse.}

  Soit $E$ un espace vectoriel sur $\mathbb{R}$ ou $\mathbb{C}$.

  \subsection{Convexité d'une fonction, d'un ensemble}

  \subsubsection{Ensembles convexes}

  \paragraph{Généralités}

  \reference[GOU21]{51}

  \begin{definition}
    \begin{itemize}
      \item Soient $a, b \in E$. On appelle \textbf{segment} d'extrémités $a$ et $b$, l'ensemble
      \[ [a,b] = \{ ta + (1-t)b \mid t \in [0,1] \} \]
      \item On dit qu'une partie $C$ de $E$ est \textbf{convexe} si
      \[ \forall a, b \in E, \, [a,b] \subseteq E \]
    \end{itemize}
  \end{definition}

  \begin{example}
    Un sous-espace vectoriel de $E$ est convexe.
  \end{example}

  \begin{remark}
    Une partie convexe est connexe.
  \end{remark}

  \reference[BMP]{26}

  \begin{proposition}
    \begin{enumerate}[label=(\roman*)]
      \item Dans $\mathbb{R}$, les intervalles sont à la fois les parties connexes et convexes.
      \item Une intersection de parties convexes est convexe.
    \end{enumerate}
  \end{proposition}

  \paragraph{Enveloppes convexes}

  \reference[GOU21]{51}

  \begin{definition}
    Soit $A \subseteq E$. On appelle \textbf{enveloppe convexe} de $A$ le plus petit (au sens de l'inclusion) convexe contenant $A$. On la note $\operatorname{Conv}(A)$.
  \end{definition}

  \begin{proposition}
    Soit $A \subseteq E$. Alors,
    \[ x \in \operatorname{Conv}(A) \iff x = \sum_{i=1}^{n} \lambda_i x_i \text{ avec } x_1, \dots, x_n \in A \text{ et } \lambda_1, \dots, \lambda_n \in \mathbb{R}^+ \text{ tels que } \sum_{i=1}^n \lambda_i = 1 \]
  \end{proposition}

  \reference{54}

  \begin{theorem}[Carathéodory]
    Soit $A \subseteq E$. On suppose que $E$ est un espace vectoriel normé de dimension finie $n$. Alors, tout élément de $\operatorname{Conv}(A)$ est combinaison convexe de $n+1$ éléments de $A$.
  \end{theorem}

  \begin{application}
    Soit $A \subseteq E$ compact. On suppose que $E$ est un espace vectoriel normé de dimension finie. Alors $\operatorname{Conv}(A)$ est compacte.
  \end{application}

  \begin{proposition}
    On suppose que $E$ est un espace vectoriel normé. Alors, pour toute partie convexe $C$ de $E$, $\overline{C}$ et $\mathring{C}$ sont convexes.
  \end{proposition}

  \reference[BMP]{97}

  \begin{theorem}[Hahn-Banach géométrique]
    On se place dans le cas où $E$ est un espace de Hilbert sur $\mathbb{R}$. Soit $C$ une partie de $E$ convexe compacte. Alors, si $x \notin C$, il existe $f \in E'$ et $\alpha \in \mathbb{R}$ tels que
    \[ \forall y \in C, \, f(x) < \alpha < f(y) \]
  \end{theorem}

  \reference{133}

  \begin{corollary}
    On se place dans le cas où $E$ est un espace de Hilbert sur $\mathbb{R}$. Soit $A \subseteq E$. Alors,
    \[ x \in \overline{\operatorname{Conv}(A)} \iff \forall f \in H', \, f(x) \leq \sup_{y \in A} f(y) \]
  \end{corollary}

  \subsubsection{Fonctions convexes}

  On munit $E$ d'une norme $\Vert . \Vert$. Soit $I$ une partie convexe de $E$.

  \reference[ROM19-1]{225}

  \begin{definition}
    \begin{itemize}
      \item Une fonction $f : I \rightarrow \mathbb{R}$ est \textbf{convexe} si
      \[ \forall x, y \in I, \, \forall t \in [0,1], \, f((1-t)x + ty) \leq (1-t)f(x) + tf(y) \]
      \item Une fonction $f : I \rightarrow \mathbb{R}$ est \textbf{concave} si $-f$ est convexe.
    \end{itemize}
  \end{definition}

  \begin{remark}
    Les définitions de $f$ \textbf{strictement convexe} et $f$ \textbf{strictement concave} s'obtiennent en remplaçant les inégalités larges par des inégalités strictes dans la définition précédente.
  \end{remark}

  \begin{example}
    \begin{itemize}
      \item $x \mapsto \Vert x \Vert$ est convexe sur $E$.
      \item $\exp$ est convexe sur $\mathbb{R}$.
    \end{itemize}
  \end{example}

  \begin{proposition}
    Une fonction $f : I \rightarrow \mathbb{R}$ est convexe si et seulement si son épigraphe est convexe dans $E \times \mathbb{R}$.
  \end{proposition}

  \begin{theorem}
    Une fonction $f : I \rightarrow \mathbb{R}$ est convexe si et seulement si $\forall x, y \in I$, $t \mapsto f((1-t)x + ty)$ est convexe sur $[0,1]$.
  \end{theorem}

  Ce dernier théorème justifie que l'étude des fonctions convexes se ramène à l'étude des fonctions convexes sur un intervalle réel.

  \begin{proposition}
    \begin{itemize}
      \item Une combinaison linéaire à coefficients positifs de fonctions convexes est convexe.
      \item La composée $\varphi \circ g$ d'une fonction convexe croissante $\varphi : J \rightarrow \mathbb{R}$ avec une fonction fonction convexe $g : I \rightarrow J$ est croissante.
      \item Une limite simple d'une suite de fonctions convexes est convexe.
    \end{itemize}
  \end{proposition}

  \subsubsection{Fonctions log-convexes}

  \reference{228}

  \begin{definition}
    On dit qu'une fonction $f : I \rightarrow \mathbb{R}^+_*$ est \textbf{log-convexe} si $\ln \circ f$ est convexe sur $I$.
  \end{definition}

  \begin{proposition}
    Une fonction log-convexe est convexe.
  \end{proposition}

  \begin{cexample}
    $x \mapsto x$ est convexe mais non log-convexe.
  \end{cexample}

  \begin{theorem}
    Pour une fonction $f : I \rightarrow \mathbb{R}^+_*$, les assertions suivantes sont équivalentes :
    \begin{enumerate}[label=(\roman*)]
      \item $f$ est log-convexe.
      \item $\forall \alpha > 0, \, x \mapsto \alpha^x f(x)$ est convexe.
      \item $\forall x, y \in I, \, \forall t \in [0,1], \, f((1-t)x + ty) \leq (f(x))^{1-t} (f(y))^t$.
      \item $\forall \alpha > 0, \, f^\alpha$ est convexe.
    \end{enumerate}
  \end{theorem}

  \reference{364}

  \begin{lemma}
    \label{253-1}
    La fonction $\Gamma$ définie pour tout $x > 0$ par $\Gamma(x) = \int_0^{+\infty} t^{x-1} e^{-t} \, \mathrm{d}t$ vérifie :
    \begin{enumerate}[label=(\roman*)]
      \item \label{253-2} $\forall x \in \mathbb{R}^+_*$, $\Gamma(x+1) = x\Gamma(x)$.
      \item \label{253-3} $\Gamma(1) = 1$.
      \item \label{253-4} $\Gamma$ est log-convexe sur $\mathbb{R}^+_*$.
    \end{enumerate}
  \end{lemma}

  \dev{caracterisation-reelle-de-gamma}

  \begin{theorem}[Bohr-Mollerup]
    Soit $f : \mathbb{R}^+_* \rightarrow \mathbb{R}^+$ vérifiant le \cref{253-2}, \cref{253-3} et \cref{253-4} du \cref{253-1}. Alors $f = \Gamma$.
  \end{theorem}

  \begin{remark}
    À la fin de la preuve, on obtient une formule due à Gauss :
    \[ \forall x \in ]0, 1], \Gamma(x) = \lim_{n \rightarrow +\infty} \frac{n^x n!}{(x+n) \dots (x+1)x} \]
    que l'on peut aisément étendre à $\mathbb{R}^+_*$ entier.
  \end{remark}

  \subsection{Inégalités de convexité}

  \subsubsection{Inégalités pour des familles de réels}

  \reference[GOU20]{97}

  \begin{proposition}[Inégalité de Hölder]
    Soient $p, q > 0$ tels que $\frac{1}{p} + \frac{1}{q} = 1$. Alors,
    \[ \forall a_1, \dots, a_n, b_1, \dots, b_n \geq 0, \, \sum_{i=1}^n a_i b_i \leq \left( \sum_{i=1}^n a_i^p \right)^{\frac{1}{p}} \left( \sum_{i=1}^n b_i^q \right)^{\frac{1}{q}} \]
  \end{proposition}

  \begin{proposition}[Inégalité de Minkowski]
    Soit $p \geq 1$. Alors,
    \[ \forall x_1, \dots, x_n, y_1, \dots, y_n \geq 0, \, \left( \sum_{i=1}^n |x_i + y_i|^p \right)^{\frac{1}{p}} \leq \left( \sum_{i=1}^n x_i^p \right)^{\frac{1}{p}} \left( \sum_{i=1}^n y_i^p \right)^{\frac{1}{p}} \]
  \end{proposition}

  \reference[ROM19-1]{242}

  \begin{proposition}[Comparaison des moyennes harmonique, géométrique et arithmétique]
    Pour toute suite finie $x = (x_i)$ de $n$ réels strictement positifs, on a :
    \[ \frac{n}{\sum_{i=1}^n \frac{1}{x_i}} \leq \left( \prod_{i=1}^n x_i \right)^{\frac{1}{n}} \leq \frac{1}{n} \sum_{i=1}^n x_i \]
  \end{proposition}

  \subsubsection{Inégalités en théorie de l'intégration}

  \begin{proposition}[Inégalité de Jensen]
    Si $f : \mathbb{R} \rightarrow \mathbb{R}$ est convexe, alors pour toute fonction $u$ continue sur un intervalle $[a, b]$, on a :
    \[ f \left( \frac{1}{b-a} \int_a^b u(t) \, \mathrm{d}t \right) \leq \frac{1}{b-a} \int_a^b f \circ u (t) \, \mathrm{d}t \]
  \end{proposition}

  \reference[G-K]{209}

  \begin{theorem}[Inégalité de Hölder]
    Soient $p, q \in ]1, +\infty[$ tels que $\frac{1}{p} + \frac{1}{q} = 1$, $f \in \mathcal{L}_p$ et $g \in \mathcal{L}_q$. Alors $fg \in \mathcal{L}_1$ et
    \[ \Vert fg \Vert_1 \leq \Vert f \Vert_p \Vert g \Vert_q \]
  \end{theorem}

  \begin{remark}
    C'est encore vrai pour $q = +\infty$ en convenant que $\frac{1}{+\infty} = 0$.
  \end{remark}

  \begin{application}
    Dans un espace de mesure finie,
    \[ 1 \leq p < q \leq +\infty \implies L_q \subseteq L_p \]
  \end{application}

  \begin{theorem}[Inégalité de Minkowski]
    \[ \forall f, g \in \mathcal{L}_p, \, \Vert f + g \Vert_p \leq \Vert f \Vert_p + \Vert g \Vert_p \]
  \end{theorem}

  \subsection{Convexité et optimisation}

  \subsubsection{Pour les fonctions convexes}

  \reference[ROM19-1]{234}

  Soit $I \subseteq \mathbb{R}$ un intervalle réel non réduit à un point.

  \begin{proposition}
    Une fonction $f : \mathbb{R} \rightarrow \mathbb{R}$ est constante si et seulement si elle est convexe et majorée.
  \end{proposition}

  \begin{cexample}
    La fonction $f$ définie sur $\mathbb{R}^+$ par $f(x) = \frac{1}{1+x}$ est convexe, majorée, mais non constante.
  \end{cexample}

  \begin{proposition}
    Si $f : I \rightarrow \mathbb{R}$ est convexe et est dérivable en un point $\alpha \in \mathring{I}$ tel que $f'(\alpha) = 0$, alors $f$ admet un minimum global en $\alpha$.
  \end{proposition}

  \begin{proposition}
    Si $f : I \rightarrow \mathbb{R}$ est convexe et admet un minimum local, alors ce minimum est global.
  \end{proposition}

  \subsubsection{Dans un espace de Hilbert}

  \reference[LI]{32}

  Pour toute la suite, on fixe $H$ un espace de Hilbert de norme $\Vert . \Vert$ et on note $\langle ., . \rangle$ le produit scalaire associé.

  \begin{lemma}[Identité du parallélogramme]
    \[ \forall x, y \in H, \, \Vert x + y \Vert^2 + \Vert x - y \Vert^2 = 2(\Vert x \Vert^2 \Vert y \Vert^2) \]
    et cette identité caractérise les normes issues d'un produit scalaire.
  \end{lemma}

  \dev{projection-sur-un-convexe-ferme}

  \begin{theorem}[Projection sur un convexe fermé]
    Soit $C \subseteq H$ un convexe fermé non-vide. Alors :
    \[ \forall x \in H, \exists! y \in C \text{ tel que } d(x, C) = \inf_{z \in C} \Vert x - z \Vert = d(x, y) \]
    On peut donc noter $y = P_C(x)$, le \textbf{projeté orthogonal de $x$ sur $C$}. Il s'agit de l'unique point de $C$ vérifiant
    \[ \forall z \in C, \, \langle x - P_C(x), z - P_C(x) \rangle \leq 0 \]
  \end{theorem}

  \begin{theorem}
    Si $F$ est un sous espace vectoriel fermé dans $H$, alors $P_F$ est une application linéaire continue. De plus, pour tout $x \in H$, $P_F(x)$ est l'unique point $y \in F$ tel que $x-y \in F^\perp$.
  \end{theorem}

  \begin{theorem}
    Si $F$ est un sous espace vectoriel fermé dans $H$, alors
    \[ H = F \oplus F^\perp \]
    et $P_F$ est la projection sur $F$ parallèlement à $F^\perp$ : c'est la \textbf{projection orthogonale} sur $F$.
  \end{theorem}

  \begin{corollary}
    Soit $F$ un sous-espace vectoriel de $H$. Alors,
    \[ \overline{F} = H \iff F^\perp = 0 \]
  \end{corollary}

  \begin{theorem}[de représentation de Riesz]
    \[ \forall \varphi \in H', \, \exists! y \in H, \text{ tel que } \forall x \in H, \, \varphi(x) = \langle x, y \rangle \]
    et de plus, $\VERT \varphi \VERT = \Vert y \Vert$.
  \end{theorem}

  \begin{corollary}
    \[ \forall T \in H', \, \exists! U \in H' \text{ tel que } \forall x, y \in H, \, \langle T(x), y \rangle = \langle x, U(y) \rangle \]
    On note alors $U = T^*$ : c'est \textbf{l'adjoint} de $T$. On a alors $\VERT T \VERT = \VERT T^* \VERT$.
  \end{corollary}

  \reference[Z-Q]{222}

  \begin{application}
    L'application
    \[
    \varphi :
    \begin{array}{ll}
      L_q &\rightarrow (L_p)' \\
      g &\mapsto \left( \varphi_g : f \mapsto \int_X f g \, \mathrm{d}\mu \right)
    \end{array}
    \qquad \text{ où } \frac{1}{p} + \frac{1}{q} = 1
    \]
    est une isométrie linéaire surjective. C'est donc un isomorphisme isométrique.
  \end{application}
  %</content>
\end{document}
