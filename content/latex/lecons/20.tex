\input{../common}

\begin{document}
  %<*content>
  \lesson{analysis}{20}{Composition  de  deux applications.}
  
   \begin{lemma}
    On considère les deux applications suivantes définies par : $ f(x)=2x+1 $  et  $ g(x)=\sqrt{x} $.
  \begin{itemize}
\item  Précise leurs ensembles de définition.
 \item Calcule $ f(4) $  puis $ g(9) $.
 \item Calcule $ f(40) $  puis $ g(f(40)) $.
 \item Calcule $ f(0) $  puis $ g(f(0)) $.
 \item Calcule  $ g(f(3)) $.
 \item Calcule $ f(a) $  puis donne l'expression de $ g(f(a)) $  en fonction du  nombre positif $ a $.
 \end{itemize}
 \end{lemma}
\subsection*{Définition - Notation - Exemples}

  \begin{definition}
  Soient  deux applications  $f $ et $g$ .
 
 On appelle composée de $ g $  par $ f $, l'application notée $ g\circ f $ définie par:
  \[\paren{g\circ f}(x)=g\croch{f(x)}\]
 \end{definition}
 
 \begin{methode}
 Pour déterminer l’expression de $\paren{g \circ f}(x)$ :

\begin{itemize}
  \item on cherche d’abord l’image de $x$ par $f$, c’est-à-dire $f(x)$ ;
  \item puis l’image de $f(x)$ par $g$, c’est-à-dire $g(f(x))$ ;
  \item on écrit alors : \[ g \circ f(x) = g[f(x)]. \]
\end{itemize}

 \end{methode}
 Dans les exemples suivants, déterminons l'expression des applications  $ g\circ f $ et de $ f\circ g $.
  \begin{example}
  Soit $ f(x)= 2x+1 $  et $ g(x)=\sqrt{x} $

 
 On a $ \paren{g\circ f}(x)=g\croch{f(x)}= g\paren{2x+1}=\sqrt{2x+1} $
 


  On a aussi $ \paren{f\circ g}(x)=f\croch{g(x)}= f\croch{\sqrt{x}}=2\sqrt{x}+1 $
  \end{example}

  \begin{example}
 
 Soit $ f(x)= x-5 $  et $ g(x)=x^{2} $
 
 
 On a $ \paren{g\circ f}(x)=g\croch{f(x)}= g\paren{x-5}=\paren{x-5}^{2} $
 


  On a aussi $ \paren{f\circ g}(x)=f\croch{g(x)}= f\croch{x^{2}}=x^{2}-5 $
 \end{example}

 
 \begin{example}
 
 Soit $ f(x)= \dfrac{x+4}{x} $  et $ g(x)=\sqrt{x} $

Alors \;  $ \paren{g\circ f}(x)=g\croch{f(x)}= \sqrt{\dfrac{x+4}{x}} $
  \end{example}
 
 \begin{remark}
  En général $ \paren{g\circ f} \neq  \paren{f\circ g}$
 \end{remark}
 \begin{example}[Reconnaissance]
 
 L'application $h$ définie sur $\mathbb{R}$ par $h(x) = (x - 2)^2$ est la composée de l'application
 \[ \begin{array}{lrcl}
   f  : & \mathbb{R}  &   \longrightarrow & \mathbb{R} \\ 
  &  x & \longmapsto & x^2
\end{array}\]
par l'application
 \[ \begin{array}{lrcl}
   g  : & \mathbb{R}  &   \longrightarrow & \mathbb{R} \\ 
  &  x & \longmapsto & x-2
\end{array}\]
Ainsi, on a : 
\[
h(x) = f[g(x)] = (f \circ g)(x).
\]
 \end{example}
  %</content>
\end{document}
