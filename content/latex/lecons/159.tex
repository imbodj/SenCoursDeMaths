\documentclass[12pt, a4paper]{report}

% LuaLaTeX :

\RequirePackage{iftex}
\RequireLuaTeX

% Packages :

\usepackage[french]{babel}
%\usepackage[utf8]{inputenc}
%\usepackage[T1]{fontenc}
\usepackage[pdfencoding=auto, pdfauthor={Hugo Delaunay}, pdfsubject={Mathématiques}, pdfcreator={agreg.skyost.eu}]{hyperref}
\usepackage{amsmath}
\usepackage{amsthm}
%\usepackage{amssymb}
\usepackage{stmaryrd}
\usepackage{tikz}
\usepackage{tkz-euclide}
\usepackage{fontspec}
\defaultfontfeatures[Erewhon]{FontFace = {bx}{n}{Erewhon-Bold.otf}}
\usepackage{fourier-otf}
\usepackage[nobottomtitles*]{titlesec}
\usepackage{fancyhdr}
\usepackage{listings}
\usepackage{catchfilebetweentags}
\usepackage[french, capitalise, noabbrev]{cleveref}
\usepackage[fit, breakall]{truncate}
\usepackage[top=2.5cm, right=2cm, bottom=2.5cm, left=2cm]{geometry}
\usepackage{enumitem}
\usepackage{tocloft}
\usepackage{microtype}
%\usepackage{mdframed}
%\usepackage{thmtools}
\usepackage{xcolor}
\usepackage{tabularx}
\usepackage{xltabular}
\usepackage{aligned-overset}
\usepackage[subpreambles=true]{standalone}
\usepackage{environ}
\usepackage[normalem]{ulem}
\usepackage{etoolbox}
\usepackage{setspace}
\usepackage[bibstyle=reading, citestyle=draft]{biblatex}
\usepackage{xpatch}
\usepackage[many, breakable]{tcolorbox}
\usepackage[backgroundcolor=white, bordercolor=white, textsize=scriptsize]{todonotes}
\usepackage{luacode}
\usepackage{float}
\usepackage{needspace}
\everymath{\displaystyle}

% Police :

\setmathfont{Erewhon Math}

% Tikz :

\usetikzlibrary{calc}
\usetikzlibrary{3d}

% Longueurs :

\setlength{\parindent}{0pt}
\setlength{\headheight}{15pt}
\setlength{\fboxsep}{0pt}
\titlespacing*{\chapter}{0pt}{-20pt}{10pt}
\setlength{\marginparwidth}{1.5cm}
\setstretch{1.1}

% Métadonnées :

\author{agreg.skyost.eu}
\date{\today}

% Titres :

\setcounter{secnumdepth}{3}

\renewcommand{\thechapter}{\Roman{chapter}}
\renewcommand{\thesubsection}{\Roman{subsection}}
\renewcommand{\thesubsubsection}{\arabic{subsubsection}}
\renewcommand{\theparagraph}{\alph{paragraph}}

\titleformat{\chapter}{\huge\bfseries}{\thechapter}{20pt}{\huge\bfseries}
\titleformat*{\section}{\LARGE\bfseries}
\titleformat{\subsection}{\Large\bfseries}{\thesubsection \, - \,}{0pt}{\Large\bfseries}
\titleformat{\subsubsection}{\large\bfseries}{\thesubsubsection. \,}{0pt}{\large\bfseries}
\titleformat{\paragraph}{\bfseries}{\theparagraph. \,}{0pt}{\bfseries}

\setcounter{secnumdepth}{4}

% Table des matières :

\renewcommand{\cftsecleader}{\cftdotfill{\cftdotsep}}
\addtolength{\cftsecnumwidth}{10pt}

% Redéfinition des commandes :

\renewcommand*\thesection{\arabic{section}}
\renewcommand{\ker}{\mathrm{Ker}}

% Nouvelles commandes :

\newcommand{\website}{https://github.com/imbodj/SenCoursDeMaths}

\newcommand{\tr}[1]{\mathstrut ^t #1}
\newcommand{\im}{\mathrm{Im}}
\newcommand{\rang}{\operatorname{rang}}
\newcommand{\trace}{\operatorname{trace}}
\newcommand{\id}{\operatorname{id}}
\newcommand{\stab}{\operatorname{Stab}}
\newcommand{\paren}[1]{\left(#1\right)}
\newcommand{\croch}[1]{\left[ #1 \right]}
\newcommand{\Grdcroch}[1]{\Bigl[ #1 \Bigr]}
\newcommand{\grdcroch}[1]{\bigl[ #1 \bigr]}
\newcommand{\abs}[1]{\left\lvert #1 \right\rvert}
\newcommand{\limi}[3]{\lim_{#1\to #2}#3}
\newcommand{\pinf}{+\infty}
\newcommand{\minf}{-\infty}
%%%%%%%%%%%%%% ENSEMBLES %%%%%%%%%%%%%%%%%
\newcommand{\ensemblenombre}[1]{\mathbb{#1}}
\newcommand{\Nn}{\ensemblenombre{N}}
\newcommand{\Zz}{\ensemblenombre{Z}}
\newcommand{\Qq}{\ensemblenombre{Q}}
\newcommand{\Qqp}{\Qq^+}
\newcommand{\Rr}{\ensemblenombre{R}}
\newcommand{\Cc}{\ensemblenombre{C}}
\newcommand{\Nne}{\Nn^*}
\newcommand{\Zze}{\Zz^*}
\newcommand{\Zzn}{\Zz^-}
\newcommand{\Qqe}{\Qq^*}
\newcommand{\Rre}{\Rr^*}
\newcommand{\Rrp}{\Rr_+}
\newcommand{\Rrm}{\Rr_-}
\newcommand{\Rrep}{\Rr_+^*}
\newcommand{\Rrem}{\Rr_-^*}
\newcommand{\Cce}{\Cc^*}
%%%%%%%%%%%%%%  INTERVALLES %%%%%%%%%%%%%%%%%
\newcommand{\intff}[2]{\left[#1\;,\; #2\right]  }
\newcommand{\intof}[2]{\left]#1 \;, \;#2\right]  }
\newcommand{\intfo}[2]{\left[#1 \;,\; #2\right[  }
\newcommand{\intoo}[2]{\left]#1 \;,\; #2\right[  }

\providecommand{\newpar}{\\[\medskipamount]}

\newcommand{\annexessection}{%
  \newpage%
  \subsection*{Annexes}%
}

\providecommand{\lesson}[3]{%
  \title{#3}%
  \hypersetup{pdftitle={#2 : #3}}%
  \setcounter{section}{\numexpr #2 - 1}%
  \section{#3}%
  \fancyhead[R]{\truncate{0.73\textwidth}{#2 : #3}}%
}

\providecommand{\development}[3]{%
  \title{#3}%
  \hypersetup{pdftitle={#3}}%
  \section*{#3}%
  \fancyhead[R]{\truncate{0.73\textwidth}{#3}}%
}

\providecommand{\sheet}[3]{\development{#1}{#2}{#3}}

\providecommand{\ranking}[1]{%
  \title{Terminale #1}%
  \hypersetup{pdftitle={Terminale #1}}%
  \section*{Terminale #1}%
  \fancyhead[R]{\truncate{0.73\textwidth}{Terminale #1}}%
}

\providecommand{\summary}[1]{%
  \textit{#1}%
  \par%
  \medskip%
}

\tikzset{notestyleraw/.append style={inner sep=0pt, rounded corners=0pt, align=center}}

%\newcommand{\booklink}[1]{\website/bibliographie\##1}
\newcounter{reference}
\newcommand{\previousreference}{}
\providecommand{\reference}[2][]{%
  \needspace{20pt}%
  \notblank{#1}{
    \needspace{20pt}%
    \renewcommand{\previousreference}{#1}%
    \stepcounter{reference}%
    \label{reference-\previousreference-\thereference}%
  }{}%
  \todo[noline]{%
    \protect\vspace{20pt}%
    \protect\par%
    \protect\notblank{#1}{\cite{[\previousreference]}\\}{}%
    \protect\hyperref[reference-\previousreference-\thereference]{p. #2}%
  }%
}

\definecolor{devcolor}{HTML}{00695c}
\providecommand{\dev}[1]{%
  \reversemarginpar%
  \todo[noline]{
    \protect\vspace{20pt}%
    \protect\par%
    \bfseries\color{devcolor}\href{\website/developpements/#1}{[DEV]}
  }%
  \normalmarginpar%
}

% En-têtes :

\pagestyle{fancy}
\fancyhead[L]{\truncate{0.23\textwidth}{\thepage}}
\fancyfoot[C]{\scriptsize \href{\website}{\texttt{https://github.com/imbodj/SenCoursDeMaths}}}

% Couleurs :

\definecolor{property}{HTML}{ffeb3b}
\definecolor{proposition}{HTML}{ffc107}
\definecolor{lemma}{HTML}{ff9800}
\definecolor{theorem}{HTML}{f44336}
\definecolor{corollary}{HTML}{e91e63}
\definecolor{definition}{HTML}{673ab7}
\definecolor{notation}{HTML}{9c27b0}
\definecolor{example}{HTML}{00bcd4}
\definecolor{cexample}{HTML}{795548}
\definecolor{application}{HTML}{009688}
\definecolor{remark}{HTML}{3f51b5}
\definecolor{algorithm}{HTML}{607d8b}
%\definecolor{proof}{HTML}{e1f5fe}
\definecolor{exercice}{HTML}{e1f5fe}

% Théorèmes :

\theoremstyle{definition}
\newtheorem{theorem}{Théorème}

\newtheorem{property}[theorem]{Propriété}
\newtheorem{proposition}[theorem]{Proposition}
\newtheorem{lemma}[theorem]{Activité d'introduction}
\newtheorem{corollary}[theorem]{Conséquence}

\newtheorem{definition}[theorem]{Définition}
\newtheorem{notation}[theorem]{Notation}

\newtheorem{example}[theorem]{Exemple}
\newtheorem{cexample}[theorem]{Contre-exemple}
\newtheorem{application}[theorem]{Application}

\newtheorem{algorithm}[theorem]{Algorithme}
\newtheorem{exercice}[theorem]{Exercice}

\theoremstyle{remark}
\newtheorem{remark}[theorem]{Remarque}

\counterwithin*{theorem}{section}

\newcommand{\applystyletotheorem}[1]{
  \tcolorboxenvironment{#1}{
    enhanced,
    breakable,
    colback=#1!8!white,
    %right=0pt,
    %top=8pt,
    %bottom=8pt,
    boxrule=0pt,
    frame hidden,
    sharp corners,
    enhanced,borderline west={4pt}{0pt}{#1},
    %interior hidden,
    sharp corners,
    after=\par,
  }
}

\applystyletotheorem{property}
\applystyletotheorem{proposition}
\applystyletotheorem{lemma}
\applystyletotheorem{theorem}
\applystyletotheorem{corollary}
\applystyletotheorem{definition}
\applystyletotheorem{notation}
\applystyletotheorem{example}
\applystyletotheorem{cexample}
\applystyletotheorem{application}
\applystyletotheorem{remark}
%\applystyletotheorem{proof}
\applystyletotheorem{algorithm}
\applystyletotheorem{exercice}

% Environnements :

\NewEnviron{whitetabularx}[1]{%
  \renewcommand{\arraystretch}{2.5}
  \colorbox{white}{%
    \begin{tabularx}{\textwidth}{#1}%
      \BODY%
    \end{tabularx}%
  }%
}

% Maths :

\DeclareFontEncoding{FMS}{}{}
\DeclareFontSubstitution{FMS}{futm}{m}{n}
\DeclareFontEncoding{FMX}{}{}
\DeclareFontSubstitution{FMX}{futm}{m}{n}
\DeclareSymbolFont{fouriersymbols}{FMS}{futm}{m}{n}
\DeclareSymbolFont{fourierlargesymbols}{FMX}{futm}{m}{n}
\DeclareMathDelimiter{\VERT}{\mathord}{fouriersymbols}{152}{fourierlargesymbols}{147}

% Code :

\definecolor{greencode}{rgb}{0,0.6,0}
\definecolor{graycode}{rgb}{0.5,0.5,0.5}
\definecolor{mauvecode}{rgb}{0.58,0,0.82}
\definecolor{bluecode}{HTML}{1976d2}
\lstset{
  basicstyle=\footnotesize\ttfamily,
  breakatwhitespace=false,
  breaklines=true,
  %captionpos=b,
  commentstyle=\color{greencode},
  deletekeywords={...},
  escapeinside={\%*}{*)},
  extendedchars=true,
  frame=none,
  keepspaces=true,
  keywordstyle=\color{bluecode},
  language=Python,
  otherkeywords={*,...},
  numbers=left,
  numbersep=5pt,
  numberstyle=\tiny\color{graycode},
  rulecolor=\color{black},
  showspaces=false,
  showstringspaces=false,
  showtabs=false,
  stepnumber=2,
  stringstyle=\color{mauvecode},
  tabsize=2,
  %texcl=true,
  xleftmargin=10pt,
  %title=\lstname
}

\newcommand{\codedirectory}{}
\newcommand{\inputalgorithm}[1]{%
  \begin{algorithm}%
    \strut%
    \lstinputlisting{\codedirectory#1}%
  \end{algorithm}%
}




\begin{document}
  %<*content>
  \lesson{algebra}{159}{Formes linéaires et dualité en dimension finie. Exemples et applications.}

  Soit $E$ un espace vectoriel sur un corps commutatif $\mathbb{K}$ de dimension finie $n$.

  \subsection{Dual d'un espace vectoriel}

  \subsubsection{Formes linéaires, espace dual}

  \reference[ROM21]{441}

  \begin{definition}
    Une \textbf{forme linéaire} sur $E$ est une application linéaire de $E$ dans $\mathbb{K}$. L'espace $\mathcal{L}(E, \mathbb{K})$ formé par l'ensemble des formes linéaires sur $E$ est appelé \textbf{dual} de $E$ et est noté $E^*$.
  \end{definition}

  \begin{example}
    \label{159-1}
    \begin{itemize}
      \item Soit $\mathcal{B} = (e_1, \dots, e_n)$ une base de $E$. Alors pour tout $j \in \llbracket 1, n \rrbracket$, la projection
      \[ p_j : \sum_{i=1}^{n} x_i e_i \mapsto x_j \]
      est une forme linéaire.
      \item Toute combinaison linéaire de formes linéaires est une forme linéaire.
    \end{itemize}
  \end{example}

  \begin{remark}
    Une forme linéaire non nulle sur $E$ est surjective.
  \end{remark}

  \begin{definition}
    On appelle \textbf{hyperplan} de $E$, le noyau d'une forme linéaire non nulle sur $E$.
  \end{definition}

  \begin{proposition}
    \begin{enumerate}[label=(\roman*)]
      \item Un hyperplan de $E$ est un sous-espace de $E$ supplémentaire d'une droite.
      \item Deux formes linéaires non nulles définissent le même hyperplan si et seulement si elles sont liées.
    \end{enumerate}
  \end{proposition}

  \subsubsection{Bases duales}

  \reference[GOU21]{133}

  \begin{definition}
    En reprenant les notations de la \cref{159-1}, les projections $p_i$ sont les \textbf{formes linéaires coordonnées}. On note $\forall i \in \llbracket 1, n \rrbracket$, $p_i = e_i^*$. La famille $\mathcal{B}^* = (e_1, \dots, e_n)$ est appelée \textbf{base duale} de $\mathcal{B}$.
  \end{definition}

  \begin{remark}
    Soit $\mathcal{B} = (e_1, \dots, e_n)$ une base de $E$. Pour tout $i, j \in \llbracket 1, n \rrbracket$, on a
    \[ e_i^*(e_j) = \delta_{i,j} = \begin{cases} 1 &\text{ si } i = j \\ 0 &\text{ sinon} \end{cases} \]
  \end{remark}

  \begin{theorem}
    Soit $\mathcal{B} = (e_1, \dots, e_n)$ une base de $E$. Alors, la base duale $\mathcal{B}^*$ est une base de $E^*$.
  \end{theorem}

  \begin{corollary}
    \begin{enumerate}[label=(\roman*)]
      \item $E^*$ est un espace vectoriel de dimension $n$.
      \item Pour tout $\varphi \in E^*$, on a $\varphi = \sum_{n=1}^n \varphi(e_i) e_i^*$.
    \end{enumerate}
  \end{corollary}

  \reference[ROM21]{446}

  \begin{corollary}
    Tout hyperplan de $E$ est de dimension $n-1$.
  \end{corollary}

  \reference[GOU20]{325}

  \begin{example}
    Soit $U \subseteq \mathbb{R}^n$ un ouvert. Soit $f : U \rightarrow \mathbb{R}$ différentiable en $a \in U$. Alors,
    \[ \mathrm{d}f_a = \sum_{i=1}^n \frac{\partial f}{\partial x_i}(a) e_i^* \]
    où $(e_i^*)_{i \in \llbracket 1, n \rrbracket}$ est la base duale de la base canonique $(e_i)_{i \in \llbracket 1, n \rrbracket}$ de $\mathbb{R}^n$.
  \end{example}

  \subsubsection{Bidual}

  \reference[GOU21]{133}

  \begin{definition}
    On appelle \textbf{bidual} de $E$ le dual $E^*$. On le note $E^{**}$.
  \end{definition}

  \begin{example}
    Pour $x \in E$, l'application $\operatorname{ev}_x : \varphi \mapsto \varphi(x)$ est un élément de $E^{**}$.
  \end{example}

  \begin{theorem}
    $x \mapsto \operatorname{ev}_x$ est un isomorphisme entre les espaces $E$ et $E^{**}$.
  \end{theorem}

  \begin{remark}
    Cet isomorphisme est canonique : il ne dépend pas du choix d'une base de $E$.
  \end{remark}

  \begin{corollary}
    Soit $(f_1, \dots, f_n)$ une base de $E^*$. Il existe une unique base $(e_1, \dots, e_n)$ de $E$ telle que, pour tout $i \in \llbracket 1, n \rrbracket$, $e_i^* = f_i$.
  \end{corollary}

  \begin{definition}
    En reprenant les notations précédentes, $(e_1, \dots, e_n)$ est appelée \textbf{base antéduale} de $(f_1, \dots, f_n)$.
  \end{definition}

  \begin{example}
    On suppose $n = 3$. Soient $(e_1, e_2, e_3)$ une base de $E$ et
    \[ f_1^* = 2e_1^* + e_2^* + e_3^*, \, f_2^* = -e_1^* + 2e_3^*, \, f_3^* = e_1^* + 3e_2^* \]
    Alors, $(f_1^*, f_2^*, f_3^*)$ est une base de $E^*$, dont une base antéduale est $(f_1, f_2, f_3)$ où
    \[ f_1 = \frac{1}{13}(6e_1 - 2e_2 + 3e_3), \, f_2 = \frac{1}{13}(-3e_1 - e_2 + 5e_3), \, f_3 = \frac{1}{13}(-2e_1 + 5e_2 - e_3) \]
  \end{example}

  \subsection{Orthogonalité au sens de la dualité}

  \subsubsection{Orthogonal d'une partie, d'une famille}

  \reference[ROM21]{446}

  \begin{definition}
    On dit qu'une forme linéaire $\varphi \in E^*$ et un vecteur $x \in E$ sont orthogonaux si $\varphi(x) = 0$.
  \end{definition}

  \begin{definition}
    \begin{itemize}
      \item L'orthogonal dans $E^*$ d'une partie non vide $X$ de $E$ est l'ensemble
      \[ X^\perp = \{ \varphi \in E^* \mid \forall x \in X, \, \varphi(x) = 0 \} \]
      \item L'orthogonal dans $E$ d'une partie non vide $Y$ de $E^*$ est l'ensemble
      \[ Y^\circ = \{ x \in E \mid \forall \varphi \in Y, \, \varphi(x) = 0 \} \]
    \end{itemize}
  \end{definition}

  \begin{theorem}
    Soient $A$, $B$ des parties non vides de $E$ et $U$, $V$ des parties non vides de $E^*$.
    \begin{enumerate}[label=(\roman*)]
      \item Si $A \subseteq B$, alors $B^\perp \subseteq A^\perp$.
      \item Si $U \subseteq V$, alors $V^\circ \subseteq U^\circ$.
      \item $A \subseteq (A^\perp)^\circ$.
      \item $U \subseteq (U^\circ)^\perp$.
      \item $A^\perp = \operatorname{Vect}(A)^\perp$.
      \item $U^\circ = \operatorname{Vect}(U)^\circ$.
      \item $\{ 0 \}^\perp = E^*, \, E^\perp = \{ 0 \}, \, \{ 0 \}^\circ = E \text{ et }, \, (E^*)^\circ = \{ 0 \}$.
    \end{enumerate}
  \end{theorem}

  \begin{corollary}
    \begin{enumerate}[label=(\roman*)]
      \item Pour tout sous-espace vectoriel $F$ de $E$, on a
      \[ \dim(F) + \dim(F^\perp) = n \]
      \item Pour tout sous-espace vectoriel $G$ de $E^*$, on a
      \[ \dim(G) + \dim(F^\circ) = n \]
      \item Pour tout sous-espace vectoriel $F$ de $E$, et pour tout sous-espace vectoriel $G$ de $E^*$, on a $F = (F^\perp)^\circ$ et $G = (G^\circ)^\perp$.
      \item Pour toute partie $X$ de $E$, on a $(X^\perp)^\circ = \operatorname{Vect}(X)$.
      \item Pour tous sous-espaces vectoriels $F_1$ et $F_2$ de $E$, on a :
      \[ (F_1 + F_2)^\perp = F_1^\perp \, \cap \, F_2^\perp \text{ et } (F_1 \, \cap \, F_2)^\perp = F_1^\perp + F_2^\perp \]
      \item Pour tous sous-espaces vectoriels $G_1$ et $G_2$ de $E^*$, on a :
      \[ (G_1 + G_2)^\circ = G_1^\circ \, \cap \, G_2^\circ \text{ et } (G_1 \, \cap \, G_2)^\circ = G_1^\circ + G_2^\circ \]
    \end{enumerate}
  \end{corollary}

  \begin{corollary}
    Si $(\varphi_i)_{i \in \llbracket 1, p \rrbracket}$ est une famille de formes linéaires sur $E$ de rang $r$, le sous-espace vectoriel $F = \bigcap_{i=1}^p \ker(\varphi_i)$ de $E$ est alors de dimension $n-r$. Réciproquement, si $F$ est un sous-espace vectoriel de $E$ de dimension $m$, il existe alors une famille de formes linéaires $(\varphi_i)_{i \in \llbracket 1, p \rrbracket}$ de rang $r = n-m$ telle que $F = \bigcap_{i=1}^p \ker(\varphi_i)$.
  \end{corollary}

  \subsubsection{Application transposée}

  \reference{452}

  \begin{definition}
    Soient $E$ et $F$ deux espaces vectoriels sur $\mathbb{K}$. Soit $u \in \mathcal{L}(E,F)$. La \textbf{transposée} de $u \in \mathcal{L}(E,F)$ est l'application
    \[
      \tr{u} :
      \begin{array}{ccc}
        F^* &\rightarrow& E^* \\
        \varphi &\mapsto& \varphi \circ u
      \end{array}
    \]
  \end{definition}

  \begin{proposition}
    $u \mapsto \tr{u}$ est linéaire, injective de $\mathcal{L}(E,F)$ dans $\mathcal{L}(F^*,E^*)$.
  \end{proposition}

  \begin{theorem}
    Soient $E$, $F$ et $G$ trois espaces vectoriels sur $\mathbb{K}$. Soient $u \in \mathcal{L}(E,F)$ et $u \in \mathcal{L}(F,G)$. On a :
    \begin{enumerate}[label=(\roman*)]
      \item $\tr{v \circ u} = \tr{u} \circ \tr{v}$.
      \item Pour $F = E$, $\tr{\operatorname{id}_E} = \operatorname{id}_{E^*}$.
      \item Si $u$ est un isomorphisme de $E$ sur $F$, alors $\tr{u}$ est un isomorphisme de $F^*$ sur $E^*$ et $(\tr{u})^{-1} = \tr{(u^{-1})}$.
      \item $\ker(\tr{u}) = (\im(u))^{\perp}$.
      \item $u$ est surjective si et seulement si $\tr{u}$ est injective.
      \item $\im(\tr{u}) = (\ker(u))^{\perp}$.
      \item $u$ est injective si et seulement si $\tr{u}$ est surjective.
      \item Si $E$ et $F$ sont de dimension finie, alors $u$ et $\tr{u}$ ont même rang.
      \item Si $A \in \mathcal{M}_n(\mathbb{K})$ est la matrice de $u$ dans des bases $\mathcal{B}$ et $\mathcal{B}'$, alors $\tr{A}$ est la matrice de $\tr{u}$ dans les bases $\mathcal{B}'^*$ et $\mathcal{B}^*$.
    \end{enumerate}
  \end{theorem}

  \reference[GOU21]{136}

  \begin{corollary}
    Soient $\mathcal{B}$ et $\mathcal{B}'$ deux bases de $E$ et $P$ la matrice de passage de $\mathcal{B}$ à $\mathcal{B}'$. Alors, la matrice de passage de $\mathcal{B}^*$ à $\mathcal{B}'^*$ est
    \[ \tr{P}^{-1} \]
  \end{corollary}

  \begin{proposition}
    Soit $u \in \mathcal{L}(E)$. Alors un sous-espace vectoriel de $E$ est stable par $u$ si et seulement si son orthogonal l'est.
  \end{proposition}

  \reference{176}
  \dev{trigonalisation-simultanee}

  \begin{application}[Trigonalisation simultanée]
    Soit $(u_i)_{i \in I}$ une famille d'endomorphismes de $E$ diagonalisables qui commutent deux-à-deux. Alors, il existe une base commune de trigonalisation.
  \end{application}

  \subsubsection{Lien avec l'orthogonalité au sens euclidien}

  \reference[ROM21]{718}

  \begin{theorem}[de représentation de Riesz]
    Soit $\langle ., . \rangle$ un produit scalaire sur $E$.
    \[ \forall \varphi \in E^*, \, \exists! a \in E \text{ tel que } \forall x \in E, \, \varphi(x) = \langle x, a \rangle \]
  \end{theorem}

  \reference{446}

  Ainsi, si $E$ est muni d'un produit scalaire $\langle ., . \rangle$, on retrouve la notion classique d'orthogonalité euclidienne avec $\varphi : x \mapsto \langle x, a \rangle$.

  \reference[GOU21]{138}

  \begin{example}
    L'application
    \[
    \begin{array}{ccc}
      \mathcal{M}_n(\mathbb{K}) &\rightarrow& \mathcal{M}_n(\mathbb{K})^* \\
      A &\mapsto& (X \mapsto \trace(AX))
    \end{array}
    \]
    est un isomorphisme.
  \end{example}

  \subsection{Applications}

  \subsubsection{Formule de Taylor}

  \reference[ROM21]{442}

  On suppose $\mathbb{K}$ de caractéristique nulle.

  \begin{application}[Formule de Taylor]
    Pour tout $j \in \llbracket 0, n \rrbracket$, on définit :
    \[
    e_j : \begin{array}{ccc}
      \mathbb{K}_n[X] &\rightarrow& \mathbb{K} \\
      P &\mapsto& \frac{P^{(j)}(0)}{j!}
    \end{array}
    \]
    Alors, $(e_i)_{i \in \llbracket 0, n \rrbracket}$ est une base de $K_n[X]^*$, dont la base antéduale est $(X^i)_{i \in \llbracket 0, n \rrbracket}$.
  \end{application}

  \reference[GOU21]{64}

  \begin{corollary}
    On suppose $P \neq 0$. Alors $a \in \mathbb{K}$ est racine d'ordre $h$ de $P$ si et seulement si
    \[ \forall i \in \llbracket 1, h-1 \rrbracket, P^{(i)}(a) = 0 \quad \text{ et } \quad F^{(h)}(a) \neq 0 \]
  \end{corollary}

  \begin{example}
    Le polynôme $P_n = \sum_{i=0}^{n} \frac{1}{i!} X^{i}$ n'a que des racines simples dans $\mathbb{C}$.
  \end{example}

  \begin{remark}
    C'est encore vrai en caractéristique non nulle pour $h = 1$.
  \end{remark}

  \subsubsection{Invariants de similitude}

  \reference[ROM21]{397}

  Soient $E$ un espace vectoriel de dimension finie $n$ et $u \in \mathcal{L}(E)$.

  \begin{definition}
    On dit que $u$ est \textbf{cyclique} s'il existe $x \in E$ tel que $\{ P(u)(x) \mid P \in \mathbb{K}[X] \} = E$.
  \end{definition}

  \begin{proposition}
    $u$ est cyclique si et seulement si $\deg(\pi_u) = n$.
  \end{proposition}

  \begin{definition}
    Soit $P = X^p + a_{p-1} X^{p-1} + \dots + a_0 \in \mathbb{K}[X]$. On appelle \textbf{matrice compagnon} de $P$ la matrice
    \[ \mathcal{C}(P) = \begin{pmatrix} 0 & \dots & \dots & 0 & -a_0 \\ 1 & 0 & \ddots & \vdots & -a_1 \\ 0 & 1 & \ddots & \vdots & \vdots \\ \vdots & \ddots & \ddots & 0 & -a_{p-2} \\ 0 & \dots & 0 & 1 & -a_{p-1} \end{pmatrix} \]
  \end{definition}

  \begin{proposition}
    $u$ est cyclique si et seulement s'il existe une base $\mathcal{B}$ de $E$ telle que $\operatorname{Mat}(u, \mathcal{B}) = \mathcal{C}(\pi_u)$.
  \end{proposition}

  \begin{theorem}
    Il existe $F_1, \dots, F_r$ des sous-espaces vectoriels de $E$ tous stables par $u$ tels que :
    \begin{itemize}
      \item $E = F_1 \oplus \dots \oplus F_r$.
      \item $u_i = u_{|F_i}$ est cyclique pour tout $i$.
      \item Si $P_i = \pi_{u_i}$, on a $P_{i+1} \mid P_i$ pour tout $i$.
    \end{itemize}
    La famille de polynômes $P_1, \dots, P_r$ ne dépend que de $u$ et non du choix de la décomposition. On l'appelle \textbf{suite des invariants de similitude} de $u$.
  \end{theorem}

  \begin{theorem}[Réduction de Frobenius]
    Si $P_1, \dots, P_r$ désigne la suite des invariants de $u$, alors il existe une base $\mathcal{B}$ de $E$ telle que :
    \[ \operatorname{Mat}(u, \mathcal{B}) = \begin{pmatrix} \mathcal{C}(P_1) & & \\ & \ddots & \\ & & \mathcal{C}(P_r) \end{pmatrix} \]
    On a d'ailleurs $P_1 = \pi_u$ et $P_1 \dots P_r = \chi_u$.
  \end{theorem}

  \begin{corollary}
    Deux endomorphismes de $E$ sont semblables si et seulement s'ils ont la même suite d'invariants de similitude.
  \end{corollary}

  \begin{application}
    Pour $n = 2$ ou $3$, deux matrices sont semblables si et seulement si elles ont mêmes polynômes minimal et caractéristique.
  \end{application}

  \begin{application}
    Soit $\mathbb{L}$ une extension de $\mathbb{K}$. Alors, si $A, B \in \mathcal{M}_n(\mathbb{K})$ sont semblables dans $\mathcal{M}_n(\mathbb{L})$, elles le sont aussi dans $\mathcal{M}_n(\mathbb{K})$.
  \end{application}

  \subsubsection{Classification des formes quadratiques}

  \reference{243}

  Soit $q$ une forme quadratique sur $E$.

  \begin{lemma}
    Il existe une base $q$-orthogonale (ie. si $\varphi$ est la forme polaire de $q$, une base $B$ où $\forall e, e' \in B, \, \varphi(e, e') = 0$ si $e \neq e'$).
  \end{lemma}

  \dev{loi-d-inertie-de-sylvester}

  \begin{theorem}[Loi d'inertie de Sylvester]
    \[ \exists p, q \in \mathbb{N} \text{ et } \exists f_1, \dots, f_{p+q} \in E^* \text{ tels que } q = \sum_{i=1}^p |f_i|^2 - \sum_{i=p+1}^{p+q} |f_i|^2 \]
    où les formes linéaires $f_i$ sont linéairement indépendantes et où $p + q \leq n$. De plus, ces entiers ne dépendent que de $q$ et pas de la décomposition choisie.
    \newpar
    Le couple $(p,q)$ est la \textbf{signature} de $q$ et le rang $q$ est égal à $p+q$.
  \end{theorem}

  \begin{example}
    La signature de la forme quadratique $q : (x,y,z) \mapsto x^2 - 2y^2 + xz + yz$ est $(2,1)$, donc son rang est $3$.
  \end{example}
  %</content>
\end{document}
