\documentclass[12pt, a4paper]{report}

% LuaLaTeX :

\RequirePackage{iftex}
\RequireLuaTeX

% Packages :

\usepackage[french]{babel}
%\usepackage[utf8]{inputenc}
%\usepackage[T1]{fontenc}
\usepackage[pdfencoding=auto, pdfauthor={Hugo Delaunay}, pdfsubject={Mathématiques}, pdfcreator={agreg.skyost.eu}]{hyperref}
\usepackage{amsmath}
\usepackage{amsthm}
%\usepackage{amssymb}
\usepackage{stmaryrd}
\usepackage{tikz}
\usepackage{tkz-euclide}
\usepackage{fontspec}
\defaultfontfeatures[Erewhon]{FontFace = {bx}{n}{Erewhon-Bold.otf}}
\usepackage{fourier-otf}
\usepackage[nobottomtitles*]{titlesec}
\usepackage{fancyhdr}
\usepackage{listings}
\usepackage{catchfilebetweentags}
\usepackage[french, capitalise, noabbrev]{cleveref}
\usepackage[fit, breakall]{truncate}
\usepackage[top=2.5cm, right=2cm, bottom=2.5cm, left=2cm]{geometry}
\usepackage{enumitem}
\usepackage{tocloft}
\usepackage{microtype}
%\usepackage{mdframed}
%\usepackage{thmtools}
\usepackage{xcolor}
\usepackage{tabularx}
\usepackage{xltabular}
\usepackage{aligned-overset}
\usepackage[subpreambles=true]{standalone}
\usepackage{environ}
\usepackage[normalem]{ulem}
\usepackage{etoolbox}
\usepackage{setspace}
\usepackage[bibstyle=reading, citestyle=draft]{biblatex}
\usepackage{xpatch}
\usepackage[many, breakable]{tcolorbox}
\usepackage[backgroundcolor=white, bordercolor=white, textsize=scriptsize]{todonotes}
\usepackage{luacode}
\usepackage{float}
\usepackage{needspace}
\everymath{\displaystyle}

% Police :

\setmathfont{Erewhon Math}

% Tikz :

\usetikzlibrary{calc}
\usetikzlibrary{3d}

% Longueurs :

\setlength{\parindent}{0pt}
\setlength{\headheight}{15pt}
\setlength{\fboxsep}{0pt}
\titlespacing*{\chapter}{0pt}{-20pt}{10pt}
\setlength{\marginparwidth}{1.5cm}
\setstretch{1.1}

% Métadonnées :

\author{agreg.skyost.eu}
\date{\today}

% Titres :

\setcounter{secnumdepth}{3}

\renewcommand{\thechapter}{\Roman{chapter}}
\renewcommand{\thesubsection}{\Roman{subsection}}
\renewcommand{\thesubsubsection}{\arabic{subsubsection}}
\renewcommand{\theparagraph}{\alph{paragraph}}

\titleformat{\chapter}{\huge\bfseries}{\thechapter}{20pt}{\huge\bfseries}
\titleformat*{\section}{\LARGE\bfseries}
\titleformat{\subsection}{\Large\bfseries}{\thesubsection \, - \,}{0pt}{\Large\bfseries}
\titleformat{\subsubsection}{\large\bfseries}{\thesubsubsection. \,}{0pt}{\large\bfseries}
\titleformat{\paragraph}{\bfseries}{\theparagraph. \,}{0pt}{\bfseries}

\setcounter{secnumdepth}{4}

% Table des matières :

\renewcommand{\cftsecleader}{\cftdotfill{\cftdotsep}}
\addtolength{\cftsecnumwidth}{10pt}

% Redéfinition des commandes :

\renewcommand*\thesection{\arabic{section}}
\renewcommand{\ker}{\mathrm{Ker}}

% Nouvelles commandes :

\newcommand{\website}{https://github.com/imbodj/SenCoursDeMaths}

\newcommand{\tr}[1]{\mathstrut ^t #1}
\newcommand{\im}{\mathrm{Im}}
\newcommand{\rang}{\operatorname{rang}}
\newcommand{\trace}{\operatorname{trace}}
\newcommand{\id}{\operatorname{id}}
\newcommand{\stab}{\operatorname{Stab}}
\newcommand{\paren}[1]{\left(#1\right)}
\newcommand{\croch}[1]{\left[ #1 \right]}
\newcommand{\Grdcroch}[1]{\Bigl[ #1 \Bigr]}
\newcommand{\grdcroch}[1]{\bigl[ #1 \bigr]}
\newcommand{\abs}[1]{\left\lvert #1 \right\rvert}
\newcommand{\limi}[3]{\lim_{#1\to #2}#3}
\newcommand{\pinf}{+\infty}
\newcommand{\minf}{-\infty}
%%%%%%%%%%%%%% ENSEMBLES %%%%%%%%%%%%%%%%%
\newcommand{\ensemblenombre}[1]{\mathbb{#1}}
\newcommand{\Nn}{\ensemblenombre{N}}
\newcommand{\Zz}{\ensemblenombre{Z}}
\newcommand{\Qq}{\ensemblenombre{Q}}
\newcommand{\Qqp}{\Qq^+}
\newcommand{\Rr}{\ensemblenombre{R}}
\newcommand{\Cc}{\ensemblenombre{C}}
\newcommand{\Nne}{\Nn^*}
\newcommand{\Zze}{\Zz^*}
\newcommand{\Zzn}{\Zz^-}
\newcommand{\Qqe}{\Qq^*}
\newcommand{\Rre}{\Rr^*}
\newcommand{\Rrp}{\Rr_+}
\newcommand{\Rrm}{\Rr_-}
\newcommand{\Rrep}{\Rr_+^*}
\newcommand{\Rrem}{\Rr_-^*}
\newcommand{\Cce}{\Cc^*}
%%%%%%%%%%%%%%  INTERVALLES %%%%%%%%%%%%%%%%%
\newcommand{\intff}[2]{\left[#1\;,\; #2\right]  }
\newcommand{\intof}[2]{\left]#1 \;, \;#2\right]  }
\newcommand{\intfo}[2]{\left[#1 \;,\; #2\right[  }
\newcommand{\intoo}[2]{\left]#1 \;,\; #2\right[  }

\providecommand{\newpar}{\\[\medskipamount]}

\newcommand{\annexessection}{%
  \newpage%
  \subsection*{Annexes}%
}

\providecommand{\lesson}[3]{%
  \title{#3}%
  \hypersetup{pdftitle={#2 : #3}}%
  \setcounter{section}{\numexpr #2 - 1}%
  \section{#3}%
  \fancyhead[R]{\truncate{0.73\textwidth}{#2 : #3}}%
}

\providecommand{\development}[3]{%
  \title{#3}%
  \hypersetup{pdftitle={#3}}%
  \section*{#3}%
  \fancyhead[R]{\truncate{0.73\textwidth}{#3}}%
}

\providecommand{\sheet}[3]{\development{#1}{#2}{#3}}

\providecommand{\ranking}[1]{%
  \title{Terminale #1}%
  \hypersetup{pdftitle={Terminale #1}}%
  \section*{Terminale #1}%
  \fancyhead[R]{\truncate{0.73\textwidth}{Terminale #1}}%
}

\providecommand{\summary}[1]{%
  \textit{#1}%
  \par%
  \medskip%
}

\tikzset{notestyleraw/.append style={inner sep=0pt, rounded corners=0pt, align=center}}

%\newcommand{\booklink}[1]{\website/bibliographie\##1}
\newcounter{reference}
\newcommand{\previousreference}{}
\providecommand{\reference}[2][]{%
  \needspace{20pt}%
  \notblank{#1}{
    \needspace{20pt}%
    \renewcommand{\previousreference}{#1}%
    \stepcounter{reference}%
    \label{reference-\previousreference-\thereference}%
  }{}%
  \todo[noline]{%
    \protect\vspace{20pt}%
    \protect\par%
    \protect\notblank{#1}{\cite{[\previousreference]}\\}{}%
    \protect\hyperref[reference-\previousreference-\thereference]{p. #2}%
  }%
}

\definecolor{devcolor}{HTML}{00695c}
\providecommand{\dev}[1]{%
  \reversemarginpar%
  \todo[noline]{
    \protect\vspace{20pt}%
    \protect\par%
    \bfseries\color{devcolor}\href{\website/developpements/#1}{[DEV]}
  }%
  \normalmarginpar%
}

% En-têtes :

\pagestyle{fancy}
\fancyhead[L]{\truncate{0.23\textwidth}{\thepage}}
\fancyfoot[C]{\scriptsize \href{\website}{\texttt{https://github.com/imbodj/SenCoursDeMaths}}}

% Couleurs :

\definecolor{property}{HTML}{ffeb3b}
\definecolor{proposition}{HTML}{ffc107}
\definecolor{lemma}{HTML}{ff9800}
\definecolor{theorem}{HTML}{f44336}
\definecolor{corollary}{HTML}{e91e63}
\definecolor{definition}{HTML}{673ab7}
\definecolor{notation}{HTML}{9c27b0}
\definecolor{example}{HTML}{00bcd4}
\definecolor{cexample}{HTML}{795548}
\definecolor{application}{HTML}{009688}
\definecolor{remark}{HTML}{3f51b5}
\definecolor{algorithm}{HTML}{607d8b}
%\definecolor{proof}{HTML}{e1f5fe}
\definecolor{exercice}{HTML}{e1f5fe}

% Théorèmes :

\theoremstyle{definition}
\newtheorem{theorem}{Théorème}

\newtheorem{property}[theorem]{Propriété}
\newtheorem{proposition}[theorem]{Proposition}
\newtheorem{lemma}[theorem]{Activité d'introduction}
\newtheorem{corollary}[theorem]{Conséquence}

\newtheorem{definition}[theorem]{Définition}
\newtheorem{notation}[theorem]{Notation}

\newtheorem{example}[theorem]{Exemple}
\newtheorem{cexample}[theorem]{Contre-exemple}
\newtheorem{application}[theorem]{Application}

\newtheorem{algorithm}[theorem]{Algorithme}
\newtheorem{exercice}[theorem]{Exercice}

\theoremstyle{remark}
\newtheorem{remark}[theorem]{Remarque}

\counterwithin*{theorem}{section}

\newcommand{\applystyletotheorem}[1]{
  \tcolorboxenvironment{#1}{
    enhanced,
    breakable,
    colback=#1!8!white,
    %right=0pt,
    %top=8pt,
    %bottom=8pt,
    boxrule=0pt,
    frame hidden,
    sharp corners,
    enhanced,borderline west={4pt}{0pt}{#1},
    %interior hidden,
    sharp corners,
    after=\par,
  }
}

\applystyletotheorem{property}
\applystyletotheorem{proposition}
\applystyletotheorem{lemma}
\applystyletotheorem{theorem}
\applystyletotheorem{corollary}
\applystyletotheorem{definition}
\applystyletotheorem{notation}
\applystyletotheorem{example}
\applystyletotheorem{cexample}
\applystyletotheorem{application}
\applystyletotheorem{remark}
%\applystyletotheorem{proof}
\applystyletotheorem{algorithm}
\applystyletotheorem{exercice}

% Environnements :

\NewEnviron{whitetabularx}[1]{%
  \renewcommand{\arraystretch}{2.5}
  \colorbox{white}{%
    \begin{tabularx}{\textwidth}{#1}%
      \BODY%
    \end{tabularx}%
  }%
}

% Maths :

\DeclareFontEncoding{FMS}{}{}
\DeclareFontSubstitution{FMS}{futm}{m}{n}
\DeclareFontEncoding{FMX}{}{}
\DeclareFontSubstitution{FMX}{futm}{m}{n}
\DeclareSymbolFont{fouriersymbols}{FMS}{futm}{m}{n}
\DeclareSymbolFont{fourierlargesymbols}{FMX}{futm}{m}{n}
\DeclareMathDelimiter{\VERT}{\mathord}{fouriersymbols}{152}{fourierlargesymbols}{147}

% Code :

\definecolor{greencode}{rgb}{0,0.6,0}
\definecolor{graycode}{rgb}{0.5,0.5,0.5}
\definecolor{mauvecode}{rgb}{0.58,0,0.82}
\definecolor{bluecode}{HTML}{1976d2}
\lstset{
  basicstyle=\footnotesize\ttfamily,
  breakatwhitespace=false,
  breaklines=true,
  %captionpos=b,
  commentstyle=\color{greencode},
  deletekeywords={...},
  escapeinside={\%*}{*)},
  extendedchars=true,
  frame=none,
  keepspaces=true,
  keywordstyle=\color{bluecode},
  language=Python,
  otherkeywords={*,...},
  numbers=left,
  numbersep=5pt,
  numberstyle=\tiny\color{graycode},
  rulecolor=\color{black},
  showspaces=false,
  showstringspaces=false,
  showtabs=false,
  stepnumber=2,
  stringstyle=\color{mauvecode},
  tabsize=2,
  %texcl=true,
  xleftmargin=10pt,
  %title=\lstname
}

\newcommand{\codedirectory}{}
\newcommand{\inputalgorithm}[1]{%
  \begin{algorithm}%
    \strut%
    \lstinputlisting{\codedirectory#1}%
  \end{algorithm}%
}



\everymath{\displaystyle}
\begin{document}
  %<*content>
  \lesson{algebra}{4}{Logarithme népérien (TS2)}

\subsection{Définition et propriétés}


\begin{definition}
On appelle fonction logarithme népérien,  notée $ \ln $, la primitive   sur $ \intoo{0}{\pinf} $ de la  fonction  $ x  \longmapsto \dfrac{1}{x} $  et qui s'annule pour  $ x=1 $.

\end{definition}


La touche \fbox{$ \ln $} de la calculatrice permet de calculer le logarithme d'un réel.
 
 \subsubsection*{Conséquences immédiates}
\begin{enumerate}
\item Le domaine de définition de la fonction $ \ln $  est  $ \intoo{0}{\pinf} $.
\item La fonction $ \ln $  est  dérivable  sur  $ \intoo{0}{\pinf} $  et pour tout $ x> 0 $ on a   $\ln'(x)=\dfrac{1}{x} $.
\item  $\ln (1)= 0$.
\item  La dérivée de la fonction $ \ln $ étant strictement positive  sur $ \intoo{0}{\pinf} $, donc la  fonction $ \ln $  est strictement croissante sur  $ \intoo{0}{\pinf} $.
\end{enumerate}

\begin{property}

\begin{align*}
    \forall a > 0, \quad  \forall b > 0 ,\quad & a< b \Longleftrightarrow \ln a < \ln b \\
    & a> b \Longleftrightarrow \ln a  >\ln b  \\
    & a= b \Longleftrightarrow \ln a  =\ln b
    \end{align*} 
   \end{property}

 \begin{property}{fondamentale}
$$\text{Pour tous}\quad  a> 0 \quad  \text{ et} \quad  b>0 ,\;\text{ on a }\quad  \ln(ab)=\ln a +\ln b $$
 \end{property}
\textbf{Démonstration}

Soit la fonction  $g : x \mapsto \ln(ax)$ où $ a > 0 $ et fixé.

g est définie et dérivable sur $]0 , \pinf[$ et d'après le théorème de dérivation d'une fonction composée,
on a :

$ \forall x > 0,\;  g'(x) = (ax)'\ln'(ax) = a \times \dfrac{1}{ax}=\dfrac{1}{x}$


On a donc $\paren{g - \ln}' (x) = \dfrac{1}{x}-\dfrac{1}{x}= 0$ donc $ \paren{g - \ln }'(x) = 0$.

Il en résulte que la fonction $g - \ln$  est une constante sur $]0 , \pinf[$.

Il existe donc  $k \in \Rr$ tel que $\forall x > 0 ;\; g (x)- \ln (x) = k \Leftrightarrow \ln(ax) - \ln x = k$

Pour  $x = 1$  on a $ \ln a - ln 1 
= k$ donc  $k = \ln a$. D’où $\ln(ax) = \ln x + \ln a$.

\begin{corollary} 
Soit $ a > 0$  et  $ b > 0$.
\begin{enumerate}
\item $  \ln \paren{\dfrac{1}{a}}= - \ln a$
\item $\ln \paren{\dfrac{a}{b} }= \ln a-  \ln b$
 \item $  \ln \sqrt{a} = \dfrac{1}{2} \ln a$
 \item $ \ln a^{r} = r\ln a \quad \forall r\in\Qq$
\end{enumerate}

\end{corollary} 




\textbf{Démonstration}

\begin{enumerate}

\item $ \ln \paren{a \times \dfrac{1}{a}}=\ln 1 =0 = \ln \paren{\dfrac{1}{a}}+ \ln a  $ \; d'où   \;$\ln \paren{\dfrac{1}{a}}= - \ln a$
\item $ \ln \paren{\dfrac{a}{b} }= \ln \paren{ a\times \dfrac{1}{b} }=\ln a+ \ln \paren{ \dfrac{1}{b}}=\ln a-  \ln b $
\item $ \ln a=\ln \paren{\sqrt{a}}^{2}=2\ln \sqrt{a} $\; d'où \;$ \ln \sqrt{a} = \dfrac{1}{2} \ln a$.
\item  Pour $ n=0 \;$ on a $ \ln a^{0}=\ln 1 =0=0\times \ln a $ \\ 
Supposons la propriété vraie pour un entier  naturel  $ n $ quelconque.  \\
On a $ \ln a^{n+1} =\ln a^{n}\times a  =\ln a^{n}+\ln a  = n\ln a+\ln a=(n+1)\ln a $ \\
Il en résulte que la propriété est vraie $ \forall n \in \Nn. $ \\
Supposons que $ n\in\Zzn $.\;   Posons $ p=-n $   donc $ p\in\Nn. $  \\
$ \ln a^{n}=\ln a^{-p}   =\ln \dfrac{1}{a^{p}}  =-\ln a^{p}=   -p\ln a  =n\ln a $  \\
Donc pour tout $ n\in \Zzn $\; on a $\ln a^{n} = n\ln a  $. \\
Soit  $ p\in \Nne $  on a $ \ln\paren{ a}^{\dfrac{p}{p}}  =             p\ln a^{\dfrac{1}{p}}  =\ln a\; $ d'où   $ \ln a^{\dfrac{1}{p}}=\dfrac{1}{p} \ln a $    \\  Soit $ r=\dfrac{n}{p}\in\Qq \;$  alors  $\; \ln \paren{a}^{\dfrac{n}{p}} =n \ln a^{\dfrac{1}{p}}  =n \paren{\dfrac{1}{p} \ln a}  =\dfrac{n}{p} \ln a=  r\ln a$ 
\end{enumerate}



\subsection{ Étude de la fonction $\ln $}

\subsection*{Limites}
 Les limites aux bornes de l'ensemble de définition  de  la fonction ln, sont données ci-dessous:


\begin{property}
\begin{itemize}
\item $ \displaystyle\lim_{x \to \pinf}\ln x=\pinf $  
 \item   $\displaystyle \lim_{x \to 0^{+}}\ln x=\minf $
\end{itemize}
\end{property}

\textbf{Démonstration}
\begin{itemize}
\item La fonction  $ \ln  $ est croissante et  n'est pas   majorée sur  $ \intoo{0}{\pinf} $.\\
 Si elle était majorée sur $ \intoo{0}{\pinf} $, elle admettrait une limite finie $L$ en
$ \pinf $. En posant $X = 5x$, on obtiendrait :

$L= \displaystyle\lim_{X \to \pinf} \ln X = \displaystyle\lim_{x \to \pinf} \ln 5x = \displaystyle\lim_{x \to \pinf} \ln 5 + \ln x = \ln 5 + L$ , on aboutit à une contradiction.
\item  Pour la limite en $ 0^{+} $ , on fait le changement de variable $ X=\dfrac{1}{x} $\\
Donc $ \displaystyle\lim_{x \to 0}\ln x=\displaystyle\lim_{X \to \pinf}\ln \dfrac{1}{X}=\displaystyle\lim_{X \to  \pinf}\paren{-\ln X}=\minf $
\end{itemize}


\subsection*{Tableau de variation }
Des propriétés précédentes, on en déduit facilement le tableau de variation suivant.

\begin{center}
\begin{tikzpicture}
\tkzTab[lgt=2]
{
	$x$ / 0.5 ,
	$\exp^{\prime}(x)$ / 0.5,
	$\exp(x)$ /1
}
{ $\minf$ , $\pinf$ }
{ , + , }
{-/ $0$ , +/$\pinf$ }
\end{tikzpicture}

\end{center}



\subsubsection*{Conséquences}
La fonction $ \ln  $  est continue et strictement croissante sur $ \intoo{0}{\pinf} $  cela entraîne que c'est une bijection de $ \intoo{0}{\pinf} $  vers $ \ln \paren{  \intoo{0}{\pinf} }=\Rr $.

Donc $ \forall y \in \Rr $,  il existe un unique $ x\in \intoo{0}{\pinf} $ tel que $ \ln x=y. $  En particulier il existe un unique réel noté $ \mathrm{e} $ tel que:  $\ln \mathrm{e}=1 $.
On démontre que $ \mathrm{e}\approx 2,718 $    et que  $  \mathrm{e}\;\notin\; \Qq\qquad$ :  $ \mathrm{e }$ est appelé la  \textbf{ constante d'Euler}.\\
On a alors  $ \forall r \in \Qq, \; \ln \eexp{r}=r\ln \mathrm{e}=r $


Ainsi:


\begin{tabularx}{\textwidth}{|X|X|X|}
\hline
$ \ln x = r  \Longleftrightarrow x = \eexp{r} $&
$ \ln x  > r \Longleftrightarrow x >\eexp{r} $&
$ \ln x  < r\Longleftrightarrow x <   \eexp{r} $\\
\hline
\end{tabularx}

\begin{exercice}

Résoudre dans $ \Rr $  l'équation:\; $ \ln x+1=\dfrac{6}{\ln x} $
\end{exercice}
\begin{proof}
Cette équation est définie lorsque:\; $ x>  0$\;  et\;  $\ln x \neq 0 \quad$ \; c-à-d \; $\quad x\neq 1. $ \\

 Donc le domaine de résolution est  \;D$ =\intoo{0}{1} \cup \intoo{1}{\pinf} $

Si $ x \in $D alors  l'équation équivaut à  $ \paren{\ln x}^{2}+ \ln x= 6 $\\

Posons \quad $  X= \ln x $.\; On a:\;  $ X^{2}+X-6=0$\quad  soit \;  $X=2 $ \;ou \; $X=-3 $\\

c-à-d \;  $ \ln x=2 $\; ou \; $ \ln x=-3 $\\

D'où\; $x=\eexp{2} $   \; ou \; $x= \eexp{-3} $ \;  d'où \; S$ =\accol{\eexp{2},\;\eexp{-3} } $

\end{proof}

\subsection*{Représentation graphique de la fonction $ \ln $}
On construit les tangentes  T$ _{1} $   et T$ _{2} $ à la courbe de $ \ln $   respectives  aux points  d'abscisses $ x=1$ et $x= \mathrm{e} $.

 T$ _{1} \;: y= \ln'(1)(x-1) +\ln1$ \; soit \; T$ _{1}:\;y=x-1 $ 
 
  T$ _{2} \;: y= \ln'(\mathrm{e})(x-\mathrm{e}) +\ln \mathrm{e}$ \; soit \; T$ _{2}:\;y=\tfrac{1}{\mathrm{e}}x $ 
  \begin{center}
\begin{tikzpicture}[>=stealth', scale=0.5]
\clip (-2,-3) rectangle (8,4);
\draw[->,thick] (0,0) -- (1,0);
\draw[->,thick] (0,0) -- (0,1);
\draw[,thick] (-2,0) -- (8,0);
\draw[thick] (0,-3) -- (0,4);
\foreach\x in {1,2,}
{
\draw[thick] (\x,0.1) -- (\x,-0.1) node[below] {\x};
}
\foreach\y in {1}
{
\draw[thick] (0.1,\y) -- (-0.1,\y) node[left] {\y};
}

\draw[thick,red] plot[domain=0.01 :7,samples=100] (\x,{ln (\x)}) node[below right] {$\mathscr{C}$};
\draw[thick,green] plot[domain=-1 :6,samples=100] (\x,{\x-1}) node[below  left] {$T_{1}$};
\draw[thick,green] plot[domain=-1 :6,samples=100] (\x,{0.4*\x}) node[below  left] {};
\draw[dashed,thick](2.7,0)--(2.7,1);
\draw[dashed,thick](2.7,1)--(0,1);
\node at(2.7,-0.4) {$\mathrm{e}$};
\node at(7,2.6) {$T_{2} $};
\node at(4.5,3) {$T_{1} $};
\end{tikzpicture}
\end{center}

\subsection*{Dérivée de la fonction $ \ln|u| $}
Soit   $ u $ une fonction dérivable sur un intervalle $I$  et  telle que:  $ \forall x \in I ,\; u(x)\neq 0 $.\\
Donc la fonction  $ x :\;\longmapsto \abs{x} $ est dérivable sur I à valeurs dans $ \Rrep $\\
Donc la fonction  g$=\ln|u|  $  est définie et dérivable  sur I.\\
\underline{Si $u(x)> 0 \quad$}: g$ (x)=\ln u( x)\; $ et\; g'$ (x)= u'(x) \times \ln'\paren{u(x)}= u'(x)\times \dfrac{1}{u(x)} =\frac{u'(x)}{u(x)} $.

\underline{Si $u(x)< 0 \quad $}: g$ (x)=\ln\paren{-u(x)} $\; et\; g'$(x)=- u'(x) \times \ln'\paren{-u(x)} =\dfrac{-u'(x)}{-u(x)}= \dfrac{u'(x)}{u(x)}$.\\
Dans tous les cas: \quad


 $  \paren{\ln|u|}^{\prime}= \dfrac{u'}{u}$ 



\begin{corollary}
\begin{itemize}
\item
 Si $u$ est dérivable et strictement positive sur I alors la fonction $ \ln u $ est dérivable sur I.
 \item Il en résulte que les primitives de la fonction $\dfrac{u'}{u}$,
sont les fonctions du type $ln |u| + C$.
\end{itemize}
\end{corollary}

\subsection*{Quelques limites classiques}

\begin{property}
\begin{itemize}
\item $\displaystyle \lim_{x \to \pinf}\dfrac{\ln x}{x}=0 $
\item   $ \displaystyle\lim_{x \to 0}x\ln x=0 $  
\item  $ \displaystyle\lim_{x \to 0} \dfrac{\ln(x+1)}{x}=1 $
\end{itemize}
\end{property}

\textbf{Démonstration}
\begin{itemize}
\item   Montrons que $ \displaystyle\lim_{x \to \pinf}\dfrac{\ln x}{x}=0 $  \\  
 Soit  g la fonction définie sur $ \intfo{1}{\pinf} $ \; par\;  g$ (x)=\ln x- x+1 $.\\ g est dérivable sur $ \intfo{1}{\pinf} $\; et                                     \;  $ \forall x> 1  \quad g'(x)=\dfrac{1}{x}-1< 0 $\; et g est donc strictement  décroissante sur  $ \intfo{1}{\pinf} $\; or\;  g$ (1)=0 $.\\ 
 D'où \;$\forall x> 1\quad g(x)\leq  0 \quad$\; soit \; $\quad \ln x \leq x-1 $.\\
 En particulier \; $ \forall x> 1 $,\quad  $ \ln \sqrt{x} \leq \sqrt{x}-1 <\sqrt{x}\quad$. D'où \;$\dfrac{1}{2}\ln x <\sqrt{x}$ \;$ \Leftrightarrow \;0< \ln x <2\sqrt{x}$\\
 En divisant par $ x $, on a \;$ 0< \dfrac{\ln x}{x} < \dfrac{2}{\sqrt{x}}$\; or $ \displaystyle\lim_{x \to \pinf }\dfrac{2}{\sqrt{x}}=0 $  et d'après le théorème des gendarmes on a $ \displaystyle\lim_{x \to \pinf}\dfrac{\ln x}{x}=0 $.\\
 
\item  Montrons que\; $ \displaystyle\lim_{x \to 0^{+}}x\ln x=0 $\\
 En posant $ X=\dfrac{1}{x} $ ,\;on obtient $ \displaystyle\lim_{x \to 0^{+}}x\ln x  =\displaystyle\lim_{X \to \pinf}\paren{-\dfrac{\ln X}{X}} =0 $
 
 \item  Montrons que\;$ \displaystyle\lim_{x \to 0} \dfrac{\ln(x+1)}{x}=1 $\\
  Soit $ \varphi (x)=\ln (1+x)$.\; On a \; $ \varphi'(x)=\dfrac{1}{1+x} \;,\quad  \varphi(0)=0 $ \;et \; $ \varphi'(0)=1 $.\\
Par suite \quad   $\displaystyle \lim_{x \to 0}  \dfrac{\ln(x+1)}{x} =\displaystyle\lim_{x \to 0}  \dfrac{\varphi(x)-\varphi(0)}{x-0} = \varphi'(0)=1$
\end{itemize}


\subsection{Fonction logarithme décimal}
\begin{definition}
On appelle fonction logarithme décimal ( ou  de base $ 10 $), notée Log ou log , la  fonction définie  sur $ \intoo{0}{\pinf} $ par: $$ x  \longmapsto \log(x)=\dfrac{\ln x}{\ln 10} $$
\end{definition}


\begin{remark}
$ \log(1)=0 $ et $ \log(10)=1 $

\end{remark}

 
  %</content>
\end{document}
