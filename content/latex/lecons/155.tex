\documentclass[12pt, a4paper]{report}

% LuaLaTeX :

\RequirePackage{iftex}
\RequireLuaTeX

% Packages :

\usepackage[french]{babel}
%\usepackage[utf8]{inputenc}
%\usepackage[T1]{fontenc}
\usepackage[pdfencoding=auto, pdfauthor={Hugo Delaunay}, pdfsubject={Mathématiques}, pdfcreator={agreg.skyost.eu}]{hyperref}
\usepackage{amsmath}
\usepackage{amsthm}
%\usepackage{amssymb}
\usepackage{stmaryrd}
\usepackage{tikz}
\usepackage{tkz-euclide}
\usepackage{fontspec}
\defaultfontfeatures[Erewhon]{FontFace = {bx}{n}{Erewhon-Bold.otf}}
\usepackage{fourier-otf}
\usepackage[nobottomtitles*]{titlesec}
\usepackage{fancyhdr}
\usepackage{listings}
\usepackage{catchfilebetweentags}
\usepackage[french, capitalise, noabbrev]{cleveref}
\usepackage[fit, breakall]{truncate}
\usepackage[top=2.5cm, right=2cm, bottom=2.5cm, left=2cm]{geometry}
\usepackage{enumitem}
\usepackage{tocloft}
\usepackage{microtype}
%\usepackage{mdframed}
%\usepackage{thmtools}
\usepackage{xcolor}
\usepackage{tabularx}
\usepackage{xltabular}
\usepackage{aligned-overset}
\usepackage[subpreambles=true]{standalone}
\usepackage{environ}
\usepackage[normalem]{ulem}
\usepackage{etoolbox}
\usepackage{setspace}
\usepackage[bibstyle=reading, citestyle=draft]{biblatex}
\usepackage{xpatch}
\usepackage[many, breakable]{tcolorbox}
\usepackage[backgroundcolor=white, bordercolor=white, textsize=scriptsize]{todonotes}
\usepackage{luacode}
\usepackage{float}
\usepackage{needspace}
\everymath{\displaystyle}

% Police :

\setmathfont{Erewhon Math}

% Tikz :

\usetikzlibrary{calc}
\usetikzlibrary{3d}

% Longueurs :

\setlength{\parindent}{0pt}
\setlength{\headheight}{15pt}
\setlength{\fboxsep}{0pt}
\titlespacing*{\chapter}{0pt}{-20pt}{10pt}
\setlength{\marginparwidth}{1.5cm}
\setstretch{1.1}

% Métadonnées :

\author{agreg.skyost.eu}
\date{\today}

% Titres :

\setcounter{secnumdepth}{3}

\renewcommand{\thechapter}{\Roman{chapter}}
\renewcommand{\thesubsection}{\Roman{subsection}}
\renewcommand{\thesubsubsection}{\arabic{subsubsection}}
\renewcommand{\theparagraph}{\alph{paragraph}}

\titleformat{\chapter}{\huge\bfseries}{\thechapter}{20pt}{\huge\bfseries}
\titleformat*{\section}{\LARGE\bfseries}
\titleformat{\subsection}{\Large\bfseries}{\thesubsection \, - \,}{0pt}{\Large\bfseries}
\titleformat{\subsubsection}{\large\bfseries}{\thesubsubsection. \,}{0pt}{\large\bfseries}
\titleformat{\paragraph}{\bfseries}{\theparagraph. \,}{0pt}{\bfseries}

\setcounter{secnumdepth}{4}

% Table des matières :

\renewcommand{\cftsecleader}{\cftdotfill{\cftdotsep}}
\addtolength{\cftsecnumwidth}{10pt}

% Redéfinition des commandes :

\renewcommand*\thesection{\arabic{section}}
\renewcommand{\ker}{\mathrm{Ker}}

% Nouvelles commandes :

\newcommand{\website}{https://github.com/imbodj/SenCoursDeMaths}

\newcommand{\tr}[1]{\mathstrut ^t #1}
\newcommand{\im}{\mathrm{Im}}
\newcommand{\rang}{\operatorname{rang}}
\newcommand{\trace}{\operatorname{trace}}
\newcommand{\id}{\operatorname{id}}
\newcommand{\stab}{\operatorname{Stab}}
\newcommand{\paren}[1]{\left(#1\right)}
\newcommand{\croch}[1]{\left[ #1 \right]}
\newcommand{\Grdcroch}[1]{\Bigl[ #1 \Bigr]}
\newcommand{\grdcroch}[1]{\bigl[ #1 \bigr]}
\newcommand{\abs}[1]{\left\lvert #1 \right\rvert}
\newcommand{\limi}[3]{\lim_{#1\to #2}#3}
\newcommand{\pinf}{+\infty}
\newcommand{\minf}{-\infty}
%%%%%%%%%%%%%% ENSEMBLES %%%%%%%%%%%%%%%%%
\newcommand{\ensemblenombre}[1]{\mathbb{#1}}
\newcommand{\Nn}{\ensemblenombre{N}}
\newcommand{\Zz}{\ensemblenombre{Z}}
\newcommand{\Qq}{\ensemblenombre{Q}}
\newcommand{\Qqp}{\Qq^+}
\newcommand{\Rr}{\ensemblenombre{R}}
\newcommand{\Cc}{\ensemblenombre{C}}
\newcommand{\Nne}{\Nn^*}
\newcommand{\Zze}{\Zz^*}
\newcommand{\Zzn}{\Zz^-}
\newcommand{\Qqe}{\Qq^*}
\newcommand{\Rre}{\Rr^*}
\newcommand{\Rrp}{\Rr_+}
\newcommand{\Rrm}{\Rr_-}
\newcommand{\Rrep}{\Rr_+^*}
\newcommand{\Rrem}{\Rr_-^*}
\newcommand{\Cce}{\Cc^*}
%%%%%%%%%%%%%%  INTERVALLES %%%%%%%%%%%%%%%%%
\newcommand{\intff}[2]{\left[#1\;,\; #2\right]  }
\newcommand{\intof}[2]{\left]#1 \;, \;#2\right]  }
\newcommand{\intfo}[2]{\left[#1 \;,\; #2\right[  }
\newcommand{\intoo}[2]{\left]#1 \;,\; #2\right[  }

\providecommand{\newpar}{\\[\medskipamount]}

\newcommand{\annexessection}{%
  \newpage%
  \subsection*{Annexes}%
}

\providecommand{\lesson}[3]{%
  \title{#3}%
  \hypersetup{pdftitle={#2 : #3}}%
  \setcounter{section}{\numexpr #2 - 1}%
  \section{#3}%
  \fancyhead[R]{\truncate{0.73\textwidth}{#2 : #3}}%
}

\providecommand{\development}[3]{%
  \title{#3}%
  \hypersetup{pdftitle={#3}}%
  \section*{#3}%
  \fancyhead[R]{\truncate{0.73\textwidth}{#3}}%
}

\providecommand{\sheet}[3]{\development{#1}{#2}{#3}}

\providecommand{\ranking}[1]{%
  \title{Terminale #1}%
  \hypersetup{pdftitle={Terminale #1}}%
  \section*{Terminale #1}%
  \fancyhead[R]{\truncate{0.73\textwidth}{Terminale #1}}%
}

\providecommand{\summary}[1]{%
  \textit{#1}%
  \par%
  \medskip%
}

\tikzset{notestyleraw/.append style={inner sep=0pt, rounded corners=0pt, align=center}}

%\newcommand{\booklink}[1]{\website/bibliographie\##1}
\newcounter{reference}
\newcommand{\previousreference}{}
\providecommand{\reference}[2][]{%
  \needspace{20pt}%
  \notblank{#1}{
    \needspace{20pt}%
    \renewcommand{\previousreference}{#1}%
    \stepcounter{reference}%
    \label{reference-\previousreference-\thereference}%
  }{}%
  \todo[noline]{%
    \protect\vspace{20pt}%
    \protect\par%
    \protect\notblank{#1}{\cite{[\previousreference]}\\}{}%
    \protect\hyperref[reference-\previousreference-\thereference]{p. #2}%
  }%
}

\definecolor{devcolor}{HTML}{00695c}
\providecommand{\dev}[1]{%
  \reversemarginpar%
  \todo[noline]{
    \protect\vspace{20pt}%
    \protect\par%
    \bfseries\color{devcolor}\href{\website/developpements/#1}{[DEV]}
  }%
  \normalmarginpar%
}

% En-têtes :

\pagestyle{fancy}
\fancyhead[L]{\truncate{0.23\textwidth}{\thepage}}
\fancyfoot[C]{\scriptsize \href{\website}{\texttt{https://github.com/imbodj/SenCoursDeMaths}}}

% Couleurs :

\definecolor{property}{HTML}{ffeb3b}
\definecolor{proposition}{HTML}{ffc107}
\definecolor{lemma}{HTML}{ff9800}
\definecolor{theorem}{HTML}{f44336}
\definecolor{corollary}{HTML}{e91e63}
\definecolor{definition}{HTML}{673ab7}
\definecolor{notation}{HTML}{9c27b0}
\definecolor{example}{HTML}{00bcd4}
\definecolor{cexample}{HTML}{795548}
\definecolor{application}{HTML}{009688}
\definecolor{remark}{HTML}{3f51b5}
\definecolor{algorithm}{HTML}{607d8b}
%\definecolor{proof}{HTML}{e1f5fe}
\definecolor{exercice}{HTML}{e1f5fe}

% Théorèmes :

\theoremstyle{definition}
\newtheorem{theorem}{Théorème}

\newtheorem{property}[theorem]{Propriété}
\newtheorem{proposition}[theorem]{Proposition}
\newtheorem{lemma}[theorem]{Activité d'introduction}
\newtheorem{corollary}[theorem]{Conséquence}

\newtheorem{definition}[theorem]{Définition}
\newtheorem{notation}[theorem]{Notation}

\newtheorem{example}[theorem]{Exemple}
\newtheorem{cexample}[theorem]{Contre-exemple}
\newtheorem{application}[theorem]{Application}

\newtheorem{algorithm}[theorem]{Algorithme}
\newtheorem{exercice}[theorem]{Exercice}

\theoremstyle{remark}
\newtheorem{remark}[theorem]{Remarque}

\counterwithin*{theorem}{section}

\newcommand{\applystyletotheorem}[1]{
  \tcolorboxenvironment{#1}{
    enhanced,
    breakable,
    colback=#1!8!white,
    %right=0pt,
    %top=8pt,
    %bottom=8pt,
    boxrule=0pt,
    frame hidden,
    sharp corners,
    enhanced,borderline west={4pt}{0pt}{#1},
    %interior hidden,
    sharp corners,
    after=\par,
  }
}

\applystyletotheorem{property}
\applystyletotheorem{proposition}
\applystyletotheorem{lemma}
\applystyletotheorem{theorem}
\applystyletotheorem{corollary}
\applystyletotheorem{definition}
\applystyletotheorem{notation}
\applystyletotheorem{example}
\applystyletotheorem{cexample}
\applystyletotheorem{application}
\applystyletotheorem{remark}
%\applystyletotheorem{proof}
\applystyletotheorem{algorithm}
\applystyletotheorem{exercice}

% Environnements :

\NewEnviron{whitetabularx}[1]{%
  \renewcommand{\arraystretch}{2.5}
  \colorbox{white}{%
    \begin{tabularx}{\textwidth}{#1}%
      \BODY%
    \end{tabularx}%
  }%
}

% Maths :

\DeclareFontEncoding{FMS}{}{}
\DeclareFontSubstitution{FMS}{futm}{m}{n}
\DeclareFontEncoding{FMX}{}{}
\DeclareFontSubstitution{FMX}{futm}{m}{n}
\DeclareSymbolFont{fouriersymbols}{FMS}{futm}{m}{n}
\DeclareSymbolFont{fourierlargesymbols}{FMX}{futm}{m}{n}
\DeclareMathDelimiter{\VERT}{\mathord}{fouriersymbols}{152}{fourierlargesymbols}{147}

% Code :

\definecolor{greencode}{rgb}{0,0.6,0}
\definecolor{graycode}{rgb}{0.5,0.5,0.5}
\definecolor{mauvecode}{rgb}{0.58,0,0.82}
\definecolor{bluecode}{HTML}{1976d2}
\lstset{
  basicstyle=\footnotesize\ttfamily,
  breakatwhitespace=false,
  breaklines=true,
  %captionpos=b,
  commentstyle=\color{greencode},
  deletekeywords={...},
  escapeinside={\%*}{*)},
  extendedchars=true,
  frame=none,
  keepspaces=true,
  keywordstyle=\color{bluecode},
  language=Python,
  otherkeywords={*,...},
  numbers=left,
  numbersep=5pt,
  numberstyle=\tiny\color{graycode},
  rulecolor=\color{black},
  showspaces=false,
  showstringspaces=false,
  showtabs=false,
  stepnumber=2,
  stringstyle=\color{mauvecode},
  tabsize=2,
  %texcl=true,
  xleftmargin=10pt,
  %title=\lstname
}

\newcommand{\codedirectory}{}
\newcommand{\inputalgorithm}[1]{%
  \begin{algorithm}%
    \strut%
    \lstinputlisting{\codedirectory#1}%
  \end{algorithm}%
}



% Bibliographie :

%\addbibresource{\bibliographypath}%
\defbibheading{bibliography}[\bibname]{\section*{#1}}
\renewbibmacro*{entryhead:full}{\printfield{labeltitle}}%
\DeclareFieldFormat{url}{\newline\footnotesize\url{#1}}%

\AtEndDocument{%
  \newpage%
  \pagestyle{empty}%
  \printbibliography%
}


\begin{document}
  %<*content>
  \lesson{algebra}{155}{Exponentielle de matrices. Applications.}
  
  \subsection{Construction}
  
  Soit $\mathbb{K} = \mathbb{R}$ ou $\mathbb{C}$. Soit $n \geq 1$ un entier.
  
  \subsubsection{Algèbres de Banach}
  
  \reference[DAN]{278}
  
  \begin{lemma}
    Pour tout réel positif $a$, la série $\sum \frac{a^n}{n!}$ est convergente.
  \end{lemma}
  
  \reference{174}
  
  \begin{definition}
    Soit $\mathcal{A}$ une algèbre.
    \begin{itemize}
      \item On dit que $\Vert . \Vert$ est une norme d'algèbre sur $\mathcal{A}$ si :
      \begin{enumerate}[label=(\roman*)]
        \item $(\mathcal{A}, \Vert . \Vert)$ est un espace vectoriel normé.
        \item $\forall x, y \in \mathcal{A}, \, \Vert x \times y \Vert \leq \Vert x \Vert \Vert y \Vert$.
      \end{enumerate}
      \item Soit $\Vert . \Vert$ une norme d'algèbre sur $\mathcal{A}$. Si $(\mathcal{A}, \Vert . \Vert)$ est un espace vectoriel complet, on dit que $\mathcal{A}$ est une \textbf{algèbre de Banach}.
    \end{itemize}
  \end{definition}
  
  \reference{183}
  
  \begin{proposition}
    Soit $\Vert . \Vert$ une norme sur $\mathbb{K}^n$. Muni de la norme
    \[ \VERT . \VERT : M \mapsto \sup_{x \neq 0} \frac{\Vert Mx \Vert}{\Vert x \Vert} \]
    l'algèbre $(\mathcal{M}_n(\mathbb{K}), \VERT . \VERT)$ est une algèbre de Banach.
  \end{proposition}
  
  \begin{cexample}
    Ce n'est pas vrai pour n'importe quelle norme : la norme infinie $\Vert . \Vert_\infty$ sur $\mathcal{M}_n(\mathbb{K})$ n'est pas une norme d'algèbre.
  \end{cexample}
  
  \reference{278}
  
  \begin{proposition}
    Soit $\mathcal{A}$ une algèbre de Banach unitaire. Pour tout élément $A \in \mathcal{A}$, la série $\sum \frac{A^n}{n!}$ est convergente.
  \end{proposition}
  
  \subsubsection{Exponentielle de matrices}
  
  \reference{345}
  
  \begin{definition}
    Soit $A \in \mathcal{M}_n(\mathbb{K})$. On appelle \textbf{exponentielle} de $A$, et on note $\exp(A)$ ou $e^A$ l'élément de $\mathcal{M}_n(\mathbb{K})$ suivant :
    \[ \exp(A) = \sum_{n=0}^{+\infty} \frac{A^n}{n!} \]
  \end{definition}
  
  \reference{356}
  
  \begin{example}
    Soient $a_1, \dots, a_n \in \mathbb{K}$ et $D = \operatorname{Diag}(a_1, \dots, a_n) \in \mathcal{M}_3(\mathbb{K})$. Alors,
    \[ \exp(D) = \operatorname{Diag}(e^{a_1}, \dots, e^{a_n}) \]
  \end{example}
  
  \reference[GRI]{378}
  
  \begin{remark}
    En particulier, $\exp(0) = I_n$.
  \end{remark}
  
  \subsubsection{Propriétés}
  
  \begin{proposition}
    Soient $A, B \in \mathcal{M}_n(\mathbb{K})$ qui commutent. Alors,
    \[ e^{A+B} = e^{A} e^{B} \]
  \end{proposition}
  
  \begin{corollary}
    Soit $A \in \mathcal{M}_n(\mathbb{K})$. Alors, $e^A \in \mathrm{GL}_n(\mathbb{K})$ et,
    \[ \left( e^A \right)^{-1} = e^{-A} \]
  \end{corollary}
  
  \begin{proposition}
    Soient $A, B \in \mathcal{M}_n(\mathbb{K})$ telles que $B = PAP^{-1}$ pour $P \in \mathrm{GL}_n(\mathbb{K})$. Alors,
    \[ e^{PAP^{-1}} = Pe^AP^{-1} \]
  \end{proposition}
  
  \begin{lemma}
    Soit $A \in \mathcal{M}_n(\mathbb{K})$ une matrice triangulaire supérieure, de la forme $A = \begin{pmatrix} \lambda_1 & & * \\ & \ddots & \\ & & \lambda_n \end{pmatrix}$. Alors,
    \[
      e^A =
      \begin{pmatrix}
        e^\lambda_1 & & * \\
        & \ddots & \\
        & & e^\lambda_n
      \end{pmatrix}
    \]
  \end{lemma}
  
  \reference[ROM21]{762}
  
  \begin{proposition}
    Soit $A \in \mathcal{M}_n(\mathbb{K})$. Alors,
    \[ \det(\exp(A)) = e^{\trace(A)} \]
  \end{proposition}
  
  \begin{proposition}
    $\exp : \mathcal{M}_n(\mathbb{K}) \rightarrow \mathrm{GL}_n(\mathbb{K})$ est continue. De plus, pour tout $A \in \mathcal{M}_n(\mathbb{K})$, $\exp(A)$ est un polynôme en $A$.
  \end{proposition}

  \subsection{Calcul pratique}
  
  \reference[GOU21]{206}
  
  \begin{proposition}
    Soit $N \in \mathcal{M}_n(\mathbb{K})$ nilpotente d'indice $q$. Alors,
    \[ e^N = \sum_{k=0}^{q-1} \frac{A^k}{k!} \]
  \end{proposition}
  
  \begin{theorem}[Décomposition de Dunford]
    Soit $A \in \mathcal{M}_n(\mathbb{K})$. On suppose que $\pi_A$ est scindé sur $\mathbb{K}$. Alors il existe un unique couple de matrices $(D, N)$ tels que :
    \begin{itemize}
      \item $D$ est diagonalisable et $N$ est nilpotente.
      \item $A = D + N$.
      \item $DN = ND$.
    \end{itemize}
  \end{theorem}
  
  \begin{corollary}
    Si $A$ vérifie les hypothèse précédentes, pour tout $k \in \mathbb{N}$, $A^k = (D + N)^k = \sum_{i=0}^m \binom{k}{i} D^i N^{k-i}$, avec $m = \min(k, l)$ où $l$ désigne l'indice de nilpotence de $N$.
  \end{corollary}
  
  \reference[ROM21]{765}
  
  \begin{example}
    Soit $A \in \mathcal{M}_n(\mathbb{K})$ qui admet une décomposition de Dunford $A = D+N$ où $D$ est diagonalisable et $N$ est nilpotente d'indice $q$. Alors,
    \begin{itemize}
      \item $e^A = e^D e^N = e^D \sum_{k=0}^{q-1} \frac{N^k}{k!}$.
      \item La décomposition de Dunford de $e^A$ est $e^A = e^D + e^D(e^N - I_n)$ avec $e^D$ diagonalisable et $e^D(e^N - I_n)$ nilpotente.
    \end{itemize}
  \end{example}
  
  \begin{application}
    Soit $A \in \mathcal{M}_n(\mathbb{K})$ dont le polynôme caractéristique est scindé sur $\mathbb{K}$. Alors $A$ est diagonalisable si et seulement si $e^A$ l'est.
  \end{application}
  
  \reference[GOU21]{209}
  
  \begin{example}
    On a
    \[ \exp \left( \begin{pmatrix} 1 & 4 & -2 \\ 0 & 6 & -3 \\ -1 & 4 & 0 \end{pmatrix} \right) = \begin{pmatrix} -6e^2 + 3e^3 & -4e^2 + 4e^3 & 10e^2 - 6e^3 \\ -6e^2 + 3e^3 & -3e^2 + 4e^3 & 9e^2 - 6e^3 \\ -7e^2 + 3e^3 & -4e^2 + 4e^3 & 11e^2 - 6e^3 \end{pmatrix} \]
  \end{example}
  
  \subsection{Étude de l'exponentielle de matrices}
  
  \subsubsection{Dérivabilité, différentiabilité}
  
  \reference{195}
  
  \begin{proposition}
    Soit $A \in \mathcal{M}_n(\mathbb{K})$. L'application $t \mapsto e^{tA}$ est dérivable, de dérivée $t \mapsto Ae^{tA}$.
  \end{proposition}
  
  \reference[C-G]{384}
  
  \begin{proposition}[Logarithme matriciel]
    \label{155-1}
    $\exp$ est différentiable en $0$ et sa différentielle est $I_n$ ; c'est un difféomorphisme local sur un voisinage de $0$. Plus précisément, si $H \in \mathcal{M}_n(\mathbb{K})$ telle que $\VERT H \VERT \leq 1$, alors
    \[ \exp^{-1}(I_n + H) = \sum_{n=1}^{+\infty} (-1)^{n-1} \frac{H^n}{n} \]
    On note alors $\ln(H) = \exp^{-1}(H)$.
  \end{proposition}
  
  \reference[ROM21]{762}
  
  \begin{theorem}
    $\exp$ est de classe $\mathcal{C}^1$ sur $\mathcal{M}_n(\mathbb{K})$ avec, pour toutes matrices $A, H \in \mathcal{M}_n(\mathbb{K})$ :
    \[ \mathrm{d}\exp_A(H) = \sum_{n=1}^{+\infty} \frac{1}{n!} \left( \sum_{\substack{i,j \in \llbracket 0, n-1 \rrbracket \\ i+j = n-1}} A^i H A^j \right) \]
  \end{theorem}
  
  \subsubsection{Image directe}
  
  \paragraph{Image de \texorpdfstring{$\mathcal{M}_n(\mathbb{C})$}{Mₙ(C)}}
  
  \reference[C-G]{387}
  
  \begin{example}
    \[ \forall k \in \mathbb{Z}, \, e^{2ik\pi} = e^0 = 1 \]
    En particulier, $\exp$ n'est pas injective pour $n \geq 1$.
  \end{example}
  
  \reference[I-P]{396}
  
  \begin{lemma}
    Soit $M \in \mathrm{GL}_n(\mathbb{C})$. Alors $M^{-1} \in \mathbb{C}[M]$.
  \end{lemma}
  
  \begin{theorem}
    $\exp : \mathcal{M}_n(\mathbb{C}) \rightarrow \mathrm{GL}_n(\mathbb{C})$ est surjective.
  \end{theorem}
  
  \begin{application}
    $\exp(\mathcal{M}_n(\mathbb{R})) = \mathrm{GL}_n(\mathbb{R})^2$, où $\mathrm{GL}_n(\mathbb{R})^2$ désigne les carrés de $\mathrm{GL}_n(\mathbb{R})$.
  \end{application}
  
  \reference[ROM21]{770}
  
  \begin{application}
    $\mathrm{GL}_n(\mathbb{C})$ est connexe par arcs.
  \end{application}
  
  \paragraph{Image de \texorpdfstring{$\mathcal{M}_n(\mathbb{R})$}{Mₙ(R)}}
  
  \reference[C-G]{387}
  
  \begin{example}
    \[ \forall k \in \mathbb{Z}, \, \exp \left( \begin{pmatrix} 0 & -2k\pi \\ 2k\pi & 0 \end{pmatrix} \right) = \exp(0) = I_2 \]
    En particulier, $\exp$ n'est pas injective pour $n \geq 2$.
  \end{example}
  
  \begin{proposition}
    En fait,
    \[ \exp(\mathcal{M}_n(\mathbb{R})) = \{ M^2 \mid M \in \mathcal{M}_n(\mathbb{R}) \} \]
  \end{proposition}
  
  \begin{example}
    La matrice $\begin{pmatrix} -1 & 0 \\ 0 & -2 \end{pmatrix}$ n'est pas dans l'image de l'exponentielle réelle.
  \end{example}
  
  \paragraph{Images de $\mathcal{S}_n(\mathbb{R})$ et $\mathcal{H}_n(\mathbb{C})$}
  
  \reference[I-P]{182}
  
  \begin{lemma}
    Soit $M \in \mathcal{S}_n(\mathbb{R})$. Alors,
    \[ \VERT M \VERT = \rho(M) \]
    où $\rho$ est l'application qui a une matrice y associe son rayon spectral.
  \end{lemma}
  
  \dev{homeomorphisme-de-l-exponentielle}
  
  \begin{theorem}
    L'application $\exp : \mathcal{S}_n(\mathbb{R}) \rightarrow \mathcal{S}^{++}_n(\mathbb{R})$ est un homéomorphisme.
  \end{theorem}
  
  \reference[C-G]{385}
  
  \begin{remark}
    On a le même résultat pour $\exp : \mathcal{H}_n(\mathbb{C}) \rightarrow \mathcal{H}^{++}_n(\mathbb{C})$.
  \end{remark}
  
  \begin{application}
    On a des homéomorphismes :
    \[ \mathrm{GL}_n(\mathbb{R}) \sim \mathcal{O}_n(\mathbb{R}) \times \mathbb{R}^{\frac{n(n+1)}{2}} \text{ et } \mathrm{GL}_n(\mathbb{C}) \sim \mathcal{U}_n(\mathbb{C}) \times \mathbb{R}^{n^2} \]
  \end{application}
  
  \paragraph{Image du cône nilpotent $\mathcal{N}_n(\mathbb{C})$}
  
  \reference[ROM21]{766}
  
  \begin{notation}
    On note $\mathcal{N}_n(\mathbb{C})$ le sous-ensemble de $\mathcal{M}_n(\mathbb{C})$ formé des matrices nilpotentes et $\mathcal{L}_n(\mathbb{C}) = \mathcal{N}_n(\mathbb{C}) - I_n$ le sous-ensemble de $\mathcal{M}_n(\mathbb{C})$ formé des matrices unipotentes.
  \end{notation}
  
  \begin{proposition}
    Soit $A \in \mathcal{N}_n(\mathbb{C})$. Alors $e^A \in \mathcal{L}_n(\mathbb{C})$ et $\ln(e^{tA}) = tA$ pour tout $t \in \mathbb{R}$.
  \end{proposition}
  
  \begin{theorem}
    L'exponentielle matricielle réalise une bijection de $\mathcal{N}_n(\mathbb{C})$ sur $\mathcal{L}_n(\mathbb{C})$ d'inverse le logarithme matriciel défini à la \cref{155-1}.
  \end{theorem}
  
  \subsection{Applications}
  
  \subsubsection{Équations différentielles}
  
  \reference[GOU20]{376}
  
  \begin{theorem}[Cauchy-Lipschitz linéaire]
    Soient $A : I \rightarrow \mathcal{M}_n(\mathbb{K})$ et $B : I \rightarrow \mathbb{K}^d$ deux fonctions continues. Alors $\forall t_0 \in I$, le problème de Cauchy
    \[ \begin{cases} Y' = A(t)Y + B(t) \\ Y(t_0) = y_0 \end{cases} \]
    admet une unique solution définie sur $I$ tout entier.
  \end{theorem}
  
  \begin{proposition}
    Une équation différentielle linéaire homogène $Y' = AY$ (où $A \in \mathcal{M}_n(\mathbb{R})$ est constante en $t$) a ses solutions maximales définies sur $\mathbb{R}$ et le problème de Cauchy
    \[ \begin{cases} Y' = AY \\ Y(0) = y_0 \end{cases} \]
    a pour (unique) solution $t \mapsto e^{tA} y_0$.
  \end{proposition}
  
  \begin{example}
    Les solutions de
    \[ Y' = \begin{pmatrix} 1 & 0 & 1 \\ 0 & -1 & -1 \\ 0 & 2 & 1 \end{pmatrix} Y \]
    sont les
    \[ t \mapsto \alpha e^t \begin{pmatrix} 1 \\ 0 \\ 0 \end{pmatrix} + \beta e^{it} \begin{pmatrix} 1+i \\ 1-i \\ -2 \end{pmatrix} + \gamma e^{-it} \begin{pmatrix} 1-i \\ 1+i \\ -2 \end{pmatrix} \]
    où $\alpha, \beta, \gamma \in \mathbb{C}$.
  \end{example}
  
  \subsubsection{Équations matricielles}
  
  \reference[I-P]{177}
  
  \begin{lemma}
    Soit $\Vert . \Vert$ une norme d'algèbre sur $\mathcal{M}_n(\mathbb{C})$, et soit $A \in \mathcal{M}_n(\mathbb{C})$ une matrice dont les valeurs propres sont de partie réelle strictement négative. Alors il existe une fonction polynômiale $P : \mathbb{R} \rightarrow \mathbb{R}$ et $\lambda > 0$ tels que $\Vert e^{tA} \Vert \leq e^{- \lambda t} P(t)$.
  \end{lemma}
  
  \dev{equation-de-sylvester}
  
  \begin{application}[Équation de Sylvester]
    Soient $A$ et $B \in \mathcal{M}_n(\mathbb{C})$ deux matrices dont les valeurs propres sont de partie réelle strictement négative. Alors pour tout $C \in \mathcal{M}_n(\mathbb{C})$, l'équation $AX + XB = C$ admet une unique solution $X$ dans $\mathcal{M}_n(\mathbb{C})$.
  \end{application}
  %</content>
\end{document}
