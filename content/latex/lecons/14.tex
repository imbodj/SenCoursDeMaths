\documentclass[12pt, a4paper]{report}

% LuaLaTeX :

\RequirePackage{iftex}
\RequireLuaTeX

% Packages :

\usepackage[french]{babel}
%\usepackage[utf8]{inputenc}
%\usepackage[T1]{fontenc}
\usepackage[pdfencoding=auto, pdfauthor={Hugo Delaunay}, pdfsubject={Mathématiques}, pdfcreator={agreg.skyost.eu}]{hyperref}
\usepackage{amsmath}
\usepackage{amsthm}
%\usepackage{amssymb}
\usepackage{stmaryrd}
\usepackage{tikz}
\usepackage{tkz-euclide}
\usepackage{fontspec}
\defaultfontfeatures[Erewhon]{FontFace = {bx}{n}{Erewhon-Bold.otf}}
\usepackage{fourier-otf}
\usepackage[nobottomtitles*]{titlesec}
\usepackage{fancyhdr}
\usepackage{listings}
\usepackage{catchfilebetweentags}
\usepackage[french, capitalise, noabbrev]{cleveref}
\usepackage[fit, breakall]{truncate}
\usepackage[top=2.5cm, right=2cm, bottom=2.5cm, left=2cm]{geometry}
\usepackage{enumitem}
\usepackage{tocloft}
\usepackage{microtype}
%\usepackage{mdframed}
%\usepackage{thmtools}
\usepackage{xcolor}
\usepackage{tabularx}
\usepackage{xltabular}
\usepackage{aligned-overset}
\usepackage[subpreambles=true]{standalone}
\usepackage{environ}
\usepackage[normalem]{ulem}
\usepackage{etoolbox}
\usepackage{setspace}
\usepackage[bibstyle=reading, citestyle=draft]{biblatex}
\usepackage{xpatch}
\usepackage[many, breakable]{tcolorbox}
\usepackage[backgroundcolor=white, bordercolor=white, textsize=scriptsize]{todonotes}
\usepackage{luacode}
\usepackage{float}
\usepackage{needspace}
\everymath{\displaystyle}

% Police :

\setmathfont{Erewhon Math}

% Tikz :

\usetikzlibrary{calc}
\usetikzlibrary{3d}

% Longueurs :

\setlength{\parindent}{0pt}
\setlength{\headheight}{15pt}
\setlength{\fboxsep}{0pt}
\titlespacing*{\chapter}{0pt}{-20pt}{10pt}
\setlength{\marginparwidth}{1.5cm}
\setstretch{1.1}

% Métadonnées :

\author{agreg.skyost.eu}
\date{\today}

% Titres :

\setcounter{secnumdepth}{3}

\renewcommand{\thechapter}{\Roman{chapter}}
\renewcommand{\thesubsection}{\Roman{subsection}}
\renewcommand{\thesubsubsection}{\arabic{subsubsection}}
\renewcommand{\theparagraph}{\alph{paragraph}}

\titleformat{\chapter}{\huge\bfseries}{\thechapter}{20pt}{\huge\bfseries}
\titleformat*{\section}{\LARGE\bfseries}
\titleformat{\subsection}{\Large\bfseries}{\thesubsection \, - \,}{0pt}{\Large\bfseries}
\titleformat{\subsubsection}{\large\bfseries}{\thesubsubsection. \,}{0pt}{\large\bfseries}
\titleformat{\paragraph}{\bfseries}{\theparagraph. \,}{0pt}{\bfseries}

\setcounter{secnumdepth}{4}

% Table des matières :

\renewcommand{\cftsecleader}{\cftdotfill{\cftdotsep}}
\addtolength{\cftsecnumwidth}{10pt}

% Redéfinition des commandes :

\renewcommand*\thesection{\arabic{section}}
\renewcommand{\ker}{\mathrm{Ker}}

% Nouvelles commandes :

\newcommand{\website}{https://github.com/imbodj/SenCoursDeMaths}

\newcommand{\tr}[1]{\mathstrut ^t #1}
\newcommand{\im}{\mathrm{Im}}
\newcommand{\rang}{\operatorname{rang}}
\newcommand{\trace}{\operatorname{trace}}
\newcommand{\id}{\operatorname{id}}
\newcommand{\stab}{\operatorname{Stab}}
\newcommand{\paren}[1]{\left(#1\right)}
\newcommand{\croch}[1]{\left[ #1 \right]}
\newcommand{\Grdcroch}[1]{\Bigl[ #1 \Bigr]}
\newcommand{\grdcroch}[1]{\bigl[ #1 \bigr]}
\newcommand{\abs}[1]{\left\lvert #1 \right\rvert}
\newcommand{\limi}[3]{\lim_{#1\to #2}#3}
\newcommand{\pinf}{+\infty}
\newcommand{\minf}{-\infty}
%%%%%%%%%%%%%% ENSEMBLES %%%%%%%%%%%%%%%%%
\newcommand{\ensemblenombre}[1]{\mathbb{#1}}
\newcommand{\Nn}{\ensemblenombre{N}}
\newcommand{\Zz}{\ensemblenombre{Z}}
\newcommand{\Qq}{\ensemblenombre{Q}}
\newcommand{\Qqp}{\Qq^+}
\newcommand{\Rr}{\ensemblenombre{R}}
\newcommand{\Cc}{\ensemblenombre{C}}
\newcommand{\Nne}{\Nn^*}
\newcommand{\Zze}{\Zz^*}
\newcommand{\Zzn}{\Zz^-}
\newcommand{\Qqe}{\Qq^*}
\newcommand{\Rre}{\Rr^*}
\newcommand{\Rrp}{\Rr_+}
\newcommand{\Rrm}{\Rr_-}
\newcommand{\Rrep}{\Rr_+^*}
\newcommand{\Rrem}{\Rr_-^*}
\newcommand{\Cce}{\Cc^*}
%%%%%%%%%%%%%%  INTERVALLES %%%%%%%%%%%%%%%%%
\newcommand{\intff}[2]{\left[#1\;,\; #2\right]  }
\newcommand{\intof}[2]{\left]#1 \;, \;#2\right]  }
\newcommand{\intfo}[2]{\left[#1 \;,\; #2\right[  }
\newcommand{\intoo}[2]{\left]#1 \;,\; #2\right[  }

\providecommand{\newpar}{\\[\medskipamount]}

\newcommand{\annexessection}{%
  \newpage%
  \subsection*{Annexes}%
}

\providecommand{\lesson}[3]{%
  \title{#3}%
  \hypersetup{pdftitle={#2 : #3}}%
  \setcounter{section}{\numexpr #2 - 1}%
  \section{#3}%
  \fancyhead[R]{\truncate{0.73\textwidth}{#2 : #3}}%
}

\providecommand{\development}[3]{%
  \title{#3}%
  \hypersetup{pdftitle={#3}}%
  \section*{#3}%
  \fancyhead[R]{\truncate{0.73\textwidth}{#3}}%
}

\providecommand{\sheet}[3]{\development{#1}{#2}{#3}}

\providecommand{\ranking}[1]{%
  \title{Terminale #1}%
  \hypersetup{pdftitle={Terminale #1}}%
  \section*{Terminale #1}%
  \fancyhead[R]{\truncate{0.73\textwidth}{Terminale #1}}%
}

\providecommand{\summary}[1]{%
  \textit{#1}%
  \par%
  \medskip%
}

\tikzset{notestyleraw/.append style={inner sep=0pt, rounded corners=0pt, align=center}}

%\newcommand{\booklink}[1]{\website/bibliographie\##1}
\newcounter{reference}
\newcommand{\previousreference}{}
\providecommand{\reference}[2][]{%
  \needspace{20pt}%
  \notblank{#1}{
    \needspace{20pt}%
    \renewcommand{\previousreference}{#1}%
    \stepcounter{reference}%
    \label{reference-\previousreference-\thereference}%
  }{}%
  \todo[noline]{%
    \protect\vspace{20pt}%
    \protect\par%
    \protect\notblank{#1}{\cite{[\previousreference]}\\}{}%
    \protect\hyperref[reference-\previousreference-\thereference]{p. #2}%
  }%
}

\definecolor{devcolor}{HTML}{00695c}
\providecommand{\dev}[1]{%
  \reversemarginpar%
  \todo[noline]{
    \protect\vspace{20pt}%
    \protect\par%
    \bfseries\color{devcolor}\href{\website/developpements/#1}{[DEV]}
  }%
  \normalmarginpar%
}

% En-têtes :

\pagestyle{fancy}
\fancyhead[L]{\truncate{0.23\textwidth}{\thepage}}
\fancyfoot[C]{\scriptsize \href{\website}{\texttt{https://github.com/imbodj/SenCoursDeMaths}}}

% Couleurs :

\definecolor{property}{HTML}{ffeb3b}
\definecolor{proposition}{HTML}{ffc107}
\definecolor{lemma}{HTML}{ff9800}
\definecolor{theorem}{HTML}{f44336}
\definecolor{corollary}{HTML}{e91e63}
\definecolor{definition}{HTML}{673ab7}
\definecolor{notation}{HTML}{9c27b0}
\definecolor{example}{HTML}{00bcd4}
\definecolor{cexample}{HTML}{795548}
\definecolor{application}{HTML}{009688}
\definecolor{remark}{HTML}{3f51b5}
\definecolor{algorithm}{HTML}{607d8b}
%\definecolor{proof}{HTML}{e1f5fe}
\definecolor{exercice}{HTML}{e1f5fe}

% Théorèmes :

\theoremstyle{definition}
\newtheorem{theorem}{Théorème}

\newtheorem{property}[theorem]{Propriété}
\newtheorem{proposition}[theorem]{Proposition}
\newtheorem{lemma}[theorem]{Activité d'introduction}
\newtheorem{corollary}[theorem]{Conséquence}

\newtheorem{definition}[theorem]{Définition}
\newtheorem{notation}[theorem]{Notation}

\newtheorem{example}[theorem]{Exemple}
\newtheorem{cexample}[theorem]{Contre-exemple}
\newtheorem{application}[theorem]{Application}

\newtheorem{algorithm}[theorem]{Algorithme}
\newtheorem{exercice}[theorem]{Exercice}

\theoremstyle{remark}
\newtheorem{remark}[theorem]{Remarque}

\counterwithin*{theorem}{section}

\newcommand{\applystyletotheorem}[1]{
  \tcolorboxenvironment{#1}{
    enhanced,
    breakable,
    colback=#1!8!white,
    %right=0pt,
    %top=8pt,
    %bottom=8pt,
    boxrule=0pt,
    frame hidden,
    sharp corners,
    enhanced,borderline west={4pt}{0pt}{#1},
    %interior hidden,
    sharp corners,
    after=\par,
  }
}

\applystyletotheorem{property}
\applystyletotheorem{proposition}
\applystyletotheorem{lemma}
\applystyletotheorem{theorem}
\applystyletotheorem{corollary}
\applystyletotheorem{definition}
\applystyletotheorem{notation}
\applystyletotheorem{example}
\applystyletotheorem{cexample}
\applystyletotheorem{application}
\applystyletotheorem{remark}
%\applystyletotheorem{proof}
\applystyletotheorem{algorithm}
\applystyletotheorem{exercice}

% Environnements :

\NewEnviron{whitetabularx}[1]{%
  \renewcommand{\arraystretch}{2.5}
  \colorbox{white}{%
    \begin{tabularx}{\textwidth}{#1}%
      \BODY%
    \end{tabularx}%
  }%
}

% Maths :

\DeclareFontEncoding{FMS}{}{}
\DeclareFontSubstitution{FMS}{futm}{m}{n}
\DeclareFontEncoding{FMX}{}{}
\DeclareFontSubstitution{FMX}{futm}{m}{n}
\DeclareSymbolFont{fouriersymbols}{FMS}{futm}{m}{n}
\DeclareSymbolFont{fourierlargesymbols}{FMX}{futm}{m}{n}
\DeclareMathDelimiter{\VERT}{\mathord}{fouriersymbols}{152}{fourierlargesymbols}{147}

% Code :

\definecolor{greencode}{rgb}{0,0.6,0}
\definecolor{graycode}{rgb}{0.5,0.5,0.5}
\definecolor{mauvecode}{rgb}{0.58,0,0.82}
\definecolor{bluecode}{HTML}{1976d2}
\lstset{
  basicstyle=\footnotesize\ttfamily,
  breakatwhitespace=false,
  breaklines=true,
  %captionpos=b,
  commentstyle=\color{greencode},
  deletekeywords={...},
  escapeinside={\%*}{*)},
  extendedchars=true,
  frame=none,
  keepspaces=true,
  keywordstyle=\color{bluecode},
  language=Python,
  otherkeywords={*,...},
  numbers=left,
  numbersep=5pt,
  numberstyle=\tiny\color{graycode},
  rulecolor=\color{black},
  showspaces=false,
  showstringspaces=false,
  showtabs=false,
  stepnumber=2,
  stringstyle=\color{mauvecode},
  tabsize=2,
  %texcl=true,
  xleftmargin=10pt,
  %title=\lstname
}

\newcommand{\codedirectory}{}
\newcommand{\inputalgorithm}[1]{%
  \begin{algorithm}%
    \strut%
    \lstinputlisting{\codedirectory#1}%
  \end{algorithm}%
}



\everymath{\displaystyle}
\begin{document}
  %<*content>
  \lesson{algebra}{14}{Nombres complexes}
\textbf{Bref historique des nombres complexes}\\
L’apparition des racines carrées de nombres négatifs remonte à Héron d’Alexandrie (env.~75--150~apr.~J.-C.), qui, en étudiant le volume d’un tronc de pyramide, rencontre l’expression $ \sqrt{81 - 144} $, représentant une quantité non réelle.

Au \textsc{xvi}\textsuperscript{e} siècle, en Italie, les mathématiciens s’affrontent dans des concours autour de la résolution des équations du troisième degré. Tartaglia (1499--1557) découvre une méthode qu’il transmet à Cardano (1501--1576), lequel introduit des expressions du type $ a + \sqrt{-b} $, où $a \in \mathbb{R}$ et $b \in \mathbb{R}^+$. Cette seconde partie, ne pouvant être additionnée à la première, sera appelée \textit{partie imaginaire}.

Descartes (1596--1650) popularise le terme \og imaginaire\fg{}, et Wallis (1616--1703) en propose une interprétation géométrique. Puis, Euler (1707--1783) introduit le symbole $ \text{i} = \sqrt{-1} $, appelé \textit{l’unité imaginaire}.

\medskip

Aujourd’hui, les nombres complexes sont essentiels en sciences, notamment pour modéliser le courant alternatif en électrotechnique, et en télécommunications (WiFi, GPS, téléphonie), où les signaux sont traités à l’aide de calculs sur les nombres complexes.

\begin{lemma}
\begin{enumerate}
        \item L'ensemble de nombres le plus simple est celui de nombres entiers naturels, noté $\mathbb{N}$ et qui contient les nombres que vous connaissez depuis longtemps : $0$ ; 1 ; 2 ; 3...
        \begin{enumerate}
            \item Quel est le nombre entier naturel qui ajouté à 7 donne 12 ?
            \item Quel est le nombre entier naturel qui ajouté à 12 donne 7 ?
        \end{enumerate}
        \item L'exemple précédent montre que l'ensemble $\mathbb{N}$ est << insuffisant>>  car certaines équations simples n'y trouvent pas de solution. On peut alors utiliser l'ensemble des entiers relatifs, noté $\mathbb{Z}$, et qui contient $\mathbb{N}$ et les opposés des entiers naturels (par exemple : $-3~;~ -2$).
        \begin{enumerate}
            \item  Résoudre dans $\mathbb{N}$ puis dans $\mathbb{Z}$ l'équation : $2x+8=0$.
            \item  Même question avec l'équation : $2x+7=0$.
        \end{enumerate}
        \item De nouveau l'ensemble $\mathbb{Z}$ est en quelque sorte insuffisant pour exprimer les solutions de certaines équations.
        \begin{enumerate}
            \item De quel autre ensemble de nombres a-t-on au minimum besoin pour que l'équation du $2x+7=0$ ait une solution ?
            \item Dans ce nouvel ensemble quelles sont les solutions de l'équation : $9x^2=16$ ?
            %\item Dans quel autre nouvel ensemble faut-il se placer pour que l'équation ait-des solutions ?
            \item Décrire l'ensemble de nombres dont on a besoin au minimum pour que l'équation précédente ait une solution. On notera $\mathbb{Q}$ cet ensemble.
        \end{enumerate}
        \item Modifier l'équation précédente pour qu'elle n'admette pas de solution dans l'ensemble des rationnels. Dans quel ensemble faut-il travailler pour pouvoir dire qu'elle a deux solutions ?
        \item Que pouvez-vous dire de l'équation  $x^2+1=0$ en terme de solutions dans les ensembles de nombres précédents ?
        \item Dresser un schéma qui montre les inclusions successives des
ensembles de nombres en donnant à chaque fois une équation qui n'a pas de solution dans
l'ensemble, mais en a une dans le suivant.
\end{enumerate}
\end{lemma}
\begin{lemma}
On considère l'équation du second degré suivant : $\; x^2+4=0 $.
\begin{enumerate}
\item Peut-on trouver  des nombres réels solutions de l'équation ? Expliquer pourquoi.

\medskip

Les mathématiciens définissent  le nombre imaginaire i tel que  $ \;\i^2=-1 $ , et que donc $ \i=\sqrt{-1} $.

\item Peut-on utiliser ce fait pour résoudre l'équation, en exprimer la réponse en fonction de i ?
\item Utiliser la forme canonique pour résoudre l'équation  $\; x^2-2x+5=0 $. Donner les racines en fonctions de i.
\end{enumerate}
\end{lemma}
 \subsection{Forme algébrique d'un nombre complexe}
 \begin{definition}
Un nombre complexe $ z $ est un nombre qui peut s'écrire  sous la forme $ z=a+\i b $  où $a $ et $ b$  sont des réels et $ \i $ un nombre imaginaire tel que $ \i^2=-1 $.
 
 \end{definition}

\medskip
\textbf{Vocabulaire}

\medskip

$ a $ est appelé \textbf{partie réelle} du nombre complexe $ z $. On note $ a = Re(z)$. 
\medskip

$ b $  est appelé \textbf{partie imaginaire} du nombre complexe $ z$ . On note $ b = Im(z)$.

\medskip
L'écriture $ a+ \i b$  est appelée  la \textbf{forme algébrique}  (ou cartésienne) du nombre complexe $ z $.

\medskip
\begin{remark}
 Les parties réelle et imaginaire d'un nombre complexe sont des nombres réels.
 \end{remark}
 \medskip
 
 Un nombre complexe de la forme \textbf{ib} est appelé un \textbf{imaginaire pur}.
 
 On note  $ \i\mathbb{R} $ l'ensemble des imaginaires purs.
\begin{example}
\begin{itemize}
\item[$  \bullet$] $ z= 3 + 2\i$ est un nombre complexe de partie réelle $Re(z) = 3$ et de partie imaginaire $Im(z) = 2$.
\item[$  \bullet$]  $z= -\frac{\sqrt{3}}{2} - 6\i$ est un nombre complexe de partie réelle $Re(z) = -\frac{\sqrt{3}}{2}$ et de partie imaginaire $Im(z) =-6$.
\item[$  \bullet$]  $z= 4\i$ est un nombre complexe de partie réelle $Re(z) = 0$ et de partie imaginaire $Im(z) = 4$.
\item[$  \bullet$] $z = −5$ est un nombre complexe de partie réelle $Re(z) = −5$ et de partie imaginaire $Im(z) = 0$.
\end{itemize}
\end{example}
\medskip

\begin{remark}
\begin{itemize}
\item L'ensemble $ \mathbb{R} $ est inclus dans $ \mathbb{C} $ car tout nombre réel  est un nombre complexe de partie imaginaire nulle.
\[\text{On a les inclusions suivantes \quad }\mathbb{N}\subset \mathbb{Z}\subset\mathbb{D}\subset\mathbb{Q}\subset\mathbb{R}\subset\mathbb{C}\]
\item Un nombre complexe est réel si et seulement si sa partie imaginaire est nulle.
\item Un nombre complexe est imaginaire pur si et seulement si sa partie réelle est nulle.
\end{itemize}
\end{remark}
\medskip

\begin{exercice}
\begin{enumerate}
\item  Pour quelles valeurs du réel $ x $, le nombre complexe :\; $ z=x(-x+2\mathrm{i})+\mathrm{i}(x-3\i) $  est-il un imaginaire pur ?
\item  Pour quelles valeurs du réel $ x $, le nombre complexe $ z=(x-\mathrm{i})\Bigl[x+4-\mathrm{i}(x-6)\Bigr] $ est-il un réel ?
\end{enumerate}
\end{exercice}
\subsection*{Somme, produit et quotient de deux nombres complexes}
L'addition $ + $ et la multiplication $ \times $ dans $ \mathbb{C} $ ont les mêmes
propriétés que les opérations analogues dans $ \mathbb{R} $.

\medskip

Soient $z =a+\i b $ et $z' =a'+\i b' $ deux nombres complexes.


\medskip
\textbf{Somme}


\medskip
$ z+z'  =(a+\i b)+(a'+\i b')=(a+a')+\i (b+b')$

\medskip


\textbf{Produit}

\medskip
$ z\times z'  =(a+\i b)\times(a'+\i b')=(aa'-bb')+\i (ab'+a'b)$

\medskip


\textbf{Quotient}

\medskip

Si $ z\neq 0 $


\medskip
$ \dfrac{1}{z}  =\dfrac{1}{a+\i b}=\dfrac{a-\i b}{(a+\i b)(a-\i b)}=\dfrac{a-\i b}{a^2+ b^2}=\dfrac{a}{a^2+ b^2} -\i\dfrac{ b}{a^2+ b^2}$

\bigskip
$ \dfrac{z'}{z}  =\dfrac{a'+\i b'}{a+\i b}=\dfrac{(a'+\i b')(a-\i b)}{(a+\i b)(a-\i b)}=\dfrac{(a'+\i b')(a-\i b)}{a^2+ b^2}$

\medskip

\begin{remark}
Pour mettre le quotient sous forme algébrique, on \textit{rend réel} le dénominateur  $ a+\i b $ en multipliant le numérateur et le dénominateur par $  a-\i b$.
\end{remark}

\bigskip
\textbf{Les puissances de i}

\medskip

Soit $ n $ et $ m $ deux entiers naturels non nuls.


\medskip

On a: $ \i^0=1,\;\; \i^1=\i ,\;\; \i^2=-1 ,\;\; \i^3=-\i ,\;\; \i^4=1\; $ et plus généralement :
\[\i^{4n}=\paren{\i^4}^n=1 \quad \text{et}\quad \i^{m}= \i^{4n+r}=\paren{\i^4}^n \times\i^r=i^r\quad \text{où}\; r \; \text{est le reste de la division de \textit{m} par 4}.  \]
\begin{example}
$ \i^{2023}=  \i^{4\times 505+3}=\i^3=-\i$
\end{example}
\medskip

\begin{exercice}
\begin{enumerate}
\item Mettre sous forme algébrique  les nombres complexes suivants :

\medskip 

$ a=(3 + 2\i) + (5 - 4\i) $,\\ $ b=(3 + 2\i) - (5 - 4\i) $,\\ $ c=(2 - 3\i)(1 - \i) $,\\ $ d=(3 - 2\i)^2$,\\ $ e=(1 + \i)^3$. 

\item Mettre sous forme algébrique  les quotients de nombres complexes suivants :

\medskip 

$ x=\dfrac{1}{4+3\i} ,\;\;$ $ y=\dfrac{\i-4}{1-2\i},\;\; $, $ z=\dfrac{1+\i}{5\i} +\dfrac{2\i}{2+\i}$.
\end{enumerate}

\end{exercice}
\begin{remark}
 La relation d'ordre n'existe pas dans $ \mathbb{C} $, en d'autres termes, on ne peut pas comparer deux nombres complexes par les symboles $ <$ et $ >$. Par contre la comparaison peut se faire par l'égalité  $ =$ ou  la différence $ \neq$.
 \end{remark}
\subsection*{Égalité de deux nombres complexes}
\begin{itemize}
\item[$  \bullet$] Deux nombres complexes sont égaux si et seulement si ils ont même partie
réelle et même partie imaginaire.
\[z=z' \Longleftrightarrow Re(z)=Re(z')\quad \text{et}\quad Im(z)=Im(z')\]
\item[$  \bullet$] En particulier $ z=0 \Longleftrightarrow Re(z)=0\quad \text{et}\quad Im(z)=0$
\end{itemize}

\subsection*{Conjugué d'un nombre complexe}
\begin{definition}
Soit $ z $ un nombre complexe de forme algébrique  $ a+ \i b.$

\medskip

On appelle conjugué de $ z $ et on note $ \overline{z} $ le nombre complexe $ \overline{z} =a- \i b$.

\medskip

Ainsi :  $ Re(z)=Re(\overline{z}) $ et $ Im(\overline{z})=-Im(z) $

\end{definition}
\begin{example}
 $ \overline{-2+5\i} =-2-5\i,
\qquad  \overline{4\i} =-4\i,\qquad 
 \overline{9} =9$
\end{example}
\medskip

\begin{corollary}
\begin{itemize}
\item[\bullet]
Soit $ z $ un nombre complexe de forme algébrique  $ a+ \i b$ et $ \overline{z} $ son conjugué. Alors $ \overline{z} z=a^2+b^2 $.


\medskip

Ainsi $  \overline{z} z $  est un réel strictement positif ou nul si $ z=0 $.
\medskip

\textbf{Démonstration}

\medskip

$ \overline{z} z= (a- \i b)(a+ \i b) =a^2+b^2 $


\medskip
\item[$  \bullet$] La notion de conjugué permet de caractériser les  nombres réels et les nombres imaginaires purs parmi les nombres complexes.

\medskip
\hspace*{2cm} $ z\in\mathbb{R} \Longleftrightarrow \overline{z} =z $
\quad et \quad  $ z\in\i\mathbb{R} \Longleftrightarrow \overline{z} =-z $

\end{itemize}
\end{corollary}
\textbf{Démonstration}

\medskip

On note $ a+ \i b$ la forme algébrique de $ z $.


\medskip
 $ \overline{z} =z \Longleftrightarrow a- \i b =a+ \i b \Longleftrightarrow -2\i b=0 \Longleftrightarrow b=0 \Longleftrightarrow z=a \Longleftrightarrow z\in\mathbb{R} $


\medskip
 $ \overline{z} =-z \Longleftrightarrow a- \i b =-a- \i b \Longleftrightarrow 2a=0 \Longleftrightarrow a=0 \Longleftrightarrow z=\i b \Longleftrightarrow z\in\i\mathbb{R} $


\medskip

\begin{remark}
\begin{itemize}
\item[$  \bullet$] $ \overline{\overline{z}}=z $
\item[$  \bullet$] $ \overline{z} +z=2 Re(z)$
\item[$  \bullet$] $ z-\overline{z} =2 \i Im(z)$
\end{itemize}

\end{remark}
\subsubsection*{Propriétés du conjugué d'un nombre complexe}
Pour tous nombres complexes $z $ et $z' $ :

\medskip

\begin{itemize}
\item[$  \bullet$]  $ \overline{z+z'} =\overline{z}+\overline{z'}$
\item[$  \bullet$]    $ \overline{z z'} =\overline{z} \overline{z'}$

\medskip

\item[$  \bullet$]  De plus si $ z'\neq 0 $, \quad $\overline{\paren{ \dfrac{1}{z'} }}= \dfrac{1}{\overline{z'}}$
\item[$  \bullet$]   $\overline{\paren{ \dfrac{z}{z'} }}= \dfrac{\overline{z}}{\overline{z'}}$
 \item[$  \bullet$] Pour tout entier naturel $ n $, \quad $ \overline{z^n} =\overline{z}^n$
\end{itemize}
\medskip

\begin{exercice}
\begin{enumerate}
\item Déterminer le conjugué des nombres complexes $ z_1= (3-5\i)(1+\i) $, $ z_2= \dfrac{2+2\i}{3-\i } $ et   $z_3=\paren{4+9\i}^3 $.
\item Déterminer les nombres complexes $ z $ tels que\; $ Z=\dfrac{1-\i z}{1+\i z} $  soit réel.
\end{enumerate}
\end{exercice}
\subsection{ Interprétation géométrique }
\begin{lemma}
Dans un repère orthonormé   $ \paren{O,\; I, \; J} $ du plan, on considère les points : 

$ A(1, 3) $,\; $ B(0, -2) $,\; $ E(-4, 3) $.
\begin{enumerate}
\item Placer ces points dans le repère.
\item Calculer les coordonnées du vecteur $ \overrightarrow{AE} $
\item Calculer les coordonnées du milieu $ C $ du segment  $ [BE] $.
\item Calculer la distance $ OE $.
\item  Déterminer les coordonnées du point $ F $  tel que le quadrilatère  $ ABEF $ soit un parallélogramme.
\item Déterminer les coordonnées du point $ E' $  symétrique du point $ E $ par rapport à l'origine $ O. $
\item Déterminer les coordonnées du point $ A' $  symétrique du point $ A $ par rapport à l'axe $ (OI). $
\item Les droites  $ (BI)$ et $(EJ) $ sont-elle perpendiculaires ?
\end{enumerate}
\end{lemma}
\subsection*{Le plan complexe}
Dans le plan muni d'un repère orthonormé   $ \paren{O,\; I, \; J} $, on associe un unique point  du plan à chaque nombre complexe et réciproquement.

\medskip


  En posant  $  \overrightarrow{u}= \overrightarrow{OI}$  et $  \overrightarrow{v}= \overrightarrow{OJ}$  le repère se note aussi  $ \paren{O,\; \overrightarrow{u}, \; \overrightarrow{v}} $.

\medskip
 Ainsi:
\begin{itemize}
\item à $ z=x+\i y $  avec $x $ et $ y$ des réels, on associe le point $ M $  de coordonnées $ \paren{ x, y} $; on dit que \textbf{$  M$ est l'image de $ z $} et on note $ M(z) $.
\item à $ M(x,\; y )$, on associe le nombre complexe $ z_M=x+\i y$; on dit que \textbf{$z_M$ est l'affixe  de $ M $}, le vecteur $ \overrightarrow{OM} $ ayant les mêmes coordonnées que le point $ M $, on dit aussi  que  $ x+\i y$ est \textbf{l'affixe du vecteur  $ \overrightarrow{OM} $}.
\item l'axe des abscisses $ \paren{ O; \overrightarrow{u} }$  est appelé \textbf{ axe réel}, celui des ordonnées $ \paren{ O; \overrightarrow{v} }$ est appelé \textbf{axe imaginaire}.
\item Le plan où les points sont repérés par leurs affixes est appelé \textbf{plan complexe.}
\end{itemize}


\begin{tikzpicture}[>=stealth,scale=2]

% Axes
\draw[->] (-2,0) -- (2,0) node[right]{Re}; % Axe réel
\draw[->] (0,-2) -- (0,2) node[above]{Im}; % Axe imaginaire

% Labels des axes
\node at (2.2,0) {Axe réel};
\node at (0,2.2) {Axe imaginaire};

% Origine
\filldraw (0,0) circle (1pt) node[below left]{O};

% Point M et son affixe
\coordinate (M) at (1.5,1);
\filldraw (M) circle (1pt) node[above right]{$M(x, y)$};
\node at (1.5,-0.2) {$x$};
\node at (-0.2,1) {$y$};

% Vecteur OM
\draw[->, thick] (0,0) -- (M) node[midway, above right]{$\overrightarrow{OM}$};

% Affixe de M
\node at (1.5,1.4) {$z_M = x + iy$};

\end{tikzpicture}


\textbf{Exemples}:\\  Les points $ O$, $ I $  et $J $ ont pour affixes respectives $0 $, $  1$  et $ \i$.

\medskip

$ \overrightarrow{IJ} $ a pour coordonnées $ \begin{pmatrix}
-1\\ 1
\end{pmatrix} $  donc le vecteur $ \overrightarrow{IJ} $ a pour affixe $ -1+\i $ notée $ z_{\overrightarrow{IJ} } $.

\medskip

\begin{remark}
\begin{itemize}
\item[$  \bullet$] Les point d'affixes $z $ et $\overline{z} $ sont symétriques par rapport à l'axe réel.
\item[$  \bullet$] Les point d'affixes $z $ et $-z $ sont symétriques par rapport à l'origine.
\end{itemize}
\end{remark} 
\begin{property}
Pour tous points $ A$ et $ B$  du plan complexe,
\begin{itemize}
\item  l'affixe du vecteur $ \overrightarrow{AB} $ \;  est  \; $ z_{\overrightarrow{AB}}=z_A-z_B $.
\item  le milieu $ I $ du segment $ [AB] $ a pour affixe $ z_I=\dfrac{z_A+z_B }{2} $.
\item le barycentre  $ G $ de $ \paren{A,\; a} $ et $ \paren{B,\; B} $ a pour affixe $ z_G=\dfrac{a z_A+b z_B }{a+b} $.
\end{itemize}
\end{property}

\medskip
\textbf{Condition d'alignement de trois points}

\medskip

Soit $ A$, $ B $   et $C $  trois points distincts et \textbf{alignés} du  du plan complexe. 

On a par exemple $\overrightarrow{AC} $ et $ \overrightarrow{AB}$  colinéaires et il existe un réel $ k $ tel que $\overrightarrow{AC}=k\overrightarrow{AB}$.

\medskip

Ainsi  : $ z_C-z_A= k(z_B-z_A) \Longleftrightarrow \dfrac{ z_C-z_A}{z_B-z_A}=k\in\mathbb{R}$.

\textbf{À retenir}\\
$ A$, $ B $   et $C $  sont trois points distincts et alignés  si et seulement si  $ \dfrac{ z_C-z_A}{z_B-z_A }$ est un réel.


\begin{exercice}
Dans le plan complexe, on considère les  points $ A(2-3\i) $ , \; $ B(4\i) $ et $ C(1-\i) $.
\begin{enumerate}
\item Calculer l'affixe du milieu $ I $ de $ [AB] $ et celle du point $ D $ tel que   $ ABCD $ soit un parallélogramme.
\item Calculer l'affixe de $ G $ barycentre de  $ \paren{A,\; 2} $ ;  $ \paren{B,\; -1} $  et $ \paren{C,\; -2} $ .
\item Soit $ A'$ le symétrique de  $ A $ par rapport à l'axe réel. Montrer que $ A'$, $  D$ et  $  G $ sont alignés.
\end{enumerate}
\end{exercice}
\subsection{ Forme trigonométrique d'un  nombre complexe}
\begin{lemma}
Le plan orienté est muni d'un repère orthonormé   $ \paren{O,\; I, \; J} $; \; $\overrightarrow{u} $ et $ \overrightarrow{v}$ deux vecteurs non nuls.

\medskip

\textbf{Vrai ou faux}\; : Préciser si les affirmations suivantes sont vraies ou fausses.
\begin{enumerate}
        \item On dit que le repère  $ \paren{O,\; I, \; J} $ \; est direct  lorsque  $ \paren{ \overrightarrow{OI},\;\overrightarrow{OJ} }=\dfrac{\pi}{2} $\quad$ [2\pi] $.
        \item Si le point $ M $, distinct de $ O $ appartient à l'axe des abscisses alors  $ \paren{ \overrightarrow{OI},\;\overrightarrow{OM} }=0 $\quad$ [2\pi] $.
        \item L'ensemble des points tels que  $ \paren{ \overrightarrow{OI},\;\overrightarrow{OM} }=\dfrac{\pi}{2} $\quad$ [2\pi] $ est l'axe des ordonnées privé de l'origine.
        \item Si  $ \paren{ \overrightarrow{OI},\;\overrightarrow{u} }= \paren{ \overrightarrow{OI},\;\overrightarrow{v} }$\quad $ [2\pi] $ alors les vecteurs $ \overrightarrow{u}$ et $\overrightarrow{v} $ sont colinéaires.
        \item Si les vecteurs $ \overrightarrow{t}$ et $\overrightarrow{w} $ sont colinéaires alors    $ \paren{ \overrightarrow{OI},\;\overrightarrow{t} }= \paren{ \overrightarrow{OI},\;\overrightarrow{w} }\quad  [2\pi] $.
        \item Si $ M $ appartient au cercle de centre $ O $ et de rayon 1 alors ses coordonnées sont de la forme $ \paren{\cos \alpha ,\; \sin \alpha} $ où $ \alpha $ est une mesure en radian de l'angle $ \paren{ \overrightarrow{OI},\;\overrightarrow{OM} } $.
        \end{enumerate}
        \end{lemma}
        \subsection*{ Module et argument   d'un  nombre complexe}
        \begin{definition}
        Le plan complexe est muni d'un repère orthonormé  direct $ \paren{O,\; I, \; J} $. 
        \begin{description}
        \item Soit $ z $ un nombre complexe et  $ M $  son image dans le plan complexe.
        Le \textbf{module de $ z $} , noté $ \abs{z} $ est la distance $ OM $:\;  $ \abs{z}=OM $.
        \item Si $ z $  est non nul, on appelle \textbf{argument} de $ z $, noté \textbf{arg}$ (z) $, toute mesure en radians de l'angle orienté $ \paren{ \overrightarrow{OI},\;\overrightarrow{OM} } $: \\  $ \text{arg}(z)= \paren{ \overrightarrow{OI},\;\overrightarrow{OM} }$\quad $ [2\pi] $.
        \end{description}
        \end{definition}
        
        
     


\begin{tikzpicture}[>=stealth, scale=2]

% Axes
\draw[->] (-2,0) -- (2,0) node[right]{Re}; % Axe réel
\draw[->] (0,-2) -- (0,2) node[above]{Im}; % Axe imaginaire

% Labels des axes
\node at (2.2,0) {Axe réel};
\node at (0,2.2) {Axe imaginaire};

% Origine
\filldraw (0,0) circle (1pt) node[below left]{O};

% Point I (unité sur l'axe réel)
\coordinate (I) at (1,0);
\filldraw (I) circle (1pt) node[below right]{I};

% Point M et son affixe
\coordinate (M) at (60:1.5); % Utilisation de coordonnées polaires pour faciliter l'angle
\filldraw (M) circle (1pt) node[above right]{$M(x, y)$};

% Vecteur OM
\draw[->, thick] (0,0) -- (M) node[midway, above right]{$\overrightarrow{OM}$};

% Affixe de M
\node at (1.5,1.4) {$z_M = x + iy$};

% Module de z
\draw[|-|] (0,-0.5) -- node[fill=white,midway,sloped] {$|z|$} (M |- {(0,-0.5)});

% Argument de z
\draw (0.5,0) arc (0:60:0.5) node[midway, right] {$\theta$};

\end{tikzpicture}





       
       \begin{example}
       $  \abs{\i}=1 $,  $ \;\text{arg}(\i) =\frac{\pi}{2}$, $ \abs{-5}=5 $,  $ \;\text{arg}(-5) =\pi$ $\;\; [2\pi] $, $\;  \abs{3}=3$, $ \;\text{arg}(3) =0\;  [2\pi] $ 
, $\;  \abs{-2\i}=2 $,  $\; \text{arg}(-2\i) =\frac{3\pi}{2}$\: $[2\pi] $.
\end{example}

\begin{corollary}
     \begin{itemize}
\item $ z $  est réel si et seulement si  $ z=0 $ ou $ \text{arg}(z) =0$\: $[\pi] $.
\item $ z $  est imaginaire pur si et seulement si $ z=0 $  $ \text{arg}(z) =\frac{\pi}{2}$\: $[\pi] $.
\end{itemize}
  \end{corollary}
\begin{remark}
\begin{itemize}
\item Si $ z=x+\i y $  avec $x $ et $y $ réels alors $ \abs{z}=\sqrt{x^2+y^2} $
\item Si les points $A $ et $ B$ ont pour affixe respectives $ x_A$ et $y_B $ alors $ AB=\abs{x_B - x_A} $.
\end{itemize}
\end{remark}
\subsubsection*{Propriétés du module d'un nombre complexe}
\begin{property}
Pour tous nombres complexes $z $ et $z' $ :

\medskip

\begin{itemize}
\item[$  \bullet$]  $ \abs{zz'} =\abs{z}\abs{z'}$
 \item[$  \bullet$] $ \abs{z} =\abs{-z} =\abs{\overline{z}}$ 
 \item[$  \bullet$]   $z\overline{z}=\abs{z}^2$


\medskip

\item[$  \bullet$]  De plus si $ z'\neq 0,  \quad   \abs{ \dfrac{z}{z'} }= \dfrac{\abs{z}}{\abs{z'}}$

\medskip
 \item[$  \bullet$] Pour tout entier naturel $ n, \quad  \abs{z^n} =\abs{z}^n$
\end{itemize}

\end{property}


\subsubsection*{Propriétés de l'argument d'un nombre complexe}
\begin{property}
Pour tous nombres complexes $z $ et $z' $ non nuls :

\medskip

\begin{itemize}
\item[$  \bullet$]  $ \text{arg}(zz') =\text{arg}(z) +\text{arg}(z') \; [2\pi]$
 \item[$  \bullet$]   De plus si $ z'\neq 0,  \quad   \text{arg}\paren{\dfrac{1}{z'}}=-\text{arg}(z')\; [2\pi]$ \item[$  \bullet$] $  \text{arg}\paren{\dfrac{z}{z'}}=\text{arg}(z)- \text{arg}(z') \; [2\pi]$ 


\medskip

\item[$  \bullet$]   $ \text{arg}(\overline{z})=-\text{arg}(z) \; [2\pi]$
\item[$  \bullet$]  $ \text{arg}(-z)=\pi +\text{arg}(z) \; [2\pi]$

\medskip
 \item[$  \bullet$] Pour tout entier naturel $ n, \quad  \text{arg}(z^n) =n \; \text{arg}(z) \; [2\pi]$
\end{itemize}


\end{property}
\subsubsection*{Interprétation géométrique }
Soit deux points distincts  $A $ et $ B$ d'affixes respectives $ x_A$ et $y_B $ alors $ \text{arg}\paren{x_B - x_A} =\paren{ \overrightarrow{OI},\;\overrightarrow{AB} }$\; $ [2\pi]$.

Soit trois points distincts  $A $, $ B$ et $ C$ d'affixes respectives $ x_A$ , $y_B $  et $y_C $ alors:

\medskip
 $ \text{arg}\paren{\dfrac{x_C - x_A}{x_B - x_A}} =\paren{ \overrightarrow{AB},\;\overrightarrow{AC} } \; [2\pi]$.
 
 \bigskip
 
 \textbf{Démonstration}
 \begin{align*}
  \text{arg}\paren{\dfrac{x_C - x_A}{x_B - x_A}}= \text{arg}\paren{x_C - x_A}- \text{arg}\paren{x_B - x_A}\; [2\pi]& \\=\paren{ \overrightarrow{OI},\;\overrightarrow{AC} }-\paren{ \overrightarrow{OI},\;\overrightarrow{AB} }\; [2\pi]&\\=\paren{ \overrightarrow{AB},\;\overrightarrow{AC} }  \; [2\pi]
  \end{align*}
  \begin{corollary}
  Soit les points distincts $M $, $ A $  et $ B$ d'affixes respectives $ z$ , $z_A $  et $z_B $ alors:
  $\;\; \text{arg}\paren{\dfrac{z - z_B}{z - z_A}} =\paren{ \overrightarrow{MA},\;\overrightarrow{MB} } \; [2\pi]$.
  
 \end{corollary}
  \subsubsection*{Condition d'orthogonalité }
  
  Les droites $(AB) $ et $(CD) $ sont perpendiculaires  $ \Longleftrightarrow  \text{arg}\paren{\dfrac{z_D - z_C}{z_B - z_A}} =\dfrac{\pi}{2} \; [\pi] \Longleftrightarrow \dfrac{z_D - z_C}{z_B - z_A}\in\i\mathbb{R}$. 
  
  \medskip
  
  \textbf{Conséquence 1: cas du triangle rectangle }
  
  \medskip
  
  $ ABC $ est un triangle rectangle en $ A $ si et seulement si   $ \dfrac{z_C - z_A}{z_B - z_A} \in\i\mathbb{R} $.
  \medskip
  
  \textbf{Conséquence 2: cas du triangle rectangle  isocèle}
  
  \medskip
  
   $ ABC $ est un triangle rectangle en $ A $ si et seulement si   $ \dfrac{z_C - z_A}{z_B - z_A} =\i $  ou $ -\i $.
   
   
 \bigskip
 
 \textbf{Démonstration}
 
  \medskip
  
 D'une part  $\abs{ \dfrac{z_C - z_A}{z_B - z_A} }=\abs{\pm \i} =1 $  càd : $ AB=AC $
  \;  d'autre part $ \text{arg}\paren{\dfrac{z_C - z_A}{z_B - z_A}} = \text{arg}(\i)=\dfrac{\pi}{2}=\paren{ \overrightarrow{AB},\;\overrightarrow{AC} } $
  
 \subsection*{Forme trigonométrique}
  \begin{definition}
  Soit $ z $ un nombre complexe non nul; on pose :
  \medskip
  
  $ x=Re(z) $, $\quad y=Im(z) $, $ \quad r=|z| $, $\quad \theta=\text{arg}(z)\;[2\pi] $
  
  \medskip
  
  On a alors :\;  $ x=r \cos \theta $\quad et \quad $ y= r\sin \theta $.
 
  
  On obtient l'écriture   $ z= r\paren{\cos \theta +\i \sin \theta}$ qui est appelée \textbf{forme trigonométrique} du nombre complexe $ z $.
   \end{definition}
  \medskip
  
  \textbf{Passage d'une forme à une autre }
  
  
  \medskip
  
  Si le nombre complexe $ z $ s'écrit   $ x+\i y $ sous  forme algébrique  et $ r\paren{\cos \theta +\i \sin \theta}$   sous forme trigonométrique alors :
  
  
   $ r=\sqrt{x^2+y^2} 
  \hspace*{1cm}  \text{et}  \hspace*{1cm}   \left\{\begin{array}{l c l}
\cos \theta & = \dfrac{x}{r} \\ 	 
\sin \theta & = \dfrac{y}{r}
\end{array}\right. $
 
  
  
  \begin{example}
  
   Déterminons la forme trigonométrique  de $ z=1+\i $.
  
  \medskip
  $ r=\sqrt{1^2+1^2}=\sqrt{2} $  \hspace*{0.5cm}  et  \hspace*{0.5cm} $  \left\{\begin{array}{l c l}
\cos \theta & = \dfrac{\sqrt{2}}{2} \\ 	 
\sin \theta & = \dfrac{\sqrt{2}}{2}
\end{array}\right. $  on trouve $ \theta=\dfrac{\pi}{4} $ \; et\; $ z=\sqrt{2}\paren{\cos \dfrac{\pi}{4} +\i \sin \dfrac{\pi}{4}}$
\end{example}

\medskip


  \begin{example} Déterminons la forme trigonométrique  de $ z=\sqrt{6}-\i\sqrt{2} $.
  
  \medskip
  $ r=\sqrt{\paren{\sqrt{6}}^2+\paren{-\sqrt{2}}^2}=2\sqrt{2}   \hspace*{0.5cm} \text{ et}  \hspace*{0.5cm}   \left\{\begin{array}{l c l}
\cos \theta & = \dfrac{\sqrt{3}}{2} \\	 
\sin \theta & =- \dfrac{1}{2}
\end{array}\right. $  on trouve $ \theta=-\dfrac{\pi}{6} $  et $ z=2\sqrt{2}\paren{\cos \paren{-\dfrac{\pi}{6}} +\i \sin \paren{-\dfrac{\pi}{6}}}$
   \end{example}
   \begin{example}
  Soit  $ z=2\paren{\cos\dfrac{\pi}{3} +\i \sin \dfrac{\pi}{3}}$.
  
   Déterminons la forme algébrique  de  $ \dfrac{1}{z} $.
  
  \medskip
  
  \textbf{Première méthode}
  
   \medskip
  $ \dfrac{1}{z} =\dfrac{1}{2}=\paren{\cos \paren{-\dfrac{\pi}{3}} +\i \sin \paren{-\dfrac{\pi}{3}}}=\dfrac{1}{2}\paren{\dfrac{1}{2}-\i\dfrac{\sqrt{3}}{2}}=\dfrac{1}{4}-\i\dfrac{\sqrt{3}}{4}$
  
  \medskip
  
  \textbf{Deuxième méthode}
  
   \medskip
  $ 2\paren{\cos\frac{\pi}{3} +\i \sin \frac{\pi}{3}}=2\paren{\frac{1}{2}+\i\frac{\sqrt{3}}{2}}=1+\i\sqrt{3}$
  
  
   \medskip
   $ \Rightarrow\dfrac{1}{1+\i\sqrt{3}}\times\frac{1-\i\sqrt{3}}{1-\i\sqrt{3}} =\dfrac{1-\i\sqrt{3}}{4}=\dfrac{1}{4}-\i\dfrac{\sqrt{3}}{4}$
    \end{example}
\begin{exercice}
\begin{enumerate}
\item Mettre sous forme trigonométrique les nombres complexes suivants:


\medskip

\textbf{a)}\;  $ -2+2\i \hspace*{1cm}$ \textbf{b)}  $\; \dfrac{3}{1+\i\sqrt{3}}  \hspace*{1cm}$ \textbf{c)}  $\; \dfrac{4-4\i}{-1+\i\sqrt{3}} $.
\item  On considère les deux nombres complexes \; $ z_1=6\paren{\cos\dfrac{3\pi}{4} +\i \sin \dfrac{3\pi}{4}} $ et $ z_2=2\paren{\cos\dfrac{2\pi}{3} +\i \sin \dfrac{2\pi}{3}} $.

\bigskip

Déterminer   \textbf{a)}  $\; z_1\times z_2 \hspace*{1cm}$ \textbf{b)}  $\; \dfrac{z_1}{z_2} \hspace*{1cm}$  \textbf{c)}  $\; \dfrac{\overline{z_2}}{-z_1} $.
\end{enumerate}
\end{exercice}
 \subsection*{Notation exponentielle de la  forme trigonométrique} 
 Le mathématicien Léonhard Euler (1707-1783) utilisa la notation $ \eexp{\i \theta } $ pour désigner le nombre complexe $ \cos \theta +\i \sin \theta $ de module 1 et d'argument $ \theta $.
 \[\text{Ainsi}:\; \cos \theta +\i \sin \theta= \eexp{\i \theta }  \]
 
 On a alors \; $ \left|\eexp{\i \theta }\right| =1$ pour tout réel $ \theta $.
 
 \medskip
 \begin{definition}
 Un nombre complexe $ z $ de module $ r $ et d'argument $ \theta $
s'écrit : $z=r \eexp{\i \theta }$ . Cette écriture est appelée  \textbf{notation exponentielle} de $ z $.
\end{definition}
 \medskip

\begin{example}
 $  \eexp{\i \frac{\pi}{2} }=\i ,\quad     \eexp{\i \pi }=-1 ,\quad    \eexp{\i 0 }=1 ,\quad    \eexp{-\i \frac{\pi}{2} }=-\i ,  \quad    2\eexp{\i \frac{\pi}{4} }=\sqrt{2}+ \i\sqrt{2}, \quad    \eexp{2k\i\pi}=1\quad   \forall k\in\mathbb{Z}$.
 \end{example}
 
 \begin{property}
 Soit $ z_1= r_1\eexp{\i\theta_1} $ \; et \; $ z_2= r_2\eexp{\i\theta_2} $
 
 \begin{itemize}
 \item[$  \bullet$]  $ \overline{z_1}= r_1\eexp{-\i\theta_1}$
  \item[$  \bullet$]  $ z_1 z_2 =r_1r_2\eexp{\i(\theta_1+\theta_2)}$
   \item[$  \bullet$] $ \dfrac{z_1}{z_2} =\dfrac{r_1}{r_2}\eexp{\i(\theta_1-\theta_2)}$ 
 
 \medskip
 \item $ z_1= z_2 \Longleftrightarrow r_1= r_2 \qquad\text{et}\qquad \theta_1= \theta_2 +2k\pi\qquad   k\in\mathbb{Z}$
 \end{itemize}
  \end{property}
 \begin{example}
 Soit à calculer  $\; \paren{1+\i}^{14} $.\\ Nous avons sous forme exponentielle   
 $ 1+\i =\sqrt{2}\eexp{\i\frac{\pi}{4}}$.
 
 
 \medskip
 
Donc  $\;\; \paren{1+\i}^{14}=\paren{ \sqrt{2}}^{14}\eexp{\i\frac{14\pi}{4}}=2^7\eexp{\frac{7\pi}{2}\i}=128\paren{\cos \dfrac{7\pi}{2} +\i \sin \dfrac{7\pi}{2}}=128\i$
 \end{example}
 \begin{exercice}
 On considère les deux nombres complexes \; $ z=\sqrt{3} +\i $\; et\; $ z'=-1+\i $
 
 \begin{itemize}
 \item[\textbf{a)}] Donner l'écriture exponentielle des nombres complexes $ z,\;\; $ $ z',\;\; $ $ zz',\;\; $ $ \dfrac{z}{z'},\;\; $ $ z^5 $.
 \item[\textbf{b)}] Déterminer l'écriture algébrique  de $\;\; \dfrac{z'^{14}}{\overline{z}^8} $
 \end{itemize}
  \end{exercice}
 \bigskip
 
 \textbf{Le triangle équilatéral}

 $ ABC $ est un triangle équilatéral si et seulement si \; $ \dfrac{z_C-z_A}{z_B-z_A} =\eexp{\pm\i\frac{\pi}{3}}$.
 
 
 \medskip
 
 En effet $ \abs{\dfrac{z_C-z_A}{z_B-z_A} }=\abs{\eexp{\pm\i\dfrac{\pi}{3}}}=1 $
 \qquad  $\text{arg}\paren{\dfrac{z_C-z_A}{z_B-z_A}} =\text{arg}\paren{\eexp{\pm\i\dfrac{\pi}{3}}}=\pm \dfrac{\pi}{3}$.
 
 
 \medskip
 
 Ainsi $ AB=AC $  et $ \widehat{BAC}=\dfrac{\pi}{3} $
\subsection*{ Formules de Moivre et d'Euler}
\begin{lemma}
Soit $ z=\cos \theta +\i \sin \theta $
\begin{enumerate}
\item Utiliser la formule de multiplication deux nombres complexes écrits en notation exponentielle pour calculer $ z^2 $, $ z^3 $, $ z^4 $ et  $ z^5 $.
\item Donner une formule générale de $ z^n $ pour $ n\in\mathbb{N} $.
\item Que devient la  formule   pour $\; n\in\mathbb{Z} $? 
\end{enumerate}
\end{lemma}

\begin{lemma}

Soit $ z=\cos \theta +\i \sin \theta $
\begin{enumerate}
\item Utiliser la formule trouvées à l'
\textbf{Activité 37}  pour calculer  les sommes suivantes :

\medskip

$ z+ \dfrac{1}{z}, \;\;  z^2 + \dfrac{1}{z^2}$, $ \;z^3+ \dfrac{1}{z^3} \; $  et   $ \; z^4 + \dfrac{1}{z^4}$ en simplifiant le résultat.
\item Quelle est la  formule générale de $ z^n + \dfrac{1}{z^n} $ pour $ n\in\mathbb{Z} $.

\end{enumerate}
\end{lemma}
\begin{property}{(Formule de Moivre)} 
Pour tout réel $ \theta $   pour tout entier relatif $ n $:
\[\paren{\cos \theta +\i \sin \theta}^n=\cos n\theta +\i \sin n\theta\]
\end{property}
\begin{corollary}
 $ \bullet\; \;\paren{\eexp{\i\theta}}^n =\eexp{n\i\theta}\quad$  et   $\quad \bullet\; \; \paren{r\eexp{\i\theta}}^n =r^n\eexp{n\i\theta}\quad $ pour tout réel $ \theta $  et pour tout entier relatif $ n $:
  \end{corollary}
\begin{example}
Soit à exprimer $ \cos 3\theta $   et $ \sin 3\theta $  en fonction de $ \cos \theta $  et $ \cos \theta $.

\medskip
En développant de deux manières  $ \paren{\cos \theta +\i\sin  \theta}^3 $

\medskip
On obtient  \;  $ \cos 3\theta +\i\sin 3 \theta=\paren{\cos \theta +\i\sin  \theta}^3$


\medskip
 $ = \cos^3\theta+ 3\i\cos^2\theta\sin\theta-3\cos\theta\sin^2\theta -\i \sin^3\theta $

\medskip
En identifiant les parties réelles et les parties imaginaires:


\medskip
$ \cos3\theta=\cos^3\theta-3\cos\theta\sin^2\theta  $\quad et \quad $ \sin 3 \theta=3\cos^2\theta\sin\theta-\sin^3\theta $.
\end{example}
\begin{property}{(Formules d'Euler)}
Pour tout réel $ \theta $:
\[\cos \theta = \dfrac{\eexp{\i\theta}+\eexp{-\i\theta}}{2}  \qquad \text{et}\qquad\sin\theta = \dfrac{\eexp{\i\theta}-\eexp{-\i\theta}}{2\i}\]
\end{property}
\textbf{Démonstration}

\medskip

$\eexp{\i\theta}=\cos \theta +\i\sin \theta  $\quad  et\quad
 $\eexp{-\i\theta}=\cos(-\theta) +\i\sin(- \theta)=\cos \theta -\i\sin \theta $ 


\medskip

On en déduit que: \quad $ 2\cos \theta=\eexp{\i\theta}+\eexp{-\i\theta} $ \quad  et \quad $ 2\i\sin \theta=\eexp{\i\theta}-\eexp{-\i\theta} $
\begin{example}
Soit à \textbf{linéariser} $ \cos^3\theta $.
\medskip

$ \cos^3\theta =\paren{ \dfrac{\eexp{\i\theta}+\eexp{-\i\theta}}{2}}^3=\frac{1}{8}\paren{\eexp{\i\theta}+\eexp{-\i\theta} }^3=\frac{1}{8}\paren{\eexp{3\i\theta}+3\eexp{\i\theta}+3\eexp{-\i\theta} +\eexp{-3\i\theta}}$

\medskip

 $ =\frac{1}{8}\paren{\eexp{3\i\theta}+\eexp{-3\i\theta}+3\eexp{\i\theta}+3\eexp{-\i\theta}} $


\medskip

$ =\frac{1}{8}\paren{2\cos 3\theta+3\times 2\cos \theta} $

 $ =\frac{1}{4}\paren{\cos 3\theta+3\cos \theta} $
\end{example}

\begin{corollary}
 $ 2\cos n\theta=\eexp{n\i\theta}+\eexp{-n\i\theta} \quad $  et $\quad  2\i\sin n \theta=\eexp{n \i\theta}-\eexp{-n \i\theta} \; $  pour tout $ \; n\in\mathbb{Z} $.
\end{corollary}

\begin{exercice}
\textit{Les questions sont indépendantes.}
\begin{enumerate}
\item  Montrer que \; $ \paren{ 3\cos\dfrac{2\pi}{3} -3\i\sin\dfrac{\pi}{3} }^9=3^9. $
\item  Soit $ z=\dfrac{1}{2} +\i\dfrac{\sqrt{3}}{2}$

Déterminer les valeurs de l'entier relatif $ n $ pour lesquelles $ z^n $ est réel.
\item 
En utilisant les formules d'Euler, montrer que : $ \sin x\cos3 x=\dfrac{1}{2} \sin 4x -\dfrac{1}{2} \sin 2x$.
\end{enumerate}
\end{exercice}
\subsection{ Racines n-ièmes d'un nombre complexe}
\begin{lemma}
On considère les équations \; $ z^2-1=0,\quad  z^3-1=0 ,\quad  z^4-1=0 $.
\begin{enumerate}
\item Factoriser les expressions puis déterminer toutes les solutions dans $ \mathbb{C} $.
\item Mettre  toutes les solutions sous forme trigonométrique  puis les représenter dans le plan complexe.
\item Quelles configurations les solutions de la deuxième et troisième forment-elles ?
\item Peut-on prévoir la configuration formée par les solutions de l'équation   \; $ z^5-1=0 $ ?
\item Quelle est la  configuration formée dans le plan complexe par les solutions de l'équation   \; $ z^n-1=0 $, \quad $ n\geq 3 $ ?
\item  Utiliser la représentation géométrique des solutions  pour prédire la forme trigonométrique des solutions de l'équation    $ \;z^n-1=0 , \quad  n\geq 3 $.
\end{enumerate}
\end{lemma}

\subsection*{Racines n-ièmes de l'unité}
\begin{theorem}
Pour tout entier naturel  non nul $ n ,$ l'équation  $ z^n=1 \;$ admet  $\;  n $ racines   distinctes  définies par
\\$\quad  z_k=\eexp{\i\frac{2k\pi}{n} }\quad k=0, \; 1, \; 2 ,\; \cdots, \;n-1$

\medskip

Les solutions (ou racines) de l'équation $ z^n=1 $ sont appelées  \textbf{racines n-ièmes de l'unité.}

\end{theorem}

\begin{remark}
Le plan complexe étant muni d'un repère orthonormé direct $\; \ouv,\; $. Lorsque $ n\geq 3 $, les
points-images des racines n-ièmes de l'unité sont les sommets d'un polygone régulier inscrit dans
le cercle trigonométrique.
\end{remark}
\begin{exercice}
Déterminer les racines sixièmes de l'unité.
\end{exercice} 
\begin{theorem}{(Admis)}
Étant donné un nombre complexe non nul $ a $, il existe $ n $ nombres complexes distincts  $ z $ tels que \; $ z^n=a $. Ces nombres sont appelés les \textbf{racines n-ièmes} de $ a $. Ils sont donnés par :
\[ z_k=\abs{a}^{\frac{1}{n}}\eexp{\i\paren{\dfrac{\text{arg}(a)}{n}+\dfrac{2k\pi}{n}} }\qquad k=0, \; 1, \; 2 ,\; \cdots, \;n-1\]
\end{theorem}
\begin{property}
Dans le plan complexe muni du repère orthonormé $ \ouv $ , les images des racines
n-ièmes d'un nombre complexe non nul $ a $ forment un polygone régulier à $ n $ côtés inscrit dans un cercle de centre O et de rayon $ \abs{a}^{\frac{1}{n}} $.
\end{property}

\subsection{Résolution d'équations du second degré}

\textbf{Résolution d'équations du second degré à coefficients réels}
\begin{theorem}
Soit l'équation  du second degré  (E) d'inconnue $ z $ : $ \; a z^2+b z+c=0 \;$ telle que $\; a\in\mathbb{R}^{\ast} $, $\; b\in\mathbb{R} \;$ et $\; \in\mathbb{R} $.

\medskip


On pose $ \Delta  =b^2- 4ac \;$ donc $\; \Delta $ est un réel.
\begin{itemize}
\item  Si  $ \;\Delta > 0 \;$ alors (E) a deux racines réelles distinctes: $ \;z_1=\dfrac{-b+\sqrt{\Delta}}{2a} $ et $\; z_2=\dfrac{-b-\sqrt{\Delta}}{2a} $
\item  Si  $ \;\Delta = 0 \;$ alors (E) a une  racine réelle double: $\; z_0=\dfrac{-b}{2a} $.
\item  Si  $\; \Delta < 0 \;$ alors (E) a deux racines complexes conjuguées : $\; z_1=\dfrac{-b+\i\sqrt{-\Delta}}{2a} \;$ et $ \; z_2=\dfrac{-b-\i\sqrt{-\Delta}}{2a} $
\end{itemize} 
\end{theorem}
\begin{example}
  Résoudre dans  $\; \mathbb{C} \;$  l'équation  $ \; 3z^2-z+5=0 $.

\medskip

  \hspace*{5cm} $\; \Delta = 1-60=-59 $

\medskip

   $ z_1= \frac{1+\i\sqrt{59}}{6}\; $ et $ \;z_2= \frac{1-\i\sqrt{59}}{6} $
   \end{example}
   \textbf{ Racines carrées d'un nombre complexe}
   \begin{definition}
   Soit un nombre complexe $ \;\Delta= a+\i b $.
   
   \medskip
   
   Un nombre complexe $\; z \;$ est une racine carrée de $ \Delta \;$ si : $\; z^2=\Delta $.
   
   \medskip
   
   Déterminer les racines carrées de $\; \Delta \;$ revient à résoudre dans $\; \mathbb{C} \;$ l'équation  : $\; z^2=\Delta $.
    \end{definition}
   \begin{property}
    Un nombre complexe a deux racines carrées opposées.
    \end{property}
    
   \medskip
   \textbf{Résolution algébrique de l'équation } $\; z^2=\Delta $
   
   \medskip
   
   Posons $ z=x+\i y ,\;$   $ x\;$ et $\; y$ des réels.
   
   \medskip
   On a : $ \; z^2= \Delta \Longleftrightarrow (x+\i y)^2=a+\i b $
   
   \medskip
   \hspace*{2cm} $ \Longleftrightarrow \; x^2 -y^2 +2\i x y =a+\i b$
   
   \medskip
   
  \hspace*{2cm}  $ \Longleftrightarrow  $ 
 $\left\{\begin{array}{l c}
 x^2 -y^2 & =a\\ 	 
 2xy& = b
\end{array}\right.$ 

\medskip

De plus $\; \abs{z^2} =\abs{\Delta}\Longleftrightarrow   x^2 +y^2=\sqrt{a^2+b^2}$.

\medskip

\textbf{Méthode générale  pour chercher les racines carrées d'un nombre 
complexe  $\; a+\i b \;$ sous forme algébrique :} 


\medskip
Soit $ \;z = x +\i y\; $ une telle racine carrée. Alors $\; x \;$ et $\; y\; $ sont solutions du système suivant :

   \medskip
   
  \hspace*{2cm}  $ \Longleftrightarrow  $ 
 $\left\{\begin{array}{l l}
 x^2 +y^2&=\sqrt{a^2+b^2}\qquad (1)\\
 x^2 -y^2 & =a\qquad \qquad (2)\\ 	 
 2xy& = b  \qquad \qquad (3)
\end{array}\right.$ 

\medskip

\begin{remark}
\begin{itemize}
\item On commence par résoudre le système formé par les deux équations (1) et (2) qui a a priori quatre
couples $(x , y )$ solutions. Et compte tenu de l'équation (3), on ne retient que les deux couples $(x , y )$
tels que le signe de $xy$ soit celui de $ b $ .
\item On se gardera d'appliquer cette méthode dans le cas où $ \Delta $ est un nombre réel. Les racines carrées de
$ \Delta $ sont alors évidentes, égales à   $ \pm \Delta $ si $ \Delta $ est positif et à  $ \pm\i\sqrt{\Delta} $ si  $ \Delta $ est négatif.
\end{itemize}
\end{remark}
\begin{example}
Déterminons les racines carrées du nombre complexe  $\; 3-4\i $.

Soit $\; x+\i y \;$ une telle racine racine carrée.

 \medskip
   
  \hspace*{2cm}  $ \Longleftrightarrow  $ 
 $\left\{\begin{array}{l l}
 x^2 +y^2&=5\qquad (1)\\
 x^2 -y^2 & =3 \qquad \qquad (2)\\ 	 
 2xy& = -4  \qquad \qquad (3)
\end{array}\right.$ 

\medskip
(1) + (2) permet d'obtenir $ x^2=4\Longleftrightarrow x=2 \; \text{ou}\; x=-2 $

\medskip
(1) - (2) permet d'obtenir $ y^2=1\Longleftrightarrow x=1 \; \text{ou}\; x=-1 $

\medskip

D'après (3)  les racines carrées de   $ 3-4\i $ sont :\;\; $2-\i $ et $-2+\i $.
\end{example}

\textbf{Résolution d'équations du second degré à coefficients complexes}
\begin{property}
Soit l'équation  du second degré  (E) d'inconnue $ z $ :\; $ a z^2+b z+c=0 $\; telle que $ a\in\mathbb{C}^{\ast} $,\; $ b\in\mathbb{C} $ et $ \in\mathbb{C} $.

\medskip


On pose $ \;\Delta  =b^2- 4ac \;$ et soient $\; x+\i y \;$ et $\; -x-\i y \;$ ses deux racines carrées opposées.

\medskip

 Alors les solutions de (E) sont:
  
  $$  z_1=\dfrac{-b+ x+\i y}{2a} \;\; \text{et} \;\; z_2=\dfrac{-b-x-\i y}{2a} $$ 
\end{property}

 
\begin{example}
  Résoudre dans $  \;\mathbb{C}  \;$  l'équation  $ \; z^2+(2+3\i)z-2+4\i=0 $.
  
  \medskip
  
On trouve    $ \; \Delta =3-4\i $
  
  \medskip


Alors 
  
  $$  z_1=\dfrac{-(2+3\i)+ 2-\i}{2}=2\i \;\; \text{ et} \;\; z_2=\dfrac{-(2+3\i)-2+\i}{2} =-2-\i$$ 

\end{example}

\begin{exercice}
  Résoudre dans  $ \;\mathbb{C}\; $  les équations:
 
\begin{enumerate}
\item  $ z^2+(\i-1)z-2(1+\i)=0 $
\item  $ (2+\i)z^2-(5-\i)z+2-2\i=0 $
\item  $ (1-\i)z^2+(3\i+2)z+1+4\i=0 $
\item  $(z^2 +3z-2)^2 =-(2z^2 -3z+2)^2$
\end{enumerate}
\end{exercice}


\subsubsection*{Un exemple  d'équation de degré supérieur ou égal à 3}

\medskip
 On considère dans  $ \;\mathbb{C} \;$  l'équation (E) d'inconnue $ z $ suivante :
 
 \medskip
 
 $ (E) :\qquad  z^3 - ( 1+2\i )z^2 +3( 1+ 3\i ) z-10(1+ \i)$
 
 \medskip
 
 \textbf{a)}\; Montrer que l'équation (E) admet une solution imaginaire et la déterminer.
 
 \textbf{b)}\; Résoudre l'équation (E).
 
 \textbf{Solution}
 
 \medskip
 
  \textbf{a)}\; Soit   $\; z_0 = \i y  \;$ une éventuelle solution imaginaire  pure de (E). On a alors :
  
  \medskip
  
  $ ( \i y )^3 + ( 1-4\i ) ( \i y )^2 -( 7+ 3\i ) ( \i y ) +6\i-2=0 $
  
  \medskip
  
  $z_0\; $ sera solution de (E) si et seulement si $\; -y^2+3 y-2+\i(- y^3 + 4 y^2-7 y+ 6)=0,\; $ soit
  
    $ \Longleftrightarrow  $ 
 $\left\{\begin{array}{l c}
 -y^2+3 y-2& =0\qquad (\ast) \\ 	 
 - y^3 + 4 y^2-7 y+ 6& = 0\qquad (\ast\ast)
\end{array}\right.$ 


L'équation $ \;(\ast)\; $ admet deux solutions qui sont 1 et 2. On vérifie que seul le réel 2 est solution
de l’équation $ \;(\ast\ast)\; $. Il en résulte que le réel 2 est l'unique solution du système précédent.

\medskip
On en déduit que  $\; z_0 = 2\i \;$ est l'unique solution imaginaire pure de (E).


\medskip
D'après le théorème précédent, il s'ensuit que :


\medskip

$ z^3 + ( 1-4\i )z^2 -( 7+ 3\i ) z+ 6\i-2=(z-2 \i)(z^2+ b z+ c) \;$  où $\; b\; $ et $\; c \;$ sont des nombres complexes.


\medskip

Un développement et une identification  terme à terme ( ou  la méthode Horner) nous donnent   $\; b = 1-2\i \;$ et $\; c = -3-\i$.


\medskip
L'équation (E) s'écrit alors $\; ( z-2\i ) ( z^2 + ( 1-2\i )-3-\i ) =0, \;$  ce qui équivaut à :

\medskip
$z = 2\i\; $ ou  $\; z^2 + ( 1-2\i )-3-\i=0$.


\medskip
On vérifie que les solutions de l'équation (E$_1$) : $\; z^2 + ( 1-2\i )z-3-\i=0\; $ sont :

\medskip
$z_1 = -2+\i\; $  et $\; z_2 = 1+\i$.

\medskip

L'ensemble des solutions de l'équation (E) est donc : $S = \bigl\{ 2\i, \; -2+\i,\;  1+\i \bigr\}$.


 
  %</content>
\end{document}