\documentclass[12pt, a4paper]{report}

% LuaLaTeX :

\RequirePackage{iftex}
\RequireLuaTeX

% Packages :

\usepackage[french]{babel}
%\usepackage[utf8]{inputenc}
%\usepackage[T1]{fontenc}
\usepackage[pdfencoding=auto, pdfauthor={Hugo Delaunay}, pdfsubject={Mathématiques}, pdfcreator={agreg.skyost.eu}]{hyperref}
\usepackage{amsmath}
\usepackage{amsthm}
%\usepackage{amssymb}
\usepackage{stmaryrd}
\usepackage{tikz}
\usepackage{tkz-euclide}
\usepackage{fontspec}
\defaultfontfeatures[Erewhon]{FontFace = {bx}{n}{Erewhon-Bold.otf}}
\usepackage{fourier-otf}
\usepackage[nobottomtitles*]{titlesec}
\usepackage{fancyhdr}
\usepackage{listings}
\usepackage{catchfilebetweentags}
\usepackage[french, capitalise, noabbrev]{cleveref}
\usepackage[fit, breakall]{truncate}
\usepackage[top=2.5cm, right=2cm, bottom=2.5cm, left=2cm]{geometry}
\usepackage{enumitem}
\usepackage{tocloft}
\usepackage{microtype}
%\usepackage{mdframed}
%\usepackage{thmtools}
\usepackage{xcolor}
\usepackage{tabularx}
\usepackage{xltabular}
\usepackage{aligned-overset}
\usepackage[subpreambles=true]{standalone}
\usepackage{environ}
\usepackage[normalem]{ulem}
\usepackage{etoolbox}
\usepackage{setspace}
\usepackage[bibstyle=reading, citestyle=draft]{biblatex}
\usepackage{xpatch}
\usepackage[many, breakable]{tcolorbox}
\usepackage[backgroundcolor=white, bordercolor=white, textsize=scriptsize]{todonotes}
\usepackage{luacode}
\usepackage{float}
\usepackage{needspace}
\everymath{\displaystyle}

% Police :

\setmathfont{Erewhon Math}

% Tikz :

\usetikzlibrary{calc}
\usetikzlibrary{3d}

% Longueurs :

\setlength{\parindent}{0pt}
\setlength{\headheight}{15pt}
\setlength{\fboxsep}{0pt}
\titlespacing*{\chapter}{0pt}{-20pt}{10pt}
\setlength{\marginparwidth}{1.5cm}
\setstretch{1.1}

% Métadonnées :

\author{agreg.skyost.eu}
\date{\today}

% Titres :

\setcounter{secnumdepth}{3}

\renewcommand{\thechapter}{\Roman{chapter}}
\renewcommand{\thesubsection}{\Roman{subsection}}
\renewcommand{\thesubsubsection}{\arabic{subsubsection}}
\renewcommand{\theparagraph}{\alph{paragraph}}

\titleformat{\chapter}{\huge\bfseries}{\thechapter}{20pt}{\huge\bfseries}
\titleformat*{\section}{\LARGE\bfseries}
\titleformat{\subsection}{\Large\bfseries}{\thesubsection \, - \,}{0pt}{\Large\bfseries}
\titleformat{\subsubsection}{\large\bfseries}{\thesubsubsection. \,}{0pt}{\large\bfseries}
\titleformat{\paragraph}{\bfseries}{\theparagraph. \,}{0pt}{\bfseries}

\setcounter{secnumdepth}{4}

% Table des matières :

\renewcommand{\cftsecleader}{\cftdotfill{\cftdotsep}}
\addtolength{\cftsecnumwidth}{10pt}

% Redéfinition des commandes :

\renewcommand*\thesection{\arabic{section}}
\renewcommand{\ker}{\mathrm{Ker}}

% Nouvelles commandes :

\newcommand{\website}{https://github.com/imbodj/SenCoursDeMaths}

\newcommand{\tr}[1]{\mathstrut ^t #1}
\newcommand{\im}{\mathrm{Im}}
\newcommand{\rang}{\operatorname{rang}}
\newcommand{\trace}{\operatorname{trace}}
\newcommand{\id}{\operatorname{id}}
\newcommand{\stab}{\operatorname{Stab}}
\newcommand{\paren}[1]{\left(#1\right)}
\newcommand{\croch}[1]{\left[ #1 \right]}
\newcommand{\Grdcroch}[1]{\Bigl[ #1 \Bigr]}
\newcommand{\grdcroch}[1]{\bigl[ #1 \bigr]}
\newcommand{\abs}[1]{\left\lvert #1 \right\rvert}
\newcommand{\limi}[3]{\lim_{#1\to #2}#3}
\newcommand{\pinf}{+\infty}
\newcommand{\minf}{-\infty}
%%%%%%%%%%%%%% ENSEMBLES %%%%%%%%%%%%%%%%%
\newcommand{\ensemblenombre}[1]{\mathbb{#1}}
\newcommand{\Nn}{\ensemblenombre{N}}
\newcommand{\Zz}{\ensemblenombre{Z}}
\newcommand{\Qq}{\ensemblenombre{Q}}
\newcommand{\Qqp}{\Qq^+}
\newcommand{\Rr}{\ensemblenombre{R}}
\newcommand{\Cc}{\ensemblenombre{C}}
\newcommand{\Nne}{\Nn^*}
\newcommand{\Zze}{\Zz^*}
\newcommand{\Zzn}{\Zz^-}
\newcommand{\Qqe}{\Qq^*}
\newcommand{\Rre}{\Rr^*}
\newcommand{\Rrp}{\Rr_+}
\newcommand{\Rrm}{\Rr_-}
\newcommand{\Rrep}{\Rr_+^*}
\newcommand{\Rrem}{\Rr_-^*}
\newcommand{\Cce}{\Cc^*}
%%%%%%%%%%%%%%  INTERVALLES %%%%%%%%%%%%%%%%%
\newcommand{\intff}[2]{\left[#1\;,\; #2\right]  }
\newcommand{\intof}[2]{\left]#1 \;, \;#2\right]  }
\newcommand{\intfo}[2]{\left[#1 \;,\; #2\right[  }
\newcommand{\intoo}[2]{\left]#1 \;,\; #2\right[  }

\providecommand{\newpar}{\\[\medskipamount]}

\newcommand{\annexessection}{%
  \newpage%
  \subsection*{Annexes}%
}

\providecommand{\lesson}[3]{%
  \title{#3}%
  \hypersetup{pdftitle={#2 : #3}}%
  \setcounter{section}{\numexpr #2 - 1}%
  \section{#3}%
  \fancyhead[R]{\truncate{0.73\textwidth}{#2 : #3}}%
}

\providecommand{\development}[3]{%
  \title{#3}%
  \hypersetup{pdftitle={#3}}%
  \section*{#3}%
  \fancyhead[R]{\truncate{0.73\textwidth}{#3}}%
}

\providecommand{\sheet}[3]{\development{#1}{#2}{#3}}

\providecommand{\ranking}[1]{%
  \title{Terminale #1}%
  \hypersetup{pdftitle={Terminale #1}}%
  \section*{Terminale #1}%
  \fancyhead[R]{\truncate{0.73\textwidth}{Terminale #1}}%
}

\providecommand{\summary}[1]{%
  \textit{#1}%
  \par%
  \medskip%
}

\tikzset{notestyleraw/.append style={inner sep=0pt, rounded corners=0pt, align=center}}

%\newcommand{\booklink}[1]{\website/bibliographie\##1}
\newcounter{reference}
\newcommand{\previousreference}{}
\providecommand{\reference}[2][]{%
  \needspace{20pt}%
  \notblank{#1}{
    \needspace{20pt}%
    \renewcommand{\previousreference}{#1}%
    \stepcounter{reference}%
    \label{reference-\previousreference-\thereference}%
  }{}%
  \todo[noline]{%
    \protect\vspace{20pt}%
    \protect\par%
    \protect\notblank{#1}{\cite{[\previousreference]}\\}{}%
    \protect\hyperref[reference-\previousreference-\thereference]{p. #2}%
  }%
}

\definecolor{devcolor}{HTML}{00695c}
\providecommand{\dev}[1]{%
  \reversemarginpar%
  \todo[noline]{
    \protect\vspace{20pt}%
    \protect\par%
    \bfseries\color{devcolor}\href{\website/developpements/#1}{[DEV]}
  }%
  \normalmarginpar%
}

% En-têtes :

\pagestyle{fancy}
\fancyhead[L]{\truncate{0.23\textwidth}{\thepage}}
\fancyfoot[C]{\scriptsize \href{\website}{\texttt{https://github.com/imbodj/SenCoursDeMaths}}}

% Couleurs :

\definecolor{property}{HTML}{ffeb3b}
\definecolor{proposition}{HTML}{ffc107}
\definecolor{lemma}{HTML}{ff9800}
\definecolor{theorem}{HTML}{f44336}
\definecolor{corollary}{HTML}{e91e63}
\definecolor{definition}{HTML}{673ab7}
\definecolor{notation}{HTML}{9c27b0}
\definecolor{example}{HTML}{00bcd4}
\definecolor{cexample}{HTML}{795548}
\definecolor{application}{HTML}{009688}
\definecolor{remark}{HTML}{3f51b5}
\definecolor{algorithm}{HTML}{607d8b}
%\definecolor{proof}{HTML}{e1f5fe}
\definecolor{exercice}{HTML}{e1f5fe}

% Théorèmes :

\theoremstyle{definition}
\newtheorem{theorem}{Théorème}

\newtheorem{property}[theorem]{Propriété}
\newtheorem{proposition}[theorem]{Proposition}
\newtheorem{lemma}[theorem]{Activité d'introduction}
\newtheorem{corollary}[theorem]{Conséquence}

\newtheorem{definition}[theorem]{Définition}
\newtheorem{notation}[theorem]{Notation}

\newtheorem{example}[theorem]{Exemple}
\newtheorem{cexample}[theorem]{Contre-exemple}
\newtheorem{application}[theorem]{Application}

\newtheorem{algorithm}[theorem]{Algorithme}
\newtheorem{exercice}[theorem]{Exercice}

\theoremstyle{remark}
\newtheorem{remark}[theorem]{Remarque}

\counterwithin*{theorem}{section}

\newcommand{\applystyletotheorem}[1]{
  \tcolorboxenvironment{#1}{
    enhanced,
    breakable,
    colback=#1!8!white,
    %right=0pt,
    %top=8pt,
    %bottom=8pt,
    boxrule=0pt,
    frame hidden,
    sharp corners,
    enhanced,borderline west={4pt}{0pt}{#1},
    %interior hidden,
    sharp corners,
    after=\par,
  }
}

\applystyletotheorem{property}
\applystyletotheorem{proposition}
\applystyletotheorem{lemma}
\applystyletotheorem{theorem}
\applystyletotheorem{corollary}
\applystyletotheorem{definition}
\applystyletotheorem{notation}
\applystyletotheorem{example}
\applystyletotheorem{cexample}
\applystyletotheorem{application}
\applystyletotheorem{remark}
%\applystyletotheorem{proof}
\applystyletotheorem{algorithm}
\applystyletotheorem{exercice}

% Environnements :

\NewEnviron{whitetabularx}[1]{%
  \renewcommand{\arraystretch}{2.5}
  \colorbox{white}{%
    \begin{tabularx}{\textwidth}{#1}%
      \BODY%
    \end{tabularx}%
  }%
}

% Maths :

\DeclareFontEncoding{FMS}{}{}
\DeclareFontSubstitution{FMS}{futm}{m}{n}
\DeclareFontEncoding{FMX}{}{}
\DeclareFontSubstitution{FMX}{futm}{m}{n}
\DeclareSymbolFont{fouriersymbols}{FMS}{futm}{m}{n}
\DeclareSymbolFont{fourierlargesymbols}{FMX}{futm}{m}{n}
\DeclareMathDelimiter{\VERT}{\mathord}{fouriersymbols}{152}{fourierlargesymbols}{147}

% Code :

\definecolor{greencode}{rgb}{0,0.6,0}
\definecolor{graycode}{rgb}{0.5,0.5,0.5}
\definecolor{mauvecode}{rgb}{0.58,0,0.82}
\definecolor{bluecode}{HTML}{1976d2}
\lstset{
  basicstyle=\footnotesize\ttfamily,
  breakatwhitespace=false,
  breaklines=true,
  %captionpos=b,
  commentstyle=\color{greencode},
  deletekeywords={...},
  escapeinside={\%*}{*)},
  extendedchars=true,
  frame=none,
  keepspaces=true,
  keywordstyle=\color{bluecode},
  language=Python,
  otherkeywords={*,...},
  numbers=left,
  numbersep=5pt,
  numberstyle=\tiny\color{graycode},
  rulecolor=\color{black},
  showspaces=false,
  showstringspaces=false,
  showtabs=false,
  stepnumber=2,
  stringstyle=\color{mauvecode},
  tabsize=2,
  %texcl=true,
  xleftmargin=10pt,
  %title=\lstname
}

\newcommand{\codedirectory}{}
\newcommand{\inputalgorithm}[1]{%
  \begin{algorithm}%
    \strut%
    \lstinputlisting{\codedirectory#1}%
  \end{algorithm}%
}




\begin{document}
  %<*content>
  \lesson{analysis}{206}{Exemples d'utilisation de la notion de dimension finie en analyse.}

  \subsection{Espaces vectoriels normés}

  \subsubsection{Complétude}

  \label{206-1}

  \reference[DAN]{52}

  Soit $(E, d)$ un espace métrique.

  \begin{definition}
    On dit que $E$ est complet si toute suite de Cauchy de $E$ est convergente dans $E$.
  \end{definition}

  \begin{example}
    \begin{itemize}
      \item $(\mathbb{R}, \vert . \vert)$ est complet.
      \item $(\mathbb{R}^p, \vert . \vert)$ est complet pour tout $p \in \mathbb{N}^*$.
    \end{itemize}
  \end{example}

  \begin{proposition}
    On suppose que $E$ est un espace métrique complet. Soit $A \subseteq E$. Alors $(A,d)$ est complet si et seulement si $A$ est une partie fermée de $E$.
  \end{proposition}

  \begin{proposition}
    On suppose que $E$ est un espace vectoriel sur $\mathbb{R}$ de dimension finie $n \geq 1$ muni de la norme infinie $\Vert . \Vert_\infty$. Alors $E$ est un espace vectoriel normé complet.
  \end{proposition}

  \begin{cexample}
    L'espace des fonctions polynômiales définies sur $[-1,1]$ et muni de la norme $\Vert . \Vert_{\infty}$ n'est pas complet.
  \end{cexample}

  \begin{application}[Théorème du point fixe de Banach]
    Soient $(E,d)$ un espace métrique complet et $f : E \rightarrow E$ une application contractant (ie. $\exists k \in ]0,1[ \text{ tel que } \forall x, y \in E, \, d(f(x), f(y)) \leq d(x, y)$). Alors,
    \[ \exists! x \in E \text{ tel que } f(x) = x \]
    De plus la suite des itérés définie par $x_0 \in E$ et $\forall n \in \mathbb{N}, x_{n+1} = f(x_n)$ converge vers $x$.
  \end{application}

  \begin{application}[Théorème de prolongement des applications uniformément continues]
    Soient $(E,d_E)$ et $(F,d_F)$ des espaces métriques. On suppose $F$ complet. Soient $A \subseteq E$ dense et $f : A \rightarrow F$ une application uniformément continue. Alors, il existe une unique application $\widehat{f} : E \rightarrow F$ uniformément continue et telle que $\widehat{f}_{|A} = f$.
  \end{application}

  \subsubsection{Compacité}

  \label{206-2}

  \reference[DAN]{51}

  Soit $(E, d)$ un espace métrique.

  \begin{definition}
    Un espace métrique est \textbf{compact} s'il vérifie la propriété de Bolzano-Weierstrass :
    \begin{center}
      \textit{De toute suite de l'espace on peut extraire une sous-suite convergente dans cet espace.}
    \end{center}
  \end{definition}

  \begin{example}
    Tout segment $[a,b]$ de $\mathbb{R}$ est compact, mais $\mathbb{R}$ n'est pas compact.
  \end{example}

  \begin{proposition}
    \begin{enumerate}[label=(\roman*)]
      \item Un espace métrique compact est complet.
      \item Un espace métrique compact est borné.
    \end{enumerate}
  \end{proposition}

  \begin{proposition}
    Soit $A \subseteq E$.
    \begin{enumerate}[label=(\roman*)]
      \item Si $A$ est compacte, alors $A$ est une partie fermée bornée de $E$.
      \item Si $E$ est compact et $A$ est fermée, alors $A$ est compacte.
    \end{enumerate}
  \end{proposition}

  \begin{proposition}
    Un produit d'espaces métriques compacts est compact pour la distance produit.
  \end{proposition}

  \begin{proposition}
    On suppose que $E$ est un espace vectoriel de dimension finie $n \geq 1$ muni de la norme infinie $\Vert . \Vert_\infty$. Les compacts de cet espace vectoriel normé sont les parties fermées et bornées.
  \end{proposition}

  \begin{application}
    Un intervalle de $\mathbb{R}$ est compact si et seulement si c'est un segment.
  \end{application}

  \subsubsection{Équivalence des normes}

  \reference[LI]{15}

  Soit $E$ un espace vectoriel.

  \begin{definition}
    On dit que deux normes $\Vert . \Vert_1$ et $\Vert . \Vert_2$ sur $E$ sont équivalentes si
    \[ \exists \alpha, \beta > 0 \text{ tels que } \forall x \in E, \, \alpha \Vert x \Vert_2 \leq \Vert x \Vert_1 \leq \beta \Vert x \Vert_2 \]
  \end{definition}

  \begin{remark}
    Deux normes équivalentes sur $E$ définissent la même topologie sur $E$.
  \end{remark}

  \dev{equivalence-des-normes-en-dimension-finie-et-theoreme-de-riesz}

  \begin{theorem}
    En dimension finie, toutes les normes sont équivalentes.
  \end{theorem}

  Le corollaire suivant justifie l'étude de la compacité dans la \cref{206-2}.

  \begin{corollary}
    Les parties compactes d'un espace vectoriel normé de dimension finie sont les parties fermées bornées.
  \end{corollary}

  Et le corollaire suivant justifie l'étude de la complétude dans la \cref{206-1}.

  \begin{corollary}
    \begin{enumerate}[label=(\roman*)]
      \item Tout espace vectoriel de dimension finie est complet.
      \item Tout espace vectoriel de dimension finie dans un espace vectoriel normé est fermé dans cet espace.
      \item Si $E$ est un espace vectoriel normé, alors toute application linéaire $T : E \rightarrow F$ (où $F$ désigne un espace vectoriel normé arbitraire) est continue.
    \end{enumerate}
  \end{corollary}

  \reference[DAN]{58}

  \begin{application}[Théorème de d'Alembert-Gauss]
    Tout polynôme non constant de $\mathbb{C}$ admet une racine dans $\mathbb{C}$.
  \end{application}

  \reference[C-G]{407}

  \begin{application}
    L'exponentielle d'une matrice est un polynôme en la matrice.
  \end{application}

  \reference[LI]{17}

  \begin{theorem}[Riesz]
    La boule unité fermée d'un espace vectoriel normé est compacte si et seulement s'il est dimension finie.
  \end{theorem}

  \subsubsection{Applications linéaires}

  \reference[GOU20]{48}

  Soient $(E, \Vert . \Vert_E)$ et $(F, \Vert . \Vert_F)$ deux espaces vectoriels normés sur $\mathbb{K} = \mathbb{R}$ ou $\mathbb{C}$.

  \begin{notation}
    On note $L(E,F)$ l'ensemble des applications linéaires de $E$ dans $F$ et $\mathcal{L}(E,F)$ l'ensemble des applications linéaires continues de $E$ dans $F$. Si $E = F$, on note $L(E,F) = L(E)$ et $\mathcal{L}(E,F) = \mathcal{L}(E)$.
  \end{notation}

  \begin{theorem}
    Soit $f \in L(E,F)$. Les assertions suivantes sont équivalentes.
    \begin{enumerate}[label=(\roman*)]
      \item $f \in \mathcal{L}(E,F)$.
      \item $f$ est continue en $0$.
      \item $f$ est bornée sur $\overline{B}(0,1) \subseteq E$.
      \item $f$ est bornée sur $S(0,1) \subseteq E$.
      \item Il existe $M \geq 0$ tel que $\Vert f(x) \Vert_F \leq M \Vert x \Vert_E$.
      \item $f$ est lipschitzienne.
      \item $f$ est uniformément continue sur $E$.
    \end{enumerate}
  \end{theorem}

  \begin{proposition}
    Toute application linéaire d'un espace vectoriel normé de dimension finie dans un espace vectoriel normé quelconque est continue.
  \end{proposition}

  \begin{cexample}
    La dérivation sur $\mathbb{K}[X]$, $P \mapsto P'$ n'est pas continue.
  \end{cexample}

  \subsection{Espaces de Hilbert}

  \subsubsection{Espaces de Hilbert et dimension finie}

  \reference[LI]{31}

  Soit $\mathbb{K} = \mathbb{R}$ ou $\mathbb{C}$.

  \begin{definition}
    Un espace vectoriel $H$ sur le corps $\mathbb{K}$ est un \textbf{espace de Hilbert} s'il est muni d'un produit scalaire $\langle . , . \rangle$ et est complet pour la norme associée $\Vert . \Vert = \sqrt{\langle . , . \rangle}$.
  \end{definition}

  \begin{example}
    Tout espace préhilbertien (ie. muni d'un produit scalaire) de dimension finie est un espace de Hilbert.
  \end{example}

  \begin{theorem}[Projection sur un convexe fermé]
    Soit $C \subseteq H$ un convexe fermé non-vide. Alors :
    \[ \forall x \in H, \exists! y \in C \text{ tel que } d(x, C) = \inf_{z \in C} \Vert x - z \Vert = d(x, y) \]
    On peut donc noter $y = P_C(x)$, le \textbf{projeté orthogonal de $x$ sur $C$}. Il s'agit de l'unique point de $C$ vérifiant
    \[ \forall z \in C, \, \langle x - P_C(x), z - P_C(x) \rangle \leq 0 \]
  \end{theorem}

  \begin{theorem}
    Si $F$ est un sous espace vectoriel fermé dans $H$ (par exemple, si $F$ est de dimension finie), alors $P_F$ est une application linéaire continue. De plus, pour tout $x \in H$, $P_F(x)$ est l'unique point $y \in F$ tel que $x-y \in F^\perp$.
  \end{theorem}

  \begin{theorem}
    Si $F$ est un sous espace vectoriel fermé dans $H$ (par exemple, si $F$ est de dimension finie), alors
    \[ H = F \oplus F^\perp \]
    et $P_F$ est la projection sur $F$ parallèlement à $F^\perp$ : c'est la \textbf{projection orthogonale} sur $F$.
  \end{theorem}

  \reference[BMP]{93}

  \begin{remark}
    En reprenant les notations précédentes, en supposant $F$ de dimension finie et en notant $(e_1, \dots, e_n)$ une base orthonormée de $F$, alors
    \[ \forall x \in H, \, p_F(x) = \sum_{i=1}^{n} \langle x, e_i \rangle e_i \]
  \end{remark}

  \subsubsection{Séries de Fourier}

  \reference[Z-Q]{73}

  \begin{notation}
    \begin{itemize}
      \item Pour tout $p \in [1, +\infty]$, on note $L_p^{2\pi}$ l'espace des fonctions $f : \mathbb{R} \rightarrow \mathbb{C}$, $2\pi$-périodiques et mesurables, telles que $\Vert f \Vert_p < +\infty$.
      \item Pour tout $n \in \mathbb{Z}$, on note $e_n$ la fonction $2\pi$-périodique définie pour tout $t \in \mathbb{R}$ par $e_n(t) = e^{int}$.
    \end{itemize}
  \end{notation}

  \begin{remark}
    \[ 1 \leq p < q \leq +\infty \implies L_q^{2\pi} \subseteq L_p^{2\pi} \text{ et } \Vert . \Vert_p \leq \Vert . \Vert_q \]
  \end{remark}

  \begin{proposition}
    $L_2^{2\pi}$ est un espace de Hilbert pour le produit scalaire
    \[ \langle ., . \rangle : (f, g) \mapsto \frac{1}{2 \pi} \int_0^{2\pi} f(t) \overline{g(t)} \, \mathrm{d}t \]
  \end{proposition}

  \reference[BMP]{123}

  \begin{theorem}
    La famille $(e_n)_{n \in \mathbb{Z}}$ est une base hilbertienne (totale et orthonormée) de $L_2^{2 \pi}$.
  \end{theorem}

  \reference[GOU21]{270}

  \begin{corollary}
    Soit $n \geq 1$. On note \[ \mathcal{P}_n = \operatorname{Vect}(e_k)_{k \in \llbracket 1, n \rrbracket} \]
    le sous-espace vectoriel des polynômes trigonométriques de degré $n$. Alors :
    \begin{enumerate}[label=(\roman*)]
      \item $L_2^{2\pi} = \mathcal{P}_n \oplus \mathcal{P}_n^\perp$.
      \item $P_{\mathcal{P}_n}(f) = S_n(f)$
      où $S_n(f)$ est la somme partielle d'ordre $n$ de la série de Fourier de $f$.
      \item $\inf_{g \in \mathcal{P}_n} \Vert f-g \Vert^2 = \Vert f-S_n(f) \Vert^2 = \frac{1}{2\pi} \int_0^{2\pi} \vert f(t) \vert^2 \, \mathrm{d}t - \sum_{k=-n}^{n} \vert c_k(f) \vert^2$
      où $c_k(f)$ est le $k$-ième coefficient de Fourier.
    \end{enumerate}
  \end{corollary}

  \begin{application}[Inégalité de Beissel]
    \[ \sum_{k=-\infty}^{+\infty} \vert c_k(f) \vert^2 \leq \frac{1}{2\pi} \int_0^{2\pi} \vert f(t) \vert^2 \, \mathrm{d}t \]
  \end{application}

  \begin{remark}
    Cette inégalité est en fait une égalité : c'est l'égalité de Parseval.
  \end{remark}

  \begin{example}
    On considère $f : x \mapsto 1 - \frac{x^2}{\pi^2}$ sur $[-\pi, \pi]$. Alors,
    \[ \frac{\pi^4}{90} = \Vert f \Vert_2 = \sum_{n=0}^{+\infty} \frac{1}{n^4} \]
  \end{example}

  \subsection{Calcul différentiel}

  \subsubsection{Différentielle et dérivées partielles}

  \reference[GOU20]{323}

  Soient $(E, \Vert . \Vert_E)$ et $(F, \Vert . \Vert_F)$ deux espaces vectoriels normés sur $\mathbb{R}$. Soient $U \subseteq E$ ouvert et $f : U \rightarrow F$ une application de $U$ dans $F$.

  \begin{definition}
    $f$ est dite \textbf{différentiable} en un point $a$ de $U$ s'il existe $\ell_a \in \mathcal{L}(E,F)$ telle que
    \[ f(a+h) = f(a) + \ell_a(h) + o(\Vert h \Vert_E) \text{ quand } h \longrightarrow 0 \]
    Si $\ell_a$ existe, alors elle est unique et on la note $\mathrm{d}f_a$ : c'est la \textbf{différentielle} de $f$ en $a$.
  \end{definition}

  \begin{remark}
    \begin{itemize}
      \item En dimension quelconque $\mathrm{d}f_a$ dépend a priori des normes $\Vert . \Vert_E$ et $\Vert . \Vert_F$. Cependant, en dimension finie, l'équivalence des normes implique que l'existence et la valeur de $\mathrm{d}f_a$ ne dépend pas des normes choisies.
      \item La définition demande à $\ell_a$ d'être continue. En dimension finie, le problème ne se pose donc pas.
    \end{itemize}
  \end{remark}

  \begin{example}
    Si $f$ est linéaire et continue, alors $\mathrm{d}f_a = f$ pour tout $a \in E$.
  \end{example}

  On se place maintenant dans le cas où $E = \mathbb{R}^n$.

  \begin{definition}
    Soit $a \in U$.
    \begin{itemize}
      \item Soit $v \in E$. Si la fonction de la variable réelle $\varphi : t \mapsto f(a+tv)$ est dérivable en $0$, on dit que $f$ est \textbf{dérivable en $a$ selon le vecteur $v$}. On note alors
      \[ f'_v(a) = \varphi'(0) \]
      \item Soit $(e_1, \dots, e_n)$ la base canonique de $\mathbb{R}^n$ et soit $i \in \llbracket 1, n \rrbracket$. On dit que $f$ admet une \textbf{$i$-ième dérivée partielle en $a$} si $f$ est dérivable en $a$ selon le vecteur $e_i$. On note alors
      \[ \frac{\partial f}{\partial x_i}(a) = f'_{e_i}(a) \]
    \end{itemize}
  \end{definition}

  \begin{proposition}
    Une fonction différentiable en un point est dérivable selon tout vecteur en ce point.
  \end{proposition}

  \begin{cexample}
    La fonction
    \[
      \begin{array}{ccc}
        \mathbb{R}^2 &\rightarrow& \mathbb{R} \\
        (x,y) &\mapsto& \begin{cases}
          \frac{y^2}{x} &\text{si } x \neq 0 \\
          y &\text{sinon}
        \end{cases}
      \end{array}
    \]
    est dérivable selon tout vecteur au point $(0,0)$ mais n'est pas continue en $(0,0)$.
  \end{cexample}

  \begin{theorem}
    Si toutes les dérivées partielles de $f$ existent et si elles sont continues en un point $a$ de $U$, alors $f$ est différentiable en $a$ et on a
    \[ \mathrm{d}f_a = \sum_{i=1}^n \frac{\partial f}{\partial x_i}(a) e_i^* \]
    où $(e_i^*)_{i \in \llbracket 1, n \rrbracket}$ est la base duale de la base canonique $(e_i)_{i \in \llbracket 1, n \rrbracket}$ de $\mathbb{R}^n$.
  \end{theorem}

  \begin{application}
    Soit $a \in U$. Si $F = \mathbb{R}^m$, la matrice de $\mathrm{d}f_a$ dans les bases canoniques de $\mathbb{R}^n$ et $\mathbb{R}^m$ est
    \[ \left(\frac{\partial f_i}{\partial x_j}(a)\right)_{\substack{i \in \llbracket 1, n \rrbracket \\ j \in \llbracket 1, m \rrbracket}} \]
    (où l'on a noté $f = (f_1, \dots, f_m))$) : c'est la \textbf{matrice jacobienne} de $f$ en $a$.
  \end{application}

  \subsubsection{Équations différentielles linéaires}

  \reference{373}

  \begin{definition}
    Soient $n \in \mathbb{N}^*$, $E$ un espace de Banach et $\Omega \subseteq \mathbb{R} \times E^n$ un ouvert. Soit $F : \Omega \times \mathbb{R}^n \rightarrow E$ une fonction.
    \begin{itemize}
      \item On appelle \textbf{équation différentielle} une équation de la forme
      \[ y^{(n)} = F(t, y, y', \dots, y^{(n-1)}) \tag{$*$} \]
      (ie. une équation portant sur les dérivées d'une fonction.)
      \item Toute application $\varphi : I \rightarrow E$ (où $I$ est un intervalle de $\mathbb{R}$) $n$ fois dérivable vérifiant :
      \begin{enumerate}[label=(\roman*)]
        \item $\forall t \in I, \, (t, \varphi(t), \dots, \varphi^{(n-1)}(t)) \in \Omega$ ;
        \item $\forall t \in I, \, F(t, \varphi(t), \dots, \varphi^{(n-1)}(t)) = \varphi^{(n)}(t)$ ;
      \end{enumerate}
      est une \textbf{solution} de $(*)$. On note $\mathcal{S}_*$ l'ensemble des solutions de $(*)$.
      \item Une solution $\varphi : I \rightarrow E$ de $(*)$ est dite \textbf{maximale} s'il n'existe pas d'autre solution $\psi : J \rightarrow E$ (où $J$ est un intervalle de $\mathbb{R}$) de $(*)$ telle que $I \subseteq J$, $I \neq J$ et $\psi = \varphi$ sir $I$.
      \item On appelle \textbf{problème de Cauchy} de $(*)$ en $(t_0, x_0, \dots, x_{n-1})$ la recherche d'une solution $\varphi : I \rightarrow E$ de $(*)$ vérifiant
      \[ \forall t_0 \in I, \, \varphi(t_0) = x_0, \dots, \varphi^{(n-1)}(t_0) = x_{n-1} \]
    \end{itemize}
  \end{definition}

  \reference{377}

  \begin{definition}
    Toute équation différentielle sur $\mathbb{K}^n$ d'ordre $p \geq 1$ du type
    \[ Y^{(p)} = A_{p-1}(t) Y^{(p-1)} + \dots + A_0(t) Y + B(t) \tag{$L$} \]
    (où $A_{p-1}, \dots, A_0$ sont des fonctions continues d'un intervalle $I$ de $\mathbb{R}$ non réduit à un point dans $\mathcal{M}_n(\mathbb{K})$ et $B : I \rightarrow \mathbb{K}^n$ est une fonction continue) est appelée \textbf{équation différentielle linéaire} d'ordre $p$.
    \newpar
    Si de plus $B = 0$, alors $(L)$ est qualifiée d'\textbf{homogène}.
  \end{definition}

  \reference[DAN]{520}
  \dev{theoreme-de-cauchy-lipschitz-lineaire}

  \begin{theorem}[Cauchy-Lipschitz linéaire]
    Soient $A : I \rightarrow \mathcal{M}_n(\mathbb{K})$ et $B : I \rightarrow \mathbb{K}^d$ deux fonctions continues. Alors $\forall t_0 \in I$, le problème de Cauchy
    \[ \begin{cases} Y' = A(t)Y + B(t) \\ Y(t_0) = y_0 \end{cases} \]
    admet une unique solution définie sur $I$ tout entier.
  \end{theorem}

  \reference[ROM19-1]{402}

  \begin{example}
    Considérons l'équation $y' - y = 0$. Comme la fonction nulle est solution maximale, il s'agit de l'unique solution qui s'annule sur $\mathbb{R}$.
  \end{example}
  %</content>
\end{document}
