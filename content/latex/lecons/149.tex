\documentclass[12pt, a4paper]{report}

% LuaLaTeX :

\RequirePackage{iftex}
\RequireLuaTeX

% Packages :

\usepackage[french]{babel}
%\usepackage[utf8]{inputenc}
%\usepackage[T1]{fontenc}
\usepackage[pdfencoding=auto, pdfauthor={Hugo Delaunay}, pdfsubject={Mathématiques}, pdfcreator={agreg.skyost.eu}]{hyperref}
\usepackage{amsmath}
\usepackage{amsthm}
%\usepackage{amssymb}
\usepackage{stmaryrd}
\usepackage{tikz}
\usepackage{tkz-euclide}
\usepackage{fontspec}
\defaultfontfeatures[Erewhon]{FontFace = {bx}{n}{Erewhon-Bold.otf}}
\usepackage{fourier-otf}
\usepackage[nobottomtitles*]{titlesec}
\usepackage{fancyhdr}
\usepackage{listings}
\usepackage{catchfilebetweentags}
\usepackage[french, capitalise, noabbrev]{cleveref}
\usepackage[fit, breakall]{truncate}
\usepackage[top=2.5cm, right=2cm, bottom=2.5cm, left=2cm]{geometry}
\usepackage{enumitem}
\usepackage{tocloft}
\usepackage{microtype}
%\usepackage{mdframed}
%\usepackage{thmtools}
\usepackage{xcolor}
\usepackage{tabularx}
\usepackage{xltabular}
\usepackage{aligned-overset}
\usepackage[subpreambles=true]{standalone}
\usepackage{environ}
\usepackage[normalem]{ulem}
\usepackage{etoolbox}
\usepackage{setspace}
\usepackage[bibstyle=reading, citestyle=draft]{biblatex}
\usepackage{xpatch}
\usepackage[many, breakable]{tcolorbox}
\usepackage[backgroundcolor=white, bordercolor=white, textsize=scriptsize]{todonotes}
\usepackage{luacode}
\usepackage{float}
\usepackage{needspace}
\everymath{\displaystyle}

% Police :

\setmathfont{Erewhon Math}

% Tikz :

\usetikzlibrary{calc}
\usetikzlibrary{3d}

% Longueurs :

\setlength{\parindent}{0pt}
\setlength{\headheight}{15pt}
\setlength{\fboxsep}{0pt}
\titlespacing*{\chapter}{0pt}{-20pt}{10pt}
\setlength{\marginparwidth}{1.5cm}
\setstretch{1.1}

% Métadonnées :

\author{agreg.skyost.eu}
\date{\today}

% Titres :

\setcounter{secnumdepth}{3}

\renewcommand{\thechapter}{\Roman{chapter}}
\renewcommand{\thesubsection}{\Roman{subsection}}
\renewcommand{\thesubsubsection}{\arabic{subsubsection}}
\renewcommand{\theparagraph}{\alph{paragraph}}

\titleformat{\chapter}{\huge\bfseries}{\thechapter}{20pt}{\huge\bfseries}
\titleformat*{\section}{\LARGE\bfseries}
\titleformat{\subsection}{\Large\bfseries}{\thesubsection \, - \,}{0pt}{\Large\bfseries}
\titleformat{\subsubsection}{\large\bfseries}{\thesubsubsection. \,}{0pt}{\large\bfseries}
\titleformat{\paragraph}{\bfseries}{\theparagraph. \,}{0pt}{\bfseries}

\setcounter{secnumdepth}{4}

% Table des matières :

\renewcommand{\cftsecleader}{\cftdotfill{\cftdotsep}}
\addtolength{\cftsecnumwidth}{10pt}

% Redéfinition des commandes :

\renewcommand*\thesection{\arabic{section}}
\renewcommand{\ker}{\mathrm{Ker}}

% Nouvelles commandes :

\newcommand{\website}{https://github.com/imbodj/SenCoursDeMaths}

\newcommand{\tr}[1]{\mathstrut ^t #1}
\newcommand{\im}{\mathrm{Im}}
\newcommand{\rang}{\operatorname{rang}}
\newcommand{\trace}{\operatorname{trace}}
\newcommand{\id}{\operatorname{id}}
\newcommand{\stab}{\operatorname{Stab}}
\newcommand{\paren}[1]{\left(#1\right)}
\newcommand{\croch}[1]{\left[ #1 \right]}
\newcommand{\Grdcroch}[1]{\Bigl[ #1 \Bigr]}
\newcommand{\grdcroch}[1]{\bigl[ #1 \bigr]}
\newcommand{\abs}[1]{\left\lvert #1 \right\rvert}
\newcommand{\limi}[3]{\lim_{#1\to #2}#3}
\newcommand{\pinf}{+\infty}
\newcommand{\minf}{-\infty}
%%%%%%%%%%%%%% ENSEMBLES %%%%%%%%%%%%%%%%%
\newcommand{\ensemblenombre}[1]{\mathbb{#1}}
\newcommand{\Nn}{\ensemblenombre{N}}
\newcommand{\Zz}{\ensemblenombre{Z}}
\newcommand{\Qq}{\ensemblenombre{Q}}
\newcommand{\Qqp}{\Qq^+}
\newcommand{\Rr}{\ensemblenombre{R}}
\newcommand{\Cc}{\ensemblenombre{C}}
\newcommand{\Nne}{\Nn^*}
\newcommand{\Zze}{\Zz^*}
\newcommand{\Zzn}{\Zz^-}
\newcommand{\Qqe}{\Qq^*}
\newcommand{\Rre}{\Rr^*}
\newcommand{\Rrp}{\Rr_+}
\newcommand{\Rrm}{\Rr_-}
\newcommand{\Rrep}{\Rr_+^*}
\newcommand{\Rrem}{\Rr_-^*}
\newcommand{\Cce}{\Cc^*}
%%%%%%%%%%%%%%  INTERVALLES %%%%%%%%%%%%%%%%%
\newcommand{\intff}[2]{\left[#1\;,\; #2\right]  }
\newcommand{\intof}[2]{\left]#1 \;, \;#2\right]  }
\newcommand{\intfo}[2]{\left[#1 \;,\; #2\right[  }
\newcommand{\intoo}[2]{\left]#1 \;,\; #2\right[  }

\providecommand{\newpar}{\\[\medskipamount]}

\newcommand{\annexessection}{%
  \newpage%
  \subsection*{Annexes}%
}

\providecommand{\lesson}[3]{%
  \title{#3}%
  \hypersetup{pdftitle={#2 : #3}}%
  \setcounter{section}{\numexpr #2 - 1}%
  \section{#3}%
  \fancyhead[R]{\truncate{0.73\textwidth}{#2 : #3}}%
}

\providecommand{\development}[3]{%
  \title{#3}%
  \hypersetup{pdftitle={#3}}%
  \section*{#3}%
  \fancyhead[R]{\truncate{0.73\textwidth}{#3}}%
}

\providecommand{\sheet}[3]{\development{#1}{#2}{#3}}

\providecommand{\ranking}[1]{%
  \title{Terminale #1}%
  \hypersetup{pdftitle={Terminale #1}}%
  \section*{Terminale #1}%
  \fancyhead[R]{\truncate{0.73\textwidth}{Terminale #1}}%
}

\providecommand{\summary}[1]{%
  \textit{#1}%
  \par%
  \medskip%
}

\tikzset{notestyleraw/.append style={inner sep=0pt, rounded corners=0pt, align=center}}

%\newcommand{\booklink}[1]{\website/bibliographie\##1}
\newcounter{reference}
\newcommand{\previousreference}{}
\providecommand{\reference}[2][]{%
  \needspace{20pt}%
  \notblank{#1}{
    \needspace{20pt}%
    \renewcommand{\previousreference}{#1}%
    \stepcounter{reference}%
    \label{reference-\previousreference-\thereference}%
  }{}%
  \todo[noline]{%
    \protect\vspace{20pt}%
    \protect\par%
    \protect\notblank{#1}{\cite{[\previousreference]}\\}{}%
    \protect\hyperref[reference-\previousreference-\thereference]{p. #2}%
  }%
}

\definecolor{devcolor}{HTML}{00695c}
\providecommand{\dev}[1]{%
  \reversemarginpar%
  \todo[noline]{
    \protect\vspace{20pt}%
    \protect\par%
    \bfseries\color{devcolor}\href{\website/developpements/#1}{[DEV]}
  }%
  \normalmarginpar%
}

% En-têtes :

\pagestyle{fancy}
\fancyhead[L]{\truncate{0.23\textwidth}{\thepage}}
\fancyfoot[C]{\scriptsize \href{\website}{\texttt{https://github.com/imbodj/SenCoursDeMaths}}}

% Couleurs :

\definecolor{property}{HTML}{ffeb3b}
\definecolor{proposition}{HTML}{ffc107}
\definecolor{lemma}{HTML}{ff9800}
\definecolor{theorem}{HTML}{f44336}
\definecolor{corollary}{HTML}{e91e63}
\definecolor{definition}{HTML}{673ab7}
\definecolor{notation}{HTML}{9c27b0}
\definecolor{example}{HTML}{00bcd4}
\definecolor{cexample}{HTML}{795548}
\definecolor{application}{HTML}{009688}
\definecolor{remark}{HTML}{3f51b5}
\definecolor{algorithm}{HTML}{607d8b}
%\definecolor{proof}{HTML}{e1f5fe}
\definecolor{exercice}{HTML}{e1f5fe}

% Théorèmes :

\theoremstyle{definition}
\newtheorem{theorem}{Théorème}

\newtheorem{property}[theorem]{Propriété}
\newtheorem{proposition}[theorem]{Proposition}
\newtheorem{lemma}[theorem]{Activité d'introduction}
\newtheorem{corollary}[theorem]{Conséquence}

\newtheorem{definition}[theorem]{Définition}
\newtheorem{notation}[theorem]{Notation}

\newtheorem{example}[theorem]{Exemple}
\newtheorem{cexample}[theorem]{Contre-exemple}
\newtheorem{application}[theorem]{Application}

\newtheorem{algorithm}[theorem]{Algorithme}
\newtheorem{exercice}[theorem]{Exercice}

\theoremstyle{remark}
\newtheorem{remark}[theorem]{Remarque}

\counterwithin*{theorem}{section}

\newcommand{\applystyletotheorem}[1]{
  \tcolorboxenvironment{#1}{
    enhanced,
    breakable,
    colback=#1!8!white,
    %right=0pt,
    %top=8pt,
    %bottom=8pt,
    boxrule=0pt,
    frame hidden,
    sharp corners,
    enhanced,borderline west={4pt}{0pt}{#1},
    %interior hidden,
    sharp corners,
    after=\par,
  }
}

\applystyletotheorem{property}
\applystyletotheorem{proposition}
\applystyletotheorem{lemma}
\applystyletotheorem{theorem}
\applystyletotheorem{corollary}
\applystyletotheorem{definition}
\applystyletotheorem{notation}
\applystyletotheorem{example}
\applystyletotheorem{cexample}
\applystyletotheorem{application}
\applystyletotheorem{remark}
%\applystyletotheorem{proof}
\applystyletotheorem{algorithm}
\applystyletotheorem{exercice}

% Environnements :

\NewEnviron{whitetabularx}[1]{%
  \renewcommand{\arraystretch}{2.5}
  \colorbox{white}{%
    \begin{tabularx}{\textwidth}{#1}%
      \BODY%
    \end{tabularx}%
  }%
}

% Maths :

\DeclareFontEncoding{FMS}{}{}
\DeclareFontSubstitution{FMS}{futm}{m}{n}
\DeclareFontEncoding{FMX}{}{}
\DeclareFontSubstitution{FMX}{futm}{m}{n}
\DeclareSymbolFont{fouriersymbols}{FMS}{futm}{m}{n}
\DeclareSymbolFont{fourierlargesymbols}{FMX}{futm}{m}{n}
\DeclareMathDelimiter{\VERT}{\mathord}{fouriersymbols}{152}{fourierlargesymbols}{147}

% Code :

\definecolor{greencode}{rgb}{0,0.6,0}
\definecolor{graycode}{rgb}{0.5,0.5,0.5}
\definecolor{mauvecode}{rgb}{0.58,0,0.82}
\definecolor{bluecode}{HTML}{1976d2}
\lstset{
  basicstyle=\footnotesize\ttfamily,
  breakatwhitespace=false,
  breaklines=true,
  %captionpos=b,
  commentstyle=\color{greencode},
  deletekeywords={...},
  escapeinside={\%*}{*)},
  extendedchars=true,
  frame=none,
  keepspaces=true,
  keywordstyle=\color{bluecode},
  language=Python,
  otherkeywords={*,...},
  numbers=left,
  numbersep=5pt,
  numberstyle=\tiny\color{graycode},
  rulecolor=\color{black},
  showspaces=false,
  showstringspaces=false,
  showtabs=false,
  stepnumber=2,
  stringstyle=\color{mauvecode},
  tabsize=2,
  %texcl=true,
  xleftmargin=10pt,
  %title=\lstname
}

\newcommand{\codedirectory}{}
\newcommand{\inputalgorithm}[1]{%
  \begin{algorithm}%
    \strut%
    \lstinputlisting{\codedirectory#1}%
  \end{algorithm}%
}




\begin{document}
  %<*content>
  \lesson{algebra}{149}{Déterminant. Exemples et applications.}

  Soient $\mathbb{K}$ un corps commutatif et $E$ un espace vectoriel de dimension finie $n$ sur $\mathbb{K}$.

  \subsection{Construction}

  \subsubsection{Formes \texorpdfstring{$n$}{n}-linéaires alternées et déterminant}

  \reference[GOU21]{140}

  \begin{definition}
    Soient $E_1, \dots, E_p$ et $F$ des espaces vectoriels sur $\mathbb{K}$ et $f : E_1, \dots, E_p \rightarrow F$.
    \begin{itemize}
      \item $f$ est dite \textbf{$p$-linéaire} si en tout point les $p$ applications partielles sont linéaires.
      \item Si $f$ est $p$-linéaire et si $E_1 = \dots = E_p$ ainsi que $F = \mathbb{K}$, $f$ est une \textbf{forme $p$-linéaire}. On note $\mathcal{L}_p(E, \mathbb{K})$ l'ensemble des formes $p$-linéaires sur $E$.
      \item Si de plus $f(x_1, \dots, x_p) = 0$ dès que deux vecteurs parmi les $x_i$ sont égaux, alors $f$ est dite \textbf{alternée}.
    \end{itemize}
  \end{definition}

  \begin{example}
    En reprenant les notations précédentes, pour $p = 2$, $f$ est bilinéaire.
  \end{example}

  \begin{proposition}
    $\mathcal{L}_p(E, \mathbb{K})$ est un espace vectoriel et, $\operatorname{dim}(\mathcal{L}_p(E, \mathbb{K})) = \operatorname{dim}(E)^p$.
  \end{proposition}

  \begin{theorem}
    L'ensemble des formes $n$-linéaires alternées sur $E$ est un $\mathbb{K}$-espace vectoriel de dimension $1$. De plus, il existe une unique forme $n$-linéaire alternée $f$ prenant la valeur $1$ sur une base $\mathcal{B}$ de $E$. On note $f = \det_{\mathcal{B}}$.
  \end{theorem}

  \begin{definition}
    $\det_{\mathcal{B}}$ est l'application \textbf{déterminant} dans la base $\mathcal{B}$. En l'absence d'ambiguïté, on s'autorise à noter $\det = \det_{\mathcal{B}}$.
  \end{definition}

  \begin{proposition}
    Soit $\mathcal{B} = (e_1, \dots, e_n)$ une base de $E$. Si $x_1, \dots, x_n \in E$ ($\forall i \in \llbracket 1, n \rrbracket$, on peut écrire $x_i = \sum_{j=1}^n x_{i,j} e_j$), on a la formule $\det_{\mathcal{B}}(x_1, \dots, x_n) = \sum_{\sigma \in S_n} \epsilon(\sigma) \prod_{i=1}^n x_{i,\sigma(i)}$.
  \end{proposition}

  \begin{proposition}
    Soit $\mathcal{B}$ une base de $E$. Si $\mathcal{B}'$ est une autre base de $E$, alors $\det_{\mathcal{B}'} = \det_{\mathcal{B}'}(\mathcal{B}) \det_{\mathcal{B}}$.
  \end{proposition}

  \begin{theorem}
    Une famille de vecteurs est liée si et seulement si son déterminant est nul dans une base quelconque de $E$.
  \end{theorem}

  \subsubsection{Déterminant d'un endomorphisme}

  \begin{lemma}
    Soient $f \in \mathcal{L}(E)$ et $\mathcal{B} = (e_1, \dots, e_n)$ une base de $E$. Le scalaire $\det_{\mathcal{B}}(f(e_1), \dots, f(e_n))$ ne dépend pas de la base $\mathcal{B}$ considérée.
  \end{lemma}

  \begin{definition}
    Soient $f \in \mathcal{L}(E)$ et $\mathcal{B} = (e_1, \dots, e_n)$ une base de $E$. On appelle \textbf{déterminant} de $f$ le scalaire $\det_{\mathcal{B}}(f(e_1), \dots, f(e_n))$. On le note $\det(f)$.
  \end{definition}

  \begin{proposition}
    Soient $f, g \in \mathcal{L}(E)$.
    \begin{enumerate}[label=(\roman*)]
      \item $\det (f \circ g) = \det(f) \times \det(g)$.
      \item $\det (\operatorname{id}_e) = 1$.
      \item $f \in \mathrm{GL}(E) \iff \det(f) \neq 0$. Dans ce cas, on a $\det(f^{-1}) = \det(f)^{-1}$.
    \end{enumerate}
  \end{proposition}

  \subsubsection{Déterminant d'une matrice carrée}

  \begin{definition}
    Soit $A \in \mathcal{M}_n(\mathbb{K})$. On appelle \textbf{déterminant} de $A$, le déterminant de ses vecteurs colonnes dans la base canonique de $\mathbb{K}^n$. On le note $\det(A)$.
  \end{definition}

  \begin{notation}
    Si $A = (a_{i,j})_{i, j \in \llbracket 1, n \rrbracket} \in \mathcal{M}_n(\mathbb{K})$, on note son déterminant sous la forme
    \[
    \det(A) =
    \begin{vmatrix}
      a_{1,1} & \dots & a_{1,n} \\
      \vdots & \ddots & \vdots \\
      a_{n,1} & \dots & a_{n,n}
    \end{vmatrix}
    \]
  \end{notation}

  \reference[GRI]{104}

  \begin{example}
    \begin{itemize}
      \item $\begin{vmatrix}
        a & c \\
        b & d
      \end{vmatrix}
      = ad - bc$.
      \item $\begin{vmatrix}
        1 & -2 & 3 \\
        2 & 1 & -1 \\
        1 & 5 & 1
      \end{vmatrix}
      = 39$.
    \end{itemize}
  \end{example}

  \reference[GOU21]{142}

  \begin{proposition}
    Soit $A \in \mathcal{M}_n(\mathbb{K})$.
    \begin{enumerate}[label=(\roman*)]
      \item $\det(A) = \det(\tr{A})$.
      \item $\det(A)$ dépend linéairement des colonnes (resp. des lignes) de $A$.
      \item $\forall \lambda \in \mathbb{K}, \, \det(\lambda A) = \lambda^n \det(A)$.
      \item $\det(A) \neq 0 \iff A \in \mathrm{GL}_n(\mathbb{K})$.
      \item Si $A$ est la matrice de $f \in \mathcal{L}(E)$ dans une base, alors $\det(f) = \det(A)$.
      \item Si $B \in \mathcal{M}_n(\mathbb{K})$, $\det(AB) = \det(A) \det(B)$.
      \item Deux matrices semblables ont le même déterminant.
    \end{enumerate}
  \end{proposition}

  \subsection{Méthodes de calcul}

  \subsubsection{Propriétés}

  \begin{proposition}
    Soit $A \in \mathcal{M}_n(\mathbb{K})$.
    \begin{enumerate}[label=(\roman*)]
      \item Si on effectue une permutation $\sigma \in S_n$ sur les colonnes ou les lignes de $A$, le déterminant est multiplié par $\epsilon(\sigma)$ (la signature de $\sigma$).
      \item Si $A$ est triangulaire, $\det(A)$ est le produit des éléments diagonaux de $A$.
      \item On ne change pas la valeur d'un déterminant en ajoutant à une colonne une combinaison linéaire des autres colonnes. Même chose sur les lignes.
    \end{enumerate}
  \end{proposition}

  \begin{example}
    \[
    \begin{vmatrix}
      1 & 2 & 1 \\
      1 & 0 & 1 \\
      -1 & 1 & m
    \end{vmatrix}
    = \begin{vmatrix}
      1 & 2 & 0 \\
      1 & 0 & 0 \\
      -1 & 1 & m+1
    \end{vmatrix}
    = -2(m+1)
    \]
  \end{example}

  \begin{proposition}[Déterminant par blocs]
    Soit $M \in \mathcal{M}_n(\mathbb{K})$ une matrice triangulaire par blocs, de la forme
    \[
    M =
    \begin{pmatrix}
      A & C \\
      0 & B
    \end{pmatrix}
    \]
    alors $\det(M) = \det(A) \det(B)$.
  \end{proposition}

  \subsubsection{Mineurs et cofacteurs}

  \begin{definition}
    Soit $A = (a_{i,j})_{i, j \in \llbracket 1, n \rrbracket} \in \mathcal{M}_n(\mathbb{K})$.
    \begin{itemize}
      \item Pour tout $i,j \in \llbracket 1, n \rrbracket$, on appelle \textbf{mineur} de l'élément $a_{i,j}$ le déterminant $\Delta_{i,j}$ de la matrice obtenue en supprimant la $i$-ième ligne et la $j$-ième colonne de $A$.
      \item Le scalaire $A_{i,j} = (-1)^{i+j} \Delta_{i,j}$ s'appelle le \textbf{cofacteur} de $a_{i,j}$.
      \item On appelle \textbf{mineurs principaux} de $A$ les déterminants $\Delta_k = \det((a_{i,j})_{i,j \in \llbracket 1, k \rrbracket})$ pour $k \in \llbracket 1, n \rrbracket$.
    \end{itemize}
  \end{definition}

  \begin{proposition}
    En reprenant les notations précédentes :
    \begin{enumerate}[label=(\roman*)]
      \item Soit $j \in \llbracket 1, n \rrbracket$. On a $\det(A) = \sum_{i=1}^n a_{i,j} A_{i,j}$ (développement par rapport à la $j$-ième colonne).
      \item Soit $i \in \llbracket 1, n \rrbracket$. On a $\det(A) = \sum_{j=1}^n a_{i,j} A_{i,j}$ (développement par rapport à la $i$-ième ligne).
    \end{enumerate}
  \end{proposition}

  \reference[GRI]{118}

  \begin{example}
    \[
    \begin{vmatrix}
      6 & 0 & -6 \\
      0 & 2 & 7 \\
      0 & 2 & 3
    \end{vmatrix}
    = 6 \begin{vmatrix}
      2 & 7 \\
      2 & 3
    \end{vmatrix}
    = 6(6-14)
    = -48
    \]
  \end{example}

  \reference[GOU21]{143}

  \begin{definition}
    Soit $A \in \mathcal{M}_n(\mathbb{K})$. La matrice $(A_{i,j})_{i,j \in \llbracket 1, n \rrbracket}$ des cofacteurs des éléments de $A$ est appelée \textbf{comatrice} de $A$, et on la note $\operatorname{com}(A)$.
  \end{definition}

  \begin{proposition}
    Soit $A \in \mathcal{M}_n(\mathbb{K})$. On a :
    \[ A \tr{\operatorname{com}(A)} = \tr{\operatorname{com}(A)} A = \det(A) I_n \]
  \end{proposition}

  \begin{corollary}
    Soit $A \in \mathrm{GL}_n(\mathbb{K})$. Alors,
    \[ A^{-1} = \frac{1}{\det(A)} \tr{\operatorname{com}(A)} \]
  \end{corollary}

  \begin{example}
    Soit $A = \begin{pmatrix} a & b \\ c & d \end{pmatrix} \in \mathrm{GL}_2(\mathbb{K})$. Alors,
    \[ A^{-1} = \frac{1}{ad-bc} \begin{pmatrix} d & -b \\ -c & a \end{pmatrix} \]
  \end{example}

  \subsubsection{Exemples classiques}

  \begin{example}[Déterminant de Vandermonde]
    Soient $a_1, \dots, a_n \in \mathbb{K}$. Alors
    \[ \begin{vmatrix} 1 & a_1 & \dots & a_1^{n-1} \\ 1 & a_2 & \dots & a_2^{n-1} \\ \vdots & \vdots & \ddots & \vdots \\ 1 & a_n & \dots & a_n^{n-1} \end{vmatrix} = \prod_{1 \leq i < j \leq n} (a_j - a_i) \]
  \end{example}

  \reference{150}

  \begin{example}[Déterminant de Cauchy]
    Soient $a_1, \dots, a_n, b_1, \dots, b_n \in \mathbb{K}$ tels que pour tout $i, j \in \llbracket 1, n \rrbracket$, $a_i + b_j \neq 0$. Alors
    \[ \det \left( \frac{1}{a_i + b_j} \right) = \frac{\prod_{1 \leq i < j \leq n} (a_j - a_i) \prod_{1 \leq i < j \leq n} (b_j - b_i)}{\prod_{i,j = 1}^n (a_i + b_j)} \]
  \end{example}

  \reference{153}

  \begin{example}[Déterminant circulant]
    Soient $a_1, \dots, a_n \in \mathbb{C}$. On pose $\omega = e^{\frac{2i\pi}{n}}$. Alors
    \[ \begin{vmatrix} a_0 & a_1 & \dots & a_{n-1} \\ a_{n-1} & a_0 & \dots & a_{n-2}\\ \vdots & \vdots & \ddots & \vdots \\ a_1 & a_2 & \dots & a_0 \end{vmatrix} = \prod_{j=0}^{n-1} P(\omega^j) \]
    où $P = \sum_{k=0}^{n-1} a_k X^k$.
  \end{example}

  \subsection{Applications}

  \subsubsection{En géométrie}

  \paragraph{Volume d'un parallélépipède}

  \reference[GRI]{130}

  \begin{theorem}
    L'aire $\mathcal{A}(v,w)$ du parallélogramme engendré par deux vecteurs $v, w \in \mathbb{R}^n$ est égale à
    \[ \mathcal{A}(v,w) = \vert \det(v,w) \vert \]
  \end{theorem}

  \begin{corollary}
    Soient $v_1, \dots, v_n \in \mathbb{R}^n$. On note $\mathcal{V}(v_1, \dots, v_n)$ le volume du parallélépipède rectangle engendré par $v_1, \dots, v_n$ (ie. l'ensemble $\{ z \in \mathbb{R}^n \mid z = \sum_{i=1}^n \lambda_i v_i, \, \lambda_i \in [0,1] \}$). On a alors :
    \[ \mathcal{V}(v_1, \dots, v_n) = \vert \det(v_1, \dots, v_n) \vert \]
  \end{corollary}

  \paragraph{Suite de polygones}

  \reference[I-P]{389}
  \dev{suite-de-polygones}

  \begin{theorem}[Suite de polygones]
    Soit $P_0$ un polygone dont les sommets sont $\{ z_{0,1}, \dots, z_{0,n} \}$. On définit la suite de polygones $(P_k)$ par récurrence en disant que, pour tout $k \in \mathbb{N}^*$, les sommets de $P_{k+1}$ sont les milieux des arêtes de $P_k$.
    \newpar
    Alors la suite $(P_k)$ converge vers l'isobarycentre de $P_0$.
  \end{theorem}

  \subsubsection{En algèbre linéaire}

  \paragraph{Aux systèmes d'équations linéaires}

  \reference{143}

  On cherche à résoudre un système d'équations linéaires de la forme

  \[ AX = B \tag{S} \]

  avec $A = (a_{i,j})_{\substack{i \in \llbracket 1, p \rrbracket \\ j \in \llbracket 1, q \rrbracket}}$ et $B = (b_i)_{i \in \llbracket 1, p \rrbracket} \in \mathbb{K}^p$.

  \begin{theorem}[Formules de Cramer]
    On se place dans le cas $p = q = n$. Alors, $(S)$ admet une unique solution si et seulement si $\det(A) \neq 0$. Dans ce cas, elle est donnée par $X = (x_i)_{i \in \llbracket 1, n \rrbracket}$ où
    \[ \forall i \in \llbracket 1, n \rrbracket, \, x_i = \frac{\det(A_i)}{\det(A)} \]
    avec $A_i$ obtenue en remplaçant la $i$-ième colonne de $A$ par $B$.
  \end{theorem}

  \begin{lemma}
    Soit $r = \rang(A)$. Il existe un déterminant $\Delta$ d'ordre $r$ extrait de $A$.
  \end{lemma}

  \begin{definition}
    \begin{itemize}
      \item Le déterminant $\Delta$ précédent est le \textbf{déterminant principal} de $A$.
      \item Les équations (resp. inconnues) dont les indices sont deux des lignes (resp. colonnes) de $\Delta$ s'appellent les \textbf{équations principales} (resp. \textbf{inconnues principales}).
      \item Si $\Delta = \det(a_{i,j})_{\substack{i \in I \\ j \in J}}$, on appelle \textbf{déterminants caractéristiques} les déterminants d'ordre $r+1$ de la forme
      \begin{center}
        \begin{tabular}{|c|c|}
          $(a_{i,j})_{\substack{i \in I \\ j \in J}}$ & $(b_i)_{i \in I}$ \\
          \hline
          $(a_{k,j})_{j \in J}$ & $b_k$
        \end{tabular}
        avec $k \notin J$.
      \end{center}
    \end{itemize}
  \end{definition}

  \begin{theorem}[Rouché-Fontené]
    Le système $(S)$ admet des solutions si et seulement si $p = r$ ou les $p-r$ déterminants caractéristiques sont nuls. Le système est alors équivalent au système des équations principales. Les inconnues principales étant déterminées par un système de Cramer à l'aide des inconnues non principales.
  \end{theorem}

  \begin{example}
    Si,
    \[
    (S) \iff
    \begin{cases}
      x + 2y + z + t = 1 \\
      x - z - t = 1 \\
      -x + y + z + 2t = m
    \end{cases}
    \quad
    m \in \mathbb{R}
    \]
    on a $\rang(A) = 2$, $(S)$ admet des solutions si et seulement si $m=-1$, et
    \[
    (S)
    \iff \begin{cases}
      x + 2y = 1 + z - t \\
      x = 1 + z + t
    \end{cases}
    \iff \begin{cases}
      x = 1 + z + t \\
      y = -t
    \end{cases}
    \]
  \end{example}

  \paragraph{À la réduction des endomorphismes}

  \reference[GOU21]{171}
  \reference{186}

  \begin{definition}
    Soit $A \in \mathcal{M}_n(\mathbb{K})$. On appelle :
    \begin{itemize}
      \item \textbf{Polynôme caractéristique} de $A$ le polynôme $\chi_A = \det(A - XI_n)$.
      \item \textbf{Polynôme minimal} de $A$ l'unique polynôme unitaire $\pi_A$ qui engendre l'idéal $\mathrm{Ann}(A) = \{ Q \in \mathbb{K}[X] \mid Q(A) = 0 \}$.
    \end{itemize}
  \end{definition}

  \reference{172}

  \begin{proposition}
    \[ \lambda \text{ est valeur propre de } A \iff \chi_A(\lambda) = 0 \iff \pi_A(\lambda) = 0 \]
  \end{proposition}

  \reference{185}

  \begin{proposition}
    \begin{itemize}
      \item $A$ est trigonalisable si et seulement si $\chi_A$ est scindé sur $\mathbb{K}$.
      \item $A$ est diagonalisable si et seulement si $\pi_A$ est scindé à racines simples sur $\mathbb{K}$.
    \end{itemize}
  \end{proposition}

  \begin{corollary}
    Si $\mathbb{K} = \mathbb{F}_q$, $A$ est diagonalisable si et seulement si $A^q = A$.
  \end{corollary}

  \begin{theorem}[Cayley-Hamilton]
    \[ \pi_u \mid \chi_u \]
  \end{theorem}

  \paragraph{À l'étude du groupe linéaire}

  \reference[ROM21]{140}

  \begin{theorem}
    Soit $u \in \mathcal{L}(E)$. Les assertions suivantes sont équivalentes :
    \begin{enumerate}[label=(\roman*)]
      \item $u \in \mathrm{GL}(E)$.
      \item $\ker(u) = \{ 0 \}$.
      \item $\im(u) = E$.
      \item $\rang(u) = n$.
      \item $\det(u) = 0$.
      \item $u$ transforme toute base de $E$ en une base de $E$.
      \item Il existe $v \in \mathcal{L}(E)$ tel que $u \circ v = \operatorname{id}_E$.
      \item Il existe $w \in \mathcal{L}(E)$ tel que $w \circ u = \operatorname{id}_E$.
    \end{enumerate}
  \end{theorem}

  \reference[PER]{95}

  \begin{proposition}
    $\det : \mathrm{GL}(E) \rightarrow \mathbb{K}^*$ est un morphisme surjectif.
  \end{proposition}

  \reference[I-P]{203}

  Soit $p$ un nombre premier $\geq 3$. On se place sur le corps $\mathbb{K} = \mathbb{F}_p$.

  \begin{definition}
    Soit $H$ un hyperplan de $E$ et $G$ un supplémentaire de $H$. On définit $f$ la \textbf{dilatation} de base $H$, de direction $G$ et de rapport $\lambda \in \mathbb{K}^*$ par
    \[ \forall x \in H, \, \forall y \in G, \, f(x+y) = x + \lambda y \]
  \end{definition}

  \begin{theorem}
    Si $|\mathbb{K}| \geq 3$, les dilatations engendre $\mathrm{GL}(E)$.
  \end{theorem}

  \begin{notation}
    Soit $a \in \mathbb{F}_p$. On note $\left( \frac{a}{p} \right)$ le symbole de Legendre de $a$ modulo $p$.
  \end{notation}

  \begin{lemma}
    $a \mapsto \left( \frac{a}{p} \right)$ est un morphisme de groupes.
  \end{lemma}

  \begin{lemma}
    Il y a $\frac{p-1}{2}$ résidus quadratiques dans $\mathbb{F}_p^*$.
  \end{lemma}

  \begin{theorem}
    Le groupe multiplicatif d'un corps fini est cyclique.
  \end{theorem}

  \dev{theoreme-de-frobenius-zolotarev}

  \begin{theorem}[Frobenius-Zolotarev]
    \[ \forall u \in \mathrm{GL}(E), \, \epsilon(u) = \left( \frac{\det(u)}{p} \right) \]
    où $u$ est vu comme une permutation des éléments de $E$.
  \end{theorem}

  \annexessection

  \reference[I-P]{389}

  \begin{figure}[h]
    \begin{center}
      \begin{tikzpicture}
        \coordinate (A) at (0:3);
        \coordinate (B) at (72:3);
        \coordinate (C) at (2*72:3);
        \coordinate (D) at (3*72:3);
        \coordinate (E) at (4*72:3);
        \coordinate (F) at (A);
        \foreach \i in {0,...,10} {
          \draw(A) node {$\bullet$};
          \draw(B) node {$\bullet$};
          \draw(C) node {$\bullet$};
          \draw(D) node {$\bullet$};
          \draw(E) node {$\bullet$};
          \draw[fill=cyan!60, fill opacity=0.2](A) -- (B) -- (C) -- (D) -- (E) -- (A);
          \coordinate (A) at ($(A)!0.5!(B)$);
          \coordinate (B) at ($(B)!0.5!(C)$);
          \coordinate (C) at ($(C)!0.5!(D)$);
          \coordinate (D) at ($(D)!0.5!(E)$);
          \coordinate (E) at ($(E)!0.5!(F)$);
          \coordinate (F) at (A);
        }
      \end{tikzpicture}
    \end{center}
    \caption{La suite de polygones.}
  \end{figure}
  %</content>
\end{document}
