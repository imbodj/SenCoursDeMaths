\documentclass[12pt, a4paper]{report}

% LuaLaTeX :

\RequirePackage{iftex}
\RequireLuaTeX

% Packages :

\usepackage[french]{babel}
%\usepackage[utf8]{inputenc}
%\usepackage[T1]{fontenc}
\usepackage[pdfencoding=auto, pdfauthor={Hugo Delaunay}, pdfsubject={Mathématiques}, pdfcreator={agreg.skyost.eu}]{hyperref}
\usepackage{amsmath}
\usepackage{amsthm}
%\usepackage{amssymb}
\usepackage{stmaryrd}
\usepackage{tikz}
\usepackage{tkz-euclide}
\usepackage{fontspec}
\defaultfontfeatures[Erewhon]{FontFace = {bx}{n}{Erewhon-Bold.otf}}
\usepackage{fourier-otf}
\usepackage[nobottomtitles*]{titlesec}
\usepackage{fancyhdr}
\usepackage{listings}
\usepackage{catchfilebetweentags}
\usepackage[french, capitalise, noabbrev]{cleveref}
\usepackage[fit, breakall]{truncate}
\usepackage[top=2.5cm, right=2cm, bottom=2.5cm, left=2cm]{geometry}
\usepackage{enumitem}
\usepackage{tocloft}
\usepackage{microtype}
%\usepackage{mdframed}
%\usepackage{thmtools}
\usepackage{xcolor}
\usepackage{tabularx}
\usepackage{xltabular}
\usepackage{aligned-overset}
\usepackage[subpreambles=true]{standalone}
\usepackage{environ}
\usepackage[normalem]{ulem}
\usepackage{etoolbox}
\usepackage{setspace}
\usepackage[bibstyle=reading, citestyle=draft]{biblatex}
\usepackage{xpatch}
\usepackage[many, breakable]{tcolorbox}
\usepackage[backgroundcolor=white, bordercolor=white, textsize=scriptsize]{todonotes}
\usepackage{luacode}
\usepackage{float}
\usepackage{needspace}
\everymath{\displaystyle}

% Police :

\setmathfont{Erewhon Math}

% Tikz :

\usetikzlibrary{calc}
\usetikzlibrary{3d}

% Longueurs :

\setlength{\parindent}{0pt}
\setlength{\headheight}{15pt}
\setlength{\fboxsep}{0pt}
\titlespacing*{\chapter}{0pt}{-20pt}{10pt}
\setlength{\marginparwidth}{1.5cm}
\setstretch{1.1}

% Métadonnées :

\author{agreg.skyost.eu}
\date{\today}

% Titres :

\setcounter{secnumdepth}{3}

\renewcommand{\thechapter}{\Roman{chapter}}
\renewcommand{\thesubsection}{\Roman{subsection}}
\renewcommand{\thesubsubsection}{\arabic{subsubsection}}
\renewcommand{\theparagraph}{\alph{paragraph}}

\titleformat{\chapter}{\huge\bfseries}{\thechapter}{20pt}{\huge\bfseries}
\titleformat*{\section}{\LARGE\bfseries}
\titleformat{\subsection}{\Large\bfseries}{\thesubsection \, - \,}{0pt}{\Large\bfseries}
\titleformat{\subsubsection}{\large\bfseries}{\thesubsubsection. \,}{0pt}{\large\bfseries}
\titleformat{\paragraph}{\bfseries}{\theparagraph. \,}{0pt}{\bfseries}

\setcounter{secnumdepth}{4}

% Table des matières :

\renewcommand{\cftsecleader}{\cftdotfill{\cftdotsep}}
\addtolength{\cftsecnumwidth}{10pt}

% Redéfinition des commandes :

\renewcommand*\thesection{\arabic{section}}
\renewcommand{\ker}{\mathrm{Ker}}

% Nouvelles commandes :

\newcommand{\website}{https://github.com/imbodj/SenCoursDeMaths}

\newcommand{\tr}[1]{\mathstrut ^t #1}
\newcommand{\im}{\mathrm{Im}}
\newcommand{\rang}{\operatorname{rang}}
\newcommand{\trace}{\operatorname{trace}}
\newcommand{\id}{\operatorname{id}}
\newcommand{\stab}{\operatorname{Stab}}
\newcommand{\paren}[1]{\left(#1\right)}
\newcommand{\croch}[1]{\left[ #1 \right]}
\newcommand{\Grdcroch}[1]{\Bigl[ #1 \Bigr]}
\newcommand{\grdcroch}[1]{\bigl[ #1 \bigr]}
\newcommand{\abs}[1]{\left\lvert #1 \right\rvert}
\newcommand{\limi}[3]{\lim_{#1\to #2}#3}
\newcommand{\pinf}{+\infty}
\newcommand{\minf}{-\infty}
%%%%%%%%%%%%%% ENSEMBLES %%%%%%%%%%%%%%%%%
\newcommand{\ensemblenombre}[1]{\mathbb{#1}}
\newcommand{\Nn}{\ensemblenombre{N}}
\newcommand{\Zz}{\ensemblenombre{Z}}
\newcommand{\Qq}{\ensemblenombre{Q}}
\newcommand{\Qqp}{\Qq^+}
\newcommand{\Rr}{\ensemblenombre{R}}
\newcommand{\Cc}{\ensemblenombre{C}}
\newcommand{\Nne}{\Nn^*}
\newcommand{\Zze}{\Zz^*}
\newcommand{\Zzn}{\Zz^-}
\newcommand{\Qqe}{\Qq^*}
\newcommand{\Rre}{\Rr^*}
\newcommand{\Rrp}{\Rr_+}
\newcommand{\Rrm}{\Rr_-}
\newcommand{\Rrep}{\Rr_+^*}
\newcommand{\Rrem}{\Rr_-^*}
\newcommand{\Cce}{\Cc^*}
%%%%%%%%%%%%%%  INTERVALLES %%%%%%%%%%%%%%%%%
\newcommand{\intff}[2]{\left[#1\;,\; #2\right]  }
\newcommand{\intof}[2]{\left]#1 \;, \;#2\right]  }
\newcommand{\intfo}[2]{\left[#1 \;,\; #2\right[  }
\newcommand{\intoo}[2]{\left]#1 \;,\; #2\right[  }

\providecommand{\newpar}{\\[\medskipamount]}

\newcommand{\annexessection}{%
  \newpage%
  \subsection*{Annexes}%
}

\providecommand{\lesson}[3]{%
  \title{#3}%
  \hypersetup{pdftitle={#2 : #3}}%
  \setcounter{section}{\numexpr #2 - 1}%
  \section{#3}%
  \fancyhead[R]{\truncate{0.73\textwidth}{#2 : #3}}%
}

\providecommand{\development}[3]{%
  \title{#3}%
  \hypersetup{pdftitle={#3}}%
  \section*{#3}%
  \fancyhead[R]{\truncate{0.73\textwidth}{#3}}%
}

\providecommand{\sheet}[3]{\development{#1}{#2}{#3}}

\providecommand{\ranking}[1]{%
  \title{Terminale #1}%
  \hypersetup{pdftitle={Terminale #1}}%
  \section*{Terminale #1}%
  \fancyhead[R]{\truncate{0.73\textwidth}{Terminale #1}}%
}

\providecommand{\summary}[1]{%
  \textit{#1}%
  \par%
  \medskip%
}

\tikzset{notestyleraw/.append style={inner sep=0pt, rounded corners=0pt, align=center}}

%\newcommand{\booklink}[1]{\website/bibliographie\##1}
\newcounter{reference}
\newcommand{\previousreference}{}
\providecommand{\reference}[2][]{%
  \needspace{20pt}%
  \notblank{#1}{
    \needspace{20pt}%
    \renewcommand{\previousreference}{#1}%
    \stepcounter{reference}%
    \label{reference-\previousreference-\thereference}%
  }{}%
  \todo[noline]{%
    \protect\vspace{20pt}%
    \protect\par%
    \protect\notblank{#1}{\cite{[\previousreference]}\\}{}%
    \protect\hyperref[reference-\previousreference-\thereference]{p. #2}%
  }%
}

\definecolor{devcolor}{HTML}{00695c}
\providecommand{\dev}[1]{%
  \reversemarginpar%
  \todo[noline]{
    \protect\vspace{20pt}%
    \protect\par%
    \bfseries\color{devcolor}\href{\website/developpements/#1}{[DEV]}
  }%
  \normalmarginpar%
}

% En-têtes :

\pagestyle{fancy}
\fancyhead[L]{\truncate{0.23\textwidth}{\thepage}}
\fancyfoot[C]{\scriptsize \href{\website}{\texttt{https://github.com/imbodj/SenCoursDeMaths}}}

% Couleurs :

\definecolor{property}{HTML}{ffeb3b}
\definecolor{proposition}{HTML}{ffc107}
\definecolor{lemma}{HTML}{ff9800}
\definecolor{theorem}{HTML}{f44336}
\definecolor{corollary}{HTML}{e91e63}
\definecolor{definition}{HTML}{673ab7}
\definecolor{notation}{HTML}{9c27b0}
\definecolor{example}{HTML}{00bcd4}
\definecolor{cexample}{HTML}{795548}
\definecolor{application}{HTML}{009688}
\definecolor{remark}{HTML}{3f51b5}
\definecolor{algorithm}{HTML}{607d8b}
%\definecolor{proof}{HTML}{e1f5fe}
\definecolor{exercice}{HTML}{e1f5fe}

% Théorèmes :

\theoremstyle{definition}
\newtheorem{theorem}{Théorème}

\newtheorem{property}[theorem]{Propriété}
\newtheorem{proposition}[theorem]{Proposition}
\newtheorem{lemma}[theorem]{Activité d'introduction}
\newtheorem{corollary}[theorem]{Conséquence}

\newtheorem{definition}[theorem]{Définition}
\newtheorem{notation}[theorem]{Notation}

\newtheorem{example}[theorem]{Exemple}
\newtheorem{cexample}[theorem]{Contre-exemple}
\newtheorem{application}[theorem]{Application}

\newtheorem{algorithm}[theorem]{Algorithme}
\newtheorem{exercice}[theorem]{Exercice}

\theoremstyle{remark}
\newtheorem{remark}[theorem]{Remarque}

\counterwithin*{theorem}{section}

\newcommand{\applystyletotheorem}[1]{
  \tcolorboxenvironment{#1}{
    enhanced,
    breakable,
    colback=#1!8!white,
    %right=0pt,
    %top=8pt,
    %bottom=8pt,
    boxrule=0pt,
    frame hidden,
    sharp corners,
    enhanced,borderline west={4pt}{0pt}{#1},
    %interior hidden,
    sharp corners,
    after=\par,
  }
}

\applystyletotheorem{property}
\applystyletotheorem{proposition}
\applystyletotheorem{lemma}
\applystyletotheorem{theorem}
\applystyletotheorem{corollary}
\applystyletotheorem{definition}
\applystyletotheorem{notation}
\applystyletotheorem{example}
\applystyletotheorem{cexample}
\applystyletotheorem{application}
\applystyletotheorem{remark}
%\applystyletotheorem{proof}
\applystyletotheorem{algorithm}
\applystyletotheorem{exercice}

% Environnements :

\NewEnviron{whitetabularx}[1]{%
  \renewcommand{\arraystretch}{2.5}
  \colorbox{white}{%
    \begin{tabularx}{\textwidth}{#1}%
      \BODY%
    \end{tabularx}%
  }%
}

% Maths :

\DeclareFontEncoding{FMS}{}{}
\DeclareFontSubstitution{FMS}{futm}{m}{n}
\DeclareFontEncoding{FMX}{}{}
\DeclareFontSubstitution{FMX}{futm}{m}{n}
\DeclareSymbolFont{fouriersymbols}{FMS}{futm}{m}{n}
\DeclareSymbolFont{fourierlargesymbols}{FMX}{futm}{m}{n}
\DeclareMathDelimiter{\VERT}{\mathord}{fouriersymbols}{152}{fourierlargesymbols}{147}

% Code :

\definecolor{greencode}{rgb}{0,0.6,0}
\definecolor{graycode}{rgb}{0.5,0.5,0.5}
\definecolor{mauvecode}{rgb}{0.58,0,0.82}
\definecolor{bluecode}{HTML}{1976d2}
\lstset{
  basicstyle=\footnotesize\ttfamily,
  breakatwhitespace=false,
  breaklines=true,
  %captionpos=b,
  commentstyle=\color{greencode},
  deletekeywords={...},
  escapeinside={\%*}{*)},
  extendedchars=true,
  frame=none,
  keepspaces=true,
  keywordstyle=\color{bluecode},
  language=Python,
  otherkeywords={*,...},
  numbers=left,
  numbersep=5pt,
  numberstyle=\tiny\color{graycode},
  rulecolor=\color{black},
  showspaces=false,
  showstringspaces=false,
  showtabs=false,
  stepnumber=2,
  stringstyle=\color{mauvecode},
  tabsize=2,
  %texcl=true,
  xleftmargin=10pt,
  %title=\lstname
}

\newcommand{\codedirectory}{}
\newcommand{\inputalgorithm}[1]{%
  \begin{algorithm}%
    \strut%
    \lstinputlisting{\codedirectory#1}%
  \end{algorithm}%
}




\begin{document}
  %<*content>
  \lesson{algebra}{104}{Groupes finis. Exemples et applications.}

  \subsection{Outils d'étude de groupes finis}

  Soit $G$ un groupe.

  \subsubsection{Ordre d'un groupe, ordre d'un élément}

  \reference[ULM21]{1}

  \begin{definition}
    \textbf{L'ordre} du groupe $G$, noté $|G|$ est le cardinal de l'ensemble sous-jacent $G$. Si $G$ est fini de cardinal $n$, on dit que $G$ est \textbf{d'ordre $n$}. Sinon, on dit que $G$ est \textbf{d'ordre infini}.
  \end{definition}

  \begin{example}
    Les multiplicatifs des corps $\mathbb{Q}$, $\mathbb{R}$ et $\mathbb{C}$ sont d'ordre infini.
  \end{example}

  \reference{6}

  \begin{definition}
    On appelle \textbf{ordre} d'un élément $g \in G$, l'ordre du sous-groupe $\langle g \rangle$ qu'il engendre.
  \end{definition}

  \begin{example}
    L'élément $i$ est d'ordre $4$ dans $\mathbb{C}^*$.
  \end{example}

  \begin{proposition}
    Soit $g \in G$ d'ordre $n$. Alors,
    \begin{enumerate}[label=(\roman*)]
      \item $n$ est le plus petit entier strictement positif ayant la propriété $g^n = e_G$.
      \item $\langle g \rangle = \{ e_G, g, \dots, g^{n-1} \}$.
      \item Pour $k \in \mathbb{Z}$, $g^k = e_G$ si et seulement si $n \mid k$.
    \end{enumerate}
  \end{proposition}

  \begin{example}
    Pour $n \in \mathbb{Z}$, on a $\langle n \rangle = \{ nk \mid k \in \mathbb{Z} \}$ et on note ce groupe $n\mathbb{Z}$.
  \end{example}

  \reference{18}

  \begin{theorem}
    \label{104-1}
    Soit $g \in G$. Alors,
    \begin{enumerate}[label=(\roman*)]
      \item $g$ est d'ordre infini si et seulement si $\langle g \rangle$ est isomorphe à $(\mathbb{Z}, +)$. Dans ce cas $g^i \neq g^j$ dès que $i \neq j$ et $\langle g \rangle = \{ \dots, g^{-1}, e_G, g, \dots \}$.
      \item \label{104-2} $g$ est d'ordre fini si et seulement si $g, \dots, g^{n-1}$ sont tous distincts et si $g^n = e_G$.
    \end{enumerate}
  \end{theorem}

  \reference{25}

  \begin{theorem}[Lagrange]
    On suppose $G$ fini. Soit $H < G$. Alors,
    \[ |H| \mid |G| \]
    En particulier, l'ordre d'un élément de $G$ divise toujours l'ordre de $G$.
  \end{theorem}

  \subsubsection{Groupes cycliques}

  \reference{6}

  \begin{definition}
    On dit que $G$ est \textbf{cyclique} s'il est engendré par un seul élément.
  \end{definition}

  \reference{26}

  \begin{proposition}
    Un groupe fini d'ordre premier est cyclique.
  \end{proposition}

  \begin{theorem}
    On suppose $G$ fini d'ordre $n$. Alors,
    \begin{enumerate}[label=(\roman*)]
      \item Si $G$ est abélien et s'il existe au plus un sous-groupe d'ordre $d$ pour tout diviseur $d$ de $n$, alors $G$ est cyclique.
      \item Si $G$ est cyclique, tous ses sous-groupes le sont aussi.
      \item $G$ est cyclique si et seulement si pour tout diviseur $d$ de $n$, $G$ admet exactement un sous-groupe d'ordre $d$.
    \end{enumerate}
  \end{theorem}

  \reference[ROM21]{25}

  \begin{theorem}
    Tout sous-groupe fini du groupe multiplicatif d'un corps commutatif est cyclique.
  \end{theorem}

  \begin{corollary}
    L'ensemble des racines $n$-ièmes de l'unité d'un corps est un sous-groupe cyclique de son groupe multiplicatif.
  \end{corollary}

  \subsubsection{Actions de groupes}

  \reference[ULM21]{29}

  Soit $X$ un ensemble.

  \begin{definition}
    On appelle \textbf{action} (à gauche) de $G$ sur $X$ toute application
    \[
    \begin{array}{ccc}
      G \times X &\rightarrow& X \\
      (g, x) &\mapsto& g \cdot x
    \end{array}
    \]
    satisfaisant les conditions suivantes :
    \begin{enumerate}[label=(\roman*)]
      \item $\forall g, h \in G$, $\forall x \in X$, $g \cdot (h \cdot x) = (gh) \cdot x$.
      \item $\forall x \in X$, $e_G \cdot x = x$.
    \end{enumerate}
  \end{definition}

  \begin{remark}
    On peut de même définir une action à droite de $G$ sur $X$.
  \end{remark}

  \begin{definition}
    On définit pour tout $x \in X$ :
    \begin{itemize}
      \item $G \cdot x = \{ g \cdot x \mid g \in G \} \subseteq X$ l'\textbf{orbite de $x$}.
      \item $\stab_G(x) = \{ g \in G \mid g \cdot x = x \} < G$ le \textbf{stabilisateur de $x$}.
    \end{itemize}
  \end{definition}

  On suppose ici que $G$ et $X$ sont finis.

  \reference{71}

  \begin{proposition}
    Soit $x \in X$. Alors :
    \begin{itemize}
      \item $|G \cdot x| = (G : \stab_G(x))$.
      \item $|G| = |\stab_G(x)| |G \cdot x|$.
      \item $|G \cdot x| = \frac{|G|}{|\stab_G(x)|}$
    \end{itemize}
  \end{proposition}

  \begin{theorem}[Formule des classes]
    Soit $\Omega$ un système de représentants des orbites de l'action de $G$ sur $X$. Alors,
    \[ |X| = \sum_{\omega \in \Omega} |G \cdot \omega| = \sum_{\omega \in \Omega} (G : \stab_G(\omega)) = \sum_{\omega \in \Omega} \frac{|G|}{|\stab_G(\omega)|} \]
  \end{theorem}

  \begin{definition}
    On définit :
    \begin{itemize}
      \item $X^G = \{ x \in X \mid \forall g \in G, \, g \cdot x = x \}$ l'ensemble des points de $X$ laissés fixes par tous les éléments de $G$.
      \item $X^g = \{ x \in X \mid g \cdot x = x \}$ l'ensemble des points de $X$ laissés fixes par $g \in G$.
    \end{itemize}
  \end{definition}

  \begin{corollary}[Formule de Burnside]
    Le nombre $r$ d'orbites de $X$ sous l'action de $G$ est donné par
    \[ r = \frac{1}{|G|} \sum_{g \in G} |X^g| \]
  \end{corollary}

  \begin{corollary}
    Soit $p$ un nombre premier. Si $G$ est un $p$-groupe (ie. l'ordre de $G$ est une puissance de $p$), alors,
    \[ |X^G| \equiv |X| \mod p \]
    où $X^G$ désigne l'ensemble des points fixes de $X$ sous l'action de $G$.
  \end{corollary}

  \begin{corollary}
    Soit $p$ un nombre premier. Le centre d'un $p$-groupe non trivial est non trivial.
  \end{corollary}

  \begin{corollary}
    Soit $p$ un nombre premier. Un groupe d'ordre $p^2$ est toujours abélien.
  \end{corollary}

  \begin{application}[Théorème de Cauchy]
    On suppose $G$ non trivial et fini. Soit $p$ un premier divisant l'ordre de $G$. Alors il existe un élément d'ordre $p$ dans $G$.
  \end{application}

  \reference[GOU21]{44}
  \dev{theoreme-de-sylow}

  \begin{application}[Premier théorème de Sylow]
    On suppose $G$ fini d'ordre $n p^\alpha$ avec $n, \alpha \in \mathbb{N}$ et $p$ premier tel que $p \nmid n$. Alors, il existe un sous-groupe de $G$ d’ordre $p^\alpha$.
  \end{application}

  \subsection{Groupes abéliens finis}

  \subsubsection{Un exemple fondamental : \texorpdfstring{$\mathbb{Z}/n\mathbb{Z}$}{Z/nZ}}

  \reference[ULM21]{45}

  \begin{proposition}
    $n\mathbb{Z}$ est un sous-groupe distingué de $(\mathbb{Z}, +)$, si bien que l'on peut définir le quotient $\mathbb{Z}/n\mathbb{Z}$.
  \end{proposition}

  \begin{proposition}
    $\mathbb{Z}/n\mathbb{Z}$ est cyclique d'ordre $n$.
  \end{proposition}

  \begin{proposition}
    On peut définir une structure d'anneau sur $\mathbb{Z}/n\mathbb{Z}$. Le groupe multiplicatif de cet anneau est alors d'ordre $\varphi(n)$.
  \end{proposition}

  \begin{corollary}
    $\mathbb{Z}/p\mathbb{Z}$ est un corps si et seulement si $p$ est premier.
  \end{corollary}

  \reference[ROM21]{14}

  \begin{proposition}
    Dans le cas du \cref{104-1} \cref{104-2}, $\langle g \rangle$ est alors isomorphe à $\mathbb{Z}/n\mathbb{Z}$.
  \end{proposition}

  \begin{example}
    \[ \mu_n \cong \mathbb{Z}/n\mathbb{Z} \]
    où $\mu_n$ désigne le groupe cyclique des racines de l'unité de $\mathbb{C}^*$.
  \end{example}

  \subsubsection{Décomposition cyclique}

  \reference[ULM21]{81}

  \begin{theorem}[Chinois]
    Soient $n$ et $m$ deux entiers premiers entre eux. Alors,
    \[ \mathbb{Z}/nm\mathbb{Z} \equiv \mathbb{Z}/n\mathbb{Z} \times \mathbb{Z}/m\mathbb{Z} \]
  \end{theorem}

  \reference{112}

  \begin{theorem}[Kronecker]
    Soit $G$ un groupe abélien d'ordre $n \geq 2$. Il existe une suite d'entiers $n_1 \geq 2$, $n_2$ multiple de $n_1$, \dots, $n_k$ multiple de $n_{k-1}$ telle que $G$ est isomorphe au groupe produit
    \[ \prod_{i=1}^k \mathbb{Z}/n_i\mathbb{Z} \]
  \end{theorem}

  \begin{example}
    Soit $G = \mathbb{Z}/5\mathbb{Z} \times \mathbb{Z}/5\mathbb{Z} \times \mathbb{Z}/90\mathbb{Z}$. Alors,
    \begin{align*}
      G &\cong \mathbb{Z}/5\mathbb{Z} \times \mathbb{Z}/5\mathbb{Z} \times (\mathbb{Z}/2\mathbb{Z} \times \mathbb{Z}/3^2\mathbb{Z} \times \mathbb{Z}/5\mathbb{Z}) \\
      &\cong \mathbb{Z}/2\mathbb{Z} \times \mathbb{Z}/3^2\mathbb{Z} \times (\mathbb{Z}/5\mathbb{Z} \times \mathbb{Z}/5\mathbb{Z} \times \mathbb{Z}/5\mathbb{Z})
    \end{align*}
  \end{example}

  \subsection{Groupes non abéliens finis}

  Les groupes qui suivent sont, sauf cas particuliers, des groupes non abéliens.

  \subsubsection{Groupes symétrique et alterné}

  \reference{55}

  \begin{definition}
    L'ensemble des permutations de $\llbracket 1, n \rrbracket$ est un groupe pour la composition des applications : c'est le \textbf{groupe symétrique}, noté $S_n$.
  \end{definition}

  \begin{remark}
    $S_n$ est fini, d'ordre $n!$.
  \end{remark}

  \begin{theorem}[Cayley]
    Tout groupe fini d'ordre $n$ est isomorphe à un sous-groupe de $S_n$.
  \end{theorem}

  \begin{definition}
    Soient $l \in \mathbb{N}^*$ et $i_1, \dots, i_l \in \llbracket 1, n \rrbracket$ des éléments distincts. La permutation $\gamma \in S_n$ définie par
    \[
      \gamma(j) =
      \begin{cases}
        j &\text{si } j \notin \{ i_1, \dots, i_l \} \\
        i_{k+1} &\text{si } j = i_k \text{ avec } k<l \\
        i_1 &\text{si } j=i_l
      \end{cases}
    \]
    et notée $\begin{pmatrix} i_1 & \dots & i_l \end{pmatrix}$ est appelée \textbf{cycle} de longueur $l$ et de \textbf{support} $\{ i_1, \dots, i_l \}$. Un cycle de longueur $2$ est une \textbf{transposition}.
  \end{definition}

  \begin{example}
    $\gamma = \begin{pmatrix} 1 & 4 & 2 & 5 \end{pmatrix} = \begin{pmatrix} 4 & 2 & 5 & 1 \end{pmatrix} = \begin{pmatrix} 2 & 5 & 1 & 4 \end{pmatrix} = \begin{pmatrix} 5 & 1 & 4 & 2 \end{pmatrix}$ est un cycle de $S_5$ de longueur $4$.
  \end{example}

  \begin{theorem}
    Toute permutation de $S_n$ s'écrit de manière unique (à l'ordre près) comme produit de cycles dont les supports sont deux à deux disjoints.
  \end{theorem}

  \begin{example}
    \label{104-3}
    \[
      \begin{pmatrix}
        1 & 2 & 3 & 4 & 5 & 6 \\
        2 & 4 & 5 & 1 & 3 & 6
      \end{pmatrix}
      =
      \begin{pmatrix} 1 & 2 & 4 \end{pmatrix}\begin{pmatrix} 3 & 5 \end{pmatrix}
    \]
  \end{example}

  \begin{definition}
    On appelle \textbf{type} d'une permutation $\sigma \in S_n$ et on note $[l_1, \dots, l_m]$ la liste des cardinaux $l_i$ des orbites dans $\llbracket 1, n \rrbracket$ de l'action du groupe $\langle \sigma \rangle$ sur $\llbracket 1, n \rrbracket$, rangée dans l'ordre croissant.
  \end{definition}

  \begin{proposition}
    Une permutation de type $[l_1, \dots, l_m]$ a pour ordre $\operatorname{ppcm}(l_1, \dots, l_m)$.
  \end{proposition}

  \begin{example}
    La permutation de l'\cref{104-3} est d'ordre $6$.
  \end{example}

  \begin{definition}
    \begin{itemize}
      \item Soit $\sigma \in S_n$. On appelle \textbf{signature} de $\sigma$, notée $\epsilon(\sigma)$ l'entier $\epsilon(\sigma) = \prod_{i \neq j} \frac{\sigma(i) - \sigma(j)}{i-j}$.
      \item $\sigma \mapsto \epsilon(\sigma)$ est un morphisme de $S_n$ dans $\{ \pm 1 \}$, on note $A_n$ son noyau.
    \end{itemize}
  \end{definition}

  \reference[PER]{15}

  \begin{lemma}
    Les $3$-cycles sont conjugués dans $A_n$ pour $n \geq 5$.
  \end{lemma}

  \reference[ROM21]{49}

  \begin{lemma}
    Le produit de deux transpositions est un produit de $3$-cycles.
  \end{lemma}

  \begin{proposition}
    $A_n$ est engendré par les $3$-cycles pour $n \geq 3$.
  \end{proposition}

  \reference[PER]{28}
  \dev{simplicite-du-groupe-alterne}

  \begin{theorem}
    $A_n$ est simple pour $n \geq 5$.
  \end{theorem}

  \subsubsection{Groupe linéaire sur un corps fini}

  \reference[ULM21]{119}

  Soit $V$ un espace vectoriel de dimension finie $n$ sur un corps $\mathbb{K}$.

  \begin{definition}
    \begin{itemize}
      \item Le \textbf{groupe linéaire} de $V$, $\mathrm{GL}(V)$ est le groupe des applications linéaires de $V$ dans lui-même qui sont inversibles.
      \item Le \textbf{groupe spécial linéaire} de $V$, $\mathrm{SL}(V)$ est le sous-groupe de $\mathrm{GL}(V)$ constitué des applications de déterminant $1$.
      \item Les quotients de ces groupes par leur centre sont respectivement notés $\mathrm{PGL}(V)$ et $\mathrm{PSL}(V)$.
    \end{itemize}
  \end{definition}

  \reference{124}

  \begin{proposition}
    On se place dans le cas où $\mathbb{K} = \mathbb{F}_q$. Alors, les groupes précédents sont finis, et :
    \begin{enumerate}[label=(\roman*)]
      \item $|\mathrm{GL}(V)| = q^{\frac{n(n-1)}{2}}((q^n-1) \dots (q-1))$.
      \item $|\mathrm{PGL}(V)| = |\mathrm{SL}(V)| = \frac{|\mathrm{GL}(V)|}{q-1}$.
      \item $|\mathrm{PSL}(V)| = |\mathrm{SL}(V)| = \frac{|\mathrm{GL}(V)|}{(q-1)\operatorname{pgcd}(n,q-1)}$.
    \end{enumerate}
  \end{proposition}

  \subsubsection{Groupe diédral}

  \reference{8}

  \begin{definition}
    Pour un entier $n \geq 1$, le \textbf{groupe diédral} $D_n$ est le sous-groupe, de $\mathrm{GL}_2(\mathbb{R})$ engendré par la symétrie axiale $s$ et la rotation d'angle $\theta = \frac{2\pi}{n}$ définies respectivement par les matrices
    \[
    S =
    \begin{pmatrix}
      1 & 0 \\
      0 & -1
    \end{pmatrix}
    \text{ et }
    R =
    \begin{pmatrix}
      \cos(\theta) & -\sin(\theta) \\
      \sin(\theta) & \cos(\theta)
    \end{pmatrix}
    \]
  \end{definition}

  \begin{example}
    $D_1 = \{ \operatorname{id}, s \}$.
  \end{example}

  \begin{proposition}
    \begin{enumerate}[label=(\roman*)]
      \item $D_n$ est un groupe d'ordre $2n$.
      \item $r^n = s^2 = \operatorname{id}$ et $sr = r^{-1}s$.
    \end{enumerate}
  \end{proposition}

  \reference{28}

  \begin{proposition}
    Un groupe non cyclique d'ordre $4$ est isomorphe à $D_2$.
  \end{proposition}

  \reference{65}

  \begin{example}
    $S_2$ est isomorphe à $D_2$.
  \end{example}

  \reference{28}

  \begin{proposition}
    Un groupe fini d'ordre $2p$ avec $p$ premier est soit cyclique, soit isomorphe à $D_p$.
  \end{proposition}

  \begin{example}
    $S_3$ est isomorphe à $D_3$.
  \end{example}

  \reference{47}

  \begin{proposition}
    Les sous-groupes de $D_n$ sont soit cyclique, soit isomorphes à un $D_m$ où $m \mid n$.
  \end{proposition}

  \subsection{Représentations linéaires de groupes finis}

  \reference{144}

  Dans cette partie, $G$ désigne un groupe d'ordre fini.

  \begin{definition}
    \begin{itemize}
      \item Une \textbf{représentation linéaire} $\rho$ est un morphisme de $G$ dans $\mathrm{GL}(V)$ où $V$ désigne un espace-vectoriel de dimension finie $n$ sur $\mathbb{C}$.
      \item On dit que $n$ est le \textbf{degré} de $\rho$.
      \item On dit que $\rho$ est \textbf{irréductible} si $V \neq \{ 0 \}$ et si aucun sous-espace vectoriel de $V$ n'est stable par $\rho(g)$ pour tout $g \in G$, hormis $\{ 0 \}$ et $V$.
    \end{itemize}
  \end{definition}

  \begin{example}
    Soit $\varphi : G \rightarrow S_n$ le morphisme structurel d'une action de $G$ sur un ensemble de cardinal $n$. On obtient une représentation de $G$ sur $\mathbb{C}^n = \{ e_1, \dots, e_n \}$ en posant
    \[ \rho(g)(e_i) = e_{\varphi(g)(i)} \]
    c'est la représentation par permutations de $G$ associé à l'action. Elle est de degré $n$.
  \end{example}

  \begin{definition}
    La représentation par permutations de $G$ associée à l'action par translation à gauche de $G$ sur lui-même est la \textbf{représentation régulière} de $G$, on la note $\rho_G$.
  \end{definition}

  \reference{150}

  \begin{definition}
    On peut associer à toute représentation linéaire $\rho$, son \textbf{caractère} $\chi = \operatorname{trace} \circ \rho$. On dit que $\chi$ est \textbf{irréductible} si $\rho$ est irréductible.
  \end{definition}

  \begin{proposition}
    \begin{enumerate}[label=(\roman*)]
      \item Les caractères sont des fonctions constantes sur les classes de conjugaison.
      \item Il y a autant de caractères irréductibles que de classes de conjugaisons.
    \end{enumerate}
  \end{proposition}

  \begin{definition}
    Soit $\rho : G \rightarrow \mathrm{GL}(V)$ une représentation linéaire de $G$. On suppose $V = W \oplus W_0$ avec $W$ et $W_0$ stables par $\rho(g)$ pour tout $g \in G$. On dit alors que $\rho$ est \textbf{somme directe} de $\rho_W$ et de $\rho_{W_0}$.
  \end{definition}

  \begin{theorem}[Maschke]
    Toute représentation linéaire de $G$ est somme directe de représentations irréductibles.
  \end{theorem}

  \reference[PEY]{231}

  \begin{theorem}
    Les sous-groupes distingués de $G$ sont exactement les
    \[ \bigcap_{i \in I} \ker(\rho_i) \text{ où } I \in \mathcal{P}(\llbracket 1, r \rrbracket) \]
  \end{theorem}

  \begin{corollary}
    $G$ est simple si et seulement si $\forall i \neq 1$, $\forall g \neq e_G$, $\chi_i(g) \neq \chi_i(e_G)$.
  \end{corollary}
  %</content>
\end{document}
