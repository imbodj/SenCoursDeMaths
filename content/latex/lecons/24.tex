\input{../common}

\begin{document}
  %<*content>
  \lesson{analysis}{24}{Étude de fonctions (TL)}
  
 \subsection{Parité et éléments de symétrie}
Soit  $ f $    une fonction  de $\Rr $ vers  $ \Rr$,  $\mathscr{D}_{f}$ est son ensemble de définition et $\mathscr{C}_{f}$ sa courbe représentative dans   un repère $ \oij. $
\begin{definition}
\begin{itemize}
\item  $ f $   est \textbf{paire} si  et  seulement si pour tout $ x\in \mathscr{D}_{f}$, $ -x\in \mathscr{D}_{f}$ et $ f(-x)=f(x). $
\item $ f $   est \textbf{impaire} si  et  seulement si pour tout $ x\in \mathscr{D}_{f}$, $ -x\in \mathscr{D}_{f}$ et $ f(-x)= -f(x). $
\end{itemize}

\end{definition}
\begin{remark}
Pour une fonction paire ou impaire, $  \mathscr{D}_{f}$  est symétrique par rapport à    $ 0. $
\end{remark}

\begin{example}
\begin{itemize}
\item[$  \bullet$]  La fonction $ x\longmapsto x(x^{2}-1) $  définie sur $ \Rr $  est   impaire.
\item[$  \bullet$]  La fonction $ x\longmapsto \dfrac{-x^{2}}{x^{2}-1} $  définie sur $ \Rr\setminus \accol{-1, 1} $  est   paire.

\item[$  \bullet$]  La fonction $ x\longmapsto \dfrac{x^{2}}{x-1} $  définie sur $ \Rr\setminus \accol{ 1} $   n'est  ni paire ni impaire.
\end{itemize}
\end{example}
\begin{definition}
\begin{itemize}
\item  Le point I$ (a,b) $   est centre de symétrie de $  \mathscr{C}_{f}$ si  et  seulement si\\ pour tout $ x\in \mathscr{D}_{f}$, $ 2a-x\in \mathscr{D}_{f}$ et $ f(2a-x)+f(x)=2b. $
\item La droite $ (D)\;: x=a $  est  axe de symétrie  de               $  \mathscr{C}_{f}$ si  et  seulement si\\ pour tout $ x\in \mathscr{D}_{f}$, $2a -x\in \mathscr{D}_{f}$ et $ f(2a-x)= f(x). $
\end{itemize}

\end{definition}

\begin{remark}

$ \bullet $  Dans ce cas on peut restreindre l'étude de $ f $  à $ \mathscr{D}_{f}\cap \intfo{a}{\pinf} $.

$ \bullet $ Si $ f $  est paire (respectivement impaire), l' axe des ordonnées du repère est  axe de symétrie (respectivement l'origine du repère est centre de symétrie ) de $  \mathscr{C}_{f}$.
\end{remark}
\begin{example}

 La  courbe  $  \mathscr{C}_{f}$ de la  fonction $ f : x\longmapsto x^{2}+x-1 $  définie sur $ \Rr $  admet comme axe de symétrie la droite d'équation $ x=-\frac{1}{2} $.
 
 En effet pour tout $ x\in\Rr $, on a $ -1-x\in\Rr $  et $ f(-1-x)= (-1-x)^{2}+(-1-x)-1=f(x) $


\end{example}

\begin{example} 

 La  courbe  $  \mathscr{C}_{f}$ de la  fonction $ f :x\longmapsto \dfrac{4x+1}{x-1} $ \;  définie sur $ \Rr\setminus \accol{1} $  admet comme centre de symétrie le point  I $ (1,4) $.\\
 
 En effet pour tout $ x\in\Rr\setminus \accol{1} $, on a $ 2-x\in\Rr\setminus \accol{1} $ \\ Et $ f(2-x)+f(x)= \dfrac{4(2-x)+1}{2-x-1}+\dfrac{4x+1}{x-1}=\dfrac{8-4x+1}{1-x}+\dfrac{4x+1}{x-1}=\dfrac{4x-9}{x-1}+\dfrac{4x+1}{x-1}=\dfrac{8(x-1)}{x-1}=8$
\end{example}



\subsection{Branches infinies  à une courbe}
Nous distinguons deux sortes de branches infinies:\; les  \textbf{asymptotes}  et les \textbf{  branches paraboliques}.\\
\textbf{Asymptote parallèle à l'axe des abscisses}
\begin{definition}
 Soit  $L$ un nombre réel.\\
   $ \bullet $ La droite D  d'équation : $ y=L $   est une  asymptote   à la courbe  $ \mathscr{C}_{f}$  en $ \pinf $  si et seulement si
$ \displaystyle\lim_{ x \to \pinf}f(x)=L$.    


$ \bullet $  La droite  $D$  d'équation : $ y=L $   est une  asymptote    à la courbe  $ \mathscr{C}_{f}$  en $ \minf $   si et seulement si
 $ \displaystyle\lim_{ x \to \minf}f(x)=L$. 
  \end{definition}
 \begin{center}
 \begin{tikzpicture}[>=stealth', scale=0.5]
\draw[->,thick] (-7,0) -- (7,0);
\draw[->,thick] (0,-2) -- (0,3);
\foreach\x in {}
{
\draw[thick] (\x,0.1) -- (\x,-0.1) node[below] {\x};
}
\node (L) at (-0.1,2.4) {L};
\foreach\y in {}
{
\draw[thick] (0.1,\y) -- (-0.1,\y) node[left] {\y};
}
\draw[thick,blue!50!black] plot[domain=-7:7,samples=100] (\x,{(2*\x*\x+\x-1)/(\x*\x+\x+1)}) node[below ] {$\mathscr{C}$};
\draw[very thick, red] (7,2) -- (-7,2) node[below right] {D};
\node () at (-0.4, -0.4) {O};
\end{tikzpicture}
\end{center}

 
 \textbf{Asymptote parallèle à l'axe des ordonnées}
\begin{definition}
 La droite $D$  d'équation : $ x=a $   est une  asymptote   à la courbe  $ \mathscr{C}_{f}$  si et seulement si\\ 
$ \displaystyle\lim_{ x \to a^{+}}f(x)=\pinf $\; ou \;$ \displaystyle\lim_{ x \to a^{+}}f(x)=\minf $
\;ou \;$ \displaystyle\lim_{ x \to a^{-}}f(x)=\pinf $\; ou\; $ \displaystyle\lim_{ x \to a^{-}}f(x)=\minf $ 
\end{definition}

\begin{tikzpicture}[>=stealth', scale=0.5]
\clip (-1,-5) rectangle (6,3);
\draw[->,thick] (-1,0) -- (3,0);
\draw[->,thick] (0,-3) -- (0,3);
\foreach\x in {}
{
\draw[thick] (\x,0.1) -- (\x,-0.1) node[below] {\x};
}
\foreach\y in {}
{
\draw[thick] (0.1,\y) -- (-0.1,\y) node[left] {\y};
}
\draw[thick,blue!50!black] plot[domain=0.26:2,samples=100] (\x,{(2*\x+1)/(4*\x-1)}) node[above left] {$\mathscr{C}$};
\draw[thick,red] (0.25,3) -- (0.25,-3) node[above right] {D};
\node () at (0.4,-0.3) {$a$};
\node () at (-0.4, -0.4) {O};
\node () at (2, -3.6) {$ \displaystyle\lim_{ x \to a^{+}}f(x)=\pinf $};
\end{tikzpicture}


\begin{tikzpicture}[>=stealth', scale=0.5]
\clip (-1,-5) rectangle (6,4);
\draw[->,thick] (-1,0) -- (3,0);
\draw[->,thick] (0,-4) -- (0,2);
\foreach\x in {}
{
\draw[thick] (\x,0.1) -- (\x,-0.1) node[below] {\x};
}
\foreach\y in {}
{
\draw[thick] (0.1,\y) -- (-0.1,\y) node[left] {\y};
}
\draw[thick,blue!50!black] plot[domain=0.26:2,samples=100] (\x,{(-2*\x-1)/(4*\x-1)}) node[above left] {};
\draw[thick,red] (0.25,2) -- (0.25,-4) node[above right] {D};
\node () at (0.4,-0.3) {$a$};
\node () at (-0.4, -0.4) {O};
\node () at (2, 3) {$\displaystyle \lim_{ x \to a^{+}}f(x)=\minf $};
\end{tikzpicture}

\begin{tikzpicture}[>=stealth', scale=0.5]
\clip (-3,-5) rectangle (6,3);
\draw[->,thick] (-3,0) -- (3,0);
\draw[->,thick] (0,-3) -- (0,3);
\foreach\x in {}
{
\draw[thick] (\x,0.1) -- (\x,-0.1) node[below] {\x};
}
\foreach\y in {}
{
\draw[thick] (0.1,\y) -- (-0.1,\y) node[left] {\y};
}
\draw[thick,blue!50!black] plot[domain=-3:-0.25,samples=100] (\x,{(-2*\x+2.5)/(-4*\x-0.06)}) node[above left] {};
\draw[very thick,red] (-0.25,4) -- (-0.25,-2) node[above left] {D};
\node () at (-0.4,-0.3) {$  a$};
\node () at (0, -3.5) {$ \displaystyle\lim_{ x \to a^{-}}f(x)=\pinf $};
\end{tikzpicture}



\begin{tikzpicture}[>=stealth',scale=0.5]
\clip (-1,-5.5) rectangle (7,3);
\draw[->,thick] (-3.5,0) -- (3,0);
\draw[->,thick] (0,-3) -- (0,3);
\foreach\x in {}
{
\draw[thick] (\x,0.1) -- (\x,-0.1) node[below] {\x};
}
\foreach\y in {}
{
\draw[thick] (0.1,\y) -- (-0.1,\y) node[left] {\y};
}
\draw[thick,blue!50!black] plot[domain=-3.5:0.15,samples=100] (\x,{(2*\x+1)/(4*\x-1)}) node[above left] {$\mathscr{C}$};
\draw[very thick,red] (0.25,3) -- (0.25,-3) node[above right] {D};
\node () at (0.4,-0.3) {$  a$};
\node () at (-0.4, -0.4) {O};
\node () at (2, -4.5) {\fbox{$ \displaystyle\lim_{ x \to a^{-}}f(x)=\minf $}};
\end{tikzpicture}


\textbf{Asymptote  oblique}
\begin{definition}
$ \bullet $  La droite $D$  d'équation : $ y=ax+b $   est  une   asymptote (oblique)    à  $ \mathscr{C}_{f}$   en $ \pinf $ si et seulement si
   $ \displaystyle\lim_{ x \to \pinf}\paren{f(x)-(ax+b)}=0$. 
  
    $ \bullet $  La droite $D$  d'équation : $ y=ax+b $   est  une   asymptote  (oblique)   à  $ \mathscr{C}_{f}$   en $ \minf $ si et seulement si
   $ \displaystyle\lim_{ x \to \minf}\paren{f(x)-(ax+b)}=0$ 
   \end{definition}
  \begin{center}
\begin{tikzpicture}[>=stealth', scale=0.5]
\clip (-7,-4) rectangle (7,6);
\draw[->,thick] (-7,0) -- (7,0);
\draw[->,thick] (0,-4) -- (0,5);
\foreach\x in {}
{
\draw[thick] (\x,0.1) -- (\x,-0.1) node[below] {\x};
}
\foreach\y in {}
{
\draw[thick] (0.1,\y) -- (-0.1,\y) node[left] {\y};
}
%\draw[thick,blue!50!black] plot[domain=-7:0.24,samples=100] (\x,{(2*\x+1)/(4*\x-1)}) node[above left] {$\mathscr{C}$};
\draw[thick,blue!50!black] plot[domain=-4:4,samples=100] (\x,{\x+0.3/(\x-1)}) node[above ] {$\mathscr{C}$};
\draw[ thick, red] (4,4) -- (-3,-3) node[above left] {D};
\end{tikzpicture}
\end{center}  

\begin{remark}
  Si $ f $ s'écrit sous la forme $ f(x)= ax+b + g(x) $ et si $\displaystyle \lim_{x \to +\infty}g(x)=0 $  alors la droite $ y=ax+b $ est une asymptote à  $ \mathcal{C}_{f} $ en $ +\infty. $\\
 
 Enoncé valable en $ \minf $.
\end{remark}


\begin{example}

Déterminons une équation de l'asymptote oblique  à la courbe de la fonction $ f: x\longmapsto 3x-5+\dfrac{2}{x+1} $


\medskip

Puisque $ \displaystyle\lim_{x\to \pinf}\dfrac{2}{x+1}  =0 $ et $ \displaystyle\lim_{x\to \minf}\dfrac{2}{x+1}  =0 $ , on en déduit que la droite $ y= 3x-5$  est une asymptote oblique  à  $ \mathscr{C}_{f}$.  
\end{example}
\begin{example}
 
  Déterminons une équation de l'asymptote oblique sachant que $ f(x)= \dfrac{2x^{2}+3x+1}{x-2}$.
\end{example}
Pour cela ,  effectuons la division euclidienne suivante. 
\[
\begin{array}{r|r}
-2x^2 + 3x + 1 & 2x - 2 \\
2x^2 - 4x      & 2x + 7 \\
\hline
-7x + 1        &        \\
7x - 14        &        \\
\hline
15             &
\end{array}
\]

 Ainsi  on a  $ f(x)= 2x+7+\dfrac{15}{x-2}$  et comme  $\displaystyle\lim_{x\to \pinf}\dfrac{15}{x-2}  =0 $ et $ \displaystyle\lim_{x\to \minf}\dfrac{15}{x-2}  =0 $ , on en déduit que la droite $\Delta:  y= 2x+7$  est une asymptote oblique  à  $ \mathscr{C}_{f}$.  

\textbf{Position relative d'une courbe par rapport à une droite}

Soit $\mathscr{C}_{f}$  la courbe d'équation  $ y=f(x) $  et  $\mathscr{D}$   la droite d'équation $ y=ax+b $.

Pour déterminer la position relative de $\mathscr{C}_{f}$   par rapport à  $\mathscr{D}$  sur un intervalle I, on étudie le signe de $ g(x)=f(x)-(ax+b). $\\

$ \bullet $  Si $ g(x)> 0 $ sur I alors $\mathscr{C}_{f}$  est au dessus de  $\mathscr{D}$  sur I .\\

$ \bullet $  Si $ g(x)< 0 $ sur I alors $\mathscr{C}_{f}$  est au dessous de  $\mathscr{D}$  sur I

\begin{example}


 En reprenant l'exemple précèdent,    étudions la position relative de $ \Delta $ et $ \mathcal{C_{f}} $.\\
Pour cela étudions le signe de $ f(x)-(2x+7) = \dfrac{15}{x-2} $. 
\[
\begin{array}{|c|ccccc|}
\hline
x & -\infty &        & 2      &        & +\infty \\
\hline
\text{signe de }\dfrac{15}{x-2} &        & -      &        & +      &        \\
\hline
\end{array}
\]

  Sur $ \intoo{\minf}{2} $  on  a  $\dfrac{15}{x-2} < 0  $ donc $ \mathcal{C_{f}} $ est au dessous de $ \Delta. $\\
  
   Sur $ \intoo{2}{\pinf} $   on a $\dfrac{15}{x-2} > 0  $ donc $ \mathcal{C_{f}} $ est au dessus de $ \Delta. $

\end{example}

\textbf{Branches paraboliques}.

\begin{property}

 \begin{itemize}
\item Si $\displaystyle \lim_{x\to \pinf} f(x) =\infty $ et $\displaystyle \lim_{x\to \pinf}{\dfrac{f(x)}{x} }=\infty $ alors $\mathscr{C}_{f}$  admet une branche parabolique de direction l'axe des ordonnées.
\item Si $\displaystyle \lim_{x\to \pinf} f(x) =\infty $ et $\displaystyle \lim_{x\to \pinf}{\dfrac{f(x)}{x} }=0 $ alors $\mathscr{C}_{f}$  admet une branche parabolique de direction l'axe des abscisses.
\end{itemize}
 

\end{property}
\begin{remark}

La courbe d'une fonction polynôme admet à l'infini une branche parabolique dans la direction l'axe des ordonnées.
\end{remark}



\subsection{Représentation graphique d'une fonction}
Pour représenter la  courbe d'une fonction, on peut procéder comme suit:
\begin{itemize}
\item[$  \blacktriangleright$] représenter les points particuliers de la courbe ( points figurants sur le tableau de variation, points d'intersection avec les axes $ \cdots $)
\item[$  \blacktriangleright$]représenter les droites particulières de la courbe ( asymptotes , tangentes $ \cdots $)
\item[$  \blacktriangleright$] tracer la courbe en s'appuyant  sur les variations de $ f $ sur chaque intervalle du tableau de variation.
\end{itemize}
\begin{remark}
Si les points particuliers et les droites remarquables  ne sont sont pas suffisants pour tracer la courbe, on peut dresser un tableau de valeurs permettant de représenter quelques points de la courbe.
\end{remark}
\textbf{Intersection d'une courbe et d'une droite}\\
\textbf{Méthode}
\begin{itemize}
\item[$  \blacktriangleright$]  Le point d'intersection de $ \mathcal{C}_{f} $  avec l'axe des ordonnées est le point ($ 0,f(0)) $  si $ 0\in \mathcal{D}_{f} $
\item[$  \blacktriangleright$]  Pour déterminer les points d'intersection de  $ \mathcal{C}_{f} $  avec l'axe des  abscisses , on résout l'équation $ f(x)=0. $
\item[$  \blacktriangleright$]   Pour déterminer les points d'intersection de  $ \mathcal{C}_{f} $  avec la droite d'équation $ y=ax+b $ , on résout l'équation $ f(x)=ax+b. $
\end{itemize}


\subsection{Résolution graphique de l'équation   f(x) = m }
\textbf{Méthode}\\
Le nombre de solutions de cette équation est égal au nombre de points d'intersection de la courbe de $ f $ avec la droite d'équation $ y=m$  parallèle à l'axe des abscisses  qu'on fera \textit{coulisser} du bas vers le haut pour déterminer le nombre de points d'intersection suivant les valeurs de $ m.$

\subsection{Représentation  graphique de la fonction | f(x)|}
\textbf{Méthode}\\
Si $ g(x)=\abs{f(x)} $ alors 
\begin{itemize}
\item[$  \blacktriangleright$]  $ \mathcal{C}_{g} $  est confondue à $ \mathcal{C}_{f} $  si $ \mathcal{C}_{f} $  est sur ou au dessus de l'axe des abscisses.
\item[$  \blacktriangleright$]  $ \mathcal{C}_{g} $  est le symétrique  de $ \mathcal{C}_{f} $ par rapport à l'axe des abscisses   si $ \mathcal{C}_{f} $  est sur ou au dessous de l'axe des abscisses.

\end{itemize}

  %</content>
\end{document}
