\documentclass[12pt, a4paper]{report}

% LuaLaTeX :

\RequirePackage{iftex}
\RequireLuaTeX

% Packages :

\usepackage[french]{babel}
%\usepackage[utf8]{inputenc}
%\usepackage[T1]{fontenc}
\usepackage[pdfencoding=auto, pdfauthor={Hugo Delaunay}, pdfsubject={Mathématiques}, pdfcreator={agreg.skyost.eu}]{hyperref}
\usepackage{amsmath}
\usepackage{amsthm}
%\usepackage{amssymb}
\usepackage{stmaryrd}
\usepackage{tikz}
\usepackage{tkz-euclide}
\usepackage{fontspec}
\defaultfontfeatures[Erewhon]{FontFace = {bx}{n}{Erewhon-Bold.otf}}
\usepackage{fourier-otf}
\usepackage[nobottomtitles*]{titlesec}
\usepackage{fancyhdr}
\usepackage{listings}
\usepackage{catchfilebetweentags}
\usepackage[french, capitalise, noabbrev]{cleveref}
\usepackage[fit, breakall]{truncate}
\usepackage[top=2.5cm, right=2cm, bottom=2.5cm, left=2cm]{geometry}
\usepackage{enumitem}
\usepackage{tocloft}
\usepackage{microtype}
%\usepackage{mdframed}
%\usepackage{thmtools}
\usepackage{xcolor}
\usepackage{tabularx}
\usepackage{xltabular}
\usepackage{aligned-overset}
\usepackage[subpreambles=true]{standalone}
\usepackage{environ}
\usepackage[normalem]{ulem}
\usepackage{etoolbox}
\usepackage{setspace}
\usepackage[bibstyle=reading, citestyle=draft]{biblatex}
\usepackage{xpatch}
\usepackage[many, breakable]{tcolorbox}
\usepackage[backgroundcolor=white, bordercolor=white, textsize=scriptsize]{todonotes}
\usepackage{luacode}
\usepackage{float}
\usepackage{needspace}
\everymath{\displaystyle}

% Police :

\setmathfont{Erewhon Math}

% Tikz :

\usetikzlibrary{calc}
\usetikzlibrary{3d}

% Longueurs :

\setlength{\parindent}{0pt}
\setlength{\headheight}{15pt}
\setlength{\fboxsep}{0pt}
\titlespacing*{\chapter}{0pt}{-20pt}{10pt}
\setlength{\marginparwidth}{1.5cm}
\setstretch{1.1}

% Métadonnées :

\author{agreg.skyost.eu}
\date{\today}

% Titres :

\setcounter{secnumdepth}{3}

\renewcommand{\thechapter}{\Roman{chapter}}
\renewcommand{\thesubsection}{\Roman{subsection}}
\renewcommand{\thesubsubsection}{\arabic{subsubsection}}
\renewcommand{\theparagraph}{\alph{paragraph}}

\titleformat{\chapter}{\huge\bfseries}{\thechapter}{20pt}{\huge\bfseries}
\titleformat*{\section}{\LARGE\bfseries}
\titleformat{\subsection}{\Large\bfseries}{\thesubsection \, - \,}{0pt}{\Large\bfseries}
\titleformat{\subsubsection}{\large\bfseries}{\thesubsubsection. \,}{0pt}{\large\bfseries}
\titleformat{\paragraph}{\bfseries}{\theparagraph. \,}{0pt}{\bfseries}

\setcounter{secnumdepth}{4}

% Table des matières :

\renewcommand{\cftsecleader}{\cftdotfill{\cftdotsep}}
\addtolength{\cftsecnumwidth}{10pt}

% Redéfinition des commandes :

\renewcommand*\thesection{\arabic{section}}
\renewcommand{\ker}{\mathrm{Ker}}

% Nouvelles commandes :

\newcommand{\website}{https://github.com/imbodj/SenCoursDeMaths}

\newcommand{\tr}[1]{\mathstrut ^t #1}
\newcommand{\im}{\mathrm{Im}}
\newcommand{\rang}{\operatorname{rang}}
\newcommand{\trace}{\operatorname{trace}}
\newcommand{\id}{\operatorname{id}}
\newcommand{\stab}{\operatorname{Stab}}
\newcommand{\paren}[1]{\left(#1\right)}
\newcommand{\croch}[1]{\left[ #1 \right]}
\newcommand{\Grdcroch}[1]{\Bigl[ #1 \Bigr]}
\newcommand{\grdcroch}[1]{\bigl[ #1 \bigr]}
\newcommand{\abs}[1]{\left\lvert #1 \right\rvert}
\newcommand{\limi}[3]{\lim_{#1\to #2}#3}
\newcommand{\pinf}{+\infty}
\newcommand{\minf}{-\infty}
%%%%%%%%%%%%%% ENSEMBLES %%%%%%%%%%%%%%%%%
\newcommand{\ensemblenombre}[1]{\mathbb{#1}}
\newcommand{\Nn}{\ensemblenombre{N}}
\newcommand{\Zz}{\ensemblenombre{Z}}
\newcommand{\Qq}{\ensemblenombre{Q}}
\newcommand{\Qqp}{\Qq^+}
\newcommand{\Rr}{\ensemblenombre{R}}
\newcommand{\Cc}{\ensemblenombre{C}}
\newcommand{\Nne}{\Nn^*}
\newcommand{\Zze}{\Zz^*}
\newcommand{\Zzn}{\Zz^-}
\newcommand{\Qqe}{\Qq^*}
\newcommand{\Rre}{\Rr^*}
\newcommand{\Rrp}{\Rr_+}
\newcommand{\Rrm}{\Rr_-}
\newcommand{\Rrep}{\Rr_+^*}
\newcommand{\Rrem}{\Rr_-^*}
\newcommand{\Cce}{\Cc^*}
%%%%%%%%%%%%%%  INTERVALLES %%%%%%%%%%%%%%%%%
\newcommand{\intff}[2]{\left[#1\;,\; #2\right]  }
\newcommand{\intof}[2]{\left]#1 \;, \;#2\right]  }
\newcommand{\intfo}[2]{\left[#1 \;,\; #2\right[  }
\newcommand{\intoo}[2]{\left]#1 \;,\; #2\right[  }

\providecommand{\newpar}{\\[\medskipamount]}

\newcommand{\annexessection}{%
  \newpage%
  \subsection*{Annexes}%
}

\providecommand{\lesson}[3]{%
  \title{#3}%
  \hypersetup{pdftitle={#2 : #3}}%
  \setcounter{section}{\numexpr #2 - 1}%
  \section{#3}%
  \fancyhead[R]{\truncate{0.73\textwidth}{#2 : #3}}%
}

\providecommand{\development}[3]{%
  \title{#3}%
  \hypersetup{pdftitle={#3}}%
  \section*{#3}%
  \fancyhead[R]{\truncate{0.73\textwidth}{#3}}%
}

\providecommand{\sheet}[3]{\development{#1}{#2}{#3}}

\providecommand{\ranking}[1]{%
  \title{Terminale #1}%
  \hypersetup{pdftitle={Terminale #1}}%
  \section*{Terminale #1}%
  \fancyhead[R]{\truncate{0.73\textwidth}{Terminale #1}}%
}

\providecommand{\summary}[1]{%
  \textit{#1}%
  \par%
  \medskip%
}

\tikzset{notestyleraw/.append style={inner sep=0pt, rounded corners=0pt, align=center}}

%\newcommand{\booklink}[1]{\website/bibliographie\##1}
\newcounter{reference}
\newcommand{\previousreference}{}
\providecommand{\reference}[2][]{%
  \needspace{20pt}%
  \notblank{#1}{
    \needspace{20pt}%
    \renewcommand{\previousreference}{#1}%
    \stepcounter{reference}%
    \label{reference-\previousreference-\thereference}%
  }{}%
  \todo[noline]{%
    \protect\vspace{20pt}%
    \protect\par%
    \protect\notblank{#1}{\cite{[\previousreference]}\\}{}%
    \protect\hyperref[reference-\previousreference-\thereference]{p. #2}%
  }%
}

\definecolor{devcolor}{HTML}{00695c}
\providecommand{\dev}[1]{%
  \reversemarginpar%
  \todo[noline]{
    \protect\vspace{20pt}%
    \protect\par%
    \bfseries\color{devcolor}\href{\website/developpements/#1}{[DEV]}
  }%
  \normalmarginpar%
}

% En-têtes :

\pagestyle{fancy}
\fancyhead[L]{\truncate{0.23\textwidth}{\thepage}}
\fancyfoot[C]{\scriptsize \href{\website}{\texttt{https://github.com/imbodj/SenCoursDeMaths}}}

% Couleurs :

\definecolor{property}{HTML}{ffeb3b}
\definecolor{proposition}{HTML}{ffc107}
\definecolor{lemma}{HTML}{ff9800}
\definecolor{theorem}{HTML}{f44336}
\definecolor{corollary}{HTML}{e91e63}
\definecolor{definition}{HTML}{673ab7}
\definecolor{notation}{HTML}{9c27b0}
\definecolor{example}{HTML}{00bcd4}
\definecolor{cexample}{HTML}{795548}
\definecolor{application}{HTML}{009688}
\definecolor{remark}{HTML}{3f51b5}
\definecolor{algorithm}{HTML}{607d8b}
%\definecolor{proof}{HTML}{e1f5fe}
\definecolor{exercice}{HTML}{e1f5fe}

% Théorèmes :

\theoremstyle{definition}
\newtheorem{theorem}{Théorème}

\newtheorem{property}[theorem]{Propriété}
\newtheorem{proposition}[theorem]{Proposition}
\newtheorem{lemma}[theorem]{Activité d'introduction}
\newtheorem{corollary}[theorem]{Conséquence}

\newtheorem{definition}[theorem]{Définition}
\newtheorem{notation}[theorem]{Notation}

\newtheorem{example}[theorem]{Exemple}
\newtheorem{cexample}[theorem]{Contre-exemple}
\newtheorem{application}[theorem]{Application}

\newtheorem{algorithm}[theorem]{Algorithme}
\newtheorem{exercice}[theorem]{Exercice}

\theoremstyle{remark}
\newtheorem{remark}[theorem]{Remarque}

\counterwithin*{theorem}{section}

\newcommand{\applystyletotheorem}[1]{
  \tcolorboxenvironment{#1}{
    enhanced,
    breakable,
    colback=#1!8!white,
    %right=0pt,
    %top=8pt,
    %bottom=8pt,
    boxrule=0pt,
    frame hidden,
    sharp corners,
    enhanced,borderline west={4pt}{0pt}{#1},
    %interior hidden,
    sharp corners,
    after=\par,
  }
}

\applystyletotheorem{property}
\applystyletotheorem{proposition}
\applystyletotheorem{lemma}
\applystyletotheorem{theorem}
\applystyletotheorem{corollary}
\applystyletotheorem{definition}
\applystyletotheorem{notation}
\applystyletotheorem{example}
\applystyletotheorem{cexample}
\applystyletotheorem{application}
\applystyletotheorem{remark}
%\applystyletotheorem{proof}
\applystyletotheorem{algorithm}
\applystyletotheorem{exercice}

% Environnements :

\NewEnviron{whitetabularx}[1]{%
  \renewcommand{\arraystretch}{2.5}
  \colorbox{white}{%
    \begin{tabularx}{\textwidth}{#1}%
      \BODY%
    \end{tabularx}%
  }%
}

% Maths :

\DeclareFontEncoding{FMS}{}{}
\DeclareFontSubstitution{FMS}{futm}{m}{n}
\DeclareFontEncoding{FMX}{}{}
\DeclareFontSubstitution{FMX}{futm}{m}{n}
\DeclareSymbolFont{fouriersymbols}{FMS}{futm}{m}{n}
\DeclareSymbolFont{fourierlargesymbols}{FMX}{futm}{m}{n}
\DeclareMathDelimiter{\VERT}{\mathord}{fouriersymbols}{152}{fourierlargesymbols}{147}

% Code :

\definecolor{greencode}{rgb}{0,0.6,0}
\definecolor{graycode}{rgb}{0.5,0.5,0.5}
\definecolor{mauvecode}{rgb}{0.58,0,0.82}
\definecolor{bluecode}{HTML}{1976d2}
\lstset{
  basicstyle=\footnotesize\ttfamily,
  breakatwhitespace=false,
  breaklines=true,
  %captionpos=b,
  commentstyle=\color{greencode},
  deletekeywords={...},
  escapeinside={\%*}{*)},
  extendedchars=true,
  frame=none,
  keepspaces=true,
  keywordstyle=\color{bluecode},
  language=Python,
  otherkeywords={*,...},
  numbers=left,
  numbersep=5pt,
  numberstyle=\tiny\color{graycode},
  rulecolor=\color{black},
  showspaces=false,
  showstringspaces=false,
  showtabs=false,
  stepnumber=2,
  stringstyle=\color{mauvecode},
  tabsize=2,
  %texcl=true,
  xleftmargin=10pt,
  %title=\lstname
}

\newcommand{\codedirectory}{}
\newcommand{\inputalgorithm}[1]{%
  \begin{algorithm}%
    \strut%
    \lstinputlisting{\codedirectory#1}%
  \end{algorithm}%
}



\everymath{\displaystyle}
\begin{document}
  %<*content>
  \lesson{algebra}{9}{Variables aléatoires}
  
  \begin{lemma}
 On lance deux fois de suite une pièce de monnaie  et on note  les côtés apparus :  Pile ( $ P$ ) ou Face ($F $).\\L'ensemble des issues est:\\ $ \Omega=\accol{(P,P);(F,F);(F,P);(P,F)} $.\\ On convient du jeu suivant: on  gagne $ 5F $  chaque fois que sort Pile et on perd $ 2F $ chaque fois sort Face. \\ Par exemple à l'issue $ (F,P) $ on perd 2F et gagne 5F donc le gain résultant est $ -2+5=3 $F.
 \begin{enumerate}
\item On note par  G un gain  possible pour un joueur. Donner toutes les valeurs $ x $ de G.
\item Justifier que G est une application et préciser  son ensemble de départ et d'arrivée.
\item Pour chacune des valeurs  $ x $  de G, calculer la probabilité de gagner $ x $ francs ?\\ On  notera cette probabilité par $P(G  =x)$.
\item Vérifier que la somme des probabilités trouvées est égale à 1.
\end{enumerate}

  \end{lemma}
\begin{definition}
$ \Omega $  est l'ensemble des issues d'une expérience aléatoire.\\ Une \textbf{variable aléatoire } sur  $ \Omega $  est une fonction qui, à chaque issue de $ \Omega $ , associe un nombre réel.
\end{definition}
 \begin{notation}
 Une variable aléatoire  est généralement notée $X $, $ Y $ ,$Z $ $ \cdots $\\ Lorsque $ x $ désigne un nombre réel, dire << $ X $ prend la valeur $ x $>>  est un événement, il est noté $ (X=x) $.\\$ (X< x) $ désigne l'événement << $ X $ prend une  valeur strictement inférieure à $ x $>> \\
  $ (X\geq x) $ désigne l'événement << $ X $ prend  au moins une fois la  valeur  $ x $>> , c'est le contraire  de l'événement précédent. 
  
   \end{notation}
   \begin{example}
   On reprend l'exercice de l'activité.\\
On lance deux fois de suite une pièce de monnaie  et on note  les côtés apparus : $ P$ ou $F $.\\L'ensemble des issues est:\\ $ \Omega=\accol{(P,P);(F,F);(F,P);(P,F)} $.\\ On gagne $ 5F $  chaque fois que sort Pile et on perd $ 2F $ chaque fois sort Face. On définit ainsi une variable aléatoire $X $ sur $ \Omega $  qui prend les valeurs ; $-4 $; \;$  3$   et $ 10$.\\L'événement $ (X=3) $ est réalisé par  les issues $ (F,P)$ et $(P,F) $
   \end{example}
   \subsection{Loi de probabilité d'une variable aléatoire}
\begin{definition}
Une loi de probabilité est définie sur un ensemble $ \Omega $  d'issues.\\ $ X $  est une variable aléatoire  définie sur $ \Omega $ et $ E=\accol{x_{1},x_{2},\cdots, x_{n}} $ est l'ensemble des valeurs prises par $ X. $\\Lorsqu'on associe  à chaque valeur  $ x_{i} $ , la probabilité de l'événement  $ (X=x_{i}) $, on définit une loi de probabilité sur $ E. $\\ Cette loi est appelée \textbf{loi de probabilité de variable aléatoire $ X $.}
\end{definition}

\begin{remark}
 On présente souvent la loi de probabilité d'une variable aléatoire  discrète $ X $ à l'aide d'un tableau.

$\begin{array}{|c|c|c|c|c|}
\hline
\text{Valeurs de}\;\;  X & x_{1}& x_{2} &\cdots  &x_{n}\\
\hline
 P(X=x_{i})  &  p_{1} &  p_{2}   &\cdots   &  p_{n} \\
\hline
\end{array}$

On a: $ P(X=x_{1})+P(X=x_{2})+\cdots +P(X=x_{n})= p_{1}+p_{2}+\cdots + p_{n}=\displaystyle\sum_{k=1}^n p_{k}= 1 $
\end{remark}
\begin{example}
On reprend l'exemple de l'activité d'introduction.

 La probabilité de l'événement  $ (X= -4) $  est la probabilité de l'issue $ (F,F) $ c'est-à-dire $ P(X=-4) =\frac{1}{4}$\\
La probabilité de l'événement  $ (X= 3) $  est la somme des  probabilités des issues $ (P,F) $  et  $ (F,P) $ c'est-à-dire $ P(X=3) =\frac{1}{4}+\frac{1}{4}=\frac{1}{2}$.\\
La probabilité de l'événement  $ (X= 10) $  est la probabilité de l'issue $ (P,P) $ c'est-à-dire $ P(X=10) =\frac{1}{4}$\\ La loi de la variable aléatoire $ X $ est résumée dans le tableau ci-dessous.\\

$\begin{array}{|c|c|c|c|c|}
\hline
  x_{i} &  -4 &  3 & 10 \\
\hline
 P(X=x_{i})  &  \frac{1}{4} &  \frac{1}{2}    &  \frac{1}{4} \\
\hline
\end{array}$
\end{example}
\subsection{Fonction de répartition}
\begin{definition}
Soit $ X $  une variable aléatoire définie sur un univers $ \Omega $ muni d'une probabilité P.\\
La fonction de répartition de $ X $  est l'application $ F $ de $ \Rr $ vers $ \intff{0}{1} $ définie par:\;$ F(x)=P (X\leq x) $
\end{definition}

\begin{example}
Reprenons l'exemple  de l'activité.\\
$ F $ est définie par: $\sysq{F(x)=0,\;\; \text{si}\;\; x< -4}{F(x)=\frac{1}{4}, \;\; \text{si}\;\; -4\leq x< 3}{F(x)=\frac{3}{4}, \;\; \text{si}\;\; 3\leq x< 10}{F(x)=1, \;\; \text{si}\;\; 10\leq x} $

\bigskip

\begin{tikzpicture}[xscale= 0.25,yscale=2]
%\draw[very thin,style=gray!60,step=0.1] (-7,-4) grid (9,9);
\draw[very thin] (-10,-0.1)  (20,1);
\draw[thick] (-10,0) -- (20,0);
\draw[thick] (0,-0.1) -- (0,1);
\foreach\x in {-4,,,,,,,3,,,,,,10}
{
\draw[thick] (\x,0.1) -- (\x,-0.1) node[below] {\x};
}
\foreach\y in { 0.25,0.75,1 }
{
\draw[thick] (0.1,\y) -- (-0.1,\y) node[left] {\y};
}
\draw[very thick,domain= -10:-4]  plot(\x,0);
\draw[ very thick,domain= -4.2:3]  plot(\x,0.25);
\draw [very thick, domain= 3:10]  plot(\x,3*0.25);
\draw[very thick, domain= 10:20]  plot(\x, 1);
%\draw[domain= -5:1.5]   plot(\x,-2*\x-1);
\node at (-4,0) {$ < $};
\node at (10,0.75) {$ < $};
\node at (3,0.25) {$ < $};
\end{tikzpicture}

\end{example}
\begin{remark}
\begin{itemize}
\item  $ F $ est une fonction croissante en escalier. 
\item  La  représentation graphique de F correspond en statistiques à la courbe des fréquences cumulées croissantes.
\end{itemize}
\end{remark}

\subsection{Paramètres d'une variable aléatoire}
\textbf{Espérance, variance et écart-type}
 \begin{definition}
 Une loi de probabilité est définie sur un ensemble $ \Omega $  d'issues.\\ $ X $  est une variable aléatoire  définie sur $ \Omega $ dont la loi de probabilité est résumée dans le tableau ci-dessous.
\bigskip

$\begin{array}{|c|c|c|c|c|}
\hline
\text{Valeurs de} \;\; X & x_{1}& x_{2} &\cdots  &x_{n}\\
\hline
 P(X=x_{i})  &  p_{1} &  p_{2}   &\cdots   &  p_{n} \\
\hline
\end{array}$


\begin{itemize}
\item L'espérance de la variable aléatoire $ X $  est le nombre réel, noté E$ (X) $  défini par :
  \[ \text{E}(X)=p_{1}x_{1} + p_{2}x_{2} +\cdots + p_{n}x_{n} =\displaystyle \sum_{k=1}^n p_{k}x_{k} \]
  
   \item  La variance  de la variable aléatoire $ X $  est le nombre réel  positif , noté V$ (X) $  défini par :
  \[ \text{V}(X)=p_{1}\paren{x_{1}-\text{E}(X)}^{2} + p_{2}\paren{x_{2}-\text{E}(X)}^{2} +\cdots + p_{n}\paren{x_{n}-\text{E}(X)}^{2} =\displaystyle\sum_{k=1}^n p_{k}\paren{x_{k}-\text{E}(X)}^{2}\]
\item L'écart-type  de la variable aléatoire $ X $  est le nombre réel  positif , noté $ \sigma(X) $  défini par :$$ \sigma(X)=\sqrt{\text{V}(X)}.$$
 \end{itemize}
\end{definition}

\begin{example}
On reprend l'exemple de la variable aléatoire $ X $ précédent.\\ $\text{E}(X)=\frac{1}{4}(-4)+\frac{1}{2}\times 3 +\frac{1}{4}\times 10 =3$ francs. $ \quad\text{E}(X)=3$ francs    signifie qu'en jouant un grand  nombre de fois  à ce jeu , un joueur peut espérer gagner $ 3$ francs    en moyenne.\\ $ \text{V}(X)= \dfrac{1}{4}(-4-3)^{2}+\dfrac{1}{2}(3-3)^{2}+\dfrac{1}{4}(10-3)^{2}=\dfrac{49}{2} $ \;\;et \quad $ \sigma(X)=\dfrac{7\sqrt{2}}{2} $.
\end{example}
\begin{remark}
\begin{itemize}
\item   L'espérance mathématique correspond, en statistiques, à la moyenne.
\item L'espérance et l'écart-type sont exprimés dans la même unité que les valeurs $ x_{i} $ prises par $ X $
\item Un jeu est dit équitable lorsque $ \text{E}(X)=0.$
\end{itemize}
\end{remark}
\begin{property}
Soit $ X  $ une  variable aléatoire. On a \; $ \text{V}(X)=  \text{E}(X^{2})- (\text{E}(X))^{2} $
\end{property}
\textbf{Démonstration}\\
On a V(X)$ =\displaystyle\sum_{k=1}^n p_{k}\paren{x_{k}-\text{E}(X)}^{2}=\displaystyle\sum_{k=1}^n p_{k}x_{k}^{2}-2\text{E}(X)\displaystyle\sum_{k=1}^n p_{k}x_{k}+ \paren{\text{E}(X)}^{2}\displaystyle\sum_{k=1}^n p_{k}$\\
Or $\displaystyle \sum_{k=1}^n p_{k}x_{k}=\text{E}(X) $,\;  $\displaystyle\sum_{k=1}^n p_{k}=1  $\; et  $\;\displaystyle \sum_{k=1}^n p_{k}x_{k}^{2}=\text{E}(X^{2}) $\\

V(X)$ =\text{E}(X^{2})-2\times \text{E}(X)\times \text{E}(X)+\paren{\text{E}(X)}^{2} =\text{E}(X^{2})- (\text{E}(X))^{2}$

\begin{property}[Admise]
$ X $  est une  variable aléatoire. Pour  tous nombres réels $a $ et $b $.
 $$\text{E}(aX+b)=a\text{E}(X)+b     \qquad  \text{et} \quad  \text{V}(aX+b)=a^{2}\text{V}(X) $$
\end{property}

   
  %</content>
\end{document}
