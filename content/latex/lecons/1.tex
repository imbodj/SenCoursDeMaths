\documentclass[12pt, a4paper]{report}

% LuaLaTeX :

\RequirePackage{iftex}
\RequireLuaTeX

% Packages :

\usepackage[french]{babel}
%\usepackage[utf8]{inputenc}
%\usepackage[T1]{fontenc}
\usepackage[pdfencoding=auto, pdfauthor={Hugo Delaunay}, pdfsubject={Mathématiques}, pdfcreator={agreg.skyost.eu}]{hyperref}
\usepackage{amsmath}
\usepackage{amsthm}
%\usepackage{amssymb}
\usepackage{stmaryrd}
\usepackage{tikz}
\usepackage{tkz-euclide}
\usepackage{fontspec}
\defaultfontfeatures[Erewhon]{FontFace = {bx}{n}{Erewhon-Bold.otf}}
\usepackage{fourier-otf}
\usepackage[nobottomtitles*]{titlesec}
\usepackage{fancyhdr}
\usepackage{listings}
\usepackage{catchfilebetweentags}
\usepackage[french, capitalise, noabbrev]{cleveref}
\usepackage[fit, breakall]{truncate}
\usepackage[top=2.5cm, right=2cm, bottom=2.5cm, left=2cm]{geometry}
\usepackage{enumitem}
\usepackage{tocloft}
\usepackage{microtype}
%\usepackage{mdframed}
%\usepackage{thmtools}
\usepackage{xcolor}
\usepackage{tabularx}
\usepackage{xltabular}
\usepackage{aligned-overset}
\usepackage[subpreambles=true]{standalone}
\usepackage{environ}
\usepackage[normalem]{ulem}
\usepackage{etoolbox}
\usepackage{setspace}
\usepackage[bibstyle=reading, citestyle=draft]{biblatex}
\usepackage{xpatch}
\usepackage[many, breakable]{tcolorbox}
\usepackage[backgroundcolor=white, bordercolor=white, textsize=scriptsize]{todonotes}
\usepackage{luacode}
\usepackage{float}
\usepackage{needspace}
\everymath{\displaystyle}

% Police :

\setmathfont{Erewhon Math}

% Tikz :

\usetikzlibrary{calc}
\usetikzlibrary{3d}

% Longueurs :

\setlength{\parindent}{0pt}
\setlength{\headheight}{15pt}
\setlength{\fboxsep}{0pt}
\titlespacing*{\chapter}{0pt}{-20pt}{10pt}
\setlength{\marginparwidth}{1.5cm}
\setstretch{1.1}

% Métadonnées :

\author{agreg.skyost.eu}
\date{\today}

% Titres :

\setcounter{secnumdepth}{3}

\renewcommand{\thechapter}{\Roman{chapter}}
\renewcommand{\thesubsection}{\Roman{subsection}}
\renewcommand{\thesubsubsection}{\arabic{subsubsection}}
\renewcommand{\theparagraph}{\alph{paragraph}}

\titleformat{\chapter}{\huge\bfseries}{\thechapter}{20pt}{\huge\bfseries}
\titleformat*{\section}{\LARGE\bfseries}
\titleformat{\subsection}{\Large\bfseries}{\thesubsection \, - \,}{0pt}{\Large\bfseries}
\titleformat{\subsubsection}{\large\bfseries}{\thesubsubsection. \,}{0pt}{\large\bfseries}
\titleformat{\paragraph}{\bfseries}{\theparagraph. \,}{0pt}{\bfseries}

\setcounter{secnumdepth}{4}

% Table des matières :

\renewcommand{\cftsecleader}{\cftdotfill{\cftdotsep}}
\addtolength{\cftsecnumwidth}{10pt}

% Redéfinition des commandes :

\renewcommand*\thesection{\arabic{section}}
\renewcommand{\ker}{\mathrm{Ker}}

% Nouvelles commandes :

\newcommand{\website}{https://github.com/imbodj/SenCoursDeMaths}

\newcommand{\tr}[1]{\mathstrut ^t #1}
\newcommand{\im}{\mathrm{Im}}
\newcommand{\rang}{\operatorname{rang}}
\newcommand{\trace}{\operatorname{trace}}
\newcommand{\id}{\operatorname{id}}
\newcommand{\stab}{\operatorname{Stab}}
\newcommand{\paren}[1]{\left(#1\right)}
\newcommand{\croch}[1]{\left[ #1 \right]}
\newcommand{\Grdcroch}[1]{\Bigl[ #1 \Bigr]}
\newcommand{\grdcroch}[1]{\bigl[ #1 \bigr]}
\newcommand{\abs}[1]{\left\lvert #1 \right\rvert}
\newcommand{\limi}[3]{\lim_{#1\to #2}#3}
\newcommand{\pinf}{+\infty}
\newcommand{\minf}{-\infty}
%%%%%%%%%%%%%% ENSEMBLES %%%%%%%%%%%%%%%%%
\newcommand{\ensemblenombre}[1]{\mathbb{#1}}
\newcommand{\Nn}{\ensemblenombre{N}}
\newcommand{\Zz}{\ensemblenombre{Z}}
\newcommand{\Qq}{\ensemblenombre{Q}}
\newcommand{\Qqp}{\Qq^+}
\newcommand{\Rr}{\ensemblenombre{R}}
\newcommand{\Cc}{\ensemblenombre{C}}
\newcommand{\Nne}{\Nn^*}
\newcommand{\Zze}{\Zz^*}
\newcommand{\Zzn}{\Zz^-}
\newcommand{\Qqe}{\Qq^*}
\newcommand{\Rre}{\Rr^*}
\newcommand{\Rrp}{\Rr_+}
\newcommand{\Rrm}{\Rr_-}
\newcommand{\Rrep}{\Rr_+^*}
\newcommand{\Rrem}{\Rr_-^*}
\newcommand{\Cce}{\Cc^*}
%%%%%%%%%%%%%%  INTERVALLES %%%%%%%%%%%%%%%%%
\newcommand{\intff}[2]{\left[#1\;,\; #2\right]  }
\newcommand{\intof}[2]{\left]#1 \;, \;#2\right]  }
\newcommand{\intfo}[2]{\left[#1 \;,\; #2\right[  }
\newcommand{\intoo}[2]{\left]#1 \;,\; #2\right[  }

\providecommand{\newpar}{\\[\medskipamount]}

\newcommand{\annexessection}{%
  \newpage%
  \subsection*{Annexes}%
}

\providecommand{\lesson}[3]{%
  \title{#3}%
  \hypersetup{pdftitle={#2 : #3}}%
  \setcounter{section}{\numexpr #2 - 1}%
  \section{#3}%
  \fancyhead[R]{\truncate{0.73\textwidth}{#2 : #3}}%
}

\providecommand{\development}[3]{%
  \title{#3}%
  \hypersetup{pdftitle={#3}}%
  \section*{#3}%
  \fancyhead[R]{\truncate{0.73\textwidth}{#3}}%
}

\providecommand{\sheet}[3]{\development{#1}{#2}{#3}}

\providecommand{\ranking}[1]{%
  \title{Terminale #1}%
  \hypersetup{pdftitle={Terminale #1}}%
  \section*{Terminale #1}%
  \fancyhead[R]{\truncate{0.73\textwidth}{Terminale #1}}%
}

\providecommand{\summary}[1]{%
  \textit{#1}%
  \par%
  \medskip%
}

\tikzset{notestyleraw/.append style={inner sep=0pt, rounded corners=0pt, align=center}}

%\newcommand{\booklink}[1]{\website/bibliographie\##1}
\newcounter{reference}
\newcommand{\previousreference}{}
\providecommand{\reference}[2][]{%
  \needspace{20pt}%
  \notblank{#1}{
    \needspace{20pt}%
    \renewcommand{\previousreference}{#1}%
    \stepcounter{reference}%
    \label{reference-\previousreference-\thereference}%
  }{}%
  \todo[noline]{%
    \protect\vspace{20pt}%
    \protect\par%
    \protect\notblank{#1}{\cite{[\previousreference]}\\}{}%
    \protect\hyperref[reference-\previousreference-\thereference]{p. #2}%
  }%
}

\definecolor{devcolor}{HTML}{00695c}
\providecommand{\dev}[1]{%
  \reversemarginpar%
  \todo[noline]{
    \protect\vspace{20pt}%
    \protect\par%
    \bfseries\color{devcolor}\href{\website/developpements/#1}{[DEV]}
  }%
  \normalmarginpar%
}

% En-têtes :

\pagestyle{fancy}
\fancyhead[L]{\truncate{0.23\textwidth}{\thepage}}
\fancyfoot[C]{\scriptsize \href{\website}{\texttt{https://github.com/imbodj/SenCoursDeMaths}}}

% Couleurs :

\definecolor{property}{HTML}{ffeb3b}
\definecolor{proposition}{HTML}{ffc107}
\definecolor{lemma}{HTML}{ff9800}
\definecolor{theorem}{HTML}{f44336}
\definecolor{corollary}{HTML}{e91e63}
\definecolor{definition}{HTML}{673ab7}
\definecolor{notation}{HTML}{9c27b0}
\definecolor{example}{HTML}{00bcd4}
\definecolor{cexample}{HTML}{795548}
\definecolor{application}{HTML}{009688}
\definecolor{remark}{HTML}{3f51b5}
\definecolor{algorithm}{HTML}{607d8b}
%\definecolor{proof}{HTML}{e1f5fe}
\definecolor{exercice}{HTML}{e1f5fe}

% Théorèmes :

\theoremstyle{definition}
\newtheorem{theorem}{Théorème}

\newtheorem{property}[theorem]{Propriété}
\newtheorem{proposition}[theorem]{Proposition}
\newtheorem{lemma}[theorem]{Activité d'introduction}
\newtheorem{corollary}[theorem]{Conséquence}

\newtheorem{definition}[theorem]{Définition}
\newtheorem{notation}[theorem]{Notation}

\newtheorem{example}[theorem]{Exemple}
\newtheorem{cexample}[theorem]{Contre-exemple}
\newtheorem{application}[theorem]{Application}

\newtheorem{algorithm}[theorem]{Algorithme}
\newtheorem{exercice}[theorem]{Exercice}

\theoremstyle{remark}
\newtheorem{remark}[theorem]{Remarque}

\counterwithin*{theorem}{section}

\newcommand{\applystyletotheorem}[1]{
  \tcolorboxenvironment{#1}{
    enhanced,
    breakable,
    colback=#1!8!white,
    %right=0pt,
    %top=8pt,
    %bottom=8pt,
    boxrule=0pt,
    frame hidden,
    sharp corners,
    enhanced,borderline west={4pt}{0pt}{#1},
    %interior hidden,
    sharp corners,
    after=\par,
  }
}

\applystyletotheorem{property}
\applystyletotheorem{proposition}
\applystyletotheorem{lemma}
\applystyletotheorem{theorem}
\applystyletotheorem{corollary}
\applystyletotheorem{definition}
\applystyletotheorem{notation}
\applystyletotheorem{example}
\applystyletotheorem{cexample}
\applystyletotheorem{application}
\applystyletotheorem{remark}
%\applystyletotheorem{proof}
\applystyletotheorem{algorithm}
\applystyletotheorem{exercice}

% Environnements :

\NewEnviron{whitetabularx}[1]{%
  \renewcommand{\arraystretch}{2.5}
  \colorbox{white}{%
    \begin{tabularx}{\textwidth}{#1}%
      \BODY%
    \end{tabularx}%
  }%
}

% Maths :

\DeclareFontEncoding{FMS}{}{}
\DeclareFontSubstitution{FMS}{futm}{m}{n}
\DeclareFontEncoding{FMX}{}{}
\DeclareFontSubstitution{FMX}{futm}{m}{n}
\DeclareSymbolFont{fouriersymbols}{FMS}{futm}{m}{n}
\DeclareSymbolFont{fourierlargesymbols}{FMX}{futm}{m}{n}
\DeclareMathDelimiter{\VERT}{\mathord}{fouriersymbols}{152}{fourierlargesymbols}{147}

% Code :

\definecolor{greencode}{rgb}{0,0.6,0}
\definecolor{graycode}{rgb}{0.5,0.5,0.5}
\definecolor{mauvecode}{rgb}{0.58,0,0.82}
\definecolor{bluecode}{HTML}{1976d2}
\lstset{
  basicstyle=\footnotesize\ttfamily,
  breakatwhitespace=false,
  breaklines=true,
  %captionpos=b,
  commentstyle=\color{greencode},
  deletekeywords={...},
  escapeinside={\%*}{*)},
  extendedchars=true,
  frame=none,
  keepspaces=true,
  keywordstyle=\color{bluecode},
  language=Python,
  otherkeywords={*,...},
  numbers=left,
  numbersep=5pt,
  numberstyle=\tiny\color{graycode},
  rulecolor=\color{black},
  showspaces=false,
  showstringspaces=false,
  showtabs=false,
  stepnumber=2,
  stringstyle=\color{mauvecode},
  tabsize=2,
  %texcl=true,
  xleftmargin=10pt,
  %title=\lstname
}

\newcommand{\codedirectory}{}
\newcommand{\inputalgorithm}[1]{%
  \begin{algorithm}%
    \strut%
    \lstinputlisting{\codedirectory#1}%
  \end{algorithm}%
}



\everymath{\displaystyle}
\begin{document}
  %<*content>
  \lesson{algebra}{1}{Fonctions numériques. Rappels et compléments}


\subsection{Limites}


\textsl{Lorsque nous écrivons $ \infty $ cela signifie que c'est valable pour $ +\infty$ comme pour $-\infty $ }

 Il existe quatre cas d'indétermination dans les opérations sur les limites:
\[ <<\pinf \minf>>;\quad <<\frac{\infty}{\infty}>>;\quad << \frac{0}{0}>>;\quad <<0\times \infty >> \]
\subsubsection*{Limites usuelles}
 $ n\in\Nne $
\begin{itemize}
\item $ \displaystyle\lim_{x \to \pinf}x^{n}= \pinf$ 
\item $\displaystyle\lim_{x \to \minf}x^{n}= \begin{cases}
\pinf & \text{si $\; n\;$ pair} \\
\minf & \text{si $\;n \;$ impair}
\end{cases}$ 
\item $\displaystyle \lim_{x \to \pinf}\sqrt{x}=\pinf$ 
\item $\displaystyle \lim_{x \to \infty} \frac{1}{x^{n}}=0$ 
\item $\displaystyle \lim_{x \to 0}\frac{\sin x}{x}=1$  $\qquad \displaystyle \lim_{x \to 0}\frac{\tan x}{x}=1$ 
\item $ \displaystyle\lim_{x \to 0}\frac{1-\cos x}{x^{2}}= \frac{1}{2}$ $\qquad  \displaystyle\lim_{x \to 0}\frac{1-\cos x}{x}= 0$ 

\end{itemize}
\begin{remark}
Les fonctions cosinus et sinus n'ont pas de limites à l'infini.
\end{remark}
\subsubsection*{Limite de la composée de deux fonctions}
\begin{property}
 Soient $f $  et $g$ deux fonctions, $a $, $b$  et  $c$ trois réels pouvant éventuellement être $ +\infty $ ou $ -\infty $. 
\[\text{Si}\displaystyle \lim_{x \to a} f(x) = b \quad \text{et} \quad \displaystyle \lim_{x \to b} g(x) = c \quad \text{alors} \quad \displaystyle \lim_{x \to a} g[f(x)] = c \]
  \end{property}
\begin{example}
 Calculons $\displaystyle \lim_{x \to +\infty} \cos\paren{\frac{x+1}{x^{2}-2}}$ 
 
On a: $\; \displaystyle \lim_{x \to +\infty} \frac{x+1}{x^{2}-2} = 0 $ et $ \displaystyle \lim_{x \to 0} \cos x = \cos0=1 $ donc  par composée $  \displaystyle\lim_{x \to +\infty}\cos\paren{\frac{x+1}{x^{2}-2}}=1 $

\end{example}  

 \subsubsection*{Comparaisons de limites}
\begin{theorem}
Soient $f $, $g$ et $h$ trois fonctions et $ l\in\Rr$ et  $ a= +\infty $ ou $ a=-\infty $ ou $ a\in\Rr $.



$\begin{array}{|c|c|c|}
\hline
\text{Hypothèse 1}  & \text{Hypothèse 2} & \text{Conclusion}\\
\hline
 f\leq g   &  \displaystyle \lim_{x \to a}f(x)=+\infty  & \displaystyle \lim_{x \to a}g(x)=+\infty  \\
 \hline
  f\leq g   & \displaystyle \lim_{x \to a}g(x)=-\infty  &  \displaystyle \lim_{x \to a}f(x)=-\infty     \\
\hline
 \abs{f(x)-l }\leq g(x)  & \displaystyle \lim_{x \to a}g(x)=0 &  \displaystyle \lim_{x \to a}f(x)=l   \\
\hline
 g \leq f \leq h  &  \displaystyle \lim_{x \to a}g(x)=l \;\text{et} \; \displaystyle \lim_{x \to a}h(x)=l  &  \displaystyle \lim_{x \to a}f(x)=l     \\
\hline
\end{array}$
\end{theorem}

\begin{remark}
Le dernier théorème est parfois appelée << le théorème des gendarmes >>.
 \end{remark}

\begin{example}
\begin{itemize}
\item  Soit $ f(x)= x+ 3\cos x $.\\
Pour tout $x\in\Rr$:  $\quad x-3\leq f(x)\leq x+3 $ \\
$ \centerdot $  On a : $ x-3\leq f(x)$ et $\displaystyle \lim_{x \to +\infty}x-3=+\infty  $ donc $ \displaystyle \lim_{x \to +\infty}f(x)=+\infty $\\
$ \centerdot $ On a :  $  f(x)\leq x+3$ et $\displaystyle \lim_{x \to -\infty}x-3=-\infty  $ donc $ \displaystyle \lim_{x \to -\infty}f(x)=-\infty $ \\

\item  Calculons $\displaystyle\lim_{x \to +\infty}\frac{\sin x}{x} $.\\

Pour tout $ x\geq 1 $: $\quad  -1\leq\sin x \leq 1 $ \\
En multipliant par $ \dfrac{1}{x} $: on a $ -\dfrac{1}{x}\leq \dfrac{\sin x}{x} \leq\dfrac{1}{x} $\\
Or $\displaystyle\lim_{x \to +\infty}(-\dfrac{1}{x})= \displaystyle\lim_{x \to +\infty}\dfrac{1}{x}=0$ donc $\displaystyle\lim_{x \to +\infty}\dfrac{\sin x}{x}=0 $
\end{itemize}

\end{example}

\subsubsection*{Limites et nombre dérivé}

\begin{theorem}
Soit $ f $ une fonction dérivable.

\[ \text{Si}\; \displaystyle \lim_{x \to a}f^{'}(x)=l,\;  ( l \;\text{réel fini ou pas) alors \;} \displaystyle\lim_{x \to a} \dfrac{f(x)-f(a)}{x-a}=l\]
\end{theorem}

\begin{example}
Calculons $\displaystyle\lim_{x \to 0}\dfrac{\sin x}{x} $

Posons $ f(x)=\sin x $\\
On a $ f(0)=0 $ et  $ f^{'}(x)=\cos x$
 \[\text{Donc \;} \displaystyle \lim_{x \to 0} \frac{f(x)-f(0)}{x-0}=\displaystyle \lim_{x \to 0}\frac{\sin x}{x}=\displaystyle\lim_{x \to 0}f^{'}(x)=\displaystyle\lim_{x \to 0}\cos x =1\]
\end{example}

\subsection{Branches infinies d'une courbe}
Soit une fonction numérique $ f $ et $ \mathcal{C} $ sa courbe représentative dans un repère orthogonal du plan.
\subsubsection{Asymptotes}
\subsubsection*{Asymptote verticale}
Elle traduit, graphiquement, le fait que la fonction $f$ admet une limite infinie en un réel $a$.


\begin{definition}
Si $\; \displaystyle\lim_{x \to a^{+}}f(x)=\infty$  ou 
$\displaystyle\lim_{x \to a^{-}}f(x)=\infty $  ou 
$\displaystyle\lim_{x \to a}f(x)=\infty $ alors  la droite  d'équation (D) $\; x=a$ est une asymptote verticale à la courbe $\mathcal{C} $.
\end{definition}


\begin{tikzpicture}[>=stealth', scale=0.5]
\clip (-1,-3) rectangle (3,3);
\draw[->,thick] (-1,0) -- (3,0);
\draw[->,thick] (0,-3) -- (0,3);
\foreach\x in {1,2}
{
\draw[thick] (\x,0.1) -- (\x,-0.1) node[below] {\x};
}
\foreach\y in {1}
{
\draw[thick] (0.1,\y) -- (-0.1,\y) node[left] {\y};
}

\draw[thick,blue!50!black] plot[domain=0.26:2,samples=100] (\x,{(2*\x+1)/(4*\x-1)}) node[above left] {$\mathscr{C}$};
\draw[red,thick] (0.25,3) -- (0.25,-3) node[above right] {(D)};
\end{tikzpicture}

$\displaystyle \lim_{ x \to a^{+}}f(x)=\pinf $


\begin{tikzpicture}[>=stealth',scale=0.5]
\clip (-3,-3) rectangle (3,3);
\draw[->,thick] (-3,0) -- (3,0);
\draw[->,thick] (0,-3) -- (0,3);
\foreach\x in {1,2}
{
\draw[thick] (\x,0.1) -- (\x,-0.1) node[below] {\x};
}
\foreach\y in {-1,1}
{
\draw[thick] (0.1,\y) -- (-0.1,\y) node[left] {\y};
}
\draw[thick,blue!50!black] plot[domain=-3:0.24,samples=100] (\x,{(2*\x+1)/(4*\x-1)}) node[above left] {$\mathscr{C}$};

\draw[red,thick] (0.25,3) -- (0.25,-3) node[above right] {(D)};
\end{tikzpicture}

$ \displaystyle\lim_{ x \to a^{-}}f(x)=\minf $




\subsubsection*{Asymptote horizontale}
Elle traduit, graphiquement, le fait que la fonction $f$ admet une limite finie à l'infini.



\begin{definition}
Si $\;\displaystyle \lim_{x \to \infty}f(x)=b \;$ (réel) alors  la droite  $\quad y = b \;$  est une  asymptote horizontale à  la courbe $\mathcal{C} $   en $\infty$.
\end{definition}

\begin{tikzpicture}[>=stealth', scale=0.5]
\draw[->,thick] (-7,0) -- (7,0);
\draw[->,thick] (0,-2) -- (0,3);
\foreach\x in {-2,1}
{
\draw[thick] (\x,0.1) -- (\x,-0.1) node[below] {\x};
}
\node (L) at (-0.2,2) {L};
\foreach\y in {1}
{
\draw[thick] (0.1,\y) -- (-0.1,\y) node[left] {\y};
}
\draw[thick,blue!50!black] plot[domain=-7:7,samples=100] (\x,{(2*\x*\x+\x-1)/(\x*\x+\x+1)}) node[above left] {$\mathscr{C}$};
\draw[red,thick] (7,2) -- (-7,2) node[below right] {D};
\end{tikzpicture}
\begin{example}
Pour la fonction $f : x\mapsto 3+ \dfrac{5}{x-1} $ 
\begin{itemize}
\item la droite d'équation   $y = 3 $ est asymptote  horizontale  
\item la droite d'équation $x = 1$ est asymptote verticale à la courbe de $f.$
\end{itemize}
\end{example}

\subsubsection*{Asymptote oblique}
\begin{definition}
Soit $f$ une fonction et $ \Delta $ la droite d'équation  $y=ax+b$.

Si $\displaystyle \lim_{x \to \infty}\left(f(x)-(ax+b)\right)=0$  alors  la droite  d'équation  $y=ax+b$  est une  asymptote oblique  à la courbe de $f$ en  $\infty$.
\end{definition}

\begin{tikzpicture}[>=stealth', scale=0.5]
\clip (-7,-4) rectangle (7,5);
\draw[->,thick] (-7,0) -- (7,0);
\draw[->,thick] (0,-4) -- (0,5);
\foreach\x in {1}
{
\draw[thick] (\x,0.1) -- (\x,-0.1) node[below] {\x};
}
\foreach\y in {1}
{
\draw[thick] (0.1,\y) -- (-0.1,\y) node[left] {\y};
}
%\draw[thick,blue!50!black] plot[domain=-7:0.24,samples=100] (\x,{(2*\x+1)/(4*\x-1)}) node[above left] {$\mathscr{C}$};
\draw[thick,blue!50!black] plot[domain=-7:7,samples=100] (\x,{\x+0.3/(\x-1)}) node[above left] {$\mathscr{C}$};
\draw[red,thick] (5,5) -- (-3,-3) node[above left] {D};
\end{tikzpicture}
\begin{example}
Pour la fonction $f : x\mapsto x+ \dfrac{2}{x-1} $ dont la courbe est représentée ci dessous,
la droite d'équation   $y = x $ est asymptote oblique à la courbe.

\end{example}

\begin{remark} 
Si $ f $ s'écrit sous la forme $ f(x)= ax+b + g(x) $ et si $ \displaystyle \lim_{x \to \infty}g(x)=l\; (\text{réel} )$   alors la droite $ y=ax+b+l $ est une asymptote à  $ \mathcal{C} $ en $ \infty. $
\end{remark}
\begin{example}
Pour la fonction $f : x\mapsto 2x+5- \dfrac{2x}{x-1} $ 
la droite d'équation   $y = 2x+3 $ est asymptote oblique à sa courbe en $ \infty$ car  $\displaystyle \lim_{x \to \infty}\dfrac{2x}{x-1}=2$.
\end{example}
\subsubsection*{Position relative d'une courbe et son asymptote}
Pour étudier la position relative de la  courbe $\mathcal{C} $ d'une fonction $ f $ par rapport à son asymptote $ \Delta : y=ax+b $, on étudie le signe de la différence $ f(x)-ax-b$.
\begin{itemize}
\item  Si $ f(x)-ax-b > 0$ alors  $ \mathcal{C}  $ est  située  au-dessus de la courbe de $ \Delta $ 
\item  Si $ f(x)-ax-b< 0$ alors la courbe de $\mathcal{C}$ est  située  en-dessous de la courbe de $\Delta$ 
\item  Si $ f(x)-ax-b= 0$ alors la courbe de $\mathcal{C}$ et  $\Delta$  sont sécantes.
\end{itemize}
On tiendra compte de l'ensemble sur lequel on doit étudier la position relative des deux courbes.
\subsubsection{Recherche de branches infinies}
Lorsque  $\displaystyle \lim_{x \to \infty}f(x)=\infty  $, la courbe $ \mathcal{C} $ présente une branche infinie qu'il faut étudier.
\begin{itemize}
\item    Si $\displaystyle \lim_{x \to \infty}\frac{f(x)}{x}=0  $  alors la courbe $ \mathcal{C} $ présente une branche parabolique dans la direction  de  l'axe des abscisses.

\item    Si $\displaystyle \lim_{x \to \infty}\frac{f(x)}{x}=\infty$  alors la courbe $ \mathcal{C} $ présente une branche parabolique dans la direction  de  l'axe des ordonnées.
\item  Si  $\displaystyle\lim_{x \to \infty}\frac{f(x)}{x}=a  $  réel  non nul alors on calcule $\displaystyle\lim_{x \to \infty}(f(x)-ax )  $ 

 $ \bullet $ Si  $\displaystyle \lim_{x \to \infty}(f(x)-ax) = b \; $  (réel) alors la droite $(D)$ d'équation : $y = a x  + b $ est asymptote à la courbe $ \mathcal{C} $ .

 $ \bullet $ Si $\displaystyle \lim_{x \to \infty}(f(x)-ax) = \infty $     alors la courbe admet une branche parabolique de direction asymptotique la droite d'équation $ y =  a x $.
\end{itemize}


\begin{exercice}
On considère la fonction $ f $ définie par :
$ f (x)=\left\{\begin{array}{l} \sqrt{x +4}\quad \text{si} \quad x\geq 2 \\ x+3-\dfrac{2}{x-1}\quad \text{si}\quad x< 2  \end{array} \right.$

\begin{enumerate}
\item Déterminer  les limites de $f$ aux bornes de $ D_{f} $.
\item Etudier la nature des branches infinies de la courbe $ \mathcal{C} $ de $f$.
\item Etudier la position relative de $ \mathcal{C} $ par rapport à son asymptote oblique.
\end{enumerate}
\end{exercice}
\begin{proof}
\begin{enumerate}
\item $ f(x) $ existe ssi $ \begin{cases} x+4 \geq 0 \\ x\geq 2\end{cases}$  ou $ \begin{cases} x-1 \neq 0 \\ x< 2\end{cases}$

 $ f(x) $ existe ssi $ \begin{cases} x \geq -4 \\ x\geq 2\end{cases}$  ou $ \begin{cases} x\neq -1 \\ x< 2\end{cases}$
 
 $ f(x) $ existe ssi  $ x \geq 2 $\quad  ou $ x\in \intoo{\minf}{-1}\cup \intoo{-1}{2} $ \\ donc $ f(x) $ existe ssi $ x\in\intoo{\minf}{-1}\cup \intoo{-1}{2} \cup\intfo{2}{\pinf}$ 

 D'où $ D_{f}=\intoo{\minf}{-1}\cup \intoo{-1}{\pinf}$
 
 Limites aux bornes de $D_{f}  $
 $\displaystyle\lim_{x \to \pinf }x+4= \pinf $  par composée $\displaystyle\lim_{x \to \pinf}\sqrt{x+4}=\pinf  $ d'où $\displaystyle\lim_{x \to \pinf}f(x)=\pinf  $\\
 $\displaystyle\lim_{x \to \minf}x+3-\frac{2}{x-1}=\displaystyle \lim_{x \to \minf}x+3-\displaystyle\lim_{x \to \minf}\frac{2}{x-1}=\minf  $ d'où $\displaystyle\lim_{x \to \pinf}f(x)=\minf  $\\
 L'étude de la limite en $ 1 $ se fait uniquement sur la restriction $ x \mapsto x+3-\frac{2}{x-1} $\\
$ \begin{cases}\displaystyle\lim_{x \to 1^{+}}x+3=4  \\ \displaystyle\lim_{x \to 1^{+}}-\frac{2}{x-1}=\minf\end{cases}$  donc $\displaystyle\lim_{x \to 1^{+}}f(x)=\minf  $\\
$ \begin{cases}\displaystyle\lim_{x \to 1^{-}}x+3=4  \\ \displaystyle\lim_{x \to 1^{-}}-\frac{2}{x-1}=\pinf\end{cases}$  donc $\displaystyle\lim_{x \to 1^{-}}f(x)=\pinf  $
\item Puisque $\displaystyle\lim_{x \to 1^{+}}f(x)=\minf  $ et $\displaystyle\lim_{x \to 1^{-}}f(x)=\pinf  $ donc la droite d'équation $ x=1 $ est une asymptote verticale à la courbe de $ f. $\\
Puisque la restriction de $ f $ sur $ \intoo{\minf}{2} $ s'écrit sous la forme $ x \mapsto x+3-\frac{2}{x-1} $ et que $ \displaystyle\lim_{x \to \minf}\frac{2}{x-1}= 0 $ donc la droite $ \Delta $ d'équation $ y=x+3 $ est une asymptote oblique à la courbe de $ f. $\\
D'autre part  $\displaystyle\lim_{x \to \pinf}f(x)=\pinf  $ doc $ \mathcal{C_{f}} $ présente une branche infinie en $ \pinf. $\\ Calculons $\displaystyle \lim_{x \to \pinf}\frac{f(x)}{x} $\\
$\displaystyle \lim_{x \to \pinf}\frac{\sqrt{x+3}}{x} = \displaystyle\lim_{x \to \pinf} \frac{x+3}{x\sqrt{x+3}}=\displaystyle\lim_{x \to \pinf} \frac{x+3}{x} \times \displaystyle\lim_{x \to \pinf}\frac{1}{\sqrt{x+3}}= 1 \times 0 =0 $ d'où $ \mathcal{C_{f}} $ admet un branche  parabolique d'axe (Ox).
\item Etudions la position relative de $ \Delta $ et $ \mathcal{C_{f}} $.\\
Pour cela étudions le signe de $ f(x)-(x+3) = -\dfrac{2}{x-1} $ pour $ x < 2 $  

  \[\begin{array}{|c|ccccc|}
\hline
x & \minf & & 1 & & 2\\ \hline
\text{signe de }-\frac{2}{x-1}  & & +\qquad &|  & \quad - & \\
\hline
\end{array}\]
 
  Sur $ \intoo{\minf}{1} $ $-\dfrac{2}{x-1} > 0  $ donc $ \mathcal{C_{f}} $ est au dessus de $ \Delta. $\\
   Sur $ \intoo{\minf}{1} $ $-\dfrac{2}{x-1} < 0  $ donc $ \mathcal{C_{f}} $ est au dessous de $ \Delta. $
\end{enumerate}
\end{proof}
\subsection{Continuité}
\subsubsection*{Continuité en un réel}

\begin{definition}
Une fonction $ f $ est continue en un réel $ a $ ssi $ a\in D_{f} $ et $\displaystyle \lim_{x \to a}f(x)=f(a) $.
\end{definition}

\textbf{\color{blue}Illustration graphique}



\begin{tikzpicture}[scale=0.375]
    % Axes
    \draw[->] (-5,0) -- (6,0);
    \draw[->] (0,-2) -- (0,8);
    
    % Graduation marks
    \foreach \x in {-4,-3,-2,-1,1,2,3,4,5} {
        \draw (\x,0.1) -- (\x,-0.1);
    }
    \foreach \y in {-1,1,2,3,4,5,6,7} {
        \draw (0.1,\y) -- (-0.1,\y);
    }
    
    % Origin
    \node[below left] at (0,0) {$O$};
    
    % Unit vectors
    \draw[thick,->] (0,0) -- (1,0);
    \node[below] at (0.5,0) {\small $\vec{\imath}$};
    \draw[thick,->] (0,0) -- (0,1);
    \node[left] at (0,0.5) {\small $\vec{\jmath}$};
    
    % Function curves
    % Left part: f(x) = sqrt(-x) for x <= 0
    \draw[red,thick,domain=-4:0,samples=100] plot (\x,{sqrt(-\x)});
    
    % Right part: f(x) = (x-1)^2 + 2 for x > 0
    \draw[red,thick,domain=0.05:3,samples=100] plot (\x,{(\x-1)^2 + 2});
    
    % Dashed lines for point (a, f(a))
    \draw[dashed] (2.8,0) -- (2.8,5.24);
    \draw[dashed] (0,5.24) -- (2.8,5.24);
    
    % Labels
    \node[below] at (2.8,0) {$a$};
    \node[left] at (0,5.24) {$f(a)$};
    
    % Point at discontinuity
    \fill[red] (0,3) circle (0.08);
    \draw[red,fill=white] (0,0) circle (0.08);
    
\end{tikzpicture}



{\color{red}$\centerdot$} $f$ continue en $a$ signifie: dans le tracé de la courbe on << \textbf{\color{magenta} ne lève pas }>> la main quand on passe au point d'abscisse $a.$ \\
La courbe n'y présente aucun saut, aucun trou, aucune asymptote verticale.\\

{\color{red}$\centerdot$} Par exemple ici la fonction n'est pas continue en $0.$

\begin{example}
Soit $ f (x)=\left\{\begin{array}{l} \dfrac{\sqrt{x}-1}{x-1}\quad \textrm{si} \quad x\neq 1 \\ \frac{1}{2}\quad \quad\textrm{si}\quad x= 1  \end{array} \right.$


Etudions la continuité   de $ f $ en 1.\\
 Pour $x\neq 1  $, $ f(x) $ existe si et seulement, si $\; x\geq 0 \quad$ et $\; x-1 \neq 0 $
 
si et seulement, si \quad $x\geq 0$ \quad   et $x\neq 1$

si et seulement, si $x\in \intfo{0}{1}\cup\intoo{1}{\pinf}$

Or $ f(1)= \frac{1}{2}$  d'où $ f(x) $ existe si et seulement, si  $x\in \intfo{0}{\pinf}  $ \\
$ \displaystyle\lim_{x \to 1}f(x)=\displaystyle\lim_{x \to 1}\frac{\sqrt{x}-1}{x-1}= \displaystyle\lim_{x \to 1}\frac{1}{\sqrt{x}+1}= \frac{1}{2}$ \\donc $\displaystyle \lim_{x \to 1}f(x)=f(1)=\frac{1}{2}\quad$ d'où $\; f $ est continue en $ 1. $
\end{example}

\subsubsection*{ Continuité à droite - continuité à gauche}
\begin{property}
$ f $ est continue en $ a $ si et seulement, si  $\displaystyle \lim_{x \to a^{+}}f(x)=\displaystyle\lim_{x \to a^{-}}f(x)=f(a) $. 
\end{property}
\subsubsection*{Prolongement par continuité}
\begin{definition}
Soit $ f $ une fonction  \textbf{non}  définie en $ a $ et $ l $ un nombre réel tel que $ \displaystyle\lim_{x \to a}f(x)=l. $\\
On appelle \textbf{prolongement par continuité } de $ f $ en $ a $, la fonction $ g $ définie par :
\[ g (x)=\left\{\begin{array}{l} f(x)\quad \textrm{si} \quad x\neq a \\ l                      \quad \quad \textrm{si}\quad x= a  \end{array} \right.\]
\textbf{NB:} La fonction $ g $ est définie et continue en $ a $.
\end{definition}
\begin{example}
Montrons que la fonction $ f: x \mapsto \dfrac{x^{2}-x-2}{x-2} $   est prolongeable par continuité en 2 et trouvons son prolongement par continuité.
\end{example}

\textsl{Réponse:}\\
$ f(x) $ existe si et seulement, si $ x\neq 2 $.\\  $\displaystyle\lim_{x \to 2 }\frac{x^{2}-x-2}{x-2}=\displaystyle\lim_{x \to 2 }\frac{(x-2)(x+1)}{x-2}= \displaystyle\lim_{x \to 2 }(x+1)=3 $   finie donc $ f $ est prolongeable par continuité en $ 2. $\\
Son prolongement par continuité en $ 2$ est la fonction $ g $ définie par : 
\[ g (x)=\left\{\begin{array}{l} \dfrac{x^{2}-x-2}{x-2} \quad \textrm{si} \quad x\neq 2 \\ 3\quad \quad \qquad \quad  \textrm{si}\quad x= 2  \end{array} \right.\]

\subsection{Dérivabilité}
\subsubsection*{Dérivabilité en un réel}
\begin{definition}
Soit $ f $ une fonction définie sur un intervalle $ I$ et $a\in I $.\\
$ f $ est dérivable en $ a $ s'il existe un nombre réel $ l $ tel que $ \displaystyle\lim_{x \to a} \dfrac{f(x)-f(a)}{x-a}=l $ \\
$ l $ est le  \textbf{nombre dérivé} de $ f $ en $ a. $ On le note $f^{'}(a)$.
\end{definition}

\textbf{Autre formulation de la définition}\\
On fait le changement de variable suivant $ h=x-a $ \\
$ f $ est dérivable en $ a $ s'il existe un nombre réel $ l $ tel que $\displaystyle \lim_{h \to 0} \frac{f(a+h)-f(a)}{h}=l $ 

\begin{example}
Soit $ f (x)=\left\{\begin{array}{l} \dfrac{\sqrt{x}-1}{x-1}\quad \textrm{si} \quad x\neq 1 \\ \frac{1}{2}\quad \textrm{si}\quad x= 1  \end{array} \right.$\\
Etudions  la dérivabilité  de $ f $ en 1.
\end{example}
\textsl{Réponse:}\\
On avait trouvé que $D_{f}=\intfo{0}{\pinf}  $ \\
$\displaystyle \lim_{x \to 1 } \dfrac{f(x)-f(1)}{x-1}=\displaystyle \lim_{x \to 1 }\dfrac{\dfrac{\sqrt{x}-1}{x-1}-\frac{1}{2}}{x-1}= \lim_{x \to 1 }\dfrac{2\sqrt{x}-(x+1)}{2(x-1)^{2}}=\lim_{x \to 1 } \dfrac{-x^{2}+2x-1}{2(x-1)^{2}(2\sqrt{x}+x+1)}=\displaystyle \lim_{x \to 1 }\dfrac{-1}{2(2\sqrt{x}+x+1}=-\dfrac{1}{8}$  \\ Donc $ f $ est dérivable en $ 1 $ et de nombre dérivé $ f'(1)= -\dfrac{1}{8}$.
\subsubsection*{Propriété}
\begin{property}
Si $ f $ est dérivable en a, alors $ f $ est continue en a
\end{property}.
\begin{cexample} 
La réciproque de cette propriété est fausse.\\
La fonction $ x\mapsto \abs{x} $ est continue en $ 0 $ mais elle n'est pas dérivable en $ 0 $.
\end{cexample}

  \subsubsection*{Propriété : Dérivabilité à droite - dérivabilité à gauche}
  \begin{property}
$ f $ est dérivable en $ a $ si et seulement, si :\[  \displaystyle\lim_{x \to a^{+}} \dfrac{f(x)-f(a)}{x-a}=\displaystyle\lim_{x \to a^{-}} \dfrac{f(x)-f(a)}{x-a}=l \quad \text{réel}\]
\[f^{'}_{d}(a)=f^{'}_{g}(a)\]
\[\textrm{Le nombre dérivé de} \  f \ \textrm{ à droite en $ a$ = Le nombre dérivé de}\  f \ \textrm{à gauche en $a $ }\]
\end{property}
\begin{notation}
Les notations \;   $f^{'}_{d}(a)$ et $f^{'}_{g}(a)$ s'utilisent que lorsque la limite du taux de variation est un réel.
\end{notation}
\subsubsection*{Cas de non dérivabilité}
$ \centerdot $ Si $\displaystyle \lim_{x \to a } \frac{f(x)-f(a)}{x-a}=\pinf$ ou $\minf $ alors $ f $ n'est pas dérivable en $ a $.\\

$ \centerdot $ Si $ f^{'}_{d}(a) \neq f^{'}_{g}(a) $ alors $ f $ n'est pas dérivable en $ a $.
\subsubsection*{Interprétation graphique de la dérivabilité}
\begin{enumerate}
\item Si $ f $ est dérivable en $ a $ alors sa courbe $ \mathcal{C} $ admet au point d'abscisse  $ a $  c-à-d le point $A(a,f(a))$ une \textbf{ tangente} de coefficient directeur ( ou pente) $  f^{'}(a)$ d'équation:  
$$ y=f^{\prime}(a)(x-a)+ f(a) $$

\begin{tikzpicture}[xscale=1,yscale=0.5]
    % Axes
    \draw[->] (-1,0) -- (7,0);
    \draw[->] (0,-1) -- (0,8);
    
    % Labels des axes
    \node[below] at (6.8,-0.3) {$x$};
    \node[left] at (-0.3,8) {$y$};
    
    % Courbe f(x) = (x-2)²/4 + 2
    \draw[red,thick,domain=0.2:7,samples=100,smooth] 
        plot (\x,{(\x-2)^2/4 + 2});
    \node at (7.5,8.5) {$C_f$};
    
    % Tangente à la courbe en A : y = 0.5(x-3) + 2.25
    \draw[blue,thick,domain=1:7,samples=50] 
        plot (\x,{0.5*(\x-3) + 2.25});
    \node at (8,4.25) {tangente à $C_f$ en $A$};
    
    % Point A
    \fill[black] (3,2.25) circle (0.08);
    \node[above right] at (2.8,2.7) {$A$};
    
    % Lignes pointillées pour les coordonnées de A
    \draw[dashed] (3,2.25) -- (3,0);
    \draw[dashed] (3,2.25) -- (0,2.25);
    
    % Labels des coordonnées
    \node[below] at (3,-0.5) {$a$};
    \node[left] at (-0.5,2.25) {$f(a)$};
    
    % Construction géométrique de la dérivée
    % Flèche horizontale (+1)
    \draw[dashed,->] (3,2.25) -- (6,2.25);
    \node[below] at (4.5,1.9) {+1};
    
    % Flèche verticale (f'(a))
    \draw[dashed,->] (6,2.25) -- (6,3.75);
    \node[right] at (6.5,3) {$f'(a)$};
    
    % Point sur la tangente
    \fill[blue] (6,3.75) circle (0.06);
\end{tikzpicture}


\begin{methode}[Méthode pour construire la tangente]
 On part de $A$, on << avance >> vers la droite du dénominateur(positif) de la pente $f'(a)$  et on << monte >> (ou << descend >>) du numérateur de  $f'(a)$: ce qui donne un deuxième point que l'on relie à $A.$.

\end{methode}

\begin{remark}
 $  f'(a)=0\; $  si et seulement si $\; \mathcal{C} \;$ admet au point d'abscisse  $ a $ une tangente horizontale d'équation $ y=f(a). $ \\ Dans ce cas, le point $A(a,f(a))$  est  soit un \textbf{extremum} ( maximum ou minimum) soit un \textbf{point d'inflexion}.
\end{remark}

\begin{center}

% Premier graphique : Minimum (parabole vers le haut)
\begin{tikzpicture}[scale=0.375]
    % Axes
    \draw[->] (-2,0) -- (7,0);
    \draw[->] (0,-1) -- (0,4);
    
    % Graduations
    \foreach \x in {-1,1,2,3,4,5,6} {
        \draw (\x,0.1) -- (\x,-0.1);
    }
    \foreach \y in {1,2,3} {
        \draw (0.1,\y) -- (-0.1,\y);
    }
    
    % Origine et vecteurs unitaires
    \node[below left] at (0,0) {$O$};
    \draw[thick,->] (0,0) -- (1,0);
    \node[below] at (0.5,0) {\small $\vec{\imath}$};
    \draw[thick,->] (0,0) -- (0,1);
    \node[left] at (0,0.5) {\small $\vec{\jmath}$};
    
    % Courbe parabole y = 2 + (x-2)²
    \draw[red,thick,domain=0.15:3.5,samples=100,smooth] 
        plot (\x,{2 + (\x-2)^2});
    
    % Point A
    \fill[black] (2,2) circle (0.1);
    \node[above] at (2,2.5) {$A$};
    
    % Ligne horizontale verte
    \draw[green,thick,<->] (1,2) -- (3.5,2);
\end{tikzpicture}

\hspace{1cm}

% Deuxième graphique : Maximum (parabole vers le bas)
\begin{tikzpicture}[scale=0.375]
    % Axes
    \draw[->] (-2,0) -- (7,0);
    \draw[->] (0,-1) -- (0,4);
    
    % Graduations
    \foreach \x in {-1,1,2,3,4,5,6} {
        \draw (\x,0.1) -- (\x,-0.1);
    }
    \foreach \y in {1,2,3} {
        \draw (0.1,\y) -- (-0.1,\y);
    }
    
    % Origine et vecteurs unitaires
    \node[below left] at (0,0) {$O$};
    \draw[thick,->] (0,0) -- (1,0);
    \node[below] at (0.5,0) {\small $\vec{\imath}$};
    \draw[thick,->] (0,0) -- (0,1);
    \node[left] at (0,0.5) {\small $\vec{\jmath}$};
    
    % Courbe parabole inversée y = 2 - (x-2)²
    \draw[red,thick,domain=-0.25:3.5,samples=100,smooth] 
        plot (\x,{2 - (\x-2)^2});
    
    % Point A
    \fill[black] (2,2) circle (0.1);
    \node[below] at (2,1.8) {$B$};
    
    % Ligne horizontale verte
    \draw[green,thick,<->] (1,2) -- (3.5,2);
\end{tikzpicture}

\vspace{0.5cm}

% Troisième graphique : Point d'inflexion (fonction cubique)
\begin{tikzpicture}[scale=0.375]
    % Axes
    \draw[->] (-2,0) -- (7,0);
    \draw[->] (0,-1) -- (0,4);
    
    % Graduations
    \foreach \x in {-1,1,2,3,4,5,6} {
        \draw (\x,0.1) -- (\x,-0.1);
    }
    \foreach \y in {1,2,3} {
        \draw (0.1,\y) -- (-0.1,\y);
    }
    
    % Origine et vecteurs unitaires
    \node[below left] at (0,0) {$O$};
    \draw[thick,->] (0,0) -- (1,0);
    \node[below] at (0.5,0) {\small $\vec{\imath}$};
    \draw[thick,->] (0,0) -- (0,1);
    \node[left] at (0,0.5) {\small $\vec{\jmath}$};
    
    % Courbe cubique y = 2 + (x-2)³
    \draw[red,thick,domain=0.5:3.2,samples=100,smooth] 
        plot (\x,{2 + (\x-2)^3});
    
    % Point A
    \fill[black] (2,2) circle (0.1);
    \node[above] at (2,2.5) {$C$};
    
    % Ligne horizontale verte
    \draw[green,thick,<->] (1,2) -- (3.5,2);
\end{tikzpicture}

\vspace{0.5cm}

$A$ est un minimum, $\quad B$ est un maximum,  $\quad C$ est un point d'inflexion

\end{center}
 \item Si $ f $ est dérivable  à droite et à gauche  de $ a $ telle que $ f^{'}_{d}(a)\neq f'_{g}(a) $ alors $ \mathcal{C} $ admet au  point $A(a,f(a))$ deux demi-tangentes de pentes respectives  $ f'_{d}(a)$ et $ f'_{g}(a) $: le point $A$ est un point \textbf{anguleux}.
 \item Détaillons les cas d'une limite infinie.

\medskip
\[
\begin{array}{|c|c|}
\hline
\begin{array}{c}
\bullet\; \displaystyle \lim_{x \to a^+} \dfrac{f(x)-f(a)}{x-a} = +\infty \\
\text{La courbe de } f \text{ admet au point } A(a, f(a)) \\
\text{une demi-tangente verticale dirigée vers le haut.}
\end{array}
&
\begin{array}{c}
\bullet\; \displaystyle \lim_{x \to a^-} \dfrac{f(x)-f(a)}{x-a} = +\infty \\
\text{La courbe de } f \text{ admet au point } \\A(a, f(a)) 
\text{une demi-tangente verticale }\\ \text{dirigée vers le bas.}
\end{array}
\\
\hline
\begin{array}{c}
\bullet\; \displaystyle \lim_{x \to a^+} \dfrac{f(x)-f(a)}{x-a} = -\infty \\
\text{La courbe de } f \text{ admet au point } A(a, f(a)) \\
\text{une demi-tangente verticale dirigée vers le bas.}
\end{array}
&
\begin{array}{c}
\bullet\; \displaystyle \lim_{x \to a^-} \dfrac{f(x)-f(a)}{x-a} = -\infty \\
\text{La courbe de } f \text{ admet au point }\\ A(a, f(a)) \\
\text{une demi-tangente verticale}\\\text{ dirigée vers le haut.}
\end{array}
\\
\hline
\end{array}
\]

\item Si $\displaystyle \lim_{x \to a^{+}} \dfrac{f(x)-f(a)}{x-a}=\pinf $   et $ \displaystyle\lim_{x \to a^{-}} \dfrac{f(x)-f(a)}{x-a}=\minf $ alors la courbe de $ f $ admet  au  point $ A (a, f (a)) $ deux demi-tangentes verticales  dirigées vers le haut  d'équation $ x=a $.  A est un \textbf{point de rebroussement}.
\item Si $\displaystyle \lim_{x \to a^{+}} \dfrac{f(x)-f(a)}{x-a}=\minf $   et $\displaystyle \lim_{x \to a^{-}} \dfrac{f(x)-f(a)}{x-a}=\pinf $ alors la courbe de $ f $ admet  au  point $ A (a, f (a)) $ deux demi-tangentes verticales  dirigées vers le bas  d'équation $ x=a $.  A est un point de rebroussement.
\item Si $ \displaystyle\lim_{x \to a^{+}} \dfrac{f(x)-f(a)}{x-a}=\pinf $   et $ \lim_{x \to a^{-}} \dfrac{f(x)-f(a)}{x-a}=\pinf $ alors la courbe de $ f $ admet  au  point $ A (a, f (a)) $ deux demi-tangentes verticales de même équation $ x=a $ l'une dirigée vers le haut et l'autre vers le bas.  A est un point d'inflexion.
\item Si $ \displaystyle \lim_{x \to a^{-}} \dfrac{f(x)-f(a)}{x-a}=\pinf $   et $ \displaystyle \lim_{x \to a^{-}} \dfrac{f(x)-f(a)}{x-a}=\minf $ alors la courbe de $ f $ admet  au  point $ A (a, f (a)) $ deux demi-tangentes verticales de même équation $ x=a $ l'une dirigée vers le haut et l'autre vers le bas.  A est un point d'inflexion à tangente verticale.
\end{enumerate}
\subsection{Continuité et dérivabilité sur un intervalle}
\begin{definition}
\begin{itemize}
\item  $ f $ est continue  ( resp. dérivable ) sur l'intervalle  $ I $ si elle est continue ( resp. dérivable ) en tout réel  $x \in I. $
\item 	La fonction qui à tout réel $ x $  de $ I $ associe le nombre dérivé de $ f $ en $ x $ s'appelle \textbf{ fonction dérivée  ou dérivée} de $ f $ et est notée $ f': x\mapsto f'(x) $. \\
  L'ensemble des réels $ x $ pour lesquels $f'(x) $ existe est appelé \textbf{ ensemble ou domaine de dérivabilité de $f$}: c'est le domaine de définition de $ f'. $
  \end{itemize}
\end{definition}

Rappelons  ci-dessous les fonctions dérivées de certaines fonctions usuelles.


\medskip
$
\begin{array}{|l|l|l|}
\hline
\text{\textbf{Fonction } } f  \text{\textbf{ définie par :}} & 
\text{\textbf{Dérivable sur}} & 
\text{\textbf{Fonction dérivée }}  f'(x)  \\
\hline
f(x) = k,\ k \in \mathbb{R} & \mathbb{R} & 0 \\
\hline
f(x) = x^n,\ n \in \mathbb{Q} & \mathbb{R} & nx^{n-1} \\
\hline
f(x) = \dfrac{1}{x} & \mathbb{R}^* & -\dfrac{1}{x^2} \\
\hline
f(x) = \sqrt{x} & \mathbb{R}_+^* & \dfrac{1}{2\sqrt{x}} \\
\hline
f(x) = \cos x & \mathbb{R} & -\sin x \\
\hline
f(x) = \sin x & \mathbb{R} & \cos x \\
\hline
f(x) = \tan x & 
\small \left] (2k{-}1)\frac{\pi}{2}, (2k{+}1)\frac{\pi}{2} \right[,\ k \in \mathbb{Z}  & 
1 + \tan^2 x\ \text{ou}\ \dfrac{1}{\cos^2 x} \\
\hline
\end{array}
$




\begin{property}
Soient $f $ et  $ g$ deux fonctions continues (resp. dérivables) sur un intervalle $ I. $

\begin{itemize}
\item les fonctions $ f+g $ et $ fg $ sont continues ( resp. dérivables ) sur   $ I. $
\item  Si de plus $ g\neq0 $ sur  $ I $ alors les fonctions $\;\dfrac{1}{g}\; $ et  $\; \dfrac{f}{g} $ sont continues (resp. dérivables) sur $ I. $
\end{itemize}
\end{property}

\textbf{Cas particuliers}
\begin{itemize}
\item Les fonctions polynômes sont continues et dérivables sur $ \Rr. $ 
\item Les fonctions rationnelles sont continues et dérivables sur tout intervalle de leur ensemble de définition.
\item Les fonctions $x\mapsto\cos x $ et  $x\mapsto\sin x $ sont continues et dérivables sur $ \Rr. $
\item La fonction $x\mapsto\tan x $ est continue et dérivable sur tout intervalle du type $\intoo{(2k-1)\frac{\pi}{2}}{(2k+1)\frac{\pi}{2}}$, $ k\in\Zz $.

\end{itemize}
\subsubsection*{Image d'un intervalle par une fonction continue}
Nous admettons le théorème suivant.
\begin{theorem}
Si $ f $ est une fonction \emph{continue} sur  un intervalle $ I $ alors $ f(I) $ est un intervalle .
\end{theorem}
\textbf{Cas particuliers}\\
Le tableau suivant donne les images de quelques intervalles simples par une fonction \textbf{continue et strictement monotone }.  $ a$ et $b $ peuvent être éventuellement $\pinf $ ou $\minf .$ 

$$
\begin{array}{|c|c|c|}
\hline
& \text{\(f(I)\)} & \text{\(f(I)\)} \\
& \text{si \(f\) continue et strictement croissante} & \text{si \(f\) continue et strictement décroissante} \\
\hline
I = [a,b] & [f(a), f(b)] & [f(b), f(a)] \\
\hline
I = [a,b[ & \left[ f(a), \lim\limits_{x \to b^-} f(x) \right[ & \left] \lim\limits_{x \to b^-} f(x), f(a) \right] \\
\hline
I = ]a,b] & \left] \lim\limits_{x \to a^+} f(x), f(b) \right] & \left[ f(b), \lim\limits_{x \to a^+} f(x) \right[ \\
\hline
I = ]a,b[ & \left] \lim\limits_{x \to a^+} f(x), \lim\limits_{x \to b^-} f(x) \right[ & \left] \lim\limits_{x \to b^-} f(x), \lim\limits_{x \to a^+} f(x) \right[ \\
\hline
\end{array}
$$

\subsubsection*{Continuité et dérivabilité de la composée de deux fonctions}
\begin{property}
$ \centerdot $ Si $ f $ est continue sur l'intervalle $ I $ et $ g $ continue sur l'intervalle $ f(I) $ alors la fonction $ g\circ f $ est continue sur  $ I $.\\
$ \centerdot $ Si $ f $ est dérivable sur l'intervalle $ I $ et $ g $ dérivable sur l'intervalle $ f(I) $ alors la fonction $ g\circ f $ est dérivable sur $ I $ et pour tout $ x\in I $, on a:
\[\paren{g\circ f(x)}'= f'(x)\times g'[f(x)] \]
\end{property}
\begin{example} 
Soit $h(x)= \cos \paren{\dfrac{1}{x}}.\quad$ Calculons  $h'(x) $.

On a   $\;\; D_{h}= \intoo{\minf}{0}\cup \intoo{0}{\pinf} $

\textbf{Attention :}  Avant de dériver une fonction, il est recommandé de justifier sa dérivabilité même si la question ne le précise pas.\\
La fonction rationnelle $ x\mapsto \dfrac{1}{x} $ est  définie sur $ \Rre $ donc dérivable sur chacun des intervalles $\intoo{\minf}{0} $ et $\intoo{0}{\pinf}. $\\
La fonction $ x\mapsto \cos x $ est dérivable sur $ \Rr $; en particulier sur chacun des intervalles $\intoo{\minf}{0} $ et $\intoo{0}{\pinf} $\\ D' où par  composée $ h $ est dérivable  sur chacun des intervalles $\intoo{\minf}{0} $ et $\intoo{0}{\pinf} $.\\
Pour tout   $ x\neq 0 \quad$   :    $h'(x)=\dfrac{1}{x^{2}}\sin\paren{\dfrac{1}{x}} $ 

\end{example}
\begin{corollary}
$ \centerdot $ Si $ f $ est dérivable  (resp. continue) sur $ I $  et $ g $ dérivable sur $ \Rr $  alors $ g \circ f $ est dérivable (resp. continue) sur $ I. $\\
$ \centerdot $ Si $ f $ est \textbf{ continue et positive} sur $ I $ alors $ \sqrt{f} $ est continue sur $ I $.\\
$ \centerdot $ Si $ f $ est \textbf{ dérivable et strictement positive } sur $ I $ alors $ \sqrt{f} $ est dérivable sur $ I $.
\end{corollary}


\subsubsection*{Formules de dérivation}
Soient $u $ et $v $ deux fonctions dérivables, $ r\in\Zze $ et $ k\in\Rr $\\

$\begin{array}{|c|c|c|c|c|c|c|c|}
\hline 
\textbf{Fonction} & ku & u+v & uv & \dfrac{u}{v} & \dfrac{1}{v} & \sqrt{u} & u^{r} \\     
\hline 
\textbf{Dérivée} & ku' & u'+v' & u'v+ v'u & \dfrac{u'v- v'u}{v^{2}} & -\dfrac{v'}{v^{2}} & \dfrac{u'}{2\sqrt{u}} & ru'u^{r-1} \\    
\hline
\end{array}$

\bigskip

$\begin{array}{|c|c|c|c|}
\hline
   v\circ u  &  \sin u  &  \cos u  & \tan u \\
\hline
 u'\times (v'\circ u)   &  u'\cos u  & -u'\sin u &  \dfrac{u'}{\cos^{2}u}\\     
 \hline
\end{array}$



\begin{exercice}
Soit $ f (x)=\dfrac{(x+1)\sqrt{x-2}}{x-1}.$
 \begin{enumerate}
\item Etudier la continuité  de $ f $ sur son $D_{f}$.
\item Etudier  la dérivabilité de $ f $ sur son $D_{f}$.
\item Calculer $ f'(x). $
\end{enumerate}
\end{exercice}
\begin{proof}
\begin{enumerate}
\item $ f(x) $ existe si et seulement, si $ x \geq 2 $ et  $ x\neq 1 $ donc $ D_{f}=\intfo{2}{\pinf} $ \\ \textbf{ 1\iere{} méthode } \\ La fonction $ x \mapsto \frac{x+1}{x-1} $ est une fonction rationnelle définie sur $\intfo{2}{\pinf}  $ donc continue  sur $\intfo{2}{\pinf}  $ \\ La fonction $ x \mapsto x-2$ est continue et positive sur $\intfo{2}{\pinf}  $  donc la fonction $ x \mapsto \sqrt{x-2} $ est continue sur $\intfo{2}{\pinf}  $  par composée.\\ On en déduit que $ f $ est continue sur $  D_{f}$ comme produit et composée de  fonctions continues. \\
\textbf{ 2\iere{} méthode } \\  La fonction $ x \mapsto x+1 $ est  continue  sur $\intfo{2}{\pinf} $, \\ La fonction $ x \mapsto x-1 $ est  continue et non nulle  sur  $\intfo{2}{\pinf} $,   \\ La fonction $ x \mapsto x-2$ est continue et positive sur $\intfo{2}{\pinf}$  donc la fonction $ x \mapsto \sqrt{x-2} $ est continue sur $\intfo{2}{\pinf} $  par composée.\\ On en déduit que $ f $ est continue sur $  D_{f}$ comme produit, quotient et composée de  fonctions continues.\\

\item  \textbf{ 1\iere{} méthode } \\ La fonction $ x \mapsto \frac{x+1}{x-1} $ est une fonction rationnelle définie sur $\intfo{2}{\pinf}  $ donc  dérivable sur $\intfo{2}{\pinf}  $ \\ La fonction $ x \mapsto x-2$ est dérivable et strictement positive sur $\intoo{2}{\pinf}$ donc la fonction $ x \mapsto \sqrt{x-2} $ est  dérivable sur $\intoo{2}{\pinf}$. par composée.\\ On en déduit que $ f $ est dérivable  sur $\intoo{2}{\pinf}$ comme produit et composée de  fonctions dérivables. \\

\textbf{ 2\iere{} méthode } \\  La fonction $ x \mapsto x+1 $ est  dérivable sur $\intfo{2}{\pinf} $ \\ La fonction $ x \mapsto x-1 $ est  dérivable et non nulle  sur sur $\intoo{2}{\pinf} $   \\ La fonction $ x \mapsto x-2$ est  dérivable et strictement positive sur $\intoo{2}{\pinf}$ donc la fonction $ x \mapsto \sqrt{x-2} $ est   dérivable sur $\intoo{2}{\pinf}$ par composée.\\ On en déduit que $ f $ est dérivable  sur $\intoo{2}{\pinf}$ comme produit, quotient et composée de  fonctions dérivables. 

 \item $ \forall x > 2  $, $ f'(x)= \frac{-2}{(x-1)^{2}}\times \sqrt{x-2}+ \frac{1}{2\sqrt{x-2}}\times \dfrac{x+1}{x-1} $\\
Soit  $ f'(x)=\dfrac{x^{2}-4x+7}{2(x-1)^{2}\sqrt{x-2}} $
\end{enumerate}
\end{proof}
\subsubsection*{Dérivée et sens de variations}
\begin{theorem}
Soit $ f $ une fonction dérivable sur un intervalle $ I. $
\begin{itemize}
\item[\textbullet] $f$ est strictement croissante sur $I$ si et seulement si : $ ∀ x\in I$, $f '(x) ≥ 0 $ \\et $f'$ ne s'annule qu'en un nombre fini de points de $I$. 
\item[\textbullet] $f$ est strictement décroissante sur $I$ si et seulement si : $ ∀ x\in I$, $f '(x)  \leq 0 $ \\ et $f'$ ne s'annule qu'en un nombre fini de points de $I$.
\item[\textbullet] $f$ est  croissante sur $I$ si et seulement si : $ ∀ x\in I$, $f '(x) ≥ 0 $. 
\item[\textbullet] $f$ est  décroissante sur $I$ si et seulement si : $ ∀ x\in I$, $f '(x) \leq 0 $ .  
\end{itemize}
\end{theorem}
 \subsubsection*{Signe d'une fonction à partir de ses variations}


\text{Les cas classiques:}

\begin{tikzpicture}
\tkzTabInit[lgt=1, espcl=2]%
{$x$ / 1, $f'(x)$ / 1, $f(x)$ / 1}%
{$\cdots$, $\alpha$, $\cdots$}

\tkzTabLine{ , - , z , + , }
\tkzTabVar{+/ , -/$f(\alpha)\geq 0$/ , +/}
\end{tikzpicture}




Si  $f(x)$  admet un minimum positif  
sur  I  alors  $f$  est positive sur  I.
\bigskip


\begin{tikzpicture}
\tkzTabInit[lgt=1, espcl=2]%
{$x$ / 1, $f'(x)$ / 1, $f(x)$ /1}%
{$\cdots$, $\alpha$, $\cdots$}

\tkzTabLine{ , + , z , - , }
\tkzTabVar{-/ , +/$f(\alpha)\leq O$/ , -/}
\end{tikzpicture}



Si  $f(x)$ admet un maximum négatif 
sur  I  alors $ f$  est  négative sur  I.

\begin{tikzpicture}
   \tkzTabInit[lgt=3,espcl=1.5]
     {$x$ / 1 , $f'(x)$ / 1, $f(x)$ / 2 }
     { $\cdots$, $\alpha$, $\cdots$}
   \tkzTabLine
     {, , + , ,}
   \tkzTabVar
     {-/ , R/, +/}
   \tkzTabIma{1}{3}{2}{0}
\end{tikzpicture}

$f(x)$  est négatif si $x \leq \alpha$. \\
$f(x)$  est positif si  $ x \geq\alpha$.

\begin{tikzpicture}
   \tkzTabInit[lgt=3,espcl=1.5]
     {$x$ / 1 , $f'(x)$ / 1, $f(x)$ / 2 }
     {$\cdots$, $\alpha$, $\cdots$}
   \tkzTabLine
     {, , - ,  , }
   \tkzTabVar
     { +/, R/ , -/}
   \tkzTabIma{1}{3}{2}{0}
  
\end{tikzpicture}

$f(x)$  est positif si $x \leq \alpha$. \\
$f(x)$  est négatif si $x \geq \alpha$.

 \subsubsection*{Dérivées successives}

 \begin{definition}
 Soit $ f $ une fonction dérivable sur $ I $, sa fonction  dérivée $ f' $ est appelée                            
  \emph{fonction dérivée première} et est notée $ f^{(1)}.$ \\
 Si $ f'$ est dérivable sur $ I $, on dit que $ f $ est deux fois dérivable alors dans ce cas la fonction  dérivée  de $ f' $ c'est à dire $ (f')' $ est  appelée \emph{fonction dérivée seconde} de $ f $ et est notée $ f'' $ ou $ f^{(2)} .$ \\ 
  Si $ f''$  est à son tour  dérivable sur $ I $, alors sa fonction  dérivée   est  appelée \emph{fonction dérivée troisième} de $ f $ et est notée $ f''' $ ou $ f^{(3)} .$ \\ 
 Par itération si la dérivée n-ième de $ f $ existe, on la note $ f^{(n)} .$   
 \end{definition}
 
  \begin{example}
 $ f(x)= x\sin x $ \\
 $ f'(x)=\sin x+x\cos x $, $ f''(x)=2\cos x+x\sin x $, $ f^{(3)}(x)=-3\sin x+x\cos x $, etc.
  \end{example}
   \begin{remark}
 $ \centerdot $ $ f^{(n)}$ est aussi appelée \emph{dérivée d'ordre n de $ f. $} \\
  $ \centerdot $ En \textbf{\color{magenta}Physique} $f' $, $ f'', $ $\cdots$, $ f^{(n)}$ sont notées respectivement $\dfrac{df}{dx} $, $ \dfrac{d^{2}f}{dx^{2}} $, $\cdots$, $ \dfrac{d^{n}f}{dv^{n}}$.
 \end{remark}
\subsubsection*{Notion de différentielle}
Une petite variation $ \Delta x $ de la variable $ x $ provoque une petite variation  $ \Delta y = f(x+ \Delta x)-f(x) $ des images.\\ Lorsque $ \Delta x $ est voisin de $ 0 $, on assimile $ dx=\Delta x $ et on peut écrire: $ \frac{dy}{dx} =f'(x)$ ou $ dy=f'(x)dx $ ou $ dy=f'(x)\Delta x $.
\begin{example}
 Pour la fonction $ y=2x^2-x $ avec $x=1 $  et $\Delta x= 0,01 $ \\ Vérifier que la différentielle $dy=0,03  $ et l'accroissement $ \Delta y= 0,0302 $
\end{example}
\subsubsection*{Position d'une courbe par rapport à sa tangente}
  Nous admettons le résultat suivant:\\ Si $ f $ est une fonction deux fois dérivable sur $ I $ et si $ f'' $ est négative sur $ I, $ alors la courbe $\mathcal{C} $ de $ f$ est en dessous de toutes ses tangentes. On dit  que  $\mathcal{C} $ est \textbf{ concave.}\\
   Si $ f $ est une fonction deux fois dérivable sur $ I $ et si $ f'' $ est positive sur $ I, $ alors la courbe $\mathcal{C} $ de $ f$ est en dessus de toutes ses tangentes. On dit  que  $\mathcal{C} $ est \textbf{ convexe.}
  \subsubsection*{Point d'inflexion}
  \begin{definition}
  On dit que la courbe de $ f $ admet un point d'inflexion d'abscisse $ x_{0} $ si la courbe y traverse sa tangente.
  \end{definition}
   \begin{tikzpicture}[scale=0.5]
  % Définition des bornes
  \def\xmin{-2}
  \def\xmax{5}
  \def\ymin{-5}
  \def\ymax{3}

  % Axes
  \draw[->] (\xmin,0) -- (\xmax,0) node[right] {};
  \draw[->] (0,\ymin) -- (0,\ymax) node[above] {};

  % Origine
  \node[below left] at (0,0) {$O$};

  % Vecteurs unité
  \draw[thick,->] (0,0) -- (1,0);
  \node[below] at (0.5,0) {\small $\vec{\imath}$};

  \draw[thick,->] (0,0) -- (0,1);
  \node[left] at (0,0.5) {\small $\vec{\jmath}$};

  % Courbe rouge : f(x) = x^3 - 3x^2
  \draw[red,thick,domain=-1:3,samples=100] plot (\x,{(\x)^3 - 3*(\x)^2});

  % Tangente bleue : y = -3x + 1
  \draw[blue,thick,domain=-0.5:2,samples=100] plot (\x,{-3*\x + 1});

\end{tikzpicture}


\medskip

 La courbe traverse sa tangente (en bleu).
 
   \begin{theorem}
   Si $ f $ est deux fois dérivable sur  un intervalle ouvert $ I $ contenant $ x_0 $  et si $ f'' $  \textit{ s'annule en changeant de signe  en $ x_0 $} alors le point de la courbe d'abscisse $ x_0 $  est un \textbf{ un point  d'inflexion}.
   \end{theorem}
   
    \begin{example}
   Reprenons l'exemple précédent $ f: x\mapsto x^{3}-3x^{2} $\\ 
   $ f $ est dérivable sur $ \Rr $ car fonction polynôme.\\
   On a $ f'(x)= 6x^2-6x$ et $ f''(x)= 6x-6 $\\
   $ f''(x)=0 \Longleftrightarrow x=1 $
   \renewcommand{\arraystretch}{1.5}

 \[\begin{array}{|c|ccccc|}
\hline
x & \minf & & 1 & & +\infty\\ \hline
\text{signe de } f''(x) & & +\qquad &\quad 0  & \quad - & \\
\hline
\end{array}\]

D'après le tableau de signes,  $ f'' $ s'annule en  $ 1 $ en changeant de signe; donc le point $ (1,-2) $ est un point d'inflexion de la courbe.
   \end{example}
   
  \subsubsection*{Inégalité des accroissements finis}
  Nous admettons le théorème de l'inégalité des accroissements finis et nous donnons ici les deux formes.\\
\textbf{\color{magenta}Première forme} \\
 Soit $ f $ une fonction dérivable sur un intervalle $ I $. On suppose qu'il existe deux réels $m $ et $ M$ tels que :\colorbox{green!20!}{ $ m\leq f'(x) \leq M $} \quad pour tout $ x\in I $. \\ Alors pour tous $a $ et $ b$ de $ I $  $(b < a)  $ on a :\colorbox{green!20!} {$ m(b-a)\leq f(b)-f(a) \leq M(b-a) $}  \\
 
 \textbf{\color{magenta}Deuxième forme} \\
 Soit $ f $ une fonction dérivable sur un intervalle $ I $. On suppose qu'il existe un réel $ M$ tel que :\colorbox{green!20!}{ $ \abs{ f'(x)} \leq M $} \quad pour tout $ x\in I $. \\ Alors pour tous $a $ et $ b$ de $ I $   on a :\colorbox{green!20!} {$ \abs{ f(b)-f(a)}\leq M \abs{b-a}$}
 
\begin{exercice}
 Soit $ f $ la fonction définie sur $ \intff{0}{\frac{\pi}{4}} $ par $ f(x)= \sin x $\\
 Démontrer que $ \forall x \in \intff{0}{\frac{\pi}{4}} $ on a: $ \frac{\sqrt{2}}{2}x \leq \sin x \leq x $\\
\end{exercice}
\begin{proof}
 La fonction $ f $ est dérivable sur $ I=\intff{0}{\frac{\pi}{4}} $  et $ \forall \in I $ on a: $ f'(x)=\cos x $ \\ Or $ \forall x \in \intff{0}{\frac{\pi}{4}} $ on a $ \frac{\sqrt{2}}{2} \leq \cos x \leq 1 $ donc pour $ a=0 $ et $ b=x\in I $,\\ le T.I.A.F donne: $-1\times \frac{\sqrt{2}}{2}(x-0) \leq \sin x -\sin 0 \leq 1\times (x-0) $\\ d'où   $ \frac{\sqrt{2}}{2}x \leq \sin x \leq x $.
 \end{proof}
 \subsection{ Théorème des valeurs intermédiaires}
 \begin{theorem}[T.V.I]
 Soit $ f $ une fonction \textbf{\color{magenta}continue} sur un intervalle $ \intff{a}{b} $. \\
 Pour tout nombre réel $ \beta $ compris entre $f(a) $ et $f(b) $, il existe \textbf{\color{magenta} au moins} un réel $ \alpha \in \intff{a}{b} $ tel que $ f(\alpha)= \beta. $
 \end{theorem}
 
 \subsubsection*{Conséquence 1}
  Si $ f $ une fonction \textbf{\color{magenta}continue et strictement monotone} sur  l'intervalle $ \intff{a}{b} $ alors pour tout nombre réel $ \beta $ compris entre $f(a) $ et $f(b) $, il existe  \textbf{\color{magenta}un unique} un réel $ \alpha \in \intff{a}{b} $ tel que $ f(\alpha)= \beta. $
 \subsubsection*{Conséquence 2}
 $ a$, $b$,  $c $  et $ d$ désignent soit des réels, soit $ \pinf $, soit $ \minf. $\\
  Soit $ f $ une fonction \textbf{\color{magenta}continue et strictement monotone} sur  l'intervalle $ \intoo{a}{b} $ telle que :
   \[\displaystyle \lim_{x \to a}f(x)=c \quad \textrm{et}\quad \displaystyle\lim_{x \to b}f(x)=d\]
Alors pour tout nombre réel $ \beta $ compris entre $c $ et $d $, l'équation $ f(x)=\beta $ admet une solution  unique $ \alpha \in \intoo{a}{b} $.\\

\begin{exercice}
Soit $ f $ la fonction définie par $ f(x)= x^{3}+x+1 $.
 \begin{enumerate}
\item Etudier les variations  de $ f $.
\item Montrer que  l'équation $ f(x)=0 $ admet une seule solution $ \alpha$ dans $\Rr $.\\ En déduire que $ \alpha\in\intoo{-1}{0} $
\item  Déterminer un encadrement de $ \alpha $ d'amplitude $ 0,01 $.
\end{enumerate}
\end{exercice}
\begin{proof}
 \begin{enumerate}
\item $ f $ est définie, continue et dérivable sur $ \Rr. $\\
$ f'(x) = 3x^{2}+1 > 0$  $\forall x\in \Rr $ Donc $ f $ est strictement croissante  sur $ \Rr $ \\
$ \displaystyle\lim_{x \to \pinf}f(x)=\displaystyle \lim_{x \to \pinf}x^{3}+x+1=\displaystyle\lim_{x \to \pinf}x^{3}=\pinf$\\
$\displaystyle \lim_{x \to \minf}f(x)=\displaystyle \lim_{x \to \minf}x^{3}+x+1=\displaystyle\lim_{x \to \minf}x^{3}=\minf$

\begin{tikzpicture}
\tkzTabInit[lgt=1.5, espcl=1]%
{$x$ / 1, $f'(x)$ / 1, $f(x)$ / 1}%
{$-\infty$, $+\infty$}

\tkzTabLine{ , + , }
\tkzTabVar{-/$-\infty$, +/$+\infty$}
\end{tikzpicture}


\item $ f $ est continue et strictement croissante sur $ \Rr$ à valeurs dans sur $ \Rr. $ Or $ 0\in \Rr $ donc d'après la conséquence  du T.V.I il existe un unique réel $ \alpha $ tel que $ f(\alpha)=0 $.\\ De plus $ f(-1)f(0)=-1\times 1 < 0$ donc $$  f(-1)<0<f(0) $$ $$ \Leftrightarrow f(-1)<f(\alpha)<f(0) $$ $$ \Leftrightarrow -1<\alpha< 0 \quad \textrm{car $ f $ est strictement croissante. }$$
\item Encadrement de $ \alpha $ d'amplitude $ 0,01 $ par \textbf{la méthode du balayage }.\\
$ \color{red}\centerdot $ Recherchons d"abord un encadrement de $ \alpha $ par deux décimaux consécutifs d'ordre $ 1. $\\
Calculons de proche en proche les images par $ f $  des nombres décimaux d'ordre $ 1 $ de l'intervalle $ \intfo{-1}{0} $ jusqu'à ce qu'on observe un changement de signe.


\renewcommand{\arraystretch}{1}
$\begin{array}{|c|c|c|c|c|c|c|c|c|c|}
\hline 
x &-0,9  &-0,8 & -0,7 &- 0,6 & -0,5 & -0,4 &-0,3 &-0,2& -0,1  \\
\hline
f(x) &-  & -  &  - &  +  &  &  &  &  &  \\
\hline
\end{array}$

\vspace{0.5cm}   On obtient $-0,7<\alpha< - 0,6  $


$ \centerdot $ Recherchons ensuite  un encadrement de $ \alpha $ par deux décimaux consécutifs d'ordre $ 2. $\\
Calculons de proche en proche les images par $ f $  des nombres décimaux d'ordre $ 2 $ de l'intervalle $ \intoo{-0,7}{-0,6} $ jusqu'à ce qu'on observe un changement de signe.

\begin{center}
$\begin{array}{|c|c|c|c|c|c|c|c|c|c|}
\hline 
x &-0,69  &-0,68 & -0,67 &- 0,66 & -0,65 & -0,64 &-0,63 &-0,62& -0,61  \\
\hline
f(x) & - &  +  &  &   &  &  &  &  &  \\
\hline
\end{array}$

\vspace{0.5cm}   On obtient $-0,69<\alpha< - 0,68  $

\end{center}
\end{enumerate}
  \end{proof}
  \subsubsection*{Conséquence 3}
  
 Si $ f $ est \textbf{\color{magenta}continue et strictement monotone} sur  l'intervalle $ \intff{a}{b} $ et si \colorbox{green!20!}{$ f(a)f(b) < 0 $ }\\
Alors l'équation $ f(x)=0 $ admet une  unique  solution $ \alpha \in \intff{a}{b} $.

\begin{remark}
Pour montrer que l'équation $ f(x)=x $ admet une  unique  solution dans l'intervalle $I$; on pose $ g(x)=f(x)-x $ et on applique le T.V.I à la fonction $ g $ sur l'intervalle $I$.
\end{remark}

\begin{example}
Montrons  que l'équation $ \cos x= x $ admet une unique solution $ \alpha$ telle que    $\; \frac{\pi}{6} \leq \alpha \leq\frac{\pi}{4}.$ \\
\emph{Réponse}\\
Remarquons que  $ \cos x= x  \Leftrightarrow \cos x- x=0$\\
Posons la fonction $ f(x)= \cos x- x$ pour $ x\in \intff{\frac{\pi}{6}}{\frac{\pi}{4}} $.\\
$ f $ est dérivable sur $\intff{\frac{\pi}{6}}{\frac{\pi}{4}} $ comme somme de deux fonctions dérivables. \\
Pour $ x\in \intff{\frac{\pi}{6}}{\frac{\pi}{4}} $, $ f'(x)=-\sin x-1< 0 $; donc $ f $ est strictement décroissante.\\
De plus $ f(\frac{\pi}{6})=\frac{\sqrt{3}}{2} -\frac{\pi}{6 }> 0$  et $ f(\frac{\pi}{4})=\frac{\sqrt{2}}{2} -\frac{\pi}{4 }< 0$.\\
 Donc $ f $ est continue et strictement décroissante sur  l'intervalle $ \intff{\frac{\pi}{6 }}{\frac{\pi}{4}} $ \\et $ f(\frac{\pi}{6})f(\frac{\pi}{4}) < 0 $ 
d'où l'équation $ f(x)=0 $ admet une  unique  solution $ \alpha \in \intff{\frac{\pi}{6}}{\frac{\pi}{4}} $.


\end{example}

\subsection{Fonction réciproque d'une fonction continue et strictement monotone} 

\begin{theorem}[Théorème de la bijection ]
Soit $ f $ une fonction continue et strictement monotone  sur  l'intervalle $I$; alors $ f $ réalise une bijection de $ I $ vers l'intervalle $ f(I)$.\\
En plus sa bijection réciproque $ f^{-1} $ est continue et strictement monotone  sur  l'intervalle $f(I)$ et a le même sens de variation que $ f. $\\
Les courbes représentatives de $f $  et $f^{-1} $, dans un repère orthonormé sont symétriques par rapport  à la droite d'équation $ y=x $ (la première  bissectrice  du repère).
\end{theorem}



$ \centerdot $ Si de plus $ f $ est dérivable sur $ I $ et  $ f'$ ne s'annule pas sur  $ I $ alors $ f^{-1} $ dérivable sur $ f(I) $ et
 \[ (f^{-1})'(y)=\frac{1}{f'(f^{-1}(y))}\quad \forall y\in f(I)\]
\begin{remark}
Posons $ f(a)=b $ \\
Si $ f $ est dérivable en $ a $ et $ f'(a)\neq 0 $ alors $ f^{-1} $ est dérivable en $ b $ et $(f^{-1})'(b)=\dfrac{1}{f'(a)}.  $\\

\textbf{\color{red}Attention}\\
Si $ f $ est dérivable en $ a $ et  $ f'(a)= 0 $ ou n'existe pas alors $ f^{-1} $ n'est pas  dérivable en $ b. $
\end{remark}

\begin{exercice}
Soit $ f $ la fonction définie par $ f(x)=4x^{2}+4x+2  $.
 \begin{enumerate}
\item Etablir  le tableau de variations   de $ f $. 
\item
\begin{enumerate}
\item Soit $ g $ la restriction de $ f $ à l'intervalle  $ \intfo{-\frac{1}{2}}{\pinf} $.
 Montrer que  $ g $  réalise une bijection  de $ \intfo{-\frac{1}{2}}{\pinf} $ vers un intervalle  $ J $ à préciser.
\item Justifier que $ g^{-1} $ est dérivable en 2 puis  calculer $(g^{-1})'(2) $.
\end{enumerate}
\end{enumerate}

\end{exercice}
\begin{proof}
\begin{enumerate}
\item $ f $ est définie, continue et dérivable sur $ \Rr. $\\
$ f'(x) = 8x+4$  $\forall x\in \Rr $\\
$\displaystyle \lim_{x \to \pinf}f(x)= \displaystyle\lim_{x \to \pinf}4x^{2}+4x+2=\displaystyle\lim_{x \to \pinf}4x^{2}=\pinf$\\
$\displaystyle \lim_{x \to \minf}f(x)=\displaystyle \lim_{x \to \minf}4x^{2}+4x+2=\displaystyle\lim_{x \to \minf}4x^{2}=\pinf$

\begin{tikzpicture}
\tkzTabInit[lgt=1.5, espcl=1]%
{$x$ / 1, $f'(x)$ / 1, $f(x)$ / 1}%
{$-\infty$, $-\dfrac{1}{2}$, $+\infty$}

\tkzTabLine{ , - , z , + , }
\tkzTabVar{+/ , -/$1$ , +/}
\end{tikzpicture}

\item 
\begin{enumerate}
\item $ g $ est continue et strictement croissante  sur  $ \intfo{-\frac{1}{2}}{\pinf} $ donc réalise une bijection $ \intfo{-\frac{1}{2}}{\pinf} $ de vers  $g( \intfo{-\frac{1}{2}}{\pinf})$ \\Or  $g( \intfo{-\frac{1}{2}}{\pinf})=\intfo{g(-\frac{1}{2})}{\displaystyle\lim_{x \to \pinf}g(x)}=\intfo{1}{\pinf} $ \\ D'où \underline{$J= \intfo{1}{\pinf} $ }.
\item Pour répondre à cette question, il  faut calculer  l' antécédent de $ 2 $ par $ g. $\\ $ g(x)=2\Leftrightarrow 4x^{2}+4x+2=2\Leftrightarrow 4x^{2}+4x=0\Leftrightarrow x= 0 \ ou\  x= -1$\\ Le seul antécédent dans $ \intfo{-\frac{1}{2}}{\pinf} $ est  0.\\
Or $ g $ est dérivable en 0 et $ g'(0)=f'(0)=4 \neq 0 $ donc  $ g^{-1} $  est dérivable en 2.\\
On a $(g^{-1})'(2)=\dfrac{1}{g'(0)}=\frac{1}{4}.  $
\end{enumerate}

\end{enumerate}
\end{proof}


  %</content>
\end{document}
