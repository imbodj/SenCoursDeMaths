\documentclass[12pt, a4paper]{report}

% LuaLaTeX :

\RequirePackage{iftex}
\RequireLuaTeX

% Packages :

\usepackage[french]{babel}
%\usepackage[utf8]{inputenc}
%\usepackage[T1]{fontenc}
\usepackage[pdfencoding=auto, pdfauthor={Hugo Delaunay}, pdfsubject={Mathématiques}, pdfcreator={agreg.skyost.eu}]{hyperref}
\usepackage{amsmath}
\usepackage{amsthm}
%\usepackage{amssymb}
\usepackage{stmaryrd}
\usepackage{tikz}
\usepackage{tkz-euclide}
\usepackage{fontspec}
\defaultfontfeatures[Erewhon]{FontFace = {bx}{n}{Erewhon-Bold.otf}}
\usepackage{fourier-otf}
\usepackage[nobottomtitles*]{titlesec}
\usepackage{fancyhdr}
\usepackage{listings}
\usepackage{catchfilebetweentags}
\usepackage[french, capitalise, noabbrev]{cleveref}
\usepackage[fit, breakall]{truncate}
\usepackage[top=2.5cm, right=2cm, bottom=2.5cm, left=2cm]{geometry}
\usepackage{enumitem}
\usepackage{tocloft}
\usepackage{microtype}
%\usepackage{mdframed}
%\usepackage{thmtools}
\usepackage{xcolor}
\usepackage{tabularx}
\usepackage{xltabular}
\usepackage{aligned-overset}
\usepackage[subpreambles=true]{standalone}
\usepackage{environ}
\usepackage[normalem]{ulem}
\usepackage{etoolbox}
\usepackage{setspace}
\usepackage[bibstyle=reading, citestyle=draft]{biblatex}
\usepackage{xpatch}
\usepackage[many, breakable]{tcolorbox}
\usepackage[backgroundcolor=white, bordercolor=white, textsize=scriptsize]{todonotes}
\usepackage{luacode}
\usepackage{float}
\usepackage{needspace}
\everymath{\displaystyle}

% Police :

\setmathfont{Erewhon Math}

% Tikz :

\usetikzlibrary{calc}
\usetikzlibrary{3d}

% Longueurs :

\setlength{\parindent}{0pt}
\setlength{\headheight}{15pt}
\setlength{\fboxsep}{0pt}
\titlespacing*{\chapter}{0pt}{-20pt}{10pt}
\setlength{\marginparwidth}{1.5cm}
\setstretch{1.1}

% Métadonnées :

\author{agreg.skyost.eu}
\date{\today}

% Titres :

\setcounter{secnumdepth}{3}

\renewcommand{\thechapter}{\Roman{chapter}}
\renewcommand{\thesubsection}{\Roman{subsection}}
\renewcommand{\thesubsubsection}{\arabic{subsubsection}}
\renewcommand{\theparagraph}{\alph{paragraph}}

\titleformat{\chapter}{\huge\bfseries}{\thechapter}{20pt}{\huge\bfseries}
\titleformat*{\section}{\LARGE\bfseries}
\titleformat{\subsection}{\Large\bfseries}{\thesubsection \, - \,}{0pt}{\Large\bfseries}
\titleformat{\subsubsection}{\large\bfseries}{\thesubsubsection. \,}{0pt}{\large\bfseries}
\titleformat{\paragraph}{\bfseries}{\theparagraph. \,}{0pt}{\bfseries}

\setcounter{secnumdepth}{4}

% Table des matières :

\renewcommand{\cftsecleader}{\cftdotfill{\cftdotsep}}
\addtolength{\cftsecnumwidth}{10pt}

% Redéfinition des commandes :

\renewcommand*\thesection{\arabic{section}}
\renewcommand{\ker}{\mathrm{Ker}}

% Nouvelles commandes :

\newcommand{\website}{https://github.com/imbodj/SenCoursDeMaths}

\newcommand{\tr}[1]{\mathstrut ^t #1}
\newcommand{\im}{\mathrm{Im}}
\newcommand{\rang}{\operatorname{rang}}
\newcommand{\trace}{\operatorname{trace}}
\newcommand{\id}{\operatorname{id}}
\newcommand{\stab}{\operatorname{Stab}}
\newcommand{\paren}[1]{\left(#1\right)}
\newcommand{\croch}[1]{\left[ #1 \right]}
\newcommand{\Grdcroch}[1]{\Bigl[ #1 \Bigr]}
\newcommand{\grdcroch}[1]{\bigl[ #1 \bigr]}
\newcommand{\abs}[1]{\left\lvert #1 \right\rvert}
\newcommand{\limi}[3]{\lim_{#1\to #2}#3}
\newcommand{\pinf}{+\infty}
\newcommand{\minf}{-\infty}
%%%%%%%%%%%%%% ENSEMBLES %%%%%%%%%%%%%%%%%
\newcommand{\ensemblenombre}[1]{\mathbb{#1}}
\newcommand{\Nn}{\ensemblenombre{N}}
\newcommand{\Zz}{\ensemblenombre{Z}}
\newcommand{\Qq}{\ensemblenombre{Q}}
\newcommand{\Qqp}{\Qq^+}
\newcommand{\Rr}{\ensemblenombre{R}}
\newcommand{\Cc}{\ensemblenombre{C}}
\newcommand{\Nne}{\Nn^*}
\newcommand{\Zze}{\Zz^*}
\newcommand{\Zzn}{\Zz^-}
\newcommand{\Qqe}{\Qq^*}
\newcommand{\Rre}{\Rr^*}
\newcommand{\Rrp}{\Rr_+}
\newcommand{\Rrm}{\Rr_-}
\newcommand{\Rrep}{\Rr_+^*}
\newcommand{\Rrem}{\Rr_-^*}
\newcommand{\Cce}{\Cc^*}
%%%%%%%%%%%%%%  INTERVALLES %%%%%%%%%%%%%%%%%
\newcommand{\intff}[2]{\left[#1\;,\; #2\right]  }
\newcommand{\intof}[2]{\left]#1 \;, \;#2\right]  }
\newcommand{\intfo}[2]{\left[#1 \;,\; #2\right[  }
\newcommand{\intoo}[2]{\left]#1 \;,\; #2\right[  }

\providecommand{\newpar}{\\[\medskipamount]}

\newcommand{\annexessection}{%
  \newpage%
  \subsection*{Annexes}%
}

\providecommand{\lesson}[3]{%
  \title{#3}%
  \hypersetup{pdftitle={#2 : #3}}%
  \setcounter{section}{\numexpr #2 - 1}%
  \section{#3}%
  \fancyhead[R]{\truncate{0.73\textwidth}{#2 : #3}}%
}

\providecommand{\development}[3]{%
  \title{#3}%
  \hypersetup{pdftitle={#3}}%
  \section*{#3}%
  \fancyhead[R]{\truncate{0.73\textwidth}{#3}}%
}

\providecommand{\sheet}[3]{\development{#1}{#2}{#3}}

\providecommand{\ranking}[1]{%
  \title{Terminale #1}%
  \hypersetup{pdftitle={Terminale #1}}%
  \section*{Terminale #1}%
  \fancyhead[R]{\truncate{0.73\textwidth}{Terminale #1}}%
}

\providecommand{\summary}[1]{%
  \textit{#1}%
  \par%
  \medskip%
}

\tikzset{notestyleraw/.append style={inner sep=0pt, rounded corners=0pt, align=center}}

%\newcommand{\booklink}[1]{\website/bibliographie\##1}
\newcounter{reference}
\newcommand{\previousreference}{}
\providecommand{\reference}[2][]{%
  \needspace{20pt}%
  \notblank{#1}{
    \needspace{20pt}%
    \renewcommand{\previousreference}{#1}%
    \stepcounter{reference}%
    \label{reference-\previousreference-\thereference}%
  }{}%
  \todo[noline]{%
    \protect\vspace{20pt}%
    \protect\par%
    \protect\notblank{#1}{\cite{[\previousreference]}\\}{}%
    \protect\hyperref[reference-\previousreference-\thereference]{p. #2}%
  }%
}

\definecolor{devcolor}{HTML}{00695c}
\providecommand{\dev}[1]{%
  \reversemarginpar%
  \todo[noline]{
    \protect\vspace{20pt}%
    \protect\par%
    \bfseries\color{devcolor}\href{\website/developpements/#1}{[DEV]}
  }%
  \normalmarginpar%
}

% En-têtes :

\pagestyle{fancy}
\fancyhead[L]{\truncate{0.23\textwidth}{\thepage}}
\fancyfoot[C]{\scriptsize \href{\website}{\texttt{https://github.com/imbodj/SenCoursDeMaths}}}

% Couleurs :

\definecolor{property}{HTML}{ffeb3b}
\definecolor{proposition}{HTML}{ffc107}
\definecolor{lemma}{HTML}{ff9800}
\definecolor{theorem}{HTML}{f44336}
\definecolor{corollary}{HTML}{e91e63}
\definecolor{definition}{HTML}{673ab7}
\definecolor{notation}{HTML}{9c27b0}
\definecolor{example}{HTML}{00bcd4}
\definecolor{cexample}{HTML}{795548}
\definecolor{application}{HTML}{009688}
\definecolor{remark}{HTML}{3f51b5}
\definecolor{algorithm}{HTML}{607d8b}
%\definecolor{proof}{HTML}{e1f5fe}
\definecolor{exercice}{HTML}{e1f5fe}

% Théorèmes :

\theoremstyle{definition}
\newtheorem{theorem}{Théorème}

\newtheorem{property}[theorem]{Propriété}
\newtheorem{proposition}[theorem]{Proposition}
\newtheorem{lemma}[theorem]{Activité d'introduction}
\newtheorem{corollary}[theorem]{Conséquence}

\newtheorem{definition}[theorem]{Définition}
\newtheorem{notation}[theorem]{Notation}

\newtheorem{example}[theorem]{Exemple}
\newtheorem{cexample}[theorem]{Contre-exemple}
\newtheorem{application}[theorem]{Application}

\newtheorem{algorithm}[theorem]{Algorithme}
\newtheorem{exercice}[theorem]{Exercice}

\theoremstyle{remark}
\newtheorem{remark}[theorem]{Remarque}

\counterwithin*{theorem}{section}

\newcommand{\applystyletotheorem}[1]{
  \tcolorboxenvironment{#1}{
    enhanced,
    breakable,
    colback=#1!8!white,
    %right=0pt,
    %top=8pt,
    %bottom=8pt,
    boxrule=0pt,
    frame hidden,
    sharp corners,
    enhanced,borderline west={4pt}{0pt}{#1},
    %interior hidden,
    sharp corners,
    after=\par,
  }
}

\applystyletotheorem{property}
\applystyletotheorem{proposition}
\applystyletotheorem{lemma}
\applystyletotheorem{theorem}
\applystyletotheorem{corollary}
\applystyletotheorem{definition}
\applystyletotheorem{notation}
\applystyletotheorem{example}
\applystyletotheorem{cexample}
\applystyletotheorem{application}
\applystyletotheorem{remark}
%\applystyletotheorem{proof}
\applystyletotheorem{algorithm}
\applystyletotheorem{exercice}

% Environnements :

\NewEnviron{whitetabularx}[1]{%
  \renewcommand{\arraystretch}{2.5}
  \colorbox{white}{%
    \begin{tabularx}{\textwidth}{#1}%
      \BODY%
    \end{tabularx}%
  }%
}

% Maths :

\DeclareFontEncoding{FMS}{}{}
\DeclareFontSubstitution{FMS}{futm}{m}{n}
\DeclareFontEncoding{FMX}{}{}
\DeclareFontSubstitution{FMX}{futm}{m}{n}
\DeclareSymbolFont{fouriersymbols}{FMS}{futm}{m}{n}
\DeclareSymbolFont{fourierlargesymbols}{FMX}{futm}{m}{n}
\DeclareMathDelimiter{\VERT}{\mathord}{fouriersymbols}{152}{fourierlargesymbols}{147}

% Code :

\definecolor{greencode}{rgb}{0,0.6,0}
\definecolor{graycode}{rgb}{0.5,0.5,0.5}
\definecolor{mauvecode}{rgb}{0.58,0,0.82}
\definecolor{bluecode}{HTML}{1976d2}
\lstset{
  basicstyle=\footnotesize\ttfamily,
  breakatwhitespace=false,
  breaklines=true,
  %captionpos=b,
  commentstyle=\color{greencode},
  deletekeywords={...},
  escapeinside={\%*}{*)},
  extendedchars=true,
  frame=none,
  keepspaces=true,
  keywordstyle=\color{bluecode},
  language=Python,
  otherkeywords={*,...},
  numbers=left,
  numbersep=5pt,
  numberstyle=\tiny\color{graycode},
  rulecolor=\color{black},
  showspaces=false,
  showstringspaces=false,
  showtabs=false,
  stepnumber=2,
  stringstyle=\color{mauvecode},
  tabsize=2,
  %texcl=true,
  xleftmargin=10pt,
  %title=\lstname
}

\newcommand{\codedirectory}{}
\newcommand{\inputalgorithm}[1]{%
  \begin{algorithm}%
    \strut%
    \lstinputlisting{\codedirectory#1}%
  \end{algorithm}%
}



\everymath{\displaystyle}
\begin{document}
  %<*content>
  \lesson{algebra}{6}{Fonction exponentielle (TS2)}

 \subsection*{Introduction}
  $ \text{La fonction } \;\;\begin{array}{lrcl}
\ln : & \intoo{0}{\pinf} & \longrightarrow & \Rr \\ 
      & x                & \longmapsto     & \ln x
\end{array} \;\text{est continue et strictement croissante.}$
    
    
 Elle admet une    \textbf{ bijection  réciproque } de $\; \Rr  \;$ vers $\;\intoo{0}{\pinf} $.

\subsection{Définition et propriétés}

\begin{definition}
On appelle fonction  exponentielle,  notée     exp  ou e, la bijection réciproque de la fonction $ \ln $.
\end{definition}

\begin{notation}
 $$ \begin{array}{lrcl}
\text{exp} : & \Rr & \longrightarrow & \intoo{0}{\pinf} \\
             & x   & \longmapsto     & \text{exp}(x) = \eexp{x}
\end{array} $$
\end{notation}
\begin{corollary}
\begin{itemize}
\item $ \forall x \in \Rr,\; \; \eexp{x}> 0 $ 

\item $  \eexp{0}= 1   \; $ et $ \;    \eexp{1}= \mathrm{e} $

 \item $\forall x> 0,\; \; \eexp{\ln{x} }=x  $ 
 
 \item $\forall y\in \Rr ,\; \; \ln {\eexp{y}}=y  $ 
  
  \item exp   est continue et dérivable sur  $\Rr$.
  \item exp   est bijective et  strictement croissante  sur $\Rr$. D'où :
  
  \item $ \eexp{a}=\eexp{b}\Longleftrightarrow a=b $
  
\item $ \eexp{a}> \eexp{b}\Longleftrightarrow a>b $
 
\end{itemize}

\end{corollary}


\begin{property}[fondamentale]

$$ \forall a \in \Rr \; \text{et} \;  \forall b \in\Rr: \quad \eexp{a+b}=\eexp{a}\times \eexp{b}  $$


\end{property}

\textbf{Démonstration}\\
$ \ln \paren{\eexp{a+b}}  =a+b $  \\
$  \ln \paren{\eexp{a} \times \eexp{b}}=\ln (\eexp{a})+ \ln {\eexp{b}} =a+b$ \\  D'où $ \;\;\eexp{a+b}=\eexp{a}\times \eexp{b}  $  

\begin{corollary}
 \begin{itemize}
 \item $ \eexp{-a}=\dfrac{1}{\eexp{a}} $
  \item $ \eexp{a-b}=\dfrac{\eexp{a}}{\eexp{b}} $
  
  \item $ \paren{\eexp{a}}^{r} = \eexp{ra} \quad \forall r \in \Qq $
 \end{itemize}
\end{corollary}


\subsection{Étude et représentation graphique}
Les  courbes de la fonction   exp   et de la fonction $ \ln  $ sont symétriques  par  rapport à la première bissectrice du repère.
  \begin{center}
\begin{tikzpicture}[>=stealth', scale=0.5]
\clip (-4,-3) rectangle (7,7);
\draw[->,thick] (0,0) -- (1,0);
\draw[->,thick] (0,0) -- (0,1);
\draw[,thick] (-4,0) -- (7,0);
\draw[thick] (0,-3) -- (0,7);
\foreach\x in {1,2,}
{
\draw[thick] (\x,0.1) -- (\x,-0.1) node[below] {\x};
}
\foreach\y in {1}
{
\draw[thick] (0.1,\y) -- (-0.1,\y) node[left] {\y};
}

\draw[thick,black] plot[domain=0.1 :7,samples=100] (\x,{ln (\x)}) node[above left] {$\mathscr{C_{\ln}}$};
\draw[thick,black] plot[domain=-3 :1.78,samples=100] (\x,{exp (\x)}) node[above left] {$\mathscr{C_{\exp}}$};
\draw[thick,dashed] plot[domain=-3 :7,samples=100] (\x, \x);
\draw[dashed,thick](2.7,0)--(2.7,1);
\draw[dashed,thick](2.7,1)--(0,1);
\node at(2.7,-0.4) {$\mathrm{e}$};
\node at(-0.4, 2.7) {$\mathrm{e}$};
\draw[dashed,thick](0,2.7)--(1,2.7);
\draw[dashed,thick](1,2.7)--(1,0);

\end{tikzpicture}
\end{center}


\subsection*{Limites} 
Aux bornes de l'ensemble de définition de la fonction exp, on obtient les limites suivantes:
\begin{property}
\begin{itemize}
\item  $\displaystyle \lim_{x \to \pinf}\eexp{x} =\pinf $ 
 \item $ \displaystyle\lim_{x \to \minf}\eexp{ x}=0 $
\end{itemize}
\end{property}

\textbf{Démonstration}
\begin{itemize}
\item  Soit $ \varphi $  la fonction définie par : $\; x\longmapsto \eexp{x}-x-1 $. \\
$ \varphi $  est dérivable sur $ \intfo{0}{\pinf} \;$    et  $ \;\forall x> 0 \quad  \varphi^{\prime}(x)= \eexp{x}-1\geq 0$.  $ \;\varphi $  est donc croissante sur  $ \intfo{0}{\pinf} \;$    or  $ \;\varphi(0)=0. $  \\ Donc $ \forall x \geq 0,\; \varphi(x)\geq 0 \;$  càd  $ \eexp{x}\geq x+1 $.  Or $ \displaystyle\lim_{x \to \pinf}x+1=\pinf $  
par comparaison  $ \displaystyle\lim_{x \to \pinf}\eexp{x} =\pinf $ \\
\item Pour calculer $ \displaystyle\lim_{x \to \minf}\eexp{ x} $\;  posons\; $ y=-x $   \\
On a alors \;$\displaystyle\lim_{x \to \minf}\eexp{ x} = \displaystyle\lim_{y \to \pinf}\eexp{-y}=\lim_{y \to \pinf}\dfrac{1}{\eexp{y}}=0 $

\end{itemize}
\subsection*{ Tableau de variations}
\begin{center}
\begin{tikzpicture}
\tkzTabInit[lgt=1.5,espcl=1]
{$x$ /1, $(e^x)'$ /1, $e^x$ /1.5}
{$-\infty$, $+\infty$}

\tkzTabLine{, +, }

\tkzTabVar{-/$0$, +/$+\infty$ , }
\end{tikzpicture}


\end{center}

\subsection*{Dérivée} 
On sait que  $\; \forall x \in \Rr ,\;  \ln{\eexp{x}}=x $. \\
Dérivons les 2 membres de cette égalité.  \\
$ \ln^{\prime}\paren{ \eexp{x}}\times \paren{\eexp{x}}^{\prime}=1$ \\
$ \dfrac{\paren{\eexp{x}}^{\prime}}{\eexp{x}}=1   \Longleftrightarrow  \paren{\eexp{x}}^{\prime}=\eexp{x}$ 

\begin{property}
 La fonction exponentielle est dérivable sur $ \Rr $ et est égale à sa propre dérivée:
 $ \paren{ \eexp{x}}^{\prime}=\eexp{x}\;\;  \quad \forall x \in \Rr $
\end{property}
\begin{corollary}
Si $u$ est une fonction  dérivable sur un intervalle I alors la fonction $\; f: x \mapsto \eexp{u(x)} \;$  est dérivable sur I  et  $ f'(x)=\paren{\text{exp}^{\prime}(u(x))}\times u'(x) = \eexp{u(x)}\times u'(x)=\eexp{u(x)}\times u'(x)$ 

 \fbox{$ \paren{\eexp{u}}^{\prime}= u^{\prime} \eexp{u}$}
\end{corollary}

 La fonction  $\;  u^{\prime}\eexp{u} \;$  a pour primitive   sur I, toute fonction du type  $\;  \eexp{u}+C \; $ $ (C\in\Rr) $.
 


\subsection{ Quelques limites classiques}

\begin{property}
\begin{itemize}
\item  $\displaystyle \lim_{x \to \pinf}\dfrac{\eexp{x}}{x}=\pinf $
\item  $\displaystyle \lim_{x \to \minf} x\eexp{x}=0 $   
\item  $\displaystyle \lim_{x \to 0} \dfrac{\eexp{x}-1}{x}=1 $
\end{itemize}
\end{property}


\textbf{Démonstration}
\begin{itemize}
\item   Montrons que $ \displaystyle\lim_{x \to \pinf}\dfrac{\eexp{x}}{x}=\pinf $ \\  
 Pour $ x > 0 $,\; $ \ln  \paren{\dfrac{\eexp{x}}{x}}  =\ln \eexp{x}-\ln x  =x-\ln x  =x\paren{1-\dfrac{\ln x}{x}}\; $   d'où par produit  $\displaystyle \lim_{x \to \pinf}x\paren{1-\dfrac{\ln x}{x}}=\pinf $
 
 \item   Montrons que\; $ \displaystyle\lim_{x \to \minf}x\eexp{x}=0 $ \\\ 
 En posant $ X=-x $ ,\;on obtient $\displaystyle \lim_{x \to \minf}x\eexp{x}=\lim_{X \to \pinf}-X\eexp{-X} = \displaystyle\lim_{X \to \pinf}\dfrac{-X}{\eexp{X}}=0$ 
 \item  Montrons que\; $ \displaystyle\lim_{x \to 0} \dfrac{\eexp{x}-1}{x}=1 $\\
  Soit $ \varphi (x)=\eexp{x}$.\\
  $\displaystyle \lim_{x \to 0}  \dfrac{\eexp{x}-1}{x} =\displaystyle\lim_{x \to 0}  \dfrac{\varphi(x)-\varphi(0)}{x-0} = \varphi'(0)=\eexp{0}=1$
  \end{itemize}




  %</content>
\end{document}
