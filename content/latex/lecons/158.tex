\documentclass[12pt, a4paper]{report}

% LuaLaTeX :

\RequirePackage{iftex}
\RequireLuaTeX

% Packages :

\usepackage[french]{babel}
%\usepackage[utf8]{inputenc}
%\usepackage[T1]{fontenc}
\usepackage[pdfencoding=auto, pdfauthor={Hugo Delaunay}, pdfsubject={Mathématiques}, pdfcreator={agreg.skyost.eu}]{hyperref}
\usepackage{amsmath}
\usepackage{amsthm}
%\usepackage{amssymb}
\usepackage{stmaryrd}
\usepackage{tikz}
\usepackage{tkz-euclide}
\usepackage{fontspec}
\defaultfontfeatures[Erewhon]{FontFace = {bx}{n}{Erewhon-Bold.otf}}
\usepackage{fourier-otf}
\usepackage[nobottomtitles*]{titlesec}
\usepackage{fancyhdr}
\usepackage{listings}
\usepackage{catchfilebetweentags}
\usepackage[french, capitalise, noabbrev]{cleveref}
\usepackage[fit, breakall]{truncate}
\usepackage[top=2.5cm, right=2cm, bottom=2.5cm, left=2cm]{geometry}
\usepackage{enumitem}
\usepackage{tocloft}
\usepackage{microtype}
%\usepackage{mdframed}
%\usepackage{thmtools}
\usepackage{xcolor}
\usepackage{tabularx}
\usepackage{xltabular}
\usepackage{aligned-overset}
\usepackage[subpreambles=true]{standalone}
\usepackage{environ}
\usepackage[normalem]{ulem}
\usepackage{etoolbox}
\usepackage{setspace}
\usepackage[bibstyle=reading, citestyle=draft]{biblatex}
\usepackage{xpatch}
\usepackage[many, breakable]{tcolorbox}
\usepackage[backgroundcolor=white, bordercolor=white, textsize=scriptsize]{todonotes}
\usepackage{luacode}
\usepackage{float}
\usepackage{needspace}
\everymath{\displaystyle}

% Police :

\setmathfont{Erewhon Math}

% Tikz :

\usetikzlibrary{calc}
\usetikzlibrary{3d}

% Longueurs :

\setlength{\parindent}{0pt}
\setlength{\headheight}{15pt}
\setlength{\fboxsep}{0pt}
\titlespacing*{\chapter}{0pt}{-20pt}{10pt}
\setlength{\marginparwidth}{1.5cm}
\setstretch{1.1}

% Métadonnées :

\author{agreg.skyost.eu}
\date{\today}

% Titres :

\setcounter{secnumdepth}{3}

\renewcommand{\thechapter}{\Roman{chapter}}
\renewcommand{\thesubsection}{\Roman{subsection}}
\renewcommand{\thesubsubsection}{\arabic{subsubsection}}
\renewcommand{\theparagraph}{\alph{paragraph}}

\titleformat{\chapter}{\huge\bfseries}{\thechapter}{20pt}{\huge\bfseries}
\titleformat*{\section}{\LARGE\bfseries}
\titleformat{\subsection}{\Large\bfseries}{\thesubsection \, - \,}{0pt}{\Large\bfseries}
\titleformat{\subsubsection}{\large\bfseries}{\thesubsubsection. \,}{0pt}{\large\bfseries}
\titleformat{\paragraph}{\bfseries}{\theparagraph. \,}{0pt}{\bfseries}

\setcounter{secnumdepth}{4}

% Table des matières :

\renewcommand{\cftsecleader}{\cftdotfill{\cftdotsep}}
\addtolength{\cftsecnumwidth}{10pt}

% Redéfinition des commandes :

\renewcommand*\thesection{\arabic{section}}
\renewcommand{\ker}{\mathrm{Ker}}

% Nouvelles commandes :

\newcommand{\website}{https://github.com/imbodj/SenCoursDeMaths}

\newcommand{\tr}[1]{\mathstrut ^t #1}
\newcommand{\im}{\mathrm{Im}}
\newcommand{\rang}{\operatorname{rang}}
\newcommand{\trace}{\operatorname{trace}}
\newcommand{\id}{\operatorname{id}}
\newcommand{\stab}{\operatorname{Stab}}
\newcommand{\paren}[1]{\left(#1\right)}
\newcommand{\croch}[1]{\left[ #1 \right]}
\newcommand{\Grdcroch}[1]{\Bigl[ #1 \Bigr]}
\newcommand{\grdcroch}[1]{\bigl[ #1 \bigr]}
\newcommand{\abs}[1]{\left\lvert #1 \right\rvert}
\newcommand{\limi}[3]{\lim_{#1\to #2}#3}
\newcommand{\pinf}{+\infty}
\newcommand{\minf}{-\infty}
%%%%%%%%%%%%%% ENSEMBLES %%%%%%%%%%%%%%%%%
\newcommand{\ensemblenombre}[1]{\mathbb{#1}}
\newcommand{\Nn}{\ensemblenombre{N}}
\newcommand{\Zz}{\ensemblenombre{Z}}
\newcommand{\Qq}{\ensemblenombre{Q}}
\newcommand{\Qqp}{\Qq^+}
\newcommand{\Rr}{\ensemblenombre{R}}
\newcommand{\Cc}{\ensemblenombre{C}}
\newcommand{\Nne}{\Nn^*}
\newcommand{\Zze}{\Zz^*}
\newcommand{\Zzn}{\Zz^-}
\newcommand{\Qqe}{\Qq^*}
\newcommand{\Rre}{\Rr^*}
\newcommand{\Rrp}{\Rr_+}
\newcommand{\Rrm}{\Rr_-}
\newcommand{\Rrep}{\Rr_+^*}
\newcommand{\Rrem}{\Rr_-^*}
\newcommand{\Cce}{\Cc^*}
%%%%%%%%%%%%%%  INTERVALLES %%%%%%%%%%%%%%%%%
\newcommand{\intff}[2]{\left[#1\;,\; #2\right]  }
\newcommand{\intof}[2]{\left]#1 \;, \;#2\right]  }
\newcommand{\intfo}[2]{\left[#1 \;,\; #2\right[  }
\newcommand{\intoo}[2]{\left]#1 \;,\; #2\right[  }

\providecommand{\newpar}{\\[\medskipamount]}

\newcommand{\annexessection}{%
  \newpage%
  \subsection*{Annexes}%
}

\providecommand{\lesson}[3]{%
  \title{#3}%
  \hypersetup{pdftitle={#2 : #3}}%
  \setcounter{section}{\numexpr #2 - 1}%
  \section{#3}%
  \fancyhead[R]{\truncate{0.73\textwidth}{#2 : #3}}%
}

\providecommand{\development}[3]{%
  \title{#3}%
  \hypersetup{pdftitle={#3}}%
  \section*{#3}%
  \fancyhead[R]{\truncate{0.73\textwidth}{#3}}%
}

\providecommand{\sheet}[3]{\development{#1}{#2}{#3}}

\providecommand{\ranking}[1]{%
  \title{Terminale #1}%
  \hypersetup{pdftitle={Terminale #1}}%
  \section*{Terminale #1}%
  \fancyhead[R]{\truncate{0.73\textwidth}{Terminale #1}}%
}

\providecommand{\summary}[1]{%
  \textit{#1}%
  \par%
  \medskip%
}

\tikzset{notestyleraw/.append style={inner sep=0pt, rounded corners=0pt, align=center}}

%\newcommand{\booklink}[1]{\website/bibliographie\##1}
\newcounter{reference}
\newcommand{\previousreference}{}
\providecommand{\reference}[2][]{%
  \needspace{20pt}%
  \notblank{#1}{
    \needspace{20pt}%
    \renewcommand{\previousreference}{#1}%
    \stepcounter{reference}%
    \label{reference-\previousreference-\thereference}%
  }{}%
  \todo[noline]{%
    \protect\vspace{20pt}%
    \protect\par%
    \protect\notblank{#1}{\cite{[\previousreference]}\\}{}%
    \protect\hyperref[reference-\previousreference-\thereference]{p. #2}%
  }%
}

\definecolor{devcolor}{HTML}{00695c}
\providecommand{\dev}[1]{%
  \reversemarginpar%
  \todo[noline]{
    \protect\vspace{20pt}%
    \protect\par%
    \bfseries\color{devcolor}\href{\website/developpements/#1}{[DEV]}
  }%
  \normalmarginpar%
}

% En-têtes :

\pagestyle{fancy}
\fancyhead[L]{\truncate{0.23\textwidth}{\thepage}}
\fancyfoot[C]{\scriptsize \href{\website}{\texttt{https://github.com/imbodj/SenCoursDeMaths}}}

% Couleurs :

\definecolor{property}{HTML}{ffeb3b}
\definecolor{proposition}{HTML}{ffc107}
\definecolor{lemma}{HTML}{ff9800}
\definecolor{theorem}{HTML}{f44336}
\definecolor{corollary}{HTML}{e91e63}
\definecolor{definition}{HTML}{673ab7}
\definecolor{notation}{HTML}{9c27b0}
\definecolor{example}{HTML}{00bcd4}
\definecolor{cexample}{HTML}{795548}
\definecolor{application}{HTML}{009688}
\definecolor{remark}{HTML}{3f51b5}
\definecolor{algorithm}{HTML}{607d8b}
%\definecolor{proof}{HTML}{e1f5fe}
\definecolor{exercice}{HTML}{e1f5fe}

% Théorèmes :

\theoremstyle{definition}
\newtheorem{theorem}{Théorème}

\newtheorem{property}[theorem]{Propriété}
\newtheorem{proposition}[theorem]{Proposition}
\newtheorem{lemma}[theorem]{Activité d'introduction}
\newtheorem{corollary}[theorem]{Conséquence}

\newtheorem{definition}[theorem]{Définition}
\newtheorem{notation}[theorem]{Notation}

\newtheorem{example}[theorem]{Exemple}
\newtheorem{cexample}[theorem]{Contre-exemple}
\newtheorem{application}[theorem]{Application}

\newtheorem{algorithm}[theorem]{Algorithme}
\newtheorem{exercice}[theorem]{Exercice}

\theoremstyle{remark}
\newtheorem{remark}[theorem]{Remarque}

\counterwithin*{theorem}{section}

\newcommand{\applystyletotheorem}[1]{
  \tcolorboxenvironment{#1}{
    enhanced,
    breakable,
    colback=#1!8!white,
    %right=0pt,
    %top=8pt,
    %bottom=8pt,
    boxrule=0pt,
    frame hidden,
    sharp corners,
    enhanced,borderline west={4pt}{0pt}{#1},
    %interior hidden,
    sharp corners,
    after=\par,
  }
}

\applystyletotheorem{property}
\applystyletotheorem{proposition}
\applystyletotheorem{lemma}
\applystyletotheorem{theorem}
\applystyletotheorem{corollary}
\applystyletotheorem{definition}
\applystyletotheorem{notation}
\applystyletotheorem{example}
\applystyletotheorem{cexample}
\applystyletotheorem{application}
\applystyletotheorem{remark}
%\applystyletotheorem{proof}
\applystyletotheorem{algorithm}
\applystyletotheorem{exercice}

% Environnements :

\NewEnviron{whitetabularx}[1]{%
  \renewcommand{\arraystretch}{2.5}
  \colorbox{white}{%
    \begin{tabularx}{\textwidth}{#1}%
      \BODY%
    \end{tabularx}%
  }%
}

% Maths :

\DeclareFontEncoding{FMS}{}{}
\DeclareFontSubstitution{FMS}{futm}{m}{n}
\DeclareFontEncoding{FMX}{}{}
\DeclareFontSubstitution{FMX}{futm}{m}{n}
\DeclareSymbolFont{fouriersymbols}{FMS}{futm}{m}{n}
\DeclareSymbolFont{fourierlargesymbols}{FMX}{futm}{m}{n}
\DeclareMathDelimiter{\VERT}{\mathord}{fouriersymbols}{152}{fourierlargesymbols}{147}

% Code :

\definecolor{greencode}{rgb}{0,0.6,0}
\definecolor{graycode}{rgb}{0.5,0.5,0.5}
\definecolor{mauvecode}{rgb}{0.58,0,0.82}
\definecolor{bluecode}{HTML}{1976d2}
\lstset{
  basicstyle=\footnotesize\ttfamily,
  breakatwhitespace=false,
  breaklines=true,
  %captionpos=b,
  commentstyle=\color{greencode},
  deletekeywords={...},
  escapeinside={\%*}{*)},
  extendedchars=true,
  frame=none,
  keepspaces=true,
  keywordstyle=\color{bluecode},
  language=Python,
  otherkeywords={*,...},
  numbers=left,
  numbersep=5pt,
  numberstyle=\tiny\color{graycode},
  rulecolor=\color{black},
  showspaces=false,
  showstringspaces=false,
  showtabs=false,
  stepnumber=2,
  stringstyle=\color{mauvecode},
  tabsize=2,
  %texcl=true,
  xleftmargin=10pt,
  %title=\lstname
}

\newcommand{\codedirectory}{}
\newcommand{\inputalgorithm}[1]{%
  \begin{algorithm}%
    \strut%
    \lstinputlisting{\codedirectory#1}%
  \end{algorithm}%
}



% Bibliographie :

%\addbibresource{\bibliographypath}%
\defbibheading{bibliography}[\bibname]{\section*{#1}}
\renewbibmacro*{entryhead:full}{\printfield{labeltitle}}%
\DeclareFieldFormat{url}{\newline\footnotesize\url{#1}}%

\AtEndDocument{%
  \newpage%
  \pagestyle{empty}%
  \printbibliography%
}


\begin{document}
  %<*content>
  \lesson{algebra}{158}{Endomorphismes remarquables d'un espace vectoriel euclidien (de dimension finie).}

  Soit $E$ un espace vectoriel sur $\mathbb{R}$ de dimension finie $n$. On munit $E$ d'un produit scalaire $\langle . , . \rangle$, qui en fait un \textbf{espace euclidien}. On note $\Vert . \Vert$ la norme associée à ce produit scalaire.

  \subsection{Conséquences du caractère euclidien de \texorpdfstring{$E$}{E}}

  \subsubsection{Adjoint d'un endomorphisme}

  \reference[ROM21]{718}

  \begin{lemma}[Théorème de représentation de Riesz]
    \[ \forall \varphi \in E^*, \, \exists! a \in E \text{ tel que } \forall x \in E, \, \varphi(x) = \langle x, a \rangle \]
  \end{lemma}

  \begin{theorem}
    \[ \forall u \in \mathcal{L}(E), \, \exists! u^* \in \mathcal{L}(E) \text{ tel que } \forall x, y \in E, \, \langle u(x), y \rangle = \langle x, u^*(y) \rangle \]
  \end{theorem}

  \begin{definition}
    Avec les notations du théorème précédent, on dit que $u^*$ est \textbf{l'adjoint} de $u$.
  \end{definition}

  \begin{theorem}
    Soient $\mathcal{B} = (e_i)_{i \in I}$ une base de $E$ et $G = (\langle e_i, e_j \rangle)_{i,j \in \llbracket 1, n \rrbracket}$ la matrice de Gram correspondante. Si $u \in \mathcal{L}(E)$ a pour matrice $A$ dans la base $\mathcal{B}$, alors la matrice de $u^*$ dans la base $\mathcal{B}$ est
    \[ B = G^{-1} \tr{A} G \]
    En particulier, si $\mathcal{B}$ est orthonormée, on a $B = \tr{A}$.
  \end{theorem}

  \reference{748}

  \begin{proposition}
    \[ \forall u \in \mathcal{L}(E), \, \VERT u \VERT = \VERT u^* \VERT \]
    Il en résulte que l'application linéaire (cf. \cref{158-1}) $u \mapsto u^*$ est continue pour la norme $\VERT . \VERT$ subordonnée à $\Vert . \Vert$.
  \end{proposition}

  \subsubsection{Propriétés de l'adjoint}

  \reference{719}

  \begin{proposition}[Propriétés de $u \mapsto u^*$]
    \label{158-1}
    Soient $u, v \in \mathcal{L}(E)$. On a :
    \begin{enumerate}[label=(\roman*)]
      \item $\forall \lambda \in \mathbb{R}, (\lambda u + v)^* = \lambda u^* + v^*$.
      \item $(u^*)^* = u$.
      \item $(u \circ v)^* = v^* \circ u^*$.
      \item $u \in \mathrm{GL}(E) \implies u^* \in \mathrm{GL}(E)$, et $(u^*)^{-1} = (u^{-1})^*$.
    \end{enumerate}
  \end{proposition}

  \begin{proposition}[Propriétés de l'endomorphisme adjoint]
    Soit $u \in \mathcal{L}(E)$. On a :
    \begin{enumerate}[label=(\roman*)]
      \item $\det(u^*) = \det(u)$.
      \item $\ker(u^*) = \mathrm{Im}(u)^\perp$.
      \item $\mathrm{Im}(u^*) = \ker(u)^\perp$.
      \item $\rang(u^*) = \rang(u)$.
      \item Si $F$ est un sous-espace vectoriel de $E$ stable par $u$, alors $F^\perp$ est stable par $u^*$.
    \end{enumerate}
  \end{proposition}

  \reference{751}

  \begin{proposition}
    Soit $u \in \mathcal{L}(E)$.
    \[ u = 0 \iff \trace(u \circ u^*) = 0 \]
  \end{proposition}

  \subsection{Endomorphismes normaux}

  \reference{743}

  \begin{definition}
    Un endomorphisme $u \in \mathcal{L}(E)$ est dit \textbf{normal} s'il est tel que $u \circ u^* = u^* \circ u$.
  \end{definition}

  \begin{remark}
    En désignant par $A \in \mathcal{M}_n(\mathbb{R})$ la matrice de $u \in \mathcal{L}(E)$ dans une base orthonormée, $u$ est normal si et seulement si,
    \[ \tr{A} A = A \tr{A} \]
    ce qui se traduit en disant que la matrice $A$ est normale.
  \end{remark}

  \begin{example}
    Les endomorphismes symétriques, anti-symétriques (\cref{158-3}) et orthogonaux (\cref{158-4}) sont des endomorphismes normaux.
  \end{example}

  \reference{758}

  \begin{proposition}
    $u \in \mathcal{L}(E)$ est normal si et seulement si $\Vert u (x) \Vert = \Vert u^* (x) \Vert$ pour tout $x \in E$ où $\Vert . \Vert$ est une norme euclidienne.
  \end{proposition}

  \reference{743}

  \begin{proposition}
    Soit $u \in \mathcal{L}(E)$ un endomorphisme normal.
    \begin{enumerate}[label=(\roman*)]
      \item Si $F$ est un sous-espace vectoriel de $E$ stable par $u$, alors $F^\perp$ est stable par $u$.
      \item Il existe un sous-espace vectoriel de $E$ de dimension $1$ ou $2$ stable par $u$.
    \end{enumerate}
  \end{proposition}

  \begin{proposition}[Réduction dans le cas $n = 2$]
    \label{158-2}
    On suppose $n = 2$. Soit $u \in \mathcal{L}(E)$ un endomorphisme normal.
    \begin{itemize}
      \item \uline{Si $u$ a une valeur propre réelle :} $u$ est diagonalisable dans une base orthonormée.
      \item \uline{Sinon :} il existe $\mathcal{B}$ une base orthonormée de $E$ telle que la matrice de $u$ dans $\mathcal{B}$ est
      \[ R(a,b) = \begin{pmatrix} a & -b \\ b & a \end{pmatrix} \]
      avec $b \neq 0$.
    \end{itemize}
  \end{proposition}

  \begin{theorem}[Réduction des endomorphismes normaux]
    Soit $u \in \mathcal{L}(E)$ un endomorphisme normal. Alors, il existe $\mathcal{B}$ une base orthonormée de $E$ telle que la matrice de $u$ dans $\mathcal{B}$ est
    \[ \begin{pmatrix} D_p & 0 & 0 & \dots & 0 \\ 0 & R(a_1,b_1) & 0 & \dots & 0 \\ 0 & 0 & R(a_2,b_2) & \ddots & \vdots \\ \vdots & \ddots & \ddots & \ddots & 0 \\ 0 & \dots & \dots & 0 & R(a_r,b_r) \end{pmatrix} \]
    où $D_p$ est diagonale d'ordre $p$ et $R(a,b)$ est définie à la \cref{158-2}.
  \end{theorem}

  \subsection{Endomorphismes symétriques}

  \label{158-3}

  \subsubsection{Définitions et propriétés}

  \reference{732}

  \begin{definition}
    Un endomorphisme $u \in \mathcal{L}(E)$ est dit \textbf{symétrique} s'il est tel que $u^* = u$.
  \end{definition}

  \begin{proposition}
    Un endomorphisme $u \in \mathcal{L}(E)$ est symétrique si et seulement si sa matrice dans une base orthonormée est symétrique.
  \end{proposition}

  \begin{corollary}
    $\mathcal{S}(E)$ est un sous-espace vectoriel de $E$ de dimension $\frac{n(n+1)}{2}$.
  \end{corollary}

  \begin{proposition}
    Si $u \in \mathcal{S}(E)$, alors $u^p \in \mathcal{S}(E)$ pour tout entier naturel $p$, et $v^* \circ u \circ v \in \mathcal{S}(E)$ pour tout $v \in \mathcal{L}(E)$.
  \end{proposition}

  \begin{theorem}[Spectral]
    Tout endomorphisme symétrique $u \in \mathcal{S}(E)$ se diagonalise dans une base orthonormée.
  \end{theorem}

  \begin{corollary}
    Toute matrice symétrique réelle se diagonalise dans une base orthonormée.
  \end{corollary}

  \subsubsection{Endomorphismes symétriques positifs}

  \begin{definition}
    \begin{itemize}
      \item Un endomorphisme $u \in \mathcal{L}(E)$ est dit \textbf{symétrique positif} (resp. \textbf{symétrique défini positif}) s'il est symétrique tel que $\langle x, u(x) \rangle \geq 0$ (resp. $\langle x, u(x) \rangle > 0$) pour tout $x \in E$. On note $\mathcal{S}^+(E)$ (resp. $\mathcal{S}^{++}(E)$) l'ensemble des endomorphismes symétriques positifs (resp. symétriques définis positifs).
      \item Une matrice $A \in \mathcal{M}_n(\mathbb{R})$ est dite \textbf{symétrique positive} (resp. \textbf{symétrique définie positive}) si elle est symétrique telle que $\langle x, Ax \rangle \geq 0$ (resp. $\langle x, Ax \rangle > 0$) pour tout $x \in E$. On note $\mathcal{S}_n^+(\mathbb{R})$ (resp. $\mathcal{S}_n^{++}(\mathbb{R})$) l'ensemble des matrices symétriques positives (resp. symétriques définies positives).
    \end{itemize}
  \end{definition}

  \begin{theorem}
    Soit $u \in \mathcal{S}(E)$. Alors, $u \in \mathcal{S}^+(E)$ (resp. $u \in \mathcal{S}^{++}(E)$) si et seulement si toutes ses valeurs propres sont positives (resp. strictement positives).
  \end{theorem}

  \begin{corollary}
    Soit $A \in \mathcal{M}_n(\mathbb{R})$. Alors, $A \in \mathcal{S}^+_n(\mathbb{R})$ si et seulement s'il existe $B \in \mathcal{S}_n(\mathbb{R})$ telle que $A = \tr{B}B$.
  \end{corollary}

  \reference{752}

  \begin{example}
    \[ \begin{pmatrix} -1 & 1 & 1 \\ 1 & -1 & 1 \\ 1 & 1 & -1 \end{pmatrix} = P^{-1} \begin{pmatrix} 1 & 0 & 0 \\ 0 & -2 & 0 \\ 0 & 0 & -2 \end{pmatrix} P \text{ avec } P = \frac{1}{\sqrt{6}} \begin{pmatrix} \sqrt{2} & \sqrt{3} & 1 \\ \sqrt{2} & -\sqrt{3} & 1 \\ \sqrt{2} & 0 & -2 \end{pmatrix} \]
  \end{example}

  \reference[I-P]{182}

  \begin{lemma}
    Soit $M \in \mathcal{S}_n(\mathbb{R})$. Alors,
    \[ \VERT M \VERT = \rho(M) \]
    où $\rho$ est l'application qui a une matrice y associe son rayon spectral.
  \end{lemma}

  \dev{homeomorphisme-de-l-exponentielle}

  \begin{theorem}
    L'application $\exp : \mathcal{S}_n(\mathbb{R}) \rightarrow \mathcal{S}^{++}_n(\mathbb{R})$ est un homéomorphisme.
  \end{theorem}

  \subsubsection{Endomorphismes antisymétriques}

  \reference[ROM21]{718}

  \begin{definition}
    Un endomorphisme $u \in \mathcal{L}(E)$ est dit \textbf{anti-symétrique} s'il est tel que $u^* = -u$.
  \end{definition}

  \reference{746}

  \begin{theorem}
    Soit $u \in \mathcal{L}(E)$ un endomorphisme anti-symétrique. Alors, les valeurs propres de $u$ sont imaginaires pures (éventuellement nulles) et il existe $\mathcal{B}$ une base orthonormée de $E$ telle que la matrice de $u$ dans $\mathcal{B}$ est
    \[ \begin{pmatrix} D_p & 0 & 0 & \dots & 0 \\ 0 & R(0,b_1) & 0 & \dots & 0 \\ 0 & 0 & R(0,b_2) & \ddots & \vdots \\ \vdots & \ddots & \ddots & \ddots & 0 \\ 0 & \dots & \dots & 0 & R(0,b_r) \end{pmatrix} \]
    où $D_p$ est diagonale d'ordre $p$ et $R(a,b)$ est définie à la \cref{158-2}.
  \end{theorem}

  \subsection{Endomorphismes orthogonaux}

  \label{158-4}

  \subsubsection{Le groupe orthogonal}

  \reference{720}

  \begin{definition}
    Un endomorphisme $u \in \mathcal{L}(E)$ est dit \textbf{orthogonal} (ou est une \textbf{isométrie}) s'il est tel que $\langle u(x), u(y) \rangle = \langle x, y \rangle$ pour tout $x, y \in E$. On note $\mathcal{O}(E)$ l'ensemble des endomorphismes orthogonaux de $E$.
  \end{definition}

  \begin{example}
    \begin{itemize}
      \item Les seules homothéties qui sont des isométries sont $-\operatorname{id}_E$ et $\operatorname{id}_E$.
      \item Si $n = 1$, on a $\mathcal{O}(E) = \{ \pm \operatorname{id}_E \}$.
    \end{itemize}
  \end{example}

  \reference{743}

  \begin{proposition}
    Soit $u \in \mathcal{L}(E)$.
    \[ u = \mathcal{O}(E) \iff \forall x \in E, \, \Vert u(x) \Vert \iff u \in \mathrm{GL}(E) \text{ et } u^{-1} = u^* \]
  \end{proposition}

  \reference{721}

  \begin{theorem}
    Les isométries sont des automorphismes. Il en résulte que $\mathcal{O}(E)$ est un sous-groupe de $\mathrm{GL}(E)$.
  \end{theorem}

  \begin{remark}
    Ce n'est pas vrai en dimension infinie.
  \end{remark}

  \begin{theorem}
    Un endomorphisme de $E$ est une isométrie si et seulement s'il transforme toute base orthonormée de $E$ en une base orthonormée.
  \end{theorem}

  \begin{theorem}
    Un endomorphisme de $E$ est une isométrie si et seulement si sa matrice $A$ dans une base orthonormée est inversible, d'inverse $\tr{A}$.
    \newpar
    On dit alors que $A$ est \textbf{orthogonale}.
  \end{theorem}

  \begin{notation}
    On note $\mathcal{O}_n(\mathbb{R})$ le groupe des matrices orthogonales.
  \end{notation}

  \begin{theorem}
    \[ \forall u \in \mathcal{O}(E), \, \det(u) = \pm 1 \]
  \end{theorem}

  \begin{remark}
    On a des résultats équivalents pour les matrices.
  \end{remark}

  \begin{theorem}[Réduction des endomorphismes orthogonaux]
    Soit $u \in \mathcal{O}(E)$. Alors, il existe $\mathcal{B}$ une base orthonormée de $E$ telle que la matrice de $u$ dans $\mathcal{B}$ est
    \[ \begin{pmatrix} I_p & 0 & 0 & \dots & 0 \\ 0 & -I_q & 0 & \dots & 0 \\ 0 & 0 & R_1 & \ddots & \vdots \\ \vdots & \ddots & \ddots & \ddots & 0 \\ 0 & \dots & \dots & 0 & R_r \end{pmatrix} \]
    où $R_i = R(\cos(\theta_i), \sin(\theta_i))$ avec $R(a,b)$ définie à la \cref{158-2} et $\forall i \in \llbracket 1, r \rrbracket, \, \theta_i \in ]0,2\pi[$.
  \end{theorem}

  \reference[C-G]{376}

  \begin{lemma}
    \[ \forall A \in \mathcal{S}_n^{++}(\mathbb{R}) \, \exists! B \in \mathcal{S}_n^{++}(\mathbb{R}) \text{ telle que } B^2 = A \]
  \end{lemma}

  \dev{decomposition-polaire}

  \begin{theorem}[Décomposition polaire]
    L'application
    \[ \mu :
    \begin{array}{ccc}
      \mathcal{O}_n(\mathbb{R}) \times \mathcal{S}_n^{++}(\mathbb{R}) &\rightarrow& \mathrm{GL}_n(\mathbb{R}) \\
      (O, S) &\mapsto& OS
    \end{array}
    \]
    est un homéomorphisme.
  \end{theorem}

  \subsubsection{Étude en dimensions \texorpdfstring{$2$}{2} et \texorpdfstring{$3$}{3}}

  \reference[GRI]{241}

  \begin{definition}
    On définit $\mathrm{SO}(E) = \{ u \in \mathcal{O}(E) \mid \det(u) = 1 \}$ et $\mathrm{SO}_n(\mathbb{R}) = \{ A \in \mathcal{O}_n(\mathbb{R}) \mid \det(A) = 1 \}$
  \end{definition}

  \reference[ROM21]{724}

  \begin{proposition}
    $\mathrm{SO}(E)$ est un sous-groupe distingué de $\mathcal{O}(E)$ d'indice $2$ (de même que $\mathrm{SO}_n(\mathbb{R})$ dans $\mathcal{O}_n(\mathbb{R})$).
  \end{proposition}

  \reference[GRI]{241}

  \begin{example}
    \[ \frac{1}{3} \begin{pmatrix} 2 & -1 & 2 \\ 2 & 2 & -1 \\ -1 & 2 & 2 \end{pmatrix} \in \mathrm{SO}_3(\mathbb{R}) \]
  \end{example}

  \begin{theorem}
    Soit $A \in \mathcal{O}_2(\mathbb{R})$. Alors :
    \begin{itemize}
      \item \uline{Si $A \in \mathrm{SO}_2(\mathbb{R})$ :}
      \[ \exists \theta \in \mathbb{R} \text{ tel que } A = \begin{pmatrix} \cos(\theta) & -\sin(\theta) \\ \sin(\theta) & \cos(\theta) \end{pmatrix} \]
      (rotation d'angle $\theta$).
      \item \uline{Si $A \notin \mathrm{SO}_2(\mathbb{R})$ :}
      \[ \exists \theta \in \mathbb{R} \text{ tel que } A = \begin{pmatrix} \cos(\theta) & \sin(\theta) \\ \sin(\theta) & -\cos(\theta) \end{pmatrix} \]
      (symétrie orthogonale par rapport à la droite d'angle polaire $\frac{\theta}{2}$).
    \end{itemize}
  \end{theorem}

  \begin{theorem}
    On suppose $n = 3$. Soit $A \in \mathcal{O}_3(\mathbb{R})$ et $u$ l'endomorphisme de $E$ dont la matrice dans la base canonique est $A$. Alors, il existe $\mathcal{B}$ une base orthonormée de $E$ telle que la matrice de $u$ dans $\mathcal{B}$ est
    \[ \begin{pmatrix} \cos(\theta) & -\sin(\theta) & 0 \\ \sin(\theta) & \cos(\theta) & 0 \\ 0 & 0 & \epsilon \end{pmatrix} \]
    avec $\epsilon = \pm 1$. On note $E_\epsilon$ le sous-espace vectoriel associé à la valeur propre $\epsilon$.
    \begin{itemize}
      \item \uline{Si $\epsilon = 1$ :} $f \in \mathrm{SO}(E)$ est la rotation d'angle $2\cos(\theta) + 1$ autour de l'axe $E_1$.
      \item \uline{Si $\epsilon = -1$ :} $f \notin \mathrm{SO}(E)$ est la composée de la rotation d'angle $2\cos(\theta) - 1$ autour de l'axe $E_{-1}$ avec la symétrie orthogonale par rapport à $E_{-1}^{\perp}$.
    \end{itemize}
  \end{theorem}

  \subsubsection{Propriétés topologiques}

  \reference[ROM21]{722}

  \begin{proposition}
    $\mathcal{O}(E)$ est une partie compacte de $\mathcal{L}(E)$.
  \end{proposition}

  \begin{proposition}
    $\mathrm{SO}(E)$ est connexe dans $\mathcal{O}(E)$.
  \end{proposition}

  \begin{corollary}
    $\mathcal{O}(E)$ est non-connexe. Ses composantes connexes sont $\mathrm{SO}(E)$ et $\{ u \in \mathcal{O}(E) \mid \det(u) = -1 \}$.
  \end{corollary}

  \reference{756}

  \begin{proposition}
    Tout sous-groupe compact de $\mathrm{GL}(E)$ qui contient $\mathcal{O}(E)$ est égal à $\mathcal{O}(E)$.
  \end{proposition}

  \newpage
  \section*{Annexes}

  \reference[GRI]{242}

  \begin{figure}[H]
    \begin{center}
      \begin{tikzpicture}
        \draw[->] (-1, 0) -- (5, 0);
        \draw[->] (0, -1) -- (0, 5);
        \draw [teal,dashed,domain=-10:100] plot ({4*cos(\x)}, {4*sin(\x)});
        \draw [->,color=teal,domain=10:60] plot ({cos(\x)}, {sin(\x)});
        \node at (35:1.3) {\color{teal}$\theta$};
        \draw (0,0) -- (10:4) node {\color{orange}$\bullet$} node[right] {\color{orange}$x$};
        \draw (0,0) -- (60:4) node {\color{orange}$\bullet$} node[above right] {\color{orange}$Ax$};
        \node at (2,-1) [below] {$A \in \mathrm{SO}_2(\mathbb{R})$};
      \end{tikzpicture}
      \hfill
      \begin{tikzpicture}
        \coordinate (A) at (20:4);
        \coordinate (B) at (80:4);
        \draw[->] (-1, 0) -- (5, 0);
        \draw[->] (0, -1) -- (0, 5);
        \draw[dashed] (A) -- (B);
        \draw[teal] (0, 0) -- (50:5) node [above right] {\color{teal}$\frac{\theta}{2}$};
        \draw (0,0) -- (A) node {\color{orange}$\bullet$} node[right] {\color{orange}$x$};
        \draw (0,0) -- (B) node {\color{orange}$\bullet$} node[above right] {\color{orange}$Ax$};
        \node at (2,-1) [below] {$A \notin \mathrm{SO}_2(\mathbb{R})$};
      \end{tikzpicture}
    \end{center}
    \caption{Le groupe $\mathcal{O}_2(\mathbb{R})$.}
  \end{figure}

  \reference{244}

  \begin{figure}[H]
    \begin{center}
      \begin{tikzpicture}
        \draw[->] (-1, 0, 0) -- (5, 0, 0);
        \draw[->] (0, -1, 0) -- (0, 5, 0) node [above] {$E_1$};
        \draw[->] (0, 0, -1) -- (0, 0, 5);
        \begin{scope}[canvas is zx plane at y=4]
          \coordinate (O) at (0,0);
          \draw[teal,dashed] (O) circle (2);
          \coordinate (A) at (10:2);
          \coordinate (B) at (60:2);
          \draw [->,color=teal,domain=10:60] plot ({cos(\x)}, {sin(\x)});
          \node at (35:1.5) {\color{teal}$\theta$};
          \draw (O) -- (A);
          \draw (O) -- (B);
        \end{scope}
        \draw (0,0,0) -- (A) node {\color{orange}$\bullet$} node[above left] {\color{orange}$x$};
        \draw (0,0,0) -- (B) node {\color{orange}$\bullet$} node[above] {\color{orange}$Ax$};
        \node at (2,-1) [below] {$A \in \mathrm{SO}_3(\mathbb{R})$};
      \end{tikzpicture}
      \hfill
      \begin{tikzpicture}
        \draw[->] (-1, 0, 0) -- (5, 0, 0);
        \draw[->] (0, -1, 0) -- (0, 5, 0) node [above] {$E_{-1}$};
        \draw[->] (0, 0, -1) -- (0, 0, 5);
        \begin{scope}[canvas is zx plane at y=4]
          \coordinate (O) at (0,0);
          \draw[teal,dashed] (O) circle (2);
          \coordinate (A) at (10:2);
          \coordinate (B) at (60:2);
          \draw [->,color=teal,domain=10:60] plot ({cos(\x)}, {sin(\x)});
          \node at (35:1.5) {\color{teal}$\theta$};
          \draw (O) -- (A);
          \draw (O) -- (B);
        \end{scope}
        \begin{scope}[canvas is zx plane at y=0]
          \coordinate (C) at (60:2);
        \end{scope}
        \draw (0,0,0) -- (A) node {\color{orange}$\bullet$} node[above left] {\color{orange}$x$};
        \draw[dashed] (0,0,0) -- (B) node {\color{orange}$\bullet$};
        \draw[dashed] (B) -- (C);
        \draw (0,0,0) -- (C) node {\color{orange}$\bullet$} node[below right] {\color{orange}$Ax$};
        \node at (2,-1) [below] {$A \notin \mathrm{SO}_3(\mathbb{R})$};
      \end{tikzpicture}
    \end{center}
    \caption{Le groupe $\mathcal{O}_3(\mathbb{R})$.}
  \end{figure}
  %</content>
\end{document}
