\documentclass[12pt, a4paper]{report}

% LuaLaTeX :

\RequirePackage{iftex}
\RequireLuaTeX

% Packages :

\usepackage[french]{babel}
%\usepackage[utf8]{inputenc}
%\usepackage[T1]{fontenc}
\usepackage[pdfencoding=auto, pdfauthor={Hugo Delaunay}, pdfsubject={Mathématiques}, pdfcreator={agreg.skyost.eu}]{hyperref}
\usepackage{amsmath}
\usepackage{amsthm}
%\usepackage{amssymb}
\usepackage{stmaryrd}
\usepackage{tikz}
\usepackage{tkz-euclide}
\usepackage{fontspec}
\defaultfontfeatures[Erewhon]{FontFace = {bx}{n}{Erewhon-Bold.otf}}
\usepackage{fourier-otf}
\usepackage[nobottomtitles*]{titlesec}
\usepackage{fancyhdr}
\usepackage{listings}
\usepackage{catchfilebetweentags}
\usepackage[french, capitalise, noabbrev]{cleveref}
\usepackage[fit, breakall]{truncate}
\usepackage[top=2.5cm, right=2cm, bottom=2.5cm, left=2cm]{geometry}
\usepackage{enumitem}
\usepackage{tocloft}
\usepackage{microtype}
%\usepackage{mdframed}
%\usepackage{thmtools}
\usepackage{xcolor}
\usepackage{tabularx}
\usepackage{xltabular}
\usepackage{aligned-overset}
\usepackage[subpreambles=true]{standalone}
\usepackage{environ}
\usepackage[normalem]{ulem}
\usepackage{etoolbox}
\usepackage{setspace}
\usepackage[bibstyle=reading, citestyle=draft]{biblatex}
\usepackage{xpatch}
\usepackage[many, breakable]{tcolorbox}
\usepackage[backgroundcolor=white, bordercolor=white, textsize=scriptsize]{todonotes}
\usepackage{luacode}
\usepackage{float}
\usepackage{needspace}
\everymath{\displaystyle}

% Police :

\setmathfont{Erewhon Math}

% Tikz :

\usetikzlibrary{calc}
\usetikzlibrary{3d}

% Longueurs :

\setlength{\parindent}{0pt}
\setlength{\headheight}{15pt}
\setlength{\fboxsep}{0pt}
\titlespacing*{\chapter}{0pt}{-20pt}{10pt}
\setlength{\marginparwidth}{1.5cm}
\setstretch{1.1}

% Métadonnées :

\author{agreg.skyost.eu}
\date{\today}

% Titres :

\setcounter{secnumdepth}{3}

\renewcommand{\thechapter}{\Roman{chapter}}
\renewcommand{\thesubsection}{\Roman{subsection}}
\renewcommand{\thesubsubsection}{\arabic{subsubsection}}
\renewcommand{\theparagraph}{\alph{paragraph}}

\titleformat{\chapter}{\huge\bfseries}{\thechapter}{20pt}{\huge\bfseries}
\titleformat*{\section}{\LARGE\bfseries}
\titleformat{\subsection}{\Large\bfseries}{\thesubsection \, - \,}{0pt}{\Large\bfseries}
\titleformat{\subsubsection}{\large\bfseries}{\thesubsubsection. \,}{0pt}{\large\bfseries}
\titleformat{\paragraph}{\bfseries}{\theparagraph. \,}{0pt}{\bfseries}

\setcounter{secnumdepth}{4}

% Table des matières :

\renewcommand{\cftsecleader}{\cftdotfill{\cftdotsep}}
\addtolength{\cftsecnumwidth}{10pt}

% Redéfinition des commandes :

\renewcommand*\thesection{\arabic{section}}
\renewcommand{\ker}{\mathrm{Ker}}

% Nouvelles commandes :

\newcommand{\website}{https://github.com/imbodj/SenCoursDeMaths}

\newcommand{\tr}[1]{\mathstrut ^t #1}
\newcommand{\im}{\mathrm{Im}}
\newcommand{\rang}{\operatorname{rang}}
\newcommand{\trace}{\operatorname{trace}}
\newcommand{\id}{\operatorname{id}}
\newcommand{\stab}{\operatorname{Stab}}
\newcommand{\paren}[1]{\left(#1\right)}
\newcommand{\croch}[1]{\left[ #1 \right]}
\newcommand{\Grdcroch}[1]{\Bigl[ #1 \Bigr]}
\newcommand{\grdcroch}[1]{\bigl[ #1 \bigr]}
\newcommand{\abs}[1]{\left\lvert #1 \right\rvert}
\newcommand{\limi}[3]{\lim_{#1\to #2}#3}
\newcommand{\pinf}{+\infty}
\newcommand{\minf}{-\infty}
%%%%%%%%%%%%%% ENSEMBLES %%%%%%%%%%%%%%%%%
\newcommand{\ensemblenombre}[1]{\mathbb{#1}}
\newcommand{\Nn}{\ensemblenombre{N}}
\newcommand{\Zz}{\ensemblenombre{Z}}
\newcommand{\Qq}{\ensemblenombre{Q}}
\newcommand{\Qqp}{\Qq^+}
\newcommand{\Rr}{\ensemblenombre{R}}
\newcommand{\Cc}{\ensemblenombre{C}}
\newcommand{\Nne}{\Nn^*}
\newcommand{\Zze}{\Zz^*}
\newcommand{\Zzn}{\Zz^-}
\newcommand{\Qqe}{\Qq^*}
\newcommand{\Rre}{\Rr^*}
\newcommand{\Rrp}{\Rr_+}
\newcommand{\Rrm}{\Rr_-}
\newcommand{\Rrep}{\Rr_+^*}
\newcommand{\Rrem}{\Rr_-^*}
\newcommand{\Cce}{\Cc^*}
%%%%%%%%%%%%%%  INTERVALLES %%%%%%%%%%%%%%%%%
\newcommand{\intff}[2]{\left[#1\;,\; #2\right]  }
\newcommand{\intof}[2]{\left]#1 \;, \;#2\right]  }
\newcommand{\intfo}[2]{\left[#1 \;,\; #2\right[  }
\newcommand{\intoo}[2]{\left]#1 \;,\; #2\right[  }

\providecommand{\newpar}{\\[\medskipamount]}

\newcommand{\annexessection}{%
  \newpage%
  \subsection*{Annexes}%
}

\providecommand{\lesson}[3]{%
  \title{#3}%
  \hypersetup{pdftitle={#2 : #3}}%
  \setcounter{section}{\numexpr #2 - 1}%
  \section{#3}%
  \fancyhead[R]{\truncate{0.73\textwidth}{#2 : #3}}%
}

\providecommand{\development}[3]{%
  \title{#3}%
  \hypersetup{pdftitle={#3}}%
  \section*{#3}%
  \fancyhead[R]{\truncate{0.73\textwidth}{#3}}%
}

\providecommand{\sheet}[3]{\development{#1}{#2}{#3}}

\providecommand{\ranking}[1]{%
  \title{Terminale #1}%
  \hypersetup{pdftitle={Terminale #1}}%
  \section*{Terminale #1}%
  \fancyhead[R]{\truncate{0.73\textwidth}{Terminale #1}}%
}

\providecommand{\summary}[1]{%
  \textit{#1}%
  \par%
  \medskip%
}

\tikzset{notestyleraw/.append style={inner sep=0pt, rounded corners=0pt, align=center}}

%\newcommand{\booklink}[1]{\website/bibliographie\##1}
\newcounter{reference}
\newcommand{\previousreference}{}
\providecommand{\reference}[2][]{%
  \needspace{20pt}%
  \notblank{#1}{
    \needspace{20pt}%
    \renewcommand{\previousreference}{#1}%
    \stepcounter{reference}%
    \label{reference-\previousreference-\thereference}%
  }{}%
  \todo[noline]{%
    \protect\vspace{20pt}%
    \protect\par%
    \protect\notblank{#1}{\cite{[\previousreference]}\\}{}%
    \protect\hyperref[reference-\previousreference-\thereference]{p. #2}%
  }%
}

\definecolor{devcolor}{HTML}{00695c}
\providecommand{\dev}[1]{%
  \reversemarginpar%
  \todo[noline]{
    \protect\vspace{20pt}%
    \protect\par%
    \bfseries\color{devcolor}\href{\website/developpements/#1}{[DEV]}
  }%
  \normalmarginpar%
}

% En-têtes :

\pagestyle{fancy}
\fancyhead[L]{\truncate{0.23\textwidth}{\thepage}}
\fancyfoot[C]{\scriptsize \href{\website}{\texttt{https://github.com/imbodj/SenCoursDeMaths}}}

% Couleurs :

\definecolor{property}{HTML}{ffeb3b}
\definecolor{proposition}{HTML}{ffc107}
\definecolor{lemma}{HTML}{ff9800}
\definecolor{theorem}{HTML}{f44336}
\definecolor{corollary}{HTML}{e91e63}
\definecolor{definition}{HTML}{673ab7}
\definecolor{notation}{HTML}{9c27b0}
\definecolor{example}{HTML}{00bcd4}
\definecolor{cexample}{HTML}{795548}
\definecolor{application}{HTML}{009688}
\definecolor{remark}{HTML}{3f51b5}
\definecolor{algorithm}{HTML}{607d8b}
%\definecolor{proof}{HTML}{e1f5fe}
\definecolor{exercice}{HTML}{e1f5fe}

% Théorèmes :

\theoremstyle{definition}
\newtheorem{theorem}{Théorème}

\newtheorem{property}[theorem]{Propriété}
\newtheorem{proposition}[theorem]{Proposition}
\newtheorem{lemma}[theorem]{Activité d'introduction}
\newtheorem{corollary}[theorem]{Conséquence}

\newtheorem{definition}[theorem]{Définition}
\newtheorem{notation}[theorem]{Notation}

\newtheorem{example}[theorem]{Exemple}
\newtheorem{cexample}[theorem]{Contre-exemple}
\newtheorem{application}[theorem]{Application}

\newtheorem{algorithm}[theorem]{Algorithme}
\newtheorem{exercice}[theorem]{Exercice}

\theoremstyle{remark}
\newtheorem{remark}[theorem]{Remarque}

\counterwithin*{theorem}{section}

\newcommand{\applystyletotheorem}[1]{
  \tcolorboxenvironment{#1}{
    enhanced,
    breakable,
    colback=#1!8!white,
    %right=0pt,
    %top=8pt,
    %bottom=8pt,
    boxrule=0pt,
    frame hidden,
    sharp corners,
    enhanced,borderline west={4pt}{0pt}{#1},
    %interior hidden,
    sharp corners,
    after=\par,
  }
}

\applystyletotheorem{property}
\applystyletotheorem{proposition}
\applystyletotheorem{lemma}
\applystyletotheorem{theorem}
\applystyletotheorem{corollary}
\applystyletotheorem{definition}
\applystyletotheorem{notation}
\applystyletotheorem{example}
\applystyletotheorem{cexample}
\applystyletotheorem{application}
\applystyletotheorem{remark}
%\applystyletotheorem{proof}
\applystyletotheorem{algorithm}
\applystyletotheorem{exercice}

% Environnements :

\NewEnviron{whitetabularx}[1]{%
  \renewcommand{\arraystretch}{2.5}
  \colorbox{white}{%
    \begin{tabularx}{\textwidth}{#1}%
      \BODY%
    \end{tabularx}%
  }%
}

% Maths :

\DeclareFontEncoding{FMS}{}{}
\DeclareFontSubstitution{FMS}{futm}{m}{n}
\DeclareFontEncoding{FMX}{}{}
\DeclareFontSubstitution{FMX}{futm}{m}{n}
\DeclareSymbolFont{fouriersymbols}{FMS}{futm}{m}{n}
\DeclareSymbolFont{fourierlargesymbols}{FMX}{futm}{m}{n}
\DeclareMathDelimiter{\VERT}{\mathord}{fouriersymbols}{152}{fourierlargesymbols}{147}

% Code :

\definecolor{greencode}{rgb}{0,0.6,0}
\definecolor{graycode}{rgb}{0.5,0.5,0.5}
\definecolor{mauvecode}{rgb}{0.58,0,0.82}
\definecolor{bluecode}{HTML}{1976d2}
\lstset{
  basicstyle=\footnotesize\ttfamily,
  breakatwhitespace=false,
  breaklines=true,
  %captionpos=b,
  commentstyle=\color{greencode},
  deletekeywords={...},
  escapeinside={\%*}{*)},
  extendedchars=true,
  frame=none,
  keepspaces=true,
  keywordstyle=\color{bluecode},
  language=Python,
  otherkeywords={*,...},
  numbers=left,
  numbersep=5pt,
  numberstyle=\tiny\color{graycode},
  rulecolor=\color{black},
  showspaces=false,
  showstringspaces=false,
  showtabs=false,
  stepnumber=2,
  stringstyle=\color{mauvecode},
  tabsize=2,
  %texcl=true,
  xleftmargin=10pt,
  %title=\lstname
}

\newcommand{\codedirectory}{}
\newcommand{\inputalgorithm}[1]{%
  \begin{algorithm}%
    \strut%
    \lstinputlisting{\codedirectory#1}%
  \end{algorithm}%
}



\everymath{\displaystyle}
\begin{document}
  %<*content>
  \lesson{algebra}{13}{Équations différentielles}
\textbf{Les équations différentielles : un outil fondamental}

Les équations différentielles occupent une place centrale en analyse mathématique depuis plusieurs siècles. Elles permettent de modéliser de nombreux phénomènes réels dans des domaines variés tels que la physique, la mécanique, l’astronomie, la chimie, la biologie ou encore l’économie.

Historiquement, elles sont nées des besoins de la physique, où il s’agissait de décrire l’évolution de systèmes dynamiques. Des mathématiciens célèbres comme Clairaut, Bernoulli, d’Alembert, Lagrange, Cauchy ou encore Lipschitz ont contribué à leur développement.

\medskip

Dans ce chapitre, nous nous concentrerons sur un cas particulier : les équations différentielles linéaires à coefficients constants et à second membre nul, qui sont à la fois simples à manipuler et riches en applications.

\begin{lemma}
Une expérience consiste à étudier l'évolution d'une population de bactéries.\\ On désigne par $ N_{0} $  le nombre de bactéries à l'instant $ t=0 $,  $ N(t) $ le nombre de bactéries à l'instant $ t $
 et  on note $ N'(t) $ la vitesse instantanée  d'évolution   des bactéries l'instant $ t $.
 \begin{enumerate}
 \item On constate que $ N(t)=9000\eexp{-0,4t} $.
 \begin{enumerate}
 \item Donner le nombre de bactéries à l'instant $ t=0 $, $ t=10 $ et $ t=20 $.
 \item Donner une relation R entre $N $ et $N' $.
 \item Déterminer la vitesse instantanée  d'évolution  aux instants $ t=10 $ et $ t=20 $.
 \end{enumerate}
 \end{enumerate}

\end{lemma}
 
 
 \subsection{Généralités}
\begin{definition}
  On appelle  \textbf{équation différentielle}, toute équation ayant pour inconnue une fonction, dans laquelle figure au moins une des dérivées successives de la fonction inconnue.
\end{definition}


\begin{notation}
La fonction inconnue est souvent notée $ y $ et ses dérivées successives $\;  y^{\prime}$, $y^{\prime \prime} ,\quad   y^{\prime \prime \prime} \cdots $
\end{notation}

\begin{example}
 $ y^{\prime } -5y=0,\quad  2y^{\prime \prime} -y^{\prime}+3y=x-3\; $ sont des équation différentielles.
\end{example}

\textbf{Vocabulaire}

  \begin{itemize}
\item  Une équation différentielle est dite \textbf{d'ordre n}  lorsque le plus grand ordre des dérivées intervenant dans cette équation est $ n $.
\item  Une équation différentielle est dite \textbf{linéaire}  si elle ne contient pas de puissances de $ y$ ou  $ y^{\prime}$, $y^{\prime \prime} $  ou  $y^{\prime \prime \prime} \cdots $ d'exposants supérieurs ou égaux à 2 , ni leur produit.
\item  Une équation différentielle est dite \textbf{homogène}  lorsque le second membre de cette équation est nul.\\
 Ainsi, \; $ 2y^{\prime \prime} -y^{\prime}+3y=0$ est une équation différentielle linéaire homogène d'ordre 2.\\
 L'expression << sans second membre >>  est un abus de langage qui signifie que le second membre  est nul.
\item Toute fonction vérifiant une équation différentielle  sur un intervalle $ I $  est appelée \textbf{solution sur $ I $} de cette équation différentielle.
\item \textbf{Résoudre ou (intégrer)} une équation différentielle sur un intervalle ouvert $ I $, c'est déterminer l'ensemble des solutions sur $ I $ de cette équation différentielle.
\end{itemize}


\subsection{Équations différentielles du premier ordre}
\begin{definition}
Une équation différentielle linéaire homogène du 1\up{er} ordre  à coefficients constants  est une équation du type  $ y^{\prime} -ay=0$\; ou toute équation s'y ramenant,  $a $ et $ b$ sont des constantes réelles et $ y $ une fonction inconnue à déterminer, définie et dérivable sur un intervalle $ I$ de  $ \Rr.$
\end{definition}
\subsection*{Résolution}
On se propose de résoudre sur $ \Rr $ l'équation différentielle (E) $ y^{\prime} -ay=0$\; ($ a\in \Rr) $\\
$ \bullet $ La fonction nulle est solution de (E) \;(évident).\\
Soit $ y $  une solution de (E) ne s'annulant pas sur $ \Rr.$
\begin{align*}
\text{On \;a:\;}   y^{\prime} -ay=0& \Longleftrightarrow  \frac{y^{\prime}}{y}=a\\
&\Longrightarrow \exists c\in \Rr, \; \ln\abs{y}=ax+c \\
&\Longrightarrow \exists c\in \Rr, \abs{y}=\eexp{c}\eexp{ax}.
\end{align*}
La fonction $ y $ ne s'annule pas sur $ \Rr $; donc elle est de signe constant.\\ On en déduit que:  $ \exists k\in \Rre,\text{tel que}\; y=k\eexp{ax} $.\\  \textbf{ Ainsi, en ajoutant la fonction nulle, les fonctions \; $ x\longmapsto k\eexp{ax} $\;  $ (k\in \Rr) $\; sont solutions de (E)}.\\
$ \bullet $ Démontrons que toute solution de (E) est de  cette forme.\\Soit
$ y $  une solution de (E) et $ h $ la fonction \;$ x  \longmapsto y(x)\eexp{-ax} $\\La fonction $ h $ est dérivable sur $ \Rr $   et sa dérivée est la fonction $ h^{\prime}: x\longmapsto \croch{y^{\prime}(x)-ay(x)}\eexp{-ax} $\\ Or \; $ y^{\prime}-ay=0 $\; donc $ h^{\prime} $ est la fonction nulle et $ h $ est une fonction constante.\\ Il existe un nombre réel  $ k $ tel que: $ \forall x\in\Rr,\; y(x)\eexp{-ax}=k $\\ c'est à dire: $ \forall x\in\Rr, \; y(x)=k\eexp{ax} $;\\ Donc toute solution de (E) est de la forme\; $ x\longmapsto k\eexp{ax}\; (k\in\Rr) $\\
De cette étude, on en déduit la propriété suivante:

\medskip

\begin{property}
Les solutions sur $ \Rr $ de l'équation différentielle\;$ y^{\prime}-ay=0 $\; sont les fonctions \; $ x\mapsto k\eexp{ax}\; (k\in\Rr) $.
\end{property}

\begin{example}
Résoudre sur $ \Rr $ les équations différentielles\;$ y^{\prime}-y=0 $\;  et\; $ y^{\prime}+3y=0 $
\end{example}
\begin{proof}
$ \bullet $ Les solutions sur $ \Rr $ de l'équation différentielle\;$ y^{\prime}-y=0 $  sont les fonctions  $ x\mapsto k\eexp{x}\; (k\in\Rr) $\\
$ \bullet $ Les solutions sur $ \Rr $ de l'équation différentielle\\                                         $ y^{\prime}+3y=0 $\;sont les fonctions  $ x\mapsto k\eexp{-3x}\; (k\in\Rr) $
\end{proof}

\subsection*{Solution vérifiant une condition initiale}

Reprenons l'équation différentielle (E)  $\quad  y^{\prime}-ay=0 $\\
Soit $ x_{0}$ et $y_{0} $  deux nombres réels.\\
On se propose de déterminer les solutions  $ y $ sur $ \Rr $  de (E)  vérifiant \; $ y(x_{0})=y_{0} $.
\begin{align*}
\text{On \;a }\;: y(x_{0})=y_{0} &\Leftrightarrow k\eexp{ax_{0}}=y_{0}\\
&\Leftrightarrow k=y_{0}\eexp{-ax_{0}}\quad \text{unique}
\end{align*}
Donc la fonction \; $ x\mapsto y_{0}\eexp{-a(x-x_{0})} $\; est l'unique solution sur $ \Rr $  de (E)  vérifiant \; $ y(x_{0})=y_{0} $.\\
On en déduit la propriété suivante:

\begin{property}
Pour tout couple \;$ (x_{0}$ , $y_{0}) $\; de nombres réels, l'équation différentielle  \; $ y^{\prime}-ay=0 \;(a\in\Rr)$\; admet une unique solution $ y $ sur $ \Rr $  qui prend la valeur $y_{0} $ en $x_{0} $.
\end{property}

\begin{exercice}
Résoudre sur $ \Rr $ l'équation  différentielle \; $ y^{\prime}-\frac{1}{3}y=0 $\\ 
En déduire la solution qui prend la valeur $1 $ en $ \ln 8$.

\end{exercice}
\begin{proof}
Les solutions sur $ \Rr $ de l'équation  différentielle \; $ y^{\prime}-\frac{1}{3}y=0 $\; ; \; sont les fonctions  $ x\mapsto k\eexp{-\frac{1}{3}x}\; (k\in\Rr) $\\
Déterminons parmi ces solutions celle qui prend la valeur $1 $ en $ \ln 8$.\\ On a:\; $ 1=k\eexp{-\frac{1}{3}\ln 8} \Longleftrightarrow k=2$\\
Donc la fonction \;$f:  x\mapsto 2\eexp{-\frac{1}{3}x} $  est l'unique solution $ \Rr $  vérifiant \;$ f(\ln 8)=1 $. C'est celle  dont la courbe passe par le point de coordonnées \;$ (1, \ln 8) $.
\end{proof}
\begin{remark}
Les équations différentielles du type $\;  y^{\prime}=ay \; $ où  $\;  a\in\mathbb{R}\;  $  modélisent des situations très diverses, où la vitesse  de variation $\;   y^{\prime} $ d'une quantité est proportionnelle à cette quantité même $ y \; $. Par exemple l'accroissement d'une population, en un temps $ t $, proportionnel à  la taille de cette population.
\end{remark}
\begin{exercice}
Une substance chimique se dissout dans l'eau. On admet que la vitesse de dissolution est
proportionnelle à la quantité non encore dissoute. A l'instant $t=0$ ($t$ en minutes), on place 20 grammes
de cette substance dans une grande quantité d'eau.\\
Sachant que les dix premiers grammes se dissolvent en cinq minutes, déterminer une expression de la
quantité non dissoute $f(t)$, en grammes, en fonction de $t$.
\end{exercice}
\subsection*{Cas d'une équation différentielle avec second membre}
\begin{exercice}
  Soient les équations différentielles :\; (E$_{0})$ \;$  y'+y=0 $\; et \; (E) \;$  y'+y=\eexp{-x}\cos x $.\\
1)  Trouver les réels $ a$ et $b $ pour que $ h $  soit solution de(E), avec $ h(x)=\paren{a\cos x+b\sin x}\eexp{-x} $.\\
2) Démontrer  que   $ f $  est solution de (E)   si et seulement si  $ f-h $  est solution de  (E$_{0})$.\\
3) Résoudre (E$_{0})$.\\
4) Déduire des questions précédentes, la solution générale de (E).
\end{exercice}

\begin{proof}
  1)\; On a $ h^{\prime}(x)+h(x)=\paren{(b-a)\cos x+(-b-a)\sin x}\eexp{-x} =\eexp{-x}\cos x$ \\Donc $ b-a=1 $  et  $ -b-a=0 $ càd $a=\frac{1}{2} $ et $ b=-\frac{1}{2}$\\
  2)\; \underline{Supposons que $ f $  est solution de (E)}
  
  \medskip
  On a:\;  $  \begin{cases}  
f'+f=0 \\
h'+h=0
\end{cases} $ 
Donc par différence membre à membre \; $ f'-h'+f-h=0 $\; càd\\$ (f-h)'+(f-h)=0$\; d'où  la fonction $ f-h $  est solution de (E).

\medskip
\underline{Supposons que $ f-h $ est solution de  (E$_{0})$}

\medskip
On a \; $ (f-h)'+(f-h)=0   \Longleftrightarrow f'+f =h'+h=\eexp{-x}\cos x$\; d'où  $ f $  est solution de (E).

\medskip
3)\; Les solutions sur $ \Rr $ de l'équation différentielle\;$ y^{\prime}+y=0 $\; sont les fonctions \; $ x\mapsto k\eexp{-x}\; (k\in\Rr) $.\\
4)\; D'après  2)\;$ f $ est solution de (E) $ \Longleftrightarrow f-h $  est solution de (E$_{0})$\\
Donc $ f $ est solution de (E) $ \Longleftrightarrow f-h =\eexp{-x}\cos x$  \\
D'où $ f $ est solution de (E) $ \Longleftrightarrow f(x) =\eexp{-x}\cos x  +\paren{\frac{1}{2}\cos x-\frac{1}{2}\sin x}\eexp{-x} $  \\
Les solutions sur $ \Rr $ de l'équation différentielle\;$ y^{\prime}+y= \eexp{-x}\cos x$\; sont les fonctions \\ $ x\mapsto \eexp{-x}\cos x  +\paren{\frac{1}{2}\cos x-\frac{1}{2}\sin x}\eexp{-x} $.
\end{proof}

\subsection{Équations différentielles du second ordre}
\begin{definition}
Une équation différentielle linéaire homogène du 2\up{nd} ordre  à coefficients constants  est une équation du type  $y^{\prime \prime} +ay^{\prime}+by= 0$\; ou toute équation s'y ramenant, et $a $ et $ b$ sont des constantes réelles et $ y $ une fonction inconnue à déterminer, définie et dérivable sur un intervalle $ I$ de  $ \Rr.$
\end{definition}


 \begin{definition}{ Équation  caractéristique}
On appelle  \textbf{équation caractéristique} de l'équation différentielle: $\; y^{\prime \prime} +ay^{\prime}+by=0 \;\;\;(A\in\Rr,\; B\in\Rr)$, l'équation du second degré  d'inconnue $ r :\;\; $  $\; r^{2}+ar+b=0 $.
\end{definition}
\begin{example}
Les équations caractéristique de chacune des   équations différentielles suivantes \\$ y^{\prime \prime}-4y+3y=0 \;$ et \;$ y^{\prime \prime}-36y=0 $ sont respectivement  $\; r^{2} -4r+3=0\;$ et $\; r^{2}-36=0.$. 
\end{example}
\subsection*{Méthode de résolution}
Pour résoudre sur $ \Rr $  une équation différentielle du type $ y^{\prime \prime}+ ay^{\prime}+by=0\;\;(A\in\Rr,\; B\in\Rr)$, on peut  résoudre l'équation caractéristique \\ $\; r^{2}+ar+b=0 \;$  et utiliser le tableau suivant.




$$
\begin{array}{|c|c|c|}
\hline
\Delta = a^2 - 4b & \text{Racines} & \text{Solutions de } y'' + ay' + by = 0 \\
\hline
\Delta = 0 & r_0 \in \mathbb{R} & (Ax + B)e^{r_0 x} \\
\hline
\Delta > 0 & r_1,\ r_2 \in \mathbb{R} & Ae^{r_1 x} + Be^{r_2 x} \\
\hline
\Delta < 0 & \alpha \pm i\beta \in \mathbb{C} & e^{\alpha x}(A\cos\beta x + B\sin\beta x) \\
\hline
\end{array}
$$

 
\begin{exercice}
 Résoudre sur $ \Rr $  les  équations différentielles suivantes:
 \begin{enumerate}
  \item $ \; y^{\prime \prime}-6y^{\prime}+9y=0 $
 \item $ \; y^{\prime \prime}+ 5y^{\prime}+4y=0 $
 \item $  \; y^{\prime \prime}+2 y^{\prime}+2y=0 $
 \item $ \; 2 y^{\prime \prime}-18y=0 $
 \end{enumerate}
\end{exercice}
 
\begin{proof}
 \begin{enumerate}
  \item
 L'équation caractéristique est $ r^{2}-6r+4=0 $ donc $ r_{0}=3 $\\D'où  les solutions sur $ \Rr $ l'équation différentielle sont les fonctions\; $ x\mapsto \paren{Ax+B}\eexp{3x}\quad  \;\;(A\in\Rr,\; B\in\Rr) $\\
   \item L'équation caractéristique est $ r^{2}+5r+4=0 $ donc $ r_{1}=-4 $  et $ r_{2} =-1$\\D'où  les solutions sur $ \Rr $ l'équation différentielle sont les fonctions\; $ x\mapsto A\eexp{-3}x+B\eexp{-x} \;\;(A\in\Rr,\; B\in\Rr $\\
 
   \item L'équation caractéristique est $ r^{2}+2r+2=0 $ donc $ r_{1}=-1-\mathrm{i} $  et $ r_{2} =-1+ \mathrm{i} $\\D'où  les solutions sur $ \Rr $ l'équation différentielle sont les fonctions\; $ x\mapsto \eexp{- x}\paren{ A\cos x)+B\sin x} \;\;(A\in\Rr,\; B\in\Rr) $\\
  \item L'équation caractéristique est $ 2r^{2}-18=0 $ donc $ r_{1}=3 $  et $ r_{2} =-3$\\D'où  les solutions sur $ \Rr $ l'équation différentielle sont les fonctions\; $ x\mapsto A\eexp{-3x}+B\eexp{ 3x} \;\;(A\in\Rr,\; B\in\Rr $
 \end{enumerate}
 \end{proof}
 \subsection*{Solution vérifiant une condition initiale}
\begin{property}
Pour tous nombres réels $x_{0} $, $ y_{0} $  et $z_{0} $,\;l'équation différentielle $ y^{\prime \prime}+ ay^{\prime}+by=0\;\;(A\in\Rr,\; B\in\Rr)$ admet une unique solution  $ y $ sur $ \Rr $ telle que: $ y(x_{0})=y_{0} $ et  $ y^{\prime}(x_{0})=z_{0} $
\end{property}
\begin{example}
Déterminons la solution $ h $ sur $ \Rr $ l'équation différentielle\;  $ y^{\prime \prime}-y^{\prime} -2y=0 $\; sachant que  $ y(0)=1 $ et $ y^{\prime}(0)=0 $ . 

\end{example}
\begin{proof}
 Les solutions sur $ \Rr $ de l'équation différentielle  \;$ y^{\prime \prime}-y^{\prime}-2y=0 $  sont les fonctions
$$ x\mapsto A\eexp{2 x}+B\eexp{-x} \;\;(A\in\Rr,\; B\in\Rr))$$  
Pour  $ y(0)=1 $ et $ y^{\prime}(0)=0 $, on a $ A+B=1 $  et $ 2A-B=0 $ \; donc  $ A=\frac{1}{3}$ et $B=\frac{2}{3} $ . \\
D'où \; $ h: x\mapsto \frac{1}{3} \eexp{2 x}+\frac{2}{3}\eexp{-x} \;\;(A\in\Rr,\; B\in\Rr)$

\end{proof}

  %</content>
\end{document}