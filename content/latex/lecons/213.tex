\documentclass[12pt, a4paper]{report}

% LuaLaTeX :

\RequirePackage{iftex}
\RequireLuaTeX

% Packages :

\usepackage[french]{babel}
%\usepackage[utf8]{inputenc}
%\usepackage[T1]{fontenc}
\usepackage[pdfencoding=auto, pdfauthor={Hugo Delaunay}, pdfsubject={Mathématiques}, pdfcreator={agreg.skyost.eu}]{hyperref}
\usepackage{amsmath}
\usepackage{amsthm}
%\usepackage{amssymb}
\usepackage{stmaryrd}
\usepackage{tikz}
\usepackage{tkz-euclide}
\usepackage{fontspec}
\defaultfontfeatures[Erewhon]{FontFace = {bx}{n}{Erewhon-Bold.otf}}
\usepackage{fourier-otf}
\usepackage[nobottomtitles*]{titlesec}
\usepackage{fancyhdr}
\usepackage{listings}
\usepackage{catchfilebetweentags}
\usepackage[french, capitalise, noabbrev]{cleveref}
\usepackage[fit, breakall]{truncate}
\usepackage[top=2.5cm, right=2cm, bottom=2.5cm, left=2cm]{geometry}
\usepackage{enumitem}
\usepackage{tocloft}
\usepackage{microtype}
%\usepackage{mdframed}
%\usepackage{thmtools}
\usepackage{xcolor}
\usepackage{tabularx}
\usepackage{xltabular}
\usepackage{aligned-overset}
\usepackage[subpreambles=true]{standalone}
\usepackage{environ}
\usepackage[normalem]{ulem}
\usepackage{etoolbox}
\usepackage{setspace}
\usepackage[bibstyle=reading, citestyle=draft]{biblatex}
\usepackage{xpatch}
\usepackage[many, breakable]{tcolorbox}
\usepackage[backgroundcolor=white, bordercolor=white, textsize=scriptsize]{todonotes}
\usepackage{luacode}
\usepackage{float}
\usepackage{needspace}
\everymath{\displaystyle}

% Police :

\setmathfont{Erewhon Math}

% Tikz :

\usetikzlibrary{calc}
\usetikzlibrary{3d}

% Longueurs :

\setlength{\parindent}{0pt}
\setlength{\headheight}{15pt}
\setlength{\fboxsep}{0pt}
\titlespacing*{\chapter}{0pt}{-20pt}{10pt}
\setlength{\marginparwidth}{1.5cm}
\setstretch{1.1}

% Métadonnées :

\author{agreg.skyost.eu}
\date{\today}

% Titres :

\setcounter{secnumdepth}{3}

\renewcommand{\thechapter}{\Roman{chapter}}
\renewcommand{\thesubsection}{\Roman{subsection}}
\renewcommand{\thesubsubsection}{\arabic{subsubsection}}
\renewcommand{\theparagraph}{\alph{paragraph}}

\titleformat{\chapter}{\huge\bfseries}{\thechapter}{20pt}{\huge\bfseries}
\titleformat*{\section}{\LARGE\bfseries}
\titleformat{\subsection}{\Large\bfseries}{\thesubsection \, - \,}{0pt}{\Large\bfseries}
\titleformat{\subsubsection}{\large\bfseries}{\thesubsubsection. \,}{0pt}{\large\bfseries}
\titleformat{\paragraph}{\bfseries}{\theparagraph. \,}{0pt}{\bfseries}

\setcounter{secnumdepth}{4}

% Table des matières :

\renewcommand{\cftsecleader}{\cftdotfill{\cftdotsep}}
\addtolength{\cftsecnumwidth}{10pt}

% Redéfinition des commandes :

\renewcommand*\thesection{\arabic{section}}
\renewcommand{\ker}{\mathrm{Ker}}

% Nouvelles commandes :

\newcommand{\website}{https://github.com/imbodj/SenCoursDeMaths}

\newcommand{\tr}[1]{\mathstrut ^t #1}
\newcommand{\im}{\mathrm{Im}}
\newcommand{\rang}{\operatorname{rang}}
\newcommand{\trace}{\operatorname{trace}}
\newcommand{\id}{\operatorname{id}}
\newcommand{\stab}{\operatorname{Stab}}
\newcommand{\paren}[1]{\left(#1\right)}
\newcommand{\croch}[1]{\left[ #1 \right]}
\newcommand{\Grdcroch}[1]{\Bigl[ #1 \Bigr]}
\newcommand{\grdcroch}[1]{\bigl[ #1 \bigr]}
\newcommand{\abs}[1]{\left\lvert #1 \right\rvert}
\newcommand{\limi}[3]{\lim_{#1\to #2}#3}
\newcommand{\pinf}{+\infty}
\newcommand{\minf}{-\infty}
%%%%%%%%%%%%%% ENSEMBLES %%%%%%%%%%%%%%%%%
\newcommand{\ensemblenombre}[1]{\mathbb{#1}}
\newcommand{\Nn}{\ensemblenombre{N}}
\newcommand{\Zz}{\ensemblenombre{Z}}
\newcommand{\Qq}{\ensemblenombre{Q}}
\newcommand{\Qqp}{\Qq^+}
\newcommand{\Rr}{\ensemblenombre{R}}
\newcommand{\Cc}{\ensemblenombre{C}}
\newcommand{\Nne}{\Nn^*}
\newcommand{\Zze}{\Zz^*}
\newcommand{\Zzn}{\Zz^-}
\newcommand{\Qqe}{\Qq^*}
\newcommand{\Rre}{\Rr^*}
\newcommand{\Rrp}{\Rr_+}
\newcommand{\Rrm}{\Rr_-}
\newcommand{\Rrep}{\Rr_+^*}
\newcommand{\Rrem}{\Rr_-^*}
\newcommand{\Cce}{\Cc^*}
%%%%%%%%%%%%%%  INTERVALLES %%%%%%%%%%%%%%%%%
\newcommand{\intff}[2]{\left[#1\;,\; #2\right]  }
\newcommand{\intof}[2]{\left]#1 \;, \;#2\right]  }
\newcommand{\intfo}[2]{\left[#1 \;,\; #2\right[  }
\newcommand{\intoo}[2]{\left]#1 \;,\; #2\right[  }

\providecommand{\newpar}{\\[\medskipamount]}

\newcommand{\annexessection}{%
  \newpage%
  \subsection*{Annexes}%
}

\providecommand{\lesson}[3]{%
  \title{#3}%
  \hypersetup{pdftitle={#2 : #3}}%
  \setcounter{section}{\numexpr #2 - 1}%
  \section{#3}%
  \fancyhead[R]{\truncate{0.73\textwidth}{#2 : #3}}%
}

\providecommand{\development}[3]{%
  \title{#3}%
  \hypersetup{pdftitle={#3}}%
  \section*{#3}%
  \fancyhead[R]{\truncate{0.73\textwidth}{#3}}%
}

\providecommand{\sheet}[3]{\development{#1}{#2}{#3}}

\providecommand{\ranking}[1]{%
  \title{Terminale #1}%
  \hypersetup{pdftitle={Terminale #1}}%
  \section*{Terminale #1}%
  \fancyhead[R]{\truncate{0.73\textwidth}{Terminale #1}}%
}

\providecommand{\summary}[1]{%
  \textit{#1}%
  \par%
  \medskip%
}

\tikzset{notestyleraw/.append style={inner sep=0pt, rounded corners=0pt, align=center}}

%\newcommand{\booklink}[1]{\website/bibliographie\##1}
\newcounter{reference}
\newcommand{\previousreference}{}
\providecommand{\reference}[2][]{%
  \needspace{20pt}%
  \notblank{#1}{
    \needspace{20pt}%
    \renewcommand{\previousreference}{#1}%
    \stepcounter{reference}%
    \label{reference-\previousreference-\thereference}%
  }{}%
  \todo[noline]{%
    \protect\vspace{20pt}%
    \protect\par%
    \protect\notblank{#1}{\cite{[\previousreference]}\\}{}%
    \protect\hyperref[reference-\previousreference-\thereference]{p. #2}%
  }%
}

\definecolor{devcolor}{HTML}{00695c}
\providecommand{\dev}[1]{%
  \reversemarginpar%
  \todo[noline]{
    \protect\vspace{20pt}%
    \protect\par%
    \bfseries\color{devcolor}\href{\website/developpements/#1}{[DEV]}
  }%
  \normalmarginpar%
}

% En-têtes :

\pagestyle{fancy}
\fancyhead[L]{\truncate{0.23\textwidth}{\thepage}}
\fancyfoot[C]{\scriptsize \href{\website}{\texttt{https://github.com/imbodj/SenCoursDeMaths}}}

% Couleurs :

\definecolor{property}{HTML}{ffeb3b}
\definecolor{proposition}{HTML}{ffc107}
\definecolor{lemma}{HTML}{ff9800}
\definecolor{theorem}{HTML}{f44336}
\definecolor{corollary}{HTML}{e91e63}
\definecolor{definition}{HTML}{673ab7}
\definecolor{notation}{HTML}{9c27b0}
\definecolor{example}{HTML}{00bcd4}
\definecolor{cexample}{HTML}{795548}
\definecolor{application}{HTML}{009688}
\definecolor{remark}{HTML}{3f51b5}
\definecolor{algorithm}{HTML}{607d8b}
%\definecolor{proof}{HTML}{e1f5fe}
\definecolor{exercice}{HTML}{e1f5fe}

% Théorèmes :

\theoremstyle{definition}
\newtheorem{theorem}{Théorème}

\newtheorem{property}[theorem]{Propriété}
\newtheorem{proposition}[theorem]{Proposition}
\newtheorem{lemma}[theorem]{Activité d'introduction}
\newtheorem{corollary}[theorem]{Conséquence}

\newtheorem{definition}[theorem]{Définition}
\newtheorem{notation}[theorem]{Notation}

\newtheorem{example}[theorem]{Exemple}
\newtheorem{cexample}[theorem]{Contre-exemple}
\newtheorem{application}[theorem]{Application}

\newtheorem{algorithm}[theorem]{Algorithme}
\newtheorem{exercice}[theorem]{Exercice}

\theoremstyle{remark}
\newtheorem{remark}[theorem]{Remarque}

\counterwithin*{theorem}{section}

\newcommand{\applystyletotheorem}[1]{
  \tcolorboxenvironment{#1}{
    enhanced,
    breakable,
    colback=#1!8!white,
    %right=0pt,
    %top=8pt,
    %bottom=8pt,
    boxrule=0pt,
    frame hidden,
    sharp corners,
    enhanced,borderline west={4pt}{0pt}{#1},
    %interior hidden,
    sharp corners,
    after=\par,
  }
}

\applystyletotheorem{property}
\applystyletotheorem{proposition}
\applystyletotheorem{lemma}
\applystyletotheorem{theorem}
\applystyletotheorem{corollary}
\applystyletotheorem{definition}
\applystyletotheorem{notation}
\applystyletotheorem{example}
\applystyletotheorem{cexample}
\applystyletotheorem{application}
\applystyletotheorem{remark}
%\applystyletotheorem{proof}
\applystyletotheorem{algorithm}
\applystyletotheorem{exercice}

% Environnements :

\NewEnviron{whitetabularx}[1]{%
  \renewcommand{\arraystretch}{2.5}
  \colorbox{white}{%
    \begin{tabularx}{\textwidth}{#1}%
      \BODY%
    \end{tabularx}%
  }%
}

% Maths :

\DeclareFontEncoding{FMS}{}{}
\DeclareFontSubstitution{FMS}{futm}{m}{n}
\DeclareFontEncoding{FMX}{}{}
\DeclareFontSubstitution{FMX}{futm}{m}{n}
\DeclareSymbolFont{fouriersymbols}{FMS}{futm}{m}{n}
\DeclareSymbolFont{fourierlargesymbols}{FMX}{futm}{m}{n}
\DeclareMathDelimiter{\VERT}{\mathord}{fouriersymbols}{152}{fourierlargesymbols}{147}

% Code :

\definecolor{greencode}{rgb}{0,0.6,0}
\definecolor{graycode}{rgb}{0.5,0.5,0.5}
\definecolor{mauvecode}{rgb}{0.58,0,0.82}
\definecolor{bluecode}{HTML}{1976d2}
\lstset{
  basicstyle=\footnotesize\ttfamily,
  breakatwhitespace=false,
  breaklines=true,
  %captionpos=b,
  commentstyle=\color{greencode},
  deletekeywords={...},
  escapeinside={\%*}{*)},
  extendedchars=true,
  frame=none,
  keepspaces=true,
  keywordstyle=\color{bluecode},
  language=Python,
  otherkeywords={*,...},
  numbers=left,
  numbersep=5pt,
  numberstyle=\tiny\color{graycode},
  rulecolor=\color{black},
  showspaces=false,
  showstringspaces=false,
  showtabs=false,
  stepnumber=2,
  stringstyle=\color{mauvecode},
  tabsize=2,
  %texcl=true,
  xleftmargin=10pt,
  %title=\lstname
}

\newcommand{\codedirectory}{}
\newcommand{\inputalgorithm}[1]{%
  \begin{algorithm}%
    \strut%
    \lstinputlisting{\codedirectory#1}%
  \end{algorithm}%
}




\begin{document}
  %<*content>
  \lesson{analysis}{213}{Espaces de Hilbert. Exemples d'applications.}

  \subsection{Généralités}

  \subsubsection{Espaces préhilbertiens}

  \reference[LI]{27}

  \begin{definition}
    Soit $H$ un espace vectoriel réel (resp. complexe). On appelle \textbf{produit scalaire} sur $H$ une forme bilinéaire $\langle . , . \rangle$ telle que :
    \begin{enumerate}[label=(\roman*)]
      \item $\forall y \in H, \, x \mapsto \langle x, y \rangle$ est une forme linéaire.
      \item $\forall x \in H, \, \langle x, x \rangle \geq 0$ avec égalité si et seulement si $x = 0$.
      \item $\forall x, y \in H, \, \langle x, y \rangle = \langle y, x \rangle$ (resp. $\langle x, y \rangle = \overline{\langle y, x \rangle}$).
    \end{enumerate}
  \end{definition}

  \begin{remark}
    Dans le cas complexe, on a donc
    \[ \forall x, y \in H, \forall \lambda \in \mathbb{C}, \, \langle x, \lambda y \rangle = \overline{\lambda} \langle x, y \rangle \]
  \end{remark}

  \begin{definition}
    En reprenant les notations de la définition, si $H$ est muni d'un produit scalaire, on dit que $H$ est un espace \textbf{préhilbertien}.
  \end{definition}

  \begin{example}
    \begin{itemize}
      \item $\mathbb{C}^n$ muni de
      \[ \langle ., . \rangle : ((x_i)_{i \in \llbracket 1, n \rrbracket},(y_i)_{i \in \llbracket 1, n \rrbracket}) \mapsto \sum_{i=1}^{n} x_i \overline{y_i} \]
      est un espace préhilbertien.
      \item Plus généralement, on peut définit d'autres produits scalaires sur $\mathbb{R}^n$ ou $\mathbb{C}^n$ en se donnant un poids $\omega = (\omega_1, \dots, \omega_n)$ où $\forall i \in \llbracket 1, n \rrbracket, \, \omega > 0$. Il suffit de munir l'espace produit du produit scalaire suivant :
      \[ \langle ., . \rangle_\omega : ((x_i)_{i \in \llbracket 1, n \rrbracket},(y_i)_{i \in \llbracket 1, n \rrbracket}) \mapsto \sum_{i=1}^{n} \omega_i x_i \overline{y_i} \]
    \end{itemize}
  \end{example}

  Dans toute la suite, on considérera un espace préhilbertien $(H, \langle ., . \rangle)$ sur le corps $\mathbb{K} = \mathbb{R} \text{ ou } \mathbb{C}$.

  \begin{notation}
    Puisque $\langle ., . \rangle \geq 0$, on peut poser
    \[ \Vert . \Vert = \sqrt{\langle ., . \rangle} \]
  \end{notation}

  \begin{proposition}[Identités de polarisation]
    \label{213-1}
    Soient $x, y \in H$.
    \begin{enumerate}[label=(\roman*)]
      \item $\Vert x + y \Vert^2 = \Vert x \Vert^2 + 2 \langle x, y \rangle + \Vert y \Vert^2$ (si $\mathbb{K} = \mathbb{R}$).
      \item $\Vert x + y \Vert^2 = \Vert x \Vert^2 + 2 \operatorname{Re}(\langle x, y \rangle) + \Vert y \Vert^2$ (si $\mathbb{K} = \mathbb{C}$).
    \end{enumerate}
  \end{proposition}

  \begin{theorem}[Inégalité de Cauchy-Schwarz]
    \[ \forall x, y \in H, \, \vert \langle x, y \rangle \vert \leq \Vert x \Vert \Vert y \Vert \]
    avec égalité si et seulement si $x$ et $y$ sont colinéaires.
  \end{theorem}

  \begin{corollary}
    $\Vert . \Vert$ définit une norme sur $H$, ce qui fait de $(H, \Vert . \Vert)$ un espace vectoriel normé.
  \end{corollary}

  \reference{62}

  \begin{proposition}[Identité du parallélogramme]
    \[ \forall x, y \in H, \, \Vert x + y \Vert^2 + \Vert x - y \Vert^2 = 2(\Vert x \Vert^2 \Vert y \Vert^2) \]
    et cette identité caractérise les normes issues d'un produit scalaire.
  \end{proposition}

  \subsubsection{Orthogonalité}

  \reference{31}

  \begin{definition}
    On dit que deux vecteurs $x$ et $y$ de $H$ sont orthogonaux si
    \[ \langle x, y \rangle = 0 \]
    et on le note $x \perp y$.
  \end{definition}

  \begin{example}
    Dans $\mathbb{R}^2$ muni de son produit scalaire usuel, on a $(-1,1) \perp (1,1)$.
  \end{example}

  \begin{remark}[Théorème de Pythagore]
    Si $\mathbb{K} = \mathbb{R}$, par la \cref{213-1},
    \[ \forall x, y \in H, \, x \perp y \iff \Vert x + y \Vert^2 = \Vert x \Vert^2 + \Vert y \Vert^2 \]
  \end{remark}

  \begin{definition}
    \textbf{L'orthogonal} d'une partie $A \subseteq H$ est l'ensemble
    \[ A^\perp = \{ y \in H \mid \forall x \in A, \, x \perp y \} \]
  \end{definition}

  \reference[BMP]{99}

  \begin{proposition}
    Soit $A \subseteq H$.
    \begin{enumerate}[label=(\roman*)]
      \item $A^\perp$ est un sous-espace vectoriel fermé de $H$.
      \item $A^\perp = (\operatorname{Vect}(A))^\perp$.
      \item $A^\perp = (\overline{A})^\perp$.
    \end{enumerate}
  \end{proposition}

  \subsubsection{Espaces de Hilbert}

  \reference{91}

  \begin{definition}
    Si $(H, \Vert . \Vert)$ est complet, on dit que $H$ est un \textbf{espace de Hilbert}.
  \end{definition}

  On suppose dans la suite que $(H, \Vert . \Vert)$ est un espace de Hilbert.

  \begin{example}
    \begin{itemize}
      \item Tout espace euclidien ou hermitien est un espace de Hilbert.
      \item L'ensemble des suites de nombres complexes de carré sommables
      \[ \ell_2(\mathbb{N}) = \{ (x_n) \in \mathbb{C}^2 \mid \sum_{n=0}^{+\infty} \vert x_n \vert^2 < +\infty \} \]
      muni du produit scalaire hermitien
      \[ \langle ., . \rangle : ((x_n)_{n \in \mathbb{N}},(y_n)_{n \in \mathbb{N}}) \mapsto \sum_{n=0}^{+\infty} x_n \overline{y_n} \]
      est un espace de Hilbert.
    \end{itemize}
  \end{example}

  \subsection{Le théorème de projection sur un convexe fermé et ses conséquences}

  \subsubsection{Théorème de projection}

  \reference[LI]{32}
  \dev{projection-sur-un-convexe-ferme}

  \begin{theorem}[Projection sur un convexe fermé]
    Soit $C \subseteq H$ un convexe fermé non-vide. Alors :
    \[ \forall x \in H, \exists! y \in C \text{ tel que } d(x, C) = \inf_{z \in C} \Vert x - z \Vert = d(x, y) \]
    On peut donc noter $y = P_C(x)$, le \textbf{projeté orthogonal de $x$ sur $C$}. Il s'agit de l'unique point de $C$ vérifiant
    \[ \forall z \in C, \, \langle x - P_C(x), z - P_C(x) \rangle \leq 0 \tag{$*$} \]
  \end{theorem}

  \reference[BMP]{96}

  \begin{remark}
    En dimension finie, dans un espace euclidien ou hermitien, on peut projeter sur tous les fermés. On perd cependant l'unicité et la caractérisation angulaire.
  \end{remark}

  \begin{proposition}
    Soit $C \subseteq H$ un convexe fermé non-vide. L'application $P_C$ est lipschitzienne de rapport $1$ et est, en particulier, continue.
  \end{proposition}

  \subsubsection{Décomposition en somme directe orthogonale}

  \begin{theorem}[Projection sur un sous-espace fermé]
    Soit $F$ un sous-espace vectoriel fermé de $H$.
    \begin{enumerate}[label=(\roman*)]
      \item Si $x \in H$, le projeté $P_F(x)$ de $x$ sur $F$ est l'unique élément $p \in H$ qui vérifie
      \[ p \in F \text{ et } x - p \in F^\perp \]
      \item $P_F : H \rightarrow F$ est linéaire, continue, surjective.
      \item $H = F \oplus F^\perp$ et $P_F$ est le projecteur sur $F$ associé à cette décomposition.
      \item Soient $x, x_1, x_2 \in H$. On a :
      \[ x = x_1 + x_2 \text{ avec } x_1 \in F, x_2 \in F^\perp \iff x_1 = P_F(x) \text{ et } x_2 = P_{F^\perp}(x) \]
    \end{enumerate}
  \end{theorem}

  \begin{cexample}
    On considère le sous-espace vectoriel de $\ell_2(\mathbb{N})$ constitué des suites nulles à partir d'un certain rang. Alors $F^\perp = \{ 0 \}$, et ainsi $H \neq F \oplus F^\perp$.
  \end{cexample}

  \begin{corollary}
    Soit $F$ un sous-espace vectoriel de $H$. Alors,
    \[ F^{\perp \perp} = \overline{F} \]
  \end{corollary}

  \begin{corollary}
    Soit $F$ un sous-espace vectoriel de $H$. Alors,
    \[ \overline{F} = H \iff F^\perp = 0 \]
  \end{corollary}

  \subsubsection{Dualité dans un espace de Hilbert}

  \begin{theorem}[de représentation de Riesz]
    L'application
    \[
    \Phi :
    \begin{array}{ccc}
      H &\rightarrow& H' \\
      y &\mapsto& (x \mapsto \langle x, y \rangle)
    \end{array}
    \]
    est une isométrie linéaire bijective de $H$ sur son dual topologique $H'$.
  \end{theorem}

  \begin{remark}
    Cela signifie que :
    \[ \forall \varphi \in H', \, \exists! y \in H, \text{ tel que } \forall x \in H, \, \varphi(x) = \langle x, y \rangle \]
    et de plus, $\VERT \varphi \VERT = \Vert y \Vert$.
  \end{remark}

  \begin{application}[Existence de l'adjoint]
    Soit $u \in \mathcal{L}(H)$. Il existe un unique $v \in \mathcal{L}(H)$ tel que :
    \[ \forall x, y \in H, \, \langle u(x), y \rangle = \langle x, v(y) \rangle \]
    On dit que $v$ est \textbf{l'adjoint} de $u$ et on note généralement $v = u^*$.
  \end{application}

  \reference[Z-Q]{222}
  \dev{dual-de-lp}

  \begin{application}[Dual de $L_p$]
    Soit $(X, \mathcal{A}, \mu)$ un espace mesuré de mesure finie. On note $\forall p \in ]1,2[$, $L_p = L_p(X, \mathcal{A}, \mu)$. L'application
    \[
    \varphi :
    \begin{array}{ll}
      L_q &\rightarrow (L_p)' \\
      g &\mapsto \left( \varphi_g : f \mapsto \int_X f g \, \mathrm{d}\mu \right)
    \end{array}
    \qquad \text{ où } \frac{1}{p} + \frac{1}{q} = 1
    \]
    est une isométrie linéaire surjective. C'est donc un isomorphisme isométrique.
  \end{application}

  \subsection{Bases hilbertiennes}

  \reference[LI]{41}

  \begin{definition}
    On dit qu'une famille $(e_i)_{i \in I}$ d'éléments de $H$ est \textbf{orthonormée} de $H$ si :
    \[ \forall i, j \in I, \, \langle e_i, e_j \rangle = \delta_{i,j} \]
  \end{definition}

  \begin{example}
    Dans $\ell_2(\mathbb{N})$, la famille $(u_n)_{n \in \mathbb{N}}$ définie par
    \[ \forall n \in \mathbb{N}, \, u_n = (0, \dots, 0, \underbrace{1}_{n \text{-ième position}}, 0 \dots) \]
    est orthonormée.
  \end{example}

  \begin{proposition}
    Toute famille orthonormée est libre.
  \end{proposition}

  \begin{proposition}[Inégalité de Bessel]
    Soient $x \in H$ et $(e_i)_{i \in I}$ une famille orthonormée de $H$. Alors,
    \[ \sum_{i \in I} \vert \langle x, e_i \rangle \vert \leq \Vert x \Vert \]
  \end{proposition}

  \begin{definition}
    On dit qu'une famille $(e_i)_{i \in I}$ d'éléments de $H$ est une \textbf{base} de $H$ si elle est orthonormée et totale (ie. $\operatorname{Vect}(e_i)_{i \in I}$ est dense dans $H$).
  \end{definition}

  \reference[BMP]{108}

  \begin{theorem}
    \begin{enumerate}[label=(\roman*)]
      \item Tout espace de Hilbert admet une base hilbertienne.
      \item Tout espace de Hilbert séparable (ie. admettant une partie dénombrable dense) admet une base hilbertienne dénombrable.
    \end{enumerate}
  \end{theorem}

  \begin{example}
    $\mathbb{K}^n$ est séparable pour tout entier $n$ et $L_p$ aussi pour tout $p \in [1,+\infty[$. On a donc existence d'une base hilbertienne dénombrable pour ces espaces.
  \end{example}

  \begin{theorem}
    \label{213-2}
    Soit $H$ un espace de Hilbert séparable et $(e_n)_{n \in I}$ une famille orthonormée dénombrable de $H$. Les propriétés suivantes sont équivalentes :
    \begin{enumerate}[label=(\roman*)]
      \item La famille orthonormée $(e_n)_{n \in I}$ est une base hilbertienne de $H$.
      \item $\forall x \in H, \, x = \sum_{n=0}^{+\infty} \langle x, e_n \rangle e_n$.
      \item \label{213-3} $\forall x \in H, \, \Vert x \Vert_2 = \sum_{n=0}^{+\infty} \vert \langle x, e_n \rangle \vert^2$.
      \item L'application
      \[
      \Delta :
      \begin{array}{ccc}
        H &\rightarrow& \ell_2(\mathbb{N}) \\
        x &\mapsto& (\langle x, e_n)_{n \in \mathbb{N}}
      \end{array}
      \]
      est une isométrie linéaire bijective.
    \end{enumerate}
  \end{theorem}

  \begin{remark}
    L'égalité du \cref{213-2} \cref{213-3} est appelée \textbf{égalité de Parseval}.
  \end{remark}

  \reference[LI]{45}

  \begin{corollary}
    Tous les espaces de Hilbert séparables sont isométriquement isomorphes à $\ell_2(\mathbb{N})$.
  \end{corollary}

  \subsection{L'espace \texorpdfstring{$L_2$}{L₂}}

  \subsubsection{Aspect hilbertien}

  \reference[BMP]{92}

  Soit $(X, \mathcal{A}, \mu)$ un espace mesuré.

  \begin{notation}
    On note $L_p = L_p(X, \mathcal{A}, \mu)$ pour tout $p \in [1, +\infty]$.
  \end{notation}

  \begin{definition}
    On considère la forme bilinéaire suivante sur $L_2$ :
    \[ \langle ., . \rangle : (f, g) \mapsto \int_X f \overline{g} \, \mathrm{d}\mu \]
    C'est un produit scalaire hermitien, ce qui confère à $(L_2, \langle ., . \rangle)$ une structure d'espace préhilbertien.
  \end{definition}

  \begin{remark}
    La norme associée au produit scalaire précédent est la norme $\Vert . \Vert_2$ de $L_2$.
  \end{remark}

  \reference[LI]{10}

  \begin{theorem}[Riesz-Fischer]
    Pour tout $p \in [1, +\infty]$, $L_p$ est complet pour la norme $\Vert . \Vert_p$.
  \end{theorem}

  \reference{31}

  \begin{corollary}
    $L_2$ est un espace de Hilbert.
  \end{corollary}

  \subsubsection{Polynômes orthogonaux}

  \reference[BMP]{110}

  Soit $I$ un intervalle de $\mathbb{R}$. On pose $\forall n \in \mathbb{N}$, $g_n : x \mapsto x^n$.

  \begin{definition}
    On appelle \textbf{fonction poids} une fonction $\rho : I \rightarrow \mathbb{R}$ mesurable, positive et telle que $\forall n \in \mathbb{N}, \rho g_n \in L_1(I)$.
  \end{definition}

  Soit $\rho : I \rightarrow \mathbb{R}$ une fonction poids.

  \begin{notation}
    On note $L_2(I, \rho)$ l'espace des fonctions de carré intégrable pour la mesure de densité $\rho$ par rapport à la mesure de Lebesgue.
  \end{notation}

  \begin{proposition}
    Muni de
    \[ \langle ., . \rangle : (f,g) \mapsto \int_I f(x) \overline{g(x)} \rho(x) \, \mathrm{d}x \]
    $L_2(I, \rho)$ est un espace de Hilbert.
  \end{proposition}

  \begin{theorem}
    Il existe une unique famille $(P_n)$ de polynômes unitaires orthogonaux deux-à-deux telle que $\deg(P_n) = n$ pour tout entier $n$. C'est la famille de \textbf{polynômes orthogonaux} associée à $\rho$ sur $I$.
  \end{theorem}

  \begin{example}[Polynômes de Hermite]
    Si $\forall x \in I, \, \rho(x) = e^{-x^2}$, alors
    \[ \forall n \in \mathbb{N}, \, \forall x \in I, \, P_n(x) = \frac{(-1)^n}{2^n} e^{x^2} \frac{\partial}{\partial x^n} \left( e^{-x^2} \right) \]
  \end{example}

  \reference{140}

  \begin{lemma}
    On suppose que $\forall n \in \mathbb{N}$, $g_n \in L_1(I, \rho)$ et on considère $(P_n)$ la famille des polynômes orthogonaux associée à $\rho$ sur $I$. Alors $\forall n \in \mathbb{N}$, $g_n \in L_2(I, \rho)$. En particulier, l'algorithme de Gram-Schmidt a bien du sens et $(P_n)$ est bien définie.
  \end{lemma}

  \begin{application}
    On considère $(P_n)$ la famille des polynômes orthogonaux associée à $\rho$ sur $I$ et on suppose qu'il existe $a > 0$ tel que
    \[ \int_I e^{a \vert x \vert} \rho(x) \, \mathrm{d}x < +\infty \]
    alors $(P_n)$ est une base hilbertienne de $L_2(I, \rho)$ pour la norme $\Vert . \Vert_2$.
  \end{application}

  \begin{cexample}
    On considère, sur $I = \mathbb{R}^+_*$, la fonction poids $\rho : x \mapsto x^{-\ln(x)}$. Alors, la famille des $g_n$ n'est pas totale. La famille des polynômes orthogonaux associée à ce poids particulier n'est donc pas totale non plus : ce n'est pas une base hilbertienne.
  \end{cexample}

  \subsubsection{Séries de Fourier}

  \reference[Z-Q]{73}

  \begin{notation}
    \begin{itemize}
      \item Pour tout $p \in [1, +\infty]$, on note $L_p^{2\pi}$ l'espace des fonctions $f : \mathbb{R} \rightarrow \mathbb{C}$, $2\pi$-périodiques et mesurables, telles que $\Vert f \Vert_p < +\infty$.
      \item Pour tout $n \in \mathbb{Z}$, on note $e_n$ la fonction $2\pi$-périodique définie pour tout $t \in \mathbb{R}$ par $e_n(t) = e^{int}$.
    \end{itemize}
  \end{notation}

  \begin{proposition}
    $L_2^{2\pi}$ est un espace de Hilbert pour le produit scalaire
    \[ \langle ., . \rangle : (f, g) \mapsto \frac{1}{2 \pi} \int_0^{2\pi} f(t) \overline{g(t)} \, \mathrm{d}t \]
  \end{proposition}

  \reference[BMP]{123}

  \begin{theorem}
    La famille $(e_n)_{n \in \mathbb{Z}}$ est une base hilbertienne de $L_2^{2 \pi}$.
  \end{theorem}

  \begin{corollary}
    \label{213-4}
    \[ \forall f \in L_2^{2 \pi}, \, f = \sum_{n = -\infty}^{+\infty} \langle f, e_n \rangle e_n \]
  \end{corollary}

  \reference[GOU20]{272}

  \begin{example}
    On considère $f : x \mapsto 1 - \frac{x^2}{\pi^2}$ sur $[-\pi, \pi]$. Alors,
    \[ \frac{\pi^4}{90} = \Vert f \Vert_2 = \sum_{n=0}^{+\infty} \frac{1}{n^4} \]
  \end{example}

  \reference[BMP]{124}

  \begin{remark}
    L'égalité du \cref{213-4} est valable dans $L_2^{2\pi}$, elle signifie donc que
    \[ \left\Vert \sum_{n = -N}^{N} \langle f, e_n \rangle e_n - f \right\Vert_2 \longrightarrow_{N \rightarrow +\infty} 0 \]
  \end{remark}

  \newpage
  \section*{Annexes}

  \reference[LI]{32}

  \begin{figure}[H]
    \begin{center}
      \begin{tikzpicture}
        \def\a{3}
        \def\b{1.5}
        \def\angle{20}

        \coordinate (O) at (0, 0);
        \coordinate (X) at (5.6, 2.54);
        \coordinate (Z) at (1.8, -0.8);
        \coordinate (P) at (\angle:{\a} and {\b});
        \coordinate (F1) at ({-sqrt(\a*\a-\b*\b)}, 0);
        \coordinate (F2) at ({+sqrt(\a*\a-\b*\b)}, 0);
        \tkzDefLine[bisector out](F1,P,F2) \tkzGetPoint{K}
        \coordinate (A) at ($(K)!+0.5!(P)$);
        \coordinate (B) at ($(P)!-0.5!(K)$);

        \draw[fill=blue!30, fill opacity=0.3] (O) ellipse ({\a} and {\b});

        \tkzMarkRightAngle[color=red](X,P,K);
        \tkzMarkAngle[->,size=0.8,color=cyan,mark=none](Z,P,X);
        \tkzLabelAngle[pos=0,shift={(2,0)}](Z,P,X){\color{cyan} Angle obtus};

        \draw(O) node {$C$};
        \draw(P) node{$\bullet$} node[left]{$P_C(x)$};
        \draw(X) node{$\bullet$} node[above right]{$x$};
        \draw(Z) node{$\bullet$} node[above left]{$z$};

        \draw[dashed] (A) -- (B);
        \draw[dashed] ($(A)!(X)!(B)$) -- (X);
        \draw[dashed] (Z) -- (P);
      \end{tikzpicture}
    \end{center}
    \caption{Illustration du théorème de projection sur un convexe fermé.}
  \end{figure}
  %</content>
\end{document}
