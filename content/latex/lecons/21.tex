\documentclass[12pt, a4paper]{report}

% LuaLaTeX :

\RequirePackage{iftex}
\RequireLuaTeX

% Packages :

\usepackage[french]{babel}
%\usepackage[utf8]{inputenc}
%\usepackage[T1]{fontenc}
\usepackage[pdfencoding=auto, pdfauthor={Hugo Delaunay}, pdfsubject={Mathématiques}, pdfcreator={agreg.skyost.eu}]{hyperref}
\usepackage{amsmath}
\usepackage{amsthm}
%\usepackage{amssymb}
\usepackage{stmaryrd}
\usepackage{tikz}
\usepackage{tkz-euclide}
\usepackage{fontspec}
\defaultfontfeatures[Erewhon]{FontFace = {bx}{n}{Erewhon-Bold.otf}}
\usepackage{fourier-otf}
\usepackage[nobottomtitles*]{titlesec}
\usepackage{fancyhdr}
\usepackage{listings}
\usepackage{catchfilebetweentags}
\usepackage[french, capitalise, noabbrev]{cleveref}
\usepackage[fit, breakall]{truncate}
\usepackage[top=2.5cm, right=2cm, bottom=2.5cm, left=2cm]{geometry}
\usepackage{enumitem}
\usepackage{tocloft}
\usepackage{microtype}
%\usepackage{mdframed}
%\usepackage{thmtools}
\usepackage{xcolor}
\usepackage{tabularx}
\usepackage{xltabular}
\usepackage{aligned-overset}
\usepackage[subpreambles=true]{standalone}
\usepackage{environ}
\usepackage[normalem]{ulem}
\usepackage{etoolbox}
\usepackage{setspace}
\usepackage[bibstyle=reading, citestyle=draft]{biblatex}
\usepackage{xpatch}
\usepackage[many, breakable]{tcolorbox}
\usepackage[backgroundcolor=white, bordercolor=white, textsize=scriptsize]{todonotes}
\usepackage{luacode}
\usepackage{float}
\usepackage{needspace}
\everymath{\displaystyle}

% Police :

\setmathfont{Erewhon Math}

% Tikz :

\usetikzlibrary{calc}
\usetikzlibrary{3d}

% Longueurs :

\setlength{\parindent}{0pt}
\setlength{\headheight}{15pt}
\setlength{\fboxsep}{0pt}
\titlespacing*{\chapter}{0pt}{-20pt}{10pt}
\setlength{\marginparwidth}{1.5cm}
\setstretch{1.1}

% Métadonnées :

\author{agreg.skyost.eu}
\date{\today}

% Titres :

\setcounter{secnumdepth}{3}

\renewcommand{\thechapter}{\Roman{chapter}}
\renewcommand{\thesubsection}{\Roman{subsection}}
\renewcommand{\thesubsubsection}{\arabic{subsubsection}}
\renewcommand{\theparagraph}{\alph{paragraph}}

\titleformat{\chapter}{\huge\bfseries}{\thechapter}{20pt}{\huge\bfseries}
\titleformat*{\section}{\LARGE\bfseries}
\titleformat{\subsection}{\Large\bfseries}{\thesubsection \, - \,}{0pt}{\Large\bfseries}
\titleformat{\subsubsection}{\large\bfseries}{\thesubsubsection. \,}{0pt}{\large\bfseries}
\titleformat{\paragraph}{\bfseries}{\theparagraph. \,}{0pt}{\bfseries}

\setcounter{secnumdepth}{4}

% Table des matières :

\renewcommand{\cftsecleader}{\cftdotfill{\cftdotsep}}
\addtolength{\cftsecnumwidth}{10pt}

% Redéfinition des commandes :

\renewcommand*\thesection{\arabic{section}}
\renewcommand{\ker}{\mathrm{Ker}}

% Nouvelles commandes :

\newcommand{\website}{https://github.com/imbodj/SenCoursDeMaths}

\newcommand{\tr}[1]{\mathstrut ^t #1}
\newcommand{\im}{\mathrm{Im}}
\newcommand{\rang}{\operatorname{rang}}
\newcommand{\trace}{\operatorname{trace}}
\newcommand{\id}{\operatorname{id}}
\newcommand{\stab}{\operatorname{Stab}}
\newcommand{\paren}[1]{\left(#1\right)}
\newcommand{\croch}[1]{\left[ #1 \right]}
\newcommand{\Grdcroch}[1]{\Bigl[ #1 \Bigr]}
\newcommand{\grdcroch}[1]{\bigl[ #1 \bigr]}
\newcommand{\abs}[1]{\left\lvert #1 \right\rvert}
\newcommand{\limi}[3]{\lim_{#1\to #2}#3}
\newcommand{\pinf}{+\infty}
\newcommand{\minf}{-\infty}
%%%%%%%%%%%%%% ENSEMBLES %%%%%%%%%%%%%%%%%
\newcommand{\ensemblenombre}[1]{\mathbb{#1}}
\newcommand{\Nn}{\ensemblenombre{N}}
\newcommand{\Zz}{\ensemblenombre{Z}}
\newcommand{\Qq}{\ensemblenombre{Q}}
\newcommand{\Qqp}{\Qq^+}
\newcommand{\Rr}{\ensemblenombre{R}}
\newcommand{\Cc}{\ensemblenombre{C}}
\newcommand{\Nne}{\Nn^*}
\newcommand{\Zze}{\Zz^*}
\newcommand{\Zzn}{\Zz^-}
\newcommand{\Qqe}{\Qq^*}
\newcommand{\Rre}{\Rr^*}
\newcommand{\Rrp}{\Rr_+}
\newcommand{\Rrm}{\Rr_-}
\newcommand{\Rrep}{\Rr_+^*}
\newcommand{\Rrem}{\Rr_-^*}
\newcommand{\Cce}{\Cc^*}
%%%%%%%%%%%%%%  INTERVALLES %%%%%%%%%%%%%%%%%
\newcommand{\intff}[2]{\left[#1\;,\; #2\right]  }
\newcommand{\intof}[2]{\left]#1 \;, \;#2\right]  }
\newcommand{\intfo}[2]{\left[#1 \;,\; #2\right[  }
\newcommand{\intoo}[2]{\left]#1 \;,\; #2\right[  }

\providecommand{\newpar}{\\[\medskipamount]}

\newcommand{\annexessection}{%
  \newpage%
  \subsection*{Annexes}%
}

\providecommand{\lesson}[3]{%
  \title{#3}%
  \hypersetup{pdftitle={#2 : #3}}%
  \setcounter{section}{\numexpr #2 - 1}%
  \section{#3}%
  \fancyhead[R]{\truncate{0.73\textwidth}{#2 : #3}}%
}

\providecommand{\development}[3]{%
  \title{#3}%
  \hypersetup{pdftitle={#3}}%
  \section*{#3}%
  \fancyhead[R]{\truncate{0.73\textwidth}{#3}}%
}

\providecommand{\sheet}[3]{\development{#1}{#2}{#3}}

\providecommand{\ranking}[1]{%
  \title{Terminale #1}%
  \hypersetup{pdftitle={Terminale #1}}%
  \section*{Terminale #1}%
  \fancyhead[R]{\truncate{0.73\textwidth}{Terminale #1}}%
}

\providecommand{\summary}[1]{%
  \textit{#1}%
  \par%
  \medskip%
}

\tikzset{notestyleraw/.append style={inner sep=0pt, rounded corners=0pt, align=center}}

%\newcommand{\booklink}[1]{\website/bibliographie\##1}
\newcounter{reference}
\newcommand{\previousreference}{}
\providecommand{\reference}[2][]{%
  \needspace{20pt}%
  \notblank{#1}{
    \needspace{20pt}%
    \renewcommand{\previousreference}{#1}%
    \stepcounter{reference}%
    \label{reference-\previousreference-\thereference}%
  }{}%
  \todo[noline]{%
    \protect\vspace{20pt}%
    \protect\par%
    \protect\notblank{#1}{\cite{[\previousreference]}\\}{}%
    \protect\hyperref[reference-\previousreference-\thereference]{p. #2}%
  }%
}

\definecolor{devcolor}{HTML}{00695c}
\providecommand{\dev}[1]{%
  \reversemarginpar%
  \todo[noline]{
    \protect\vspace{20pt}%
    \protect\par%
    \bfseries\color{devcolor}\href{\website/developpements/#1}{[DEV]}
  }%
  \normalmarginpar%
}

% En-têtes :

\pagestyle{fancy}
\fancyhead[L]{\truncate{0.23\textwidth}{\thepage}}
\fancyfoot[C]{\scriptsize \href{\website}{\texttt{https://github.com/imbodj/SenCoursDeMaths}}}

% Couleurs :

\definecolor{property}{HTML}{ffeb3b}
\definecolor{proposition}{HTML}{ffc107}
\definecolor{lemma}{HTML}{ff9800}
\definecolor{theorem}{HTML}{f44336}
\definecolor{corollary}{HTML}{e91e63}
\definecolor{definition}{HTML}{673ab7}
\definecolor{notation}{HTML}{9c27b0}
\definecolor{example}{HTML}{00bcd4}
\definecolor{cexample}{HTML}{795548}
\definecolor{application}{HTML}{009688}
\definecolor{remark}{HTML}{3f51b5}
\definecolor{algorithm}{HTML}{607d8b}
%\definecolor{proof}{HTML}{e1f5fe}
\definecolor{exercice}{HTML}{e1f5fe}

% Théorèmes :

\theoremstyle{definition}
\newtheorem{theorem}{Théorème}

\newtheorem{property}[theorem]{Propriété}
\newtheorem{proposition}[theorem]{Proposition}
\newtheorem{lemma}[theorem]{Activité d'introduction}
\newtheorem{corollary}[theorem]{Conséquence}

\newtheorem{definition}[theorem]{Définition}
\newtheorem{notation}[theorem]{Notation}

\newtheorem{example}[theorem]{Exemple}
\newtheorem{cexample}[theorem]{Contre-exemple}
\newtheorem{application}[theorem]{Application}

\newtheorem{algorithm}[theorem]{Algorithme}
\newtheorem{exercice}[theorem]{Exercice}

\theoremstyle{remark}
\newtheorem{remark}[theorem]{Remarque}

\counterwithin*{theorem}{section}

\newcommand{\applystyletotheorem}[1]{
  \tcolorboxenvironment{#1}{
    enhanced,
    breakable,
    colback=#1!8!white,
    %right=0pt,
    %top=8pt,
    %bottom=8pt,
    boxrule=0pt,
    frame hidden,
    sharp corners,
    enhanced,borderline west={4pt}{0pt}{#1},
    %interior hidden,
    sharp corners,
    after=\par,
  }
}

\applystyletotheorem{property}
\applystyletotheorem{proposition}
\applystyletotheorem{lemma}
\applystyletotheorem{theorem}
\applystyletotheorem{corollary}
\applystyletotheorem{definition}
\applystyletotheorem{notation}
\applystyletotheorem{example}
\applystyletotheorem{cexample}
\applystyletotheorem{application}
\applystyletotheorem{remark}
%\applystyletotheorem{proof}
\applystyletotheorem{algorithm}
\applystyletotheorem{exercice}

% Environnements :

\NewEnviron{whitetabularx}[1]{%
  \renewcommand{\arraystretch}{2.5}
  \colorbox{white}{%
    \begin{tabularx}{\textwidth}{#1}%
      \BODY%
    \end{tabularx}%
  }%
}

% Maths :

\DeclareFontEncoding{FMS}{}{}
\DeclareFontSubstitution{FMS}{futm}{m}{n}
\DeclareFontEncoding{FMX}{}{}
\DeclareFontSubstitution{FMX}{futm}{m}{n}
\DeclareSymbolFont{fouriersymbols}{FMS}{futm}{m}{n}
\DeclareSymbolFont{fourierlargesymbols}{FMX}{futm}{m}{n}
\DeclareMathDelimiter{\VERT}{\mathord}{fouriersymbols}{152}{fourierlargesymbols}{147}

% Code :

\definecolor{greencode}{rgb}{0,0.6,0}
\definecolor{graycode}{rgb}{0.5,0.5,0.5}
\definecolor{mauvecode}{rgb}{0.58,0,0.82}
\definecolor{bluecode}{HTML}{1976d2}
\lstset{
  basicstyle=\footnotesize\ttfamily,
  breakatwhitespace=false,
  breaklines=true,
  %captionpos=b,
  commentstyle=\color{greencode},
  deletekeywords={...},
  escapeinside={\%*}{*)},
  extendedchars=true,
  frame=none,
  keepspaces=true,
  keywordstyle=\color{bluecode},
  language=Python,
  otherkeywords={*,...},
  numbers=left,
  numbersep=5pt,
  numberstyle=\tiny\color{graycode},
  rulecolor=\color{black},
  showspaces=false,
  showstringspaces=false,
  showtabs=false,
  stepnumber=2,
  stringstyle=\color{mauvecode},
  tabsize=2,
  %texcl=true,
  xleftmargin=10pt,
  %title=\lstname
}

\newcommand{\codedirectory}{}
\newcommand{\inputalgorithm}[1]{%
  \begin{algorithm}%
    \strut%
    \lstinputlisting{\codedirectory#1}%
  \end{algorithm}%
}




\begin{document}
  %<*content>
  \lesson{analysis}{21}{Factorisation des polynômes.}
  
  \subsection{Rappels sur le trinôme du second degré}
\subsubsection*{Equations du second degré }
\begin{definition} 
 On appelle équation du second degré toute équation pouvant se ramener sous la forme: $ ax^{2}+bx+c=0  \quad $ avec  $ a\neq0 $.
\end{definition}

Nous  rappelons la méthode de résolution vue en classe de  seconde.

Soit l'équation (E) suivante :$ \; ax^{2}+bx+c=0$.

On utilise le discriminant $\; \Delta=b^{2}-4ac $.

\begin{enumerate}
\item Si $ \Delta < 0 $ alors l'équation (E) n'a  pas de  solutions et  $ ax^{2}+bx+c $ n'est pas factorisable.
   \item Si $ \Delta = 0 $ alors l'équation (E) a  une seule  solution $x_{0}=-\dfrac{b}{2a}$    et   $ ax^{2}+bx+c= a\paren{x-x_{0}}^{2}$ 
\item Si $ \Delta > 0 $ alors l'équation (E) a deux solutions (ou racines) distinctes:


$ x_{1}=\dfrac{-b-\sqrt{\Delta}}{2a}$  ;  $x_{2}=\dfrac{-b+\sqrt{\Delta}}{2a}\quad$ et  $ ax^{2}+bx+c= a\paren{x-x_{1}}\paren{x-x_{2}}$ 
\end{enumerate}


\begin{remark}

 Si l'équation du  second degré est incomplète du type  $ax^{2}+bx=0 $  ou    $ ax^{2}+c=0$ alors il est inutile de calculer $ \Delta $ : on peut faire une factorisation pour trouver les racines.
\end{remark}

\begin{example}

Résolvons dans $ \Rr $  les équations suivantes puis factoriser le trinôme figurant au 1{\ier} membre.
\begin{multicols}{3}
\begin{enumerate}
\item $ 3x^{2}-2x-16=0 $
 \item $ -5x^{2}+x-1=0 $
 \item $ -4x^{2}+20x-25=0 $
   \item $ 2x^{2}+3x-1=0 $
    \item $ 7x^{2}+3x=0 $
\end{enumerate}
\end{multicols}
 \end{example}
 

  \begin{enumerate}
\item $ 3x^{2}-2x-16=0 $

On a : $ \Delta=(-2)^{2}-4\times3 (-16)=4-4(-48) =4+192=196$.

Donc $ \Delta>0 $ \; et \; $\sqrt{\Delta}=14$.

$ x_{1} =\dfrac{2-14}{6}=\dfrac{-12}{6}=-2$ \;  et\;  $ x_{2} =\dfrac{2+14}{6}=\dfrac{16}{6}=\dfrac{8}{3}$  \; ainsi :\; $ S=\accol{-2,\dfrac{8}{3} }$

Factorisation:\; $ 3x^{2}-2x-16=3\paren{x-\dfrac{8}{3}}\paren{x+2} =\paren{3x-8}\paren{x+2}$
 \item $ -5x^{2}+x-1=0 $
 
 On a : $ \Delta=(1)^{2}-4\times(-5)(-1)=1-20 =-19$
 
 Donc $ \Delta<0 $ \; ainsi\;  $ S=\emptyset $ et on ne peut  pas factoriser $ -5x^{2}+x-1 $
 \item $ -4x^{2}+20x-25=0 $
 
  On a : $ \Delta=(20)^{2}-4\times (-4)(-25)=400-400 =0$
  
Donc $ \Delta=0 $:\;  il y a une seule solution $ x_{0}=\dfrac{-20}{-8}=\dfrac{5}{2}$ \; ainsi \;$ S=\accol{\dfrac{5}{2} }$

Factorisation:\; $ -4x^{2}+20x-25=-4\paren{x-\dfrac{5}{2}}^{2}$
    \item $ 7x^{2}+3x=0 $.
    
    Ici, il est inutile de calculer $ \Delta $.
    
   On a:\;   $ 7x^{2}+3x=x\paren{7x+3}=0 $
   
   Donc \; $ x=0$ \; ou \; $7x+3=0 $\; (produit de facteurs nul)
   
   soit $ x=0 $\; ou \; $ x=-\dfrac{3}{7} $ \; et\; $ S=\accol{0,-\dfrac{3}{7} }$
   
   Factorisation:\;   déjà faite \;$ 7x^{2}+3x=x\paren{7x+3}.$
\end{enumerate}

 

\subsection*{Somme et produit des racines}
\begin{property}
\begin{itemize}
\item  Si l'équation $ax^2+bx+c=0$ \;a deux racines distinctes ou confondues (c'est-à-dire  $ \Delta  \geq 0  $),\; alors leur somme \; :  $ x_1+x_2=-\dfrac{b}{a} $ \;et\; leur produit \; : $ x_1 \times x_2=\dfrac{c}{a} $.

\item  Réciproquement, si deux nombres ont pour somme S et pour produit P, alors ils sont les solutions de l'équation du second degré : $ X^{2} - SX + P = 0 $    ou  du système  $ \begin{cases} x+y=S \\ xy=P \end{cases} $.
  \end{itemize}
\end{property}
\begin{example}

Résolvons dans $ \Rr^{2} $ le système suivant:\; $ \begin{cases} x+y=5 \\ xy=6 \end{cases} $.
 \end{example}

 On a:\; $ S=5 $ et $ P=6 $
 
   Résoudre un tel système, revient à résoudre l'équation $X^{2} - 5X + 6 = 0   $.
   
On trouve  $\Delta= 25- 24=1~~~\text{et} ~~ x_{1}=3~,~~x_{2}=2$.\; (faire les calculs).

Les solutions du système  sont les couples $(2,3) $ et $(3,2) $.


\subsection*{Equations bicarrées}

\begin{definition}

On appelle équation bicarrée, toute équation (E) pouvant se ramener sous la forme:      $ ax^{4}+bx^{2} +c=0$. 
\end{definition}

\medskip
Pour résoudre une telle équation, on procède par un changement d'inconnue en posant $ X=x^{2} $ qui mène à l'équation du  second degré  (E'): $ aX^{2}+bX +c=0$, ensuite on résout si possible les équations d'inconnue $ x $ suivantes ; $ x^{2}=X_{1} $   et $ x^{2}=X_{2} $ où $X_{1} $ et $ X_{2}$ sont les solutions possibles (E').

\begin{example}

Soit à résoudre l'équation: $ x^{4}-4x^{2}+3=0 $
\end{example}

En posant $ X=x^{2} $ ,\;l'équation devient  $ X^{2}-4X+3=0 $.\\Après calcul, on trouve comme solutions:\;  $ X_{1} =1$\;  et\;  $ X_{2}=3 $.
\medskip

On a  $ x^{2}=1 $ soit $ x=1 $ ou $ x=-1 $

\medskip
On a  $ x^{2}=3 $ soit $ x=\sqrt{3} $ ou $ x=-\sqrt{3} $ 

\medskip
D'où  $ S=\accol{-\sqrt{3},\;\sqrt{3},\;-1,\;1} $


\subsubsection*{Signe du trinôme $ax^2+bx+c$}
\begin{property}

Soit $ax^2+bx+c$ un trinôme  du second degré.
\begin{itemize}
	\item[$ \bullet $] Si $ \Delta <0 $ alors $ax^2+bx+c$ est  du signe de $a$ pour tout $x\in\Rr$.
	\renewcommand{\arraystretch}{1}
  \[\begin{array}{|c|c|}
\hline
x   &\minf~~~~~~~~~~~~\qquad   \qquad ~~~~\pinf\\ 
\hline
 ax^{2}+bx+c &   \text{signe\ de\ a }\\
\hline
\end{array}\]
	\item[$ \bullet $] Si $ \Delta =0 $ alors $ax^2+bx+c$ est  du signe de $a$ pour tout $x\neq-\dfrac{b}{2a}$ et s'annule en $-\dfrac{b}{2a}$.
	
	\begin{tikzpicture}
\tkzTabInit{$x$/0.5,$ ax^{2}+bx+c$/0.5}{$\minf$,  $-\tfrac{b}{2a}$,$\pinf$}
\tkzTabLine{,\text{signe\ de} \ $ a $,z,\text{signe\ de}\ $ a $}
\end{tikzpicture}
	\item[$ \bullet $] Si $ \Delta  > 0 $ alors  $ax^2+bx+c$ est :
\begin{itemize}
	\item  du signe de $a$ quand $x\in ]\minf;x_1[\cup]x_2;\pinf[$ \; (on suppose $ x_{1} < x_{2})$;
	\item du signe opposé de $a$ quand $x\in]x_1;x_2[$ ;
	\item s'annule en $x_1$ et en $x_2$.
\end{itemize}
\begin{tikzpicture}
\tkzTabInit{$x$/0.5,$ ax^{2}+bx+c$/0.5}{$\minf$, $ x_{1} $, $x_{2}$,$\pinf$}
\tkzTabLine{,\text{signe\ de} \ $ a $,z, \text{signe\ de} \ $ -a $,z,\text{signe\ de}\ $ a $}
\end{tikzpicture}
\end{itemize}
\end{property}

\begin{example}
Résolvons  dans $ \Rr $  les inéquations suivantes.
\begin{multicols}{3}

\begin{enumerate}[label=\textbf{\arabic*.}]
\item  $ 4x^{2}-x+2 \leq 0 $
\item  $ -x^{2}+x+2 >0 $
\item  $ x^{2}-\sqrt{28}x+7 > 0 $
  \end{enumerate}
\end{multicols}
 \end{example}
 
  $ \textbf{1.} \; 4x^{2}-x+2 \leq 0 $\qquad On a:\; $ \Delta=(-1)^{2}-4\times4\times2=1-32=-31 $ 

\begin{tikzpicture}[scale=0.8]
\tkzTabInit{$x$/0.5,$4x^{2}-x+2 $/0.5}{$\minf$,$\pinf$}
\tkzTabLine{,+,}
\end{tikzpicture}

$ S=\varnothing $

\vspace*{0.5cm}
$ \textbf{2.}\; -x^{2}+x+2 >0$ \qquad On a:\; $ \Delta=1^{2}-4\times(-1)\times2=1+8=9 $\\ 
On trouve $x_1= \dfrac{-1-3}{-2}=2$ \;et\; $x_2=\dfrac{-1+3}{-2}=-1 $

\begin{tikzpicture}[scale=0.8]
\tkzTabInit{$x$/0.5,$ -x^{2}+x+2$/0.5}{$\minf$, $ -1 $,$2$,$\pinf$}
\tkzTabLine{,-,z,+,z,-}
\end{tikzpicture}

$ S=\intoo{-1}{2} $


\vspace*{0.5cm}
$ \textbf{3.}\;  x^{2}-\sqrt{28}x+7 > 0$ \qquad On a:\; $ \Delta=(-\sqrt{28})^{2}-4\times 1\times7=28-28=0 $\\
Donc \; $ x_{0}=\dfrac{\sqrt{28}}{2} =\dfrac{2\sqrt{7}}{2}=\sqrt{7}$

\begin{tikzpicture}[scale=1.5]
\tkzTabInit{$x$/0.5,$\scriptsize{x^{2}-\sqrt{28}x+7}$/0.5}{$\minf$,$\sqrt{7}$, $\pinf$}
\tkzTabLine{,+,z,+,}
\end{tikzpicture}

\vspace*{0.3cm}
$ S=\intoo{\minf}{\sqrt{7}}\cup \intoo{\sqrt{7}}{\pinf}$

\subsection{Factorisation d'un polynôme}
\begin{definition}

Dire que le réel  $ \alpha $   est une  \textbf{racine}  ou un  \textbf{zéro} d'un polynôme  $ P(x) $,  signifie   que : $ P(\alpha)=0 $ .
\end{definition}

\begin{remark}

 Déterminer les racines d'un polynôme $ P(x) $, c'est résoudre l'équation $ P(x)=0 $. 
\end{remark}

Nous admettons le théorème  suivant.

\begin{theorem}

Soit $ P(x) $  un polynôme  et $ \alpha $  un réel.

$ \alpha $  est une racine de $ P(x) $  si et seulement si $ P(x) $  est factorisable par $ \paren{x-\alpha}$.\\ Dans ce cas il existe un polynôme $ Q(x) $ tel que : $  P(x)=\paren{x-\alpha}Q(x) $.

$ Q(x) $  est le quotient de  $ P(x) $ par $ \paren{x-\alpha}$  et $ d^{\circ}Q=d^{\circ}P-1 $.

\end{theorem}
\begin{remark}

 Si $\alpha $ et $\beta $ sont deux racines de  $ P(x) $  alors  $ P(x) $ est factorisable par $ \paren{x-\alpha} \paren{x-\beta} $ et dans ce cas il existe un polynôme $ Q(x)$  tel que $  P(x)=\paren{x-\alpha}\paren{x-\beta}Q(x) $  et $ d^{\circ}Q=d^{\circ}P-2 $.
\end{remark}
\begin{example}

Considérons le polynôme suivant:\;$ P(x)=  2x^{3}-5x^{2}-6x+9 $

 Montrons que $ P(x) $  est factorisable par $ (x-3) $.
 
 Ensuite déterminons le polynôme quotient $ Q(x) $ tel que: $  P(x)=\paren{x-3}Q(x) $ puis factorisons $ P(x) $ .
\end{example}

\begin{itemize}
\item  On a $ P(3)=  2\times 3^{3}-5\times 3^{2}-6\times 3+9 = 54-45 -18 +9=9-9=0$

Donc 3 est une racine de $ P(x) $  c'est-à-dire que $ P(x) $  est factorisable par $ \paren{x-3}.$ \\D'après le théorème précédent, il existe un polynôme $ Q(x) $ tel que :                         $  P(x)=\paren{x-3}Q(x) $.


 \item  Or  $ P(x) $  est de degré trois donc $ Q(x) $ sera de degré  deux. Par conséquent nous devons déterminer trois réels $ a$, $ b $  et $ c$ tels que 
$  P(x)=\paren{x-3}\paren{ax^{2} +bx +c} $ 
\end{itemize}
 Nous proposons ici la  méthode  Hörner\footnote{William George Hörner mathématicien allemand(1819-1845}  pour  déterminer de $ Q(x) $.
 \bigskip
 
$ \blacktriangleright $ \textbf{ Méthode de   Hörner}
 

On utilise la disposition suivante appelée méthode de Hörner:
\renewcommand{\arraystretch}{1.3}
\[\begin{array}{|c|c|c|c|c|}
\hline   & 2 & -5 &  -6 &  9  \\
\hline
3  & \boxtimes   & 6 & 3 & -9   \\
\hline
 & 2 & 1 & -3 & 0 \\
\hline
\end{array}\]
Les valeurs $ 2 $, $ 1 $ et $ -3 $ figurant dans la dernière ligne, correspondent respectivement à celles des coefficients   $a ,$ $ b $ et  $c $ de \; $ Q(x). $  Soit \; $ Q(x)=2x^{2}+x-3 $.

$ P(\alpha) $  correspond à la valeur  $ 0 $ figurant dans la dernière case de la dernière ligne du tableau de Hörner. Cette valeur n'est pas nécessairement nulle.\\ Ce tableau permet donc de calculer $ P(\alpha) $ et de trouver en même temps les coefficients du polynôme 
$ Q(x). $ 
\bigskip

\textbf{Factorisation de $ P(x) $ }

Maintenant factorisons au mieux $ P(x) $.

On a : $  P(x)=\paren{x-3}\paren{2x^{2}+x-3} $ \; (attention ceci n'est pas la factorisation demandée  !)

On va  continuer la factorisation  \textbf{si possible} dans\; $2x^{2}+x-3$.

$ \Delta=1-4\times2(-3)=25 $\; et $ x_{1}=\dfrac{-1-5}{4}=-\dfrac{3}{2} $,\; $ x_{2}=\dfrac{-1+5}{4}=1 $.\\Donc \;$2x^{2}+x-3=2\paren{x+\dfrac{3}{2}}\paren{x-1}=\paren{2x+3}\paren{x-1}$. \; (attention ceci n'est pas la factorisation demandée  !)\\

On remplace $ \paren{2x^{2}+x-3} $  par $ \paren{2x+3}\paren{x-1} $ dans $  P(x) $.\\
Finalement \; $ P(x)=\paren{x-3}\paren{2x+3}\paren{x-1} $\; cette expression est la factorisation de $ P(x) $.

\bigskip

\begin{remark}
 Dans la démarche précédente, on a trouvé  toutes les racines du polynôme $ P(x) $.  C'est-à-dire:\; $ 3$,  $ -\dfrac{3}{2} $ et $1 $.
 
On pourrait aussi vous demander  d'étudier le signe  $ P(x) $ à l'aide d'un tableau de signes puis de résoudre une inéquation comme nous le verrons dans les exercices.
\end{remark}

\begin{remark}

 On pourrait aussi utliser les méthodes vues en classe de première : la division ou l'identification des coefficients.
\end{remark}
 
  %</content>
\end{document}
