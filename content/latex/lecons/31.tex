\documentclass[12pt, a4paper]{report}

% LuaLaTeX :

\RequirePackage{iftex}
\RequireLuaTeX

% Packages :

\usepackage[french]{babel}
%\usepackage[utf8]{inputenc}
%\usepackage[T1]{fontenc}
\usepackage[pdfencoding=auto, pdfauthor={Hugo Delaunay}, pdfsubject={Mathématiques}, pdfcreator={agreg.skyost.eu}]{hyperref}
\usepackage{amsmath}
\usepackage{amsthm}
%\usepackage{amssymb}
\usepackage{stmaryrd}
\usepackage{tikz}
\usepackage{tkz-euclide}
\usepackage{fontspec}
\defaultfontfeatures[Erewhon]{FontFace = {bx}{n}{Erewhon-Bold.otf}}
\usepackage{fourier-otf}
\usepackage[nobottomtitles*]{titlesec}
\usepackage{fancyhdr}
\usepackage{listings}
\usepackage{catchfilebetweentags}
\usepackage[french, capitalise, noabbrev]{cleveref}
\usepackage[fit, breakall]{truncate}
\usepackage[top=2.5cm, right=2cm, bottom=2.5cm, left=2cm]{geometry}
\usepackage{enumitem}
\usepackage{tocloft}
\usepackage{microtype}
%\usepackage{mdframed}
%\usepackage{thmtools}
\usepackage{xcolor}
\usepackage{tabularx}
\usepackage{xltabular}
\usepackage{aligned-overset}
\usepackage[subpreambles=true]{standalone}
\usepackage{environ}
\usepackage[normalem]{ulem}
\usepackage{etoolbox}
\usepackage{setspace}
\usepackage[bibstyle=reading, citestyle=draft]{biblatex}
\usepackage{xpatch}
\usepackage[many, breakable]{tcolorbox}
\usepackage[backgroundcolor=white, bordercolor=white, textsize=scriptsize]{todonotes}
\usepackage{luacode}
\usepackage{float}
\usepackage{needspace}
\everymath{\displaystyle}

% Police :

\setmathfont{Erewhon Math}

% Tikz :

\usetikzlibrary{calc}
\usetikzlibrary{3d}

% Longueurs :

\setlength{\parindent}{0pt}
\setlength{\headheight}{15pt}
\setlength{\fboxsep}{0pt}
\titlespacing*{\chapter}{0pt}{-20pt}{10pt}
\setlength{\marginparwidth}{1.5cm}
\setstretch{1.1}

% Métadonnées :

\author{agreg.skyost.eu}
\date{\today}

% Titres :

\setcounter{secnumdepth}{3}

\renewcommand{\thechapter}{\Roman{chapter}}
\renewcommand{\thesubsection}{\Roman{subsection}}
\renewcommand{\thesubsubsection}{\arabic{subsubsection}}
\renewcommand{\theparagraph}{\alph{paragraph}}

\titleformat{\chapter}{\huge\bfseries}{\thechapter}{20pt}{\huge\bfseries}
\titleformat*{\section}{\LARGE\bfseries}
\titleformat{\subsection}{\Large\bfseries}{\thesubsection \, - \,}{0pt}{\Large\bfseries}
\titleformat{\subsubsection}{\large\bfseries}{\thesubsubsection. \,}{0pt}{\large\bfseries}
\titleformat{\paragraph}{\bfseries}{\theparagraph. \,}{0pt}{\bfseries}

\setcounter{secnumdepth}{4}

% Table des matières :

\renewcommand{\cftsecleader}{\cftdotfill{\cftdotsep}}
\addtolength{\cftsecnumwidth}{10pt}

% Redéfinition des commandes :

\renewcommand*\thesection{\arabic{section}}
\renewcommand{\ker}{\mathrm{Ker}}

% Nouvelles commandes :

\newcommand{\website}{https://github.com/imbodj/SenCoursDeMaths}

\newcommand{\tr}[1]{\mathstrut ^t #1}
\newcommand{\im}{\mathrm{Im}}
\newcommand{\rang}{\operatorname{rang}}
\newcommand{\trace}{\operatorname{trace}}
\newcommand{\id}{\operatorname{id}}
\newcommand{\stab}{\operatorname{Stab}}
\newcommand{\paren}[1]{\left(#1\right)}
\newcommand{\croch}[1]{\left[ #1 \right]}
\newcommand{\Grdcroch}[1]{\Bigl[ #1 \Bigr]}
\newcommand{\grdcroch}[1]{\bigl[ #1 \bigr]}
\newcommand{\abs}[1]{\left\lvert #1 \right\rvert}
\newcommand{\limi}[3]{\lim_{#1\to #2}#3}
\newcommand{\pinf}{+\infty}
\newcommand{\minf}{-\infty}
%%%%%%%%%%%%%% ENSEMBLES %%%%%%%%%%%%%%%%%
\newcommand{\ensemblenombre}[1]{\mathbb{#1}}
\newcommand{\Nn}{\ensemblenombre{N}}
\newcommand{\Zz}{\ensemblenombre{Z}}
\newcommand{\Qq}{\ensemblenombre{Q}}
\newcommand{\Qqp}{\Qq^+}
\newcommand{\Rr}{\ensemblenombre{R}}
\newcommand{\Cc}{\ensemblenombre{C}}
\newcommand{\Nne}{\Nn^*}
\newcommand{\Zze}{\Zz^*}
\newcommand{\Zzn}{\Zz^-}
\newcommand{\Qqe}{\Qq^*}
\newcommand{\Rre}{\Rr^*}
\newcommand{\Rrp}{\Rr_+}
\newcommand{\Rrm}{\Rr_-}
\newcommand{\Rrep}{\Rr_+^*}
\newcommand{\Rrem}{\Rr_-^*}
\newcommand{\Cce}{\Cc^*}
%%%%%%%%%%%%%%  INTERVALLES %%%%%%%%%%%%%%%%%
\newcommand{\intff}[2]{\left[#1\;,\; #2\right]  }
\newcommand{\intof}[2]{\left]#1 \;, \;#2\right]  }
\newcommand{\intfo}[2]{\left[#1 \;,\; #2\right[  }
\newcommand{\intoo}[2]{\left]#1 \;,\; #2\right[  }

\providecommand{\newpar}{\\[\medskipamount]}

\newcommand{\annexessection}{%
  \newpage%
  \subsection*{Annexes}%
}

\providecommand{\lesson}[3]{%
  \title{#3}%
  \hypersetup{pdftitle={#2 : #3}}%
  \setcounter{section}{\numexpr #2 - 1}%
  \section{#3}%
  \fancyhead[R]{\truncate{0.73\textwidth}{#2 : #3}}%
}

\providecommand{\development}[3]{%
  \title{#3}%
  \hypersetup{pdftitle={#3}}%
  \section*{#3}%
  \fancyhead[R]{\truncate{0.73\textwidth}{#3}}%
}

\providecommand{\sheet}[3]{\development{#1}{#2}{#3}}

\providecommand{\ranking}[1]{%
  \title{Terminale #1}%
  \hypersetup{pdftitle={Terminale #1}}%
  \section*{Terminale #1}%
  \fancyhead[R]{\truncate{0.73\textwidth}{Terminale #1}}%
}

\providecommand{\summary}[1]{%
  \textit{#1}%
  \par%
  \medskip%
}

\tikzset{notestyleraw/.append style={inner sep=0pt, rounded corners=0pt, align=center}}

%\newcommand{\booklink}[1]{\website/bibliographie\##1}
\newcounter{reference}
\newcommand{\previousreference}{}
\providecommand{\reference}[2][]{%
  \needspace{20pt}%
  \notblank{#1}{
    \needspace{20pt}%
    \renewcommand{\previousreference}{#1}%
    \stepcounter{reference}%
    \label{reference-\previousreference-\thereference}%
  }{}%
  \todo[noline]{%
    \protect\vspace{20pt}%
    \protect\par%
    \protect\notblank{#1}{\cite{[\previousreference]}\\}{}%
    \protect\hyperref[reference-\previousreference-\thereference]{p. #2}%
  }%
}

\definecolor{devcolor}{HTML}{00695c}
\providecommand{\dev}[1]{%
  \reversemarginpar%
  \todo[noline]{
    \protect\vspace{20pt}%
    \protect\par%
    \bfseries\color{devcolor}\href{\website/developpements/#1}{[DEV]}
  }%
  \normalmarginpar%
}

% En-têtes :

\pagestyle{fancy}
\fancyhead[L]{\truncate{0.23\textwidth}{\thepage}}
\fancyfoot[C]{\scriptsize \href{\website}{\texttt{https://github.com/imbodj/SenCoursDeMaths}}}

% Couleurs :

\definecolor{property}{HTML}{ffeb3b}
\definecolor{proposition}{HTML}{ffc107}
\definecolor{lemma}{HTML}{ff9800}
\definecolor{theorem}{HTML}{f44336}
\definecolor{corollary}{HTML}{e91e63}
\definecolor{definition}{HTML}{673ab7}
\definecolor{notation}{HTML}{9c27b0}
\definecolor{example}{HTML}{00bcd4}
\definecolor{cexample}{HTML}{795548}
\definecolor{application}{HTML}{009688}
\definecolor{remark}{HTML}{3f51b5}
\definecolor{algorithm}{HTML}{607d8b}
%\definecolor{proof}{HTML}{e1f5fe}
\definecolor{exercice}{HTML}{e1f5fe}

% Théorèmes :

\theoremstyle{definition}
\newtheorem{theorem}{Théorème}

\newtheorem{property}[theorem]{Propriété}
\newtheorem{proposition}[theorem]{Proposition}
\newtheorem{lemma}[theorem]{Activité d'introduction}
\newtheorem{corollary}[theorem]{Conséquence}

\newtheorem{definition}[theorem]{Définition}
\newtheorem{notation}[theorem]{Notation}

\newtheorem{example}[theorem]{Exemple}
\newtheorem{cexample}[theorem]{Contre-exemple}
\newtheorem{application}[theorem]{Application}

\newtheorem{algorithm}[theorem]{Algorithme}
\newtheorem{exercice}[theorem]{Exercice}

\theoremstyle{remark}
\newtheorem{remark}[theorem]{Remarque}

\counterwithin*{theorem}{section}

\newcommand{\applystyletotheorem}[1]{
  \tcolorboxenvironment{#1}{
    enhanced,
    breakable,
    colback=#1!8!white,
    %right=0pt,
    %top=8pt,
    %bottom=8pt,
    boxrule=0pt,
    frame hidden,
    sharp corners,
    enhanced,borderline west={4pt}{0pt}{#1},
    %interior hidden,
    sharp corners,
    after=\par,
  }
}

\applystyletotheorem{property}
\applystyletotheorem{proposition}
\applystyletotheorem{lemma}
\applystyletotheorem{theorem}
\applystyletotheorem{corollary}
\applystyletotheorem{definition}
\applystyletotheorem{notation}
\applystyletotheorem{example}
\applystyletotheorem{cexample}
\applystyletotheorem{application}
\applystyletotheorem{remark}
%\applystyletotheorem{proof}
\applystyletotheorem{algorithm}
\applystyletotheorem{exercice}

% Environnements :

\NewEnviron{whitetabularx}[1]{%
  \renewcommand{\arraystretch}{2.5}
  \colorbox{white}{%
    \begin{tabularx}{\textwidth}{#1}%
      \BODY%
    \end{tabularx}%
  }%
}

% Maths :

\DeclareFontEncoding{FMS}{}{}
\DeclareFontSubstitution{FMS}{futm}{m}{n}
\DeclareFontEncoding{FMX}{}{}
\DeclareFontSubstitution{FMX}{futm}{m}{n}
\DeclareSymbolFont{fouriersymbols}{FMS}{futm}{m}{n}
\DeclareSymbolFont{fourierlargesymbols}{FMX}{futm}{m}{n}
\DeclareMathDelimiter{\VERT}{\mathord}{fouriersymbols}{152}{fourierlargesymbols}{147}

% Code :

\definecolor{greencode}{rgb}{0,0.6,0}
\definecolor{graycode}{rgb}{0.5,0.5,0.5}
\definecolor{mauvecode}{rgb}{0.58,0,0.82}
\definecolor{bluecode}{HTML}{1976d2}
\lstset{
  basicstyle=\footnotesize\ttfamily,
  breakatwhitespace=false,
  breaklines=true,
  %captionpos=b,
  commentstyle=\color{greencode},
  deletekeywords={...},
  escapeinside={\%*}{*)},
  extendedchars=true,
  frame=none,
  keepspaces=true,
  keywordstyle=\color{bluecode},
  language=Python,
  otherkeywords={*,...},
  numbers=left,
  numbersep=5pt,
  numberstyle=\tiny\color{graycode},
  rulecolor=\color{black},
  showspaces=false,
  showstringspaces=false,
  showtabs=false,
  stepnumber=2,
  stringstyle=\color{mauvecode},
  tabsize=2,
  %texcl=true,
  xleftmargin=10pt,
  %title=\lstname
}

\newcommand{\codedirectory}{}
\newcommand{\inputalgorithm}[1]{%
  \begin{algorithm}%
    \strut%
    \lstinputlisting{\codedirectory#1}%
  \end{algorithm}%
}




\begin{document}
  %<*content>
  \lesson{analysis}{31}{Calcul intégral}
  

\subsection{Primitives d'une fonction numérique}
\begin{lemma}
Soit la fonction $ f : x \longmapsto 2x+3 $.

Calcule la dérivée de chacune des fonctions F ; G ; H définies par :

F$ (x)=x^{2} +3x+7$;  G$ (x)= x^{2} +3x-30$  et  H$ (x)=\paren{x+\frac{3}{2}}^{2} +10$. Que remarques-tu ?
\medskip

Pour tout $ x\in \mathscr{D}f $,  F$' (x)=f(x)$;  H$' (x)=f(x)$;  G$' (x)=f(x)$;
\end{lemma}

\medskip
On dit que F ; G ; H sont  \textbf{des primitives de f sur Df}.

\begin{definition}
Soit $ f $ une fonction définie sur un intervalle $I$.

 On appelle fonction primitive de $f$ sur $I$ , toute fonction $F$ telle que :\\
$ \text{pour tout } \; x\in \text{I},\;  \text{F}'(x) = f (x).$
\end{definition}

\medskip


\begin{example}

Vérifions que la fonction : $\; F(x)=\dfrac{\eexp{2x}+1}{\eexp{x}+5}\; $  est une primitive sur $ \;\Rr\;  $ de la fonction $\; f\; $  définie sur $\; \Rr\; $  par: \\

 $  f(x)= \dfrac{\eexp{3x}+10\eexp{2x}-\eexp{x}}{(\eexp{x}+5)^{2}}$

\bigskip
Pour cela dérivons la fonction $F$.

\medskip
On a  $F'(x)=\dfrac{2\eexp{2x}(\eexp{x}+5)-\eexp{x}(\eexp{2x}+1)}{(\eexp{x}+5)^{2}}  =\dfrac{2\eexp{3x}+10\eexp{2x}-\eexp{3x}-\eexp{x}}{(\eexp{x}+5)^{2}}=\dfrac{\eexp{3x}+10\eexp{2x}-\eexp{x}}{(\eexp{x}+5)^{2}}$

Ainsi $F$ est une primitive de $ f. $
\end{example}



\begin{remark}
 Si $F$ est une primitive de $ f $ sur I alors  toute  autre primitive   de $ f $ est de la  forme $ F(x)+c $ où $ c $ est une constante réelle.
\end{remark}

\medskip
\subsection*{Primitives des fonctions usuelles}

$$\begin{array}{|c|c|}
\hline
\textbf{Fonctions} \;f  &\textbf{ Primitives} \; F   \\ 
\hline
 a \; \text{un réel}  & ax \\
\hline
x^{n}    & \dfrac{x^{n+1}}{n+1} \\
\hline
 \dfrac{1}{\sqrt{x}}   &  2\sqrt{x}   \\
\hline
 \dfrac{1}{x}     &  \ln |x|     \\
\hline
\eexp{x}     &\eexp{x}    \\
\hline
\end{array}$$


\subsection*{Opérations sur les primitives}
\begin{property}
\noindent
Si F est une primitive de $f$ sur I et si G est une primitive de $g$ sur I alors:

\medskip
\noindent
$ \bullet $  F $+$ G est une primitive de $f + g$ sur I.

\medskip
\noindent
$ \bullet  $  Pour tout réel  $k$ ,\;  $k$F est une primitive de $kf$ sur I.
\end{property}
\medskip

Le tableau suivant découle des règles de dérivation des fonctions.

$ u $ désigne une fonction dérivable sur un intervalle I.


$$\begin{array}{|c|c|}
\hline
\textbf{Fonction}  &\textbf{ Primitive}     \\ 
\hline
 u'u^{n}   & \dfrac{u^{n+1}}{n+1}   \\
\hline
\dfrac{u'}{\sqrt{u}}    &  2\sqrt{u}    \\
\hline
\dfrac{u'}{u}  & \ln |u|      \\
\hline
u'\eexp{u}   &   \eexp{u}    \\
\hline
\dfrac{u'}{u^{n}}     &- \dfrac{1}{n-1} \frac{1}{u^{n-1}} \\
\hline

\end{array}$$



\medskip 
\begin{example}

Déterminons une primitive de chacune des fonctions suivantes.
\begin{enumerate}
\item $ f(x)= x^{2}-2x+5 $
\item  $ g(x)= 2x(x^{2}+3)^{2} $
\item $ h(x)=\eexp{x}(\eexp{x}+2)^{2} $

\item $ i(x)=\dfrac{2\eexp{x}}{\eexp{x}+1} $
\item $ j(x)=2+\dfrac{3}{(3x+4)^{2}} $
\item $ k(x)= x + 2-\dfrac{4}{2x-2}$
\end{enumerate}

\end{example}
\begin{proof}
\begin{enumerate}
\item  On a:\; F$(x)= \dfrac{1}{3}x^{3}-x^{2}+5x$
\item  $ g(x)= 2x(x^{2}+3)^{2} $  est de la forme  $f(x)= u'u^{n} $.

 Par suite  G$(x)= \dfrac{1}{3}(x^{2}+3)^{3} $.
\item $ h(x)=\eexp{x}(\eexp{x}+2)^{2} $ est de la forme  $f(x)= u'u^{n} $.                   

Par suite $ H(x)=\dfrac{1}{3}(\eexp{x}+2)^{3} $.
\item $ i(x)=\dfrac{2\eexp{x}}{\eexp{x}+1} $ est de la forme   $\dfrac{a u'}{u}$.


 Par suite   I$(x)=\ln (\eexp{x}+1) $.

\item $ j(x)=2+\dfrac{3}{(3x+4)^{2}} $  est   une somme de deux fonctions:  l'une étant une constante égale à $ 2 $  et l'autre de la forme $ \dfrac{ u'}{u^{2}}$.

Par conséquent  J$(x)=2x-\dfrac{1}{3x+4} $.

\item $ k(x)= x + 2-\dfrac{4}{2x-2}$\;  on fait la somme des deux primitives d'où:

K$(x)= \dfrac{1}{2}x^{2} + 2x-2\ln (2x-2)$
\end{enumerate}
 \end{proof}

\subsection{Intégrales de fonctions}
\begin{definition}
Soient $ f $ une fonction définie sur un intervalle I et $  F $ une de ces primitives, soient $ a $ et $ b $ deux réels   de I.

Le nombre réel $ F(b)-F(a) $ est appelé intégrale de $ f $ entre $ a $ et $ b $ et est notée   $\; \inte{a}{b}{f(x)}{x} $. 
\\ Ainsi on a:
$$ \inte{a}{b}{f(x)}{x} =F(b)-F(a) $$

\end{definition}
 \textbf{Attention}\\
L'ordre de $a$ et de $b$ est important.\\
Le nombre $a$ est appelé borne inférieure et $b$ la borne supérieure de l'intégrale.
\medskip
 
 \textbf{Vocabulaire  et notations}
 \begin{itemize}
 \item Le réel  $\; \inte{a}{b}{f(x)}{x}\;  $ \; se lit  <<  intégrale  de $a$  à $b$  \;f$(x)\; dx$  >>
\item Pour toute primitive F de f, on écrit \;   $ \inte{a}{b}{f(x)}{x}=\croch {F(x)}^{b}_{a} =F(b)-F(a) $.\\
L'expression \;$ \croch {F(x)}^{b}_{a} $\; se lit <<  $ F(x) $>>  pris entre a et b.
\item  Dans l'écriture  $ \inte{a}{b}{f(x)}{x} $, on peut remplacer la lettre $ x $ par n'importe quelle    lettre  et on peut écrire   $ \;\inte{a}{b}{f(x)}{x}  =  \inte{a}{b}{f(u)}{u} =  \inte{a}{b}{f(t)}{t} $.\; On dit que $ x $ est une variable muette,  elle n'intervient pas dans le résultat.
  \end{itemize}
  
  \medskip
  \begin{example}
  
Calculons $\;  \inte{0}{1}{(x^{2}-1)}{x} \quad$   et  \quad$\inte{-1}{0}{\eexp{-2x}}{x}  $.
 
  \bigskip
$ \bullet \; $   Une primitive de la fonction  $ x\longmapsto x^{2}-1 \; $  sur $\;  \intff{0}{1} $ est la fonction  F: $ x\longmapsto \frac{1}{3}x^{3}-x $\\ On a donc  $\;  \inte{0}{1}{(x^{2}-1)}{x}= \croch{\frac{1}{3}x^{3}- x}^{1}_{0}=F(1)-F(0) =-\frac{2}{3}$.

\medskip

$ \bullet \; $   Une primitive de la fonction  $ x\longmapsto \eexp{-2x}$  sur $ \Rr $ est la fonction  G: $ x\longmapsto  \frac{-\eexp{-2x}}{2} $\\


On a donc  $\;  \inte{-1}{0}{\eexp{-2x}}{x}=G(0)-G(-1)= \frac{\eexp{2} -1}{2}$
\end{example}
 

  \subsection*{Propriétés  de l'intégrale}
  \begin{property}
 Soit $f$  et $g$ deux fonctions continues sur un intervalle I contenant les réels  $a $, $b  $  et $c $. Alors:     
 \begin{itemize}
  \item[$  \bullet$]  $ \inte{a}{a}{f(x)}{x} =0$ 
  \item[$  \bullet$] $ \inte{a}{b}{f(x)}{x} = -\inte{b}{a}{f(x)}{x}$
  \item[$  \bullet$] $ \inte{a}{c}{f(x)}{x} +\inte{c}{b}{f(x)}{x}=\inte{a}{b}{f(x)}{x}$\;
  
  (Relation de Chasles)
\item[$  \bullet$]   $ \inte{a}{b}{\alpha f(x)}{x} =\alpha\inte{a}{b}{f(x)}{x}$;\; $\alpha\in\Rr$.
   \item[$  \bullet$] $ \inte{a}{b}{f(x)}{x} +\inte{a}{b}{g(x)}{x}=\inte{a}{b}{f(x)+g(x)}{x}$
 
\end{itemize}


 \end{property}
 

 
 \subsection{Calcul d'aires}

Le plan est muni d'un repère orthogonal  $ \oij $. \\ L'unité d'aire notée par \textbf{u.a},\;  est l'aire du rectangle de dimensions     $ ||\overrightarrow{i}|| $  et   $ ||\overrightarrow{j}|| $.

\begin{tikzpicture}
\draw[thick,black,dashed] (-1,0)  (2,2);
\draw[thick] (-1,0) -- (2,0);
\draw [thick](0,0) -- (0,2);
\draw[thick,red,->,>=stealth'] (0,0) -- (0.5,0) node[midway,below] {$\overrightarrow{i}$};
\draw[thick,red,->,>=stealth'] (0,0) -- (0,1) node[midway,left] {$\overrightarrow{j}$};
\node[below left] at (0,0) { O};
\clip (-1,0) rectangle (2,2);
\draw[thick,black,dashed] (0.5,1) -- (0.5,0);
\draw[thick,black,dashed] (0.5,1) -- (0,1);
\end{tikzpicture}


 
 \begin{definition}
 

 Le  plan est  muni d'un repère orthogonal.\\ Soit $f$ une fonction continue et\textbf{ positive} sur un intervalle $ \intff{a}{b} $  et  $F$ une primitive de $f$ sur $ \intff{a}{b} $.\\

  L'aire ( en u.a) de la partie du plan délimitée par la courbe de $f$, l'axe des abscisses et les droites d'équations $\; x=a\;$ et $ \;x=b$, est égale à l'intégrale  $\; \inte{a}{b}{f(x)}{x} $.
\end{definition}
\begin{center}
\begin{tikzpicture}[scale=0.8]
\draw[gray!25] (-4,-1) grid (3,3);
\fill[bottom color=blue!25,top color=blue!35!black,opacity=0.5] (-3,0)  plot[domain=-3:2,samples=100] (\x,{0.125*\x*\x*\x+0.125*\x*\x-\x+1}) -- (2,0) -- (-3,0);
\draw[thick,->,>=stealth'] (-4,0) -- (3,0);
\draw[thick,->,>=stealth'] (0,-1) -- (0,3);
\foreach\x in {}
{
	\draw[thick] (\x,0.1) -- (\x,-0.1) node[below] {\tiny\x};
}
\foreach\y in {}
{
	\draw[thick] (-.1,\y) -- (0.1,\y) node[right] {\tiny\y};
}
\node[below left] at (0,0) {\tiny O};
\draw[thick,blue!50!black] plot[domain=-4:3,samples=100] (\x,{0.125*\x*\x*\x+0.125*\x*\x-\x+1}) node[right] {$\mathscr{C}_f$};
\draw[very thick,blue!50!black] plot[domain=-3:2,samples=100] (\x,{0.125*\x*\x*\x+0.125*\x*\x-\x+1});
\node[white] at (-1.5,1) {$\pmb{ \inte{a}{b}{f(x)}{x}}$};
\node  at (-3,-0.5) {$ a $};
\node  at (2,-0.5) {$ b $};
\end{tikzpicture}
\end{center}

\begin{remark}

Lorsque $ f $ une fonction continue et\textbf{ négative} sur un intervalle $\; \intff{a}{b} $.\\
 L'aire ( en u.a) de la partie du plan délimitée par la courbe de $f$, l'axe des abscisses et les droites d'équations $ x=a$ et $ x=b$, est égale à l'intégrale  $ \;-\inte{a}{b}{f(x)}{x} $.
\end{remark}

\medskip

\begin{exercice}

Soit  la fonction $ f $  définie sur  $ \Rr $  par $ f(x)=x^{3}-3x $.\\
Etudier les variations de $ f $ et construire sa courbe dans un repère  orthonormé d'unité 2 cm.\\
En déduire l'aire en cm$ ^{2} $ de la partie comprise entre les
droites d'équation $x = 0$ et $x = \sqrt{3}$, l'axe des abscisses
et la courbe représentative de $ f$.
\end{exercice}

\begin{proof}
$ f $ est une fonction dérivable sur $ \Rr $  et $ f'(x)=3x^{2}-3=3(x-1)(x+1) $.  D'où le tableau de variation suivant.

\begin{tikzpicture}[scale=0.6]
\tkzTab[lgt=1.5]
{
	$x$ / 1 ,
	$f^{\prime}(x)$ / 1,
	$f(x)$ / 2
}
{ $-\infty$ , $-1$ , $1$ , $+\infty$ }
{ , + , z , - , z , + , }
{ -/$-\infty$ ,  +/$2$, -/$-2$ , +/$+\infty$ }
\end{tikzpicture}

\begin{tikzpicture}[scale=0.8]
\draw[gray!25] (-4,-1) grid (3,3);
%\fill[bottom color=blue!25,top color=blue!35!black,opacity=0.5] (-3,0)  plot[domain=-3:2,samples=100] (\x,{0.125*\x*\x*\x+0.125*\x*\x-\x+1}) -- (2,0) -- (-3,0);
\draw[thick,->,>=stealth'] (-3,0) -- (3,0);
\draw[thick,->,>=stealth'] (0,-2) -- (0,2);
\foreach\x in {}
{
	\draw[thick] (\x,0.1) -- (\x,-0.1) node[below] {\tiny\x};
}
\foreach\y in {}
{
	\draw[thick] (-.1,\y) -- (0.1,\y) node[right] {\tiny\y};
}
\node[below left] at (0,0) { O};
\draw[thick,blue!50!black] plot[domain=-1.9:2,samples=50] (\x,{\x*\x*\x-3*\x}) node[right] {$\mathscr{C}_f$};
%\draw[very thick,blue!50!black] plot[domain=-3:2,samples=100] (\x,%{0.125*\x*\x*\x+0.125*\x*\x-\x+1});
%\node[white] at (-1.5,1) {$\pmb{ \inte{a}{b}{f(x)}{x}}$};
%\node  at (-3,-0.5) {$ a $};
\node  at (2,-0.5) {$ \sqrt{3} $};
\end{tikzpicture}


Sur l'intervalle $ \intff{0}{\sqrt{3}} $  la fonction $ f $  est continue et négative  et a pour primitive $ F(x)=\dfrac{1}{4}x^{4}-\dfrac{3}{2}x^{2}$.

Une unité d'aire est égale à 4 cm$ ^{2} $.

L'aire de la partie en question est égale à:

$\mathcal{A}=-\inte{0}{\sqrt{3}}{f(x)}{x} = -\inte{0}{\sqrt{3}}{(x^{3}-3x)}{x}=-(F(\sqrt{3})-F(0))=2.25$

Soit  en unité d'aire  $\mathcal{A}= 2.25\times 4 cm^{2}=9 cm^{2} $
\end{proof}

 
 \end{document}  