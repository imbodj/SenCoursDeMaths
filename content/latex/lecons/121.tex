\documentclass[12pt, a4paper]{report}

% LuaLaTeX :

\RequirePackage{iftex}
\RequireLuaTeX

% Packages :

\usepackage[french]{babel}
%\usepackage[utf8]{inputenc}
%\usepackage[T1]{fontenc}
\usepackage[pdfencoding=auto, pdfauthor={Hugo Delaunay}, pdfsubject={Mathématiques}, pdfcreator={agreg.skyost.eu}]{hyperref}
\usepackage{amsmath}
\usepackage{amsthm}
%\usepackage{amssymb}
\usepackage{stmaryrd}
\usepackage{tikz}
\usepackage{tkz-euclide}
\usepackage{fontspec}
\defaultfontfeatures[Erewhon]{FontFace = {bx}{n}{Erewhon-Bold.otf}}
\usepackage{fourier-otf}
\usepackage[nobottomtitles*]{titlesec}
\usepackage{fancyhdr}
\usepackage{listings}
\usepackage{catchfilebetweentags}
\usepackage[french, capitalise, noabbrev]{cleveref}
\usepackage[fit, breakall]{truncate}
\usepackage[top=2.5cm, right=2cm, bottom=2.5cm, left=2cm]{geometry}
\usepackage{enumitem}
\usepackage{tocloft}
\usepackage{microtype}
%\usepackage{mdframed}
%\usepackage{thmtools}
\usepackage{xcolor}
\usepackage{tabularx}
\usepackage{xltabular}
\usepackage{aligned-overset}
\usepackage[subpreambles=true]{standalone}
\usepackage{environ}
\usepackage[normalem]{ulem}
\usepackage{etoolbox}
\usepackage{setspace}
\usepackage[bibstyle=reading, citestyle=draft]{biblatex}
\usepackage{xpatch}
\usepackage[many, breakable]{tcolorbox}
\usepackage[backgroundcolor=white, bordercolor=white, textsize=scriptsize]{todonotes}
\usepackage{luacode}
\usepackage{float}
\usepackage{needspace}
\everymath{\displaystyle}

% Police :

\setmathfont{Erewhon Math}

% Tikz :

\usetikzlibrary{calc}
\usetikzlibrary{3d}

% Longueurs :

\setlength{\parindent}{0pt}
\setlength{\headheight}{15pt}
\setlength{\fboxsep}{0pt}
\titlespacing*{\chapter}{0pt}{-20pt}{10pt}
\setlength{\marginparwidth}{1.5cm}
\setstretch{1.1}

% Métadonnées :

\author{agreg.skyost.eu}
\date{\today}

% Titres :

\setcounter{secnumdepth}{3}

\renewcommand{\thechapter}{\Roman{chapter}}
\renewcommand{\thesubsection}{\Roman{subsection}}
\renewcommand{\thesubsubsection}{\arabic{subsubsection}}
\renewcommand{\theparagraph}{\alph{paragraph}}

\titleformat{\chapter}{\huge\bfseries}{\thechapter}{20pt}{\huge\bfseries}
\titleformat*{\section}{\LARGE\bfseries}
\titleformat{\subsection}{\Large\bfseries}{\thesubsection \, - \,}{0pt}{\Large\bfseries}
\titleformat{\subsubsection}{\large\bfseries}{\thesubsubsection. \,}{0pt}{\large\bfseries}
\titleformat{\paragraph}{\bfseries}{\theparagraph. \,}{0pt}{\bfseries}

\setcounter{secnumdepth}{4}

% Table des matières :

\renewcommand{\cftsecleader}{\cftdotfill{\cftdotsep}}
\addtolength{\cftsecnumwidth}{10pt}

% Redéfinition des commandes :

\renewcommand*\thesection{\arabic{section}}
\renewcommand{\ker}{\mathrm{Ker}}

% Nouvelles commandes :

\newcommand{\website}{https://github.com/imbodj/SenCoursDeMaths}

\newcommand{\tr}[1]{\mathstrut ^t #1}
\newcommand{\im}{\mathrm{Im}}
\newcommand{\rang}{\operatorname{rang}}
\newcommand{\trace}{\operatorname{trace}}
\newcommand{\id}{\operatorname{id}}
\newcommand{\stab}{\operatorname{Stab}}
\newcommand{\paren}[1]{\left(#1\right)}
\newcommand{\croch}[1]{\left[ #1 \right]}
\newcommand{\Grdcroch}[1]{\Bigl[ #1 \Bigr]}
\newcommand{\grdcroch}[1]{\bigl[ #1 \bigr]}
\newcommand{\abs}[1]{\left\lvert #1 \right\rvert}
\newcommand{\limi}[3]{\lim_{#1\to #2}#3}
\newcommand{\pinf}{+\infty}
\newcommand{\minf}{-\infty}
%%%%%%%%%%%%%% ENSEMBLES %%%%%%%%%%%%%%%%%
\newcommand{\ensemblenombre}[1]{\mathbb{#1}}
\newcommand{\Nn}{\ensemblenombre{N}}
\newcommand{\Zz}{\ensemblenombre{Z}}
\newcommand{\Qq}{\ensemblenombre{Q}}
\newcommand{\Qqp}{\Qq^+}
\newcommand{\Rr}{\ensemblenombre{R}}
\newcommand{\Cc}{\ensemblenombre{C}}
\newcommand{\Nne}{\Nn^*}
\newcommand{\Zze}{\Zz^*}
\newcommand{\Zzn}{\Zz^-}
\newcommand{\Qqe}{\Qq^*}
\newcommand{\Rre}{\Rr^*}
\newcommand{\Rrp}{\Rr_+}
\newcommand{\Rrm}{\Rr_-}
\newcommand{\Rrep}{\Rr_+^*}
\newcommand{\Rrem}{\Rr_-^*}
\newcommand{\Cce}{\Cc^*}
%%%%%%%%%%%%%%  INTERVALLES %%%%%%%%%%%%%%%%%
\newcommand{\intff}[2]{\left[#1\;,\; #2\right]  }
\newcommand{\intof}[2]{\left]#1 \;, \;#2\right]  }
\newcommand{\intfo}[2]{\left[#1 \;,\; #2\right[  }
\newcommand{\intoo}[2]{\left]#1 \;,\; #2\right[  }

\providecommand{\newpar}{\\[\medskipamount]}

\newcommand{\annexessection}{%
  \newpage%
  \subsection*{Annexes}%
}

\providecommand{\lesson}[3]{%
  \title{#3}%
  \hypersetup{pdftitle={#2 : #3}}%
  \setcounter{section}{\numexpr #2 - 1}%
  \section{#3}%
  \fancyhead[R]{\truncate{0.73\textwidth}{#2 : #3}}%
}

\providecommand{\development}[3]{%
  \title{#3}%
  \hypersetup{pdftitle={#3}}%
  \section*{#3}%
  \fancyhead[R]{\truncate{0.73\textwidth}{#3}}%
}

\providecommand{\sheet}[3]{\development{#1}{#2}{#3}}

\providecommand{\ranking}[1]{%
  \title{Terminale #1}%
  \hypersetup{pdftitle={Terminale #1}}%
  \section*{Terminale #1}%
  \fancyhead[R]{\truncate{0.73\textwidth}{Terminale #1}}%
}

\providecommand{\summary}[1]{%
  \textit{#1}%
  \par%
  \medskip%
}

\tikzset{notestyleraw/.append style={inner sep=0pt, rounded corners=0pt, align=center}}

%\newcommand{\booklink}[1]{\website/bibliographie\##1}
\newcounter{reference}
\newcommand{\previousreference}{}
\providecommand{\reference}[2][]{%
  \needspace{20pt}%
  \notblank{#1}{
    \needspace{20pt}%
    \renewcommand{\previousreference}{#1}%
    \stepcounter{reference}%
    \label{reference-\previousreference-\thereference}%
  }{}%
  \todo[noline]{%
    \protect\vspace{20pt}%
    \protect\par%
    \protect\notblank{#1}{\cite{[\previousreference]}\\}{}%
    \protect\hyperref[reference-\previousreference-\thereference]{p. #2}%
  }%
}

\definecolor{devcolor}{HTML}{00695c}
\providecommand{\dev}[1]{%
  \reversemarginpar%
  \todo[noline]{
    \protect\vspace{20pt}%
    \protect\par%
    \bfseries\color{devcolor}\href{\website/developpements/#1}{[DEV]}
  }%
  \normalmarginpar%
}

% En-têtes :

\pagestyle{fancy}
\fancyhead[L]{\truncate{0.23\textwidth}{\thepage}}
\fancyfoot[C]{\scriptsize \href{\website}{\texttt{https://github.com/imbodj/SenCoursDeMaths}}}

% Couleurs :

\definecolor{property}{HTML}{ffeb3b}
\definecolor{proposition}{HTML}{ffc107}
\definecolor{lemma}{HTML}{ff9800}
\definecolor{theorem}{HTML}{f44336}
\definecolor{corollary}{HTML}{e91e63}
\definecolor{definition}{HTML}{673ab7}
\definecolor{notation}{HTML}{9c27b0}
\definecolor{example}{HTML}{00bcd4}
\definecolor{cexample}{HTML}{795548}
\definecolor{application}{HTML}{009688}
\definecolor{remark}{HTML}{3f51b5}
\definecolor{algorithm}{HTML}{607d8b}
%\definecolor{proof}{HTML}{e1f5fe}
\definecolor{exercice}{HTML}{e1f5fe}

% Théorèmes :

\theoremstyle{definition}
\newtheorem{theorem}{Théorème}

\newtheorem{property}[theorem]{Propriété}
\newtheorem{proposition}[theorem]{Proposition}
\newtheorem{lemma}[theorem]{Activité d'introduction}
\newtheorem{corollary}[theorem]{Conséquence}

\newtheorem{definition}[theorem]{Définition}
\newtheorem{notation}[theorem]{Notation}

\newtheorem{example}[theorem]{Exemple}
\newtheorem{cexample}[theorem]{Contre-exemple}
\newtheorem{application}[theorem]{Application}

\newtheorem{algorithm}[theorem]{Algorithme}
\newtheorem{exercice}[theorem]{Exercice}

\theoremstyle{remark}
\newtheorem{remark}[theorem]{Remarque}

\counterwithin*{theorem}{section}

\newcommand{\applystyletotheorem}[1]{
  \tcolorboxenvironment{#1}{
    enhanced,
    breakable,
    colback=#1!8!white,
    %right=0pt,
    %top=8pt,
    %bottom=8pt,
    boxrule=0pt,
    frame hidden,
    sharp corners,
    enhanced,borderline west={4pt}{0pt}{#1},
    %interior hidden,
    sharp corners,
    after=\par,
  }
}

\applystyletotheorem{property}
\applystyletotheorem{proposition}
\applystyletotheorem{lemma}
\applystyletotheorem{theorem}
\applystyletotheorem{corollary}
\applystyletotheorem{definition}
\applystyletotheorem{notation}
\applystyletotheorem{example}
\applystyletotheorem{cexample}
\applystyletotheorem{application}
\applystyletotheorem{remark}
%\applystyletotheorem{proof}
\applystyletotheorem{algorithm}
\applystyletotheorem{exercice}

% Environnements :

\NewEnviron{whitetabularx}[1]{%
  \renewcommand{\arraystretch}{2.5}
  \colorbox{white}{%
    \begin{tabularx}{\textwidth}{#1}%
      \BODY%
    \end{tabularx}%
  }%
}

% Maths :

\DeclareFontEncoding{FMS}{}{}
\DeclareFontSubstitution{FMS}{futm}{m}{n}
\DeclareFontEncoding{FMX}{}{}
\DeclareFontSubstitution{FMX}{futm}{m}{n}
\DeclareSymbolFont{fouriersymbols}{FMS}{futm}{m}{n}
\DeclareSymbolFont{fourierlargesymbols}{FMX}{futm}{m}{n}
\DeclareMathDelimiter{\VERT}{\mathord}{fouriersymbols}{152}{fourierlargesymbols}{147}

% Code :

\definecolor{greencode}{rgb}{0,0.6,0}
\definecolor{graycode}{rgb}{0.5,0.5,0.5}
\definecolor{mauvecode}{rgb}{0.58,0,0.82}
\definecolor{bluecode}{HTML}{1976d2}
\lstset{
  basicstyle=\footnotesize\ttfamily,
  breakatwhitespace=false,
  breaklines=true,
  %captionpos=b,
  commentstyle=\color{greencode},
  deletekeywords={...},
  escapeinside={\%*}{*)},
  extendedchars=true,
  frame=none,
  keepspaces=true,
  keywordstyle=\color{bluecode},
  language=Python,
  otherkeywords={*,...},
  numbers=left,
  numbersep=5pt,
  numberstyle=\tiny\color{graycode},
  rulecolor=\color{black},
  showspaces=false,
  showstringspaces=false,
  showtabs=false,
  stepnumber=2,
  stringstyle=\color{mauvecode},
  tabsize=2,
  %texcl=true,
  xleftmargin=10pt,
  %title=\lstname
}

\newcommand{\codedirectory}{}
\newcommand{\inputalgorithm}[1]{%
  \begin{algorithm}%
    \strut%
    \lstinputlisting{\codedirectory#1}%
  \end{algorithm}%
}



% Bibliographie :

%\addbibresource{\bibliographypath}%
\defbibheading{bibliography}[\bibname]{\section*{#1}}
\renewbibmacro*{entryhead:full}{\printfield{labeltitle}}%
\DeclareFieldFormat{url}{\newline\footnotesize\url{#1}}%

\AtEndDocument{%
  \newpage%
  \pagestyle{empty}%
  \printbibliography%
}


\begin{document}
  %<*content>
  \lesson{algebra}{121}{Nombres premiers. Applications.}

  \subsection{Généralités}

  \subsubsection{Nombres premiers et premiers entre eux}

  \reference[GOU21]{9}

  \begin{definition}
    Soient $a, b \in \mathbb{Z}$. On dit que $a$ \textbf{divise} $b$ (ou que $b$ est un \textbf{multiple} de $a$), et on note $a \mid b$ s'il existe $n \in \mathbb{Z}$ tel que $b = an$. Dans le cas contraire, on note $a \nmid b$.
  \end{definition}

  \begin{theorem}[Division euclidienne dans $\mathbb{Z}$]
    \[ \forall (a, b) \in \mathbb{Z}^2, \, \exists! (q, r) \in \mathbb{Z}^2 \text{ tel que } a=bq+r \text{ et } r \in \llbracket 0, \vert b \vert \rrbracket \]
  \end{theorem}

  \begin{definition}
    Soient $a_1, \dots, a_n \in \mathbb{Z}$. Par principalité de $\mathbb{Z}$, il existe un unique $d \in \mathbb{N}$ tel que
    \[ a_1\mathbb{Z} + \dots + a_n\mathbb{Z} = d\mathbb{Z} \]
    Ainsi défini, $d$ s'appelle le \textbf{pgcd} de $a_1, \dots, a_n$ et on note $d = \operatorname{pgcd}(a_1, \dots, a_n)$.
  \end{definition}

  \begin{remark}
    Dans la définition précédente, $d$ est le plus entier naturel divisant tous les $a_i$.
  \end{remark}

  \begin{definition}
    Soient $a_1, \dots, a_n \in \mathbb{Z}$. Lorsque $\operatorname{pgcd}(a_1, \dots, a_n) = 1$, on dit que $a_1, \dots, a_n$ sont \textbf{premiers entre eux} \textit{dans leur ensemble}. Lorsque $\operatorname{pgcd}(a_i, a_j) = 1$ dès que $i \neq j$, on dit que $a_1, \dots, a_n$ sont \textbf{premiers entre eux} \textit{deux à deux}.
  \end{definition}

  \begin{theorem}[Bézout]
    Soient $a_1, \dots, a_n \in \mathbb{Z}$.
    \[ \operatorname{pgcd}(a_1, \dots, a_n) = 1 \iff \exists u_1, \dots, u_n \in \mathbb{Z} \text{ tels que } \sum_{i=1}^{n} u_i a_i = 1 \]
  \end{theorem}

  \begin{theorem}[Gauss]
    Soient $a, b, c \in \mathbb{Z}$.
    \[ a \mid bc \text{ et } \operatorname{pgcd}(a,b) = 1 \implies a \mid c \]
  \end{theorem}

  \reference[ROM21]{304}

  \begin{definition}
    On dit qu'un entier naturel $p$ est \textbf{premier} s'il est supérieur ou égal à $2$ et si ses seuls diviseurs positifs sont $1$ et $p$.
  \end{definition}

  \begin{example}
    Les nombres de Fermat $F_n = 2^{2^n}+1$ sont premiers pour $n \in \llbracket 0,4 \rrbracket$, mais pas pour $n \in \llbracket 5,32 \rrbracket$.
  \end{example}

  \begin{theorem}[Euclide]
    L'ensemble $\mathcal{P}$ des nombres premiers est infini.
  \end{theorem}

  \begin{theorem}[Fondamental de l'arithmétique]
    Tout entier naturel $n \geq 2$ se décompose de manière unique sous la forme :
    \[ n = \prod_{k=1}^r p_k^{\alpha_k} \]
    où les $p_k$ sont des nombres premiers distincts et où les $\alpha_k$ sont des entiers naturels non nuls.
  \end{theorem}

  \reference[GOU21]{11}

  \begin{proposition}
    \begin{enumerate}[label=(\roman*)]
      \item Si $n = \prod_{i=1}^k p_i^{\alpha_i}$ et $m = \prod_{i=1}^k p_i^{\beta_i}$, alors $\operatorname{pgcd}(n,m) = \prod_{i=1}^k p_i^{\inf(\alpha_i, \beta_i)}$.
      \item Soient $p \in \mathcal{P}$ et $k \in \llbracket 1, p-1 \rrbracket$. Alors $p \mid \binom{p}{k}$.
    \end{enumerate}
  \end{proposition}

  \begin{theorem}[Fermat]
    Soient $p \in \mathcal{P}$ et $a \in \mathbb{Z}$. Alors :
    \begin{enumerate}[label=(\roman*)]
      \item $a^p \equiv a \mod p$.
      \item $p \nmid a \implies a^{p-1} \equiv 1 \mod p$.
    \end{enumerate}
  \end{theorem}

  \subsubsection{Fonctions arithmétiques}

  \reference[GOZ]{3}

  \begin{definition}
    On définit :
    \begin{itemize}
      \item L'\textbf{indicatrice d'Euler} $\varphi$ est la fonction qui à un entier $k$, associe le nombre d'entiers compris entre $1$ et $n$ qui sont premiers avec $k$.
      \item La \textbf{fonction de Möbius}, notée $\mu$, par
      \[
      \mu :
      \begin{array}{ccc}
        \mathbb{Z} &\rightarrow& \mathbb{Z} \\
        n &\mapsto& \begin{cases}
          1 &\text{si } n = 1 \\
          (-1)^k &\text{si } n = p_1 \dots p_k \text{ avec } p_1, \dots, p_k \text{ premiers distincts} \\
          0 &\text{sinon}
        \end{cases}
      \end{array}
      \]
    \end{itemize}
  \end{definition}

  \begin{proposition}
    \begin{enumerate}[label=(\roman*)]
      \item $\forall m, n \in \mathbb{Z}$ premiers entre eux, $\varphi(mn) = \varphi(m)\varphi(n)$.
      \item Pour tout entier relatif $a$ premier avec $n$, $a^{\varphi(n)} \equiv 1 \mod n$.
      \item Pour tout entier naturel $n$, $\sum_{d \mid n} \varphi(d) = n$.
    \end{enumerate}
  \end{proposition}

  \reference{89}

  \begin{theorem}[Formule d'inversion de Möbius]
    Soient $f$ et $g$ des fonctions de $\mathbb{N}^*$ dans $\mathbb{C}$ telles que $\forall n \in \mathbb{N}^*, \, f(n) = \sum_{d \mid n} g(d)$. Alors,
    \[ \forall n \in \mathbb{N}^*, \, g(n) = \sum_{d \mid n} \mu(d) f \left( \frac{n}{d} \right) \]
  \end{theorem}

  \begin{corollary}
    \[ \forall n \in \mathbb{N}^*, \, \varphi(n) = \sum_{d \mid n} d\mu \left( \frac{n}{d} \right) \]
  \end{corollary}

  \subsubsection{Répartition des nombres premiers}

  \reference{67}

  \begin{definition}
    L'ensemble des générateurs de $\mu_n$, noté $\mu_n^*$, est formé des \textbf{racines primitives $n$-ièmes de l'unité}.
  \end{definition}

  \begin{proposition}
    \begin{enumerate}[label=(\roman*)]
      \item $\mu_n^* = \{ e^{\frac{2ik\pi}{n}} \mid k \in \llbracket 0, n-1 \rrbracket, \, \operatorname{pgcd}(k, m) = 1 \}$.
      \item $\vert \mu_n^* \vert = \varphi(n)$, où $\varphi$ désigne l'indicatrice d'Euler.
    \end{enumerate}
  \end{proposition}

  \begin{definition}
    On appelle \textbf{$n$-ième polynôme cyclotomique} le polynôme
    \[ \Phi_n = \prod_{\xi \in \mu_n^*} (X - \xi) \]
  \end{definition}

  \begin{theorem}
    \begin{enumerate}[label=(\roman*)]
      \item $X^n - 1 = \prod_{d \mid n} \Phi_d$.
      \item $\Phi_n \in \mathbb{Z}[X]$.
      \item $\Phi_n$ est irréductible sur $\mathbb{Q}$.
    \end{enumerate}
  \end{theorem}

  \begin{corollary}
    Le polynôme minimal sur $\mathbb{Q}$ de tout élément $\xi$ de $\mu_n^*$ est $\Phi_n$. En particulier,
    \[ [\mathbb{Q}(\xi):\mathbb{Q}]=\varphi(m) \]
  \end{corollary}

  \reference[GOU21]{99}
  \dev{theoreme-de-dirichlet-faible}

  \begin{theorem}[Dirichlet faible]
    Pour tout entier $n$, il existe une infinité de nombres premiers congrus à $1$ modulo $n$.
  \end{theorem}

  \begin{remark}
    La version forte de ce théorème est que, pour tout entiers naturels $a$, $b$ non nuls, il existe une infinité de nombres premiers de la forme $ak+b$, $k \in \mathbb{N}$.
  \end{remark}

  \reference{16}

  \begin{theorem}[des nombres premiers]
    Si $x > 0$, on note $\pi(x)$ le nombre de nombres premiers inférieurs à $x$. Alors,
    \[ \pi(x) \sim \frac{x}{\ln(x)} \]
  \end{theorem}

  \subsection{Théorie des corps}

  \subsubsection{Corps finis}

  \reference[GOZ]{3}

  \begin{proposition}
    Les conditions suivantes sont équivalentes :
    \begin{enumerate}[label=(\roman*)]
      \item $n$ est un nombre premier.
      \item $\mathbb{Z}/n\mathbb{Z}$ est un anneau intègre.
      \item $\mathbb{Z}/n\mathbb{Z}$ est un corps.
    \end{enumerate}
  \end{proposition}

  \begin{notation}
    On note $\mathbb{F}_p = \mathbb{Z}/p\mathbb{Z}$.
  \end{notation}

  \reference{7}

  \begin{definition}
    Soit $A$ un anneau. L'application
    \[
      f_A :
      \begin{array}{ccc}
        \mathbb{Z} &\rightarrow& A \\
        n &\mapsto& \underbrace{1 + \dots + 1}_{n \text{ fois}}
      \end{array}
    \]
    On note $\operatorname{car}(A)$ l'unique $n \in \mathbb{N}$ tel que $\operatorname{Ker}(f_A) = n\mathbb{Z}$ : c'est la \textbf{caractéristique} de $A$.
  \end{definition}

  \begin{proposition}
    \begin{enumerate}[label=(\roman*)]
      \item \label{121-1} Soit $A$ un anneau intègre. Alors, $\operatorname{car}(A) = 0 \text{ ou } p$ avec $p$ premier.
      \item Soit $A$ un anneau fini. Alors, $\operatorname{car}(A) \neq 0$ et $\operatorname{car}(A) \mid |A|$.
      \item Un anneau et un quelconque de ses sous-anneaux ont la même caractéristique.
    \end{enumerate}
  \end{proposition}

  \begin{remark}
    \begin{itemize}
      \item Le \cref{121-1} est en particulier vrai pour un corps.
      \item Si $\operatorname{car}(A) = 0$, $A$ est infini.
    \end{itemize}
  \end{remark}

  \reference{81}

  \begin{proposition}
    Soit $\mathbb{K}$ un corps fini.
    \begin{enumerate}[label=(\roman*)]
      \item $\operatorname{car}(\mathbb{K})$ est un nombre premier $p$.
      \item Le sous-corps premier de $\mathbb{K}$ est isomorphe à $\mathbb{F}_p$.
      \item $\vert \mathbb{K} \vert = p^n$ pour $n \geq 2$.
    \end{enumerate}
  \end{proposition}

  \begin{proposition}
    Soit $\mathbb{K}$ un corps de caractéristique $p$. L'application
    \[
      \operatorname{Frob} :
      \begin{array}{ccc}
        \mathbb{K} &\rightarrow& \mathbb{K} \\
        x &\mapsto& x^p
      \end{array}
    \]
    est un morphisme de corps.
    \begin{enumerate}[label=(\roman*)]
      \item Si $\mathbb{K}$ est fini, c'est un automorphisme.
      \item Si $\mathbb{K} = \mathbb{F}_p$, c'est l'identité.
    \end{enumerate}
  \end{proposition}

  \begin{theorem}
    Soient $p \in \mathcal{P}$ et $n \in \mathbb{N}^*$. On pose $q = p^n$. Alors :
    \begin{enumerate}[label=(\roman*)]
      \item Il existe un corps $\mathbb{K}$ à $q$ éléments : c'est le corps de décomposition de $X^q - X$ sur $\mathbb{F}_p$.
      \item $\mathbb{K}$ est unique à isomorphisme près : on le note $\mathbb{F}_q$.
    \end{enumerate}
  \end{theorem}

  \begin{corollary}[Théorème de Wilson]
    Soit $n \geq 2$ un entier. Alors,
    \[ n \text{ est premier} \iff (n-1)!+1 \equiv 0 \mod n \]
  \end{corollary}

  \reference[PER]{74}

  \begin{theorem}
    $\mathbb{F}_q^*$ est cyclique, isomorphe à $\mathbb{Z}/(q-1)\mathbb{Z}$.
  \end{theorem}

  \begin{remark}
    En fait, tout sous-groupe fini du groupe multiplicatif d'un corps commutatif est cyclique.
  \end{remark}

  \reference[GOU21]{100}

  \begin{theorem}[Wedderburn]
    Tout corps fini est commutatif.
  \end{theorem}

  \subsubsection{Carrés dans les corps finis}

  \reference[GOZ]{93}

  Soit $q = p^n$ avec $p$ premier et $n \geq 2$.

  \begin{proposition}
    On note $\mathbb{F}_q^2 = \{ x^2 \mid x \in \mathbb{F}_q \}$ et $\mathbb{F}_q^{*2} = \mathbb{F}_q^2 \, \cap \, \mathbb{F}_q^*$. Alors $\mathbb{F}_q^{*2}$ est un sous-groupe de $\mathbb{F}_q^*$.
  \end{proposition}

  \begin{proposition}
    \begin{enumerate}[label=(\roman*)]
      \item Si $p = 2$, $\mathbb{F}_q^2 = \mathbb{F}_q$, donc $\mathbb{F}_q^{*2} = \mathbb{F}_q^*$.
      \item Si $p > 2$, alors :
      \begin{itemize}
        \item $\mathbb{F}_q^{*2}$ est le noyau de l'endomorphisme de $\mathbb{F}_q^*$ défini par $x \mapsto x^{\frac{q-1}{2}}$.
        \item $\mathbb{F}_q^{*2}$ est un sous-groupe d'indice $2$ de $\mathbb{F}_q^*$.
        \item $\vert \mathbb{F}_q^{*2} \vert = \frac{q-1}{2}$ et $\vert \mathbb{F}_q^2 \vert = \frac{q+1}{2}$.
        \item $(-1) \in \mathbb{F}_q^{*2} \iff q \equiv 1 \mod 4$.
      \end{itemize}
    \end{enumerate}
  \end{proposition}

  \reference[I-P]{203}

  \begin{notation}
    Soit $a \in \mathbb{F}_p$. On note $\left( \frac{a}{p} \right)$ le symbole de Legendre de $a$ modulo $p$. On a ainsi $\left( \frac{a}{p} \right) = \pm 1$ avec $\left( \frac{a}{p} \right) = 1$ si et seulement si $a \in \mathbb{F}_p^2$.
  \end{notation}

  \begin{application}[Frobenius-Zolotarev]
    Soient $p \geq 3$ un nombre premier et $V$ un espace vectoriel sur $\mathbb{F}_p$ de dimension finie.
    \[ \forall u \in \mathrm{GL}(V), \, \epsilon(u) = \left( \frac{\det(u)}{p} \right) \]
    où $u$ est vu comme une permutation des éléments de $V$.
  \end{application}

  \subsubsection{Réduction modulo \texorpdfstring{$p$}{p}}

  Le résultat suivant justifie que l'on s'intéresse aux polynômes irréductibles en théorie des corps.

  \reference[GOZ]{57}

  \begin{theorem}
    Soit $P \in \mathbb{K}[X]$ un polynôme irréductible sur un corps $\mathbb{K}$.
    \begin{itemize}
      \item Il existe un corps de rupture de $P$.
      \item Si $\mathbb{L} = \mathbb{K}[\alpha]$ et $\mathbb{L}' = \mathbb{K}[\beta]$ sont deux corps de rupture de $P$, alors il existe un unique $\mathbb{K}$-isomorphisme $\varphi : \mathbb{L} \rightarrow \mathbb{L}'$ tel que $\varphi(\alpha) = \beta$.
      \item $\mathbb{K}[X]/(P)$ est un corps de rupture de $P$.
    \end{itemize}
  \end{theorem}

  \reference{10}

  \begin{lemma}[Gauss]
    \begin{enumerate}[label=(\roman*)]
      \item Le produit de deux polynômes primitifs est primitif (ie. dont le PGCD des coefficients est égal à $1$).
      \item $\forall P, Q \in \mathbb{Z}[X] \setminus \{ 0 \}$, $\gamma(PQ) = \gamma(P) \gamma(Q)$ (où $\gamma(P)$ est le contenu du polynôme $P$).
    \end{enumerate}
  \end{lemma}

  \dev{critere-d-eisenstein}

  \begin{theorem}[Critère d'Eisenstein]
    Soit $P = \sum_{i=0}^n a_i X^i \in \mathbb{Z}[X]$ de degré $n \geq 1$. On suppose qu'il existe $p$ premier tel que :
    \begin{enumerate}[label=(\roman*)]
      \item $p \mid a_i$, $\forall i \in \llbracket 0, n-1 \rrbracket$.
      \item $p \nmid a_n$.
      \item $p^2 \nmid a_0$.
    \end{enumerate}
    Alors $P$ est irréductible dans $\mathbb{Q}[X]$.
  \end{theorem}

  \reference[PER]{67}

  \begin{application}
    Soit $n \in \mathbb{N}^*$. Il existe des polynômes irréductibles de degré $n$ sur $\mathbb{Z}$.
  \end{application}

  \reference[GOZ]{12}

  \begin{theorem}[Critère d'irréductibilité modulo $p$]
    Soit $P = \sum_{i=0}^n a_i X^i \in \mathbb{Z}[X]$ de degré $n \geq 1$. Soit $p$ un premier. On suppose $p \nmid a_n$.
    \newpar
    Si $\overline{P}$ est irréductible dans $(\mathbb{Z}/p\mathbb{Z})[X]$, alors $P$ est irréductible dans $\mathbb{Q}[X]$.
  \end{theorem}

  \begin{example}
    Le polynôme $X^3-127X^2+3608X+19$ est irréductible dans $\mathbb{Z}[X]$.
  \end{example}

  \subsection{Autres applications en algèbre}

  \subsubsection{Entiers sommes de deux carrés}

  \reference[I-P]{137}

  \begin{notation}
    On note \[ N :
    \begin{array}{ccc}
      \mathbb{Z}[i] &\rightarrow& \mathbb{N} \\
      a+ib &\mapsto& a^2 + b^2
    \end{array}
    \] et $\Sigma$ l'ensemble des entiers qui sont somme de deux carrés.
  \end{notation}

  \begin{remark}
    $n \in \Sigma \iff \exists z \in \mathbb{Z}[i] \text{ tel que } N(z)=n$.
  \end{remark}

  \begin{theorem}[Deux carrés de Fermat]
    Soit $n \in \mathbb{N}^*$. Alors $n \in \Sigma$ si et seulement si $v_p(n)$ est pair pour tout $p$ premier tel que $p \equiv 3 \mod 4$ (où $v_p(n)$ désigne la valuation $p$-adique de $n$).
  \end{theorem}

  \subsubsection{En théorie des groupes}

  \reference[ROM21]{22}

  Soit $G$ un groupe fini opérant sur un ensemble fini $X$.

  \begin{definition}
    On dit que $G$ est un \textbf{$p$-groupe} s'il est d'ordre une puissance d'un nombre premier $p$.
  \end{definition}

  \begin{theorem}[Formule des classes]
    Soit $\Omega$ un système de représentants des orbites de l'action de $G$ sur $X$. Alors,
    \[ |X| = \sum_{\omega \in \Omega} |G \cdot \omega| = \sum_{\omega \in \Omega} (G : \stab_G(\omega)) = \sum_{\omega \in \Omega} \frac{|G|}{|\stab_G(\omega)|} \]
  \end{theorem}

  \begin{corollary}
    Soit $p$ un nombre premier. Si $G$ est un $p$-groupe opérant sur $X$, alors,
    \[ |X^G| \equiv |X| \mod p \]
    où $X^G$ désigne l'ensemble des points fixes de $X$ sous l'action de $G$.
  \end{corollary}

  \begin{corollary}
    On note $G \cdot h_1, \dots, G \cdot h_r$ les classes de conjugaison de $G$. Alors,
    \begin{align*}
      \vert G \vert &= \vert Z(G) \vert + \sum_{\substack{i=1 \\ \vert G \cdot h_i \vert = 2}}^{r} \vert G \cdot h_i \vert \\
      &= \vert Z(G) \vert + \sum_{\substack{i=1 \\ \vert G \cdot h_i \vert = 2}}^{r} \frac{\vert G \vert}{\vert \stab_G(h_i) \vert}
    \end{align*}
  \end{corollary}

  \begin{corollary}
    Soit $p$ un nombre premier. Le centre d'un $p$-groupe non trivial est non trivial.
  \end{corollary}

  \begin{corollary}
    Soit $p$ un nombre premier. Un groupe d'ordre $p^2$ est toujours abélien.
  \end{corollary}

  \begin{application}[Théorème de Cauchy]
    On suppose $G$ non trivial et fini. Soit $p$ un premier divisant l'ordre de $G$. Alors il existe un élément d'ordre $p$ dans $G$.
  \end{application}

  \reference[GOU21]{44}

  \begin{application}[Premier théorème de Sylow]
    On suppose $G$ fini d'ordre $n p^\alpha$ avec $n, \alpha \in \mathbb{N}$ et $p$ premier tel que $p \nmid n$. Alors, il existe un sous-groupe de $G$ d’ordre $p^\alpha$.
  \end{application}

  \subsubsection{RSA}

  \reference[ULM18]{62}

  \begin{definition}
    Afin de chiffrer un \textbf{message} (tout entier découpé en séquence d'entiers de taille bornée) en utilisant RSA, on doit a besoin de deux clés :
    \begin{itemize}
      \item Une \textbf{clé privée}, qui est un couple de nombres premiers $(p,q)$.
      \item La \textbf{clé publique} correspondante, qui est le couple $(n,e)$ où $n = pq$ et $e$ est l'inverse de $d$ modulo $\phi(n)$ où $d$ désigne un nombre premier à $\phi(n)$.
    \end{itemize}
  \end{definition}

  Nous conserverons ces notations pour la suite.

  \begin{theorem}[Chiffrement RSA]
    Soit $m = (m_i)_{i \in \llbracket 1, r \rrbracket}$ un message où pour tout $i$, $m_i < n$.
    \begin{enumerate}[label=(\roman*)]
      \item Possédant la clé publique, on peut \textit{chiffrer} ce message en un message $m'$ :
      \[ m' = (m_i^e)_{i \in \llbracket 1, r \rrbracket} \]
      \item Possédant la clé privée, on peut \textit{déchiffrer} le message $m'$ pour reconstituer $m$ :
      \[ \forall i \in \llbracket 1, r \rrbracket, \, (m_i^e)^d \equiv d \mod n \]
    \end{enumerate}
  \end{theorem}

  \begin{remark}
    \begin{itemize}
      \item L'intérêt vient pour des premiers $p$ et $q$ très grands : il devient alors très compliqué de factoriser $n$ et d'obtenir la clé privée.
      \item Les inverses peuvent se calculer à l'aide de l'algorithme de Bézout.
    \end{itemize}
  \end{remark}
  %</content>
\end{document}
