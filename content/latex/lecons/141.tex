\documentclass[12pt, a4paper]{report}

% LuaLaTeX :

\RequirePackage{iftex}
\RequireLuaTeX

% Packages :

\usepackage[french]{babel}
%\usepackage[utf8]{inputenc}
%\usepackage[T1]{fontenc}
\usepackage[pdfencoding=auto, pdfauthor={Hugo Delaunay}, pdfsubject={Mathématiques}, pdfcreator={agreg.skyost.eu}]{hyperref}
\usepackage{amsmath}
\usepackage{amsthm}
%\usepackage{amssymb}
\usepackage{stmaryrd}
\usepackage{tikz}
\usepackage{tkz-euclide}
\usepackage{fontspec}
\defaultfontfeatures[Erewhon]{FontFace = {bx}{n}{Erewhon-Bold.otf}}
\usepackage{fourier-otf}
\usepackage[nobottomtitles*]{titlesec}
\usepackage{fancyhdr}
\usepackage{listings}
\usepackage{catchfilebetweentags}
\usepackage[french, capitalise, noabbrev]{cleveref}
\usepackage[fit, breakall]{truncate}
\usepackage[top=2.5cm, right=2cm, bottom=2.5cm, left=2cm]{geometry}
\usepackage{enumitem}
\usepackage{tocloft}
\usepackage{microtype}
%\usepackage{mdframed}
%\usepackage{thmtools}
\usepackage{xcolor}
\usepackage{tabularx}
\usepackage{xltabular}
\usepackage{aligned-overset}
\usepackage[subpreambles=true]{standalone}
\usepackage{environ}
\usepackage[normalem]{ulem}
\usepackage{etoolbox}
\usepackage{setspace}
\usepackage[bibstyle=reading, citestyle=draft]{biblatex}
\usepackage{xpatch}
\usepackage[many, breakable]{tcolorbox}
\usepackage[backgroundcolor=white, bordercolor=white, textsize=scriptsize]{todonotes}
\usepackage{luacode}
\usepackage{float}
\usepackage{needspace}
\everymath{\displaystyle}

% Police :

\setmathfont{Erewhon Math}

% Tikz :

\usetikzlibrary{calc}
\usetikzlibrary{3d}

% Longueurs :

\setlength{\parindent}{0pt}
\setlength{\headheight}{15pt}
\setlength{\fboxsep}{0pt}
\titlespacing*{\chapter}{0pt}{-20pt}{10pt}
\setlength{\marginparwidth}{1.5cm}
\setstretch{1.1}

% Métadonnées :

\author{agreg.skyost.eu}
\date{\today}

% Titres :

\setcounter{secnumdepth}{3}

\renewcommand{\thechapter}{\Roman{chapter}}
\renewcommand{\thesubsection}{\Roman{subsection}}
\renewcommand{\thesubsubsection}{\arabic{subsubsection}}
\renewcommand{\theparagraph}{\alph{paragraph}}

\titleformat{\chapter}{\huge\bfseries}{\thechapter}{20pt}{\huge\bfseries}
\titleformat*{\section}{\LARGE\bfseries}
\titleformat{\subsection}{\Large\bfseries}{\thesubsection \, - \,}{0pt}{\Large\bfseries}
\titleformat{\subsubsection}{\large\bfseries}{\thesubsubsection. \,}{0pt}{\large\bfseries}
\titleformat{\paragraph}{\bfseries}{\theparagraph. \,}{0pt}{\bfseries}

\setcounter{secnumdepth}{4}

% Table des matières :

\renewcommand{\cftsecleader}{\cftdotfill{\cftdotsep}}
\addtolength{\cftsecnumwidth}{10pt}

% Redéfinition des commandes :

\renewcommand*\thesection{\arabic{section}}
\renewcommand{\ker}{\mathrm{Ker}}

% Nouvelles commandes :

\newcommand{\website}{https://github.com/imbodj/SenCoursDeMaths}

\newcommand{\tr}[1]{\mathstrut ^t #1}
\newcommand{\im}{\mathrm{Im}}
\newcommand{\rang}{\operatorname{rang}}
\newcommand{\trace}{\operatorname{trace}}
\newcommand{\id}{\operatorname{id}}
\newcommand{\stab}{\operatorname{Stab}}
\newcommand{\paren}[1]{\left(#1\right)}
\newcommand{\croch}[1]{\left[ #1 \right]}
\newcommand{\Grdcroch}[1]{\Bigl[ #1 \Bigr]}
\newcommand{\grdcroch}[1]{\bigl[ #1 \bigr]}
\newcommand{\abs}[1]{\left\lvert #1 \right\rvert}
\newcommand{\limi}[3]{\lim_{#1\to #2}#3}
\newcommand{\pinf}{+\infty}
\newcommand{\minf}{-\infty}
%%%%%%%%%%%%%% ENSEMBLES %%%%%%%%%%%%%%%%%
\newcommand{\ensemblenombre}[1]{\mathbb{#1}}
\newcommand{\Nn}{\ensemblenombre{N}}
\newcommand{\Zz}{\ensemblenombre{Z}}
\newcommand{\Qq}{\ensemblenombre{Q}}
\newcommand{\Qqp}{\Qq^+}
\newcommand{\Rr}{\ensemblenombre{R}}
\newcommand{\Cc}{\ensemblenombre{C}}
\newcommand{\Nne}{\Nn^*}
\newcommand{\Zze}{\Zz^*}
\newcommand{\Zzn}{\Zz^-}
\newcommand{\Qqe}{\Qq^*}
\newcommand{\Rre}{\Rr^*}
\newcommand{\Rrp}{\Rr_+}
\newcommand{\Rrm}{\Rr_-}
\newcommand{\Rrep}{\Rr_+^*}
\newcommand{\Rrem}{\Rr_-^*}
\newcommand{\Cce}{\Cc^*}
%%%%%%%%%%%%%%  INTERVALLES %%%%%%%%%%%%%%%%%
\newcommand{\intff}[2]{\left[#1\;,\; #2\right]  }
\newcommand{\intof}[2]{\left]#1 \;, \;#2\right]  }
\newcommand{\intfo}[2]{\left[#1 \;,\; #2\right[  }
\newcommand{\intoo}[2]{\left]#1 \;,\; #2\right[  }

\providecommand{\newpar}{\\[\medskipamount]}

\newcommand{\annexessection}{%
  \newpage%
  \subsection*{Annexes}%
}

\providecommand{\lesson}[3]{%
  \title{#3}%
  \hypersetup{pdftitle={#2 : #3}}%
  \setcounter{section}{\numexpr #2 - 1}%
  \section{#3}%
  \fancyhead[R]{\truncate{0.73\textwidth}{#2 : #3}}%
}

\providecommand{\development}[3]{%
  \title{#3}%
  \hypersetup{pdftitle={#3}}%
  \section*{#3}%
  \fancyhead[R]{\truncate{0.73\textwidth}{#3}}%
}

\providecommand{\sheet}[3]{\development{#1}{#2}{#3}}

\providecommand{\ranking}[1]{%
  \title{Terminale #1}%
  \hypersetup{pdftitle={Terminale #1}}%
  \section*{Terminale #1}%
  \fancyhead[R]{\truncate{0.73\textwidth}{Terminale #1}}%
}

\providecommand{\summary}[1]{%
  \textit{#1}%
  \par%
  \medskip%
}

\tikzset{notestyleraw/.append style={inner sep=0pt, rounded corners=0pt, align=center}}

%\newcommand{\booklink}[1]{\website/bibliographie\##1}
\newcounter{reference}
\newcommand{\previousreference}{}
\providecommand{\reference}[2][]{%
  \needspace{20pt}%
  \notblank{#1}{
    \needspace{20pt}%
    \renewcommand{\previousreference}{#1}%
    \stepcounter{reference}%
    \label{reference-\previousreference-\thereference}%
  }{}%
  \todo[noline]{%
    \protect\vspace{20pt}%
    \protect\par%
    \protect\notblank{#1}{\cite{[\previousreference]}\\}{}%
    \protect\hyperref[reference-\previousreference-\thereference]{p. #2}%
  }%
}

\definecolor{devcolor}{HTML}{00695c}
\providecommand{\dev}[1]{%
  \reversemarginpar%
  \todo[noline]{
    \protect\vspace{20pt}%
    \protect\par%
    \bfseries\color{devcolor}\href{\website/developpements/#1}{[DEV]}
  }%
  \normalmarginpar%
}

% En-têtes :

\pagestyle{fancy}
\fancyhead[L]{\truncate{0.23\textwidth}{\thepage}}
\fancyfoot[C]{\scriptsize \href{\website}{\texttt{https://github.com/imbodj/SenCoursDeMaths}}}

% Couleurs :

\definecolor{property}{HTML}{ffeb3b}
\definecolor{proposition}{HTML}{ffc107}
\definecolor{lemma}{HTML}{ff9800}
\definecolor{theorem}{HTML}{f44336}
\definecolor{corollary}{HTML}{e91e63}
\definecolor{definition}{HTML}{673ab7}
\definecolor{notation}{HTML}{9c27b0}
\definecolor{example}{HTML}{00bcd4}
\definecolor{cexample}{HTML}{795548}
\definecolor{application}{HTML}{009688}
\definecolor{remark}{HTML}{3f51b5}
\definecolor{algorithm}{HTML}{607d8b}
%\definecolor{proof}{HTML}{e1f5fe}
\definecolor{exercice}{HTML}{e1f5fe}

% Théorèmes :

\theoremstyle{definition}
\newtheorem{theorem}{Théorème}

\newtheorem{property}[theorem]{Propriété}
\newtheorem{proposition}[theorem]{Proposition}
\newtheorem{lemma}[theorem]{Activité d'introduction}
\newtheorem{corollary}[theorem]{Conséquence}

\newtheorem{definition}[theorem]{Définition}
\newtheorem{notation}[theorem]{Notation}

\newtheorem{example}[theorem]{Exemple}
\newtheorem{cexample}[theorem]{Contre-exemple}
\newtheorem{application}[theorem]{Application}

\newtheorem{algorithm}[theorem]{Algorithme}
\newtheorem{exercice}[theorem]{Exercice}

\theoremstyle{remark}
\newtheorem{remark}[theorem]{Remarque}

\counterwithin*{theorem}{section}

\newcommand{\applystyletotheorem}[1]{
  \tcolorboxenvironment{#1}{
    enhanced,
    breakable,
    colback=#1!8!white,
    %right=0pt,
    %top=8pt,
    %bottom=8pt,
    boxrule=0pt,
    frame hidden,
    sharp corners,
    enhanced,borderline west={4pt}{0pt}{#1},
    %interior hidden,
    sharp corners,
    after=\par,
  }
}

\applystyletotheorem{property}
\applystyletotheorem{proposition}
\applystyletotheorem{lemma}
\applystyletotheorem{theorem}
\applystyletotheorem{corollary}
\applystyletotheorem{definition}
\applystyletotheorem{notation}
\applystyletotheorem{example}
\applystyletotheorem{cexample}
\applystyletotheorem{application}
\applystyletotheorem{remark}
%\applystyletotheorem{proof}
\applystyletotheorem{algorithm}
\applystyletotheorem{exercice}

% Environnements :

\NewEnviron{whitetabularx}[1]{%
  \renewcommand{\arraystretch}{2.5}
  \colorbox{white}{%
    \begin{tabularx}{\textwidth}{#1}%
      \BODY%
    \end{tabularx}%
  }%
}

% Maths :

\DeclareFontEncoding{FMS}{}{}
\DeclareFontSubstitution{FMS}{futm}{m}{n}
\DeclareFontEncoding{FMX}{}{}
\DeclareFontSubstitution{FMX}{futm}{m}{n}
\DeclareSymbolFont{fouriersymbols}{FMS}{futm}{m}{n}
\DeclareSymbolFont{fourierlargesymbols}{FMX}{futm}{m}{n}
\DeclareMathDelimiter{\VERT}{\mathord}{fouriersymbols}{152}{fourierlargesymbols}{147}

% Code :

\definecolor{greencode}{rgb}{0,0.6,0}
\definecolor{graycode}{rgb}{0.5,0.5,0.5}
\definecolor{mauvecode}{rgb}{0.58,0,0.82}
\definecolor{bluecode}{HTML}{1976d2}
\lstset{
  basicstyle=\footnotesize\ttfamily,
  breakatwhitespace=false,
  breaklines=true,
  %captionpos=b,
  commentstyle=\color{greencode},
  deletekeywords={...},
  escapeinside={\%*}{*)},
  extendedchars=true,
  frame=none,
  keepspaces=true,
  keywordstyle=\color{bluecode},
  language=Python,
  otherkeywords={*,...},
  numbers=left,
  numbersep=5pt,
  numberstyle=\tiny\color{graycode},
  rulecolor=\color{black},
  showspaces=false,
  showstringspaces=false,
  showtabs=false,
  stepnumber=2,
  stringstyle=\color{mauvecode},
  tabsize=2,
  %texcl=true,
  xleftmargin=10pt,
  %title=\lstname
}

\newcommand{\codedirectory}{}
\newcommand{\inputalgorithm}[1]{%
  \begin{algorithm}%
    \strut%
    \lstinputlisting{\codedirectory#1}%
  \end{algorithm}%
}



% Bibliographie :

%\addbibresource{\bibliographypath}%
\defbibheading{bibliography}[\bibname]{\section*{#1}}
\renewbibmacro*{entryhead:full}{\printfield{labeltitle}}%
\DeclareFieldFormat{url}{\newline\footnotesize\url{#1}}%

\AtEndDocument{%
  \newpage%
  \pagestyle{empty}%
  \printbibliography%
}


\begin{document}
  %<*content>
  \lesson{algebra}{141}{Polynômes irréductibles à une indéterminée. Corps de rupture. Exemples et applications.}

  Sauf mention contraire, les corps sont supposés commutatifs.

  \subsection{Irréductibilité de polynômes}

  \subsubsection{Racines et polynômes irréductibles}

  \reference[GOZ]{8}

  \begin{definition}
    Soit $A$ un anneau. Un polynôme $P$ de $A[X]$ est dit \textbf{irréductible} si $\deg(A) \geq 1$ et ses seuls diviseurs dans $A[X]$ sont les polynômes $uP$ où $u \in A^{\times}$.
  \end{definition}

  \begin{remark}
    Soit $\mathbb{K}$ un corps. Alors, $\mathbb{K}[X]$ est euclidien, donc principal, donc factoriel.
  \end{remark}

  \begin{definition}
    Soient $\mathbb{L}$ un corps et $\mathbb{K}$ un sous-corps de $\mathbb{L}$. Soit $P \in \mathbb{K}[X]$.
    \begin{itemize}
      \item Une \textbf{racine} est un élément $\alpha \in \mathbb{K}$ tel que $P(\alpha) = 0$.
      \item La \textbf{multiplicité} de $\alpha$ comme racine de $P$ est le plus grand $n \in \mathbb{N}$ tel que $(X-a)^n$ divise $P$ dans $\mathbb{K}[X]$.
      \item La somme des multiplicités des racines de $P$ dans $\mathbb{K}$ est inférieure ou égale à $\deg(P)$. En cas d'égalité, on dit que $P$ est \textbf{scindé} \textit{sur} $\mathbb{K}$ (ou \textit{dans} $\mathbb{K}[X]$).
    \end{itemize}
  \end{definition}

  \begin{proposition}
    \label{141-1}
    \begin{enumerate}[label=(\roman*)]
      \item Tout polynôme de degré $1$ est irréductible.
      \item \label{141-2} Tout polynôme irréductible de degré strictement supérieur à $1$ n'a pas de racine dans $\mathbb{K}$.
    \end{enumerate}
  \end{proposition}

  \begin{cexample}
    $(X^2+1)^2$ n'a pas de racine dans $\mathbb{Q}$, mais est réductible dans $\mathbb{Q}[X]$.
  \end{cexample}

  \begin{proposition}
    La réciproque de la \cref{141-1} \cref{141-2} est vraie pour les polynômes de degré $2$ ou $3$.
  \end{proposition}

  \begin{proposition}
    Soit $P = \sum_{k=0}^n a_k X^k \in \mathbb{Z}[X]$ un polynôme de degré $n$ tel que $a_0 \neq 0$. Si $\alpha = \frac{p}{q} \in \mathbb{Q}$ est une racine de $P$, en supposant $\frac{p}{q}$ irréductible, alors $p \mid a_0$ et $q \mid a_n$.
  \end{proposition}

  \reference{19}

  \begin{example}
    $X^3 + X + 1$ n'a pas de racine dans $\mathbb{Q}$.
  \end{example}

  \subsubsection{Quelques critères d'irréductibilité}

  Soit $A$ un anneau factoriel.

  \reference{10}

  \begin{definition}
    Pour tout polynôme non nul $P \in A[X]$, on appelle \textbf{contenu} de $P$, noté $\gamma(P)$, le PGCD des coefficients de $P$. $P$ est dit \textbf{primitif} si $\gamma(P) = 1$.
  \end{definition}

  \begin{lemma}[Gauss]
    \begin{enumerate}[label=(\roman*)]
      \item Le produit de deux polynômes primitifs est primitif.
      \item $\forall P, Q \in A[X] \setminus \{ 0 \}$, $\gamma(PQ) = \gamma(P) \gamma(Q)$.
    \end{enumerate}
  \end{lemma}

  \begin{theorem}
    Soient $\mathbb{K}$ le corps des fractions de $A$ et $P \in A[X]$ de degré supérieur ou égal à $1$. Alors, $P$ est irréductible dans $A[X]$ si et seulement si $P$ est irréductible dans $\mathbb{K}[X]$ et $\gamma(P)=1$.
  \end{theorem}

  \dev{critere-d-eisenstein}

  \begin{theorem}[Critère d'Eisenstein]
    Soient $\mathbb{K}$ le corps des fractions de $A$ et $P = \sum_{i=0}^n a_i X^i \in A[X]$ de degré $n \geq 1$. On suppose qu'il existe $p \in A$ irréductible tel que :
    \begin{enumerate}[label=(\roman*)]
      \item $p \mid a_i$, $\forall i \in \llbracket 0, n-1 \rrbracket$.
      \item $p \nmid a_n$.
      \item $p^2 \nmid a_0$.
    \end{enumerate}
    Alors $P$ est irréductible dans $\mathbb{K}[X]$.
  \end{theorem}

  \begin{example}
    Soit $p$ un nombre premier. Le polynôme $\Phi_p = \sum_{k=0}^{p-1} X^k$ est irréductible dans $\mathbb{Z}[X]$.
  \end{example}

  \reference[PER]{67}

  \begin{application}
    \label{141-3}
    Soit $n \in \mathbb{N}^*$. Il existe des polynômes irréductibles de degré $n$ sur $\mathbb{Z}$.
  \end{application}

  \reference[GOZ]{12}

  \begin{theorem}[Critère d'irréductibilité modulo $p$]
    Soient $\mathbb{K}$ le corps des fractions de $A$ et $P = \sum_{i=0}^n a_i X^i \in A[X]$ de degré $n \geq 1$. Soit $I$ un idéal premier de $A$. On pose $B=A/I$ et $\mathbb{L}$ le corps des fractions de $B$. On suppose $a_n \notin I$.
    \newpar
    Si $\overline{P}$ est irréductible dans $\mathbb{L}[X]$, alors $P$ est irréductible dans $\mathbb{K}[X]$.
  \end{theorem}

  \begin{example}
    Le polynôme $X^3-127X^2+3608X+19$ est irréductible dans $\mathbb{Z}[X]$.
  \end{example}

  \subsection{Adjonction de racines}

  Soit $\mathbb{K}$ un corps commutatif.

  \subsubsection{Éléments algébriques, transcendants}

  \reference[PER]{66}

  \begin{definition}
    Soient $\mathbb{L}$ une extension de $\mathbb{K}$ et $\alpha \in \mathbb{L}$. Soit $\operatorname{ev}_\alpha : \mathbb{K}[X] \rightarrow \mathbb{L}$ le morphisme d'évaluation en $\alpha$.
    \begin{itemize}
      \item On note $\mathrm{Ann}(\alpha)$ l'idéal des polynômes annulateurs de $\alpha$. Notons qu'on a $\mathrm{Ann}(\alpha) = \ker(\operatorname{ev}_\alpha)$.
      \item Si $\operatorname{ev}_\alpha$ est injectif, on dit que $\alpha$ est \textbf{transcendant} sur $\mathbb{K}$.
      \item Sinon, $\alpha$ est dit \textbf{algébrique} sur $\mathbb{K}$.
    \end{itemize}
  \end{definition}

  \begin{example}
    \begin{itemize}
      \item $e$ et $\pi$ sont transcendants sur $\mathbb{Q}$ (théorèmes d'Hermite et de Lindemann).
      \item $\sqrt{2}$, $i$, ... sont algébriques sur $\mathbb{Q}$.
    \end{itemize}
  \end{example}

  \begin{proposition}
    Soient $\mathbb{L}$ une extension de $\mathbb{K}$ et $\alpha \in \mathbb{L}$. Les assertions suivantes sont équivalentes.
    \begin{enumerate}[label=(\roman*)]
      \item $\alpha$ est algébrique sur $\mathbb{K}$.
      \item $\mathbb{K}[\alpha] = \mathbb{K}(\alpha)$.
      \item $[\mathbb{K}[\alpha]:\mathbb{K}] < +\infty$.
    \end{enumerate}
  \end{proposition}

  \begin{proposition}
    En reprenant les notations précédentes, si $\alpha$ est transcendant, on a
    \[ \mathbb{K}[\alpha] \cong \mathbb{K}[X] \text{ et } \mathbb{K}(\alpha) \cong \mathbb{K}(X) \]
  \end{proposition}

  \begin{definition}
    Soient $\mathbb{L}$ une extension de $\mathbb{K}$ et $\alpha \in \mathbb{L}$. Si $\alpha$ est algébrique sur $\mathbb{K}$, alors $\mathrm{Ann}(\alpha)$ est un idéal principal non nul. Donc, il existe $P \in \mathbb{K}[X]$ unitaire tel que $\mathrm{Ann}(\alpha) = (P)$. On note $\pi_\alpha$ ce polynôme $P$ : c'est le \textbf{polynôme minimal} de $\alpha$ sur $\mathbb{K}$.
  \end{definition}

  \begin{example}
    Sur $\mathbb{Q}$, on a $\pi_{\sqrt{2}} = X^2 - 2$ et $\pi_i = X^2 + 1$.
  \end{example}

  \reference[GOZ]{31}

  \begin{proposition}
    Soient $\mathbb{L}$ une extension de $\mathbb{K}$ et $\alpha \in \mathbb{L}$. Soient $P \in \mathbb{K}[X]$. Les assertions suivantes sont équivalentes :
    \begin{enumerate}[label=(\roman*)]
      \item $P = \mu_\alpha$.
      \item $P \in \mathrm{Ann}(\alpha)$ et est unitaire et $\forall R \in \mathrm{Ann}(\alpha) \setminus \{ 0 \}, \, \deg(P) \leq \deg(R)$.
      \item $P \in \mathrm{Ann}(\alpha)$ et est unitaire et irréductible dans $\mathbb{K}[X]$.
    \end{enumerate}
  \end{proposition}

  \subsubsection{Corps de rupture}

  \reference{57}

  \begin{definition}
    Soient $\mathbb{L}$ une extension de $\mathbb{K}$ et $P \in \mathbb{K}[X]$ irréductible. On dit que $\mathbb{L}$ est un \textbf{corps de rupture} de $P$ si $\mathbb{L} = \mathbb{K}[\alpha]$ où $\alpha \in \mathbb{L}$ est une racine de $P$.
  \end{definition}

  \begin{example}
    En reprenant les notations précédentes, si $\deg(P) = 1$, alors $\mathbb{K}$ est un corps de rupture de $P$.
  \end{example}

  \begin{theorem}
    Soit $P \in \mathbb{K}[X]$ un polynôme irréductible sur $\mathbb{K}$.
    \begin{itemize}
      \item Il existe un corps de rupture de $P$.
      \item Si $\mathbb{L} = \mathbb{K}[\alpha]$ et $\mathbb{L}' = \mathbb{K}[\beta]$ sont deux corps de rupture de $P$, alors il existe un unique $\mathbb{K}$-isomorphisme $\varphi : \mathbb{L} \rightarrow \mathbb{L}'$ tel que $\varphi(\alpha) = \beta$.
    \end{itemize}
  \end{theorem}

  \begin{application}
    $X^2 + 1$ est un polynôme irréductible sur $\mathbb{R}$ dont $\mathbb{R}[X]/(X^2+1)$ est un corps de rupture. On pose alors $\mathbb{C} = \mathbb{R}[X]/(X^2+1)$, le corps des nombres complexes, et on note $i$ la classe de $X$ dans l'anneau quotient.
  \end{application}

  \begin{remark}
    Si $\mathbb{L}$ est un corps de rupture d'un polynôme $P \in \mathbb{K}[X]$, on a $[\mathbb{L}:\mathbb{K}] = \deg(P)$. Plus précisément, une base de $\mathbb{L}$ en tant que $\mathbb{K}$-espace vectoriel est $(1, \alpha, \dots, \alpha^{\deg(P)-1})$.
  \end{remark}

  \subsubsection{Corps de décomposition}

  \begin{definition}
    Soit $P \in \mathbb{K}[X]$ de degré $n \geq 1$. On dit que $\mathbb{L}$ est un \textbf{corps de décomposition} de $P$ si :
    \begin{itemize}
      \item Il existe $a \in \mathbb{L}$ et $\alpha_1, \dots, \alpha_n \in \mathbb{L}$ tels que $P = a(X-\alpha_1) \dots (X-\alpha_n)$.
      \item $\mathbb{L} = \mathbb{K}[\alpha_1, \dots, \alpha_n]$.
    \end{itemize}
  \end{definition}

  \begin{example}
    \begin{itemize}
      \item $\mathbb{K}$ est un corps de décomposition de tout polynôme de degré $1$ sur $\mathbb{K}$.
      \item $\mathbb{C}$ est un corps de décomposition de $X^2+1$ sur $\mathbb{R}$.
    \end{itemize}
  \end{example}

  \begin{theorem}
    Soit $P \in \mathbb{K}[X]$ un polynôme de degré supérieur ou égal à $1$.
    \begin{itemize}
      \item Il existe un corps de décomposition de $P$.
      \item Deux corps de décomposition de $P$ sont $\mathbb{K}$-isomorphes.
    \end{itemize}
  \end{theorem}

  \reference[FGN2]{160}

  \begin{application}
    Soit $A \in \mathcal{M}_n(\mathbb{K})$. On note $\mathcal{C}(A)$ le commutant de $A$. Alors,
    \[ \mathbb{K}[A] = \mathcal{C}(A) \iff \pi_A = \chi_A = \det(XI_n - A) \]
  \end{application}

  \subsubsection{Clôture algébrique}

  \reference[GOZ]{62}

  \begin{proposition}
    \label{141-4}
    Les assertions suivantes sont équivalentes :
    \begin{enumerate}[label=(\roman*)]
      \item Tout polynôme de $\mathbb{K}[X]$ de degré supérieur ou égal à $1$ est scindé sur $\mathbb{K}$.
      \item Tout polynôme de $\mathbb{K}[X]$ de degré supérieur ou égal à $1$ admet au moins une racine dans $\mathbb{K}$.
      \item Les seuls polynômes irréductibles de $\mathbb{K}[X]$ sont ceux de degré $1$.
      \item Toute extension algébrique de $\mathbb{K}$ est égale à $\mathbb{K}$.
    \end{enumerate}
  \end{proposition}

  \begin{definition}
    Si $\mathbb{K}$ vérifie un des points de la \cref{141-4}, $\mathbb{K}$ est dit \textbf{algébriquement clos}.
  \end{definition}

  \begin{proposition}
    Tout corps algébriquement clos est infini.
  \end{proposition}

  \begin{cexample}
    $\mathbb{Q}$ et même $\mathbb{R}$ ne sont pas algébriquement clos.
  \end{cexample}

  \begin{theorem}[D'Alembert-Gauss]
    $\mathbb{C}$ est algébriquement clos.
  \end{theorem}

  \begin{definition}
    On dit que $\mathbb{L}$ est une \textbf{clôture algébrique} de $\mathbb{K}$ si $\mathbb{L}$ est une extension de $\mathbb{K}$ algébriquement close et si
    \[ \forall x \in \mathbb{L}, \, \exists P \in \mathbb{K}[X] \text{ tel que } P(x) = 0 \]
  \end{definition}

  \begin{example}
    \begin{itemize}
      \item $\mathbb{C}$ est une clôture algébrique de $\mathbb{R}$.
      \item $\overline{\mathbb{Q}} = \{ \alpha \in \mathbb{C} \mid \alpha \text{ est algébrique sur } \mathbb{Q} \}$ est une clôture algébrique de $\mathbb{Q}$.
    \end{itemize}
  \end{example}

  \begin{theorem}[Steinitz]
    \begin{enumerate}[label=(\roman*)]
      \item Il existe une clôture algébrique de $\mathbb{K}$.
      \item Deux clôtures algébriques de $\mathbb{K}$ sont $\mathbb{K}$-isomorphes.
    \end{enumerate}
  \end{theorem}

  \subsection{Polynômes cyclotomiques}

  \begin{definition}
    On appelle \textbf{$m$-ième polynôme cyclotomique} le polynôme
    \[ \Phi_m = \prod_{\xi \in \mu_m^*} (X - \xi) \]
  \end{definition}

  \begin{theorem}
    \begin{enumerate}[label=(\roman*)]
      \item $X^m - 1 = \prod_{d \mid m} \Phi_d$.
      \item $\Phi_m \in \mathbb{Z}[X]$.
      \item $\Phi_m$ est irréductible sur $\mathbb{Q}$.
    \end{enumerate}
  \end{theorem}

  \begin{corollary}
    Le polynôme minimal sur $\mathbb{Q}$ de tout élément $\xi$ de $\mu_m^*$ est $\Phi_m$. En particulier,
    \[ [\mathbb{Q}(\xi):\mathbb{Q}]=\varphi(m) \]
  \end{corollary}

  \begin{application}[Théorème de Wedderburn]
    Tout corps fini est commutatif.
  \end{application}

  \reference[GOU21]{99}

  \begin{lemma}
    Soient $a \in \mathbb{N}$ et $p$ premier tels que $p \mid \Phi_n(a)$ mais $p \nmid \Phi_d(a)$ pour tout diviseur strict $d$ de $n$. Alors $p \equiv 1 \mod n$.
  \end{lemma}

  \dev{theoreme-de-dirichlet-faible}

  \begin{application}[Dirichlet faible]
    Pour tout entier $n$, il existe une infinité de nombres premiers congrus à $1$ modulo $n$.
  \end{application}

  \subsection{Polynômes irréductibles sur \texorpdfstring{$\mathbb{F}_q$}{Fq}}

  \reference[GOZ]{87}

  Soient $p$ un nombre premier et $n \in \mathbb{N}^*$. On pose $q = p^n$.

  \begin{theorem}
    \[ \mathbb{F}_q = \mathbb{F}_p[X] / (P) \]
    où $P \in \mathbb{F}_p[X]$ est un polynôme irréductible de degré $n$ sur $\mathbb{F}_p$.
  \end{theorem}

  \begin{corollary}
    \begin{enumerate}[label=(\roman*)]
      \item Il existe des polynômes irréductibles de tout degré dans $\mathbb{F}_p[X]$.
      \item Si $P \in \mathbb{F}_p[X]$ est un polynôme irréductible sur $\mathbb{F}_p$ de degré $n$, alors $P$ divise $X^q - X$. En particulier, il est scindé sur $\mathbb{F}_q$. Donc son corps de rupture $\mathbb{F}_q = \mathbb{F}_p[X] / (P)$ est aussi son corps de décomposition.
    \end{enumerate}
  \end{corollary}

  \begin{theorem}
    Pour tout $j \in \mathbb{N}^*$, on note $I(p,q)$ l'ensemble des polynômes irréductibles unitaires de degré $j$ sur $\mathbb{F}_p$. Alors,
    \[ X^q - X = \prod_{d \mid n} \prod_{Q \in I(p,q)} Q \]
  \end{theorem}

  \begin{corollary}
    \[ q = \sum_{d \mid n} d \vert I(p,d) \vert \]
  \end{corollary}

  \begin{definition}
    On définit la \textbf{fonction de Möbius}, notée $\mu$, par
    \[
      \mu :
      \begin{array}{ccc}
        \mathbb{Z} &\rightarrow& \mathbb{Z} \\
        n &\mapsto& \begin{cases}
          1 &\text{si } n = 1 \\
          (-1)^k &\text{si } n = p_1 \dots p_k \text{ avec } p_1, \dots, p_k \text{ premiers distincts} \\
          0 &\text{sinon}
        \end{cases}
      \end{array}
    \]
  \end{definition}

  \begin{theorem}[Formule d'inversion de Möbius]
    Soient $f$ et $g$ des fonctions de $\mathbb{N}^*$ dans $\mathbb{C}$ telles que $\forall n \in \mathbb{N}^*, \, f(n) = \sum_{d \mid n} g(d)$. Alors,
    \[ \forall n \in \mathbb{N}^*, \, g(n) = \sum_{d \mid n} \mu(d) f \left( \frac{n}{d} \right) \]
  \end{theorem}

  \begin{corollary}
    \[ \forall n \in \mathbb{N}^*, \, \vert I(p,q) \vert = \frac{1}{n} \sum_{d \mid n} \mu(d) p^{\frac{n}{d}} = \frac{1}{n} \sum_{d \mid n} \mu \left( \frac{n}{d} \right) p^{d} \]
  \end{corollary}
  %</content>
\end{document}
