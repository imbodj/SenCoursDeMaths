\documentclass[12pt, a4paper]{report}

% LuaLaTeX :

\RequirePackage{iftex}
\RequireLuaTeX

% Packages :

\usepackage[french]{babel}
%\usepackage[utf8]{inputenc}
%\usepackage[T1]{fontenc}
\usepackage[pdfencoding=auto, pdfauthor={Hugo Delaunay}, pdfsubject={Mathématiques}, pdfcreator={agreg.skyost.eu}]{hyperref}
\usepackage{amsmath}
\usepackage{amsthm}
%\usepackage{amssymb}
\usepackage{stmaryrd}
\usepackage{tikz}
\usepackage{tkz-euclide}
\usepackage{fontspec}
\defaultfontfeatures[Erewhon]{FontFace = {bx}{n}{Erewhon-Bold.otf}}
\usepackage{fourier-otf}
\usepackage[nobottomtitles*]{titlesec}
\usepackage{fancyhdr}
\usepackage{listings}
\usepackage{catchfilebetweentags}
\usepackage[french, capitalise, noabbrev]{cleveref}
\usepackage[fit, breakall]{truncate}
\usepackage[top=2.5cm, right=2cm, bottom=2.5cm, left=2cm]{geometry}
\usepackage{enumitem}
\usepackage{tocloft}
\usepackage{microtype}
%\usepackage{mdframed}
%\usepackage{thmtools}
\usepackage{xcolor}
\usepackage{tabularx}
\usepackage{xltabular}
\usepackage{aligned-overset}
\usepackage[subpreambles=true]{standalone}
\usepackage{environ}
\usepackage[normalem]{ulem}
\usepackage{etoolbox}
\usepackage{setspace}
\usepackage[bibstyle=reading, citestyle=draft]{biblatex}
\usepackage{xpatch}
\usepackage[many, breakable]{tcolorbox}
\usepackage[backgroundcolor=white, bordercolor=white, textsize=scriptsize]{todonotes}
\usepackage{luacode}
\usepackage{float}
\usepackage{needspace}
\everymath{\displaystyle}

% Police :

\setmathfont{Erewhon Math}

% Tikz :

\usetikzlibrary{calc}
\usetikzlibrary{3d}

% Longueurs :

\setlength{\parindent}{0pt}
\setlength{\headheight}{15pt}
\setlength{\fboxsep}{0pt}
\titlespacing*{\chapter}{0pt}{-20pt}{10pt}
\setlength{\marginparwidth}{1.5cm}
\setstretch{1.1}

% Métadonnées :

\author{agreg.skyost.eu}
\date{\today}

% Titres :

\setcounter{secnumdepth}{3}

\renewcommand{\thechapter}{\Roman{chapter}}
\renewcommand{\thesubsection}{\Roman{subsection}}
\renewcommand{\thesubsubsection}{\arabic{subsubsection}}
\renewcommand{\theparagraph}{\alph{paragraph}}

\titleformat{\chapter}{\huge\bfseries}{\thechapter}{20pt}{\huge\bfseries}
\titleformat*{\section}{\LARGE\bfseries}
\titleformat{\subsection}{\Large\bfseries}{\thesubsection \, - \,}{0pt}{\Large\bfseries}
\titleformat{\subsubsection}{\large\bfseries}{\thesubsubsection. \,}{0pt}{\large\bfseries}
\titleformat{\paragraph}{\bfseries}{\theparagraph. \,}{0pt}{\bfseries}

\setcounter{secnumdepth}{4}

% Table des matières :

\renewcommand{\cftsecleader}{\cftdotfill{\cftdotsep}}
\addtolength{\cftsecnumwidth}{10pt}

% Redéfinition des commandes :

\renewcommand*\thesection{\arabic{section}}
\renewcommand{\ker}{\mathrm{Ker}}

% Nouvelles commandes :

\newcommand{\website}{https://github.com/imbodj/SenCoursDeMaths}

\newcommand{\tr}[1]{\mathstrut ^t #1}
\newcommand{\im}{\mathrm{Im}}
\newcommand{\rang}{\operatorname{rang}}
\newcommand{\trace}{\operatorname{trace}}
\newcommand{\id}{\operatorname{id}}
\newcommand{\stab}{\operatorname{Stab}}
\newcommand{\paren}[1]{\left(#1\right)}
\newcommand{\croch}[1]{\left[ #1 \right]}
\newcommand{\Grdcroch}[1]{\Bigl[ #1 \Bigr]}
\newcommand{\grdcroch}[1]{\bigl[ #1 \bigr]}
\newcommand{\abs}[1]{\left\lvert #1 \right\rvert}
\newcommand{\limi}[3]{\lim_{#1\to #2}#3}
\newcommand{\pinf}{+\infty}
\newcommand{\minf}{-\infty}
%%%%%%%%%%%%%% ENSEMBLES %%%%%%%%%%%%%%%%%
\newcommand{\ensemblenombre}[1]{\mathbb{#1}}
\newcommand{\Nn}{\ensemblenombre{N}}
\newcommand{\Zz}{\ensemblenombre{Z}}
\newcommand{\Qq}{\ensemblenombre{Q}}
\newcommand{\Qqp}{\Qq^+}
\newcommand{\Rr}{\ensemblenombre{R}}
\newcommand{\Cc}{\ensemblenombre{C}}
\newcommand{\Nne}{\Nn^*}
\newcommand{\Zze}{\Zz^*}
\newcommand{\Zzn}{\Zz^-}
\newcommand{\Qqe}{\Qq^*}
\newcommand{\Rre}{\Rr^*}
\newcommand{\Rrp}{\Rr_+}
\newcommand{\Rrm}{\Rr_-}
\newcommand{\Rrep}{\Rr_+^*}
\newcommand{\Rrem}{\Rr_-^*}
\newcommand{\Cce}{\Cc^*}
%%%%%%%%%%%%%%  INTERVALLES %%%%%%%%%%%%%%%%%
\newcommand{\intff}[2]{\left[#1\;,\; #2\right]  }
\newcommand{\intof}[2]{\left]#1 \;, \;#2\right]  }
\newcommand{\intfo}[2]{\left[#1 \;,\; #2\right[  }
\newcommand{\intoo}[2]{\left]#1 \;,\; #2\right[  }

\providecommand{\newpar}{\\[\medskipamount]}

\newcommand{\annexessection}{%
  \newpage%
  \subsection*{Annexes}%
}

\providecommand{\lesson}[3]{%
  \title{#3}%
  \hypersetup{pdftitle={#2 : #3}}%
  \setcounter{section}{\numexpr #2 - 1}%
  \section{#3}%
  \fancyhead[R]{\truncate{0.73\textwidth}{#2 : #3}}%
}

\providecommand{\development}[3]{%
  \title{#3}%
  \hypersetup{pdftitle={#3}}%
  \section*{#3}%
  \fancyhead[R]{\truncate{0.73\textwidth}{#3}}%
}

\providecommand{\sheet}[3]{\development{#1}{#2}{#3}}

\providecommand{\ranking}[1]{%
  \title{Terminale #1}%
  \hypersetup{pdftitle={Terminale #1}}%
  \section*{Terminale #1}%
  \fancyhead[R]{\truncate{0.73\textwidth}{Terminale #1}}%
}

\providecommand{\summary}[1]{%
  \textit{#1}%
  \par%
  \medskip%
}

\tikzset{notestyleraw/.append style={inner sep=0pt, rounded corners=0pt, align=center}}

%\newcommand{\booklink}[1]{\website/bibliographie\##1}
\newcounter{reference}
\newcommand{\previousreference}{}
\providecommand{\reference}[2][]{%
  \needspace{20pt}%
  \notblank{#1}{
    \needspace{20pt}%
    \renewcommand{\previousreference}{#1}%
    \stepcounter{reference}%
    \label{reference-\previousreference-\thereference}%
  }{}%
  \todo[noline]{%
    \protect\vspace{20pt}%
    \protect\par%
    \protect\notblank{#1}{\cite{[\previousreference]}\\}{}%
    \protect\hyperref[reference-\previousreference-\thereference]{p. #2}%
  }%
}

\definecolor{devcolor}{HTML}{00695c}
\providecommand{\dev}[1]{%
  \reversemarginpar%
  \todo[noline]{
    \protect\vspace{20pt}%
    \protect\par%
    \bfseries\color{devcolor}\href{\website/developpements/#1}{[DEV]}
  }%
  \normalmarginpar%
}

% En-têtes :

\pagestyle{fancy}
\fancyhead[L]{\truncate{0.23\textwidth}{\thepage}}
\fancyfoot[C]{\scriptsize \href{\website}{\texttt{https://github.com/imbodj/SenCoursDeMaths}}}

% Couleurs :

\definecolor{property}{HTML}{ffeb3b}
\definecolor{proposition}{HTML}{ffc107}
\definecolor{lemma}{HTML}{ff9800}
\definecolor{theorem}{HTML}{f44336}
\definecolor{corollary}{HTML}{e91e63}
\definecolor{definition}{HTML}{673ab7}
\definecolor{notation}{HTML}{9c27b0}
\definecolor{example}{HTML}{00bcd4}
\definecolor{cexample}{HTML}{795548}
\definecolor{application}{HTML}{009688}
\definecolor{remark}{HTML}{3f51b5}
\definecolor{algorithm}{HTML}{607d8b}
%\definecolor{proof}{HTML}{e1f5fe}
\definecolor{exercice}{HTML}{e1f5fe}

% Théorèmes :

\theoremstyle{definition}
\newtheorem{theorem}{Théorème}

\newtheorem{property}[theorem]{Propriété}
\newtheorem{proposition}[theorem]{Proposition}
\newtheorem{lemma}[theorem]{Activité d'introduction}
\newtheorem{corollary}[theorem]{Conséquence}

\newtheorem{definition}[theorem]{Définition}
\newtheorem{notation}[theorem]{Notation}

\newtheorem{example}[theorem]{Exemple}
\newtheorem{cexample}[theorem]{Contre-exemple}
\newtheorem{application}[theorem]{Application}

\newtheorem{algorithm}[theorem]{Algorithme}
\newtheorem{exercice}[theorem]{Exercice}

\theoremstyle{remark}
\newtheorem{remark}[theorem]{Remarque}

\counterwithin*{theorem}{section}

\newcommand{\applystyletotheorem}[1]{
  \tcolorboxenvironment{#1}{
    enhanced,
    breakable,
    colback=#1!8!white,
    %right=0pt,
    %top=8pt,
    %bottom=8pt,
    boxrule=0pt,
    frame hidden,
    sharp corners,
    enhanced,borderline west={4pt}{0pt}{#1},
    %interior hidden,
    sharp corners,
    after=\par,
  }
}

\applystyletotheorem{property}
\applystyletotheorem{proposition}
\applystyletotheorem{lemma}
\applystyletotheorem{theorem}
\applystyletotheorem{corollary}
\applystyletotheorem{definition}
\applystyletotheorem{notation}
\applystyletotheorem{example}
\applystyletotheorem{cexample}
\applystyletotheorem{application}
\applystyletotheorem{remark}
%\applystyletotheorem{proof}
\applystyletotheorem{algorithm}
\applystyletotheorem{exercice}

% Environnements :

\NewEnviron{whitetabularx}[1]{%
  \renewcommand{\arraystretch}{2.5}
  \colorbox{white}{%
    \begin{tabularx}{\textwidth}{#1}%
      \BODY%
    \end{tabularx}%
  }%
}

% Maths :

\DeclareFontEncoding{FMS}{}{}
\DeclareFontSubstitution{FMS}{futm}{m}{n}
\DeclareFontEncoding{FMX}{}{}
\DeclareFontSubstitution{FMX}{futm}{m}{n}
\DeclareSymbolFont{fouriersymbols}{FMS}{futm}{m}{n}
\DeclareSymbolFont{fourierlargesymbols}{FMX}{futm}{m}{n}
\DeclareMathDelimiter{\VERT}{\mathord}{fouriersymbols}{152}{fourierlargesymbols}{147}

% Code :

\definecolor{greencode}{rgb}{0,0.6,0}
\definecolor{graycode}{rgb}{0.5,0.5,0.5}
\definecolor{mauvecode}{rgb}{0.58,0,0.82}
\definecolor{bluecode}{HTML}{1976d2}
\lstset{
  basicstyle=\footnotesize\ttfamily,
  breakatwhitespace=false,
  breaklines=true,
  %captionpos=b,
  commentstyle=\color{greencode},
  deletekeywords={...},
  escapeinside={\%*}{*)},
  extendedchars=true,
  frame=none,
  keepspaces=true,
  keywordstyle=\color{bluecode},
  language=Python,
  otherkeywords={*,...},
  numbers=left,
  numbersep=5pt,
  numberstyle=\tiny\color{graycode},
  rulecolor=\color{black},
  showspaces=false,
  showstringspaces=false,
  showtabs=false,
  stepnumber=2,
  stringstyle=\color{mauvecode},
  tabsize=2,
  %texcl=true,
  xleftmargin=10pt,
  %title=\lstname
}

\newcommand{\codedirectory}{}
\newcommand{\inputalgorithm}[1]{%
  \begin{algorithm}%
    \strut%
    \lstinputlisting{\codedirectory#1}%
  \end{algorithm}%
}




\begin{document}
  %<*content>
  \lesson{algebra}{150}{Polynômes d'endomorphisme en dimension finie. Réduction d'un endomorphisme en dimension finie. Applications.}

  Soit $E$ un espace vectoriel de dimension finie $n$ sur un corps commutatif $\mathbb{K}$. Soit $u \in \mathcal{L}(E)$.

  \subsection{Polynômes d'endomorphismes}

  \subsubsection{L'algèbre \texorpdfstring{$\mathbb{K}[u]$}{K[u]}}

  \reference[ROM21]{603}

  \begin{notation}
    On note $u^0 = \operatorname{id}_E$ et
    \[ u^k = \underbrace{u \circ \dots \circ u}_{k \text{ fois}} \]
  \end{notation}

  \begin{definition}
    À tout polynôme $P = \sum_{k=0}^{n} a_k X^k \in \mathbb{K}[X]$ on fait correspondre l'endomorphisme $P(u) = \sum_{k=1}^{n} a_k u^k$.
  \end{definition}

  \begin{proposition}
    L'ensemble,
    \[ \mathbb{K}[u] = \{ P(u) \mid P \in \mathbb{K}[X] \} \]
    est une sous-algèbre commutative de $\mathcal{L}(E)$, de dimension inférieure ou égale à $n^2$.
  \end{proposition}

  \begin{remark}
    Au vu de l'isomorphisme entre $\mathcal{L}(E)$ et $\mathcal{M}_n(\mathbb{K})$, on définit de même $\mathbb{K}[A]$ pour une matrice $A \in \mathcal{M}_n(\mathbb{K})$. Si $A$ est la matrice de $u$ dans une base de $E$, alors pour tout $P \in \mathbb{K}[X]$, $P(A)$ est la matrice de $P(u)$ dans cette même base. Toute les propriétés énoncées pour les endomorphismes sont vraies pour les matrices, et réciproquement.
  \end{remark}

  \reference[GOU21]{184}

  \begin{proposition}
    Soient $M \in \mathcal{M}_n(\mathbb{K})$ triangulaire de la forme
    \[
      M = \begin{pmatrix}
        \alpha_1 & * & \dots & * \\
        0 & \alpha_2 & \ddots & \vdots \\
        \vdots & \ddots & \ddots & * \\
        0 & \dots & 0 & \alpha_n
      \end{pmatrix}
    \]
    et $P \in \mathbb{K}[X]$. Alors, $P(M)$ est de la forme
    \[
    P(M) = \begin{pmatrix}
      P(\alpha_1) & * & \dots & * \\
      0 & P(\alpha_2) & \ddots & \vdots \\
      \vdots & \ddots & \ddots & * \\
      0 & \dots & 0 & P(\alpha_n)
    \end{pmatrix}
    \]
  \end{proposition}

  \subsubsection{Polynôme caractéristique de \texorpdfstring{$u$}{u}}

  \reference[ROM21]{643}

  \begin{definition}
    \label{150-1}
    Soit $\lambda \in \mathbb{K}$.
    \begin{itemize}
      \item On dit que $\lambda$ est \textbf{valeur propre} de $u$ si $E_\lambda = \ker(u - \lambda \operatorname{id}_E)$ n'est pas réduit à $\{ 0 \}$.
      \item Un vecteur $x \neq 0$ tel que $u(x) = \lambda x$ est un \textbf{vecteur propre} de $u$ associé à la valeur propre $\lambda$.
      \item $E_\lambda$ est le \textbf{sous-espace propre} associé à la valeur propre $\lambda$.
      \item L'ensemble des valeurs propres de $u$ est appelé \textbf{spectre} de $u$. On le note $\operatorname{Sp}(u)$.
    \end{itemize}
  \end{definition}

  \begin{proposition}
    En notant $\chi_u = \det(X \operatorname{id}_E - u)$,
    \[ \operatorname{Sp}(u) = \{ \lambda \in \mathbb{K} \mid \chi_u(\lambda) = 0 \} \]
  \end{proposition}

  \reference{604}

  \begin{theorem}
    Soit $P \in \mathbb{K}[X]$. Pour tout valeur propre $\lambda$ de $u$ (voir \cref{150-1}), $P(\lambda)$ est une valeur propre de $P(u)$. Si le corps $\mathbb{K}$ est algébriquement clos, on a alors
    \[ \operatorname{Sp}(P(u)) = \{ P(\lambda) \mid \lambda \in \operatorname{Sp}(u) \} \]
  \end{theorem}

  \begin{cexample}
    Pour $A = \begin{pmatrix} 0 & -1 \\ 1 & 0 \end{pmatrix} \in \mathcal{M}_2(\mathbb{R})$ et $P = X^2$, on a $A^2 = -I_2$ et $\operatorname{Sp}(A) = \emptyset$.
  \end{cexample}

  \reference{644}

  \begin{definition}
    Le polynôme $\chi_u$ précédent est appelé \textbf{polynôme caractéristique} de $u$.
  \end{definition}

  \begin{remark}
    On peut définir de la même manière les mêmes notions pour une matrice de $\mathcal{M}_n(\mathbb{K})$ (une valeur est propre pour une matrice si et seulement si elle l'est pour l'endomorphisme associé). On reprendra les mêmes notations.
  \end{remark}

  \begin{example}
    Pour $A = \begin{pmatrix} a & b \\ c & d \end{pmatrix} \in \mathcal{M}_2(\mathbb{K})$, on a $\chi_A = X^2 - \trace(A)X + \det(A)$.
  \end{example}

  \begin{proposition}
    Soit $\lambda$ une valeur propre de $u$ de multiplicité $\alpha$ en tant que racine de $\chi_u$. Alors,
    \[ \dim(E_\lambda) \in \llbracket 1, \alpha \rrbracket \]
  \end{proposition}

  \reference[GOU21]{172}

  \begin{proposition}
    \begin{enumerate}[label=(\roman*)]
      \item Le polynôme caractéristique est un invariant de similitude.
      \item Soit $A \in \mathcal{M}_n(\mathbb{K})$. On note $\chi_A = \sum_{k=0}^n a_k X^k$. Alors, $a_0 = \det(A)$ et $a_{n-1} = \trace(A)$ (à un signe près).
    \end{enumerate}
  \end{proposition}

  \subsubsection{Polynôme minimal de \texorpdfstring{$u$}{u}}

  \reference[ROM21]{604}

  \begin{lemma}
    \begin{enumerate}[label=(\roman*)]
      \item $\mathrm{Ann}(u) = \{ P \in \mathbb{K}[X] \mid P(u) = 0 \}$ est un sous-ensemble de $\mathbb{K}[u]$ non réduit au polynôme nul.
      \item $\mathrm{Ann}(u)$ est le noyau de $P \mapsto P(u)$ : c'est un idéal de $\mathbb{K}[u]$.
      \item Il existe un unique polynôme unitaire engendrant cet idéal.
    \end{enumerate}
  \end{lemma}

  \begin{definition}
    On appelle \textbf{idéal annulateur} de $u$ l'idéal $\mathrm{Ann}(u)$. Le polynôme unitaire générateur est noté $\pi_u$ et est appelé \textbf{polynôme minimal} de $u$.
  \end{definition}

  \begin{remark}
    \begin{itemize}
      \item $\pi_u$ est le polynôme unitaire de plus petit degré annulant $u$.
      \item Si $A \in \mathcal{M}_n(\mathbb{K})$ est la matrice de $u$ dans une base de $E$, on a $\mathrm{Ann}(u) = \mathrm{Ann}(A)$ et $\pi_u = \pi_A$.
    \end{itemize}
  \end{remark}

  \begin{example}
    Un endomorphisme est nilpotent d'indice $q$ si et seulement si son polynôme minimal est $X^q$.
  \end{example}

  \begin{proposition}
    Soit $F$ un sous-espace vectoriel de $E$ stable par $u$. Alors, le polynôme minimal de l'endomorphisme $u_{|F} : F \rightarrow F$ divise $\pi_u$.
  \end{proposition}

  \begin{proposition}
    \begin{enumerate}[label=(\roman*)]
      \item Les valeurs propres de $u$ sont racines de tout polynôme annulateur.
      \item Les valeurs propres de $u$ sont exactement les racines de $\pi_u$.
    \end{enumerate}
  \end{proposition}

  \reference[GOU21]{186}

  \begin{remark}
    $\pi_u$ et $\chi_u$ partagent dont les mêmes racines.
  \end{remark}

  \reference[ROM21]{606}

  \begin{theorem}
    $P \mapsto P(u)$ induit un isomorphisme :
    \[ \mathbb{K}[X]/(\pi_u) \cong \mathbb{K}[u] \]
  \end{theorem}

  \begin{corollary}
    L'espace vectoriel $\mathbb{K}[u]$ est de dimension égale à $p_u = \deg(\pi_u)$, une base étant donnée par $(u^k)_{k \in \llbracket 1, p_u \rrbracket}$.
  \end{corollary}

  \begin{corollary}
    \[ \mathbb{K}[u] \text{ est un corps} \iff \mathbb{K}[u] \text{ est intègre} \iff u \text{ est irréductible} \]
  \end{corollary}

  \begin{theorem}[Cayley-Hamilton]
    \[ \pi_u \mid \chi_u \]
  \end{theorem}

  \begin{corollary}
    \[ \dim(\mathbb{K}[u]) \leq n \]
  \end{corollary}

  \begin{corollary}
    Si $u$ est inversible,
    \[ u^{-1} = -\frac{1}{\det(u)} \sum_{k=1}^n a_k u^{k-1} \]
    En particulier, $u^{-1} \in \mathbb{K}[u]$.
  \end{corollary}

  \begin{corollary}
    $u$ est nilpotent si et seulement si $\chi_u = X^n$.
  \end{corollary}

  \subsection{Réduction d'endomorphismes}

  \subsubsection{Diagonalisation}

  \reference{683}

  \begin{definition}
    \begin{itemize}
      \item On dit que $u$ est \textbf{diagonalisable} s'il existe une base de $E$ dans laquelle la matrice de $u$ est diagonale.
      \item On dit qu'une matrice $A \in \mathcal{M}_n(\mathbb{K})$ est \textbf{diagonalisable} si elle est semblable à une matrice diagonale.
    \end{itemize}
  \end{definition}

  \begin{remark}
    $u$ est diagonalisable si et seulement si sa matrice dans n'importe quelle base de $E$ l'est.
  \end{remark}

  \reference[BMP]{166}

  \begin{example}
    \begin{itemize}
      \item Les projecteurs (ie. les endomorphismes $p \in \mathcal{L}(E)$ tels que $p^2 = p$) sont toujours diagonalisables, à valeurs propres dans $\{ 0, 1 \}$.
      \item Les symétries (ie. les endomorphismes $s \in \mathcal{L}(E)$ tels que $s^2 = \operatorname{id}_E$) sont toujours diagonalisables, à valeurs propres dans $\{ \pm 1 \}$. Par exemple, l'endomorphisme de transposition $A \mapsto \tr{A}$ est diagonalisable.
    \end{itemize}
  \end{example}

  \reference[ROM21]{683}

  \begin{proposition}
    Si $u$ a $n$ valeurs propres distinctes dans $\mathbb{K}$, alors il est diagonalisable.
  \end{proposition}

  \reference{609}

  \begin{theorem}[Lemme des noyaux]
    Soit $P = P_1 \dots P_k \in \mathbb{K}[X]$ où les polynômes $P_1, \dots, P_k$ sont premiers entre eux deux à deux. Alors,
    \[ \ker(P(u)) = \bigoplus_{i=1}^k \ker(P_i(u)) \]
  \end{theorem}

  \reference{683}

  \begin{theorem}
    Soit $\operatorname{Sp}(u) = \{ \lambda_1, \dots, \lambda_p \}$. Les assertions suivantes sont équivalentes :
    \begin{enumerate}[label=(\roman*)]
      \item $u$ est diagonalisable sur $\mathbb{K}$.
      \item $E = \bigoplus_{k=1}^p E_{\lambda_k}$.
      \item $\sum_{k=1}^p \dim(E_{\lambda_k}) = n$.
      \item $\chi_n$ est scindé sur $\mathbb{K}$ et pour tout $k \in \llbracket 1, p \rrbracket$, la dimension de $E_{\lambda_k}$ est égale à la multiplicité de $\lambda_k$ dans $\chi_u$.
      \item $\exists P \in \mathrm{Ann}(u)$ scindé à racines simples.
      \item $\pi_u$ est scindé à racines simples.
    \end{enumerate}
  \end{theorem}

  \reference[GOU21]{177}

  \begin{example}
    $\begin{pmatrix} 0 & 2 & -1 \\ 3 & -2 & 0 \\ -2 & 2 & 1 \end{pmatrix}$ est diagonalisable, semblable à $\begin{pmatrix} 1 & 0 & 0 \\ 0 & 2 & 0 \\ 0 & 0 & -4 \end{pmatrix}$.
  \end{example}

  \reference[ROM21]{684}

  \begin{theorem}[Diagonalisation simultanée]
    Soit $(u_i)_{i \in I}$ une famille d'endomorphismes de $E$ diagonalisables. Il existe une base commune de diagonalisation dans $E$ pour $(u_i)_{i \in I}$ si et seulement si ces endomorphismes commutent deux-à-deux.
  \end{theorem}

  \reference{734}

  \begin{theorem}[Spectral]
    Tout endomorphisme symétrique se diagonalise dans une base orthonormée.
  \end{theorem}

  \subsubsection{Trigonalisation}

  \reference{675}

  \begin{definition}
    \begin{itemize}
      \item On dit que $u$ est \textbf{trigonalisable} s'il existe une base de $E$ dans laquelle la matrice de $u$ est triangulaire supérieure.
      \item On dit qu'une matrice $A \in \mathcal{M}_n(\mathbb{K})$ est \textbf{trigonalisable} si elle est semblable à une matrice diagonale.
    \end{itemize}
  \end{definition}

  \begin{remark}
    $u$ est trigonalisable si et seulement si sa matrice dans n'importe quelle base de $E$ l'est.
  \end{remark}

  \begin{example}
    Une matrice à coefficients réels ayant des valeurs propres imaginaires pures n'est pas trigonalisable dans $\mathcal{M}_n(\mathbb{R})$.
  \end{example}

  \begin{theorem}
    $u$ est trigonalisable sur $\mathbb{K}$ si et seulement si $\chi_u$ est scindé sur $\mathbb{K}$.
  \end{theorem}

  \begin{corollary}
    Si $\mathbb{K}$ est algébriquement clos, tout endomorphisme de $u$ est trigonalisable sur $\mathbb{K}$.
  \end{corollary}

  \begin{proposition}
    Si $u$ est trigonalisable, sa trace est la somme de ses valeurs propres et son déterminant est le produit de ses valeurs propres.
  \end{proposition}

  \begin{theorem}[Trigonalisation simultanée]
    Soit $(u_i)_{i \in I}$ une famille d'endomorphismes de $E$ diagonalisables qui commutent deux-à-deux. Alors, il existe une base commune de trigonalisation.
  \end{theorem}

  \subsubsection{Décomposition de Dunford}

  \reference[GOU21]{203}
  \dev{decomposition-de-dunford}

  \begin{theorem}[Décomposition de Dunford]
    On suppose que $\pi_u$ est scindé sur $\mathbb{K}$. Alors il existe un unique couple d'endomorphismes $(d, n)$ tels que :
    \begin{itemize}
      \item $d$ est diagonalisable et $n$ est nilpotent.
      \item $u = d + n$.
      \item $d n = n d$.
    \end{itemize}
  \end{theorem}

  \begin{corollary}
    Si $u$ vérifie les hypothèse précédentes, pour tout $k \in \mathbb{N}$, $u^k = (d + n)^k = \sum_{i=0}^m \binom{k}{i} d^i n^{k-i}$, avec $m = \min(k, l)$ où $l$ désigne l'indice de nilpotence de $n$.
  \end{corollary}

  \begin{remark}
    On peut montrer de plus que $d$ et $n$ sont des polynômes en $u$.
  \end{remark}

  \subsection{Applications}

  \subsubsection{Commutant}

  \reference[FGN2]{160}

  Soit $A \in \mathcal{M}_n(\mathbb{K})$.

  \begin{notation}
    On note $\mathcal{C}(A)$ le commutant de $A$.
  \end{notation}

  \begin{lemma}
    \[ \dim_{\mathbb{K}}(\mathcal{C}(A)) \geq n \]
  \end{lemma}

  \dev{dimension-du-commutant}

  \begin{application}
    Soit $A \in \mathcal{M}_n(\mathbb{K})$. On note $\mathcal{C}(A)$ le commutant de $A$. Alors,
    \[ \mathbb{K}[A] = \mathcal{C}(A) \iff \pi_A = \chi_A = \det(XI_n - A) \]
  \end{application}

  \subsubsection{Exponentielles de matrices}

  \reference[ROM21]{761}

  \begin{lemma}
    \begin{enumerate}[label=(\roman*)]
      \item La série entière $\sum \frac{z^k}{k!}$ a un rayon de convergence infini.
      \item $\sum \frac{A^k}{k!}$ est convergente pour toute matrice $A \in \mathcal{M}_n(\mathbb{K})$.
    \end{enumerate}
  \end{lemma}

  \begin{definition}
    Soit $A \in \mathcal{M}_n(\mathbb{K})$. On définit \textbf{l'exponentielle} de $A$ par
    \[ \sum_{k=0}^{+\infty} \frac{A^k}{k!} \]
    on la note aussi $\exp(A)$ ou $e^A$.
  \end{definition}

  \begin{theorem}
    Soit $A \in \mathcal{M}_n(\mathbb{K})$.
    \begin{enumerate}[label=(\roman*)]
      \item $\exp : \mathcal{M}_n(\mathbb{K}) \rightarrow \mathcal{M}_n(\mathbb{K})$ est continue.
      \item Si $A$ est nilpotente d'indice $q$, $\exp(A) = \sum_{k=0}^{q-1} \frac{A^k}{k!}$.
      \item $\exp(A) \in \mathbb{K}[A]$. En particulier, $\exp(A)$ commute avec $A$.
      \item Si $A = \operatorname{Diag}(\lambda_1, \dots, \lambda_n)$, alors $\exp(A) = \operatorname{Diag}(e^\lambda_1, \dots, e^\lambda_n)$.
      \item Si $B = PAP^{-1}$ pour $P \in \mathrm{GL}_n(\mathbb{K})$, alors $e^B = P^{-1} e^A P$.
      \item $\det(e^A) = e^{\trace(A)}$.
      \item $t \mapsto e^{tA}$ est de classe $\mathcal{C}^\infty$, de dérivée $t \mapsto e^{tA}A$.
    \end{enumerate}
  \end{theorem}

  \begin{proposition}
    Soient $A, B \in \mathcal{M}_n(\mathbb{K})$ qui commutent. Alors,
    \[ e^A e^B = e^{A+B} = e^B e^A \]
  \end{proposition}

  \begin{corollary}
    Soit $A \in \mathcal{M}_n(\mathbb{K})$. Alors, $e^A$ est inversible, d'inverse $e^{-A}$.
  \end{corollary}

  \begin{example}
    Soit $A \in \mathcal{M}_n(\mathbb{K})$ qui admet une décomposition de Dunford $A = D+N$ où $D$ est diagonalisable et $N$ est nilpotente d'indice $q$. Alors,
    \begin{itemize}
      \item $e^A = e^D e^N = e^D \sum_{k=0}^{q-1} \frac{N^k}{k!}$.
      \item La décomposition de Dunford de $e^A$ est $e^A = e^D + e^D(e^N - I_n)$ avec $e^D$ diagonalisable et $e^D(e^N - I_n)$ nilpotente.
    \end{itemize}
  \end{example}

  \reference[GOU20]{380}

  \begin{application}
    Une équation différentielle linéaire homogène $(H) : Y' = AY$ (où $A$ est constante en $t$) a ses solutions maximales définies sur $\mathbb{R}$ et le problème de Cauchy
    \[ \begin{cases} Y' = AY \\ Y(0) = y_0 \end{cases} \]
    a pour (unique) solution $t \mapsto e^{tA} y_0$.
  \end{application}

  \reference[I-P]{177}

  \begin{application}[Équation de Sylvester]
    Soient $A$ et $B \in \mathcal{M}_n(\mathbb{C})$ deux matrices dont les valeurs propres sont de partie réelle strictement négative. Alors pour tout $C \in \mathcal{M}_n(\mathbb{C})$, l'équation $AX + XB = C$ admet une unique solution $X$ dans $\mathcal{M}_n(\mathbb{C})$.
  \end{application}

  \subsubsection{Étude d'une suite de polygones}

  \reference[GOU21]{153}

  \begin{lemma}[Déterminant circulant]
    Soient $n \in \mathbb{N}^*$ et $a_1, \dots, a_n \in \mathbb{C}$. On pose $\omega = e^{\frac{2i\pi}{n}}$. Alors
    \[ \begin{vmatrix} a_0 & a_1 & \dots & a_{n-1} \\ a_{n-1} & a_0 & \dots & a_{n-2}\\ \vdots & \vdots & \ddots & \vdots \\ a_1 & a_2 & \dots & a_0 \end{vmatrix} = \prod_{j=0}^{n-1} P(\omega^j) \]
    où $P = \sum_{k=0}^{n-1} a_k X^k$.
  \end{lemma}

  \reference[I-P]{389}

  \begin{application}[Suite de polygones]
    Soit $P_0$ un polygone dont les sommets sont $\{ z_{0,1}, \dots, z_{0,n} \}$. On définit la suite de polygones $(P_k)$ par récurrence en disant que, pour tout $k \in \mathbb{N}^*$, les sommets de $P_{k+1}$ sont les milieux des arêtes de $P_k$.
    \newpar
    Alors la suite $(P_k)$ converge vers l'isobarycentre de $P_0$.
  \end{application}

  \annexessection

  \reference[I-P]{389}

  \begin{figure}[h]
    \begin{center}
      \begin{tikzpicture}
        \coordinate (A) at (0:3);
        \coordinate (B) at (72:3);
        \coordinate (C) at (2*72:3);
        \coordinate (D) at (3*72:3);
        \coordinate (E) at (4*72:3);
        \coordinate (F) at (A);
        \foreach \i in {0,...,10} {
          \draw(A) node {$\bullet$};
          \draw(B) node {$\bullet$};
          \draw(C) node {$\bullet$};
          \draw(D) node {$\bullet$};
          \draw(E) node {$\bullet$};
          \draw[fill=cyan!60, fill opacity=0.2](A) -- (B) -- (C) -- (D) -- (E) -- (A);
          \coordinate (A) at ($(A)!0.5!(B)$);
          \coordinate (B) at ($(B)!0.5!(C)$);
          \coordinate (C) at ($(C)!0.5!(D)$);
          \coordinate (D) at ($(D)!0.5!(E)$);
          \coordinate (E) at ($(E)!0.5!(F)$);
          \coordinate (F) at (A);
        }
      \end{tikzpicture}
    \end{center}
    \caption{La suite de polygones.}
  \end{figure}
  %</content>
\end{document}
