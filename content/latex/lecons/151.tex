\documentclass[12pt, a4paper]{report}

% LuaLaTeX :

\RequirePackage{iftex}
\RequireLuaTeX

% Packages :

\usepackage[french]{babel}
%\usepackage[utf8]{inputenc}
%\usepackage[T1]{fontenc}
\usepackage[pdfencoding=auto, pdfauthor={Hugo Delaunay}, pdfsubject={Mathématiques}, pdfcreator={agreg.skyost.eu}]{hyperref}
\usepackage{amsmath}
\usepackage{amsthm}
%\usepackage{amssymb}
\usepackage{stmaryrd}
\usepackage{tikz}
\usepackage{tkz-euclide}
\usepackage{fontspec}
\defaultfontfeatures[Erewhon]{FontFace = {bx}{n}{Erewhon-Bold.otf}}
\usepackage{fourier-otf}
\usepackage[nobottomtitles*]{titlesec}
\usepackage{fancyhdr}
\usepackage{listings}
\usepackage{catchfilebetweentags}
\usepackage[french, capitalise, noabbrev]{cleveref}
\usepackage[fit, breakall]{truncate}
\usepackage[top=2.5cm, right=2cm, bottom=2.5cm, left=2cm]{geometry}
\usepackage{enumitem}
\usepackage{tocloft}
\usepackage{microtype}
%\usepackage{mdframed}
%\usepackage{thmtools}
\usepackage{xcolor}
\usepackage{tabularx}
\usepackage{xltabular}
\usepackage{aligned-overset}
\usepackage[subpreambles=true]{standalone}
\usepackage{environ}
\usepackage[normalem]{ulem}
\usepackage{etoolbox}
\usepackage{setspace}
\usepackage[bibstyle=reading, citestyle=draft]{biblatex}
\usepackage{xpatch}
\usepackage[many, breakable]{tcolorbox}
\usepackage[backgroundcolor=white, bordercolor=white, textsize=scriptsize]{todonotes}
\usepackage{luacode}
\usepackage{float}
\usepackage{needspace}
\everymath{\displaystyle}

% Police :

\setmathfont{Erewhon Math}

% Tikz :

\usetikzlibrary{calc}
\usetikzlibrary{3d}

% Longueurs :

\setlength{\parindent}{0pt}
\setlength{\headheight}{15pt}
\setlength{\fboxsep}{0pt}
\titlespacing*{\chapter}{0pt}{-20pt}{10pt}
\setlength{\marginparwidth}{1.5cm}
\setstretch{1.1}

% Métadonnées :

\author{agreg.skyost.eu}
\date{\today}

% Titres :

\setcounter{secnumdepth}{3}

\renewcommand{\thechapter}{\Roman{chapter}}
\renewcommand{\thesubsection}{\Roman{subsection}}
\renewcommand{\thesubsubsection}{\arabic{subsubsection}}
\renewcommand{\theparagraph}{\alph{paragraph}}

\titleformat{\chapter}{\huge\bfseries}{\thechapter}{20pt}{\huge\bfseries}
\titleformat*{\section}{\LARGE\bfseries}
\titleformat{\subsection}{\Large\bfseries}{\thesubsection \, - \,}{0pt}{\Large\bfseries}
\titleformat{\subsubsection}{\large\bfseries}{\thesubsubsection. \,}{0pt}{\large\bfseries}
\titleformat{\paragraph}{\bfseries}{\theparagraph. \,}{0pt}{\bfseries}

\setcounter{secnumdepth}{4}

% Table des matières :

\renewcommand{\cftsecleader}{\cftdotfill{\cftdotsep}}
\addtolength{\cftsecnumwidth}{10pt}

% Redéfinition des commandes :

\renewcommand*\thesection{\arabic{section}}
\renewcommand{\ker}{\mathrm{Ker}}

% Nouvelles commandes :

\newcommand{\website}{https://github.com/imbodj/SenCoursDeMaths}

\newcommand{\tr}[1]{\mathstrut ^t #1}
\newcommand{\im}{\mathrm{Im}}
\newcommand{\rang}{\operatorname{rang}}
\newcommand{\trace}{\operatorname{trace}}
\newcommand{\id}{\operatorname{id}}
\newcommand{\stab}{\operatorname{Stab}}
\newcommand{\paren}[1]{\left(#1\right)}
\newcommand{\croch}[1]{\left[ #1 \right]}
\newcommand{\Grdcroch}[1]{\Bigl[ #1 \Bigr]}
\newcommand{\grdcroch}[1]{\bigl[ #1 \bigr]}
\newcommand{\abs}[1]{\left\lvert #1 \right\rvert}
\newcommand{\limi}[3]{\lim_{#1\to #2}#3}
\newcommand{\pinf}{+\infty}
\newcommand{\minf}{-\infty}
%%%%%%%%%%%%%% ENSEMBLES %%%%%%%%%%%%%%%%%
\newcommand{\ensemblenombre}[1]{\mathbb{#1}}
\newcommand{\Nn}{\ensemblenombre{N}}
\newcommand{\Zz}{\ensemblenombre{Z}}
\newcommand{\Qq}{\ensemblenombre{Q}}
\newcommand{\Qqp}{\Qq^+}
\newcommand{\Rr}{\ensemblenombre{R}}
\newcommand{\Cc}{\ensemblenombre{C}}
\newcommand{\Nne}{\Nn^*}
\newcommand{\Zze}{\Zz^*}
\newcommand{\Zzn}{\Zz^-}
\newcommand{\Qqe}{\Qq^*}
\newcommand{\Rre}{\Rr^*}
\newcommand{\Rrp}{\Rr_+}
\newcommand{\Rrm}{\Rr_-}
\newcommand{\Rrep}{\Rr_+^*}
\newcommand{\Rrem}{\Rr_-^*}
\newcommand{\Cce}{\Cc^*}
%%%%%%%%%%%%%%  INTERVALLES %%%%%%%%%%%%%%%%%
\newcommand{\intff}[2]{\left[#1\;,\; #2\right]  }
\newcommand{\intof}[2]{\left]#1 \;, \;#2\right]  }
\newcommand{\intfo}[2]{\left[#1 \;,\; #2\right[  }
\newcommand{\intoo}[2]{\left]#1 \;,\; #2\right[  }

\providecommand{\newpar}{\\[\medskipamount]}

\newcommand{\annexessection}{%
  \newpage%
  \subsection*{Annexes}%
}

\providecommand{\lesson}[3]{%
  \title{#3}%
  \hypersetup{pdftitle={#2 : #3}}%
  \setcounter{section}{\numexpr #2 - 1}%
  \section{#3}%
  \fancyhead[R]{\truncate{0.73\textwidth}{#2 : #3}}%
}

\providecommand{\development}[3]{%
  \title{#3}%
  \hypersetup{pdftitle={#3}}%
  \section*{#3}%
  \fancyhead[R]{\truncate{0.73\textwidth}{#3}}%
}

\providecommand{\sheet}[3]{\development{#1}{#2}{#3}}

\providecommand{\ranking}[1]{%
  \title{Terminale #1}%
  \hypersetup{pdftitle={Terminale #1}}%
  \section*{Terminale #1}%
  \fancyhead[R]{\truncate{0.73\textwidth}{Terminale #1}}%
}

\providecommand{\summary}[1]{%
  \textit{#1}%
  \par%
  \medskip%
}

\tikzset{notestyleraw/.append style={inner sep=0pt, rounded corners=0pt, align=center}}

%\newcommand{\booklink}[1]{\website/bibliographie\##1}
\newcounter{reference}
\newcommand{\previousreference}{}
\providecommand{\reference}[2][]{%
  \needspace{20pt}%
  \notblank{#1}{
    \needspace{20pt}%
    \renewcommand{\previousreference}{#1}%
    \stepcounter{reference}%
    \label{reference-\previousreference-\thereference}%
  }{}%
  \todo[noline]{%
    \protect\vspace{20pt}%
    \protect\par%
    \protect\notblank{#1}{\cite{[\previousreference]}\\}{}%
    \protect\hyperref[reference-\previousreference-\thereference]{p. #2}%
  }%
}

\definecolor{devcolor}{HTML}{00695c}
\providecommand{\dev}[1]{%
  \reversemarginpar%
  \todo[noline]{
    \protect\vspace{20pt}%
    \protect\par%
    \bfseries\color{devcolor}\href{\website/developpements/#1}{[DEV]}
  }%
  \normalmarginpar%
}

% En-têtes :

\pagestyle{fancy}
\fancyhead[L]{\truncate{0.23\textwidth}{\thepage}}
\fancyfoot[C]{\scriptsize \href{\website}{\texttt{https://github.com/imbodj/SenCoursDeMaths}}}

% Couleurs :

\definecolor{property}{HTML}{ffeb3b}
\definecolor{proposition}{HTML}{ffc107}
\definecolor{lemma}{HTML}{ff9800}
\definecolor{theorem}{HTML}{f44336}
\definecolor{corollary}{HTML}{e91e63}
\definecolor{definition}{HTML}{673ab7}
\definecolor{notation}{HTML}{9c27b0}
\definecolor{example}{HTML}{00bcd4}
\definecolor{cexample}{HTML}{795548}
\definecolor{application}{HTML}{009688}
\definecolor{remark}{HTML}{3f51b5}
\definecolor{algorithm}{HTML}{607d8b}
%\definecolor{proof}{HTML}{e1f5fe}
\definecolor{exercice}{HTML}{e1f5fe}

% Théorèmes :

\theoremstyle{definition}
\newtheorem{theorem}{Théorème}

\newtheorem{property}[theorem]{Propriété}
\newtheorem{proposition}[theorem]{Proposition}
\newtheorem{lemma}[theorem]{Activité d'introduction}
\newtheorem{corollary}[theorem]{Conséquence}

\newtheorem{definition}[theorem]{Définition}
\newtheorem{notation}[theorem]{Notation}

\newtheorem{example}[theorem]{Exemple}
\newtheorem{cexample}[theorem]{Contre-exemple}
\newtheorem{application}[theorem]{Application}

\newtheorem{algorithm}[theorem]{Algorithme}
\newtheorem{exercice}[theorem]{Exercice}

\theoremstyle{remark}
\newtheorem{remark}[theorem]{Remarque}

\counterwithin*{theorem}{section}

\newcommand{\applystyletotheorem}[1]{
  \tcolorboxenvironment{#1}{
    enhanced,
    breakable,
    colback=#1!8!white,
    %right=0pt,
    %top=8pt,
    %bottom=8pt,
    boxrule=0pt,
    frame hidden,
    sharp corners,
    enhanced,borderline west={4pt}{0pt}{#1},
    %interior hidden,
    sharp corners,
    after=\par,
  }
}

\applystyletotheorem{property}
\applystyletotheorem{proposition}
\applystyletotheorem{lemma}
\applystyletotheorem{theorem}
\applystyletotheorem{corollary}
\applystyletotheorem{definition}
\applystyletotheorem{notation}
\applystyletotheorem{example}
\applystyletotheorem{cexample}
\applystyletotheorem{application}
\applystyletotheorem{remark}
%\applystyletotheorem{proof}
\applystyletotheorem{algorithm}
\applystyletotheorem{exercice}

% Environnements :

\NewEnviron{whitetabularx}[1]{%
  \renewcommand{\arraystretch}{2.5}
  \colorbox{white}{%
    \begin{tabularx}{\textwidth}{#1}%
      \BODY%
    \end{tabularx}%
  }%
}

% Maths :

\DeclareFontEncoding{FMS}{}{}
\DeclareFontSubstitution{FMS}{futm}{m}{n}
\DeclareFontEncoding{FMX}{}{}
\DeclareFontSubstitution{FMX}{futm}{m}{n}
\DeclareSymbolFont{fouriersymbols}{FMS}{futm}{m}{n}
\DeclareSymbolFont{fourierlargesymbols}{FMX}{futm}{m}{n}
\DeclareMathDelimiter{\VERT}{\mathord}{fouriersymbols}{152}{fourierlargesymbols}{147}

% Code :

\definecolor{greencode}{rgb}{0,0.6,0}
\definecolor{graycode}{rgb}{0.5,0.5,0.5}
\definecolor{mauvecode}{rgb}{0.58,0,0.82}
\definecolor{bluecode}{HTML}{1976d2}
\lstset{
  basicstyle=\footnotesize\ttfamily,
  breakatwhitespace=false,
  breaklines=true,
  %captionpos=b,
  commentstyle=\color{greencode},
  deletekeywords={...},
  escapeinside={\%*}{*)},
  extendedchars=true,
  frame=none,
  keepspaces=true,
  keywordstyle=\color{bluecode},
  language=Python,
  otherkeywords={*,...},
  numbers=left,
  numbersep=5pt,
  numberstyle=\tiny\color{graycode},
  rulecolor=\color{black},
  showspaces=false,
  showstringspaces=false,
  showtabs=false,
  stepnumber=2,
  stringstyle=\color{mauvecode},
  tabsize=2,
  %texcl=true,
  xleftmargin=10pt,
  %title=\lstname
}

\newcommand{\codedirectory}{}
\newcommand{\inputalgorithm}[1]{%
  \begin{algorithm}%
    \strut%
    \lstinputlisting{\codedirectory#1}%
  \end{algorithm}%
}




\begin{document}
  %<*content>
  \lesson{algebra}{151}{Sous-espaces stables par un endomorphisme ou une famille d'endomorphismes d'un espace vectoriel de dimension finie. Applications.}

  Soit $E$ un espace vectoriel sur un corps $\mathbb{K}$ de dimension finie $n$. Soit $u \in \mathcal{L}(E)$ un endomorphisme de $E$.

  \subsection{Stabilité}

  \subsubsection{Définitions, endomorphismes induits}

  \reference[BMP]{158}

  \begin{definition}
    Soit $F$ un sous-espace vectoriel de $E$. On dit que $F$ est \textbf{stable} par $u$ si $u(F) \subseteq F$.
  \end{definition}

  \begin{example}
    Le noyau et l'image de $u$ sont stables par $u$.
  \end{example}

  \begin{proposition}
    Si $\mathbb{K} = \mathbb{R}$, alors $u$ admet au moins une droite ou un plan stable.
  \end{proposition}

  \begin{proposition}
    Soit $F$ un sous-espace de $E$ stable par $u$. Alors $u$ induit deux endomorphismes :
    \begin{itemize}
      \item $u_{|F} : F \rightarrow F$ la restriction de $u$ à $F$.
      \item $\overline{u} : E/F \rightarrow E/F$ obtenu par passage au quotient.
    \end{itemize}
  \end{proposition}

  \reference{163}

  \begin{definition}
    Soit $A$ la matrice de l'endomorphisme $u$ dans une base quelconque de $E$. On définit le \textbf{polynôme caractéristique} de $u$ par $\chi_u = \det{(X I_n - A)}$.
  \end{definition}

  \reference{158}

  \begin{proposition}
    Soit $F$ un sous-espace de $E$ stable par $u$ de dimension $r$. Soit $\mathcal{B} = (e_1, \dots, e_n)$ une base de $E$ telle que les $r$ premiers vecteurs forment une base $\mathcal{B}_F$ de $F$. Alors :
    \begin{enumerate}[label=(\roman*)]
      \item La matrice de $u$ dans la base $\mathcal{B}$ est de la forme
      \[ \operatorname{Mat}(u, \mathcal{B}) = \begin{pmatrix} A & C \\ 0 & B \end{pmatrix} \]
      \item $\mathcal{B}_{E/F} = \pi_F(\mathcal{B} \setminus \mathcal{B}_F)$ est une base de $E/F$ où $\pi_F : E \rightarrow E/F$ désigne la projection canonique sur le quotient.
      \item $A = \operatorname{Mat}(u_{|F}, \mathcal{B}_F)$ et $B = \operatorname{Mat}(\overline{u}, \mathcal{B}_{E/F})$.
      \item $\chi_u = \chi_{u_{|F}} \chi_{\overline{u}}$.
    \end{enumerate}
  \end{proposition}

  \subsubsection{Sous-espaces stables et polynôme minimal}

  \reference{161}

  \begin{proposition}
    Il existe un polynôme qui engendre l'idéal $\{ P \in \mathbb{K}[X] \mid P(u) = 0 \}$. Il s'agit du \textbf{polynôme minimal} de $u$ noté $\pi_u$.
  \end{proposition}

  \begin{theorem}[Cayley-Hamilton]
    \[ \pi_u \mid \chi_u \]
  \end{theorem}

  \begin{proposition}
    Soit $F$ un sous-espace de $E$ stable par $u$. Alors $\pi_{u_{|F}} \mid \pi_u$.
  \end{proposition}

  \begin{proposition}
    Si $E = F_1 \oplus F_2$ avec $F_1$ et $F_2$ deux sous-espaces stables par $u$, alors $\pi_u = \operatorname{ppcm}(\pi_{u_{|F_1}}, \pi_{u_{|F_2}})$.
  \end{proposition}

  \begin{proposition}
    Soient $P$ et $Q$ deux polynômes unitaires tels que $\pi_u = PQ$. On note $F = \ker{(P(u))}$. Alors $\pi_{u_{|F}} = P$.
  \end{proposition}

  \subsubsection{Recherche de sous-espaces stables}

  \reference[GOU21]{201}

  \begin{definition}
    On suppose que le polynôme caractéristique de $u$ est scindé sur $\mathbb{K}$ :
    \[ \chi_u = \prod_{i=1}^p (X - \lambda_i)^{\alpha_i} \text{ où les } \lambda_i \text{ sont distincts deux-à-deux} \]
    Pour tout $i \in \llbracket 1, p \rrbracket$, le sous-espace vectoriel $N_i = \ker{(f - \lambda_i \operatorname{id}_E)^{\alpha_i}}$ s'appelle le \textbf{sous-espace caractéristique} de $f$ associé à $\lambda_i$.
  \end{definition}

  \reference{185}

  \begin{proposition}[Lemme des noyaux]
    Soient $P_1, \dots, P_r \in \mathbb{K}[X]$ premiers entre eux. Alors
    \[ \bigoplus_{i=1}^r \ker{(P_i(u))} = \ker \left( \left( \prod_{i=1}^r P_i \right) \left ( u \right) \right) \]
  \end{proposition}

  \reference{202}

  \begin{proposition}
    On suppose que le polynôme caractéristique de $u$ est scindé sur $\mathbb{K}$. On note $N_1, \dots, N_p$ les sous-espaces caractéristiques de $u$.
    \begin{itemize}
      \item $\forall i \in \llbracket 1, p \rrbracket$, $N_i$ est stable par $u$.
      \item $E = N_1 \oplus \dots \oplus N_p$.
      \item $\forall i \in \llbracket 1, p \rrbracket$, $\dim{N_i} = \alpha_i$ où $\alpha_i$ est la multiplicité de $\lambda_i$ dans $\chi_u$.
    \end{itemize}
  \end{proposition}

  \reference[BMP]{159}

  \begin{remark}
    Plus généralement, $\forall \lambda \in \mathbb{K}, \, \forall i \in \mathbb{N}, \, \ker{(u - \lambda \operatorname{id}_E)^i}$ est stable par $u$. C'est en fait un corollaire de la proposition suivante.
  \end{remark}

  \begin{proposition}
    Soient $u, v \in \mathcal{L}(E)$ tels que $uv = vu$ (pour la composition). Alors le noyau et l'image de $v$ sont stables par $u$ (et réciproquement).
  \end{proposition}

  \reference[GOU21]{202}

  \begin{proposition}
    On suppose que le polynôme caractéristique de $u$ est scindé sur $\mathbb{K}$.
    \[ \chi_u = \prod_{i=1}^p (X - \lambda_i)^{\alpha_i} \text{ où les } \lambda_i \text{ sont distincts deux-à-deux} \]
    Alors :
    \begin{enumerate}[label=(\roman*)]
      \item $\pi_u$ est de la forme :
      \[ \pi_u = \prod_{i=1}^p (X - \lambda_i)^{r_i} \text{ où les } \lambda_i \text{ sont distincts deux-à-deux} \]
      \item $\forall i \in \llbracket 1, p \rrbracket$, $N_i = \ker{(f - \lambda_i \operatorname{id}_E )^{r_i}}$.
      \item $\forall i \in \llbracket 1, p \rrbracket$, $r_i$ est l'indice de nilpotence de l'endomorphisme $f_{|N_i} - \lambda_i \operatorname{id}_{N_i}$.
    \end{enumerate}
  \end{proposition}

  \subsubsection{Utilisation de la dualité}

  \reference{132}

  \begin{definition}
    On appelle \textbf{forme linéaire} de $E$ toute application linéaire de $E$ dans $\mathbb{K}$ et on note $E^*$ appelé \textbf{dual} de $E$ l'ensemble des formes linéaires de $E$.
  \end{definition}

  \begin{proposition}
    $E^*$ est un espace vectoriel sur $\mathbb{K}$ de dimension $n$.
  \end{proposition}

  \begin{definition}
    Si $A \subset E$, on note $A^\perp = \{ \varphi \in E^* \mid \forall x \in A, \, \varphi(x) = 0 \}$ l'\textbf{orthogonal} (au sens de la dualité) de $A$ qui est un sous-espace vectoriel de $E^*$.
  \end{definition}

  \begin{proposition}
    Si $F$ est un sous-espace vectoriel de $E$, on a $\dim F + \dim F^\perp = \dim E$.
  \end{proposition}

  \begin{definition}
    On définit $\tr{u} : E^* \rightarrow E^*$ l'\textbf{application transposée} de $u$ par
    \[ \forall \varphi \in E^*, \, \tr{u} (\varphi) = \varphi \circ u \]
  \end{definition}

  \begin{proposition}
    Un sous-espace vectoriel $F$ de $E$ est stable par $u$ si et seulement si $F^\perp$ est stable par $\tr{u}$.
  \end{proposition}

  \begin{remark}
    C'est un résultat qui peut s'avérer utile dans les démonstrations par récurrence s'appuyant sur la dimension d'un sous-espace stable (cf. \cref{151-1}).
  \end{remark}

  \subsection{Application à la réduction d'endomorphismes}

  \subsubsection{Diagonalisation et trigonalisation}

  \reference{171}

  \begin{definition}
    On dit que $\lambda \in \mathbb{K}$ est \textbf{valeur propre} de $u$ s'il existe $x \neq 0$ tel que $u(x) = \lambda x$. $x$ est alors un \textbf{vecteur propre} de $u$ associé à $\lambda$. Le sous-espace
    \[ E_\lambda = \{ x \in E \mid u(x) = \lambda x \} = \ker{(u - \lambda \operatorname{Id})} \]
    est le \textbf{sous-espace propre} associé à $\lambda$.
  \end{definition}

  \begin{definition}
    On dit que $u$ est \textbf{diagonalisable} (resp. \textbf{trigonalisable}) s'il existe une base $\mathcal{B}$ de $E$ telle que $\operatorname{Mat}(u, \mathcal{B})$ soit diagonale (resp. triangulaire supérieure).
  \end{definition}

  \reference[BMP]{165}

  \begin{theorem}
    Les assertions suivantes sont équivalentes :
    \begin{enumerate}[label=(\roman*)]
      \item $u$ est diagonalisable.
      \item $\pi_u$ est scindé à racines simples.
      \item $\chi_u$ est scindé et, pour toute valeur propre $\lambda$, la dimension du sous-espace propre $E_\lambda$ est égale à la multiplicité de $\lambda$ dans $\chi_u$.
      \item $E$ est somme directe des sous-espaces propres de $u$.
    \end{enumerate}
  \end{theorem}

  \begin{example}
    \begin{itemize}
      \item Soit $p \in \mathcal{L}(E)$ tel que $p^2 = p$. Alors $p$ est annulé par $X^2 - X$ donc est diagonalisable et à valeurs propres dans $\{ 0, 1 \}$.
      \item Soit $s \in \mathcal{L}(E)$ tel que $s^2 = \operatorname{id}_E$. Alors si $\operatorname{car}(\mathbb{K}) \neq 2$, $s$ est annulé par $X^2 - 1$ donc est diagonalisable et à valeurs propres dans $\{ \pm 1 \}$.
    \end{itemize}
  \end{example}

  \begin{theorem}
    Les assertions suivantes sont équivalentes :
    \begin{enumerate}[label=(\roman*)]
      \item $u$ est trigonalisable.
      \item $\pi_u$ est scindé.
      \item $\chi_u$ est scindé.
    \end{enumerate}
  \end{theorem}

  \begin{example}
    Si $\mathbb{K}$ est algébriquement clos, tout endomorphisme de $E$ est trigonalisable.
  \end{example}

  \dev{trigonalisation-simultanee}

  \begin{theorem}
    \label{151-1}
    Soit $(u_i)_{i \in I}$ une famille d'endomorphismes telle que $\forall i, j \in I, \, u_i u_j = u_j u_i$. Si tous les $u_i$ sont trigonalisables (resp. diagonalisables), on peut co-trigonaliser (resp. co-diagonaliser) la famille $(u_i)_{i \in I}$.
  \end{theorem}

  \begin{remark}
    Dans le cas de la diagonalisabilité, cette condition est à la fois nécessaire et suffisante.
  \end{remark}

  \reference[GOU21]{174}

  \begin{proposition}
    On suppose que $u$ est diagonalisable. Soit $F$ un sous-espace de $E$ stable par $u$. Alors $u_{|F}$ est diagonalisable.
  \end{proposition}

  \reference[BMP]{170}

  \begin{application}
    Les assertions suivantes sont équivalentes :
    \begin{enumerate}[label=(\roman*)]
      \item $u$ est trigonalisable avec des zéros sur la diagonale.
      \item $u$ est nilpotent (ie. $\exists m \in \mathbb{N}$ tel que $u^m = 0$).
      \item $\chi_u = X^n$.
      \item $\pi_u = X^p$ où $p$ est l'indice de nilpotence de $u$.
    \end{enumerate}
  \end{application}

  \subsubsection{Décomposition de Dunford}

  \reference[GOU21]{203}
  \dev{decomposition-de-dunford}

  \begin{theorem}[Décomposition de Dunford]
    On suppose que $\pi_u$ est scindé sur $\mathbb{K}$. Alors il existe un unique couple d'endomorphismes $(d, n)$ tels que :
    \begin{itemize}
      \item $d$ est diagonalisable et $n$ est nilpotent.
      \item $u = d + n$.
      \item $d n = n d$.
    \end{itemize}
  \end{theorem}

  \begin{corollary}
    Si $u$ vérifie les hypothèse précédentes, pour tout $k \in \mathbb{N}$, $u^k = (d + n)^k = \sum_{i=0}^m \binom{k}{i} d^i n^{k-i}$, avec $m = \min(k, l)$ où $l$ désigne l'indice de nilpotence de $n$.
  \end{corollary}

  \begin{remark}
    \begin{itemize}
      \item Un autre intérêt est le calcul d'exponentielles de matrices.
      \item On peut montrer de plus que $d$ et $n$ sont des polynômes en $u$.
    \end{itemize}
  \end{remark}

  \subsubsection{Réduction de Jordan}

  \reference[BMP]{171}

  \begin{definition}
    Un \textbf{bloc de Jordan} de taille $m$ associé à $\lambda \in \mathbb{K}$ désigne la matrice $J_m(\lambda)$ suivante :
    \[ J_m(\lambda) = \begin{pmatrix} \lambda & 1 & \\ & \ddots & \ddots & \\ & & \ddots & 1 \\ & & & \lambda \end{pmatrix} \in \mathcal{M}_m(\mathbb{K}) \]
  \end{definition}

  \begin{proposition}
    Les assertions suivantes sont équivalentes :
    \begin{enumerate}[label=(\roman*)]
      \item Il existe une base de $E$ telle que la matrice de $u$ est $J_n(0)$.
      \item $u$ est nilpotent et cyclique (voir \cref{151-2}).
      \item $u$ est nilpotent d'indice de nilpotence $n$.
    \end{enumerate}
  \end{proposition}

  \begin{theorem}[Réduction de Jordan d'un endomorphisme nilpotent]
    On suppose que $u$ est nilpotent. Alors il existe des entiers $n_1 \geq \dots \geq n_p$ et une base $\mathcal{B}$ de $E$ tels que :
    \[ \operatorname{Mat}(u, \mathcal{B}) = \begin{pmatrix} J_{n_1}(0) & & \\ & \ddots & \\ & & J_{n_p}(0) \end{pmatrix} \]
    De plus, on a unicité dans cette décomposition.
  \end{theorem}

  \begin{remark}
    Comme l'indice de nilpotence d'un bloc de Jordan est égal à sa taille, l'indice de nilpotence de $u$ est la plus grande des tailles des blocs de Jordan de la réduite.
  \end{remark}

  \reference[GOU21]{209}

  \begin{theorem}[Réduction de Jordan d'un endomorphisme]
    On suppose que le polynôme caractéristique de $u$ est scindé sur $\mathbb{K}$ :
    \[ \chi_u = \prod_{i=1}^p (X - \lambda_i)^{\alpha_i} \text{ où les } \lambda_i \text{ sont distincts deux-à-deux} \]
    Alors il existe des entiers $n_1 \geq \dots \geq n_p$ et une base $\mathcal{B}$ de $E$ tels que :
    \[ \operatorname{Mat}(u, \mathcal{B}) = \begin{pmatrix} J_{n_1}(\lambda_1) & & \\ & \ddots & \\ & & J_{n_p}(\lambda_p) \end{pmatrix} \]
    De plus, on a unicité dans cette décomposition.
  \end{theorem}

  \subsubsection{Réduction de Frobenius}

  \reference{397}

  \begin{definition}
    \label{151-2}
    On dit que $u$ est \textbf{cyclique} s'il existe $x \in E$ tel que $\{ P(u)(x) \mid P \in \mathbb{K}[X] \} = E$.
  \end{definition}

  \begin{proposition}
    $u$ est cyclique si et seulement si $\deg(\pi_u) = n$.
  \end{proposition}

  \begin{definition}
    Soit $P = X^p + a_{p-1} X^{p-1} + \dots + a_0 \in \mathbb{K}[X]$. On appelle \textbf{matrice compagnon} de $P$ la matrice
    \[ \mathcal{C}(P) = \begin{pmatrix} 0 & \dots & \dots & 0 & -a_0 \\ 1 & 0 & \ddots & \vdots & -a_1 \\ 0 & 1 & \ddots & \vdots & \vdots \\ \vdots & \ddots & \ddots & 0 & -a_{p-2} \\ 0 & \dots & 0 & 1 & -a_{p-1} \end{pmatrix} \]
  \end{definition}

  \begin{proposition}
    $u$ est cyclique si et seulement s'il existe une base $\mathcal{B}$ de $E$ telle que $\operatorname{Mat}(u, \mathcal{B}) = \mathcal{C}(\pi_u)$.
  \end{proposition}

  \begin{theorem}
    Il existe $F_1, \dots, F_r$ des sous-espaces vectoriels de $E$ tous stables par $u$ tels que :
    \begin{itemize}
      \item $E = F_1 \oplus \dots \oplus F_r$.
      \item $u_i = u_{|F_i}$ est cyclique pour tout $i$.
      \item Si $P_i = \pi_{u_i}$, on a $P_{i+1} \mid P_i$ pour tout $i$.
    \end{itemize}
    La famille de polynômes $P_1, \dots, P_r$ ne dépend que de $u$ et non du choix de la décomposition. On l'appelle \textbf{suite des invariants de similitude} de $u$.
  \end{theorem}

  \begin{theorem}[Réduction de Frobenius]
    Si $P_1, \dots, P_r$ désigne la suite des invariants de $u$, alors il existe une base $\mathcal{B}$ de $E$ telle que :
    \[ \operatorname{Mat}(u, \mathcal{B}) = \begin{pmatrix} \mathcal{C}(P_1) & & \\ & \ddots & \\ & & \mathcal{C}(P_r) \end{pmatrix} \]
    On a d'ailleurs $P_1 = \pi_u$ et $P_1 \dots P_r = \chi_u$.
  \end{theorem}

  \begin{corollary}
    Deux endomorphismes de $E$ sont semblables si et seulement s'ils ont la même suite d'invariants de similitude.
  \end{corollary}

  \begin{application}
    Toute matrice est semblable à sa transposée.
  \end{application}

  \subsection{Endomorphismes remarquables}

  \subsubsection{Endomorphismes normaux}

  Soit $E$ un espace vectoriel sur $\mathbb{C}$ de dimension finie $n$. On munit $E$ d'un produit scalaire $\langle . , . \rangle$, qui en fait un espace hermitien.

  \reference[GRI]{286}

  \begin{notation}
    On note $u^*$ \textbf{l'adjoint} de $u$.
  \end{notation}

  \begin{definition}
    Un endomorphisme $u \in \mathcal{L}(E)$ est dit \textbf{normal} s'il est tel que $u \circ u^* = u^* \circ u$.
  \end{definition}

  \begin{proposition}
    On suppose $u$ normal. Soit $\lambda \in \mathbb{C}$ une valeur propre de $u$. Alors :
    \begin{enumerate}[label=(\roman*)]
      \item $E_\lambda^\perp = \{ x \in E^\lambda \mid \forall y \in E^\lambda, \, \langle x, y \rangle = 0 \}$ est stable par $u$.
      \item $u_{| E_\lambda^\perp}$ est normal.
    \end{enumerate}
  \end{proposition}

  \begin{corollary}
    On suppose $u$ normal. Alors $u$ est diagonalisable dans une base orthonormée.
  \end{corollary}

  \subsubsection{Sous-représentations}

  \reference[ULM21]{144}

  Soit $G$ un groupe d'ordre fini.

  \begin{definition}
    \begin{itemize}
      \item Une \textbf{représentation linéaire} $\rho$ est un morphisme de $G$ dans $\mathrm{GL}(V)$ où $V$ désigne un espace-vectoriel de dimension finie $n$ sur $\mathbb{C}$.
      \item On dit que $n$ est le \textbf{degré} de $\rho$.
      \item On dit que $\rho$ est \textbf{irréductible} si $V \neq \{ 0 \}$ et si aucun sous-espace vectoriel de $V$ n'est stable par $\rho(g)$ pour tout $g \in G$, hormis $\{ 0 \}$ et $V$.
    \end{itemize}
  \end{definition}

  \begin{example}
    Soit $\varphi : G \rightarrow S_n$ le morphisme structurel d'une action de $G$ sur un ensemble de cardinal $n$. On obtient une représentation de $G$ sur $\mathbb{C}^n = \{ e_1, \dots, e_n \}$ en posant
    \[ \rho(g)(e_i) = e_{\varphi(g)(i)} \]
    c'est la représentation par permutations de $G$ associé à l'action. Elle est de degré $n$.
  \end{example}

  \begin{definition}
    La représentation par permutations de $G$ associée à l'action par translation à gauche de $G$ sur lui-même est la \textbf{représentation régulière} de $G$, on la note $\rho_G$.
  \end{definition}

  \begin{definition}
    Soit $\rho : G \rightarrow \mathrm{GL}(V)$ une représentation linéaire de $G$. On suppose $V = W \oplus W_0$ avec $W$ et $W_0$ stables par $\rho(g)$ pour tout $g \in G$. On dit alors que $\rho$ est \textbf{somme directe} de $\rho_W$ et de $\rho_{W_0}$.
  \end{definition}

  \begin{theorem}[Maschke]
    Toute représentation linéaire de $G$ est somme directe de représentations irréductibles.
  \end{theorem}

  \annexessection

  \reference[BMP]{158}

  \begin{figure}[h]
    \begin{center}
      \begin{tikzpicture}
        \node[draw=none](a) at (0,2) {$F$};
        \node[draw=none](b) at (0,0) {$F$};
        \node[draw=none](c) at (2,0) {$E$};
        \node[draw=none](d) at (2,2) {$E$};
        \node[draw=none](e) at (4,0) {$E/F$};
        \node[draw=none](f) at (4,2) {$E/F$};
        \node[draw=none] at (3.4,1) {$\overline{u}$};
        \node[draw=none] at (1.4,1) {$u$};
        \node[draw=none] at (-0.6,1) {$u_{|F}$};
        \draw[right hook->] (a)--(d);
        \draw[right hook->] (b)--(c);
        \draw[->>] (c)--(e);
        \draw[->>] (d)--(f);
        \draw[->] (a)--(b);
        \draw[->] (d)--(c);
        \draw[->] (f)--(e);
      \end{tikzpicture}
    \end{center}
    \caption{Endomorphismes induits par $u$ sur un sous-espace stable $F$.}
  \end{figure}

  \reference{157}

  \begin{figure}[h]
    \begin{whitetabularx}{|X|X|X|X|}
      \hline
      $u$ & Diagonalisable & Trigonalisable & Quelconque \\
      \hline
      Décomposition & de $E$ suivant les vecteurs propres & de Dunford & de Frobenius \\
      \hline
      Sous-espace stable $F$ & espace propre & espace caractéristique & engendré par un élément \\
      \hline
      $u_{|F}$ & homothétie & homothétie + nilpotent & cyclique \\
      \hline
    \end{whitetabularx}
    \caption{Réduction d'un endomorphisme en fonction de ses propriétés.}
  \end{figure}
  %</content>
\end{document}
