\input{../common}
\everymath{\displaystyle}
\begin{document}
  %<*content>
  \lesson{algebra}{7}{Fonctions puissances d'exposant rationnel}
\subsection{Définition  et propriétés}

\begin{definition}
 $ r $ étant un nombre rationnel non nul, on appelle \emph{fonction puissance d'exposant r,} la fonction : 
\[\begin{array}{lrcl}
    &  \Rrp  & \longrightarrow & \Rrp \\ 
  &  x & \longmapsto & x^{r}
   \end{array}\]
\end{definition}
\begin{notation}
$\hspace*{1cm} x^{r}=\eexp{r \ln x} $
\end{notation}
\begin{property}
\begin{enumerate}
\item $ r$ et $r' $ étant des nombres rationnels non nuls, $x $ et $ y$ des réels strictement positifs.
\begin{itemize}
\item $ x^{r}\times y^{r}= (xy)^{r} $ 
\item  $ (x^{r})^{r'}=x^{rr'} $

\item $ \dfrac{x^{r}}{y^{r}}=(\dfrac{x}{y})^{r} $
\item   $ x^{r}\times x^{r'}=x^{r+r'} $
 \end{itemize}
 \end{enumerate}
 \end{property}
 
\subsection{Dérivabilité}
La fonction $ x\rightarrow x^{r} $  est dérivable sur $ \Rrep $ car composée des deux fonctions dérivables :  $ x \longmapsto r \ln x \;$ et  $\; x\longmapsto \eexp{x} $. \\ On a : $ \;\forall  x >0,\quad \paren{x^{r}}^{\prime} =\paren{\eexp{r \ln x}}^{\prime} =\dfrac{r}{x}\times \eexp{r \ln x} =\dfrac{r}{x}\times x^{r}  =r x^{r-1}$.\\  Ainsi  la fonction \;$ x \longmapsto  x^{r}  $\; est dérivable sur \;$ \intoo{0}{\pinf} $\;   et sa dérivée est la fonction\; $ x\longmapsto r x^{r -1} $. 
 \subsection{Fonction composée}
Soit $r $ un rationnel  et $u$  une fonction strictement positive sur un intervalle I. \\
La fonction \; $ x\longmapsto \paren{u(x)}^{r} $\; est la composée de la fonction $ x\longmapsto u(x) \;$  suivie de la fonction  \; $ x \longmapsto  x^{r} $.\\
 De plus on a:\; $ \paren{u(x)}^{r}=\eexp{r\ln (u(x))} $\\
On en déduit la propriété suivante.

\begin{property}
Soit $r $ un rationnel et  $u$  une fonction  dérivable et strictement positive sur un intervalle I. \\La fonction \;$ u^{r} $\; est dérivable sur I et on a:\; $\paren{ u^{r}}^{\prime}=r u^{\prime}u^{r -1} $
\end{property}
 \begin{example}
 $ f(x)= \sqrt[3]{2x+1} $\\
 $ f(x) $  existe $ \Leftrightarrow 2x+1 \Leftrightarrow x \geq -\frac{1}{2} $ donc $ f $ est dérivable sur $ \intoo{-\frac{1}{2}}{\pinf} $\\
  $ f(x)= \sqrt[3]{2x+1}=(2x+1)^{\frac{1}{3}}\Longrightarrow f'(x)=\frac{2}{3}(2x+1)^{-\frac{2}{3}}=\dfrac{2}{3\sqrt[3]{(2x+1)^2}} $
 \end{example}

\begin{corollary}
Soit $r $ un rationnel  différent de \;$ -1 $\;   et  u  une fonction  dérivable et  strictement positive sur un intervalle I. \\La fonction  $ \;                                       u^{\prime}u^{r} $\;   admet pour primitives sur I les fonctions   $ \;                                      \dfrac{u^{r +1}}{r+1} +C$.

\end{corollary}
\begin{example}
La fonction $ \; x\longmapsto 2x \paren{3-x^{2}}^{\frac{1}{4}} $   admet pour primitives sur $ \;\intoo{-\sqrt{3}}{\sqrt{3}} \;$ les fonctions  $ x\longmapsto -\dfrac{\paren{3-x^{2}}^{\frac{5}{4}}}{\frac{5}{4}} +C$
\end{example}


\subsection{Croissances comparées}
\begin{property}
Soit $\alpha $ un rationnel    strictement positif.  On a :  
\begin{itemize}
\item  $ \displaystyle\lim_{x \to \pinf} \dfrac{\ln x}{x^{\alpha}}=0$ 

%\vspace*{0.5cm}
\item  $\displaystyle \lim_{x \to 0^{+}} x^{\alpha}\ln x=0$

%\vspace*{0.5cm}
 \item  $\displaystyle \lim_{x \to \pinf} \frac{\eexp{x}}{x^{\alpha}}=\pinf$
 
% \vspace*{0.5cm}
 \item  $\displaystyle \lim_{x \to \pinf} x^{\alpha}\eexp{-x}=0$ 
\end{itemize}
\end{property}
\begin{remark}
Lorsqu'on ne peut pas conclure directement, on peut conjecturer la limite d'une fonction comportant des  fonctions logarithmes, puissances ou exponentielles en remarquant que :
\begin{itemize}
\item la fonction exponentielle <<  l'emporte >> sur la fonction puissance.
\item la fonction puissance <<  l'emporte >> sur la fonction logarithme.
\end{itemize}
\end{remark}


\textbf{Démonstration}
\begin{itemize}
\item Posons:\;$ X=x^{\alpha} $\quad on a :\;    $ \displaystyle\lim_{x \to \pinf} \dfrac{\ln x}{x^{\alpha}}  =\displaystyle \lim_{x \to \pinf} \dfrac{1}{\alpha}\dfrac{\ln x^{\alpha}}{x^{\alpha}} =\displaystyle\lim_{X \to \pinf}\dfrac{1}{\alpha} \dfrac{\ln X}{X}=0$

\item Posons:\;$ X=x^{\alpha} $ on a :
\;$ \displaystyle\lim_{x \to 0^{+}} x^{\alpha}\ln x=\displaystyle\lim_{x \to 0^{+}}\dfrac{1}{\alpha} x^{\alpha}\ln x^{\alpha} =\displaystyle\lim_{X \to 0^{+}} \dfrac{1}{\alpha}X\ln X=0$ 
\item  Posons:\;$ X=\dfrac{x}{\alpha} $\quad on a :                                   \; $ \displaystyle\lim_{x \to \pinf} \dfrac{\eexp{x}}{x^{\alpha}}= \displaystyle \lim_{x \to \pinf} \dfrac{1}{\alpha}\paren{\dfrac{\eexp{\dfrac{x}{\alpha}}}{\dfrac{x}{\alpha}}}^{\alpha} =\lim_{X\to \pinf} \dfrac{1}{\alpha}\paren{\dfrac{\eexp{X}}{X}}^{\alpha}= \pinf$
\item On a:\; $ \displaystyle\lim_{x \to \pinf} x^{\alpha}\eexp{-x}=\displaystyle\lim_{x \to \pinf}\dfrac{x^{\alpha}}{\eexp{x}} =\displaystyle\lim_{x \to \pinf}\dfrac{1}{\dfrac{\eexp{x}}{x^{\alpha}}}= 0$

\end{itemize}


\begin{example}

\begin{itemize}
\item $\displaystyle \lim_{x \to \pinf} \sqrt{x}-\ln x =\displaystyle\lim_{x \to \pinf}x^{\frac{1}{2}}\croch{1-\dfrac{\ln x}{x^{\frac{1}{2}}}}$,   or $\displaystyle \displaystyle\lim_{x \to \pinf}\paren{1-\dfrac{\ln x}{x^{\frac{1}{2}}}} =1-0=1 $     et $ \displaystyle\lim_{x \to \pinf}x^{\frac{1}{2}}=\pinf $     donc $\displaystyle\lim_{x \to \pinf} \sqrt{x}-\ln x=\pinf  $  
\item $\displaystyle\lim_{x \to \pinf}\eexp{x} -\ln x =\displaystyle\lim_{x \to \pinf}\ln x\paren{\dfrac{\eexp{x}}{\ln x} -1}$     or $\displaystyle \lim_{x \to \pinf}\ln x=\pinf $    et $\displaystyle \lim_{x \to \pinf}\dfrac{\eexp{x}}{\ln x}= \displaystyle \lim_{x \to \pinf}\dfrac{\eexp{x}}{ x}\times\frac{x}{\ln x} =\pinf$ \\ Par suite $\displaystyle \lim_{x \to \pinf}\ln x\paren{\dfrac{\eexp{x}}{\ln x} -1}=\pinf $ 
\item $\displaystyle \lim_{n \to \pinf} \sqrt[n]{n} $. \;    On a:\;$\sqrt[n]{n}=n^{\frac{1}{n}}  =\eexp{\frac{1}{n}\ln n} $.\;   Or $ \displaystyle\lim_{n \to \pinf}\dfrac{\ln n}{n} =0$\;     par  composée , $ \displaystyle \lim_{n \to \pinf} \eexp{\frac{1}{n}\ln n}=\eexp{0}=1$   il vient: $ \displaystyle\lim_{n \to \pinf} \sqrt[n]{n}=1 $  
%\item  $\displaystyle \lim_{x \to \pinf} 2^{x}-x^{100} =\displaystyle\lim_{x \to \pinf}2^{x}\paren{1-\dfrac{x^{100}}{2^{x}}}$.
%Or $ \displaystyle\lim_{x \to \pinf} 2^{x}=\pinf $  \; et \; $\displaystyle \lim_{x \to \pinf}\paren{1-\dfrac{x^{100}}{2^{x}}} = 1$,\;    on obtient \; $ \displaystyle\lim_{x \to \pinf} 2^{x}-x^{100} =\pinf $

\end{itemize}

\end{example}
  %</content>
\end{document}
