\input{../common}
\everymath{\displaystyle}
\begin{document}
  %<*content>
  \lesson{algebra}{6}{Fonctions puissances}



\subsection{Définition  et propriétés}

\begin{definition}
Soit $\alpha $ un réel\\
On appelle  fonction puissance d'exposant $\alpha $,  la fonction $ f_{\alpha} $ définie par  :
\begin{center}
 $ \begin{array}{lrcl}
     & \Rrep  &   \longrightarrow &  \Rrep \\ 
  &  x & \longmapsto & x^{\alpha}=\eexp{\alpha \ln x}  \end{array} $ 
\end{center}
\end{definition}

\begin{property}
La fonction $ f_{\alpha} $  est dérivable sur $ \Rrep $ car composée de deux fonctions dérivables:  \\ $ x \longmapsto \alpha \ln x $\; et \; $ x\longmapsto \eexp{x} $.  On a :\; $ \forall  x >0,\quad f_{\alpha}^{\prime} (x) =\dfrac{\alpha}{x} \eexp{\alpha \ln x} =\dfrac{\alpha}{x}\times x^{\alpha}  =\alpha x^{\alpha-1}$.\\  Ainsi  la fonction \;$ x \longmapsto  x^{\alpha}  $\; est dérivable sur \;$ \intoo{0}{\pinf} $\;   et sa dérivée est la fonction\; $ x\longmapsto \alpha x^{\alpha -1} $. 

\end{property}

  \fbox{$\paren{x^{\alpha}}^{\prime}=\alpha x^{\alpha-1} $}
 \subsection{Fonction composée}

Soit $\alpha $ un réel  et $u$  une fonction strictement positive sur un intervalle I. \\
La fonction \; $ x\longmapsto \paren{u(x)}^{\alpha} $\; est la composée de la fonction $ x\longmapsto u(x) $  suivie de la fonction  \; $ x \longmapsto  x^{\alpha} $. De plus on a:\; $ \paren{u(x)}^{\alpha}=\eexp{\alpha\ln (u(x))} $\\
On en déduit les propriétés suivantes.

\begin{property}
Soit $\alpha $ un réel et  $u$  une fonction  dérivable et strictement positive sur un intervalle I. \\La fonction \;$ u^{\alpha} $\; est dérivable sur I et on a:\; $\paren{ u^{\alpha}}^{\prime}=\alpha u^{\prime}u^{\alpha -1} $
\end{property}
\begin{example}
La fonction \;$ x\longmapsto  \paren{\cos x}^{\sqrt{2}} $  est dérivable sur \;$ \intfo{0}{\frac{\pi}{2}} $\;   et sa dérivée est la \\ fonction \;$ x\longmapsto -\sqrt{2}\sin x \paren{\cos x}^{\sqrt{2}-1}$

\end{example}

\begin{property}
Soit $\alpha $ un réel  différent de \;$ -1 $\;   et  u  une fonction  dérivable et  strictement positive sur un intervalle I. \\La fonction \;                                       $ u^{\prime}u^{\alpha} $\;   admet pour primitive sur I la fonction \;  $ \dfrac{u^{\alpha +1}}{\alpha+1} $.
\end{property}
\begin{example}
La fonction \;$ x\longmapsto 2x \paren{3-x^{2}}^{\pi} $   admet pour primitive sur $ \intoo{-\sqrt{3}}{\sqrt{3}} $\;    la fonction \\ $ x\longmapsto -\dfrac{\paren{3-x^{2}}^{\pi+1}}{\pi+1} $
\end{example}


\subsection{Croissances comparées}
\begin{property}
Soit $\alpha $ un réel    strictement positif.  On a :  
\begin{itemize}
\item  $ \displaystyle\lim_{x \to \pinf} \frac{\ln x}{x^{\alpha}}=0$ 

%\vspace*{0.5cm}
\item  $ \lim_{x \to 0^{+}} x^{\alpha}\ln x=0$

%\vspace*{0.5cm}
 \item  $\displaystyle \lim_{x \to \pinf} \frac{\eexp{x}}{x^{\alpha}}=\pinf$
 
% \vspace*{0.5cm}
 \item  $\displaystyle \lim_{x \to \pinf} x^{\alpha}\eexp{-x}=0$ 
\end{itemize}
\end{property}
\begin{remark}
Lorsqu'on ne peut pas conclure directement, on peut conjecturer la limite d'une fonction comportant des  fonctions logarithmes, puissances ou exponentielles en remarquant que :
\begin{itemize}
\item la fonction exponentielle <<  l'emporte >> sur la fonction puissance.
\item la fonction puissance <<  l'emporte >> sur la fonction logarithme.
\end{itemize}
\end{remark}


\textbf{Démonstration}
\begin{itemize}
\item Posons:\;$ X=x^{\alpha} $\quad on a :\;    $ \displaystyle\lim_{x \to \pinf} \dfrac{\ln x}{x^{\alpha}}  =\displaystyle \lim_{x \to \pinf} \dfrac{1}{\alpha}\dfrac{\ln x^{\alpha}}{x^{\alpha}} =\displaystyle\lim_{X \to \pinf}\dfrac{1}{\alpha} \dfrac{\ln X}{X}=0$

\item Posons:\;$ X=x^{\alpha} $ on a :
\;$ \displaystyle\lim_{x \to 0^{+}} x^{\alpha}\ln x=\displaystyle\lim_{x \to 0^{+}}\dfrac{1}{\alpha} x^{\alpha}\ln x^{\alpha} =\displaystyle\lim_{X \to 0^{+}} \dfrac{1}{\alpha}X\ln X=0$ 
\item  Posons:\;$ X=\dfrac{x}{\alpha} $\quad on a :                                   \; $ \displaystyle\lim_{x \to \pinf} \dfrac{\eexp{x}}{x^{\alpha}}= \displaystyle \lim_{x \to \pinf} \dfrac{1}{\alpha}\paren{\dfrac{\eexp{\dfrac{x}{\alpha}}}{\dfrac{x}{\alpha}}}^{\alpha} =\lim_{X\to \pinf} \dfrac{1}{\alpha}\paren{\dfrac{\eexp{X}}{X}}^{\alpha}= \pinf$
\item On a:\; $ \displaystyle\lim_{x \to \pinf} x^{\alpha}\eexp{-x}=\displaystyle\lim_{x \to \pinf}\dfrac{x^{\alpha}}{\eexp{x}} =\displaystyle\lim_{x \to \pinf}\dfrac{1}{\dfrac{\eexp{x}}{x^{\alpha}}}= 0$

\end{itemize}


\begin{example}

\begin{itemize}
\item $\displaystyle \lim_{x \to \pinf} \sqrt{x}-\ln x =\displaystyle\lim_{x \to \pinf}x^{\dfrac{1}{2}}\croch{1-\dfrac{\ln x}{x^{\dfrac{1}{2}}}}$,   or $\displaystyle \displaystyle\lim_{x \to \pinf}\paren{1-\dfrac{\ln x}{x^{\dfrac{1}{2}}}} =1-0=1 $     et $ \displaystyle\lim_{x \to \pinf}x^{\dfrac{1}{2}}=\pinf $     donc $\displaystyle\lim_{x \to \pinf} \sqrt{x}-\ln x=\pinf  $  
\item $\displaystyle\lim_{x \to \pinf}\eexp{x} -\ln x =\displaystyle\lim_{x \to \pinf}\ln x\paren{\dfrac{\eexp{x}}{\ln x} -1}$     or $\displaystyle \lim_{x \to \pinf}\ln x=\pinf $    et $\displaystyle \lim_{x \to \pinf}\dfrac{\eexp{x}}{\ln x}= \displaystyle \lim_{x \to \pinf}\dfrac{\eexp{x}}{ x}\times\frac{x}{\ln x} =\pinf$ \\ Par suite $\displaystyle \lim_{x \to \pinf}\ln x\paren{\dfrac{\eexp{x}}{\ln x} -1}=\pinf $ 
\item $\displaystyle \lim_{n \to \pinf} \sqrt[n]{n} $. \;    On a:\;$\sqrt[n]{n}=n^{\dfrac{1}{n}}  =\eexp{\dfrac{1}{n}\ln n} $.\;   Or $ \displaystyle\lim_{n \to \pinf}\dfrac{\ln n}{n} =0$\;     par  composée , $ \displaystyle \lim_{n \to \pinf} \eexp{\dfrac{1}{n}\ln n}=\eexp{0}=1$   il vient: $ \displaystyle\lim_{n \to \pinf} \sqrt[n]{n}=1 $  
\item  $ \lim_{x \to \pinf} 2^{x}-x^{100} =\displaystyle\lim_{x \to \pinf}2^{x}\paren{1-\dfrac{x^{100}}{2^{x}}}$.
Or $ \displaystyle\lim_{x \to \pinf} 2^{x}=\pinf $  \; et \; $\displaystyle \lim_{x \to \pinf}\paren{1-\dfrac{x^{100}}{2^{x}}} = 1$,\;    on obtient \; $ \displaystyle\lim_{x \to \pinf} 2^{x}-x^{100} =\pinf $

\end{itemize}

\end{example}

  %</content>
\end{document}
