\input{../common}
\everymath{\displaystyle}
\begin{document}
  %<*content>
  \lesson{algebra}{15}{Similitudes directes}
\begin{definition}
Une transformation du plan complexe $ \mathcal{P} $ est une application 
 \[ \begin{array}{lrcl}
  f : & \mathcal{P} &   \longrightarrow & \mathcal{P} \\ 
  &  M & \longmapsto & M'
  \end{array}\quad \text{qui est bijective}.\]
\end{definition}
\medskip

A toute transformation $ f $ du plan complexe, on peut associer une unique application
\[ \begin{array}{lrcl}
  \varphi : & \mathbb{C}&   \longrightarrow & \mathbb{C}\\ 
  &  z& \longmapsto & z'
  \end{array}\] telle que si  $M'( z') = f(M( z ))$, alors \; $\varphi( z ) = z'$.
\medskip

$ \varphi $ , qui est également bijective, est appelée transformation complexe associée à $ f $ .

 La formule   $z'=\varphi (z ) $ \, est appelée écriture ou expression complexe de la transformation $ f $.
\medskip

\begin{example}
\begin{itemize}
\item
La translation, l'homothétie et la rotation sont des transformations du plan.
\item La composée de deux transformations  du plan est une transformation du plan.
\end{itemize}
\end{example}
Voici les écritures complexes de ces transformations du plan.
\subsubsection*{Écriture complexe d'une translation}
\begin{theorem}
La translation de vecteur  $ \overrightarrow{u} $, d'affixe $ a $ ,
transforme un point   $M( z ) $ en un point  $M'( z') $ tel que :  $z'=z +a  $.
\end{theorem}

\textbf{Démonstration}

\medskip

 Dire que $ M' $ est l'image de $ M $ par la translation de vecteur $ \overrightarrow{u} $  revient à dire
que  $ \overrightarrow{M M'} = \overrightarrow{a}$, ce qui se traduit en termes d'affixes par  $z'-z =a  $ soit  $z'=z +a  $. 


 \begin{remark}
 \begin{itemize}
 \item La translation réciproque a pour vecteur $ - \overrightarrow{u} $.
 \item  Ajouter un nombre $ a $ revient géométriquement à translater d'un vecteur
d'affixe $ a $.
 \end{itemize}
  \end{remark}
 \subsubsection*{Écriture complexe d'une homothétie}
\begin{theorem}
 L'homothétie de centre $ \Omega  (\omega ) $ et de rapport $ k\in\mathbb{R} $ transforme un point  $ M( z )$  en un point $ M'( z ')$  tel que : $ z'-\omega =k(z-\omega) $.
\end{theorem}

\textbf{Démonstration}

\medskip

Dire que $ M' $ est l'image de $ M $ par l'homothétie de centre $ \Omega $ et de rapport $ k $, signifie par définition que : $\overrightarrow{ \Omega M'}= k  \overrightarrow{ \Omega M}$, ce qui se traduit en termes d'affixes par : $ z'-\omega =k (z-\omega) $ ou  $ z'=kz +\omega(1-k) $.

\medskip
\begin{example}
Soit $ f $ la transformation du plan qui, à tout point $M(z)$ du plan associe le point  $M'(z')$ tel que : $ z'=-\frac{5}{2}z+2\i $.

\medskip

Montrons d'abord que $ f $ admet un unique point invariant.

 Pour cela, résolvons l'équation $ f(\omega)=\omega $.

$ f(\omega)=\omega \Longleftrightarrow -\frac{5}{2}\omega+2\i=\omega  \Longleftrightarrow \omega=\frac{4}{7} \i$.

\medskip

La transformation $ f $ admet donc un unique point invariant $ \Omega $ d'affixe $ \omega= \frac{4}{7} \i$.

 Pour déterminer la nature de $ f $ , exprimons  $ z'-\omega $ en fonction de $ z-\omega $. On a :
$$  \left\{\begin{array}{l c l}
z'&=-\frac{5}{2}z+2\i \\ 	 
\omega &=-\frac{5}{2}\omega +2\i\\

\end{array}\right.  $$
D'où en soustrayant membre à membre :  $ z'-\omega =-\frac{5}{2}(z-\omega) $..

On en déduit, grâce à son écriture complexe, que $ f $ est l'homothétie de centre $ \Omega $ et de rapport $ k=-\frac{5}{2} $. 
\end{example}
\medskip

\begin{remark}
\begin{itemize}
\item Comme cas particulier d'une homothétie, on a la symétrie centrale, qui est 
une homothétie de rapport $ -1 $.

 L'écriture complexe de la symétrie s de centre $ \Omega $ d'affixe $ \omega $  est s$(z)= z'=-z+2 \omega $.
 \item Lorsque O (origine du repère )est le centre de l'homothétie alors $ z'= k z $.
\end{itemize}
\end{remark}
\subsubsection*{Écriture complexe d'une rotation}
\begin{theorem}
 La rotation de centre $ \Omega  (\omega ) $ et d'angle  $ \theta\in\mathbb{R} $ transforme un point  $ M( z )$  en un point $ M'(z ')$  tel que : $ z'-\omega =\eexp{\i \theta}(z-\omega) $ ou  $ z'=\eexp{\i \theta}z +\omega(1-\eexp{\i \theta}) $.
 \end{theorem}
 \medskip

\textbf{Démonstration}

\medskip
Si $ M=\Omega $, la relation  $ z'-\omega =\eexp{\i \theta}(z-\omega) $  est triviale.

 Si $ M \neq\Omega  $ , dire que $ M' $
est l'image de $ M $ par la rotation de centre $ \Omega $ et d'angle $ \theta $ signifie  que : $$  \left\{\begin{array}{l c l}
\Omega M'=\Omega M \\ 	 
 \paren{\overrightarrow{\Omega M} \; ;\;  \overrightarrow{ \Omega M'}} =\theta \quad  [2\pi]\\

\end{array}\right.  $$


Ce qui se traduit en termes d'affixes par :$$  \left\{\begin{array}{l c l}
\abs{z'-\omega}=\abs{z-\omega}\\ 	 
\text{arg}\paren{\dfrac{z'-\omega}{z-\omega}} =\theta \quad  [2\pi]\\

\end{array}\right.  $$
On en déduit  que :  $ \dfrac{z'-\omega}{z-\omega}= \eexp{\i \theta} $ d'où  $ z'-\omega =\eexp{\i \theta}(z-\omega) $.

\medskip
\textbf{Cas particulier}

\medskip

Si  $ \Omega= $O, l'écriture complexe de la rotation devient : $ z'=\eexp{\i \theta}z  $.

\medskip
\begin{example}
On donne deux points distincts $A(a)$ et $B(b)$. On construit le carré ABCD de sens
direct. Quelle est l'affixe $ \omega $ du centre $ \Omega $ du carré ABCD ?


Il suffit de remarquer que B est l'image de A par la rotation de centre $ \Omega $ et d'angle $ \frac{\pi}{2} $.

 $ b-\omega =\eexp{\i \frac{\pi}{2}}(a-\omega)=\i (a-\omega)\Longleftrightarrow \omega(1-\i)=a\i-b\Longleftrightarrow \omega=\dfrac{a\i-b}{1-\i}$.
 
\end{example}
 \begin{exercice}
 \begin{enumerate}
 \item 
 Le plan complexe est muni d'un repère orthonormé direct $ \ouv $.
 \medskip
 
 Soit $A$, $B$ et $C$ les points d'affixes respectives $z_A = 4$,  $z_B = 1 + 3\i$ et $z_C = 1 -\i $.

 \begin{enumerate}
 \item   
 Déterminer l'écriture complexe de la translation   de vecteur $ \overrightarrow{AB}$ puis trouver l'image $ C' $ du  point  $ C $.
 \item Déterminer l'écriture complexe de la rotation   de centre $ O $  et d'angle $ \frac{\pi}{3} $  puis trouver l'image $ A' $ de  $ A $.
 \item Déterminer le rapport de l'homothétie de centre $ \Omega (1) $   qui transforme  $ B$     en $C $.
  \item Déterminer le centre de la rotation d'angle $ \frac{\pi}{4} $ qui transforme  $ B$     en $C $.
 \end{enumerate}
 \item  Déterminer la nature et  les éléments caractéristiques des transformations suivantes :
 
 \textbf{a)}  $ \;T_1\;: \:z'=\paren{-\tfrac{1}{2} +\i \tfrac{\sqrt{3}}{2}}z +2\i(1-\eexp{\i\frac{2\pi}{3}}) \hspace*{0.5cm}$  \textbf{b)} $\; T_2\;: \:z'= -\sqrt{3}z +(4+\i)(1+\sqrt{3})$
 \end{enumerate}
 \end{exercice}
 
 \subsection{Similitudes directes}
 \begin{definition}[Rappel]

 On appelle  similitude plane directe, toute transformation du plan  $ \mathcal{P} $ dans lui-même qui multiplie les distances  par un nombre réel $ k>0 $, appelé \textit{rapport} et qui conserve les mesures d'angles.
 \end{definition}
 \medskip
 
 Les éléments caractéristiques d'une similitude directe sont:
 \begin{description}
 \item le rapport,
 \item le centre,
 \item l'angle.
 \end{description}
 \begin{example}
\begin{itemize}
\item Une translation de vecteur non nul est une similitude directe de rapport $ k=1 $ et d'angle $ \theta=0 $.
\item Une homothétie  de centre $ \Omega $  et de rapport $ k $ est une similitude directe de centre $ \Omega $ , de rapport $ |k| $ et d'angle $ \theta=0 $ si $ k>0 $ ou $ \theta=\pi $ si $ k<0 $.
\item Une rotation de centre $ \Omega $  et d'angle  $ \theta $ est une similitude directe de rapport $ k=1 $, de centre $ \Omega $  et d'angle  $ \theta $.
\end{itemize}
Toute similitude  directe S de centre $ \Omega $, de rapport  $ k $ et d'angle $ \theta $ est notée par S($ \Omega $, $ k $, $ \theta $).

 $ \Omega $, $ k $ et $ \theta $  sont les éléments caractéristiques de la similitude directe.
 \end{example}


\bigskip
 \textbf{Propriétés  géométriques d'une similitude directe}
 
 \medskip
 
 Soit $ S $ la similitude directe   de centre $ \Omega $, de rapport  $ k $ et d'angle $ \theta $ qui transforme le point $ M(z) $ en $ M'(z') $. 
 $$ S(M)=M'\Longleftrightarrow \left\{\begin{array}{l c l}
\Omega M'=k \,\Omega M\\ 	 
\paren{\overrightarrow{\Omega M}\; ,\;  \overrightarrow{\Omega M'}} =\theta \quad  [2\pi]\\

\end{array}\right.  $$
\begin{remark}
\begin{itemize}
\item Le centre de la similitude est le seul \textit{point invariant}.
\item Une similitude directe qui n'admet pas de point invariant est une translation.
\end{itemize}
\end{remark}
 \begin{property}
 \begin{itemize}
 \item Une similitude directe multiplie:
 \begin{itemize}
 \item[$-$] les distances par $ k $,
  \item[$-$] les aires par $ k^2 $.
 \end{itemize}
 \item Une similitude directe conserve:
 \begin{itemize}
  \item[$-$] l'alignement des points,
 \item[$-$] le parallélisme,
  \item[$-$] l'orthogonalité,
   \item[$-$] le contact.
    \item[$-$] le barycentre
     \item[$-$] les angles orientés.
 \end{itemize}
 \item Par une  similitude directe S l'image d'une droite $ \Delta $ passant les points A et B est une droite $ \Delta' $ passant par les images A' et B' de  A et B par S.
 \item Par une  similitude directe S l'image d'un cercle $ \mathcal{C} $ de centre $ I $ et de rayon $ r $ est un cercle $ \mathcal{C}' $ de centre $ I' $ image de $ I $ par S et de rayon $ k r $ .
 \item La réciproque d'une similitude directe  $S( \Omega $, $ k $, $ \theta $) est une   similitude directe $S^{-1}$ $\paren{ \Omega , \; \frac{1}{k}, \; -\theta }$.
 \item La composée de deux similitudes directes  de même centre , est une similitude directe  de même centre, de rapport le produit des rapports et d'angle, la somme des angles.
 \medskip
 
\hspace*{4cm} S($ \Omega $, $ k $, $ \theta $)$ \circ $S'($ \Omega' $, $ k' $, $ \theta' $)$ = $S($ \Omega $, $ k +k'$, $ \theta +\theta' $)
 \end{itemize}
\end{property}
 
 \begin{remark}
 Soit la similitude directe $S( \Omega $, $ k $, $ \theta $). Alors :
 
 \medskip
 
 $ \underbrace{ S \circ S\circ S\circ \cdots S\circ S \circ S}_{n \text{\; fois}}= S( \Omega , \; k^n , \; n\:\theta )$
\end{remark}
\subsection*{ Expression complexe d'une similitude directe}
 \begin{lemma}
Le plan complexe $ \mathcal{P} $ est muni du repère orthonormé (O;\; I ,\; J).

\medskip
On considère l'application $ f $ du plan complexe $ \mathcal{P} $  dans lui-même qui à tout point $ M $ d'affixe $ z $ associe le point $ M' $ d'affixe $ z'=(1+\i) z+2-3\i $. 
\begin{enumerate}
        \item Déterminer l'image $ A' $ du point $ A$ d'affixe $ 2+\i $.
        \item Quelle est l'affixe de l'image  du point O ? du point I ? du point J?
        \item  Déterminer l'affixe de l'antécédent du point $ B'(-2\i) $ .
        \item Quelle est l'affixe de l'antécédent  du point O ?
        \item Déterminer le point $ \Omega $ dont l'affixe $ \omega $ vérifie $ f(\omega)=\omega $.
        \item Exprimer $ z $ en fonction de $ z' $ sous la forme $ z=az'+b $ où $ a$ et $ b$ sont des nombres complexes écrits sous forme algébrique.
\end{enumerate}
 \end{lemma}


 \begin{lemma}

On considère l'application $ f $ du plan complexe $ \mathcal{P} $  dans lui-même qui à tout point $ M $ d'affixe $ z $ associe  son image $ M' $ d'affixe $ z' $ telle que $ z'=a z+b$  où $ a$ et $ b$ sont des nombres complexes non nuls. 

\medskip
On donne les points $ A(1) $, $ B(\i)$,  $ A'(1+2\i)$ et  $ B'(-1+6\i)$
\begin{enumerate}
\item Déterminer $ a$ et $ b$ sachant que $ f(A)=A' $  et $ f(B)=B' $.
\item Déterminer $ \omega $ tel que $ f(\omega)=\omega $.
\item Exprimer  $ z'-\omega $ en fonction de $ z-\omega $.
\item En posant  $ z'=x'+\i y' $ et $ z=x+\i y$ exprimer $ x' $ et $ y' $ en fonction de $ x$ et $ y$.
\end{enumerate}
 \end{lemma}
\begin{theorem}[Écriture complexe]
La similitude directe  $ S $ de centre $ \Omega $  d'affixe $ \omega $, de rapport $ k $ et d'angle $ \theta $  transforme un point  $ M( z )$  en un point $ M'(z ')$  tel que : $ z'-\omega =k\eexp{\i \theta}(z-\omega) $ ou  $ z'=k\eexp{\i \theta}z +\omega(1-k\eexp{\i \theta}) $.
 
\end{theorem}

\textbf{Démonstration}

\medskip
 $ S(M)=M'\Longleftrightarrow \left\{\begin{array}{l c l}
\Omega M'=k \,\Omega M\\	 
\paren{\overrightarrow{\Omega M}\; ,\;  \overrightarrow{\Omega M'}} =\theta \quad  [2\pi]
\end{array}\right.  \Longleftrightarrow   \left\{\begin{array}{l c l}
\dfrac{\Omega M'}{\Omega M}=k \\ 	 
\paren{\overrightarrow{\Omega M}\; ,\;  \overrightarrow{\Omega M'}} =\theta \quad  [2\pi]
\end{array}\right. \Longleftrightarrow  \left\{\begin{array}{l c l}
\abs{\dfrac{z'-\omega}{z-\omega}}=k \\ 	 
\text{arg}\paren{\frac{z'-\omega}{z-\omega}} =\theta \quad  [2\pi]
\end{array}\right. $

\medskip
$ \Longleftrightarrow \dfrac{z'-\omega}{z-\omega}=k\eexp{\i \theta}\Longleftrightarrow   z'-\omega =k\eexp{\i \theta}(z-\omega)$   ou  $ z'=k\eexp{\i \theta}z +\omega(1-k\eexp{\i \theta}) $.

 \begin{corollary}
Toute similitude directe a une écriture complexe de la forme :\; $ z'=az+b $\; où $ a\in\mathbb{C}^{\star} $, $ b\in\mathbb{C} $  et $ z' $ l'affixe de l'image du point d'affixe $ z $.
 \end{corollary}
 
\medskip

\textbf{Réciproque}

\medskip

Toute transformation $f$ admettant une écriture de la forme : $z' = a z + b $  avec  $ a\neq 0 $ 
est une similitude directe de rapport  $ k = |a| $  et d'angle  $ \theta= \text{arg}\; a $.

 \medskip

\textbf{Démonstration}

\medskip
Soient $ M $ et $ N $ points quelconques du plan d'images respectives $ M' $ et $ N' $ par $f$.

\medskip
$  \left\{\begin{array}{l c l}
z_{N'}&=a z_{N}+b \\ 	 
z_{M'} &= a z_{M}+b\\

\end{array}\right.  $ alors $ z_{N'}-z_{M'}=a ( z_{N}-z_{M}) $ d'où $\abs{ z_{N'}-z_{M'}}=|a| \abs{ z_{N}-z_{M} }$.
\medskip

 D'où $ M'N'=|a|\times MN $
 
 \medskip
 
 Et  $a \neq 0 $, donc f est une similitude de rapport $ |a| $.
 
 \medskip
 
 De plus, comme $a \neq 0 $, son argument existe et $\text{arg}\; (z_N' - z_M') = \text{arg}\; a + \text{arg}\;(z_N -z_M)$

\medskip

Donc : $ \paren{\overrightarrow{u}, \overrightarrow{M'N'}} =\text{arg}\; a+\paren{\overrightarrow{u}, \overrightarrow{MN}}$.


\medskip

D’où : $ \paren{\overrightarrow{MN}, \overrightarrow{M'N'}} =\text{arg}\; a$

\medskip

$f$ est une similitude et l'angle entre un vecteur et son image est constant donc :

\medskip

$f$ est donc  une similitude directe et son angle vaut cette constante : $ \text{arg}\; a $.

\begin{theorem}
 Soient A, B, A' et B' quatre points donnés du plan tels que A $ \neq $ B et A' $ \neq $ B'.
 
Alors, il existe une unique similitude directe $ s $ telle que : $ s $(A) = A' et $ s $(B) = B'.
\end{theorem}
 

 

\textbf{Démonstration}

\medskip

Si une telle similitude $ s $ existe alors il existe  deux nombres complexes $ a $ et $ b $ , avec  $ a\neq $ 0 tels que:

$z_{A'} = az_A + b$ \;  et \; $z_{B'} = az_B + b$


alors : $z_{B'} - z_{A'} = a (z_B - z_A)$  soit  $a=\dfrac{z_{B'}-z_{A'}}{z_B-z_A}   $ et on a : $   b = z_{A'} - az_A$

Si $ s $ existe, le couple $( a , b )$ est unique et $ s $ est donc elle aussi unique.

Soit $s$  la similitude directe dont l'écriture complexe est $ z'=a z +b\; $ où  $ \; a=\dfrac{z_{B'}-z_{A'}}{z_B-z_A}\; $ et  $ \; b = z_{A'}-az_A $.

\medskip

$B$ étant différent de $A$, donc $a$ est défini.

$z_{A'} = az_A + b\;$   et  $\; z_{B'} - z_{A'} = az_B - az_A$

Donc $z_{B'} = az_B - az_A+ z_{A'} = az_B + b$

De plus, comme  $B’ \neq A’$, donc  $ a $ est non nul et $ s $ est donc définie.

D’où : $s(A) = A'$ et $s(B) = B'$.

\medskip
Une similitude directe transformant A en A' et B en B' existe donc et est unique.
\medskip

\textbf{ Expression analytique d'une similitude directe}

\medskip

À partir de l'écriture complexe d'une similitude s directe, on peut en déduire l'écriture analytique.  Pour cela on remplace $ z $ par $ x+\i y $ et $ z' $ par $ x'+\i y' $ dans $ z'=a z+b $. Puis on exprime $ x' $ et $ y' $ en fonction de $ x $ et $ y $.

\begin{example}
Soit s : $ z'=(1-\i )z+1+5\i $

$ x'+\i y'=(1-\i )(x+\i y)+ 1+5\i \Longleftrightarrow x'+\i y'=x+ y +1 + \i(-x+y +5) \Leftrightarrow $ $\left\{\begin{array}{l c l}
x'&=x+y+1      \\ 	 
 y'&=-x+y+5
\end{array}\right. $ 
\end{example}


\bigskip

 \textbf{ Utilisation des nombres complexes pour
déterminer la nature d'une transformation
géométrique}
 
 \begin{theorem}
Soit  une similitude directe s d'écriture   complexe : $z' = a z + b$ avec  $a \neq 0$.

\begin{itemize}
\item   si $a = 1$ : s est la translation de vecteur  d'affixe $ b $. 
\item si $ a\neq 1 $ : alors s admet un unique point invariant  d'affixe : $ \omega=\dfrac{b}{1-a} $  et s est la composée :
\begin{itemize}
\item de l'homothétie de centre  $ \Omega(\omega) $ et de rapport $ |a| $ (rapport de s) et
\item de la rotation de centre  $ \Omega(\omega) $ et d'angle : \, $ \text{arg} a $ (angle de s)

 $\Omega $ est appelé le centre de la similitude directe s.
 
 Et une écriture complexe de s est alors $ z'-\omega =|a|e^{\i \text{arg} a}(z-\omega) $.
\end{itemize}
\item  si $ |a|=1 $ et $ a\neq 1 $ alors  s est une  rotation de centre d'affixe  $ \omega=\dfrac{b}{1-a} $ et d'angle   \, $ \text{arg} a $.
\item  si $ a\in\mathbb{R} \setminus \accol{0\, , \,1}$ alors s une homothétie de centre d'affixe  $ \omega=\dfrac{b}{1-a} $ et de rapport $ a $.
\end{itemize}
 \end{theorem}
\subsubsection*{Récapitulatif des écritures complexes}

$$
\begin{array}{|l|l|}
\hline
\textbf{Transformation} & \textbf{Écriture complexe} \\
\hline
\text{Translation de vecteur } \vec{u} & z' = z + b \quad (b = \text{affixe de } \vec{u}) \\
\hline
\text{Homothétie de centre } \Omega,\ \text{rapport } k & z' - \omega = k(z - \omega) \quad (\omega = \text{affixe de } \Omega) \\
\hline
\text{Rotation de centre } \Omega,\ \text{angle } \theta & z' - \omega = e^{i\theta}(z - \omega) \quad (\omega = \text{affixe de } \Omega) \\
\hline
\text{Similitude directe de centre } \Omega,\ k > 0,\ \theta\in\mathbb{R} & z' - \omega = ke^{i\theta}(z - \omega) \quad (\omega = \text{affixe de } \Omega) \\
\hline
\end{array}
$$
\begin{exercice}
\begin{enumerate}
\item Identifier la transformation définie par l'écriture complexe donnée et préciser ses éléments
caractéristiques

\medskip

\textbf{a)} $\; z'=z-\i\sqrt{3} $ \\
\textbf{b)} $\; z'=\tfrac{\sqrt{2}}{2}(1-\i)z-\i $\\ \textbf{c)} $\; z'=4 z- 2\i $ \\ 
\textbf{d)} $ \; z'=-\i z+1+\i $
\item Donner l'écriture complexe des similitudes directes ci-dessous de centre $ \Omega $ d'affixe $ \omega $ , de rapport $ k $ et
d'angle $ \theta $.

\medskip

\textbf{a)} $\; \omega=2+\i,\; $ $\; k= 2$,  $\; \theta= \frac{\pi}{2}$\\
\textbf{b)}  $ \;\omega=-\i $,\ $ \;k= \dfrac{1}{2}$,  $\; \theta= \dfrac{\pi}{2}$ \\
\textbf{c)}  $\; \omega= 1+2\i $,  $ \;k= 3$,   $\; \theta= \dfrac{\pi}{2}$  

\textbf{d)}  $\; \omega=0$,  $\; k= \sqrt{3}$, $\;\; \theta= \dfrac{3\pi}{4}$ \\
\textbf{d)}  $ \;\omega= 1+\i $ ,  $ \;k= 2$,  $\; \theta=\pi$  \\
\textbf{e)}  $\; \omega= -1 $ ,  $ \;k= 7$,  $\; \theta= 0$  
\end{enumerate}
 \end{exercice}
  %</content>
\end{document}