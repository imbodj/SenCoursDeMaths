\documentclass[12pt, a4paper]{report}

% LuaLaTeX :

\RequirePackage{iftex}
\RequireLuaTeX

% Packages :

\usepackage[french]{babel}
%\usepackage[utf8]{inputenc}
%\usepackage[T1]{fontenc}
\usepackage[pdfencoding=auto, pdfauthor={Hugo Delaunay}, pdfsubject={Mathématiques}, pdfcreator={agreg.skyost.eu}]{hyperref}
\usepackage{amsmath}
\usepackage{amsthm}
%\usepackage{amssymb}
\usepackage{stmaryrd}
\usepackage{tikz}
\usepackage{tkz-euclide}
\usepackage{fontspec}
\defaultfontfeatures[Erewhon]{FontFace = {bx}{n}{Erewhon-Bold.otf}}
\usepackage{fourier-otf}
\usepackage[nobottomtitles*]{titlesec}
\usepackage{fancyhdr}
\usepackage{listings}
\usepackage{catchfilebetweentags}
\usepackage[french, capitalise, noabbrev]{cleveref}
\usepackage[fit, breakall]{truncate}
\usepackage[top=2.5cm, right=2cm, bottom=2.5cm, left=2cm]{geometry}
\usepackage{enumitem}
\usepackage{tocloft}
\usepackage{microtype}
%\usepackage{mdframed}
%\usepackage{thmtools}
\usepackage{xcolor}
\usepackage{tabularx}
\usepackage{xltabular}
\usepackage{aligned-overset}
\usepackage[subpreambles=true]{standalone}
\usepackage{environ}
\usepackage[normalem]{ulem}
\usepackage{etoolbox}
\usepackage{setspace}
\usepackage[bibstyle=reading, citestyle=draft]{biblatex}
\usepackage{xpatch}
\usepackage[many, breakable]{tcolorbox}
\usepackage[backgroundcolor=white, bordercolor=white, textsize=scriptsize]{todonotes}
\usepackage{luacode}
\usepackage{float}
\usepackage{needspace}
\everymath{\displaystyle}

% Police :

\setmathfont{Erewhon Math}

% Tikz :

\usetikzlibrary{calc}
\usetikzlibrary{3d}

% Longueurs :

\setlength{\parindent}{0pt}
\setlength{\headheight}{15pt}
\setlength{\fboxsep}{0pt}
\titlespacing*{\chapter}{0pt}{-20pt}{10pt}
\setlength{\marginparwidth}{1.5cm}
\setstretch{1.1}

% Métadonnées :

\author{agreg.skyost.eu}
\date{\today}

% Titres :

\setcounter{secnumdepth}{3}

\renewcommand{\thechapter}{\Roman{chapter}}
\renewcommand{\thesubsection}{\Roman{subsection}}
\renewcommand{\thesubsubsection}{\arabic{subsubsection}}
\renewcommand{\theparagraph}{\alph{paragraph}}

\titleformat{\chapter}{\huge\bfseries}{\thechapter}{20pt}{\huge\bfseries}
\titleformat*{\section}{\LARGE\bfseries}
\titleformat{\subsection}{\Large\bfseries}{\thesubsection \, - \,}{0pt}{\Large\bfseries}
\titleformat{\subsubsection}{\large\bfseries}{\thesubsubsection. \,}{0pt}{\large\bfseries}
\titleformat{\paragraph}{\bfseries}{\theparagraph. \,}{0pt}{\bfseries}

\setcounter{secnumdepth}{4}

% Table des matières :

\renewcommand{\cftsecleader}{\cftdotfill{\cftdotsep}}
\addtolength{\cftsecnumwidth}{10pt}

% Redéfinition des commandes :

\renewcommand*\thesection{\arabic{section}}
\renewcommand{\ker}{\mathrm{Ker}}

% Nouvelles commandes :

\newcommand{\website}{https://github.com/imbodj/SenCoursDeMaths}

\newcommand{\tr}[1]{\mathstrut ^t #1}
\newcommand{\im}{\mathrm{Im}}
\newcommand{\rang}{\operatorname{rang}}
\newcommand{\trace}{\operatorname{trace}}
\newcommand{\id}{\operatorname{id}}
\newcommand{\stab}{\operatorname{Stab}}
\newcommand{\paren}[1]{\left(#1\right)}
\newcommand{\croch}[1]{\left[ #1 \right]}
\newcommand{\Grdcroch}[1]{\Bigl[ #1 \Bigr]}
\newcommand{\grdcroch}[1]{\bigl[ #1 \bigr]}
\newcommand{\abs}[1]{\left\lvert #1 \right\rvert}
\newcommand{\limi}[3]{\lim_{#1\to #2}#3}
\newcommand{\pinf}{+\infty}
\newcommand{\minf}{-\infty}
%%%%%%%%%%%%%% ENSEMBLES %%%%%%%%%%%%%%%%%
\newcommand{\ensemblenombre}[1]{\mathbb{#1}}
\newcommand{\Nn}{\ensemblenombre{N}}
\newcommand{\Zz}{\ensemblenombre{Z}}
\newcommand{\Qq}{\ensemblenombre{Q}}
\newcommand{\Qqp}{\Qq^+}
\newcommand{\Rr}{\ensemblenombre{R}}
\newcommand{\Cc}{\ensemblenombre{C}}
\newcommand{\Nne}{\Nn^*}
\newcommand{\Zze}{\Zz^*}
\newcommand{\Zzn}{\Zz^-}
\newcommand{\Qqe}{\Qq^*}
\newcommand{\Rre}{\Rr^*}
\newcommand{\Rrp}{\Rr_+}
\newcommand{\Rrm}{\Rr_-}
\newcommand{\Rrep}{\Rr_+^*}
\newcommand{\Rrem}{\Rr_-^*}
\newcommand{\Cce}{\Cc^*}
%%%%%%%%%%%%%%  INTERVALLES %%%%%%%%%%%%%%%%%
\newcommand{\intff}[2]{\left[#1\;,\; #2\right]  }
\newcommand{\intof}[2]{\left]#1 \;, \;#2\right]  }
\newcommand{\intfo}[2]{\left[#1 \;,\; #2\right[  }
\newcommand{\intoo}[2]{\left]#1 \;,\; #2\right[  }

\providecommand{\newpar}{\\[\medskipamount]}

\newcommand{\annexessection}{%
  \newpage%
  \subsection*{Annexes}%
}

\providecommand{\lesson}[3]{%
  \title{#3}%
  \hypersetup{pdftitle={#2 : #3}}%
  \setcounter{section}{\numexpr #2 - 1}%
  \section{#3}%
  \fancyhead[R]{\truncate{0.73\textwidth}{#2 : #3}}%
}

\providecommand{\development}[3]{%
  \title{#3}%
  \hypersetup{pdftitle={#3}}%
  \section*{#3}%
  \fancyhead[R]{\truncate{0.73\textwidth}{#3}}%
}

\providecommand{\sheet}[3]{\development{#1}{#2}{#3}}

\providecommand{\ranking}[1]{%
  \title{Terminale #1}%
  \hypersetup{pdftitle={Terminale #1}}%
  \section*{Terminale #1}%
  \fancyhead[R]{\truncate{0.73\textwidth}{Terminale #1}}%
}

\providecommand{\summary}[1]{%
  \textit{#1}%
  \par%
  \medskip%
}

\tikzset{notestyleraw/.append style={inner sep=0pt, rounded corners=0pt, align=center}}

%\newcommand{\booklink}[1]{\website/bibliographie\##1}
\newcounter{reference}
\newcommand{\previousreference}{}
\providecommand{\reference}[2][]{%
  \needspace{20pt}%
  \notblank{#1}{
    \needspace{20pt}%
    \renewcommand{\previousreference}{#1}%
    \stepcounter{reference}%
    \label{reference-\previousreference-\thereference}%
  }{}%
  \todo[noline]{%
    \protect\vspace{20pt}%
    \protect\par%
    \protect\notblank{#1}{\cite{[\previousreference]}\\}{}%
    \protect\hyperref[reference-\previousreference-\thereference]{p. #2}%
  }%
}

\definecolor{devcolor}{HTML}{00695c}
\providecommand{\dev}[1]{%
  \reversemarginpar%
  \todo[noline]{
    \protect\vspace{20pt}%
    \protect\par%
    \bfseries\color{devcolor}\href{\website/developpements/#1}{[DEV]}
  }%
  \normalmarginpar%
}

% En-têtes :

\pagestyle{fancy}
\fancyhead[L]{\truncate{0.23\textwidth}{\thepage}}
\fancyfoot[C]{\scriptsize \href{\website}{\texttt{https://github.com/imbodj/SenCoursDeMaths}}}

% Couleurs :

\definecolor{property}{HTML}{ffeb3b}
\definecolor{proposition}{HTML}{ffc107}
\definecolor{lemma}{HTML}{ff9800}
\definecolor{theorem}{HTML}{f44336}
\definecolor{corollary}{HTML}{e91e63}
\definecolor{definition}{HTML}{673ab7}
\definecolor{notation}{HTML}{9c27b0}
\definecolor{example}{HTML}{00bcd4}
\definecolor{cexample}{HTML}{795548}
\definecolor{application}{HTML}{009688}
\definecolor{remark}{HTML}{3f51b5}
\definecolor{algorithm}{HTML}{607d8b}
%\definecolor{proof}{HTML}{e1f5fe}
\definecolor{exercice}{HTML}{e1f5fe}

% Théorèmes :

\theoremstyle{definition}
\newtheorem{theorem}{Théorème}

\newtheorem{property}[theorem]{Propriété}
\newtheorem{proposition}[theorem]{Proposition}
\newtheorem{lemma}[theorem]{Activité d'introduction}
\newtheorem{corollary}[theorem]{Conséquence}

\newtheorem{definition}[theorem]{Définition}
\newtheorem{notation}[theorem]{Notation}

\newtheorem{example}[theorem]{Exemple}
\newtheorem{cexample}[theorem]{Contre-exemple}
\newtheorem{application}[theorem]{Application}

\newtheorem{algorithm}[theorem]{Algorithme}
\newtheorem{exercice}[theorem]{Exercice}

\theoremstyle{remark}
\newtheorem{remark}[theorem]{Remarque}

\counterwithin*{theorem}{section}

\newcommand{\applystyletotheorem}[1]{
  \tcolorboxenvironment{#1}{
    enhanced,
    breakable,
    colback=#1!8!white,
    %right=0pt,
    %top=8pt,
    %bottom=8pt,
    boxrule=0pt,
    frame hidden,
    sharp corners,
    enhanced,borderline west={4pt}{0pt}{#1},
    %interior hidden,
    sharp corners,
    after=\par,
  }
}

\applystyletotheorem{property}
\applystyletotheorem{proposition}
\applystyletotheorem{lemma}
\applystyletotheorem{theorem}
\applystyletotheorem{corollary}
\applystyletotheorem{definition}
\applystyletotheorem{notation}
\applystyletotheorem{example}
\applystyletotheorem{cexample}
\applystyletotheorem{application}
\applystyletotheorem{remark}
%\applystyletotheorem{proof}
\applystyletotheorem{algorithm}
\applystyletotheorem{exercice}

% Environnements :

\NewEnviron{whitetabularx}[1]{%
  \renewcommand{\arraystretch}{2.5}
  \colorbox{white}{%
    \begin{tabularx}{\textwidth}{#1}%
      \BODY%
    \end{tabularx}%
  }%
}

% Maths :

\DeclareFontEncoding{FMS}{}{}
\DeclareFontSubstitution{FMS}{futm}{m}{n}
\DeclareFontEncoding{FMX}{}{}
\DeclareFontSubstitution{FMX}{futm}{m}{n}
\DeclareSymbolFont{fouriersymbols}{FMS}{futm}{m}{n}
\DeclareSymbolFont{fourierlargesymbols}{FMX}{futm}{m}{n}
\DeclareMathDelimiter{\VERT}{\mathord}{fouriersymbols}{152}{fourierlargesymbols}{147}

% Code :

\definecolor{greencode}{rgb}{0,0.6,0}
\definecolor{graycode}{rgb}{0.5,0.5,0.5}
\definecolor{mauvecode}{rgb}{0.58,0,0.82}
\definecolor{bluecode}{HTML}{1976d2}
\lstset{
  basicstyle=\footnotesize\ttfamily,
  breakatwhitespace=false,
  breaklines=true,
  %captionpos=b,
  commentstyle=\color{greencode},
  deletekeywords={...},
  escapeinside={\%*}{*)},
  extendedchars=true,
  frame=none,
  keepspaces=true,
  keywordstyle=\color{bluecode},
  language=Python,
  otherkeywords={*,...},
  numbers=left,
  numbersep=5pt,
  numberstyle=\tiny\color{graycode},
  rulecolor=\color{black},
  showspaces=false,
  showstringspaces=false,
  showtabs=false,
  stepnumber=2,
  stringstyle=\color{mauvecode},
  tabsize=2,
  %texcl=true,
  xleftmargin=10pt,
  %title=\lstname
}

\newcommand{\codedirectory}{}
\newcommand{\inputalgorithm}[1]{%
  \begin{algorithm}%
    \strut%
    \lstinputlisting{\codedirectory#1}%
  \end{algorithm}%
}




\begin{document}
  %<*content>
  \lesson{algebra}{105}{Groupe des permutations d'un ensemble fini. Applications.}

  Pour toute cette leçon, on fixe un entier $n \geq 1$.

  \subsection{Généralités}

  \subsubsection{Définitions}

  \reference[ROM21]{37}

  \begin{definition}
    Soit $E$ un ensemble. On appelle \textbf{groupe des permutations} de $E$ le groupe des bijections de $E$ dans lui-même. On le note $S(E)$.
  \end{definition}

  \begin{notation}
    Si $E = \llbracket 1, n \rrbracket$, on note $S(E) = S_n$, le groupe symétrique à $n$ éléments.
  \end{notation}

  \begin{notation}
    Soit $\sigma \in S_n$. On note :
    \[
      \sigma =
      \begin{pmatrix}
        1 & 2 & \dots & n \\
        \sigma(1) & \sigma(2) & \dots & \sigma(n)
      \end{pmatrix}
    \]
    pour signifier que $\sigma$ est la bijection $\sigma : k \mapsto \sigma(k)$.
  \end{notation}

  Le théorème suivant justifie que, pour un ensemble à $n$ éléments, on peut se contenter d'étudier $S_n$ en lieu et place de $S(E)$.

  \begin{theorem}
    \begin{enumerate}[label=(\roman*)]
      \item Soient $E$ et $F$ deux ensembles en bijection. Alors $S(E)$ et $S(F)$ sont isomorphes.
      \item \[ |S_n| = n! \]
    \end{enumerate}
  \end{theorem}

  \reference{53}

  \begin{theorem}[Cayley]
    Tout groupe $G$ est isomorphe à un sous-groupe de $S(G)$.
  \end{theorem}

  \subsubsection{Orbites et cycles}

  \reference{41}

  \begin{definition}
    Soit $\sigma \in \llbracket 1, n \rrbracket$. On a une action naturelle de $H = \langle \sigma \rangle$ sur $\llbracket 1, n \rrbracket$ définie par
    \[ \forall k \in \mathbb{Z}, \forall j \in \llbracket 1, n \rrbracket, \, \sigma^k \cdot j = \sigma^k(j) \]
    Les orbites pour cette action sont les $H \cdot j = \{ \sigma(j) \mid j \in \llbracket 1, n \rrbracket \}$. On les note $\mathcal{O}_\sigma(j)$.
  \end{definition}

  \begin{remark}
    \begin{itemize}
      \item Les orbites selon $\sigma$ sont décrites par la relation
      \[ x \sim y \iff \exists k \in \mathbb{Z} \text{ tel que } y = \sigma^k(x) \]
      \item Une orbite $\mathcal{O}_\sigma(j)$ est réduite à un point si et seulement si $\sigma(j) = j$.
    \end{itemize}
  \end{remark}

  \reference{37}

  \begin{definition}
    Soient $l \leq n$ et $i_1, \dots, i_l \in \llbracket 1, n \rrbracket$ des éléments distincts. La permutation $\gamma \in S_n$ définie par
    \[
    \gamma(j) =
    \begin{cases}
      j &\text{si } j \notin \{ i_1, \dots, i_l \} \\
      i_{k+1} &\text{si } j = i_k \text{ avec } k<l \\
      i_1 &\text{si } j=i_l
    \end{cases}
    \]
    et notée $\begin{pmatrix} i_1 & \dots & i_l \end{pmatrix}$ est appelée \textbf{cycle} de longueur $l$ et de \textbf{support} $\{ i_1, \dots, i_l \}$. Un cycle de longueur $2$ est une \textbf{transposition}.
  \end{definition}

  \reference{42}

  \begin{proposition}
    Une permutation $\sigma$ est cycle si et seulement s'il n'y a qu'une seule orbite $\mathcal{O}_\sigma(j)$ non réduite à un point.
  \end{proposition}

  \begin{remark}
    La composée de deux cycles n'est pas un cycle en général.
  \end{remark}

  \begin{example}
    Avec $\sigma = \begin{pmatrix} 1 & 2 & 3 & 4 \end{pmatrix} \in S_4$, on a $\sigma^2 = \begin{pmatrix} 1 & 2 & 3 & 4 \\ 3 & 4 & 1 & 2 \end{pmatrix}$ qui n'est pas un cycle.
  \end{example}

  \begin{proposition}
    L'ordre d'un cycle est égal à sa longueur.
  \end{proposition}

  \reference[ULM21]{56}

  \begin{proposition}
    Soient $\sigma$ et $\tau$ deux cycles de $S_n$ dont on note respectivement $\operatorname{Supp}(\sigma)$ et $\operatorname{Supp}(\tau)$ les supports. Si $\operatorname{Supp}(\sigma) \, \cap \, \operatorname{Supp}(\tau) = \emptyset$, alors $\operatorname{Supp}(\sigma\tau) = \operatorname{Supp}(\sigma) \, \cup \, \operatorname{Supp}(\tau)$ et dans ce cas :
    \begin{enumerate}[label=(\roman*)]
      \item $\sigma\tau = \tau\sigma$.
      \item $\sigma\tau = \operatorname{id} \implies \sigma = \tau = \operatorname{id}$.
    \end{enumerate}
  \end{proposition}

  \begin{theorem}
    Toute permutation de $S_n$ s'écrit de manière unique (à l'ordre près) comme produit de cycles dont les supports sont deux à deux disjoints.
  \end{theorem}

  \begin{example}
    \label{105-1}
    \[
    \begin{pmatrix}
      1 & 2 & 3 & 4 & 5 & 6 \\
      2 & 4 & 5 & 1 & 3 & 6
    \end{pmatrix}
    =
    \begin{pmatrix} 1 & 2 & 4 \end{pmatrix}\begin{pmatrix} 3 & 5 \end{pmatrix}
    \]
  \end{example}

  \begin{definition}
    On appelle \textbf{type} d'une permutation $\sigma \in S_n$ et on note $[l_1, \dots, l_m]$ la liste des cardinaux $l_i$ des orbites dans $\llbracket 1, n \rrbracket$ de l'action du groupe $\langle \sigma \rangle$ sur $\llbracket 1, n \rrbracket$, rangée dans l'ordre croissant.
  \end{definition}

  \begin{proposition}
    Une permutation de type $[l_1, \dots, l_m]$ a pour ordre $\operatorname{ppcm}(l_1, \dots, l_m)$.
  \end{proposition}

  \begin{example}
    La permutation de l'\cref{105-1} est d'ordre $6$.
  \end{example}

  \subsubsection{Signature}

  \begin{definition}
    Soit $\sigma \in S_n$. On appelle \textbf{signature} de $\sigma$, notée $\epsilon(\sigma)$ le nombre rationnel
    \[ \epsilon(\sigma) = \prod_{i \neq j} \frac{\sigma(i) - \sigma(j)}{i-j} \]
  \end{definition}

  \begin{example}
    \[ \epsilon(\begin{pmatrix} 1 & 2 \end{pmatrix}) = -1 \]
  \end{example}

  \begin{proposition}
    $\epsilon : S_n \rightarrow \mathbb{Q}^*$ est un morphisme de groupes. Pour une permutation $\sigma \in S_n$, on a les propriétés suivantes :
    \begin{enumerate}[label=(\roman*)]
      \item Si $\sigma$ est un transposition, $\epsilon(\sigma) = -1$.
      \item Si $l$ est le nombre de transpositions qui apparaît dans une décomposition de $\sigma$ en produit de transpositions, alors $\epsilon(\sigma) = (-1)^l$.
      \item Si $\sigma$ est de type $[l_1, \dots, l_m]$, alors $\epsilon(\sigma) = (-1)^{l_1 + \dots + l_m - m}$.
    \end{enumerate}
    En particulier, si $n \geq 2$, l'image de $\epsilon$ est le sous-groupe $\{ \pm 1 \}$ de $\mathbb{Q}^*$.
  \end{proposition}

  \reference[PEY]{20}

  \begin{proposition}
    Le seul morphisme non trivial de $S_n$ dans $\mathbb{C}^*$ est $\epsilon$.
  \end{proposition}

  \reference[ULM21]{64}

  \begin{definition}
    \begin{itemize}
      \item Soit $\sigma \in S_n$. Si $\epsilon(\sigma) = 1$, on dit que $\sigma$ est \textbf{paire}. Sinon, on dit qu'elle est \textbf{impaire}.
      \item Le noyau de $\epsilon$ (constitué donc des permutations paires) est un sous-groupe distingué de $S_n$ appelé \textbf{groupe alterné} et noté $A_n$.
    \end{itemize}
  \end{definition}

  \begin{proposition}
    Pour $n \geq 2$,
    \[ |A_n| = \frac{n!}{2} \]
  \end{proposition}

  \subsection{Structure}

  \subsubsection{Conjugaison}

  \reference{60}

  \begin{proposition}
    Deux permutations $\sigma$ et $\tau$ de $S_n$ sont conjuguées si et seulement si elles sont du même type. En particulier, pour $\omega \in S_n$ et tout cycle $\begin{pmatrix} i_1 & \dots & i_l \end{pmatrix} \in S_n$, on a :
    \[ \omega \begin{pmatrix} i_1 & \dots & i_l \end{pmatrix} \omega^{-1} = \begin{pmatrix} \omega(i_1) & \dots & \omega(i_l) \end{pmatrix} \]
  \end{proposition}

  \begin{example}
    Les types possibles d'une permutation de $S_4$ sont $[1]$ (l'identité), $[2]$ (les transpositions), $[2,2]$ (les doubles transpositions), $[3]$ (les $3$-cycles) et $[4]$ (les $4$-cycles) : on a $5$ classes de conjugaison de tailles respectives $1$, $6$, $3$, $8$ et $6$.
  \end{example}

  \reference[PER]{13}

  \begin{proposition}
    Pour tout $n \geq 3$, $Z(S_n) = \{ \sigma \in S_n \mid \forall \tau \in S_n, \, \sigma \tau = \tau \sigma \} = \{ \operatorname{id} \}$.
  \end{proposition}

  \reference{15}

  \begin{lemma}
    Les $3$-cycles sont conjugués dans $A_n$ pour $n \geq 5$.
  \end{lemma}

  \subsubsection{Générateurs}

  \reference[ROM21]{44}

  \begin{proposition}
    \begin{enumerate}[label=(\roman*)]
      \item $S_n$ est engendré par les transpositions. On peut même se limiter aux transpositions de la forme $\begin{pmatrix} 1 & k \end{pmatrix}$ ou encore  $\begin{pmatrix} k & k+1 \end{pmatrix}$ (pour $k \leq n$).
      \item $S_n$ est engendré par $\begin{pmatrix} 1 & 2 \end{pmatrix}$ et $\begin{pmatrix} 1 & \dots & n \end{pmatrix}$.
    \end{enumerate}
  \end{proposition}

  \begin{example}
    Pour $\sigma = \begin{pmatrix} 1 & 2 & 3 & 4 & 5 \end{pmatrix}\begin{pmatrix} 6 & 7 \end{pmatrix}$, on a $\sigma = \begin{pmatrix} 1 & 2 \end{pmatrix} \begin{pmatrix} 2 & 3 \end{pmatrix} \begin{pmatrix} 3 & 4 \end{pmatrix} \begin{pmatrix} 4 & 5 \end{pmatrix} \begin{pmatrix} 6 & 7 \end{pmatrix}$.
  \end{example}

  \begin{proposition}
    $A_n$ est engendré par les $3$-cycles pour $n \geq 3$.
  \end{proposition}

  \subsubsection{Simplicité}

  \reference[PER]{15}

  \begin{lemma}
    Les $3$-cycles sont conjugués dans $A_n$ pour $n \geq 5$.
  \end{lemma}

  \reference[ROM21]{49}

  \begin{lemma}
    Le produit de deux transpositions est un produit de $3$-cycles.
  \end{lemma}

  \reference[PER]{28}
  \dev{simplicite-du-groupe-alterne}

  \begin{theorem}
    $A_n$ est simple pour $n \geq 5$.
  \end{theorem}

  \begin{corollary}
    Le groupe dérivé de $A_n$ est $A_n$ pour $n \geq 5$, et le groupe dérivé de $S_n$ est $A_n$ pour $n \geq 2$.
  \end{corollary}

  \begin{corollary}
    Pour $n \geq 5$, les sous-groupes distingués de $S_n$ sont $S_n$, $A_n$ et $\{ \operatorname{id} \}$.
  \end{corollary}

  \begin{corollary}
    Soit $H$ un sous-groupe d'indice $n$ de $S_n$. Alors, $H$ est isomorphe à $S_{n-1}$.
  \end{corollary}

  \subsection{Applications}

  \subsubsection{Déterminant}

  \reference[GOU21]{140}

  Soit $\mathbb{K}$ un corps et soit $E$ un espace vectoriel de dimension $n$ sur $\mathbb{K}$.

  \begin{definition}
    Soient $E_1, \dots, E_p$ et $F$ des espaces vectoriels sur $\mathbb{K}$ et $f : E_1, \dots, E_p \rightarrow F$.
    \begin{itemize}
      \item $f$ est dite \textbf{$p$-linéaire} si en tout point les $p$ applications partielles sont linéaires.
      \item Si $f$ est $p$-linéaire et si $E_1 = \dots = E_p$ ainsi que $F = \mathbb{K}$, $f$ est une \textbf{forme $p$-linéaire}. On note $\mathcal{L}_p(E, \mathbb{K})$ l'ensemble des formes $p$-linéaires sur $E$.
      \item Si de plus $f(x_1, \dots, x_p) = 0$ dès que deux vecteurs parmi les $x_i$ sont égaux, alors $f$ est dite \textbf{alternée}.
    \end{itemize}
  \end{definition}

  \begin{example}
    En reprenant les notations précédentes, pour $p = 2$, $f$ est bilinéaire.
  \end{example}

  \begin{proposition}
    $\mathcal{L}_p(E, \mathbb{K})$ est un espace vectoriel et, $\operatorname{dim}(\mathcal{L}_p(E, \mathbb{K})) = |\operatorname{dim}(E)|^p$.
  \end{proposition}

  \begin{theorem}
    L'ensemble des formes $p$-linéaires alternées sur $E$ est un $\mathbb{K}$-espace vectoriel de dimension $1$. De plus, il existe une unique forme $p$-linéaire alternée $f$ prenant la valeur $1$ sur une base $\mathcal{B}$ de $E$. On note $f = \det_{\mathcal{B}}$.
  \end{theorem}

  \begin{definition}
    $\det_{\mathcal{B}}$ est l'application \textbf{déterminant} dans la base $\mathcal{B}$. En l'absence d'ambiguïté, on s'autorise à noter $\det = \det_{\mathcal{B}}$.
  \end{definition}

  \begin{proposition}
    Soit $\mathcal{B} = (e_1, \dots, e_n)$ une base de $E$. Si $x_1, \dots, x_n \in E$ ($\forall i \in \llbracket 1, n \rrbracket$, on peut écrire $x_i = \sum_{j=1}^n x_{i,j} e_j$), on a la formule $\det_{\mathcal{B}}(x_1, \dots, x_n) = \sum_{\sigma \in S_n} \epsilon(\sigma) \prod_{i=1}^n x_{i,\sigma(i)}$.
  \end{proposition}

  \begin{corollary}
    Soit $\mathcal{B}$ une base de $E$.
    \begin{enumerate}[label=(\roman*)]
      \item Si $\mathcal{B}'$ est une autre base de $E$, alors $\det_{\mathcal{B}'} = \det_{\mathcal{B}'}(\mathcal{B}) \det_{\mathcal{B}}$.
      \item Une famille de vecteurs est liée si et seulement si son déterminant est nul dans une base quelconque de $E$.
      \item Soient $A, B \in \mathcal{M}_n(\mathbb{K})$, alors $\det_{\mathcal{B}}(AB) = \det_{\mathcal{B}}(A) \det_{\mathcal{B}}(B)$.
      \item Soit $A \in \mathcal{M}_n(\mathbb{K})$, alors $\det_{\mathcal{B}}(A) = \det_{\mathcal{B}}(\tr{A})$ et pour tout $\lambda \in \mathbb{K}$, $\det_{\mathcal{B}}(\lambda A) = \lambda^n \det_{\mathcal{B}}(A)$.
      \item Si on effectue une permutation $\sigma \in S_n$ sur les colonnes d'une matrice $A$, alors le déterminant de $A$ est multiplié par $\epsilon(\sigma)$.
    \end{enumerate}
  \end{corollary}

  \reference[I-P]{203}

  \begin{notation}
    Soit $a \in \mathbb{F}_p$. On note $\left( \frac{a}{p} \right)$ le symbole de Legendre de $a$ modulo $p$.
  \end{notation}

  \begin{lemma}
    Soient $p \geq 3$ un nombre premier et $V$ un espace vectoriel sur $\mathbb{F}_p$ de dimension finie. Les dilatations engendrent $\mathrm{GL}(V)$.
  \end{lemma}

  \dev{theoreme-de-frobenius-zolotarev}

  \begin{theorem}[Frobenius-Zolotarev]
    Soient $p \geq 3$ un nombre premier et $V$ un espace vectoriel sur $\mathbb{F}_p$ de dimension finie.
    \[ \forall u \in \mathrm{GL}(V), \, \epsilon(u) = \left( \frac{\det(u)}{p} \right) \]
    où $u$ est vu comme une permutation des éléments de $V$.
  \end{theorem}

  \subsubsection{Matrices de permutation}

  \reference[ROM21]{54}

  Soit $\mathbb{K}$ un corps et soit $E$ un espace vectoriel de dimension $n$ sur $\mathbb{K}$.

  \begin{definition}
    À tout $\sigma \in S_n$ on associe la matrice de passage de la base canonique $(e_i)_{i \in \llbracket 1, n \rrbracket}$ à la base $(e_\sigma(i))_{i \in \llbracket 1, n \rrbracket}$ que l'on note $P_{\sigma}$ : c'est la \textbf{matrice de permutation} associée à $\sigma$.
  \end{definition}

  \begin{remark}
    En reprenant les notations précédentes, $\forall j \in \llbracket 1, n \rrbracket$, $P_{\sigma} e_j = \sigma(e_j)$.
  \end{remark}

  \begin{proposition}
    $\sigma \mapsto P_{\sigma}$ est un morphisme de groupes injectif de $S_n$ dans $\mathrm{GL}_n(\mathbb{K})$. De plus, on a
    \[ \det(P_{\sigma}) = \epsilon(\sigma) \]
  \end{proposition}

  \begin{corollary}
    Tout groupe fini d'ordre $n$ est isomorphe à un sous groupe de $\mathrm{GL}_n(\mathbb{F}_p)$ pour un premier $p \geq 2$.
  \end{corollary}

  \subsubsection{Polynômes symétriques}

  \reference[GOU21]{83}

  Soit $\mathbb{K}$ un corps de caractéristique différente de $2$.

  \begin{definition}
    Soit $P \in \mathbb{K}[X_1, \dots, X_n]$. On dit que $P$ est \textbf{symétrique} si
    \[ \forall \sigma \in S_n, \, P(X_{\sigma(1)}, \dots, X_{\sigma(n)}) = P(X_1, \dots, X_n) \]
  \end{definition}

  \begin{example}
    Dans $\mathbb{R}[X]$, le polynôme $XY + YZ + ZX$ est symétrique.
  \end{example}

  \begin{definition}
    On appelle \textbf{polynômes symétriques élémentaires} de $A[X_1, \dots, X_n]$ les polynômes noté $\Sigma_p$ où $p \in \llbracket 1, n \rrbracket$ définis par
    \[ \Sigma_p = \sum_{1 \leq i_1 < \dots < i_p \leq n} X_{i_1} \dots X_{i_p} \]
  \end{definition}

  \begin{example}
    \begin{itemize}
      \item $\Sigma_1 = X_1 + \dots + X_n$.
      \item $\Sigma_2 = \sum_{1 \leq i < j \leq n} X_i X_j$.
      \item $\Sigma_n = X_1 \dots X_n$.
    \end{itemize}
  \end{example}

  \begin{remark}
    Si $P \in A[X_1, \dots, X_n]$, alors $P(\Sigma_1(X_1, \dots, X_n), \dots, \Sigma_n(X_1, \dots, X_n))$ est symétrique. Et la réciproque est vraie.
  \end{remark}

  \begin{theorem}[Théorème fondamental des polynômes symétriques]
    Soit $P \in A[X_1, \dots, X_n]$ un polynôme symétrique. Alors,
    \[ \exists! \Phi \in A[X_1, \dots, X_n] \text{ tel que } \Phi(\Sigma_1, \dots, \Sigma_n) \]
  \end{theorem}

  \begin{example}
    $P = X^3 + Y^3 + Z^3$ s'écrit $P = \Sigma_1^3 - 3 \Sigma_1 \Sigma_2 + 3 \Sigma_3$.
  \end{example}

  \reference{64}

  \begin{application}[Relations coefficients - racines]
    Soit $P = a_0X^n + \dots + a_n \in \mathbb{K}[X]$ avec $a_0 \neq 0$ scindé sur $\mathbb{K}$, dont les racines (comptées avec leur ordre de multiplicité) sont $x_1, \dots, x_n$. Alors
    \[ \forall p \in \llbracket 1, n \rrbracket, \, \Sigma_p(x_1, \dots, x_n) = (-1)^p \frac{a_p}{a_0} \]
    En particulier,
    \begin{itemize}
      \item $\Sigma_1(x_1, \dots, x_n) = \sum_{i=1}^n x_i = -\frac{a_1}{a_0}$.
      \item $\Sigma_n(x_1, \dots, x_n) = \prod_{i=1}^n x_i = (-1)^n \frac{a_n}{a_0}$.
    \end{itemize}
  \end{application}

  \reference[I-P]{279}

  \begin{application}[Théorème de Kronecker]
    Soit $P \in \mathbb{Z}[X]$ unitaire tel que toutes ses racines complexes appartiennent au disque unité épointé en l'origine (que l'on note $D$). Alors toutes ses racines sont des racines de l'unité.
  \end{application}

  \begin{corollary}
    Soit $P \in \mathbb{Z}[X]$ unitaire et irréductible sur $\mathbb{Q}$ tel que toutes ses racines complexes soient de module inférieur ou égal à $1$. Alors $P = X$ ou $P$ est un polynôme cyclotomique.
  \end{corollary}
  %</content>
\end{document}
