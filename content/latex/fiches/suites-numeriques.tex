\input{../common}

\begin{document}
  %<*content>
  \sheet{analysis}{suites-numeriques}{Suites numériques (TL)}
 \summary{Aide-mémoire sur les suites}
\begin{proposition}
\[\begin{array}{|c|c|c|}
\hline
\textbf{Propriétés} & \textbf{Suite arithmétique} & \textbf{Suite géométrique} \\
\hline
\textbf{Définition} & u_{n+1} = u_n + r & u_{n+1} = u_n \times q \\
\hline
\textbf{Terme général} & 
\begin{cases}
u_n = u_0 + n r \\
u_n = u_1 + (n-1)r
\end{cases}
& 
\begin{cases}
u_n = u_0 \cdot q^n \\
u_n = u_1 \cdot q^{n-1}
\end{cases}
\\
\hline
\textbf{Raison} & r = u_{n+1} - u_n & q = \dfrac{u_{n+1}}{u_n} \quad (u_n \ne 0) \\
\hline
\textbf{Variation} & 
\begin{cases}
r > 0 \Rightarrow \text{croissante} \\
r < 0 \Rightarrow \text{décroissante} \\
r = 0 \Rightarrow \text{constante}
\end{cases}
&
\begin{cases}
q > 1 \Rightarrow \text{croissante si } u_0 > 0 \\
0 < q < 1 \Rightarrow \text{décroissante si } u_0 > 0 \\
q < 0 \Rightarrow \text{oscillante}
\end{cases}
\\
\hline
\textbf{Somme} & 
S_n = \dfrac{n}{2} (u_0 + u_{n-1}) 
& 
\begin{cases}
q \ne 1 \Rightarrow S_n = u_0 \cdot \dfrac{1 - q^n}{1 - q} \\
q = 1 \Rightarrow S_n = n \cdot u_0
\end{cases}
\\
\hline
\textbf{Modèle} &
\begin{array}{l}
\text{Évolution linéaire} \\
\text{Ex : salaires, loyers} \\
\text{ajout ou retrait fixe}
\end{array}
&
\begin{array}{l}
\text{Évolution exponentielle} \\
\text{Ex : intérêts, population} \\
\text{+ }p\% : q = 1 + \dfrac{p}{100} \\
\text{– }p\% : q = 1 - \dfrac{p}{100}
\end{array}
\\
\hline
\end{array}
\]

\end{proposition}

  %</content>
\end{document}
