\input{../common}
\everymath{\displaystyle}
\begin{document}
  %<*content>
  \sheet{algebra}{fonction-racine-nieme}{Fonction racine n-ième}
 \summary{}
   \begin{theorem}[et définition]
   
$ n\in \Nne $\\
La fonction $x\mapsto x^{n} $  est continue et strictement croissante sur $ \Rrp $  donc elle est bijective de $ \Rrp $ vers $ \Rrp $ et admet une bijection réciproque appelée \emph{fonction racine n-ième} et notée  $x\mapsto x^{\frac{1}{n}} $ ou $ x\mapsto \sqrt[n]{x} $.
\end{theorem}
\begin{example}
\begin{itemize}
\item $ \sqrt[1]{x}=x $,
\item  $ \sqrt[2]{x}=\sqrt{x}=x^{\frac{1}{2}}$\; (racine carrée) ,
\item  $ \sqrt[3]{x} =x^{\frac{1}{3}}$  appelée la racine cubique de $ x. $

\end{itemize}
\end{example}
\begin{notation}
\[\sqrt[n]{x^{p}}= (x^{p})^{\frac{1}{n}}= x^{\frac{p}{n}}\]
\end{notation}

 \colorbox{red!20!}{Résolution de l'équation $ x^{n}=a $}
\begin{itemize}
\item[\textbullet] si $ n $ est pair  et  $ a\geq 0 $ alors $ x= \sqrt[n]{a}$ ou $ x= -\sqrt[n]{a}$
\item[\textbullet] si $ n $ est impair  et  $ a\geq 0 $ alors $ x= \sqrt[n]{a}$
\item[\textbullet] si $ n $ est pair  et  $ a < 0 $ alors pas de solution. 
\item[\textbullet] si $ n $ est impair  et  $ a\leq 0 $ alors  $ x= -\sqrt[n]{-a}$
\end{itemize}

\begin{example}
\begin{enumerate}
\item $ x^3=8 \Longleftrightarrow x=\sqrt[3]{8}= 8^{\frac{1}{3}} = (2^3)^{\frac{1}{3}}=2$
\item $ x^3+1=0\Longleftrightarrow x^3=-1 \Longleftrightarrow x=-\sqrt[3]{1}= -1$
\item $ x^4= 3 \Longleftrightarrow x= \sqrt[4]{3}$ ou $ x= -\sqrt[4]{3} $  
 \end{enumerate}
\end{example}




  %</content>
\end{document}
