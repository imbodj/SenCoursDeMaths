\input{../common}

\begin{document}
  %<*content>
  \sheet{algebra, analysis}{histoire-des-logarithmes}{Naissance des logarithmes}

  \summary{Un bref historique sur les logarithmes}


\change


L'histoire de la naissance des logarithmes  traverse le XVIIe siècle.


\change


En ce temps là, les calculatrices n'existaient pas, et le calcul numérique était fastidieux.


\change


 Elle commence par la création de tables de logarithmes  permettant de faciliter les calculs astronomiques  et se poursuit par des tentatives de calcul d'aire sous l'hyperbole.


\change


Les calculs astronomiques qui se développent au cours du XVIe siècle poussent les mathématiciens à chercher des outils facilitant les calculs de produits et de quotients.

\change


La publication en 1619, par  l'écossais Jean Néper, de son \texttt{ Mirifici logarithmorum canonis constructio} , en fait l'inventeur officiel des tables logarithmiques.


\change


 C'est également à lui que l'on doit le terme de logarithme (arithmos = nombre, logos = raison, rapport). Celles-ci consistent  en une correspondance entre des nombres en progression géométrique et des nombres en progression arithmétique.
 
 
 \change
 
  Ainsi une multiplication peut se ramener à une addition et une division à une différence.

\change

La seconde rencontre des mathématiciens avec les logarithmes concerne la recherche de  l'aire sous l'hyperbole d'équation   $ y=\dfrac{1}{x} $  entre les points d'abscisse a et b. 

\change


Alphonse Antoine de Sarasa, qui, en 1649, à l'occasion d'un problème posé par  Mersenne signale le comportement logarithmique de l'aire sous l'hyperbole. L'aire sous l'hyperbole entre le point d'abscisse 1 et le point d'abscisse $ x $ est alors appelé  le logarithme naturel   du réel x  ou  encore  primitive  de la fonction   $x\mapsto  \dfrac{1}{x} $.

\change

De nos  jours les logarithmes sont utilisés dans plusieurs domaines:\; en arithmétique, en sciences physiques , en  acoustique ...

  %</content>
\end{document}
