\documentclass[12pt, a4paper]{report}

% LuaLaTeX :

\RequirePackage{iftex}
\RequireLuaTeX

% Packages :

\usepackage[french]{babel}
%\usepackage[utf8]{inputenc}
%\usepackage[T1]{fontenc}
\usepackage[pdfencoding=auto, pdfauthor={Hugo Delaunay}, pdfsubject={Mathématiques}, pdfcreator={agreg.skyost.eu}]{hyperref}
\usepackage{amsmath}
\usepackage{amsthm}
%\usepackage{amssymb}
\usepackage{stmaryrd}
\usepackage{tikz}
\usepackage{tkz-euclide}
\usepackage{fontspec}
\defaultfontfeatures[Erewhon]{FontFace = {bx}{n}{Erewhon-Bold.otf}}
\usepackage{fourier-otf}
\usepackage[nobottomtitles*]{titlesec}
\usepackage{fancyhdr}
\usepackage{listings}
\usepackage{catchfilebetweentags}
\usepackage[french, capitalise, noabbrev]{cleveref}
\usepackage[fit, breakall]{truncate}
\usepackage[top=2.5cm, right=2cm, bottom=2.5cm, left=2cm]{geometry}
\usepackage{enumitem}
\usepackage{tocloft}
\usepackage{microtype}
%\usepackage{mdframed}
%\usepackage{thmtools}
\usepackage{xcolor}
\usepackage{tabularx}
\usepackage{xltabular}
\usepackage{aligned-overset}
\usepackage[subpreambles=true]{standalone}
\usepackage{environ}
\usepackage[normalem]{ulem}
\usepackage{etoolbox}
\usepackage{setspace}
\usepackage[bibstyle=reading, citestyle=draft]{biblatex}
\usepackage{xpatch}
\usepackage[many, breakable]{tcolorbox}
\usepackage[backgroundcolor=white, bordercolor=white, textsize=scriptsize]{todonotes}
\usepackage{luacode}
\usepackage{float}
\usepackage{needspace}
\everymath{\displaystyle}

% Police :

\setmathfont{Erewhon Math}

% Tikz :

\usetikzlibrary{calc}
\usetikzlibrary{3d}

% Longueurs :

\setlength{\parindent}{0pt}
\setlength{\headheight}{15pt}
\setlength{\fboxsep}{0pt}
\titlespacing*{\chapter}{0pt}{-20pt}{10pt}
\setlength{\marginparwidth}{1.5cm}
\setstretch{1.1}

% Métadonnées :

\author{agreg.skyost.eu}
\date{\today}

% Titres :

\setcounter{secnumdepth}{3}

\renewcommand{\thechapter}{\Roman{chapter}}
\renewcommand{\thesubsection}{\Roman{subsection}}
\renewcommand{\thesubsubsection}{\arabic{subsubsection}}
\renewcommand{\theparagraph}{\alph{paragraph}}

\titleformat{\chapter}{\huge\bfseries}{\thechapter}{20pt}{\huge\bfseries}
\titleformat*{\section}{\LARGE\bfseries}
\titleformat{\subsection}{\Large\bfseries}{\thesubsection \, - \,}{0pt}{\Large\bfseries}
\titleformat{\subsubsection}{\large\bfseries}{\thesubsubsection. \,}{0pt}{\large\bfseries}
\titleformat{\paragraph}{\bfseries}{\theparagraph. \,}{0pt}{\bfseries}

\setcounter{secnumdepth}{4}

% Table des matières :

\renewcommand{\cftsecleader}{\cftdotfill{\cftdotsep}}
\addtolength{\cftsecnumwidth}{10pt}

% Redéfinition des commandes :

\renewcommand*\thesection{\arabic{section}}
\renewcommand{\ker}{\mathrm{Ker}}

% Nouvelles commandes :

\newcommand{\website}{https://github.com/imbodj/SenCoursDeMaths}

\newcommand{\tr}[1]{\mathstrut ^t #1}
\newcommand{\im}{\mathrm{Im}}
\newcommand{\rang}{\operatorname{rang}}
\newcommand{\trace}{\operatorname{trace}}
\newcommand{\id}{\operatorname{id}}
\newcommand{\stab}{\operatorname{Stab}}
\newcommand{\paren}[1]{\left(#1\right)}
\newcommand{\croch}[1]{\left[ #1 \right]}
\newcommand{\Grdcroch}[1]{\Bigl[ #1 \Bigr]}
\newcommand{\grdcroch}[1]{\bigl[ #1 \bigr]}
\newcommand{\abs}[1]{\left\lvert #1 \right\rvert}
\newcommand{\limi}[3]{\lim_{#1\to #2}#3}
\newcommand{\pinf}{+\infty}
\newcommand{\minf}{-\infty}
%%%%%%%%%%%%%% ENSEMBLES %%%%%%%%%%%%%%%%%
\newcommand{\ensemblenombre}[1]{\mathbb{#1}}
\newcommand{\Nn}{\ensemblenombre{N}}
\newcommand{\Zz}{\ensemblenombre{Z}}
\newcommand{\Qq}{\ensemblenombre{Q}}
\newcommand{\Qqp}{\Qq^+}
\newcommand{\Rr}{\ensemblenombre{R}}
\newcommand{\Cc}{\ensemblenombre{C}}
\newcommand{\Nne}{\Nn^*}
\newcommand{\Zze}{\Zz^*}
\newcommand{\Zzn}{\Zz^-}
\newcommand{\Qqe}{\Qq^*}
\newcommand{\Rre}{\Rr^*}
\newcommand{\Rrp}{\Rr_+}
\newcommand{\Rrm}{\Rr_-}
\newcommand{\Rrep}{\Rr_+^*}
\newcommand{\Rrem}{\Rr_-^*}
\newcommand{\Cce}{\Cc^*}
%%%%%%%%%%%%%%  INTERVALLES %%%%%%%%%%%%%%%%%
\newcommand{\intff}[2]{\left[#1\;,\; #2\right]  }
\newcommand{\intof}[2]{\left]#1 \;, \;#2\right]  }
\newcommand{\intfo}[2]{\left[#1 \;,\; #2\right[  }
\newcommand{\intoo}[2]{\left]#1 \;,\; #2\right[  }

\providecommand{\newpar}{\\[\medskipamount]}

\newcommand{\annexessection}{%
  \newpage%
  \subsection*{Annexes}%
}

\providecommand{\lesson}[3]{%
  \title{#3}%
  \hypersetup{pdftitle={#2 : #3}}%
  \setcounter{section}{\numexpr #2 - 1}%
  \section{#3}%
  \fancyhead[R]{\truncate{0.73\textwidth}{#2 : #3}}%
}

\providecommand{\development}[3]{%
  \title{#3}%
  \hypersetup{pdftitle={#3}}%
  \section*{#3}%
  \fancyhead[R]{\truncate{0.73\textwidth}{#3}}%
}

\providecommand{\sheet}[3]{\development{#1}{#2}{#3}}

\providecommand{\ranking}[1]{%
  \title{Terminale #1}%
  \hypersetup{pdftitle={Terminale #1}}%
  \section*{Terminale #1}%
  \fancyhead[R]{\truncate{0.73\textwidth}{Terminale #1}}%
}

\providecommand{\summary}[1]{%
  \textit{#1}%
  \par%
  \medskip%
}

\tikzset{notestyleraw/.append style={inner sep=0pt, rounded corners=0pt, align=center}}

%\newcommand{\booklink}[1]{\website/bibliographie\##1}
\newcounter{reference}
\newcommand{\previousreference}{}
\providecommand{\reference}[2][]{%
  \needspace{20pt}%
  \notblank{#1}{
    \needspace{20pt}%
    \renewcommand{\previousreference}{#1}%
    \stepcounter{reference}%
    \label{reference-\previousreference-\thereference}%
  }{}%
  \todo[noline]{%
    \protect\vspace{20pt}%
    \protect\par%
    \protect\notblank{#1}{\cite{[\previousreference]}\\}{}%
    \protect\hyperref[reference-\previousreference-\thereference]{p. #2}%
  }%
}

\definecolor{devcolor}{HTML}{00695c}
\providecommand{\dev}[1]{%
  \reversemarginpar%
  \todo[noline]{
    \protect\vspace{20pt}%
    \protect\par%
    \bfseries\color{devcolor}\href{\website/developpements/#1}{[DEV]}
  }%
  \normalmarginpar%
}

% En-têtes :

\pagestyle{fancy}
\fancyhead[L]{\truncate{0.23\textwidth}{\thepage}}
\fancyfoot[C]{\scriptsize \href{\website}{\texttt{https://github.com/imbodj/SenCoursDeMaths}}}

% Couleurs :

\definecolor{property}{HTML}{ffeb3b}
\definecolor{proposition}{HTML}{ffc107}
\definecolor{lemma}{HTML}{ff9800}
\definecolor{theorem}{HTML}{f44336}
\definecolor{corollary}{HTML}{e91e63}
\definecolor{definition}{HTML}{673ab7}
\definecolor{notation}{HTML}{9c27b0}
\definecolor{example}{HTML}{00bcd4}
\definecolor{cexample}{HTML}{795548}
\definecolor{application}{HTML}{009688}
\definecolor{remark}{HTML}{3f51b5}
\definecolor{algorithm}{HTML}{607d8b}
%\definecolor{proof}{HTML}{e1f5fe}
\definecolor{exercice}{HTML}{e1f5fe}

% Théorèmes :

\theoremstyle{definition}
\newtheorem{theorem}{Théorème}

\newtheorem{property}[theorem]{Propriété}
\newtheorem{proposition}[theorem]{Proposition}
\newtheorem{lemma}[theorem]{Activité d'introduction}
\newtheorem{corollary}[theorem]{Conséquence}

\newtheorem{definition}[theorem]{Définition}
\newtheorem{notation}[theorem]{Notation}

\newtheorem{example}[theorem]{Exemple}
\newtheorem{cexample}[theorem]{Contre-exemple}
\newtheorem{application}[theorem]{Application}

\newtheorem{algorithm}[theorem]{Algorithme}
\newtheorem{exercice}[theorem]{Exercice}

\theoremstyle{remark}
\newtheorem{remark}[theorem]{Remarque}

\counterwithin*{theorem}{section}

\newcommand{\applystyletotheorem}[1]{
  \tcolorboxenvironment{#1}{
    enhanced,
    breakable,
    colback=#1!8!white,
    %right=0pt,
    %top=8pt,
    %bottom=8pt,
    boxrule=0pt,
    frame hidden,
    sharp corners,
    enhanced,borderline west={4pt}{0pt}{#1},
    %interior hidden,
    sharp corners,
    after=\par,
  }
}

\applystyletotheorem{property}
\applystyletotheorem{proposition}
\applystyletotheorem{lemma}
\applystyletotheorem{theorem}
\applystyletotheorem{corollary}
\applystyletotheorem{definition}
\applystyletotheorem{notation}
\applystyletotheorem{example}
\applystyletotheorem{cexample}
\applystyletotheorem{application}
\applystyletotheorem{remark}
%\applystyletotheorem{proof}
\applystyletotheorem{algorithm}
\applystyletotheorem{exercice}

% Environnements :

\NewEnviron{whitetabularx}[1]{%
  \renewcommand{\arraystretch}{2.5}
  \colorbox{white}{%
    \begin{tabularx}{\textwidth}{#1}%
      \BODY%
    \end{tabularx}%
  }%
}

% Maths :

\DeclareFontEncoding{FMS}{}{}
\DeclareFontSubstitution{FMS}{futm}{m}{n}
\DeclareFontEncoding{FMX}{}{}
\DeclareFontSubstitution{FMX}{futm}{m}{n}
\DeclareSymbolFont{fouriersymbols}{FMS}{futm}{m}{n}
\DeclareSymbolFont{fourierlargesymbols}{FMX}{futm}{m}{n}
\DeclareMathDelimiter{\VERT}{\mathord}{fouriersymbols}{152}{fourierlargesymbols}{147}

% Code :

\definecolor{greencode}{rgb}{0,0.6,0}
\definecolor{graycode}{rgb}{0.5,0.5,0.5}
\definecolor{mauvecode}{rgb}{0.58,0,0.82}
\definecolor{bluecode}{HTML}{1976d2}
\lstset{
  basicstyle=\footnotesize\ttfamily,
  breakatwhitespace=false,
  breaklines=true,
  %captionpos=b,
  commentstyle=\color{greencode},
  deletekeywords={...},
  escapeinside={\%*}{*)},
  extendedchars=true,
  frame=none,
  keepspaces=true,
  keywordstyle=\color{bluecode},
  language=Python,
  otherkeywords={*,...},
  numbers=left,
  numbersep=5pt,
  numberstyle=\tiny\color{graycode},
  rulecolor=\color{black},
  showspaces=false,
  showstringspaces=false,
  showtabs=false,
  stepnumber=2,
  stringstyle=\color{mauvecode},
  tabsize=2,
  %texcl=true,
  xleftmargin=10pt,
  %title=\lstname
}

\newcommand{\codedirectory}{}
\newcommand{\inputalgorithm}[1]{%
  \begin{algorithm}%
    \strut%
    \lstinputlisting{\codedirectory#1}%
  \end{algorithm}%
}



% Bibliographie :

%\addbibresource{\bibliographypath}%
\defbibheading{bibliography}[\bibname]{\section*{#1}}
\renewbibmacro*{entryhead:full}{\printfield{labeltitle}}%
\DeclareFieldFormat{url}{\newline\footnotesize\url{#1}}%

\AtEndDocument{%
  \newpage%
  \pagestyle{empty}%
  \printbibliography%
}


\begin{document}
  %<*content>
  \sheet{algebra}{transformee-de-fourier-discrete}{Transformée de Fourier discrète}

  \summary{On dispose en mathématiques de quatre opérations dites ``élémentaires'' : l'addition, la soustraction, la division et donc la multiplication. On sait tous multiplier deux entiers en base $10$ : il suffit de faire la multiplication de chaque chiffre du multiplicateur par chaque chiffre du multiplicande, puis d'additionner le tout. Pour deux nombres de taille $n$, cela donne un algorithme de complexité $O(n^2)$. Mais dès que l'on veut multiplier de très grands chiffres (en informatique par exemple), cet algorithme montre très vite ses limites. Nous allons étudier ici le cas des polynômes en donnant un algorithme de multiplication utilisant la transformée de Fourier rapide.}

  \subsection{Transformée de Fourier discrète sur $\mathbb{C}$}

  \subsubsection{Définitions}

  L'idée va être d'identifier les polynômes de degré inférieur à $n-1$ de la forme $\sum_{k=0}^{n-1} a_k x^k$ au vecteur de $\mathbb{C}^n$ $(a_0, \dots, a_{n-1})$. On fixe, pour toute la suite, $\omega$ une racine primitive $n$-ième de l'unité.

  \begin{definition}
    On appelle \textbf{transformée de Fourier discrète} l'application
    \[
    \operatorname{DFT}_\omega :
    \begin{array}{ccc}
      \mathbb{C}^n &\rightarrow& \mathbb{C}^n \\
      (f_0, \dots, f_{n-1}) &\mapsto& (\sum_{k=0}^{n-1} f_k, \sum_{k=0}^{n-1} f_k \omega^k, \dots, \sum_{k=0}^{n-1} f_k \omega^{(n-1)k})
    \end{array}
    \]
  \end{definition}

  \begin{remark}
    Si $F$ est le polynôme associé au vecteur $f = (f_0, \dots, f_{n-1})$, on a
    \[ \operatorname{DFT}_\omega(f) = (F(1), \dots, F(\omega^{n-1})) \]
    fait que nous utiliserons plusieurs fois dans la suite.
  \end{remark}

  Dans un premier temps, on peut écrire l'algorithme de calcul suivant.

  \inputalgorithm{transformee-de-fourier-discrete-1.py}

  \begin{definition}
    Le \textbf{produit de convolution} de deux vecteurs $f = (f_0, \dots, f_{n-1})$ et $g = (g_0, \dots, g_{n-1})$ de $\mathbb{C}^n$ est le vecteur de $\mathbb{C}^n$ noté $f *_n g$ défini par
    \[ f *_n g = \left( \sum_{i+j \equiv k \mod n} f_ig_j \right)_{k \in \llbracket 0, n-1 \rrbracket} \]
  \end{definition}

  \begin{example}
    $\operatorname{DFT}_{-1}(1,3) = (4, -2)$ et $(1,3) *_2 (1,3) = (1 + 9, 3 + 3) = (10, 6)$.
  \end{example}

  \subsubsection{Propriétés}

  \begin{proposition}
    \label{transformee-de-fourier-discrete-1}
    Soient $f = (f_0, \dots, f_{n-1})$, $g = (g_0, \dots, g_{n-1})$ deux vecteurs de $\mathbb{C}^n$. On pose $h = f *_n g = (h_0, \dots, h_{n-1})$ ainsi que $F$, $G$ et $H$ les polynômes associés respectivement à $f$, $g$ et $h$. Modulo $X^n - 1$, on a :
    \[ \overline{H} = \overline{FG} \]
  \end{proposition}

  \begin{proof}
    Écrivons :
    \[ H = \sum_{k=0}^{n-1} \left( \sum_{i+j \equiv k \mod n} f_i g_j \right) X^k \]
    Comme $H$ est au plus de degré $n-1$, on a $H = \overline{H}$ (par abus de notation). Maintenant avec $F = \sum_{k=0}^{n-1} f_k X^k$ et $G = \sum_{k=0}^{n-1} g_k X^k$, on a :
    \begin{align*}
      FG &= \sum_{k=0}^{2n-2} \left( \sum_{i+j=k} f_i g_{k-i} \right) X^k \\
      &= \sum_{k=0}^{n-1} \left( \sum_{i+j=k} f_i g_{k-i} \right) X^k + \sum_{k=n}^{2n-2} \left( \sum_{i+j=k} f_i g_{k-i} \right) X^k \\
      &= \sum_{k=0}^{n-1} \left( \sum_{i+j=k} f_i g_{k-i} \right) X^k + X^n \left( \sum_{k=0}^{n-2} \left( \sum_{i+j=n+k} f_i g_{k-i} \right) X^k \right)
    \end{align*}
    En passant modulo $X^n - 1$,
    \begin{align*}
      \overline{F}\overline{G} &= \overline{FG} \\
      &= \sum_{k=0}^{n-1} \left( \sum_{i=0}^k f_i g_{k-i} \right) \overline{X^k} + \sum_{k=n}^{2n-2} \left( \sum_{i=0}^k f_i g_{k-i} \right) \overline{X^k} \\
      \overset{\overline{X^n}=\overline{1}}&{=} \sum_{k=0}^{n-1} \left( \sum_{i=0}^k f_i g_{k-i} \right) \overline{X^k} + \sum_{k=0}^{n-2} \left( \sum_{i+j=n+k} f_i g_{k-i} \right) \overline{X^k} \\
      &= \sum_{k=0}^{n-1} \left( \sum_{i+j \equiv k \mod n} f_i g_j \right) \overline{X^k} \\
      &= \overline{H}
    \end{align*}
  \end{proof}

  \begin{theorem}
    \label{transformee-de-fourier-discrete-2}
    $\operatorname{DFT}_\omega$ est un isomorphisme d'algèbres entre $(\mathbb{C}^n, +, *_n)$ et $(\mathbb{C}^n, +, \cdot)$ dont la matrice dans la base canonique est la matrice de Vandermonde :
    \[
    V_\omega =
    \begin{pmatrix}
      1 & 1 & 1 & \dots & 1 \\
      1 & \omega & \omega^2 & \dots & \omega^{n-1} \\
      1 & \omega^2 & \omega^4 & \dots & \omega^{2(n-1)} \\
      \vdots & \vdots & \vdots & \ddots & \vdots \\
      1 & \omega^{n-1} & \omega^{2(n-1)} & \dots & \omega^{(n-1)^2}
    \end{pmatrix}
    \]
    (où $\cdot$ est le produit sur $\mathbb{C}^n$ effectué composante par composante.)
  \end{theorem}

  \begin{proof}
    Soient $f = (f_0, \dots, f_{n-1})$ et $g = (g_0, \dots, g_{n-1})$ deux vecteurs de $\mathbb{C}^n$.
    \begin{itemize}
      \item $\forall \lambda \in \mathbb{C}$,
      \begin{align*}
        \operatorname{DFT}_\omega(\lambda f + g) &= \left(\sum_{k=0}^{n-1} \lambda f_k + g_k, \dots, \sum_{k=0}^{n-1} (\lambda f_k + g_k) \omega^{(n-1)k} \right) \\
        &= \lambda \left(\sum_{k=0}^{n-1} f_k, \dots, \sum_{k=0}^{n-1} f_k \omega^{(n-1)k} \right) + \left(\sum_{k=0}^{n-1} g_k, \dots, \sum_{k=0}^{n-1} g_k \omega^{(n-1)k} \right) \\
        &= \lambda \operatorname{DFT}_\omega(f) + \operatorname{DFT}_\omega(g)
      \end{align*}
      $\operatorname{DFT}_\omega$ est bien une application linéaire.
      \item Soit $h = f *_n g$. On note $F$, $G$ et $H$ les polynômes respectivement associés à $f$, $g$ et $h$.
      \newpar
      Soit $i \in \llbracket 0, n-1 \rrbracket$. Par la \cref{transformee-de-fourier-discrete-1}, on a $H = FG + Q(X^n - 1)$. Ainsi,
      \[ H(\omega^i) = (FG)(\omega^i) + Q(\omega^i)((\omega^i)^n - 1) = (FG)(\omega^i) \]
      Or, le $(i + 1)$-ième coefficient de $\operatorname{DFT}_\omega(f) \cdot \operatorname{DFT}_\omega(g)$ est
      \[ F(\omega^i) G(\omega^i) = (FG)(\omega^i) = H(\omega^i) \]
      Donc $\operatorname{DFT}_\omega(f) \cdot \operatorname{DFT}_\omega(g) = \operatorname{DFT}_\omega(h)$, et ainsi, $\operatorname{DFT}_\omega$ est bien un morphisme d'algèbres.
      \item Clairement,
      \[ \operatorname{DFT}_\omega(f_0, \dots, f_{n-1}) = V_\omega \, \tr{(f_0, \dots, f_{n-1})} \]
      donc la matrice dans la base canonique de $\operatorname{DFT}_\omega$ est $V_\omega$. Or, les $\omega^k$ sont distincts deux-à-deux, donc $V_\omega$ est inversible (car de déterminant non nul, en vertu de la formule de Vandermonde).
    \end{itemize}
  \end{proof}

  \begin{proposition}
    \label{transformee-de-fourier-discrete-3}
    L'inverse de $\operatorname{DFT}_\omega$ est donné par $\frac{1}{n} \operatorname{DFT}_{\omega^{-1}}$.
  \end{proposition}

  \begin{proof}
    Déjà, remarquons que si $\omega$ est une racine primitive $n$-ième de l'unité, alors $\omega^{-1}$ aussi :
    \begin{itemize}
      \item $(\omega^{-1})^n = (\omega^n)^{-1} = 1$.
      \item Et si on a $m < n$ tel que $(\omega^{-1})^m = (\omega^m)^{-1} = 1$, alors $\omega^m = 1$, ce qui est absurde.
    \end{itemize}
    Il suffit donc de montrer que $V_\omega V_{\omega^{-1}} = n I_n$. Soient $i, j \in \llbracket 1, n \rrbracket$. En notant $c_{i,j}$ le coefficient à la $i$-ième ligne et à la $j$-ième colonne de $V_\omega V_{\omega^{-1}}$, on a :
    \[
    c_{i,j}
    = \sum_{k=1}^n \omega^{(i-1)(k-1)} \omega^{(1-j)(k-1)}
    = \sum_{k=0}^{n-1} \omega^{(i-j)k}
    = \begin{cases}
      n &\text{ si } i = j \\
      \frac{1 - \omega^{(i-j)n}}{1 - \omega^{i-j}} \overset{\omega^n = 1}{=} 0 &\text{ sinon}
    \end{cases}
    \]
    Ce qu'on voulait.
  \end{proof}

  \subsubsection{Application à la multiplication de polynômes}

  Notre but ici va être de trouver un moyen de calculer la transformée de Fourier discrète d'un polynôme de manière efficace, puis d'en déduire un algorithme de multiplication de deux polynômes.

  \begin{proposition}
    \label{transformee-de-fourier-discrete-4}
    Soit $n = 2^k$, soit $\omega$ une racine $n$-ième de l'unité et soit $F \in \mathbb{C}[x]$ de degré inférieur ou égal à $n$. On suppose qu'il existe $F_0$ et $F_1$ tels que $F = F_1 X^{\frac{n}{2}} + F_0$ et on pose $R_0 = F_0 + F_1$ et $R_1 = F_0 - F_1$. Alors
    \[ \forall k \in \llbracket 0, \frac{n}{2} \rrbracket, \, F(\omega^{2k}) = R_0(\omega^{2k}) \text{ et } F(\omega^{2k+1}) = R_1(\omega^{2k+1}) \]
  \end{proposition}

  \begin{proof}
    Écrivons $F = F_1 X^{\frac{n}{2}} + F_0$. On a donc
    \[ F - F_0 - F_1 = F_1(x^{\frac{n}{2}} - 1) \iff F = F_1(x^{\frac{n}{2}} - 1) + R_0 \]
    Soit $ k \in \llbracket 0, \frac{n}{2} \rrbracket$, on évalue en $\omega^{2k}$ :
    \[ F(\omega^{2k}) = F_1(\omega^{2k})(\omega^{nk} - 1) + R_0(\omega^{2k}) = R_0(\omega^{2k}) \]
    et la deuxième égalité s'obtient par un calcul similaire (en utilisant le fait que $\omega^{\frac{n}{2}} = -1$).
  \end{proof}

  \begin{proposition}
    \label{transformee-de-fourier-discrete-5}
    Si $\omega$ est une racine primitive $n$-ième de l'unité, alors $\omega^2$ est une racine primitive $\frac{n}{2}$-ième de l'unité.
  \end{proposition}

  \begin{proof}
    Clairement, $\omega^2$ est une racine $\frac{n}{2}$-ième de l'unité. Maintenant, si $d \mid \frac{n}{2}$, alors
    \[ (\omega^2)^d = 1 \iff \omega^{2d} = 1 \iff 2d = n \iff d = \frac{n}{2} \]
  \end{proof}

  On déduit de la \cref{transformee-de-fourier-discrete-4} et de la \cref{transformee-de-fourier-discrete-5} un algorithme récursif de calcul de $\operatorname{DFT}_\omega$ qui a une complexité de $O(n \ln(n))$.

  \inputalgorithm{transformee-de-fourier-discrete-2.py}

  \begin{theorem}
    Soient $F$ et $G$ deux polynômes de degré strictement inférieur à $\frac{n}{2}$ dont on note $f$ et $g$ les vecteurs de $\mathbb{C}^n$ associés. Alors
    \[ FG = H \]
    où $H$ est le polynôme associé au vecteur $\frac{1}{n} \operatorname{DFT}_{\omega^{-1}} (\operatorname{DFT}_\omega(f) \cdot \operatorname{DFT}_\omega(g))$.
  \end{theorem}

  \begin{proof}
    Comme $\deg(FG) < n$, on a $FG = H = \sum_{i=1}^n h_k X^k$ où $h = (h_0, \dots, h_{n-1}) = f *_n g$ par la \cref{transformee-de-fourier-discrete-1}. Par le \cref{transformee-de-fourier-discrete-2}, on a
    \[ \operatorname{DFT}_\omega (h) = \operatorname{DFT}_\omega(f *_n g) = \operatorname{DFT}_\omega(f) \cdot \operatorname{DFT}_\omega(g) \]
    Et $\operatorname{DFT}_\omega^{-1} = \frac{1}{n} \operatorname{DFT}_{\omega^{-1}}$ par la \cref{transformee-de-fourier-discrete-3}. On obtient le résultat voulu.
  \end{proof}

  En utilisant l'algorithme écrit précédemment, on peut donc écrire un nouvel algorithme permettant de calculer le produit de deux polynômes de degré strictement inférieur à $\frac{n}{2}$ en $O(n \ln(n))$.

  \inputalgorithm{transformee-de-fourier-discrete-3.py}

  \begin{remark}
    On pourrait imaginer un algorithme calculant le produit de deux polynômes de degrés quelconques $n$ et $m$ sur le même modèle en considérant $F$ et $G$ comme des polynômes de degré $2^k$ où $k$ est tel que $2^{(k-1)} \leq
    \max(n, m) \leq 2^k$. Il suffit ensuite de choisir $\omega$ racine primitive $2^k$-ième de l'unité.
  \end{remark}
  %</content>
\end{document}
