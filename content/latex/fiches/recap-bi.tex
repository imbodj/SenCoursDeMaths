\input{../common}
\everymath{\displaystyle}
\begin{document}
  %<*content>
  \sheet{algebra}{recap-bi}{Détermination de la nature d'une branche infinie}
 \summary{Récapitulatif de la détermination de la nature d'une branche infinie en $ \pinf $ ou en $ \pinf $}
   
\begin{center}
\begin{tikzpicture}[
  man/.style={rectangle,draw,fill=blue!20,text width=2.5cm,align=center,minimum height=0.8cm},
  woman/.style={rectangle,draw,fill=red!20,rounded corners=.8ex,text width=3cm,align=center,minimum height=0.8cm},
  node distance=1.5cm and 2cm]
    
  % Placement manuel des nœuds
  \node[man] (root) {$ \displaystyle\lim_{x \to +\infty}f(x) $};
  
  % Niveau 1
  \node[man, below left=of root] (inf) {$ \infty $};
  \node[man, below right=of root] (b) {$ b $};
  
  % Niveau 2
  \node[woman, below=of inf] (ratio) {$ \displaystyle\lim_{x \to +\infty}\dfrac{f(x)}{x} $};
  \node[woman, below=of b] (horiz) {asymptote horizontale $y=b$};
  
  % Niveau 3
  \node[man, below left=1.5cm and 3cm of ratio] (zero) {$ 0 $};
  \node[man, below=1.5cm of ratio] (a) {$ a \neq 0 $};
  \node[man, below right=1.5cm and 3cm of ratio] (inf2) {$ \infty $};
  
  % Niveau 4
  \node[woman, below=of zero] (bp_ox) {BP de direction $(Ox)$};
  \node[woman, below=of a] (limit_ax) {$ \displaystyle\lim_{x \to +\infty}(f(x)-ax) $};
  \node[woman, below=of inf2] (bp_oy) {BP de direction $(Oy)$};
  
  % Niveau 5
  \node[man, below left=1.5cm and 2cm of limit_ax] (b_final) {$ b $};
  \node[man, below=1.5cm of limit_ax] (nexiste) {n'existe pas};
  \node[man, below right=1.5cm and 2cm of limit_ax] (inf3) {$ \infty $};
  
  % Niveau 6
  \node[woman, below=of b_final] (asymp) {asymptote $y=ax+b$};
  \node[woman, below=of nexiste] (dir_asymp) {direction asymptotique $y=ax$};
  \node[woman, below=of inf3] (bp_ax) {BP de direction $y=ax$};
  
  % Connexions
  \draw (root) -- (inf);
  \draw (root) -- (b);
  \draw (inf) -- (ratio);
  \draw (b) -- (horiz);
  \draw (ratio) -- (zero);
  \draw (ratio) -- (a);
  \draw (ratio) -- (inf2);
  \draw (zero) -- (bp_ox);
  \draw (a) -- (limit_ax);
  \draw (inf2) -- (bp_oy);
  \draw (limit_ax) -- (b_final);
  \draw (limit_ax) -- (nexiste);
  \draw (limit_ax) -- (inf3);
  \draw (b_final) -- (asymp);
  \draw (nexiste) -- (dir_asymp);
  \draw (inf3) -- (bp_ax);
  
\end{tikzpicture}
\end{center}
\begin{remark}

Si $ f $ s'écrit sous la forme $ f(x)= ax+b + g(x) $ et si $\displaystyle \lim_{x \to \infty}g(x)=c $  alors la droite $ y=ax+b+c $ est une asymptote à  $ \mathcal{C}_{f} $ en $ \infty. $
\end{remark}
  %</content>
\end{document}
