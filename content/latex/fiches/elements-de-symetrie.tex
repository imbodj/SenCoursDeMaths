\documentclass[12pt, a4paper]{report}

% LuaLaTeX :

\RequirePackage{iftex}
\RequireLuaTeX

% Packages :

\usepackage[french]{babel}
%\usepackage[utf8]{inputenc}
%\usepackage[T1]{fontenc}
\usepackage[pdfencoding=auto, pdfauthor={Hugo Delaunay}, pdfsubject={Mathématiques}, pdfcreator={agreg.skyost.eu}]{hyperref}
\usepackage{amsmath}
\usepackage{amsthm}
%\usepackage{amssymb}
\usepackage{stmaryrd}
\usepackage{tikz}
\usepackage{tkz-euclide}
\usepackage{fontspec}
\defaultfontfeatures[Erewhon]{FontFace = {bx}{n}{Erewhon-Bold.otf}}
\usepackage{fourier-otf}
\usepackage[nobottomtitles*]{titlesec}
\usepackage{fancyhdr}
\usepackage{listings}
\usepackage{catchfilebetweentags}
\usepackage[french, capitalise, noabbrev]{cleveref}
\usepackage[fit, breakall]{truncate}
\usepackage[top=2.5cm, right=2cm, bottom=2.5cm, left=2cm]{geometry}
\usepackage{enumitem}
\usepackage{tocloft}
\usepackage{microtype}
%\usepackage{mdframed}
%\usepackage{thmtools}
\usepackage{xcolor}
\usepackage{tabularx}
\usepackage{xltabular}
\usepackage{aligned-overset}
\usepackage[subpreambles=true]{standalone}
\usepackage{environ}
\usepackage[normalem]{ulem}
\usepackage{etoolbox}
\usepackage{setspace}
\usepackage[bibstyle=reading, citestyle=draft]{biblatex}
\usepackage{xpatch}
\usepackage[many, breakable]{tcolorbox}
\usepackage[backgroundcolor=white, bordercolor=white, textsize=scriptsize]{todonotes}
\usepackage{luacode}
\usepackage{float}
\usepackage{needspace}
\everymath{\displaystyle}

% Police :

\setmathfont{Erewhon Math}

% Tikz :

\usetikzlibrary{calc}
\usetikzlibrary{3d}

% Longueurs :

\setlength{\parindent}{0pt}
\setlength{\headheight}{15pt}
\setlength{\fboxsep}{0pt}
\titlespacing*{\chapter}{0pt}{-20pt}{10pt}
\setlength{\marginparwidth}{1.5cm}
\setstretch{1.1}

% Métadonnées :

\author{agreg.skyost.eu}
\date{\today}

% Titres :

\setcounter{secnumdepth}{3}

\renewcommand{\thechapter}{\Roman{chapter}}
\renewcommand{\thesubsection}{\Roman{subsection}}
\renewcommand{\thesubsubsection}{\arabic{subsubsection}}
\renewcommand{\theparagraph}{\alph{paragraph}}

\titleformat{\chapter}{\huge\bfseries}{\thechapter}{20pt}{\huge\bfseries}
\titleformat*{\section}{\LARGE\bfseries}
\titleformat{\subsection}{\Large\bfseries}{\thesubsection \, - \,}{0pt}{\Large\bfseries}
\titleformat{\subsubsection}{\large\bfseries}{\thesubsubsection. \,}{0pt}{\large\bfseries}
\titleformat{\paragraph}{\bfseries}{\theparagraph. \,}{0pt}{\bfseries}

\setcounter{secnumdepth}{4}

% Table des matières :

\renewcommand{\cftsecleader}{\cftdotfill{\cftdotsep}}
\addtolength{\cftsecnumwidth}{10pt}

% Redéfinition des commandes :

\renewcommand*\thesection{\arabic{section}}
\renewcommand{\ker}{\mathrm{Ker}}

% Nouvelles commandes :

\newcommand{\website}{https://github.com/imbodj/SenCoursDeMaths}

\newcommand{\tr}[1]{\mathstrut ^t #1}
\newcommand{\im}{\mathrm{Im}}
\newcommand{\rang}{\operatorname{rang}}
\newcommand{\trace}{\operatorname{trace}}
\newcommand{\id}{\operatorname{id}}
\newcommand{\stab}{\operatorname{Stab}}
\newcommand{\paren}[1]{\left(#1\right)}
\newcommand{\croch}[1]{\left[ #1 \right]}
\newcommand{\Grdcroch}[1]{\Bigl[ #1 \Bigr]}
\newcommand{\grdcroch}[1]{\bigl[ #1 \bigr]}
\newcommand{\abs}[1]{\left\lvert #1 \right\rvert}
\newcommand{\limi}[3]{\lim_{#1\to #2}#3}
\newcommand{\pinf}{+\infty}
\newcommand{\minf}{-\infty}
%%%%%%%%%%%%%% ENSEMBLES %%%%%%%%%%%%%%%%%
\newcommand{\ensemblenombre}[1]{\mathbb{#1}}
\newcommand{\Nn}{\ensemblenombre{N}}
\newcommand{\Zz}{\ensemblenombre{Z}}
\newcommand{\Qq}{\ensemblenombre{Q}}
\newcommand{\Qqp}{\Qq^+}
\newcommand{\Rr}{\ensemblenombre{R}}
\newcommand{\Cc}{\ensemblenombre{C}}
\newcommand{\Nne}{\Nn^*}
\newcommand{\Zze}{\Zz^*}
\newcommand{\Zzn}{\Zz^-}
\newcommand{\Qqe}{\Qq^*}
\newcommand{\Rre}{\Rr^*}
\newcommand{\Rrp}{\Rr_+}
\newcommand{\Rrm}{\Rr_-}
\newcommand{\Rrep}{\Rr_+^*}
\newcommand{\Rrem}{\Rr_-^*}
\newcommand{\Cce}{\Cc^*}
%%%%%%%%%%%%%%  INTERVALLES %%%%%%%%%%%%%%%%%
\newcommand{\intff}[2]{\left[#1\;,\; #2\right]  }
\newcommand{\intof}[2]{\left]#1 \;, \;#2\right]  }
\newcommand{\intfo}[2]{\left[#1 \;,\; #2\right[  }
\newcommand{\intoo}[2]{\left]#1 \;,\; #2\right[  }

\providecommand{\newpar}{\\[\medskipamount]}

\newcommand{\annexessection}{%
  \newpage%
  \subsection*{Annexes}%
}

\providecommand{\lesson}[3]{%
  \title{#3}%
  \hypersetup{pdftitle={#2 : #3}}%
  \setcounter{section}{\numexpr #2 - 1}%
  \section{#3}%
  \fancyhead[R]{\truncate{0.73\textwidth}{#2 : #3}}%
}

\providecommand{\development}[3]{%
  \title{#3}%
  \hypersetup{pdftitle={#3}}%
  \section*{#3}%
  \fancyhead[R]{\truncate{0.73\textwidth}{#3}}%
}

\providecommand{\sheet}[3]{\development{#1}{#2}{#3}}

\providecommand{\ranking}[1]{%
  \title{Terminale #1}%
  \hypersetup{pdftitle={Terminale #1}}%
  \section*{Terminale #1}%
  \fancyhead[R]{\truncate{0.73\textwidth}{Terminale #1}}%
}

\providecommand{\summary}[1]{%
  \textit{#1}%
  \par%
  \medskip%
}

\tikzset{notestyleraw/.append style={inner sep=0pt, rounded corners=0pt, align=center}}

%\newcommand{\booklink}[1]{\website/bibliographie\##1}
\newcounter{reference}
\newcommand{\previousreference}{}
\providecommand{\reference}[2][]{%
  \needspace{20pt}%
  \notblank{#1}{
    \needspace{20pt}%
    \renewcommand{\previousreference}{#1}%
    \stepcounter{reference}%
    \label{reference-\previousreference-\thereference}%
  }{}%
  \todo[noline]{%
    \protect\vspace{20pt}%
    \protect\par%
    \protect\notblank{#1}{\cite{[\previousreference]}\\}{}%
    \protect\hyperref[reference-\previousreference-\thereference]{p. #2}%
  }%
}

\definecolor{devcolor}{HTML}{00695c}
\providecommand{\dev}[1]{%
  \reversemarginpar%
  \todo[noline]{
    \protect\vspace{20pt}%
    \protect\par%
    \bfseries\color{devcolor}\href{\website/developpements/#1}{[DEV]}
  }%
  \normalmarginpar%
}

% En-têtes :

\pagestyle{fancy}
\fancyhead[L]{\truncate{0.23\textwidth}{\thepage}}
\fancyfoot[C]{\scriptsize \href{\website}{\texttt{https://github.com/imbodj/SenCoursDeMaths}}}

% Couleurs :

\definecolor{property}{HTML}{ffeb3b}
\definecolor{proposition}{HTML}{ffc107}
\definecolor{lemma}{HTML}{ff9800}
\definecolor{theorem}{HTML}{f44336}
\definecolor{corollary}{HTML}{e91e63}
\definecolor{definition}{HTML}{673ab7}
\definecolor{notation}{HTML}{9c27b0}
\definecolor{example}{HTML}{00bcd4}
\definecolor{cexample}{HTML}{795548}
\definecolor{application}{HTML}{009688}
\definecolor{remark}{HTML}{3f51b5}
\definecolor{algorithm}{HTML}{607d8b}
%\definecolor{proof}{HTML}{e1f5fe}
\definecolor{exercice}{HTML}{e1f5fe}

% Théorèmes :

\theoremstyle{definition}
\newtheorem{theorem}{Théorème}

\newtheorem{property}[theorem]{Propriété}
\newtheorem{proposition}[theorem]{Proposition}
\newtheorem{lemma}[theorem]{Activité d'introduction}
\newtheorem{corollary}[theorem]{Conséquence}

\newtheorem{definition}[theorem]{Définition}
\newtheorem{notation}[theorem]{Notation}

\newtheorem{example}[theorem]{Exemple}
\newtheorem{cexample}[theorem]{Contre-exemple}
\newtheorem{application}[theorem]{Application}

\newtheorem{algorithm}[theorem]{Algorithme}
\newtheorem{exercice}[theorem]{Exercice}

\theoremstyle{remark}
\newtheorem{remark}[theorem]{Remarque}

\counterwithin*{theorem}{section}

\newcommand{\applystyletotheorem}[1]{
  \tcolorboxenvironment{#1}{
    enhanced,
    breakable,
    colback=#1!8!white,
    %right=0pt,
    %top=8pt,
    %bottom=8pt,
    boxrule=0pt,
    frame hidden,
    sharp corners,
    enhanced,borderline west={4pt}{0pt}{#1},
    %interior hidden,
    sharp corners,
    after=\par,
  }
}

\applystyletotheorem{property}
\applystyletotheorem{proposition}
\applystyletotheorem{lemma}
\applystyletotheorem{theorem}
\applystyletotheorem{corollary}
\applystyletotheorem{definition}
\applystyletotheorem{notation}
\applystyletotheorem{example}
\applystyletotheorem{cexample}
\applystyletotheorem{application}
\applystyletotheorem{remark}
%\applystyletotheorem{proof}
\applystyletotheorem{algorithm}
\applystyletotheorem{exercice}

% Environnements :

\NewEnviron{whitetabularx}[1]{%
  \renewcommand{\arraystretch}{2.5}
  \colorbox{white}{%
    \begin{tabularx}{\textwidth}{#1}%
      \BODY%
    \end{tabularx}%
  }%
}

% Maths :

\DeclareFontEncoding{FMS}{}{}
\DeclareFontSubstitution{FMS}{futm}{m}{n}
\DeclareFontEncoding{FMX}{}{}
\DeclareFontSubstitution{FMX}{futm}{m}{n}
\DeclareSymbolFont{fouriersymbols}{FMS}{futm}{m}{n}
\DeclareSymbolFont{fourierlargesymbols}{FMX}{futm}{m}{n}
\DeclareMathDelimiter{\VERT}{\mathord}{fouriersymbols}{152}{fourierlargesymbols}{147}

% Code :

\definecolor{greencode}{rgb}{0,0.6,0}
\definecolor{graycode}{rgb}{0.5,0.5,0.5}
\definecolor{mauvecode}{rgb}{0.58,0,0.82}
\definecolor{bluecode}{HTML}{1976d2}
\lstset{
  basicstyle=\footnotesize\ttfamily,
  breakatwhitespace=false,
  breaklines=true,
  %captionpos=b,
  commentstyle=\color{greencode},
  deletekeywords={...},
  escapeinside={\%*}{*)},
  extendedchars=true,
  frame=none,
  keepspaces=true,
  keywordstyle=\color{bluecode},
  language=Python,
  otherkeywords={*,...},
  numbers=left,
  numbersep=5pt,
  numberstyle=\tiny\color{graycode},
  rulecolor=\color{black},
  showspaces=false,
  showstringspaces=false,
  showtabs=false,
  stepnumber=2,
  stringstyle=\color{mauvecode},
  tabsize=2,
  %texcl=true,
  xleftmargin=10pt,
  %title=\lstname
}

\newcommand{\codedirectory}{}
\newcommand{\inputalgorithm}[1]{%
  \begin{algorithm}%
    \strut%
    \lstinputlisting{\codedirectory#1}%
  \end{algorithm}%
}



\everymath{\displaystyle}
\begin{document}
  %<*content>
  \sheet{algebra}{elements-de-symetrie}{Éléments de symétrie d'une courbe}
 \summary{}
     
   
Soit $ f $ une fonction numérique et $ \mathcal{C} $ sa courbe représentative dans un repère orthogonal.

\subsection{Axe de symétrie}
Pour montrer que la droite $ \Delta $ d'équation $ x=a $ est un axe de symétrie de la courbe $ \mathcal{C} $,  on peut utiliser l'une des méthodes suivantes: 
\begin{itemize}
\item Démontrer que : $ \forall x\in D_{f}$  on a   $ 2a-x\in D_{f}$ et $ f(a-x)=f(x) $
\item Démontrer que : $ \forall x\in D_{f}$  on a  $ a-x\in D_{f}$, $ a+x\in D_{f}$ et $ f(2a-x)=f(a+x) $
\item Démontrer que : la fonction $ g(x)=f(a-x) $ est paire.
\end{itemize}
Dans ce cas on peut restreindre l'étude de $ f $ à $ \intfo{a}{\pinf}\cap D_{f}$ et on obtient la courbe complète par symétrie par rapport à la droite $ \Delta $.

\subsection{Centre de symétrie}
Pour montrer que le point $ I(a, b) $ est un axe de symétrie de la courbe $ \mathcal{C} $,  on peut utiliser l'une des méthodes suivantes : 
\begin{itemize}
\item Démontrer que : $ \forall x\in D_{f}$  on a  $ 2a-x\in D_{f}$ et $ f(2a-x)+f(x)=2b $
\item Démontrer que : $ \forall x\in D_{f}$  on a   $ a-x\in D_{f}$,  $ a+x\in D_{f}$ et $ f(a-x)+f(a+x)=2b $
\item Démontrer que : la fonction $ g(x)=f(a-x)+b$ est impaire.
\end{itemize}
Dans ce cas on peut restreindre l'étude de $ f $ à $ \intfo{a}{\pinf}\cap D_{f}$ et on obtient la courbe complète par symétrie par rapport à $ I. $

\subsection{Fonction périodique}
\begin{definition}
 Une fonction $ f $ est dite
\textbf{\color{magenta} périodique de période $ t $} (ou t- périodique) ssi :\\
 $ \centerdot $ t est non nul, \\
$ \centerdot $  pour tout $x\in D_{f} $, \   $  x+t$ et $ x-t$   sont dans $ D_{f} $           et  $f(x+t)=f(x) $.\\
On dit que \textbf{\color{magenta} $t$ est une période de $f$}, et la plus petite période strictement positive est  \textbf{\color{magenta} la période de $f$}.  En général la période est notée $ T $. \\
Pour tout $x$ de $ D_{f} $ et tout \textbf{\color{magenta} entier relatif } $ k $, \colorbox{gray!20!}{$ f(x+kT) = f(x) $. }
\end{definition}

\textbf{\color{blue}Conséquences}

Pour représenter graphiquement une fonction $ f $ de période $ T $, il suffit de :
\begin{itemize}
\item choisir un intervalle $ I $ de longueur $ T $ inclus dans $ D_{f} $;
\item tracer (en rouge )la partie $ \mathcal{C} $ de la courbe de $ f $ restreinte à cet intervalle $ I: $
\item translater la partie $ \mathcal{C} $ par les translations de vecteurs $ (kT)\overrightarrow{i} $ avec $ k $ entier relatif.
\end{itemize}
\begin{tikzpicture}[scale=0.375]
    % Configuration des limites
    \def\xmin{-15}
    \def\xmax{15}
    \def\ymin{-3}
    \def\ymax{3}
    
    % Axes avec flèches
    \draw[thick,->] (\xmin,0) -- (\xmax,0);
    \draw[thick,->] (0,\ymin) -- (0,\ymax);
    
    % Graduations sur les axes
    \foreach \x in {\xmin,...,-1,1,2,...,\xmax} {
        \draw (\x,0.1) -- (\x,-0.1);
    }
    \foreach \y in {\ymin,...,-1,1,2,...,\ymax} {
        \draw (0.1,\y) -- (-0.1,\y);
    }
    
    % Origine
    \node[below left] at (0,0) {$O$};
    
    % Vecteurs unitaires
    \draw[thick,->] (0,0) -- (1,0);
    \node[below] at (0.5,0) {$\vec{\imath}$};
    \draw[thick,->] (0,0) -- (0,1);
    \node[left] at (0,0.5) {$\vec{\jmath}$};
    
    % Fonction sin(x) sur [-π, π] en rouge
    \draw[red, thick, domain=-3.14159:3.14159, samples=200, smooth] 
        plot (\x, {sin(\x r)});
    
    % Fonction sin(x) sur [π, xmax] en noir
    \draw[black, thick, domain=3.14159:\xmax, samples=200, smooth] 
        plot (\x, {sin(\x r)});
    
    % Fonction sin(x) sur [xmin, -π] en noir
    \draw[black, thick, domain=\xmin:-3.14159, samples=200, smooth] 
        plot (\x, {sin(\x r)});
\end{tikzpicture}

\textbf{\color{blue}Cas des fonctions trigonométriques}\\
$ - $ Les fonctions $x\mapsto\cos x $ et $x\mapsto\sin x $ sont périodiques de période $ 2\pi $  c'est à dire: \colorbox{green!20!}{ $ \cos (x+2\pi)=\cos x $ et  $ \sin (x+2\pi)=\sin x $ }\\

$ - $ La fonction $x\mapsto\tan x $ est périodique de période $ \pi $ c'est à dire:       \colorbox{green!20!} {$ \tan (x+\pi)=\tan x $}\\

\colorbox{magenta!20!}{Cas général }: \\
Les fonctions $x \mapsto \cos (ax+b) $ et $x \mapsto \sin (ax+b) $ ont pour période $ T=\dfrac{2\pi}{\abs{a}}$ \\
La fonction $x \mapsto \tan (ax+b) $ a pour période $ T=\dfrac{\pi}{\abs{a}}$ 

\subsubsection*{Réduction de domaine d'étude}

\begin{itemize}
\item[\textbullet] Si $ f $ est T-périodique alors on peut restreindre le domaine d'étude à tout domaine du type $ \intff{a}{a+T}\cap D_{f} $ pour tout réel $ a, $ ainsi on obtient la courbe complète de $ f$ sur ce domaine.
\item[\textbullet] Si $ f $ est T-périodique   et paire ( resp. impaire ) alors on peut restreindre le domaine d'étude au domaine  $ \intff{0}{\frac{T}{2}}\cap D_{f} $  ainsi on obtient la courbe complète sur $ \intff{-\frac{T}{2}}{\frac{T}{2}}\cap D_{f} $ par symétrie par rapport à l'axe des ordonnées (resp. par symétrie par rapport à O origine du repère ).
\item[\textbullet] Si $ f $ est T-périodique   et $ \mathcal{C} $  admet un axe de symétrie $ \Delta $ ( resp. un centre de symétrie $ I $ ) alors on peut restreindre le domaine d'étude au domaine \\ $ \intff{a}{a+\frac{T}{2}}\cap D_{f} $ ou au domaine  $ \intff{a-\frac{T}{2}}{a}\cap D_{f} $ ainsi on obtient la courbe complète sur $ \intff{a-\frac{T}{2}}{a+\frac{T}{2}}\cap D_{f} $ par symétrie par rapport à l'axe $ \Delta $  (resp. par symétrie par rapport un point $ I $ ).
\end{itemize}

 \begin{exercice}
On considère  la fonction $ f $ définie par: $ f(x)= \dfrac{\sin x }{\sin x+\cos x} $ 
\begin{enumerate}
\item Déterminer $ D_{f} $ et  puis montrer que  $ f$ est de période  $ \pi. $
\item Montrer que le point  $ A(-\frac{\pi}{4}; \frac{1}{2}) $ est un centre de symétrie de $ \mathcal{C}_{f} $. En déduire un domaine simple pour l'étude de $ f $ 
\end{enumerate}
 \end{exercice}

\begin{proof}
\begin{enumerate}
\item $ f(x) $ existe    $\Leftrightarrow \sin x+\cos x \neq 0  $\\
 $\Leftrightarrow \sin x \neq -\cos x $\\
 $\Leftrightarrow \sin x \neq \sin (x- \frac{\pi}{2}) $\\
 $ \Leftrightarrow x\neq \frac{3\pi}{4}+ k\pi , k\in \Zz$\\
 Montrons que $ \pi $ est la période de $ f. $\\
 $ x \in D_{f}\Leftrightarrow x\neq \frac{3\pi}{4}+ k\pi \Leftrightarrow x+\pi\neq \frac{7\pi}{4}+ k\pi \Leftrightarrow x+\pi\in D_{f}.$\\
  $ x \in D_{f}\Leftrightarrow x\neq \frac{3\pi}{4}+ k\pi \Leftrightarrow x-\pi\neq -\frac{\pi}{4}+ k\pi \Leftrightarrow x-\pi\in D_{f}.$\\
 $ f(x+\pi)=\frac{\sin (x+\pi) }{\sin (x+\pi)+\cos (x+\pi)}= \frac{-\sin x }{-\sin x-\cos x}=\frac{\sin x }{\sin x+\cos x}=f(x)$
 \item $ 2a-x=-\frac{\pi}{2}-x $\\
  $ x \in D_{f}\Leftrightarrow x\neq \frac{3\pi}{4}+ k\pi \Leftrightarrow -\frac{\pi}{2}-x \neq -\frac{7\pi}{4}+ k\pi \Leftrightarrow -\frac{\pi}{2}-x\in D_{f} \\\Leftrightarrow 2a-x\in D_{f}$
 $$ f(2a-x)+f(x)= \frac{\sin (-\frac{\pi}{2}-x ) }{\sin (-\frac{\pi}{2}-x )+\cos (-\frac{\pi}{2}-x )}+\frac{\sin x }{\sin x+\cos x} $$
 $$ f(2a-x)+f(x)= \frac{\cos x }{\cos x +\sin x }+\frac{\sin x }{\sin x+\cos x}=1 $$
 Donc le point $ A $ est bien un centre de symétrie de $ \mathcal{C}_{f} $.\\
 Proposons un d'étude   de $f. $\\
$ f $ est de période $ T= \pi$ et l'abscisse  du centre de symétrie est $ a=-\frac{\pi}{4}  $ , on peut appliquer la formule $ \intff{a}{a+\frac{T}{2}}\cap D_{f} $ \\
$ \intff{-\frac{\pi}{4}}{-\frac{\pi}{4}+\frac{\pi}{2}}\cap D_{f} =\intff{-\frac{\pi}{4}}{\frac{\pi}{4}}\cap D_{f}=\intof{-\frac{\pi}{4}}{\frac{\pi}{4}}$\\
\underline{Conclusion }: on peut étudier $ f $ sur $\intof{-\frac{\pi}{4}}{\frac{\pi}{4}}$ puis obtenir la courbe complète par symétrie par rapport à $ A $ sur $ \intfo{-\frac{3\pi}{4}}{-\frac{\pi}{4}}\cup \intof{-\frac{\pi}{4}}{\frac{\pi}{4}} $.
 
 
\end{enumerate}
\end{proof}
  %</content>
\end{document}
