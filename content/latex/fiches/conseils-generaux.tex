\documentclass[12pt, a4paper]{report}

% LuaLaTeX :

\RequirePackage{iftex}
\RequireLuaTeX

% Packages :

\usepackage[french]{babel}
%\usepackage[utf8]{inputenc}
%\usepackage[T1]{fontenc}
\usepackage[pdfencoding=auto, pdfauthor={Hugo Delaunay}, pdfsubject={Mathématiques}, pdfcreator={agreg.skyost.eu}]{hyperref}
\usepackage{amsmath}
\usepackage{amsthm}
%\usepackage{amssymb}
\usepackage{stmaryrd}
\usepackage{tikz}
\usepackage{tkz-euclide}
\usepackage{fontspec}
\defaultfontfeatures[Erewhon]{FontFace = {bx}{n}{Erewhon-Bold.otf}}
\usepackage{fourier-otf}
\usepackage[nobottomtitles*]{titlesec}
\usepackage{fancyhdr}
\usepackage{listings}
\usepackage{catchfilebetweentags}
\usepackage[french, capitalise, noabbrev]{cleveref}
\usepackage[fit, breakall]{truncate}
\usepackage[top=2.5cm, right=2cm, bottom=2.5cm, left=2cm]{geometry}
\usepackage{enumitem}
\usepackage{tocloft}
\usepackage{microtype}
%\usepackage{mdframed}
%\usepackage{thmtools}
\usepackage{xcolor}
\usepackage{tabularx}
\usepackage{xltabular}
\usepackage{aligned-overset}
\usepackage[subpreambles=true]{standalone}
\usepackage{environ}
\usepackage[normalem]{ulem}
\usepackage{etoolbox}
\usepackage{setspace}
\usepackage[bibstyle=reading, citestyle=draft]{biblatex}
\usepackage{xpatch}
\usepackage[many, breakable]{tcolorbox}
\usepackage[backgroundcolor=white, bordercolor=white, textsize=scriptsize]{todonotes}
\usepackage{luacode}
\usepackage{float}
\usepackage{needspace}
\everymath{\displaystyle}

% Police :

\setmathfont{Erewhon Math}

% Tikz :

\usetikzlibrary{calc}
\usetikzlibrary{3d}

% Longueurs :

\setlength{\parindent}{0pt}
\setlength{\headheight}{15pt}
\setlength{\fboxsep}{0pt}
\titlespacing*{\chapter}{0pt}{-20pt}{10pt}
\setlength{\marginparwidth}{1.5cm}
\setstretch{1.1}

% Métadonnées :

\author{agreg.skyost.eu}
\date{\today}

% Titres :

\setcounter{secnumdepth}{3}

\renewcommand{\thechapter}{\Roman{chapter}}
\renewcommand{\thesubsection}{\Roman{subsection}}
\renewcommand{\thesubsubsection}{\arabic{subsubsection}}
\renewcommand{\theparagraph}{\alph{paragraph}}

\titleformat{\chapter}{\huge\bfseries}{\thechapter}{20pt}{\huge\bfseries}
\titleformat*{\section}{\LARGE\bfseries}
\titleformat{\subsection}{\Large\bfseries}{\thesubsection \, - \,}{0pt}{\Large\bfseries}
\titleformat{\subsubsection}{\large\bfseries}{\thesubsubsection. \,}{0pt}{\large\bfseries}
\titleformat{\paragraph}{\bfseries}{\theparagraph. \,}{0pt}{\bfseries}

\setcounter{secnumdepth}{4}

% Table des matières :

\renewcommand{\cftsecleader}{\cftdotfill{\cftdotsep}}
\addtolength{\cftsecnumwidth}{10pt}

% Redéfinition des commandes :

\renewcommand*\thesection{\arabic{section}}
\renewcommand{\ker}{\mathrm{Ker}}

% Nouvelles commandes :

\newcommand{\website}{https://github.com/imbodj/SenCoursDeMaths}

\newcommand{\tr}[1]{\mathstrut ^t #1}
\newcommand{\im}{\mathrm{Im}}
\newcommand{\rang}{\operatorname{rang}}
\newcommand{\trace}{\operatorname{trace}}
\newcommand{\id}{\operatorname{id}}
\newcommand{\stab}{\operatorname{Stab}}
\newcommand{\paren}[1]{\left(#1\right)}
\newcommand{\croch}[1]{\left[ #1 \right]}
\newcommand{\Grdcroch}[1]{\Bigl[ #1 \Bigr]}
\newcommand{\grdcroch}[1]{\bigl[ #1 \bigr]}
\newcommand{\abs}[1]{\left\lvert #1 \right\rvert}
\newcommand{\limi}[3]{\lim_{#1\to #2}#3}
\newcommand{\pinf}{+\infty}
\newcommand{\minf}{-\infty}
%%%%%%%%%%%%%% ENSEMBLES %%%%%%%%%%%%%%%%%
\newcommand{\ensemblenombre}[1]{\mathbb{#1}}
\newcommand{\Nn}{\ensemblenombre{N}}
\newcommand{\Zz}{\ensemblenombre{Z}}
\newcommand{\Qq}{\ensemblenombre{Q}}
\newcommand{\Qqp}{\Qq^+}
\newcommand{\Rr}{\ensemblenombre{R}}
\newcommand{\Cc}{\ensemblenombre{C}}
\newcommand{\Nne}{\Nn^*}
\newcommand{\Zze}{\Zz^*}
\newcommand{\Zzn}{\Zz^-}
\newcommand{\Qqe}{\Qq^*}
\newcommand{\Rre}{\Rr^*}
\newcommand{\Rrp}{\Rr_+}
\newcommand{\Rrm}{\Rr_-}
\newcommand{\Rrep}{\Rr_+^*}
\newcommand{\Rrem}{\Rr_-^*}
\newcommand{\Cce}{\Cc^*}
%%%%%%%%%%%%%%  INTERVALLES %%%%%%%%%%%%%%%%%
\newcommand{\intff}[2]{\left[#1\;,\; #2\right]  }
\newcommand{\intof}[2]{\left]#1 \;, \;#2\right]  }
\newcommand{\intfo}[2]{\left[#1 \;,\; #2\right[  }
\newcommand{\intoo}[2]{\left]#1 \;,\; #2\right[  }

\providecommand{\newpar}{\\[\medskipamount]}

\newcommand{\annexessection}{%
  \newpage%
  \subsection*{Annexes}%
}

\providecommand{\lesson}[3]{%
  \title{#3}%
  \hypersetup{pdftitle={#2 : #3}}%
  \setcounter{section}{\numexpr #2 - 1}%
  \section{#3}%
  \fancyhead[R]{\truncate{0.73\textwidth}{#2 : #3}}%
}

\providecommand{\development}[3]{%
  \title{#3}%
  \hypersetup{pdftitle={#3}}%
  \section*{#3}%
  \fancyhead[R]{\truncate{0.73\textwidth}{#3}}%
}

\providecommand{\sheet}[3]{\development{#1}{#2}{#3}}

\providecommand{\ranking}[1]{%
  \title{Terminale #1}%
  \hypersetup{pdftitle={Terminale #1}}%
  \section*{Terminale #1}%
  \fancyhead[R]{\truncate{0.73\textwidth}{Terminale #1}}%
}

\providecommand{\summary}[1]{%
  \textit{#1}%
  \par%
  \medskip%
}

\tikzset{notestyleraw/.append style={inner sep=0pt, rounded corners=0pt, align=center}}

%\newcommand{\booklink}[1]{\website/bibliographie\##1}
\newcounter{reference}
\newcommand{\previousreference}{}
\providecommand{\reference}[2][]{%
  \needspace{20pt}%
  \notblank{#1}{
    \needspace{20pt}%
    \renewcommand{\previousreference}{#1}%
    \stepcounter{reference}%
    \label{reference-\previousreference-\thereference}%
  }{}%
  \todo[noline]{%
    \protect\vspace{20pt}%
    \protect\par%
    \protect\notblank{#1}{\cite{[\previousreference]}\\}{}%
    \protect\hyperref[reference-\previousreference-\thereference]{p. #2}%
  }%
}

\definecolor{devcolor}{HTML}{00695c}
\providecommand{\dev}[1]{%
  \reversemarginpar%
  \todo[noline]{
    \protect\vspace{20pt}%
    \protect\par%
    \bfseries\color{devcolor}\href{\website/developpements/#1}{[DEV]}
  }%
  \normalmarginpar%
}

% En-têtes :

\pagestyle{fancy}
\fancyhead[L]{\truncate{0.23\textwidth}{\thepage}}
\fancyfoot[C]{\scriptsize \href{\website}{\texttt{https://github.com/imbodj/SenCoursDeMaths}}}

% Couleurs :

\definecolor{property}{HTML}{ffeb3b}
\definecolor{proposition}{HTML}{ffc107}
\definecolor{lemma}{HTML}{ff9800}
\definecolor{theorem}{HTML}{f44336}
\definecolor{corollary}{HTML}{e91e63}
\definecolor{definition}{HTML}{673ab7}
\definecolor{notation}{HTML}{9c27b0}
\definecolor{example}{HTML}{00bcd4}
\definecolor{cexample}{HTML}{795548}
\definecolor{application}{HTML}{009688}
\definecolor{remark}{HTML}{3f51b5}
\definecolor{algorithm}{HTML}{607d8b}
%\definecolor{proof}{HTML}{e1f5fe}
\definecolor{exercice}{HTML}{e1f5fe}

% Théorèmes :

\theoremstyle{definition}
\newtheorem{theorem}{Théorème}

\newtheorem{property}[theorem]{Propriété}
\newtheorem{proposition}[theorem]{Proposition}
\newtheorem{lemma}[theorem]{Activité d'introduction}
\newtheorem{corollary}[theorem]{Conséquence}

\newtheorem{definition}[theorem]{Définition}
\newtheorem{notation}[theorem]{Notation}

\newtheorem{example}[theorem]{Exemple}
\newtheorem{cexample}[theorem]{Contre-exemple}
\newtheorem{application}[theorem]{Application}

\newtheorem{algorithm}[theorem]{Algorithme}
\newtheorem{exercice}[theorem]{Exercice}

\theoremstyle{remark}
\newtheorem{remark}[theorem]{Remarque}

\counterwithin*{theorem}{section}

\newcommand{\applystyletotheorem}[1]{
  \tcolorboxenvironment{#1}{
    enhanced,
    breakable,
    colback=#1!8!white,
    %right=0pt,
    %top=8pt,
    %bottom=8pt,
    boxrule=0pt,
    frame hidden,
    sharp corners,
    enhanced,borderline west={4pt}{0pt}{#1},
    %interior hidden,
    sharp corners,
    after=\par,
  }
}

\applystyletotheorem{property}
\applystyletotheorem{proposition}
\applystyletotheorem{lemma}
\applystyletotheorem{theorem}
\applystyletotheorem{corollary}
\applystyletotheorem{definition}
\applystyletotheorem{notation}
\applystyletotheorem{example}
\applystyletotheorem{cexample}
\applystyletotheorem{application}
\applystyletotheorem{remark}
%\applystyletotheorem{proof}
\applystyletotheorem{algorithm}
\applystyletotheorem{exercice}

% Environnements :

\NewEnviron{whitetabularx}[1]{%
  \renewcommand{\arraystretch}{2.5}
  \colorbox{white}{%
    \begin{tabularx}{\textwidth}{#1}%
      \BODY%
    \end{tabularx}%
  }%
}

% Maths :

\DeclareFontEncoding{FMS}{}{}
\DeclareFontSubstitution{FMS}{futm}{m}{n}
\DeclareFontEncoding{FMX}{}{}
\DeclareFontSubstitution{FMX}{futm}{m}{n}
\DeclareSymbolFont{fouriersymbols}{FMS}{futm}{m}{n}
\DeclareSymbolFont{fourierlargesymbols}{FMX}{futm}{m}{n}
\DeclareMathDelimiter{\VERT}{\mathord}{fouriersymbols}{152}{fourierlargesymbols}{147}

% Code :

\definecolor{greencode}{rgb}{0,0.6,0}
\definecolor{graycode}{rgb}{0.5,0.5,0.5}
\definecolor{mauvecode}{rgb}{0.58,0,0.82}
\definecolor{bluecode}{HTML}{1976d2}
\lstset{
  basicstyle=\footnotesize\ttfamily,
  breakatwhitespace=false,
  breaklines=true,
  %captionpos=b,
  commentstyle=\color{greencode},
  deletekeywords={...},
  escapeinside={\%*}{*)},
  extendedchars=true,
  frame=none,
  keepspaces=true,
  keywordstyle=\color{bluecode},
  language=Python,
  otherkeywords={*,...},
  numbers=left,
  numbersep=5pt,
  numberstyle=\tiny\color{graycode},
  rulecolor=\color{black},
  showspaces=false,
  showstringspaces=false,
  showtabs=false,
  stepnumber=2,
  stringstyle=\color{mauvecode},
  tabsize=2,
  %texcl=true,
  xleftmargin=10pt,
  %title=\lstname
}

\newcommand{\codedirectory}{}
\newcommand{\inputalgorithm}[1]{%
  \begin{algorithm}%
    \strut%
    \lstinputlisting{\codedirectory#1}%
  \end{algorithm}%
}



% Bibliographie :

%\addbibresource{\bibliographypath}%
\defbibheading{bibliography}[\bibname]{\section*{#1}}
\renewbibmacro*{entryhead:full}{\printfield{labeltitle}}%
\DeclareFieldFormat{url}{\newline\footnotesize\url{#1}}%

\AtEndDocument{%
  \newpage%
  \pagestyle{empty}%
  \printbibliography%
}


\begin{document}
  %<*content>
  \sheet{algebra, analysis}{conseils-generaux}{Conseils généraux}

  \summary{Dans cette fiche un peu différente, je liste quelques conseils de préparation qui ont fonctionné pour moi.}

  Avant de commencer la lecture de ce document, je tiens à avertir tous ses lecteurs : les conseils qui vont suivre sont très personnels. Je les rédige en toute modestie ici avec l'expérience acquise au cours de cette année de préparation et je ne pourrai résumer à quelques phrases tout ce que j'ai appris. Bref, c'est à vous de trouver les méthodes de travail qui marchent le mieux pour vous. Ci-dessous ne sont listés que de modestes exemples...
  \newpar
  Je précise également que j'ai obtenu l'agrégation en deux fois : admissible en 2021, puis admis en 2024 en ``candidat libre''. J'ai continué à étudier parallèlement à mon travail d'enseignant en mathématiques, et j'ai donc dû adapter mon plan de travail en conséquence.

  \subsection{Conseils pour la préparation pendant l'année}

  \subsubsection{Travail de l'écrit}

  À mon sens, il faut commencer à travailler les écrits dès la rentrée. Personnellement, j'ai commencé par des petits exercices très ciblés, puis j'ai assez vite enchaîné sur des sujets d'agrégation. Un travail de l'ordre d'un sujet par semaine ; pas tout d'un coup, mais étalé sur les sept jours. Environ deux mois avant les oraux, je me suis imposé un écrit blanc par semaine : six heures de suite le mercredi à la bibliothèque universitaire, puis correction dans la foulée.
  \newpar
  L'idée pour les écrits était de bien faire environ la moitié du sujet, le reste serait du bonus étant donné que je ne vise aucun classement.

  \subsubsection{Travail de l'oral}

  Mes développements étant déjà rédigés suite à mon parcours de 2021, je n'avais ``que'' les leçons à retravailler et à rerédiger.
  \newpar
  Ma méthode était assez simple : un plan de leçon par semaine jusqu'aux écrits, puis un plan par jour sur cinq jours de la semaine jusqu'à avoir terminé. Je me suis autorisé trois impasses : deux en algèbre et une en analyse. Je conçois que la seconde moitié de cette préparation est assez lourde, j'aurais sans doute dû équilibrer un peu mieux au cours de l'année. Une fois mes plans terminés, j'ai fait des révisions thématiques (groupes, anneaux, séries entières, probabilités, etc.) notamment à l'aide des retours d'oraux disponibles sur le site \href{https://agreg-maths.fr}{agreg-maths.fr}. Je ne peux que vous conseiller d'en faire de même, certaines questions sont très classiques et il y a de grandes chances que vous les retrouviez le jour J.
  \newpar
  Pour mémoriser mes plans, j'ai utilisé le logiciel \href{https://apps.ankiweb.net/}{Anki}. Cela fonctionne très bien, et j'ai pu tout apprendre en un mois et demi environ.
  \newpar
  En ce qui concerne le contenu des plans, il me semble inutile et même contreproductif d'inclure des résultats non maîtrisés. Le jury interroge en priorité sur les résultats qui figurent dans le plan. À mon avis, il vaut mieux un plan simple d'un niveau moindre mais dont on est sûr du contenu, qu'un plan trop ambitieux.

  \subsection{Conseils pour le jour J}

  \subsubsection{Pour les écrits}

  Déjà, un point simple mais essentiel : rester pendant les six heures d'épreuve et tout donner jusqu'à la fin. Il n'y a rien de plus frustrant que de rater l'admission à un point, il faut mettre toutes les chances de son côté.
  \newpar
  Ensuite, on le lit partout et les rapports du jury le confirme, mais il faut soigner au maximum le début de sa copie. Les premières questions servent régulièrement de discriminant entre les candidats, il convient d'être du bon côté.
  \newpar
  Dernier détail ; les copies sont numérisées, puis corrigées sur ordinateur. Donc il vaut mieux écrire avec un stylo à encre foncée.

  \subsubsection{Pour les oraux}

  J'ai organisé mes préparations de la manière suivante :
  \begin{itemize}
    \item une heure et demi consacrée à l'écriture du plan ;
    \item le reste pour apprendre mes développements et travailler les résultats qui figurent dans mon plan selon le temps restant.
  \end{itemize}
  Cela ne peut fonctionner que si les développements ont été travaillés et appris en amont. Pour les oraux de 2024, nous avions réellement trois heures de préparation et nous disposions d'une petite minute pour relire le développement choisi par le jury. Pour cette raison, il convient de l'avoir rédigé proprement sur une des feuilles de brouillon.
  \newpar
  Je n'ai pas de conseil à donner pour la défense du plan autre qu'il faut indiquer ses développements et motiver leur place dans la leçon choisie.
  \newpar
  Pour le développement, il ne faut faire ni trop court ni trop long. Je pense qu'un développement n'est jamais trop court : on peut mettre à profit le temps superflu en faisant un plan de la preuve au tableau, en prenant le temps de se retourner vers le jury pour expliquer son raisonnement, en appliquant le résultat démontré à un exemple, etc. Le jury attend une certaine aisance à l'oral, et faire preuve de qualités pédagogiques dans un tel contexte ne peut être que valorisé.
  \newpar
  À ce sujet, je vous conseille de porter une montre digitale le jour de l'oral (on peut en trouver des très bonnes pour une quinzaine d'euros dans un magasin de sport très connu), et de lancer le chronomètre, que ce soit pour la défense du plan ou pour le développement. Cela aide vraiment dans la gestion du temps. Ce conseil vaut d'autant plus pour l'épreuve de modélisation où la partie orale en autonomie dure tout de même 35 minutes.
  \newpar
  Ensuite, pour les questions, les notes réalisées pendant la préparation sont autorisées. Donc il ne faut pas hésiter à noter quelques éléments de preuve de résultats non triviaux. Et, je vais sûrement me répéter, mais il faut maîtriser son plan car le jury peut interroger sur tous les éléments qui s'y trouvent.
  \newpar
  En ce qui concerne l'attitude des membres du jury, de mon expérience, ils sont généralement bienveillants. Les questions servent à tester le degré de maîtrise du sujet par le candidat. À ce titre, il faut être honnête et ne pas tenter d'entourloupes (les membres du jury sauront de toute manière le détecter, et le restant de l'heure pourra vite tourner au clavaire).
  \newpar
  Dernier conseil, plus personnel, mais il ne faut pas se décourager. Un tirage défavorable, ça existe. Deux tirages défavorables, aussi. Ce fut mon cas cette année, et je confirme que la préparation est d'autant plus stressante que de savoir que nous allons rester une heure devant des spécialistes de leur domaine pour parler d'un sujet que nous ne maîtrisons pas. Même dans ce cas là, n'oubliez pas que vous savez des choses, même si elles sont simples, et le jury saura (aussi) le mettre en valeur. Et même si vous n'arrivez à rien, une mauvaise note ça arrive et cela ne vous disqualifie pas d'office. Il ne faut pas se décourager et donner le maximum les autres jours en pensant à tout le chemin parcouru dans l'année pour en arriver là où vous êtes...
  %</content>
\end{document}
