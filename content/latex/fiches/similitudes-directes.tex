
 \input{../common}

\begin{document}
  %<*content>
  \sheet{algebra}{similitudes-directes}{Similitudes directes}
 \summary{}
 


\subsection*{1. Transformations du plan complexe}

Une transformation du plan complexe $\mathcal{P}$ est une application bijective :
\[
f : \mathcal{P} \longrightarrow \mathcal{P}, \quad M \longmapsto M'
\]
On lui associe une unique application complexe bijective $\varphi : \mathbb{C} \to \mathbb{C}$, telle que $z' = \varphi(z)$.

L’expression $z' = \varphi(z)$ est appelée \textbf{écriture complexe} de la transformation.

\subsection*{2. Écriture complexe des transformations usuelles}

\begin{itemize}
  \item \textbf{Translation de vecteur $\vec{u}$ d'affixe $a$} : 
  \[
  z' = z + a
  \]
  
  \item \textbf{Homothétie de centre $\Omega(\omega)$ et de rapport $k \in \mathbb{R}$} :
  \[
  z' - \omega = k(z - \omega) \quad \text{ou} \quad z' = kz + \omega(1 - k)
  \]
  
  \item \textbf{Rotation de centre $\Omega(\omega)$ et d’angle $\theta$} :
  \[
  z' - \omega = e^{i\theta}(z - \omega) \quad \text{ou} \quad z' = e^{i\theta}z + \omega(1 - e^{i\theta})
  \]
\end{itemize}

\subsection*{3. Similitudes directes}

\begin{definition}
Une \textbf{similitude directe} est une transformation du plan qui conserve les angles orientés et multiplie les distances par un réel $k > 0$, appelé \textbf{rapport}.
\end{definition}

Les éléments caractéristiques d’une similitude directe sont :
\begin{itemize}
  \item le \textbf{centre} $\Omega$ (point invariant),
  \item le \textbf{rapport} $k > 0$,
  \item l’\textbf{angle} $\theta$.
\end{itemize}

Elle est notée : $S(\Omega,\;k,\;\theta)$

\subsection*{4. Propriétés géométriques d'une similitude directe}

\begin{itemize}
  \item Le centre est le seul point invariant (sauf pour les translations).
  \item Une similitude directe :
    \begin{itemize}
      \item multiplie les longueurs par $k$ ;
      \item multiplie les aires par $k^2$ ;
      \item conserve les alignements, parallélismes, orthogonalités, barycentres, contacts et angles orientés.
    \end{itemize}
  \item L'image d'une droite est une droite ; l'image d’un cercle est un cercle.
  \item La réciproque de $S(\Omega,\;k,\;\theta)$ est $S^{-1}(\Omega,\;\tfrac{1}{k},\;-\theta)$.
  \item La composée de deux similitudes directes de même centre est une similitude directe de même centre :
  \[
  S(\Omega,\;k,\;\theta)\circ S(\Omega,\;k',\;\theta') = S(\Omega,\;kk',\;\theta+\theta')
  \]
\end{itemize}

\subsection*{5. Écriture complexe d'une similitude directe}

Toute similitude directe de centre $\Omega$ d'affixe $\omega$, de rapport $k > 0$ et d'angle $\theta$ admet l’écriture complexe :
\[
z' - \omega = k e^{i\theta}(z - \omega) \quad \text{ou} \quad z' = k e^{i\theta}z + \omega(1 - k e^{i\theta})
\]

\begin{corollary}
Toute similitude directe a une écriture complexe de la forme :
\[
z' = az + b \quad \text{où } a \in \mathbb{C}^*,\; b \in \mathbb{C}
\]
avec $k = |a|$ et $\theta = \arg(a)$.
\end{corollary}

\subsection*{6. Détermination de la nature d’une transformation}

Soit $f(z) = az + b$ avec $a \neq 0$ :
\begin{itemize}
  \item si $a = 1$ : $f$ est une \textbf{translation} d’affixe $b$ ;
  \item si $|a| = 1$ et $a \neq 1$ : $f$ est une \textbf{rotation} ;
  \item si $a \in \mathbb{R} \setminus \{0,1\}$ : $f$ est une \textbf{homothétie} ;
  \item si $a \notin \mathbb{R}$ et $|a| \ne 1$ : $f$ est une \textbf{similitude directe} non isométrique.
\end{itemize}

Le point invariant $\omega$ est donné par :
\[
\omega = \frac{b}{1 - a}
\]

\subsection*{7. Tableau récapitulatif}

\[
\begin{array}{|l|l|}
\hline
\textbf{Transformation} & \textbf{Écriture complexe} \\
\hline
\text{Translation de vecteur } \vec{u} & z' = z + b \quad (b = \text{affixe de } \vec{u}) \\
\hline
\text{Homothétie de centre } \Omega,\ \text{rapport } k & z' - \omega = k(z - \omega) \\
\hline
\text{Rotation de centre } \Omega,\ \text{angle } \theta & z' - \omega = e^{i\theta}(z - \omega) \\
\hline
\text{Similitude directe de centre } \Omega,\ k > 0,\ \theta\in\mathbb{R} & z' - \omega = ke^{i\theta}(z - \omega) \\
\hline
\end{array}
\]


\end{document}