
 \documentclass[12pt, a4paper]{report}

% LuaLaTeX :

\RequirePackage{iftex}
\RequireLuaTeX

% Packages :

\usepackage[french]{babel}
%\usepackage[utf8]{inputenc}
%\usepackage[T1]{fontenc}
\usepackage[pdfencoding=auto, pdfauthor={Hugo Delaunay}, pdfsubject={Mathématiques}, pdfcreator={agreg.skyost.eu}]{hyperref}
\usepackage{amsmath}
\usepackage{amsthm}
%\usepackage{amssymb}
\usepackage{stmaryrd}
\usepackage{tikz}
\usepackage{tkz-euclide}
\usepackage{fontspec}
\defaultfontfeatures[Erewhon]{FontFace = {bx}{n}{Erewhon-Bold.otf}}
\usepackage{fourier-otf}
\usepackage[nobottomtitles*]{titlesec}
\usepackage{fancyhdr}
\usepackage{listings}
\usepackage{catchfilebetweentags}
\usepackage[french, capitalise, noabbrev]{cleveref}
\usepackage[fit, breakall]{truncate}
\usepackage[top=2.5cm, right=2cm, bottom=2.5cm, left=2cm]{geometry}
\usepackage{enumitem}
\usepackage{tocloft}
\usepackage{microtype}
%\usepackage{mdframed}
%\usepackage{thmtools}
\usepackage{xcolor}
\usepackage{tabularx}
\usepackage{xltabular}
\usepackage{aligned-overset}
\usepackage[subpreambles=true]{standalone}
\usepackage{environ}
\usepackage[normalem]{ulem}
\usepackage{etoolbox}
\usepackage{setspace}
\usepackage[bibstyle=reading, citestyle=draft]{biblatex}
\usepackage{xpatch}
\usepackage[many, breakable]{tcolorbox}
\usepackage[backgroundcolor=white, bordercolor=white, textsize=scriptsize]{todonotes}
\usepackage{luacode}
\usepackage{float}
\usepackage{needspace}
\everymath{\displaystyle}

% Police :

\setmathfont{Erewhon Math}

% Tikz :

\usetikzlibrary{calc}
\usetikzlibrary{3d}

% Longueurs :

\setlength{\parindent}{0pt}
\setlength{\headheight}{15pt}
\setlength{\fboxsep}{0pt}
\titlespacing*{\chapter}{0pt}{-20pt}{10pt}
\setlength{\marginparwidth}{1.5cm}
\setstretch{1.1}

% Métadonnées :

\author{agreg.skyost.eu}
\date{\today}

% Titres :

\setcounter{secnumdepth}{3}

\renewcommand{\thechapter}{\Roman{chapter}}
\renewcommand{\thesubsection}{\Roman{subsection}}
\renewcommand{\thesubsubsection}{\arabic{subsubsection}}
\renewcommand{\theparagraph}{\alph{paragraph}}

\titleformat{\chapter}{\huge\bfseries}{\thechapter}{20pt}{\huge\bfseries}
\titleformat*{\section}{\LARGE\bfseries}
\titleformat{\subsection}{\Large\bfseries}{\thesubsection \, - \,}{0pt}{\Large\bfseries}
\titleformat{\subsubsection}{\large\bfseries}{\thesubsubsection. \,}{0pt}{\large\bfseries}
\titleformat{\paragraph}{\bfseries}{\theparagraph. \,}{0pt}{\bfseries}

\setcounter{secnumdepth}{4}

% Table des matières :

\renewcommand{\cftsecleader}{\cftdotfill{\cftdotsep}}
\addtolength{\cftsecnumwidth}{10pt}

% Redéfinition des commandes :

\renewcommand*\thesection{\arabic{section}}
\renewcommand{\ker}{\mathrm{Ker}}

% Nouvelles commandes :

\newcommand{\website}{https://github.com/imbodj/SenCoursDeMaths}

\newcommand{\tr}[1]{\mathstrut ^t #1}
\newcommand{\im}{\mathrm{Im}}
\newcommand{\rang}{\operatorname{rang}}
\newcommand{\trace}{\operatorname{trace}}
\newcommand{\id}{\operatorname{id}}
\newcommand{\stab}{\operatorname{Stab}}
\newcommand{\paren}[1]{\left(#1\right)}
\newcommand{\croch}[1]{\left[ #1 \right]}
\newcommand{\Grdcroch}[1]{\Bigl[ #1 \Bigr]}
\newcommand{\grdcroch}[1]{\bigl[ #1 \bigr]}
\newcommand{\abs}[1]{\left\lvert #1 \right\rvert}
\newcommand{\limi}[3]{\lim_{#1\to #2}#3}
\newcommand{\pinf}{+\infty}
\newcommand{\minf}{-\infty}
%%%%%%%%%%%%%% ENSEMBLES %%%%%%%%%%%%%%%%%
\newcommand{\ensemblenombre}[1]{\mathbb{#1}}
\newcommand{\Nn}{\ensemblenombre{N}}
\newcommand{\Zz}{\ensemblenombre{Z}}
\newcommand{\Qq}{\ensemblenombre{Q}}
\newcommand{\Qqp}{\Qq^+}
\newcommand{\Rr}{\ensemblenombre{R}}
\newcommand{\Cc}{\ensemblenombre{C}}
\newcommand{\Nne}{\Nn^*}
\newcommand{\Zze}{\Zz^*}
\newcommand{\Zzn}{\Zz^-}
\newcommand{\Qqe}{\Qq^*}
\newcommand{\Rre}{\Rr^*}
\newcommand{\Rrp}{\Rr_+}
\newcommand{\Rrm}{\Rr_-}
\newcommand{\Rrep}{\Rr_+^*}
\newcommand{\Rrem}{\Rr_-^*}
\newcommand{\Cce}{\Cc^*}
%%%%%%%%%%%%%%  INTERVALLES %%%%%%%%%%%%%%%%%
\newcommand{\intff}[2]{\left[#1\;,\; #2\right]  }
\newcommand{\intof}[2]{\left]#1 \;, \;#2\right]  }
\newcommand{\intfo}[2]{\left[#1 \;,\; #2\right[  }
\newcommand{\intoo}[2]{\left]#1 \;,\; #2\right[  }

\providecommand{\newpar}{\\[\medskipamount]}

\newcommand{\annexessection}{%
  \newpage%
  \subsection*{Annexes}%
}

\providecommand{\lesson}[3]{%
  \title{#3}%
  \hypersetup{pdftitle={#2 : #3}}%
  \setcounter{section}{\numexpr #2 - 1}%
  \section{#3}%
  \fancyhead[R]{\truncate{0.73\textwidth}{#2 : #3}}%
}

\providecommand{\development}[3]{%
  \title{#3}%
  \hypersetup{pdftitle={#3}}%
  \section*{#3}%
  \fancyhead[R]{\truncate{0.73\textwidth}{#3}}%
}

\providecommand{\sheet}[3]{\development{#1}{#2}{#3}}

\providecommand{\ranking}[1]{%
  \title{Terminale #1}%
  \hypersetup{pdftitle={Terminale #1}}%
  \section*{Terminale #1}%
  \fancyhead[R]{\truncate{0.73\textwidth}{Terminale #1}}%
}

\providecommand{\summary}[1]{%
  \textit{#1}%
  \par%
  \medskip%
}

\tikzset{notestyleraw/.append style={inner sep=0pt, rounded corners=0pt, align=center}}

%\newcommand{\booklink}[1]{\website/bibliographie\##1}
\newcounter{reference}
\newcommand{\previousreference}{}
\providecommand{\reference}[2][]{%
  \needspace{20pt}%
  \notblank{#1}{
    \needspace{20pt}%
    \renewcommand{\previousreference}{#1}%
    \stepcounter{reference}%
    \label{reference-\previousreference-\thereference}%
  }{}%
  \todo[noline]{%
    \protect\vspace{20pt}%
    \protect\par%
    \protect\notblank{#1}{\cite{[\previousreference]}\\}{}%
    \protect\hyperref[reference-\previousreference-\thereference]{p. #2}%
  }%
}

\definecolor{devcolor}{HTML}{00695c}
\providecommand{\dev}[1]{%
  \reversemarginpar%
  \todo[noline]{
    \protect\vspace{20pt}%
    \protect\par%
    \bfseries\color{devcolor}\href{\website/developpements/#1}{[DEV]}
  }%
  \normalmarginpar%
}

% En-têtes :

\pagestyle{fancy}
\fancyhead[L]{\truncate{0.23\textwidth}{\thepage}}
\fancyfoot[C]{\scriptsize \href{\website}{\texttt{https://github.com/imbodj/SenCoursDeMaths}}}

% Couleurs :

\definecolor{property}{HTML}{ffeb3b}
\definecolor{proposition}{HTML}{ffc107}
\definecolor{lemma}{HTML}{ff9800}
\definecolor{theorem}{HTML}{f44336}
\definecolor{corollary}{HTML}{e91e63}
\definecolor{definition}{HTML}{673ab7}
\definecolor{notation}{HTML}{9c27b0}
\definecolor{example}{HTML}{00bcd4}
\definecolor{cexample}{HTML}{795548}
\definecolor{application}{HTML}{009688}
\definecolor{remark}{HTML}{3f51b5}
\definecolor{algorithm}{HTML}{607d8b}
%\definecolor{proof}{HTML}{e1f5fe}
\definecolor{exercice}{HTML}{e1f5fe}

% Théorèmes :

\theoremstyle{definition}
\newtheorem{theorem}{Théorème}

\newtheorem{property}[theorem]{Propriété}
\newtheorem{proposition}[theorem]{Proposition}
\newtheorem{lemma}[theorem]{Activité d'introduction}
\newtheorem{corollary}[theorem]{Conséquence}

\newtheorem{definition}[theorem]{Définition}
\newtheorem{notation}[theorem]{Notation}

\newtheorem{example}[theorem]{Exemple}
\newtheorem{cexample}[theorem]{Contre-exemple}
\newtheorem{application}[theorem]{Application}

\newtheorem{algorithm}[theorem]{Algorithme}
\newtheorem{exercice}[theorem]{Exercice}

\theoremstyle{remark}
\newtheorem{remark}[theorem]{Remarque}

\counterwithin*{theorem}{section}

\newcommand{\applystyletotheorem}[1]{
  \tcolorboxenvironment{#1}{
    enhanced,
    breakable,
    colback=#1!8!white,
    %right=0pt,
    %top=8pt,
    %bottom=8pt,
    boxrule=0pt,
    frame hidden,
    sharp corners,
    enhanced,borderline west={4pt}{0pt}{#1},
    %interior hidden,
    sharp corners,
    after=\par,
  }
}

\applystyletotheorem{property}
\applystyletotheorem{proposition}
\applystyletotheorem{lemma}
\applystyletotheorem{theorem}
\applystyletotheorem{corollary}
\applystyletotheorem{definition}
\applystyletotheorem{notation}
\applystyletotheorem{example}
\applystyletotheorem{cexample}
\applystyletotheorem{application}
\applystyletotheorem{remark}
%\applystyletotheorem{proof}
\applystyletotheorem{algorithm}
\applystyletotheorem{exercice}

% Environnements :

\NewEnviron{whitetabularx}[1]{%
  \renewcommand{\arraystretch}{2.5}
  \colorbox{white}{%
    \begin{tabularx}{\textwidth}{#1}%
      \BODY%
    \end{tabularx}%
  }%
}

% Maths :

\DeclareFontEncoding{FMS}{}{}
\DeclareFontSubstitution{FMS}{futm}{m}{n}
\DeclareFontEncoding{FMX}{}{}
\DeclareFontSubstitution{FMX}{futm}{m}{n}
\DeclareSymbolFont{fouriersymbols}{FMS}{futm}{m}{n}
\DeclareSymbolFont{fourierlargesymbols}{FMX}{futm}{m}{n}
\DeclareMathDelimiter{\VERT}{\mathord}{fouriersymbols}{152}{fourierlargesymbols}{147}

% Code :

\definecolor{greencode}{rgb}{0,0.6,0}
\definecolor{graycode}{rgb}{0.5,0.5,0.5}
\definecolor{mauvecode}{rgb}{0.58,0,0.82}
\definecolor{bluecode}{HTML}{1976d2}
\lstset{
  basicstyle=\footnotesize\ttfamily,
  breakatwhitespace=false,
  breaklines=true,
  %captionpos=b,
  commentstyle=\color{greencode},
  deletekeywords={...},
  escapeinside={\%*}{*)},
  extendedchars=true,
  frame=none,
  keepspaces=true,
  keywordstyle=\color{bluecode},
  language=Python,
  otherkeywords={*,...},
  numbers=left,
  numbersep=5pt,
  numberstyle=\tiny\color{graycode},
  rulecolor=\color{black},
  showspaces=false,
  showstringspaces=false,
  showtabs=false,
  stepnumber=2,
  stringstyle=\color{mauvecode},
  tabsize=2,
  %texcl=true,
  xleftmargin=10pt,
  %title=\lstname
}

\newcommand{\codedirectory}{}
\newcommand{\inputalgorithm}[1]{%
  \begin{algorithm}%
    \strut%
    \lstinputlisting{\codedirectory#1}%
  \end{algorithm}%
}




\begin{document}
  %<*content>
  \sheet{algebra}{nombres-complexes}{Nombres complexes}

 \summary{}

\subsection{Comment écrire un nombre complexe sous forme algébrique ?}

\begin{methode}
Pour obtenir la forme algébrique d'un nombre complexe, on développe en utilisant les propriétés de l'addition et de la multiplication et en tenant compte de ce que $\i^2 = -1$. Dans le cas d'un quotient, si le dénominateur est un nombre complexe, on multiplie numérateur et dénominateur par le conjugué du dénominateur pour rendre ce dernier réel.
\end{methode}

\begin{example}
Soit $f$ l'application définie dans $\Cce$ par $f(z) = \dfrac{1}{z}$, donner la forme algébrique de $f(2+3\i)$.

\begin{proof}
$f(2+3\i) = \dfrac{1}{2+3\i} = \dfrac{1}{2+3\i} \cdot \dfrac{2-3\i}{2-3\i} = \dfrac{2-3\i}{4+9} = \dfrac{2-3\i}{13} = \dfrac{2}{13} - \dfrac{3}{13}\i$
\end{proof}
\end{example}

\begin{remark}
Remarques utiles :
\begin{itemize}
\item $\overline{z} = \overline{a+b\i} = a-b\i$
\item $z \cdot \overline{z} = |z|^2 = a^2 + b^2$
\end{itemize}
\end{remark}

\subsection{Comment résoudre une équation dans $\Cc$ ?}

\begin{methode}
\begin{itemize}
\item Pour résoudre une équation du premier degré d'inconnue $z$, on isole $z$ et on donne sa forme algébrique.
\item Pour résoudre une équation du premier degré d'inconnue $z$ dans laquelle figurent $\overline{z}$, on pose $z = x + y\i$ et on remplace dans l'équation donnée.
\item Pour résoudre une équation du second degré d'inconnue $z$ à coefficients réels, on calcule le discriminant $\Delta$, suivant son signe on a alors des solutions réelles ou complexes conjuguées.
\end{itemize}
\end{methode}

\begin{example}
Résoudre dans $\Cc$ chacune des équations suivantes :
\begin{enumerate}
\item $2z + 3\i = 5 - z$
\item $z + 2\overline{z} = 1 + 3\i$
\item $z^2 - 2z + 5 = 0$
\end{enumerate}

\begin{proof}
\begin{enumerate}
\item $2z + 3\i = 5 - z \Rightarrow 3z = 5 - 3\i \Rightarrow z = \dfrac{5-3\i}{3} = \dfrac{5}{3} - \i$
\item Posons $z = x + y\i$, alors $\overline{z} = x - y\i$.
L'équation devient : $(x + y\i) + 2(x - y\i) = 1 + 3\i$
$\Rightarrow 3x - y\i = 1 + 3\i$
Par identification : $3x = 1$ et $-y = 3$, donc $x = \frac{1}{3}$ et $y = -3$.
Ainsi $z = \frac{1}{3} - 3\i$
\item $\Delta = 4 - 20 = -16 < 0$
Les solutions sont : $z = \dfrac{2 \pm \sqrt{-16}}{2} = \dfrac{2 \pm 4\i}{2} = 1 \pm 2\i$
\end{enumerate}
\end{proof}
\end{example}

\subsection{Comment déterminer un ensemble de points à partir de la forme algébrique ?}

\begin{methode}
Il faut savoir << traduire l'énoncé >>, les remarques suivantes sont souvent utiles pour le faire :
\begin{itemize}
\item $z$ est un nombre réel signifie que $\text{Im}(z) = 0$ ou que $z = \overline{z}$ ou que l'image de $z$ est un point de l'axe réel
\item $z$ est un nombre imaginaire pur signifie que $\text{Re}(z) = 0$ ou que $z = -\overline{z}$ ou que l'image de $z$ appartient à l'axe imaginaire
\end{itemize}

Il faut aussi savoir reconnaître les ensembles de points à partir de leur équation :
\begin{itemize}
\item $ax + by + c = 0$ pour une droite
\item $(x-a)^2 + (y-b)^2 = R^2$ pour le cercle de centre le point $\Omega$ d'affixe $a+b\i$ et de rayon $R$
\end{itemize}
\end{methode}

\begin{example}
À tout nombre complexe $z$ différent de $\i$, on associe le nombre complexe $Z = \dfrac{z-\i}{z+\i}$. On pose $z = x + y\i$ où $x$ et $y$ sont deux réels.

\begin{enumerate}
\item Exprimer en fonction de $x$ et $y$ la partie réelle $X$ et la partie imaginaire $Y$ de $Z$.
\item En déduire l'ensemble $\mathcal{E}$ des points $M$ du plan complexe, d'affixe $z$ tels que $Z$ est réel.
\item En déduire l'ensemble $\mathcal{F}$ des points $M$ du plan complexe, d'affixe $z$ tels que $Z$ est imaginaire pur.
\end{enumerate}

\begin{proof}
\begin{enumerate}
\item $Z = \dfrac{x + y\i - \i}{x + y\i + \i} = \dfrac{x + (y-1)\i}{x + (y+1)\i}$

En multipliant par le conjugué :
$Z = \dfrac{[x + (y-1)\i][x - (y+1)\i]}{x^2 + (y+1)^2} = \dfrac{x^2 + (y-1)(y+1) + \i[x(y-1) - x(y+1)]}{x^2 + (y+1)^2}$

$Z = \dfrac{x^2 + y^2 - 1 - 2x\i}{x^2 + (y+1)^2}$

Donc $X = \dfrac{x^2 + y^2 - 1}{x^2 + (y+1)^2}$ et $Y = \dfrac{-2x}{x^2 + (y+1)^2}$

\item $Z$ est réel $\Leftrightarrow Y = 0 \Leftrightarrow x = 0$
Donc $\mathcal{E}$ est l'axe imaginaire privé du point d'affixe $\i$.

\item $Z$ est imaginaire pur $\Leftrightarrow X = 0 \Leftrightarrow x^2 + y^2 - 1 = 0$
Donc $\mathcal{F}$ est le cercle de centre $O$ et de rayon $1$ privé du point d'affixe $\i$.
\end{enumerate}
\end{proof}
\end{example}

\subsection{Comment utiliser les différentes formes d'un nombre complexe ?}

\[
\begin{array}{|c|c|c|}
\hline
\textbf{Que faut-il faire ?} & \textbf{Quelle forme faut-il choisir ?} & \textbf{Utilité} \\
\hline
\begin{array}{c}
\text{Vérifier que } a \text{ et } b \\
\text{ sont réels}
\end{array}
&
\begin{array}{c}
\text{Forme algébrique } \\
z = a + b\i
\end{array}
&
\begin{array}{c}
\text{Cette forme facilite les calculs} \\
\text{de somme et de différence et} \\
\text{permet de faire le lien entre les} \\
\text{complexes et les coordonnées} \\
\text{cartésiennes des points images}
\end{array}
\\
\hline
\begin{array}{c}
\text{Vérifier que } r \text{ est} \\
\text{un réel positif}
\end{array}
&
\begin{array}{c}
\text{Forme trigonométrique } \\
z = r(\cos\theta + \i\sin\theta)
\end{array}
&
\begin{array}{c}
\text{Cette forme établit le lien} \\
\text{entre les complexes et la géométrie,} \\
\text{elle permet le calcul des distances} \\
\text{et des angles}
\end{array}
\\
\hline
\begin{array}{c}
\text{Vérifier que } r \text{ est positif}
\end{array}
&
\begin{array}{c}
\text{Forme exponentielle } \\
z = re^{\i\theta}
\end{array}
&
\begin{array}{c}
\text{Cette forme facilite les calculs} \\
\text{de produit, de quotient et de} \\
\text{puissance de nombres complexes}
\end{array}
\\
\hline
\end{array}
\]

\medskip

\textbf{Pour passer d'une forme à l'autre :}

\begin{align}
\text{Forme algébrique} &\leftrightarrow \text{Forme trigonométrique/exponentielle} \\
x &= r\cos\theta \text{ et } y = r\sin\theta \\
r &= \sqrt{x^2 + y^2}, \cos\theta = \frac{x}{r}, \sin\theta = \frac{y}{r}
\end{align}

\begin{example}
Relier les nombres complexes proposés à leurs différentes formes :
\bigskip

$\begin{array}{|c|c|c|c|}
\hline
\textbf{Complexes} & \textbf{Forme algébrique} & \textbf{Forme trigonométrique} & \textbf{Forme exponentielle} \\
\hline
1 + \i & 1 + \i & \sqrt{2}\paren{\cos\frac{\pi}{4} + \i\sin\frac{\pi}{4}} & \sqrt{2}e^{\i\pi/4} \\
\hline
-2 + 2\i & -2 + 2\i & 2\sqrt{2}\paren{\cos\frac{3\pi}{4} + \i\sin\frac{3\pi}{4}} & 2\sqrt{2}e^{\i3\pi/4} \\
\hline
\end{array}$
\end{example}

\begin{remark}
Cas particuliers à bien savoir :
\begin{itemize}
\item[$  \bullet$] $\i = e^{\i\pi/2}$
\item[$  \bullet$] $-1 = e^{\i\pi}$
\item[$  \bullet$] $-\i = e^{\i3\pi/2}$
\item[$  \bullet$] $1 = e^{\i0}$
\end{itemize}
\end{remark}

\subsection{Comment utiliser la forme trigonométrique des nombres complexes en géométrie ?}

\begin{methode}
Pour utiliser les nombres complexes en géométrie, il est utile de connaître leur forme trigonométrique, les règles de calcul sur celle-ci et l'interprétation géométrique des modules et arguments suivants :

\begin{itemize}
\item $|z_B - z_A| = AB$ (distance)
\item $\arg\paren{\dfrac{z_C - z_A}{z_B - z_A}} = \paren{\overrightarrow{AB}, \overrightarrow{AC}}\;\; $ (angle orienté)
\item $\left|\dfrac{z_C - z_A}{z_B - z_A}\right| = \dfrac{AC}{AB}\;\;$ (rapport de distances)
\item $\arg\paren{\dfrac{z_C - z_A}{z_B - z_A}} \equiv 0 \pmod{2\pi}\;\;$ signifie que $A$, $B$ et $C$ sont alignés
\end{itemize}

Dans ce qui précède $A$, $B$ et $C$ sont trois points tels que $A \neq B$ et $A \neq C$.
\end{methode}

\begin{example}
\begin{enumerate}
\item Soit $z_A = 1$, $z_B = 1 + \i\sqrt{3}$, $z_C = -1 + \i\sqrt{3}$ les affixes respectives de trois points $A$, $B$ et $C$. Déterminer la forme exponentielle de $\dfrac{z_C - z_A}{z_B - z_A}$. En déduire la nature du triangle $ABC$.

\item Déterminer l'ensemble des points $M$ d'affixe $z$ tels que $|z - \i| = 2$.

\item Déterminer l'ensemble des points $M$ d'affixe $z$ tels que $|z - 1| = |z - \i|$.

\item Déterminer l'ensemble des points $M$ d'affixe $z$ tels que $|z - 1| = |z + 1|$.
\end{enumerate}

\begin{proof}
\begin{enumerate}
\item $\dfrac{z_C - z_A}{z_B - z_A} = \dfrac{-1 + \i\sqrt{3} - 1}{\i\sqrt{3}} = \dfrac{-2 + \i\sqrt{3}}{\i\sqrt{3}} = \frac{(-2 + \i\sqrt{3})(-\i\sqrt{3})}{3} = \dfrac{2\i\sqrt{3} + 3}{3} = 1 + \dfrac{2\i\sqrt{3}}{3}$

Le module est $\sqrt{1 + \frac{4 \cdot 3}{9}} = \sqrt{\frac{13}{9}} = \frac{\sqrt{13}}{3}$ et l'argument est $\tan^{-1}\paren{\frac{2\sqrt{3}}{3}}$.

\item L'ensemble est le cercle de centre le point d'affixe $\i$ et de rayon $2$.

\item $|z - 1| = |z - \i|$ signifie que $M$ est équidistant des points d'affixes $1$ et $\i$. C'est la médiatrice du segment joignant ces deux points.

\item $|z - 1| = |z + 1|$ signifie que $M$ est équidistant des points d'affixes $1$ et $-1$. C'est l'axe imaginaire.
\end{enumerate}
\end{proof}
\end{example}

\subsection{Comment préciser la position relative de trois points ?}

\begin{methode}
Dans le plan complexe, $z_A$, $z_B$ et $z_C$ sont trois nombres complexes distincts, d'images respectives $A$, $B$ et $C$. On considère le nombre complexe $\frac{z_C - z_A}{z_B - z_A}$. On a :

\begin{itemize}
\item Si $\dfrac{z_C - z_A}{z_B - z_A} \in \Rr$, alors $A$, $B$ et $C$ sont alignés
\item Si $\left|\dfrac{z_C - z_A}{z_B - z_A}\right| = 1$, alors $AC = AB$
\item Si $\arg\paren{\dfrac{z_C - z_A}{z_B - z_A}} = \pm\frac{\pi}{2}$, alors le triangle $ABC$ est rectangle en $A$
\item Si $\dfrac{z_C - z_A}{z_B - z_A} = \pm\i$, alors le triangle $ABC$ est rectangle et isocèle en $A$
\end{itemize}
\end{methode}

\begin{example}
Dans chacun des cas que peut-on dire des points $A$, $B$ et $C$ ?
\begin{enumerate}
\item $\dfrac{z_C - z_A}{z_B - z_A} = 2$
\item $\dfrac{z_C - z_A}{z_B - z_A} = \i$
\item $\dfrac{z_C - z_A}{z_B - z_A} = -\dfrac{\sqrt{2}}{2} + \dfrac{\sqrt{2}}{2}\i$
\item $\dfrac{z_C - z_A}{z_B - z_A} = 1 + \i\sqrt{3}$
\end{enumerate}

\begin{proof}
\begin{enumerate}
\item Le rapport est réel positif, donc $A$, $B$ et $C$ sont alignés et $AC = 2AB$.
\item $|\i| = 1$ et $\arg(\i) = \frac{\pi}{2}$, donc le triangle $ABC$ est rectangle et isocèle en $A$.
\item Le module est $1$ et l'argument est $\frac{3\pi}{4}$, donc $AC = AB$ et $\paren{\overrightarrow{AB}, \overrightarrow{AC}} = \frac{3\pi}{4}$.
\item Le module est $2$ et l'argument est $\frac{\pi}{3}$, donc $AC = 2AB$ et $\paren{\overrightarrow{AB}, \overrightarrow{AC}} = \frac{\pi}{3}$.
\end{enumerate}
\end{proof}
\end{example}




\end{document}