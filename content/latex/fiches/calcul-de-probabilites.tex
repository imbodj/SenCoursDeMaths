\input{../common}

\begin{document}
  %<*content>
  \sheet{analysis}{calcul-de-probabilités}{Calcul de probabilité (TL)}
 \summary{Aide-mémoire de probabilité}
 \textbf{Vocabulaire}
\begin{itemize}
    \item \textbf{Expérience aléatoire} : expérience dont le résultat n’est pas prévisible.
    \item \textbf{Issue} : un des résultats possibles de l'expérience.
    \item \textbf{Univers} $\Omega$ : ensemble de toutes les issues possibles.
    \item \textbf{Événement} : sous-ensemble de l’univers (peut être simple ou composé).
    \item \textbf{Événement certain} : se réalise dans tous les cas, c’est $\Omega$.
    \item \textbf{Événement impossible} : ne se réalise jamais, c’est $\varnothing$.
    \item \textbf{Événements incompatibles} : ils ne peuvent pas se produire en même temps.
\end{itemize}
 \begin{property}
 \begin{itemize}
 \item Soient $A $ et $ B$ deux événements : \[P\paren{A\cup B}=P(A)+P(B)-P(A\cap B)\]
 \item Soient $A $ un  événement et $ \overline{A}$  son événement contraire: \[P\paren{\overline{A}}=1-P(A)\]
 \end{itemize}
 \end{property}
   \begin{definition}
   Lorsque tous les événements élémentaires de l'univers $ \Omega $ ont la même probabilité; on dit qu'il y a \textbf{équiprobabilité}.
  \end{definition}

  
\textbf{Conseils pratiques}

- Une probabilité est un nombre entre $ 0 $ et $ 1 $.\\
- La somme des probabilités de toutes les issues = 1.\\
- Bien identifier l’univers \( \Omega \) dès le départ.


  \begin{property}
  Dans un cas d'équiprobabilité, la probabilité d'un événement $A$ est : $$ P(A)=\dfrac{\text{card}A }{\text{card}\Omega}=\dfrac{\text{nombre d’issues favorables à } A}{\text{nombre total d’issues}}$$
  \end{property}

   \textbf{Tableau résumant les   différents tirages}\\
 Tirage de $p$ éléments dans un ensemble à $n$ éléments.
 
\[
\begin{array}{|c|c|c|c|c|}
\hline
\text{Type} & \text{Ordre ?} & \text{Distinction ?} & \text{Outil} & \text{Nb tirages} \\
\hline
\text{Avec remise}  &   \text{Oui}  & \text{Non} & p\text{-liste} &  n^{p}  \\
\hline
\text{Sans remise} & \text{Oui} & \text{Oui} & \text{Arrgt.} &  A_n^p \\
\hline
\text{Simultanés}  & \text{Non} & \text{Oui} & \text{Comb.} & C_n^p \\
\hline
\end{array}
\]

  %</content>
\end{document}
