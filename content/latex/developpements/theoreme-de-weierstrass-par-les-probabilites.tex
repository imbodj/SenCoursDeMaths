\documentclass[12pt, a4paper]{report}

% LuaLaTeX :

\RequirePackage{iftex}
\RequireLuaTeX

% Packages :

\usepackage[french]{babel}
%\usepackage[utf8]{inputenc}
%\usepackage[T1]{fontenc}
\usepackage[pdfencoding=auto, pdfauthor={Hugo Delaunay}, pdfsubject={Mathématiques}, pdfcreator={agreg.skyost.eu}]{hyperref}
\usepackage{amsmath}
\usepackage{amsthm}
%\usepackage{amssymb}
\usepackage{stmaryrd}
\usepackage{tikz}
\usepackage{tkz-euclide}
\usepackage{fontspec}
\defaultfontfeatures[Erewhon]{FontFace = {bx}{n}{Erewhon-Bold.otf}}
\usepackage{fourier-otf}
\usepackage[nobottomtitles*]{titlesec}
\usepackage{fancyhdr}
\usepackage{listings}
\usepackage{catchfilebetweentags}
\usepackage[french, capitalise, noabbrev]{cleveref}
\usepackage[fit, breakall]{truncate}
\usepackage[top=2.5cm, right=2cm, bottom=2.5cm, left=2cm]{geometry}
\usepackage{enumitem}
\usepackage{tocloft}
\usepackage{microtype}
%\usepackage{mdframed}
%\usepackage{thmtools}
\usepackage{xcolor}
\usepackage{tabularx}
\usepackage{xltabular}
\usepackage{aligned-overset}
\usepackage[subpreambles=true]{standalone}
\usepackage{environ}
\usepackage[normalem]{ulem}
\usepackage{etoolbox}
\usepackage{setspace}
\usepackage[bibstyle=reading, citestyle=draft]{biblatex}
\usepackage{xpatch}
\usepackage[many, breakable]{tcolorbox}
\usepackage[backgroundcolor=white, bordercolor=white, textsize=scriptsize]{todonotes}
\usepackage{luacode}
\usepackage{float}
\usepackage{needspace}
\everymath{\displaystyle}

% Police :

\setmathfont{Erewhon Math}

% Tikz :

\usetikzlibrary{calc}
\usetikzlibrary{3d}

% Longueurs :

\setlength{\parindent}{0pt}
\setlength{\headheight}{15pt}
\setlength{\fboxsep}{0pt}
\titlespacing*{\chapter}{0pt}{-20pt}{10pt}
\setlength{\marginparwidth}{1.5cm}
\setstretch{1.1}

% Métadonnées :

\author{agreg.skyost.eu}
\date{\today}

% Titres :

\setcounter{secnumdepth}{3}

\renewcommand{\thechapter}{\Roman{chapter}}
\renewcommand{\thesubsection}{\Roman{subsection}}
\renewcommand{\thesubsubsection}{\arabic{subsubsection}}
\renewcommand{\theparagraph}{\alph{paragraph}}

\titleformat{\chapter}{\huge\bfseries}{\thechapter}{20pt}{\huge\bfseries}
\titleformat*{\section}{\LARGE\bfseries}
\titleformat{\subsection}{\Large\bfseries}{\thesubsection \, - \,}{0pt}{\Large\bfseries}
\titleformat{\subsubsection}{\large\bfseries}{\thesubsubsection. \,}{0pt}{\large\bfseries}
\titleformat{\paragraph}{\bfseries}{\theparagraph. \,}{0pt}{\bfseries}

\setcounter{secnumdepth}{4}

% Table des matières :

\renewcommand{\cftsecleader}{\cftdotfill{\cftdotsep}}
\addtolength{\cftsecnumwidth}{10pt}

% Redéfinition des commandes :

\renewcommand*\thesection{\arabic{section}}
\renewcommand{\ker}{\mathrm{Ker}}

% Nouvelles commandes :

\newcommand{\website}{https://github.com/imbodj/SenCoursDeMaths}

\newcommand{\tr}[1]{\mathstrut ^t #1}
\newcommand{\im}{\mathrm{Im}}
\newcommand{\rang}{\operatorname{rang}}
\newcommand{\trace}{\operatorname{trace}}
\newcommand{\id}{\operatorname{id}}
\newcommand{\stab}{\operatorname{Stab}}
\newcommand{\paren}[1]{\left(#1\right)}
\newcommand{\croch}[1]{\left[ #1 \right]}
\newcommand{\Grdcroch}[1]{\Bigl[ #1 \Bigr]}
\newcommand{\grdcroch}[1]{\bigl[ #1 \bigr]}
\newcommand{\abs}[1]{\left\lvert #1 \right\rvert}
\newcommand{\limi}[3]{\lim_{#1\to #2}#3}
\newcommand{\pinf}{+\infty}
\newcommand{\minf}{-\infty}
%%%%%%%%%%%%%% ENSEMBLES %%%%%%%%%%%%%%%%%
\newcommand{\ensemblenombre}[1]{\mathbb{#1}}
\newcommand{\Nn}{\ensemblenombre{N}}
\newcommand{\Zz}{\ensemblenombre{Z}}
\newcommand{\Qq}{\ensemblenombre{Q}}
\newcommand{\Qqp}{\Qq^+}
\newcommand{\Rr}{\ensemblenombre{R}}
\newcommand{\Cc}{\ensemblenombre{C}}
\newcommand{\Nne}{\Nn^*}
\newcommand{\Zze}{\Zz^*}
\newcommand{\Zzn}{\Zz^-}
\newcommand{\Qqe}{\Qq^*}
\newcommand{\Rre}{\Rr^*}
\newcommand{\Rrp}{\Rr_+}
\newcommand{\Rrm}{\Rr_-}
\newcommand{\Rrep}{\Rr_+^*}
\newcommand{\Rrem}{\Rr_-^*}
\newcommand{\Cce}{\Cc^*}
%%%%%%%%%%%%%%  INTERVALLES %%%%%%%%%%%%%%%%%
\newcommand{\intff}[2]{\left[#1\;,\; #2\right]  }
\newcommand{\intof}[2]{\left]#1 \;, \;#2\right]  }
\newcommand{\intfo}[2]{\left[#1 \;,\; #2\right[  }
\newcommand{\intoo}[2]{\left]#1 \;,\; #2\right[  }

\providecommand{\newpar}{\\[\medskipamount]}

\newcommand{\annexessection}{%
  \newpage%
  \subsection*{Annexes}%
}

\providecommand{\lesson}[3]{%
  \title{#3}%
  \hypersetup{pdftitle={#2 : #3}}%
  \setcounter{section}{\numexpr #2 - 1}%
  \section{#3}%
  \fancyhead[R]{\truncate{0.73\textwidth}{#2 : #3}}%
}

\providecommand{\development}[3]{%
  \title{#3}%
  \hypersetup{pdftitle={#3}}%
  \section*{#3}%
  \fancyhead[R]{\truncate{0.73\textwidth}{#3}}%
}

\providecommand{\sheet}[3]{\development{#1}{#2}{#3}}

\providecommand{\ranking}[1]{%
  \title{Terminale #1}%
  \hypersetup{pdftitle={Terminale #1}}%
  \section*{Terminale #1}%
  \fancyhead[R]{\truncate{0.73\textwidth}{Terminale #1}}%
}

\providecommand{\summary}[1]{%
  \textit{#1}%
  \par%
  \medskip%
}

\tikzset{notestyleraw/.append style={inner sep=0pt, rounded corners=0pt, align=center}}

%\newcommand{\booklink}[1]{\website/bibliographie\##1}
\newcounter{reference}
\newcommand{\previousreference}{}
\providecommand{\reference}[2][]{%
  \needspace{20pt}%
  \notblank{#1}{
    \needspace{20pt}%
    \renewcommand{\previousreference}{#1}%
    \stepcounter{reference}%
    \label{reference-\previousreference-\thereference}%
  }{}%
  \todo[noline]{%
    \protect\vspace{20pt}%
    \protect\par%
    \protect\notblank{#1}{\cite{[\previousreference]}\\}{}%
    \protect\hyperref[reference-\previousreference-\thereference]{p. #2}%
  }%
}

\definecolor{devcolor}{HTML}{00695c}
\providecommand{\dev}[1]{%
  \reversemarginpar%
  \todo[noline]{
    \protect\vspace{20pt}%
    \protect\par%
    \bfseries\color{devcolor}\href{\website/developpements/#1}{[DEV]}
  }%
  \normalmarginpar%
}

% En-têtes :

\pagestyle{fancy}
\fancyhead[L]{\truncate{0.23\textwidth}{\thepage}}
\fancyfoot[C]{\scriptsize \href{\website}{\texttt{https://github.com/imbodj/SenCoursDeMaths}}}

% Couleurs :

\definecolor{property}{HTML}{ffeb3b}
\definecolor{proposition}{HTML}{ffc107}
\definecolor{lemma}{HTML}{ff9800}
\definecolor{theorem}{HTML}{f44336}
\definecolor{corollary}{HTML}{e91e63}
\definecolor{definition}{HTML}{673ab7}
\definecolor{notation}{HTML}{9c27b0}
\definecolor{example}{HTML}{00bcd4}
\definecolor{cexample}{HTML}{795548}
\definecolor{application}{HTML}{009688}
\definecolor{remark}{HTML}{3f51b5}
\definecolor{algorithm}{HTML}{607d8b}
%\definecolor{proof}{HTML}{e1f5fe}
\definecolor{exercice}{HTML}{e1f5fe}

% Théorèmes :

\theoremstyle{definition}
\newtheorem{theorem}{Théorème}

\newtheorem{property}[theorem]{Propriété}
\newtheorem{proposition}[theorem]{Proposition}
\newtheorem{lemma}[theorem]{Activité d'introduction}
\newtheorem{corollary}[theorem]{Conséquence}

\newtheorem{definition}[theorem]{Définition}
\newtheorem{notation}[theorem]{Notation}

\newtheorem{example}[theorem]{Exemple}
\newtheorem{cexample}[theorem]{Contre-exemple}
\newtheorem{application}[theorem]{Application}

\newtheorem{algorithm}[theorem]{Algorithme}
\newtheorem{exercice}[theorem]{Exercice}

\theoremstyle{remark}
\newtheorem{remark}[theorem]{Remarque}

\counterwithin*{theorem}{section}

\newcommand{\applystyletotheorem}[1]{
  \tcolorboxenvironment{#1}{
    enhanced,
    breakable,
    colback=#1!8!white,
    %right=0pt,
    %top=8pt,
    %bottom=8pt,
    boxrule=0pt,
    frame hidden,
    sharp corners,
    enhanced,borderline west={4pt}{0pt}{#1},
    %interior hidden,
    sharp corners,
    after=\par,
  }
}

\applystyletotheorem{property}
\applystyletotheorem{proposition}
\applystyletotheorem{lemma}
\applystyletotheorem{theorem}
\applystyletotheorem{corollary}
\applystyletotheorem{definition}
\applystyletotheorem{notation}
\applystyletotheorem{example}
\applystyletotheorem{cexample}
\applystyletotheorem{application}
\applystyletotheorem{remark}
%\applystyletotheorem{proof}
\applystyletotheorem{algorithm}
\applystyletotheorem{exercice}

% Environnements :

\NewEnviron{whitetabularx}[1]{%
  \renewcommand{\arraystretch}{2.5}
  \colorbox{white}{%
    \begin{tabularx}{\textwidth}{#1}%
      \BODY%
    \end{tabularx}%
  }%
}

% Maths :

\DeclareFontEncoding{FMS}{}{}
\DeclareFontSubstitution{FMS}{futm}{m}{n}
\DeclareFontEncoding{FMX}{}{}
\DeclareFontSubstitution{FMX}{futm}{m}{n}
\DeclareSymbolFont{fouriersymbols}{FMS}{futm}{m}{n}
\DeclareSymbolFont{fourierlargesymbols}{FMX}{futm}{m}{n}
\DeclareMathDelimiter{\VERT}{\mathord}{fouriersymbols}{152}{fourierlargesymbols}{147}

% Code :

\definecolor{greencode}{rgb}{0,0.6,0}
\definecolor{graycode}{rgb}{0.5,0.5,0.5}
\definecolor{mauvecode}{rgb}{0.58,0,0.82}
\definecolor{bluecode}{HTML}{1976d2}
\lstset{
  basicstyle=\footnotesize\ttfamily,
  breakatwhitespace=false,
  breaklines=true,
  %captionpos=b,
  commentstyle=\color{greencode},
  deletekeywords={...},
  escapeinside={\%*}{*)},
  extendedchars=true,
  frame=none,
  keepspaces=true,
  keywordstyle=\color{bluecode},
  language=Python,
  otherkeywords={*,...},
  numbers=left,
  numbersep=5pt,
  numberstyle=\tiny\color{graycode},
  rulecolor=\color{black},
  showspaces=false,
  showstringspaces=false,
  showtabs=false,
  stepnumber=2,
  stringstyle=\color{mauvecode},
  tabsize=2,
  %texcl=true,
  xleftmargin=10pt,
  %title=\lstname
}

\newcommand{\codedirectory}{}
\newcommand{\inputalgorithm}[1]{%
  \begin{algorithm}%
    \strut%
    \lstinputlisting{\codedirectory#1}%
  \end{algorithm}%
}



% Bibliographie :

%\addbibresource{\bibliographypath}%
\defbibheading{bibliography}[\bibname]{\section*{#1}}
\renewbibmacro*{entryhead:full}{\printfield{labeltitle}}%
\DeclareFieldFormat{url}{\newline\footnotesize\url{#1}}%

\AtEndDocument{%
  \newpage%
  \pagestyle{empty}%
  \printbibliography%
}


\begin{document}
  %<*content>
  \development{analysis}{theoreme-de-weierstrass-par-les-probabilites}{Théorème de Weierstrass (par les probabilités)}

  \summary{On montre le théorème de Weierstrass en faisant un raisonnement sur des variables aléatoires suivant une loi de Bernoulli.}

  \reference[G-K]{195}

  \begin{theorem}[Bernstein]
    \label{theoreme-de-weierstrass-par-les-probabilites-1}
    Soit $f : [0,1] \rightarrow \mathbb{R}$ continue. On note
    \[ \forall n \in \mathbb{N}^*, \, B_n(f) : x \mapsto \sum_{k=0}^n \binom{n}{k} f \left( \frac{k}{n} \right) x^k (1-x)^{n-k} \]
    le $n$-ième polynôme de Bernstein associé à $f$. Alors le suite de fonctions $(B_n(f))$ converge uniformément vers $f$.
  \end{theorem}

  \begin{proof}
    Soit $x \in ]0, 1[$. On se place sur un espace probabilité $(\Omega, \mathcal{A}, \mathbb{P})$ et considère $(X_k)$ une suite de variables aléatoires indépendantes de même loi $\mathcal{B}(x)$. On note $\forall n \in \mathbb{N}^*$, $S_n = \sum_{k=1}^n X_k$. Ainsi, $S_n \sim \mathcal{B}(n, x)$ et donc par la formule de transfert,
    \[ \mathbb{E} \left( f \left( \frac{S_n}{n} \right) \right) = \sum_{k=0}^n \binom{n}{k} f \left( \frac{k}{n} \right) x^k (1-x)^{n-k} = B_n(f)(x) \]
    La fonction $f$ est continue sur $[0,1]$ qui est un compact de $\mathbb{R}$, donc par le théorème de Heine; elle y est uniformément continue. Soit donc $\epsilon > 0$,
    \[ \exists \eta > 0 \text{ tel que } \forall x, y \in [0,1], \, |x-y| < \eta \implies |f(x) - f(y)| < \epsilon \]
    On a,
    \begin{align*}
      |B_n(f)(x) - f(x)| &= \left| \mathbb{E} \left( f \left( \frac{S_n}{n} \right) \right) - f(x) \right| \\
      &= \left| \mathbb{E} \left( f \left( \frac{S_n}{n} \right) - f(x) \right) \right| \\
      &\leq \mathbb{E} \left| f \left( \frac{S_n}{n} \right) - f(x) \right| \\
      &\leq \mathbb{E} \left( \mathbb{1}_{\left \{ \left| \frac{S_n}{n} - x \right| < \eta \right \}} \left| f \left( \frac{S_n}{n} \right) - f(x) \right| \right) + \mathbb{E} \left( \mathbb{1}_{\left \{ \left| \frac{S_n}{n} - x \right| \geq \eta \right \}} \left| f \left( \frac{S_n}{n} \right) - f(x) \right| \right) \\
      &\leq \mathbb{E} (\epsilon) + 2 \Vert f \Vert_\infty \mathbb{E} \left( \mathbb{1}_{\left \{ \left| \frac{S_n}{n} - x \right| \geq \eta \right \}} \right) \\
      &= \epsilon + 2 \Vert f \Vert_\infty \mathbb{P} \left( \left| \frac{S_n}{n} - x \right| \geq \eta \right) \tag{$*$}
    \end{align*}
    Comme $\mathbb{E} \left( \frac{S_n}{n} \right) = x$, on peut appliquer l'inégalité de Bienaymé-Tchebychev :
    \[ \mathbb{P} \left( \left| \frac{S_n}{n} - x \right| \geq \eta \right) = \mathbb{P} \left( \left| \frac{S_n}{n} - \mathbb{E} \left( \frac{S_n}{n} \right) \right| \geq \eta \right) \leq \frac{1}{\eta^2} \operatorname{Var} \left( \frac{S_n}{n} \right) \]
    Comme les $X_k$ sont indépendantes et de même loi :
    \[ \operatorname{Var} \left( \frac{S_n}{n} \right) = \frac{1}{n^2} \operatorname{Var} (S_n) = \frac{1}{n} \operatorname{Var}(X_1) = \frac{x(1-x)}{n} \leq \frac{1}{n} \]
    En réinjectant cela dans $(*)$, cela donne
    \[ |B_n(f)(x) - f(x)| \leq \epsilon + \frac{2 \Vert f \Vert_\infty}{n \eta^2} \]
    qui est une majoration indépendante de $x$. Comme la fonction $B_n(f) - f$ est continue sur $[0, 1]$, on peut passer à la borne supérieure :
    \[ \Vert B_n(f) - f \Vert_\infty = \sup_{x \in [0, 1]} |B_n(f)(x) - f(x)| \leq \epsilon + \frac{2 \Vert f \Vert_\infty}{n \eta^2} \]
    ce qui donne après un passage à la limite supérieure :
    \begin{align*}
      &\limsup_{n \rightarrow +\infty} \Vert B_n(f) - f \Vert_\infty \leq \epsilon \\
      \overset{\epsilon \longrightarrow 0}{\implies} &\limsup_{n \rightarrow +\infty} \Vert B_n(f) - f \Vert_\infty = 0 \\
      \implies &\lim_{n \rightarrow +\infty} \Vert B_n(f) - f \Vert_\infty = 0
    \end{align*}
  \end{proof}

  \begin{theorem}[Weierstrass]
    Toute fonction continue $f : [a,b] \rightarrow \mathbb{R}$ (avec $a, b \in \mathbb{R}$ tels que $a \leq b$) est limite uniforme de fonctions polynômiales sur $[a, b]$.
  \end{theorem}

  \begin{proof}
    On va avoir besoin du changement de variable suivant :
    \[ \varphi :
    \begin{array}{ccc}
      [0,1] &\rightarrow& [a, b] \\
      x &\mapsto& a + (b-a) x
    \end{array}
    \]
    Par le \cref{theoreme-de-weierstrass-par-les-probabilites-1}, la fonction $f \circ \varphi^{-1}$ est limite uniforme d'une suite de fonctions polynômiales $(p_n)$. Donc $f$ est limite uniforme de la suite $(p_n \circ \varphi)$ où $\forall n \in \mathbb{N}$, $p_n \circ \varphi$ est bien une fonction polynômiale car $\varphi$ est affine.
  \end{proof}
  %</content>
\end{document}
