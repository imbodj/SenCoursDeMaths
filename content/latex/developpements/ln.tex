\input{../common}

\begin{document}
	%<*content>
	\development{analysis}{ln}{Logarithme népérien}

 \summary{}
 
	\begin{exercice}
Exprimer en fonction de $ \ln 3 $ chacun des nombres suivants

\begin{enumerate}
\item $ \ln \frac{1}{9} $ 
\item $ \ln 63-\ln 7 $
\item  $ \ln \sqrt{27} $                                               
\item $ 4\ln 6-\ln 16  $
\item $ \ln(3\eexp{2})  $
\end{enumerate}

  \end{exercice}
  
  \begin{exercice}
Exprimer en fonction de $ \ln 2 $  les nombres suivants.

\begin{enumerate}
\item $ \ln 32 $
\item $ \ln \frac{1}{16} $ 
\item $ \ln 40-\ln 5 $
\item  $ \ln 4\sqrt{2} $                                               
\item $ 4\ln 2-\ln 8  $
\item $ \ln \frac{1}{1024} $
\end{enumerate}


  \end{exercice}
  
  
 
   \begin{exercice}

Exprimer en fonction de $ \ln 3 $  et $ \ln 5 $ les nombres suivants.

\begin{enumerate}
\item $ \ln \frac{27}{25} $
\item $4\ln 15+\ln 81  $ 
\item $ \ln 25-\ln 15 $
\item  $ \ln 15\sqrt{25} $                                               
\item $ 4\ln6 -2\ln 20  $
\item $ \ln 675 $
\end{enumerate}

  \end{exercice}
  
   \begin{exercice}

Exprimer  chacun des nombres suivants sous la forme $ \ln A $ où $ A $ est un réel strictement positif.

\begin{enumerate}
\item $ \ln 4+\ln 5 $
\item $4\ln 6-\ln 7 $ 
\item $ \frac{1}{2}\ln 3-\ln 5 $
\item  $ 1-2\ln 6 $                                               
\item $ -\ln2 +1  $
\item $ 3\ln 5 +2\ln 3$
\item $ -2\ln 3 +2\ln 2$
\end{enumerate}

  \end{exercice}
  
   \begin{exercice}

Simplifier au maximum.

\begin{enumerate}
\item $ \ln 8-\ln 2 $
\item $4\ln 6+\ln 3 $ 
\item $\ln 25 - \ln 30 + \ln 10 $
\item  $ \ln 50 + \ln 2 - \ln 10 $                                               
\item $ 2 \ln 2 - \ln 16 + \ln 128  $
\item $ 3\ln \eexp{} +2\ln \eexp{2}$
\item $ -2\ln \eexp{3} +\ln \eexp{-2}-\ln \eexp{2}$
\item $ 3\ln \paren{\frac{2}{\eexp{} }} +\ln 2\eexp{3} +1$
\end{enumerate}

  \end{exercice}

 
   \begin{exercice}
    Résoudre dans $ \Rr $  les équations suivantes.
  
\begin{enumerate}
\item $ \ln x=5 $
\item $ \ln x+4=0 $
\item $ \ln (3-2x)=5 $  
\item $ 2\ln x- 6=0$  
\item  $ 1-4\ln x=\ln x -9 $  
\item $ \paren{\ln x}^{2}=1$  
\end{enumerate}

  \end{exercice}
  
  \begin{exercice}
Résoudre dans $ \Rr $  les équations suivantes.
   
\begin{enumerate}
\item $ \ln (x +2)=\ln2 $  
\item $ \ln (2x- 6)=1$  
\item  $ 4\ln(1-x)=8 $  
\item $ \ln(x+1)=\ln x$  
\item $ \ln(2x-3)=\ln (x-2)$ 
\item $ \ln(2x)=\ln (x+1)$ 
\end{enumerate}


  \end{exercice}
  
   \begin{exercice}
Résoudre dans $ \Rr $  les équations suivantes.

\begin{enumerate}
\item $ \ln (x +1)+\ln x=0 $  
\item $ \ln (3- x )=3\ln 2$  
\item  $ \ln(3-x)\times \ln(x+1)=0  $  
\item $ \ln(5x-6)-2\ln x=0$  
\end{enumerate}


  \end{exercice}
  
  

\begin{exercice}
Résoudre dans $ \Rr $  les équations suivantes.
   %\begin{multicols}{2}
\begin{enumerate}
\item $ \ln (x -2)-\ln (x+1)=2\ln 2 $  
\item $ \ln ( x-2 )+\ln (x+3)=\ln (5x-9)$  
\item  $ \ln(x-1)=\ln(2-x)  $ 
 \item $ \ln(-x+1)+\ln (-x+2)=\ln (x+7)$ 
\item $ 2\ln(x+1)+\ln (x-1)=3\ln x$ 
\item  $ \ln \paren{\frac{x-1}{2x-1}} =0$ 
\end{enumerate}
%\end{multicols}
  \end{exercice}
  
  \begin{exercice}
Résoudre dans $ \Rr $   les équations suivantes.
\begin{enumerate} 
\item $\paren{\ln x}^{2} -\ln x-2=0  $ 
\item  $\paren{\ln (x-1)}^{2} -\ln (x-1)-2=0  $ 
\item $3\paren{\ln x}^{2} +\ln x-1=0  $                                                   
\item $\paren{\ln x}^{2} -6\ln x+9=0  $
\item $\ln x^{2} -6\ln x+4=0  $  
\end{enumerate}

  \end{exercice}
  
  \begin{exercice}
Résoudre dans $ \Rr $   les équations suivantes.
\begin{enumerate} 
\item $\ln (3x - 4) = \ln(2x + 1)$
\item $ \ln(4 - 2x) = \ln(x - 1)$
\item  $ (\ln x)^{2} - 3\ln x + 2 = 0$
\item $ 2(\ln x)^{2} - 5\ln x - 3 = 0$
\item $ \ln(x^{2} -3 x + 2) = 2\ln(x + 4)$
\item $  \ln(2x^{2} - 10x + 8) = \ln(3x^{2} - 3x - 18)$
\end{enumerate}

  \end{exercice}
 
   \begin{exercice}
    Résoudre dans $ \Rr $  les inéquations suivantes.
   
\begin{enumerate}
\item $ \ln x \leq1 $  
\item $ 2\ln x> \ln 3$  
\item  $ 4\ln x+6\geq0 $  
\item $ 3\ln x-4\leq\ln x$ 
\item  $( 1,2)^{n} \geq 4 \quad n\in\Nn$ 
\item  $( 0,02)^{n} \geq 4 \quad n\in\Nn$ 
\item  $( 5,5)^{n} < 20 \quad n\in\Nn$ 
\item  $( 0,007)^{n} \leq 0.001 \quad n\in\Nn$ 
\end{enumerate}

  \end{exercice}
  
  \begin{exercice}
Résoudre dans $ \Rr $  les inéquations suivantes.
   
\begin{enumerate}
\item $ \ln (x +1)\leq0 $  
\item $ \ln (x- 6)>1$   
\item $ 2\ln(3-x)<1$ 
\item  $ \ln(x-2)>\ln x $
\item  $ \ln(x-2)>1 $
\item $ (1-3x)\ln x\geq 0 $
\end{enumerate}

 \end{exercice}
 
  \begin{exercice}
Résoudre dans $ \Rr $  les inéquations suivantes.
\begin{enumerate}
\item  $ \ln(5x + 20) > \ln(3x - 9) $
\item $ \ln(8 - 2x) \leq \ln(5x- 25) $ 
\item $ \ln(x^{2}-1) \leq \ln(2x+2 ) $ 
\item $ \ln(x^{2} + 1) < \ln(2x^{2} + x + 2) $

\end{enumerate}

  \end{exercice}
  
   \begin{exercice}
  On considère le polynôme $ P(x)=2x^{3}-9x^{2}+x+12$.
\begin{enumerate}
\item Montrer que $ -1 $  est une racine de $ P(x) $.
\item En déduire une factorisation de $  P(x) $.
\item Résoudre dans $ \Rr $  l'équation $ P(x)=0 $.
\item En déduire les  solutions des équation  et inéquation suivantes.
\begin{enumerate}
\item $ 2(\ln(x))^{3}-9(\ln(x))^{2}+\ln(x)+12=0 $.
\item $ 2(\ln(2x+3))^{3}-9(\ln(2x+3))^{2}+\ln(2x+3)+12=0 $.
\item $ 2(\ln(x))^{3}-9(\ln(x))^{2}+\ln(x)+12<0 $.
\item $\ln(2x-3) +2\ln(x-2) =\ln(-2x^{2}+19x-24)$
\end{enumerate}
\end{enumerate}
 \end{exercice}
  
 \begin{exercice}

\begin{enumerate} 
\item Résoudre dans $ \Rr $   l'équation $x^{3}+2x^{2}-x-2=0  $ 
\item En déduire  la résolution des équations suivantes.
\begin{enumerate} 
\item $\paren{\ln x}^{3}+2\paren{\ln x}^{2}-\ln x-2=0  $                                                   
\item $\paren{\ln (x-1)}^{3}+2\paren{\ln (x-1)}^{2}-\ln (x-1)-2=0  $                                                   
\item $ \ln (x^{2} +2x-1)=\ln 2-\ln x$ 

\end{enumerate}
\end{enumerate}

  \end{exercice}
  

  
  \begin{exercice}
Résoudre les systèmes d'équations suivants
\begin{enumerate}
\item $  \left\{\begin{array}{l}-\ln x+ 2\ln y=1  \\ 3\ln x-5\ln y =-1 \end{array}\right.$
\item $  \left\{\begin{array}{l}2\ln x-3\ln y=5  \\ \ln x+2\ln y =-1 \end{array}\right.$
\item $  \left\{\begin{array}{l} x+y=2  \\ \ln x-\ln y =\ln 3\end{array}\right.$
\item $  \left\{\begin{array}{l} x+y=3  \\ \ln x+\ln y =\ln 2\end{array}\right.$
\item $  \left\{\begin{array}{l} \ln \paren{xy}=4  \\ \paren{\ln x}\paren{\ln y} =-12 \end{array}\right.$
\item $  \left\{\begin{array}{l} \ln \paren{xy}=-2  \\ \paren{\ln x}\paren{\ln y} =-15 \end{array}\right.$
\item $  \left\{\begin{array}{l} \ln x+\ln y=2  \\ \paren{\ln x}\paren{\ln y} =-24 \end{array}\right.$
\item $  \left\{\begin{array}{l}2\ln (x+3)+ 3\ln (4-y)=4  \\ 5\ln (x+3)-3\ln (4-y) =-11 \end{array}\right.$
\item $  \left\{\begin{array}{l}\ln x^{3}-\ln y^{2}=-4  \\ \ln x+\ln y^{4} =1 \end{array}\right.$
\end{enumerate}

  \end{exercice}
 
  \begin{exercice}
Dans chaque cas déterminer l'ensemble de définition de la fonction $ f. $

\begin{enumerate}
\item  $ f(x)=x+\ln \paren{x +3} $ 
\item  $ f(x)=\ln x+\ln(x +3) $ 
\item $ f(x)=\ln \paren{-x^{2}+2x +3} $  
\item  $ f(x)=\ln \paren{x^{2} -1} $                                                   
\item $ f(x)=\dfrac{4}{\ln (x-2)} $ 
\item $ f(x)=\dfrac{1}{\ln x-1}$
\item $ f(x)= \dfrac{\ln (1-x)}{x+2} $

\end{enumerate}

  \end{exercice}
 
  \begin{exercice}
 Calculer la dérivée de  $ f $  dans chaque cas.

\begin{enumerate}
\item $ f(x)=\ln x-x-1 $ 
\item $ f(x)=2x+\ln (3x-1)$  
\item $f(x)= x\ln x $  
\item  $f(x)= \ln x\ln(2-x)  $  
 \item $ f(x)= \dfrac{1}{\ln x}$ 
  \item $ f(x)= \dfrac{x+1}{\ln x}$
   \item $ f(x)= \paren{\ln x}^{2}$ 
   \item  $f(x)= \ln(2x^{2}+3x)  $  
   \item $ f(x)=\ln \paren{\dfrac{2x+3}{3x-6}} $
   
\end{enumerate}

 \end{exercice}

  \begin{exercice}

 Etudier et représenter graphiquement  $ f $  dans chaque cas.

\begin{enumerate}
\item $ f(x)=\ln x $ 
\item $ f(x)=\ln x^{2} $
\item $ f(x)=\paren{\ln x }^{2}$   
\item $f(x)= x\ln x $  
\item $f(x)= x^{2}\ln x $ 
\item $f(x)= x\ln \abs{x} $   
\item  $f(x)= \dfrac{x}{\ln x}  $  
 \item $ f(x)= \dfrac{\ln x}{x}$ 
  \item  $f(x)= \ln(x-2)  $ 
   \item  $f(x)= \ln(4-2x)  $ 
 % \item  $f(x)= \ln(x^{2}-2x)  $  
 % \item  $f(x)= \ln(-x^{2}+2x)  $ 
%  \item  $f(x)= \ln(x^{2}-4)  $ 
 % \item  $f(x)= \ln(4-x^{2})  $ 
  % \item $ f(x)=\ln \paren{\dfrac{x+1}{x-1}} $
   % \item $ f(x)=\ln \paren{\dfrac{x}{x+2}} $
   
\end{enumerate}

 \end{exercice}
 
 \begin{exercice}
 
 Soit la fonction $ f $ définie par : $ f(x)=\ln(-2x^{2}+x+1)  $\\ et  de  représentation $ \mathscr{C} $.
 \begin{enumerate}
 \item Montrer que le domaine de définition de $ f $  est\\ $D=\intoo{-\frac{1}{2}}{1} $.
 \item Calculer les limites aux bornes de $ D $.
 \item Démontrer que pour tout $ x\in D$,\\ $ f^{\prime}(x)=\dfrac{-4x+1}{-2x^{2}+x+1} $
  \item Dresser le tableau de variation de $ f. $
  \item Déterminer les points d'intersection de $ \mathscr{C} $ avec l'axe des abscisses.
  \item Déterminer l'équation de la tangente au point d'abscisse $ 0 $.
 \end{enumerate}
  \end{exercice}
 
  \begin{exercice}

\medskip

 Soit la fonction $ f $ définie par : \\$ f(x)=\ln\paren{\dfrac{3x-6}{x}}  $, de  représentation $ \mathscr{C} $.
 \begin{enumerate}
 \item Montrer que le domaine de définition de $ f $  est\\ $D=\intoo{\minf}{0} \cup\intoo{2}{\pinf} $.
 \item Calculer les limites aux bornes de $ D $. \\Préciser les asymptotes à $ \mathscr{C} $.
 \item Démontrer que pour tout $ x\in D$,\; $ f^{\prime}(x)=\dfrac{6}{x(3x-6)} $
  \item Etudier le signe de  $  f^{\prime}(x) $ puis dresser le tableau de variation de $ f. $
  \item Déterminer le point   A intersection de $ \mathscr{C} $ avec l'axe des abscisses.
  \item Déterminer l'équation de la tangente au point d'abscisse $ 3 $.
  \item Montrer que le point I($ 1,\; \ln 3) $  est un centre de symétrie de $ \mathscr{C} $.
  \item Construire $ \mathscr{C} $.
 \end{enumerate}
  \end{exercice}
  
   \begin{exercice}


 Soit la fonction $ f $ définie par : $ f(x)=\frac{1}{x}+2\ln\paren{x+1}  $, de  représentation $ \mathscr{C} $.
 \begin{enumerate}
 \item Déterminer le domaine de définition $ D $   de $ f. $
 \item Calculer les limites aux bornes de  ce domaine. \\Préciser les asymptotes à $ \mathscr{C} $.
 \item Démontrer que pour tout $ x\in D$,\\ $ f^{\prime}(x)=\dfrac{2x^{2}-x-1}{x^{2}(x+1)} $
  \item Dresser le tableau de variation de $ f. $
  \item Construire $ \mathscr{C} $.
 \end{enumerate}
  \end{exercice}
  
   \begin{exercice}


 Soit la fonction $ f $ définie par : $ f(x)=\paren{\ln x-2} \ln x $, de  représentation $ \mathscr{C} $.
 \begin{enumerate}
 \item Déterminer le domaine de définition $ D $   de $ f. $
 \item Calculer les limites aux bornes de  ce domaine.
 \item Déterminer $ f^{\prime}(x)$.
  \item Dresser le tableau de variation de $ f. $
  \item Montrer que $ \mathscr{C} $.  coupe l'axe des abscisses en deux points A et B dont on précisera les coordonnées.
  \item Déterminer les équations des tangentes en A et B.
 \end{enumerate}
  \end{exercice}
  
  \begin{exercice}

 \begin{enumerate}
 \item 
 Soit la fonction $ f $ définie par : $ f(x)=\paren{\ln ax+b} $  où $ a$ et $b $  sont des réels  et  $ \mathscr{C} $ sa  représentation graphique.
 \begin{enumerate}
 \item Déterminer  $ f^{\prime}(x) $ en fonction de $ a$ et $b $.
 \item Calculer les réels $ a$ et $b $  pour que $ \mathscr{C} $  passe par le point I($1\;$, $\;0$) et admette en ce point une tangente  (T) parallèle à la droite (D) : $ y=-x. $
 \end{enumerate}
 \item
  Dans la suite on prend $a=-1 $ et $ b=2$ et  donc $ f(x)=\ln \paren{-x+2} $ 
 \begin{enumerate}
  \item Dresser le tableau de variation de $ f. $
  \item Ecrire une équation de la tangente (T).
  \item Caluler $ \limi{x}{\pinf}{\dfrac{f(x)}{x}} $.\\
  
   En déduire la nature de la branche infinie à $ \mathscr{C} $.
   \item Déterminer les coordonnées du point J intersection de $ \mathscr{C} $  avec l'axe des ordonnées.
  \item Tracer la courbe $ \mathscr{C} $.
  \end{enumerate}
 \end{enumerate}
  \end{exercice}
  
  \begin{exercice}

 Soit la fonction $ f $ définie par : $ f(x)=x-2+\ln\paren{\dfrac{x-2}{x+2}}  $, de  représentation $ \mathscr{C} $.
 \begin{enumerate}
 \item Etudier le signe de    $ \dfrac{x-2}{x+2} $ en déduire le domaine de définition de $ f $. 
 \item Calculer les limites aux bornes du domaine de définition et préciser les asymptotes à $ \mathscr{C} $.
 \item Calculer  $ f^{\prime}(x) $.
  \item Dresser le tableau de variation de $ f. $
  \item Montrer que la droite d'équation $ y=x-2 $  est une asymptote à $ \mathscr{C} $ 
  \item Montrer que le point I($ 0,\; 2) $  est un centre de symétrie de $ \mathscr{C} $.
  \item Construire $ \mathscr{C} $.
 \end{enumerate}
  \end{exercice}
  
  \begin{exercice}

 Soit la fonction $ f $ définie par : $ f(x)=-\dfrac{1}{2}x+\ln\paren{\dfrac{x-1}{x}}  $, de  représentation $ \mathscr{C} $.
 \begin{enumerate}
 \item Résoudre l'inéquation   $ \dfrac{x-1}{x}>0 $ 
 \item En déduire le domaine de définition  D de $ f $. 
 \item Calculer les limites aux bornes du domaine de définition.
 \item Montrer que   $ f^{\prime}(x) =\dfrac{-x^{2}+x+2}{2x(x-1)}$  pour $ x\in D $.
  \item Dresser le tableau de variation de $ f. $
  \item Montrer que la droite d'équation $ (\Delta) $:\;  $ y=-\dfrac{1}{2}x $  est une asymptote oblique  à $ \mathscr{C} $ .
  \item Etudier la position de $ \mathscr{C} $   par rapport à $ (\Delta) $ sur les intervalles $\intoo{\minf}{0} $ et $\intoo{0}{\pinf} $.
  \item Montrer que le point I$\paren{ \dfrac{1}{2},\; -\dfrac{1}{4}} $  est un centre de symétrie de $ \mathscr{C} $.
  \item Construire $ \mathscr{C} $.
 \end{enumerate}
  \end{exercice}
  
  
   \begin{exercice}

 Soit la fonction $ f $ définie par : $ f(x)=\dfrac{1+2\ln x}{2x} $, de  représentation $ \mathscr{C} $.
 \begin{enumerate} 
 \item Déterminer  le domaine de définition  D de $ f $. 
 \item Calculer les limites aux bornes du domaine de définition. \\On précisera les asymptotes éventuelles.
 \item Calculer   $ f^{\prime}(x)$  pour $ x\in D $.
  \item Dresser le tableau de variation de $ f. $
   \item Déterminer le point   A intersection de $ \mathscr{C} $ avec l'axe des abscisses.
   \item Déterminer l'équation de la tangente au point A.
  \item Construire  les tangentes, les asymptote et la courbe $ \mathscr{C} $.
 \end{enumerate}
  \end{exercice}

	%</content>
\end{document}
