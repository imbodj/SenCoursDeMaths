\documentclass[12pt, a4paper]{report}

% LuaLaTeX :

\RequirePackage{iftex}
\RequireLuaTeX

% Packages :

\usepackage[french]{babel}
%\usepackage[utf8]{inputenc}
%\usepackage[T1]{fontenc}
\usepackage[pdfencoding=auto, pdfauthor={Hugo Delaunay}, pdfsubject={Mathématiques}, pdfcreator={agreg.skyost.eu}]{hyperref}
\usepackage{amsmath}
\usepackage{amsthm}
%\usepackage{amssymb}
\usepackage{stmaryrd}
\usepackage{tikz}
\usepackage{tkz-euclide}
\usepackage{fontspec}
\defaultfontfeatures[Erewhon]{FontFace = {bx}{n}{Erewhon-Bold.otf}}
\usepackage{fourier-otf}
\usepackage[nobottomtitles*]{titlesec}
\usepackage{fancyhdr}
\usepackage{listings}
\usepackage{catchfilebetweentags}
\usepackage[french, capitalise, noabbrev]{cleveref}
\usepackage[fit, breakall]{truncate}
\usepackage[top=2.5cm, right=2cm, bottom=2.5cm, left=2cm]{geometry}
\usepackage{enumitem}
\usepackage{tocloft}
\usepackage{microtype}
%\usepackage{mdframed}
%\usepackage{thmtools}
\usepackage{xcolor}
\usepackage{tabularx}
\usepackage{xltabular}
\usepackage{aligned-overset}
\usepackage[subpreambles=true]{standalone}
\usepackage{environ}
\usepackage[normalem]{ulem}
\usepackage{etoolbox}
\usepackage{setspace}
\usepackage[bibstyle=reading, citestyle=draft]{biblatex}
\usepackage{xpatch}
\usepackage[many, breakable]{tcolorbox}
\usepackage[backgroundcolor=white, bordercolor=white, textsize=scriptsize]{todonotes}
\usepackage{luacode}
\usepackage{float}
\usepackage{needspace}
\everymath{\displaystyle}

% Police :

\setmathfont{Erewhon Math}

% Tikz :

\usetikzlibrary{calc}
\usetikzlibrary{3d}

% Longueurs :

\setlength{\parindent}{0pt}
\setlength{\headheight}{15pt}
\setlength{\fboxsep}{0pt}
\titlespacing*{\chapter}{0pt}{-20pt}{10pt}
\setlength{\marginparwidth}{1.5cm}
\setstretch{1.1}

% Métadonnées :

\author{agreg.skyost.eu}
\date{\today}

% Titres :

\setcounter{secnumdepth}{3}

\renewcommand{\thechapter}{\Roman{chapter}}
\renewcommand{\thesubsection}{\Roman{subsection}}
\renewcommand{\thesubsubsection}{\arabic{subsubsection}}
\renewcommand{\theparagraph}{\alph{paragraph}}

\titleformat{\chapter}{\huge\bfseries}{\thechapter}{20pt}{\huge\bfseries}
\titleformat*{\section}{\LARGE\bfseries}
\titleformat{\subsection}{\Large\bfseries}{\thesubsection \, - \,}{0pt}{\Large\bfseries}
\titleformat{\subsubsection}{\large\bfseries}{\thesubsubsection. \,}{0pt}{\large\bfseries}
\titleformat{\paragraph}{\bfseries}{\theparagraph. \,}{0pt}{\bfseries}

\setcounter{secnumdepth}{4}

% Table des matières :

\renewcommand{\cftsecleader}{\cftdotfill{\cftdotsep}}
\addtolength{\cftsecnumwidth}{10pt}

% Redéfinition des commandes :

\renewcommand*\thesection{\arabic{section}}
\renewcommand{\ker}{\mathrm{Ker}}

% Nouvelles commandes :

\newcommand{\website}{https://github.com/imbodj/SenCoursDeMaths}

\newcommand{\tr}[1]{\mathstrut ^t #1}
\newcommand{\im}{\mathrm{Im}}
\newcommand{\rang}{\operatorname{rang}}
\newcommand{\trace}{\operatorname{trace}}
\newcommand{\id}{\operatorname{id}}
\newcommand{\stab}{\operatorname{Stab}}
\newcommand{\paren}[1]{\left(#1\right)}
\newcommand{\croch}[1]{\left[ #1 \right]}
\newcommand{\Grdcroch}[1]{\Bigl[ #1 \Bigr]}
\newcommand{\grdcroch}[1]{\bigl[ #1 \bigr]}
\newcommand{\abs}[1]{\left\lvert #1 \right\rvert}
\newcommand{\limi}[3]{\lim_{#1\to #2}#3}
\newcommand{\pinf}{+\infty}
\newcommand{\minf}{-\infty}
%%%%%%%%%%%%%% ENSEMBLES %%%%%%%%%%%%%%%%%
\newcommand{\ensemblenombre}[1]{\mathbb{#1}}
\newcommand{\Nn}{\ensemblenombre{N}}
\newcommand{\Zz}{\ensemblenombre{Z}}
\newcommand{\Qq}{\ensemblenombre{Q}}
\newcommand{\Qqp}{\Qq^+}
\newcommand{\Rr}{\ensemblenombre{R}}
\newcommand{\Cc}{\ensemblenombre{C}}
\newcommand{\Nne}{\Nn^*}
\newcommand{\Zze}{\Zz^*}
\newcommand{\Zzn}{\Zz^-}
\newcommand{\Qqe}{\Qq^*}
\newcommand{\Rre}{\Rr^*}
\newcommand{\Rrp}{\Rr_+}
\newcommand{\Rrm}{\Rr_-}
\newcommand{\Rrep}{\Rr_+^*}
\newcommand{\Rrem}{\Rr_-^*}
\newcommand{\Cce}{\Cc^*}
%%%%%%%%%%%%%%  INTERVALLES %%%%%%%%%%%%%%%%%
\newcommand{\intff}[2]{\left[#1\;,\; #2\right]  }
\newcommand{\intof}[2]{\left]#1 \;, \;#2\right]  }
\newcommand{\intfo}[2]{\left[#1 \;,\; #2\right[  }
\newcommand{\intoo}[2]{\left]#1 \;,\; #2\right[  }

\providecommand{\newpar}{\\[\medskipamount]}

\newcommand{\annexessection}{%
  \newpage%
  \subsection*{Annexes}%
}

\providecommand{\lesson}[3]{%
  \title{#3}%
  \hypersetup{pdftitle={#2 : #3}}%
  \setcounter{section}{\numexpr #2 - 1}%
  \section{#3}%
  \fancyhead[R]{\truncate{0.73\textwidth}{#2 : #3}}%
}

\providecommand{\development}[3]{%
  \title{#3}%
  \hypersetup{pdftitle={#3}}%
  \section*{#3}%
  \fancyhead[R]{\truncate{0.73\textwidth}{#3}}%
}

\providecommand{\sheet}[3]{\development{#1}{#2}{#3}}

\providecommand{\ranking}[1]{%
  \title{Terminale #1}%
  \hypersetup{pdftitle={Terminale #1}}%
  \section*{Terminale #1}%
  \fancyhead[R]{\truncate{0.73\textwidth}{Terminale #1}}%
}

\providecommand{\summary}[1]{%
  \textit{#1}%
  \par%
  \medskip%
}

\tikzset{notestyleraw/.append style={inner sep=0pt, rounded corners=0pt, align=center}}

%\newcommand{\booklink}[1]{\website/bibliographie\##1}
\newcounter{reference}
\newcommand{\previousreference}{}
\providecommand{\reference}[2][]{%
  \needspace{20pt}%
  \notblank{#1}{
    \needspace{20pt}%
    \renewcommand{\previousreference}{#1}%
    \stepcounter{reference}%
    \label{reference-\previousreference-\thereference}%
  }{}%
  \todo[noline]{%
    \protect\vspace{20pt}%
    \protect\par%
    \protect\notblank{#1}{\cite{[\previousreference]}\\}{}%
    \protect\hyperref[reference-\previousreference-\thereference]{p. #2}%
  }%
}

\definecolor{devcolor}{HTML}{00695c}
\providecommand{\dev}[1]{%
  \reversemarginpar%
  \todo[noline]{
    \protect\vspace{20pt}%
    \protect\par%
    \bfseries\color{devcolor}\href{\website/developpements/#1}{[DEV]}
  }%
  \normalmarginpar%
}

% En-têtes :

\pagestyle{fancy}
\fancyhead[L]{\truncate{0.23\textwidth}{\thepage}}
\fancyfoot[C]{\scriptsize \href{\website}{\texttt{https://github.com/imbodj/SenCoursDeMaths}}}

% Couleurs :

\definecolor{property}{HTML}{ffeb3b}
\definecolor{proposition}{HTML}{ffc107}
\definecolor{lemma}{HTML}{ff9800}
\definecolor{theorem}{HTML}{f44336}
\definecolor{corollary}{HTML}{e91e63}
\definecolor{definition}{HTML}{673ab7}
\definecolor{notation}{HTML}{9c27b0}
\definecolor{example}{HTML}{00bcd4}
\definecolor{cexample}{HTML}{795548}
\definecolor{application}{HTML}{009688}
\definecolor{remark}{HTML}{3f51b5}
\definecolor{algorithm}{HTML}{607d8b}
%\definecolor{proof}{HTML}{e1f5fe}
\definecolor{exercice}{HTML}{e1f5fe}

% Théorèmes :

\theoremstyle{definition}
\newtheorem{theorem}{Théorème}

\newtheorem{property}[theorem]{Propriété}
\newtheorem{proposition}[theorem]{Proposition}
\newtheorem{lemma}[theorem]{Activité d'introduction}
\newtheorem{corollary}[theorem]{Conséquence}

\newtheorem{definition}[theorem]{Définition}
\newtheorem{notation}[theorem]{Notation}

\newtheorem{example}[theorem]{Exemple}
\newtheorem{cexample}[theorem]{Contre-exemple}
\newtheorem{application}[theorem]{Application}

\newtheorem{algorithm}[theorem]{Algorithme}
\newtheorem{exercice}[theorem]{Exercice}

\theoremstyle{remark}
\newtheorem{remark}[theorem]{Remarque}

\counterwithin*{theorem}{section}

\newcommand{\applystyletotheorem}[1]{
  \tcolorboxenvironment{#1}{
    enhanced,
    breakable,
    colback=#1!8!white,
    %right=0pt,
    %top=8pt,
    %bottom=8pt,
    boxrule=0pt,
    frame hidden,
    sharp corners,
    enhanced,borderline west={4pt}{0pt}{#1},
    %interior hidden,
    sharp corners,
    after=\par,
  }
}

\applystyletotheorem{property}
\applystyletotheorem{proposition}
\applystyletotheorem{lemma}
\applystyletotheorem{theorem}
\applystyletotheorem{corollary}
\applystyletotheorem{definition}
\applystyletotheorem{notation}
\applystyletotheorem{example}
\applystyletotheorem{cexample}
\applystyletotheorem{application}
\applystyletotheorem{remark}
%\applystyletotheorem{proof}
\applystyletotheorem{algorithm}
\applystyletotheorem{exercice}

% Environnements :

\NewEnviron{whitetabularx}[1]{%
  \renewcommand{\arraystretch}{2.5}
  \colorbox{white}{%
    \begin{tabularx}{\textwidth}{#1}%
      \BODY%
    \end{tabularx}%
  }%
}

% Maths :

\DeclareFontEncoding{FMS}{}{}
\DeclareFontSubstitution{FMS}{futm}{m}{n}
\DeclareFontEncoding{FMX}{}{}
\DeclareFontSubstitution{FMX}{futm}{m}{n}
\DeclareSymbolFont{fouriersymbols}{FMS}{futm}{m}{n}
\DeclareSymbolFont{fourierlargesymbols}{FMX}{futm}{m}{n}
\DeclareMathDelimiter{\VERT}{\mathord}{fouriersymbols}{152}{fourierlargesymbols}{147}

% Code :

\definecolor{greencode}{rgb}{0,0.6,0}
\definecolor{graycode}{rgb}{0.5,0.5,0.5}
\definecolor{mauvecode}{rgb}{0.58,0,0.82}
\definecolor{bluecode}{HTML}{1976d2}
\lstset{
  basicstyle=\footnotesize\ttfamily,
  breakatwhitespace=false,
  breaklines=true,
  %captionpos=b,
  commentstyle=\color{greencode},
  deletekeywords={...},
  escapeinside={\%*}{*)},
  extendedchars=true,
  frame=none,
  keepspaces=true,
  keywordstyle=\color{bluecode},
  language=Python,
  otherkeywords={*,...},
  numbers=left,
  numbersep=5pt,
  numberstyle=\tiny\color{graycode},
  rulecolor=\color{black},
  showspaces=false,
  showstringspaces=false,
  showtabs=false,
  stepnumber=2,
  stringstyle=\color{mauvecode},
  tabsize=2,
  %texcl=true,
  xleftmargin=10pt,
  %title=\lstname
}

\newcommand{\codedirectory}{}
\newcommand{\inputalgorithm}[1]{%
  \begin{algorithm}%
    \strut%
    \lstinputlisting{\codedirectory#1}%
  \end{algorithm}%
}



% Bibliographie :

%\addbibresource{\bibliographypath}%
\defbibheading{bibliography}[\bibname]{\section*{#1}}
\renewbibmacro*{entryhead:full}{\printfield{labeltitle}}%
\DeclareFieldFormat{url}{\newline\footnotesize\url{#1}}%

\AtEndDocument{%
  \newpage%
  \pagestyle{empty}%
  \printbibliography%
}


\begin{document}
  %<*content>
  \development{algebra}{theoreme-de-kronecker}{Théorème de Kronecker}

  \summary{En utilisant les polynômes symétriques, nous montrons ici que toutes les racines d'un polynôme unitaire à coefficients entiers dont les racines sont dans $D(0, 1) \setminus \{ 0 \}$, sont en fait des racines de l'unité.}

  \begin{lemma}[Relations de Viète]
    \label{theoreme-de-kronecker-1}
    Soient $A$ un anneau commutatif unitaire intègre et $P = \sum_{i=1}^n a_iX^i \in A[X]$ que l'on suppose scindé dans $A[X]$ et tel que $a_n \in A^*$. Si on note $\Sigma_k(X_1, \dots, X_n)$ le $k$-ième polynôme symétrique élémentaire en $n$ variables et $\alpha_1$, ..., $\alpha_n$ les racines de $P$ (comptées avec multiplicité), alors $\Sigma_k(\alpha_1, \dots, \alpha_n) = (-1)^k a_{n-k} a_n^{-1}$.
  \end{lemma}

  \begin{proof}
    On a $P = a_n \prod_{i=1}^n (X-\alpha_i)$. En développant partiellement $P$, on a de même :
    \[ P = a_n X^n - a_n (\alpha_1 + \dots + \alpha_n)X^{n-1} + \dots + (-1)^n a_n \alpha_1 \dots \alpha_n \]
    Par identification avec la forme développée, les coefficients de $X^{n-1}$ doivent être égaux. En particulier :
    \[ a_{n-1} = -a_n (\alpha_1 + \dots + \alpha_n) \iff \underbrace{\alpha_1 + \dots + \alpha_n}_{= \Sigma_1(\alpha_1, \dots \alpha_n)} = - a_{n-1} a_n^{-1} \]
    Et on procède de même pour trouver les autres coefficients. Par exemple, $a_0 = (-1)^n a_n \alpha_1 \dots \alpha_n \iff \Sigma_n(\alpha_1, \dots \alpha_n) = (-1)^n a_0 a_n^{-1}$.
  \end{proof}

  \begin{remark}
    Tout au long de ce développement, nous utiliserons implicitement le fait que tout polynôme à coefficients dans $\mathbb{C}$ (donc à fortiori aussi dans $\mathbb{Z}$) admet $n$ racines complexes comptées avec multiplicité. Il s'agit du théorème de d'Alembert-Gauss.
  \end{remark}

  \reference[I-P]{279}

  \begin{theorem}[Kronecker]
    \label{theoreme-de-kronecker-2}
    Soit $P \in \mathbb{Z}[X]$ unitaire tel que toutes ses racines complexes appartiennent au disque unité épointé en l'origine (que l'on note $D$). Alors toutes ses racines sont des racines de l'unité.
  \end{theorem}

  \begin{proof}
    Notons par $\Omega_n$ l'ensemble des polynômes unitaires à coefficients dans $\mathbb{Z}$ tels que toutes leurs racines complexes appartiennent à $D$. Soit $P \in \Omega_n$ dont on note $a_0, \dots, a_n$ les coefficients et $z_1, \dots, z_n$ les racines complexes. On note $\forall k \in \llbracket 0, n \rrbracket$, $\sigma_k = \Sigma_k(z_1, \dots, z_n)$. D'après le \cref{theoreme-de-kronecker-1}, on a :
    \[ \forall k \in \llbracket 0, n \rrbracket, \, \sigma_k = (-1)^k a_{n-k} \tag{$*$} \]
    D'où $\forall k \in \llbracket 0, n \rrbracket$ :
    \begin{align*}
      |\sigma_k| &= \left| \sum_{I \in \mathcal{P}_k(\llbracket 1, n \rrbracket)} \prod_{i \in I} z_i \right| \\
      &\leq \sum_{I \in \mathcal{P}_k(\llbracket 1, n \rrbracket)} \prod_{i \in I} |z_i| \\
      &\leq |\mathcal{P}_k(\llbracket 1, n \rrbracket)| \times 1 \\
      &= \binom{n}{k}
    \end{align*}
    Et par $(*)$,
    \[ \forall k \in \llbracket 0, n \rrbracket, \, |a_k| \leq \binom{n}{n-k} = \binom{n}{k} \]
    $\Omega_n$ est donc un ensemble fini (car on n'a qu'un nombre limité de choix possibles pour les coefficients $a_k$).
    \newpar
    On pose maintenant
    \[ \forall k \in \mathbb{N}, \, P_k = \prod_{j=0}^n (X-z_j^k) \]
    qui sont des polynômes unitaires de degré $n$ dont les racines $z_1^k, \dots, z_n^k$ appartiennent toutes à $D$. Soient $k \in \mathbb{N}$ et $r \in \llbracket 0, n \rrbracket$. D'après le \cref{theoreme-de-kronecker-1}, le coefficient de $X^{n-r}$ de $P_k$ est $(-1)^r \Sigma_r(z_1^k, \dots, z_n^k)$. Mais, $\Sigma_r(X_1^k, \dots, X_n^k) \in \mathbb{Z}[X]$, donc on peut y appliquer le théorème fondamental des polynômes symétriques :
    \[ \exists Q_{r,k} \in \mathbb{Z}[X] \text{ tel que } \Sigma_r(X_1^k, \dots, X_n^k) = Q_{r,k}(\Sigma_1(X_1, \dots, X_n), \dots, \Sigma_n(X_1, \dots, X_n)) \]
    Or, comme $P \in \mathbb{Z}[X]$, on a $\forall j \in \llbracket 0, n \rrbracket$, $\Sigma_j(z_1, \dots, z_n) \in \mathbb{Z}$ d'après le \cref{theoreme-de-kronecker-1}. En particulier, on a $\Sigma_r(X_1^k, \dots, X_n^k) \in \mathbb{Z}[X]$ car $Q_{r,k} \in \mathbb{Z}[X]$. On en déduit que $\forall k \in \mathbb{N}$, $P_k \in \Omega_n$.
    \newpar
    Comme $\Omega_n$ est fini, l'ensemble des racines de tous les $P_k$ ; qui est $\{ z \in \mathbb{C} \mid \exists k \in \mathbb{N}, \, P_k(z) = 0 \}$ est fini. Soit $j \in \llbracket 1, n \rrbracket$. L'ensemble $\{ z_j^k \mid k \in \mathbb{N} \}$ est inclus dans l'ensemble de ces racines, qui est fini ; il est donc lui-même fini :
    \[ \exists k \neq k' \text{ tel que } z_j^k = z_j^{k'} \]
    Quitte à échanger les deux, on peut supposer $k \geq k'$. Comme $z_j \neq 0$, on a $z_j^{k-k'} = 1$. Donc $z_j$ est une racine de l'unité ; ce que l'on voulait.
  \end{proof}

  \begin{corollary}
    Soit $P \in \mathbb{Z}[X]$ unitaire et irréductible sur $\mathbb{Q}$ tel que toutes ses racines complexes soient de module inférieur ou égal à $1$. Alors $P = X$ ou $P$ est un polynôme cyclotomique.
  \end{corollary}

  \begin{proof}
    Si $0$ est racine de $P$, alors $X \mid P$, donc $P = X$ par irréductibilité et unitarité. Sinon, $0$ n'est pas racine de $P$. On peut donc appliquer le \cref{theoreme-de-kronecker-2} à $P$ ; et donc les racines de $P$ sont des racines de l'unité. Ainsi, en notant $N$ le maximum des ordres des racines de $P$, on a :
    \[ P \mid (X^N - 1)^n \text{ où } n = \deg(P) \]
    Or, la décomposition en irréductibles de $X^N - 1$ est
    \[ X^N - 1 = \prod_{d \mid N} \Phi_d \]
    Puisque $\mathbb{Q}[X]$ est un anneau factoriel, $P$ est premier. Donc d'après le lemme de Gauss, comme $P \mid X^N - 1$ :
    \[ \exists d \mid N \text{ tel que } P = \Phi_d \]
  \end{proof}
  %</content>
\end{document}
