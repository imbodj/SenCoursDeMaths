\documentclass[12pt, a4paper]{report}

% LuaLaTeX :

\RequirePackage{iftex}
\RequireLuaTeX

% Packages :

\usepackage[french]{babel}
%\usepackage[utf8]{inputenc}
%\usepackage[T1]{fontenc}
\usepackage[pdfencoding=auto, pdfauthor={Hugo Delaunay}, pdfsubject={Mathématiques}, pdfcreator={agreg.skyost.eu}]{hyperref}
\usepackage{amsmath}
\usepackage{amsthm}
%\usepackage{amssymb}
\usepackage{stmaryrd}
\usepackage{tikz}
\usepackage{tkz-euclide}
\usepackage{fontspec}
\defaultfontfeatures[Erewhon]{FontFace = {bx}{n}{Erewhon-Bold.otf}}
\usepackage{fourier-otf}
\usepackage[nobottomtitles*]{titlesec}
\usepackage{fancyhdr}
\usepackage{listings}
\usepackage{catchfilebetweentags}
\usepackage[french, capitalise, noabbrev]{cleveref}
\usepackage[fit, breakall]{truncate}
\usepackage[top=2.5cm, right=2cm, bottom=2.5cm, left=2cm]{geometry}
\usepackage{enumitem}
\usepackage{tocloft}
\usepackage{microtype}
%\usepackage{mdframed}
%\usepackage{thmtools}
\usepackage{xcolor}
\usepackage{tabularx}
\usepackage{xltabular}
\usepackage{aligned-overset}
\usepackage[subpreambles=true]{standalone}
\usepackage{environ}
\usepackage[normalem]{ulem}
\usepackage{etoolbox}
\usepackage{setspace}
\usepackage[bibstyle=reading, citestyle=draft]{biblatex}
\usepackage{xpatch}
\usepackage[many, breakable]{tcolorbox}
\usepackage[backgroundcolor=white, bordercolor=white, textsize=scriptsize]{todonotes}
\usepackage{luacode}
\usepackage{float}
\usepackage{needspace}
\everymath{\displaystyle}

% Police :

\setmathfont{Erewhon Math}

% Tikz :

\usetikzlibrary{calc}
\usetikzlibrary{3d}

% Longueurs :

\setlength{\parindent}{0pt}
\setlength{\headheight}{15pt}
\setlength{\fboxsep}{0pt}
\titlespacing*{\chapter}{0pt}{-20pt}{10pt}
\setlength{\marginparwidth}{1.5cm}
\setstretch{1.1}

% Métadonnées :

\author{agreg.skyost.eu}
\date{\today}

% Titres :

\setcounter{secnumdepth}{3}

\renewcommand{\thechapter}{\Roman{chapter}}
\renewcommand{\thesubsection}{\Roman{subsection}}
\renewcommand{\thesubsubsection}{\arabic{subsubsection}}
\renewcommand{\theparagraph}{\alph{paragraph}}

\titleformat{\chapter}{\huge\bfseries}{\thechapter}{20pt}{\huge\bfseries}
\titleformat*{\section}{\LARGE\bfseries}
\titleformat{\subsection}{\Large\bfseries}{\thesubsection \, - \,}{0pt}{\Large\bfseries}
\titleformat{\subsubsection}{\large\bfseries}{\thesubsubsection. \,}{0pt}{\large\bfseries}
\titleformat{\paragraph}{\bfseries}{\theparagraph. \,}{0pt}{\bfseries}

\setcounter{secnumdepth}{4}

% Table des matières :

\renewcommand{\cftsecleader}{\cftdotfill{\cftdotsep}}
\addtolength{\cftsecnumwidth}{10pt}

% Redéfinition des commandes :

\renewcommand*\thesection{\arabic{section}}
\renewcommand{\ker}{\mathrm{Ker}}

% Nouvelles commandes :

\newcommand{\website}{https://github.com/imbodj/SenCoursDeMaths}

\newcommand{\tr}[1]{\mathstrut ^t #1}
\newcommand{\im}{\mathrm{Im}}
\newcommand{\rang}{\operatorname{rang}}
\newcommand{\trace}{\operatorname{trace}}
\newcommand{\id}{\operatorname{id}}
\newcommand{\stab}{\operatorname{Stab}}
\newcommand{\paren}[1]{\left(#1\right)}
\newcommand{\croch}[1]{\left[ #1 \right]}
\newcommand{\Grdcroch}[1]{\Bigl[ #1 \Bigr]}
\newcommand{\grdcroch}[1]{\bigl[ #1 \bigr]}
\newcommand{\abs}[1]{\left\lvert #1 \right\rvert}
\newcommand{\limi}[3]{\lim_{#1\to #2}#3}
\newcommand{\pinf}{+\infty}
\newcommand{\minf}{-\infty}
%%%%%%%%%%%%%% ENSEMBLES %%%%%%%%%%%%%%%%%
\newcommand{\ensemblenombre}[1]{\mathbb{#1}}
\newcommand{\Nn}{\ensemblenombre{N}}
\newcommand{\Zz}{\ensemblenombre{Z}}
\newcommand{\Qq}{\ensemblenombre{Q}}
\newcommand{\Qqp}{\Qq^+}
\newcommand{\Rr}{\ensemblenombre{R}}
\newcommand{\Cc}{\ensemblenombre{C}}
\newcommand{\Nne}{\Nn^*}
\newcommand{\Zze}{\Zz^*}
\newcommand{\Zzn}{\Zz^-}
\newcommand{\Qqe}{\Qq^*}
\newcommand{\Rre}{\Rr^*}
\newcommand{\Rrp}{\Rr_+}
\newcommand{\Rrm}{\Rr_-}
\newcommand{\Rrep}{\Rr_+^*}
\newcommand{\Rrem}{\Rr_-^*}
\newcommand{\Cce}{\Cc^*}
%%%%%%%%%%%%%%  INTERVALLES %%%%%%%%%%%%%%%%%
\newcommand{\intff}[2]{\left[#1\;,\; #2\right]  }
\newcommand{\intof}[2]{\left]#1 \;, \;#2\right]  }
\newcommand{\intfo}[2]{\left[#1 \;,\; #2\right[  }
\newcommand{\intoo}[2]{\left]#1 \;,\; #2\right[  }

\providecommand{\newpar}{\\[\medskipamount]}

\newcommand{\annexessection}{%
  \newpage%
  \subsection*{Annexes}%
}

\providecommand{\lesson}[3]{%
  \title{#3}%
  \hypersetup{pdftitle={#2 : #3}}%
  \setcounter{section}{\numexpr #2 - 1}%
  \section{#3}%
  \fancyhead[R]{\truncate{0.73\textwidth}{#2 : #3}}%
}

\providecommand{\development}[3]{%
  \title{#3}%
  \hypersetup{pdftitle={#3}}%
  \section*{#3}%
  \fancyhead[R]{\truncate{0.73\textwidth}{#3}}%
}

\providecommand{\sheet}[3]{\development{#1}{#2}{#3}}

\providecommand{\ranking}[1]{%
  \title{Terminale #1}%
  \hypersetup{pdftitle={Terminale #1}}%
  \section*{Terminale #1}%
  \fancyhead[R]{\truncate{0.73\textwidth}{Terminale #1}}%
}

\providecommand{\summary}[1]{%
  \textit{#1}%
  \par%
  \medskip%
}

\tikzset{notestyleraw/.append style={inner sep=0pt, rounded corners=0pt, align=center}}

%\newcommand{\booklink}[1]{\website/bibliographie\##1}
\newcounter{reference}
\newcommand{\previousreference}{}
\providecommand{\reference}[2][]{%
  \needspace{20pt}%
  \notblank{#1}{
    \needspace{20pt}%
    \renewcommand{\previousreference}{#1}%
    \stepcounter{reference}%
    \label{reference-\previousreference-\thereference}%
  }{}%
  \todo[noline]{%
    \protect\vspace{20pt}%
    \protect\par%
    \protect\notblank{#1}{\cite{[\previousreference]}\\}{}%
    \protect\hyperref[reference-\previousreference-\thereference]{p. #2}%
  }%
}

\definecolor{devcolor}{HTML}{00695c}
\providecommand{\dev}[1]{%
  \reversemarginpar%
  \todo[noline]{
    \protect\vspace{20pt}%
    \protect\par%
    \bfseries\color{devcolor}\href{\website/developpements/#1}{[DEV]}
  }%
  \normalmarginpar%
}

% En-têtes :

\pagestyle{fancy}
\fancyhead[L]{\truncate{0.23\textwidth}{\thepage}}
\fancyfoot[C]{\scriptsize \href{\website}{\texttt{https://github.com/imbodj/SenCoursDeMaths}}}

% Couleurs :

\definecolor{property}{HTML}{ffeb3b}
\definecolor{proposition}{HTML}{ffc107}
\definecolor{lemma}{HTML}{ff9800}
\definecolor{theorem}{HTML}{f44336}
\definecolor{corollary}{HTML}{e91e63}
\definecolor{definition}{HTML}{673ab7}
\definecolor{notation}{HTML}{9c27b0}
\definecolor{example}{HTML}{00bcd4}
\definecolor{cexample}{HTML}{795548}
\definecolor{application}{HTML}{009688}
\definecolor{remark}{HTML}{3f51b5}
\definecolor{algorithm}{HTML}{607d8b}
%\definecolor{proof}{HTML}{e1f5fe}
\definecolor{exercice}{HTML}{e1f5fe}

% Théorèmes :

\theoremstyle{definition}
\newtheorem{theorem}{Théorème}

\newtheorem{property}[theorem]{Propriété}
\newtheorem{proposition}[theorem]{Proposition}
\newtheorem{lemma}[theorem]{Activité d'introduction}
\newtheorem{corollary}[theorem]{Conséquence}

\newtheorem{definition}[theorem]{Définition}
\newtheorem{notation}[theorem]{Notation}

\newtheorem{example}[theorem]{Exemple}
\newtheorem{cexample}[theorem]{Contre-exemple}
\newtheorem{application}[theorem]{Application}

\newtheorem{algorithm}[theorem]{Algorithme}
\newtheorem{exercice}[theorem]{Exercice}

\theoremstyle{remark}
\newtheorem{remark}[theorem]{Remarque}

\counterwithin*{theorem}{section}

\newcommand{\applystyletotheorem}[1]{
  \tcolorboxenvironment{#1}{
    enhanced,
    breakable,
    colback=#1!8!white,
    %right=0pt,
    %top=8pt,
    %bottom=8pt,
    boxrule=0pt,
    frame hidden,
    sharp corners,
    enhanced,borderline west={4pt}{0pt}{#1},
    %interior hidden,
    sharp corners,
    after=\par,
  }
}

\applystyletotheorem{property}
\applystyletotheorem{proposition}
\applystyletotheorem{lemma}
\applystyletotheorem{theorem}
\applystyletotheorem{corollary}
\applystyletotheorem{definition}
\applystyletotheorem{notation}
\applystyletotheorem{example}
\applystyletotheorem{cexample}
\applystyletotheorem{application}
\applystyletotheorem{remark}
%\applystyletotheorem{proof}
\applystyletotheorem{algorithm}
\applystyletotheorem{exercice}

% Environnements :

\NewEnviron{whitetabularx}[1]{%
  \renewcommand{\arraystretch}{2.5}
  \colorbox{white}{%
    \begin{tabularx}{\textwidth}{#1}%
      \BODY%
    \end{tabularx}%
  }%
}

% Maths :

\DeclareFontEncoding{FMS}{}{}
\DeclareFontSubstitution{FMS}{futm}{m}{n}
\DeclareFontEncoding{FMX}{}{}
\DeclareFontSubstitution{FMX}{futm}{m}{n}
\DeclareSymbolFont{fouriersymbols}{FMS}{futm}{m}{n}
\DeclareSymbolFont{fourierlargesymbols}{FMX}{futm}{m}{n}
\DeclareMathDelimiter{\VERT}{\mathord}{fouriersymbols}{152}{fourierlargesymbols}{147}

% Code :

\definecolor{greencode}{rgb}{0,0.6,0}
\definecolor{graycode}{rgb}{0.5,0.5,0.5}
\definecolor{mauvecode}{rgb}{0.58,0,0.82}
\definecolor{bluecode}{HTML}{1976d2}
\lstset{
  basicstyle=\footnotesize\ttfamily,
  breakatwhitespace=false,
  breaklines=true,
  %captionpos=b,
  commentstyle=\color{greencode},
  deletekeywords={...},
  escapeinside={\%*}{*)},
  extendedchars=true,
  frame=none,
  keepspaces=true,
  keywordstyle=\color{bluecode},
  language=Python,
  otherkeywords={*,...},
  numbers=left,
  numbersep=5pt,
  numberstyle=\tiny\color{graycode},
  rulecolor=\color{black},
  showspaces=false,
  showstringspaces=false,
  showtabs=false,
  stepnumber=2,
  stringstyle=\color{mauvecode},
  tabsize=2,
  %texcl=true,
  xleftmargin=10pt,
  %title=\lstname
}

\newcommand{\codedirectory}{}
\newcommand{\inputalgorithm}[1]{%
  \begin{algorithm}%
    \strut%
    \lstinputlisting{\codedirectory#1}%
  \end{algorithm}%
}



\everymath{\displaystyle}
\begin{document}
  %<*content>
  \development{algebra}{complexes-b}{Nombres complexes  2}

  \summary{Initiation sur les nombres complexes}

\begin{exercice}
\begin{enumerate}
\item Rappeler la forme trigonométrique d'un nombre complexe $ z $.
\item Mettre  sous la  forme trigonométrique les nombres complexes $ z $ suivants.

 $\textbf{a)} \;\;z=2\sqrt{3}-6\i
\hspace*{0.5cm}\textbf{b)} \;\;  z=-\dfrac{3}{2}+\i\dfrac{\sqrt{3}}{2} \hspace*{0.5cm}\textbf{c)} \;\; z=\paren{2+2\i}\paren{-\sqrt{3}+\i}^2
\hspace*{0.5cm}\textbf{d)} \;\; z=2\i\eexp{\i\frac{\pi}{6}} $
\medskip

 $\textbf{e)} \;\;z=\paren{-3+3\i}\eexp{\i\frac{\pi}{3}}
\hspace*{0.5cm}\textbf{f)} \;\;  z=1+\cos 2\theta +\i\sin 2\theta \hspace*{0.5cm}\textbf{g)} \;\; z=\sin \dfrac{\pi}{5} +\i\cos \dfrac{\pi}{5}$

$\textbf{h)} \;\; z=\dfrac{1+\sqrt{2}+\i}{1-\sqrt{2}+\i} $

\end{enumerate}
\end{exercice}

\begin{exercice}
Mettre sous forme algébrique les nombres complexes suivants.
\medskip

 $  z_1=\paren{1+\i}^{17}
\hspace*{0.5cm}  z_2=\paren{-\sqrt{3}+\i}^{2021}  \hspace*{0.5cm}  z_3=\dfrac{\paren{1+\i}^{3}}{\paren{\sqrt{3}+\i}^{4}} \hspace*{0.5cm} z_4=\eexp{\i\frac{\pi}{3}}+\eexp{-\i\frac{\pi}{6}} \hspace*{0.5cm} z_5=\dfrac{-\i\paren{\sqrt{3}-\i}^{2}}{2\paren{1-\i\sqrt{3}}^{7}}$

\end{exercice}


\begin{exercice}
 Le plan complexe  est rapporté  d'un repère orthonormé direct  $ (O ;\overrightarrow{u},\overrightarrow{v}) $. 
On considère les points  $ A$  , $B $ et $  C $  d'affixes respectives $ 1 +\i$  , $ \; 3+2\i  \; $  \; et  \; $ \; 3\i  $.
\begin{enumerate}
\item Donner une mesure de chacun des angles  orientés suivants :
  $ (\overrightarrow{u} , \overrightarrow{OA} )$ , $ (\overrightarrow{u} , \overrightarrow{OC} )$,  $ (\overrightarrow{v} , \overrightarrow{OA} )$ ,$ (\overrightarrow{v} , \overrightarrow{OC} )$, $ (\overrightarrow{CA} , \overrightarrow{CB} )$  et $ (\overrightarrow{AB}, \overrightarrow{AC} )$.
\item  Soit  $ Z = \dfrac{z_{C}-z_{A}}{z_{B}-z_{A}} $.
\begin{enumerate}
 \item Calculer  $  |Z| $ et un argument $ Z $. 
 \item Interpréter géométriquement $  |Z| $ et un argument $ Z $.
  En déduire la nature du triangle $ ABC $ . 
\end{enumerate} 
\end{enumerate} 
 \end{exercice}
\begin{exercice}
Soit  $\; z_1=\sqrt{2}+\i\sqrt{6} $, $\quad z_2=2-2\i\quad $  et   $\quad Z=\dfrac{z_1}{z_2} $.
    \begin{enumerate} 
   \item Ecrire $ Z $ sous forme algébrique.
  \item Ecrire  $z_1 $ et $ z_2 $  sous forme trigonométrique.
   \item En déduire  $ Z $ sous forme trigonométrique.
   \item Déterminer les valeurs de $\cos \dfrac{\pi}{12} $ et $\sin \dfrac{\pi}{12} $.
 \end{enumerate}
 
\end{exercice}

\begin{exercice}
 Soit $ \omega =\sqrt{3}+1 +\i\paren{\sqrt{3}-1}$
\begin{enumerate}
\item Ecrire $ \omega^{2} $ sous forme algébrique.
\item Déterminer le module et un argument de $ \omega ^{2} $. En déduire  le module et un argument de $ \omega  $.

\end{enumerate}
 \end{exercice}
\begin{exercice}
 Identifier la réponse juste et donner la justification.
\begin{enumerate}
\item  Si $\; \dfrac{\pi}{6}\; $  est un argument de  $\; \dfrac{9}{z}$ \; alors un argument de $\; \dfrac{\i}{z^{2}} $  est:\\
  $\textbf{a)\;} \dfrac{\pi}{6} \hspace*{0.5cm} \textbf{b)\;\;}   -\dfrac{5\pi}{6} \hspace*{0.5cm}   \textbf{c)\;}   \dfrac{5\pi}{6} $
\item Soit $ z $ un nombre complexe non nul d’argument $ \theta $. Un argument de $ \dfrac{-1+\i\sqrt{3}}{\overline{z}} $  est:\\
  $ \textbf{a)\;} -\dfrac{\pi}{3}+\theta \hspace*{0.5cm} \textbf{b)\;\;}   \dfrac{2\pi}{3}+\theta \hspace*{0.5cm}   \textbf{c)\;}   \dfrac{2\pi}{3}-\theta  $
\item Un argument de   $ \;\sin(x)+\i\cos(x)\; $ est:\\  $\textbf{a)\;} -x\hspace*{0.5cm} \textbf{b)\;\;}   x  \hspace*{0.5cm}   \textbf{c)\;}   \dfrac{\pi}{2}-x    v \hspace*{0.5cm} \textbf{d)\;}   \dfrac{\pi}{2}+x  $
\item Le nombre complexe $\;  (\sqrt{3}+\i)^{1689} $  \\ \textbf{a/\;}  est un réel $ \hspace*{0.5cm} $ \textbf{b/\;}  est un imaginaire pur \textbf{c/\;} n'est ni  réel ni imaginaire pur. 
\item Le conjugué de $ \eexp{\i\theta}$ est :


$\textbf{a)\;}   -\eexp{\i\theta}  \hspace*{1cm }\textbf{b)\;}   \eexp{-\i\theta}    \hspace*{1cm }  \textbf{c)\;}   \eexp{\i\theta} $
\end{enumerate}

\end{exercice}

\begin{exercice}
On consid\`ere les trois nombres complexes suivants : $z_{1} =(1 - i)(1+2i)$, $\;z_{2}=\dfrac{2 +6i}{3-i}\;$  et $ \;z_{3}=\dfrac{4i}{i-1} $.
\medskip

 Soit $ M_{1} $ , $ M_{2} $ et  $ M_{3} $  leurs images respectives dans le plan.
 \begin{enumerate}
 \item  Donner leurs formes ag\'ebriques. 
 \item  Placer  $ M_{1} $ , $ M_{2} $  et  $ M_{3} $ dans le plan complexe.
 \item Calculer  $\dfrac{z_{3} -z_{1}}{z_{2} -z_{1}}$ .  En d\'eduire que le triangle   $ M_{1} $$ M_{2} $$ M_{3}$  est rectangle isoc\`ele  . 
 \item   D\'eterminer l'affixe du point $ M_{4} $ telle  que le quadrilat\`ere  $ M_{1}M_{2}  M_{4} M_{3}$  soit un carr\'e .
 \item Montrer que les points $ M_{1} $ ,$ M_{2} $ , $ M_{3}$   et  $ M_{4} $  appartiennent \`a  un m\^eme cercle dont on pr\'ecisera les  \'el\'ements .
 \end{enumerate}

\end{exercice}

\begin{exercice}
 $ x\in\mathbb{R} $. Soient les nombres complexes suivants:
\medskip

$ Z'= -2\paren{\cos \frac{\pi}{3}+\i\sin\frac{\pi}{3}} $ et $ Z= (1-x)\paren{\cos \frac{\pi}{3}+\i\sin\frac{\pi}{3}} $
\begin{enumerate}
\item Calculer le module et un argument de $ Z' $.
\item Calculer le module et un argument de $ Z $.

( On discutera selon les valeurs de $ x) $


Donner pour chaque cas la forme trigonométrique et la forme algébrique de  $ Z $.
\item Montrer que $ Z^{2004} $ est un nombre réel dont on précisera le signe.
\item Montrer que l'équation  $ |Z|=2 $ a deux solutions $ Z_1$ et $ Z_2$.

Ecrire  $ Z_1$ et $ Z_2$ forme algébrique.
\item Placer  les points A et B d'affixes respectives $ 2\text{e}^{\i\frac{\pi}{3}} $  et $ -2\text{e}^{\i\frac{\pi}{3}} $ dans le plan complexe muni d'un repère orthonormé $ \ouv. $

Vérifier que les points A, B et $ O $ sont alignés.
\end{enumerate}
\end{exercice}
  %</content>
\end{document}
