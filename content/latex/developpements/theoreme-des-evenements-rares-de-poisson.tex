\documentclass[12pt, a4paper]{report}

% LuaLaTeX :

\RequirePackage{iftex}
\RequireLuaTeX

% Packages :

\usepackage[french]{babel}
%\usepackage[utf8]{inputenc}
%\usepackage[T1]{fontenc}
\usepackage[pdfencoding=auto, pdfauthor={Hugo Delaunay}, pdfsubject={Mathématiques}, pdfcreator={agreg.skyost.eu}]{hyperref}
\usepackage{amsmath}
\usepackage{amsthm}
%\usepackage{amssymb}
\usepackage{stmaryrd}
\usepackage{tikz}
\usepackage{tkz-euclide}
\usepackage{fontspec}
\defaultfontfeatures[Erewhon]{FontFace = {bx}{n}{Erewhon-Bold.otf}}
\usepackage{fourier-otf}
\usepackage[nobottomtitles*]{titlesec}
\usepackage{fancyhdr}
\usepackage{listings}
\usepackage{catchfilebetweentags}
\usepackage[french, capitalise, noabbrev]{cleveref}
\usepackage[fit, breakall]{truncate}
\usepackage[top=2.5cm, right=2cm, bottom=2.5cm, left=2cm]{geometry}
\usepackage{enumitem}
\usepackage{tocloft}
\usepackage{microtype}
%\usepackage{mdframed}
%\usepackage{thmtools}
\usepackage{xcolor}
\usepackage{tabularx}
\usepackage{xltabular}
\usepackage{aligned-overset}
\usepackage[subpreambles=true]{standalone}
\usepackage{environ}
\usepackage[normalem]{ulem}
\usepackage{etoolbox}
\usepackage{setspace}
\usepackage[bibstyle=reading, citestyle=draft]{biblatex}
\usepackage{xpatch}
\usepackage[many, breakable]{tcolorbox}
\usepackage[backgroundcolor=white, bordercolor=white, textsize=scriptsize]{todonotes}
\usepackage{luacode}
\usepackage{float}
\usepackage{needspace}
\everymath{\displaystyle}

% Police :

\setmathfont{Erewhon Math}

% Tikz :

\usetikzlibrary{calc}
\usetikzlibrary{3d}

% Longueurs :

\setlength{\parindent}{0pt}
\setlength{\headheight}{15pt}
\setlength{\fboxsep}{0pt}
\titlespacing*{\chapter}{0pt}{-20pt}{10pt}
\setlength{\marginparwidth}{1.5cm}
\setstretch{1.1}

% Métadonnées :

\author{agreg.skyost.eu}
\date{\today}

% Titres :

\setcounter{secnumdepth}{3}

\renewcommand{\thechapter}{\Roman{chapter}}
\renewcommand{\thesubsection}{\Roman{subsection}}
\renewcommand{\thesubsubsection}{\arabic{subsubsection}}
\renewcommand{\theparagraph}{\alph{paragraph}}

\titleformat{\chapter}{\huge\bfseries}{\thechapter}{20pt}{\huge\bfseries}
\titleformat*{\section}{\LARGE\bfseries}
\titleformat{\subsection}{\Large\bfseries}{\thesubsection \, - \,}{0pt}{\Large\bfseries}
\titleformat{\subsubsection}{\large\bfseries}{\thesubsubsection. \,}{0pt}{\large\bfseries}
\titleformat{\paragraph}{\bfseries}{\theparagraph. \,}{0pt}{\bfseries}

\setcounter{secnumdepth}{4}

% Table des matières :

\renewcommand{\cftsecleader}{\cftdotfill{\cftdotsep}}
\addtolength{\cftsecnumwidth}{10pt}

% Redéfinition des commandes :

\renewcommand*\thesection{\arabic{section}}
\renewcommand{\ker}{\mathrm{Ker}}

% Nouvelles commandes :

\newcommand{\website}{https://github.com/imbodj/SenCoursDeMaths}

\newcommand{\tr}[1]{\mathstrut ^t #1}
\newcommand{\im}{\mathrm{Im}}
\newcommand{\rang}{\operatorname{rang}}
\newcommand{\trace}{\operatorname{trace}}
\newcommand{\id}{\operatorname{id}}
\newcommand{\stab}{\operatorname{Stab}}
\newcommand{\paren}[1]{\left(#1\right)}
\newcommand{\croch}[1]{\left[ #1 \right]}
\newcommand{\Grdcroch}[1]{\Bigl[ #1 \Bigr]}
\newcommand{\grdcroch}[1]{\bigl[ #1 \bigr]}
\newcommand{\abs}[1]{\left\lvert #1 \right\rvert}
\newcommand{\limi}[3]{\lim_{#1\to #2}#3}
\newcommand{\pinf}{+\infty}
\newcommand{\minf}{-\infty}
%%%%%%%%%%%%%% ENSEMBLES %%%%%%%%%%%%%%%%%
\newcommand{\ensemblenombre}[1]{\mathbb{#1}}
\newcommand{\Nn}{\ensemblenombre{N}}
\newcommand{\Zz}{\ensemblenombre{Z}}
\newcommand{\Qq}{\ensemblenombre{Q}}
\newcommand{\Qqp}{\Qq^+}
\newcommand{\Rr}{\ensemblenombre{R}}
\newcommand{\Cc}{\ensemblenombre{C}}
\newcommand{\Nne}{\Nn^*}
\newcommand{\Zze}{\Zz^*}
\newcommand{\Zzn}{\Zz^-}
\newcommand{\Qqe}{\Qq^*}
\newcommand{\Rre}{\Rr^*}
\newcommand{\Rrp}{\Rr_+}
\newcommand{\Rrm}{\Rr_-}
\newcommand{\Rrep}{\Rr_+^*}
\newcommand{\Rrem}{\Rr_-^*}
\newcommand{\Cce}{\Cc^*}
%%%%%%%%%%%%%%  INTERVALLES %%%%%%%%%%%%%%%%%
\newcommand{\intff}[2]{\left[#1\;,\; #2\right]  }
\newcommand{\intof}[2]{\left]#1 \;, \;#2\right]  }
\newcommand{\intfo}[2]{\left[#1 \;,\; #2\right[  }
\newcommand{\intoo}[2]{\left]#1 \;,\; #2\right[  }

\providecommand{\newpar}{\\[\medskipamount]}

\newcommand{\annexessection}{%
  \newpage%
  \subsection*{Annexes}%
}

\providecommand{\lesson}[3]{%
  \title{#3}%
  \hypersetup{pdftitle={#2 : #3}}%
  \setcounter{section}{\numexpr #2 - 1}%
  \section{#3}%
  \fancyhead[R]{\truncate{0.73\textwidth}{#2 : #3}}%
}

\providecommand{\development}[3]{%
  \title{#3}%
  \hypersetup{pdftitle={#3}}%
  \section*{#3}%
  \fancyhead[R]{\truncate{0.73\textwidth}{#3}}%
}

\providecommand{\sheet}[3]{\development{#1}{#2}{#3}}

\providecommand{\ranking}[1]{%
  \title{Terminale #1}%
  \hypersetup{pdftitle={Terminale #1}}%
  \section*{Terminale #1}%
  \fancyhead[R]{\truncate{0.73\textwidth}{Terminale #1}}%
}

\providecommand{\summary}[1]{%
  \textit{#1}%
  \par%
  \medskip%
}

\tikzset{notestyleraw/.append style={inner sep=0pt, rounded corners=0pt, align=center}}

%\newcommand{\booklink}[1]{\website/bibliographie\##1}
\newcounter{reference}
\newcommand{\previousreference}{}
\providecommand{\reference}[2][]{%
  \needspace{20pt}%
  \notblank{#1}{
    \needspace{20pt}%
    \renewcommand{\previousreference}{#1}%
    \stepcounter{reference}%
    \label{reference-\previousreference-\thereference}%
  }{}%
  \todo[noline]{%
    \protect\vspace{20pt}%
    \protect\par%
    \protect\notblank{#1}{\cite{[\previousreference]}\\}{}%
    \protect\hyperref[reference-\previousreference-\thereference]{p. #2}%
  }%
}

\definecolor{devcolor}{HTML}{00695c}
\providecommand{\dev}[1]{%
  \reversemarginpar%
  \todo[noline]{
    \protect\vspace{20pt}%
    \protect\par%
    \bfseries\color{devcolor}\href{\website/developpements/#1}{[DEV]}
  }%
  \normalmarginpar%
}

% En-têtes :

\pagestyle{fancy}
\fancyhead[L]{\truncate{0.23\textwidth}{\thepage}}
\fancyfoot[C]{\scriptsize \href{\website}{\texttt{https://github.com/imbodj/SenCoursDeMaths}}}

% Couleurs :

\definecolor{property}{HTML}{ffeb3b}
\definecolor{proposition}{HTML}{ffc107}
\definecolor{lemma}{HTML}{ff9800}
\definecolor{theorem}{HTML}{f44336}
\definecolor{corollary}{HTML}{e91e63}
\definecolor{definition}{HTML}{673ab7}
\definecolor{notation}{HTML}{9c27b0}
\definecolor{example}{HTML}{00bcd4}
\definecolor{cexample}{HTML}{795548}
\definecolor{application}{HTML}{009688}
\definecolor{remark}{HTML}{3f51b5}
\definecolor{algorithm}{HTML}{607d8b}
%\definecolor{proof}{HTML}{e1f5fe}
\definecolor{exercice}{HTML}{e1f5fe}

% Théorèmes :

\theoremstyle{definition}
\newtheorem{theorem}{Théorème}

\newtheorem{property}[theorem]{Propriété}
\newtheorem{proposition}[theorem]{Proposition}
\newtheorem{lemma}[theorem]{Activité d'introduction}
\newtheorem{corollary}[theorem]{Conséquence}

\newtheorem{definition}[theorem]{Définition}
\newtheorem{notation}[theorem]{Notation}

\newtheorem{example}[theorem]{Exemple}
\newtheorem{cexample}[theorem]{Contre-exemple}
\newtheorem{application}[theorem]{Application}

\newtheorem{algorithm}[theorem]{Algorithme}
\newtheorem{exercice}[theorem]{Exercice}

\theoremstyle{remark}
\newtheorem{remark}[theorem]{Remarque}

\counterwithin*{theorem}{section}

\newcommand{\applystyletotheorem}[1]{
  \tcolorboxenvironment{#1}{
    enhanced,
    breakable,
    colback=#1!8!white,
    %right=0pt,
    %top=8pt,
    %bottom=8pt,
    boxrule=0pt,
    frame hidden,
    sharp corners,
    enhanced,borderline west={4pt}{0pt}{#1},
    %interior hidden,
    sharp corners,
    after=\par,
  }
}

\applystyletotheorem{property}
\applystyletotheorem{proposition}
\applystyletotheorem{lemma}
\applystyletotheorem{theorem}
\applystyletotheorem{corollary}
\applystyletotheorem{definition}
\applystyletotheorem{notation}
\applystyletotheorem{example}
\applystyletotheorem{cexample}
\applystyletotheorem{application}
\applystyletotheorem{remark}
%\applystyletotheorem{proof}
\applystyletotheorem{algorithm}
\applystyletotheorem{exercice}

% Environnements :

\NewEnviron{whitetabularx}[1]{%
  \renewcommand{\arraystretch}{2.5}
  \colorbox{white}{%
    \begin{tabularx}{\textwidth}{#1}%
      \BODY%
    \end{tabularx}%
  }%
}

% Maths :

\DeclareFontEncoding{FMS}{}{}
\DeclareFontSubstitution{FMS}{futm}{m}{n}
\DeclareFontEncoding{FMX}{}{}
\DeclareFontSubstitution{FMX}{futm}{m}{n}
\DeclareSymbolFont{fouriersymbols}{FMS}{futm}{m}{n}
\DeclareSymbolFont{fourierlargesymbols}{FMX}{futm}{m}{n}
\DeclareMathDelimiter{\VERT}{\mathord}{fouriersymbols}{152}{fourierlargesymbols}{147}

% Code :

\definecolor{greencode}{rgb}{0,0.6,0}
\definecolor{graycode}{rgb}{0.5,0.5,0.5}
\definecolor{mauvecode}{rgb}{0.58,0,0.82}
\definecolor{bluecode}{HTML}{1976d2}
\lstset{
  basicstyle=\footnotesize\ttfamily,
  breakatwhitespace=false,
  breaklines=true,
  %captionpos=b,
  commentstyle=\color{greencode},
  deletekeywords={...},
  escapeinside={\%*}{*)},
  extendedchars=true,
  frame=none,
  keepspaces=true,
  keywordstyle=\color{bluecode},
  language=Python,
  otherkeywords={*,...},
  numbers=left,
  numbersep=5pt,
  numberstyle=\tiny\color{graycode},
  rulecolor=\color{black},
  showspaces=false,
  showstringspaces=false,
  showtabs=false,
  stepnumber=2,
  stringstyle=\color{mauvecode},
  tabsize=2,
  %texcl=true,
  xleftmargin=10pt,
  %title=\lstname
}

\newcommand{\codedirectory}{}
\newcommand{\inputalgorithm}[1]{%
  \begin{algorithm}%
    \strut%
    \lstinputlisting{\codedirectory#1}%
  \end{algorithm}%
}



% Bibliographie :

%\addbibresource{\bibliographypath}%
\defbibheading{bibliography}[\bibname]{\section*{#1}}
\renewbibmacro*{entryhead:full}{\printfield{labeltitle}}%
\DeclareFieldFormat{url}{\newline\footnotesize\url{#1}}%

\AtEndDocument{%
  \newpage%
  \pagestyle{empty}%
  \printbibliography%
}


\begin{document}
  %<*content>
  \development{analysis}{theoreme-des-evenements-rares-de-poisson}{Théorème des événements rares de Poisson}

  \summary{On établit la convergence en loi vers une loi de Poisson d'une suite de variables aléatoires.}

  \reference[G-K]{372}

  \begin{lemma}
    \label{theoreme-des-evenements-rares-de-poisson-1}
    Soient $z_1, \dots, z_n, u_1, \dots u_n \in \mathbb{C}$ de module inférieur ou égal à $1$. Alors
    \[ |z_1 \dots z_n - u_1 \dots u_n| \leq |z_1-u_1| + \dots |z_n - u_n| \]
  \end{lemma}

  \begin{proof}
    $|z_1 z_2 - u_1 u_2| = |z_1(z_2 - u_2) + u_2(z_1 - u_1)| \leq |z_1 - u_1| + |z_2 - u_2|$. On procède ensuite par récurrence pour montrer le résultat.
  \end{proof}

  \reference{390}

  \begin{theorem}[des événements rares de Poisson]
    Soit $(N_n)_{n \geq 1}$ une suite d'entiers tendant vers l'infini. On suppose que pour tout $n$, $A_{n,N_1}, \dots , A_{n,N_n}$ sont des événements indépendants avec $\mathbb{P}(A_{n,N_k}) = p_{n,k}$. On suppose également que :
    \begin{enumerate}[label=(\roman*)]
      \item $\lim_{n \rightarrow +\infty} s_n = \lambda > 0$ où $\forall n \in \mathbb{N}, s_n = \sum_{k=1}^{N_n} p_{n,k}$.
      \item $\lim_{n \rightarrow +\infty} \sup_{k \in \llbracket 1, N_n \rrbracket} p_{n,k} = 0$.
    \end{enumerate}
    Alors, la suite de variables aléatoires $(S_n)$ définie par
    \[ \forall n \in \mathbb{N}^*, \, S_n = \sum_{k=1}^{N_n} \mathbb{1}_{A_{n,k}} \]
    converge en loi vers la loi de Poisson de paramètre $\lambda$.
  \end{theorem}

  \begin{proof}
    Pour la suite, on note $\forall n \in \mathbb{N}, \, m_n = \max_{k \in \llbracket 1, N_n \rrbracket} p_{n,k}$. On calcule
    \begin{align*}
      \phi_{S_n}(t) &= \mathbb{E} \left(e^{itS_n} \right) \\
      &= \mathbb{E} \left(e^{it \sum_{k=1}^{N_n} \mathbb{1}_{A_{n,k}}} \right) \\
      &= \mathbb{E} \left(\prod_{k=1}^{N_n} e^{it \mathbb{1}_{A_{n,k}}} \right) \\
      &= \prod_{k=1}^{N_n} \mathbb{E} \left(e^{it \mathbb{1}_{A_{n,k}}} \right) \text{ par indépendance} \\
      &= \prod_{k=1}^{N_n} \left( (1 - p_{n,k}) + e^{it} p_{n,k} \right) \\
      &= \prod_{k=1}^{N_n} \left( p_{n,k}(e^{it}-1) + 1 \right)
    \end{align*}
    l'avant-dernière égalité étant justifiée par le fait que
    \[ \mathbb{P}(e^{it \mathbb{1}_{A_{n,k}}} = e^{it}) = \mathbb{P}(A_{n,k} = 1) = p_{n,k} \text{ et } \mathbb{P}(e^{it \mathbb{1}_{A_{n,k}}} = 1) = \mathbb{P}(A_{n,k} = 0) = 1 - p_{n,k} \]
    Soient $P_{n,k}$ des variables aléatoires indépendantes suivant les lois de Poisson de paramètres respectifs $p_{n,k}$. On pose
    \[ S_n' = \sum_{k=1}^{N_n} P_{n,k} \]
    et on calcule la fonction caractéristique de cette nouvelle variable aléatoire :
    \begin{align*}
      \phi_{S_n'}(t) &= \prod_{k=1}^{N_n} \phi_{P_{n,k}} (t) \text{ par indépendance} \\
      &= \prod_{k=1}^{N_n} \exp ( p_{n,k}(e^{it} - 1 ))) \\
      &= \exp(s_n(e^{it} - 1))
    \end{align*}
    Par différence, on obtient
    \[ \vert \phi_{S_n}(t) - \phi_{S_n'}(t) \vert = \left| \prod_{k=1}^{N_n} \left( p_{n,k}(e^{it}-1) + 1 \right) - \prod_{k=1}^{N_n} \exp ( p_{n,k}(e^{it} - 1 ))) \right| \]
    ce qui, après application du \cref{theoreme-des-evenements-rares-de-poisson-1}, donne l'inégalité
    \[ \vert \phi_{S_n}(t) - \phi_{S_n'}(t) \vert \leq \sum_{k=1}^{N_n} g(p_{n,k}(e^{it}-1)) \]
    avec $g : z \mapsto \vert e^z - 1 - z \vert$. Mais, par développement en série entière :
    \begin{align*}
      g(z) &= \left\vert \sum_{k=2}^{+\infty} \frac{z^k}{k!} \right\vert \\
      &= \left\vert \sum_{k=0}^{+\infty} \frac{z^{k+2}}{(k+2)!} \right\vert  \\
      &= \left\vert z^2 \sum_{k=0}^{+\infty} \frac{z^k}{k!} \frac{1}{(k+1)(k+2)} \right\vert \\
      &\leq \vert z \vert^2 \sum_{k=0}^{+\infty} \frac{\vert z \vert^k}{k!} \left| \frac{1}{(k+1)(k+2)} \right| \\
      &\leq \vert z \vert^2 \frac{e^{\vert z \vert}}{2}
    \end{align*}
    Mais, comme $\vert p_{n,k}(e^{it} - 1) \vert \leq 2 p_{n,k} \leq 2$, on a :
    \begin{align*}
      \vert \phi_{S_n}(t) - \phi_{S_n'}(t) \vert &\leq \sum_{k=1}^{N_n} (2p_{n,k})^2 \frac{e^2}{2} \\
      &= 2e^2 \sum_{k=1}^{N_n} 2p_{n,k}^2 \\
      &\leq 2e^2 \underbrace{s_n}_{\longrightarrow \lambda} \underbrace{m_n}_{\longrightarrow 0} \\
      &\longrightarrow 0
    \end{align*}
    Enfin,
    \begin{align*}
      \vert \phi_{S_n}(t) - \exp(\lambda(e^{it} - 1)) \vert &\leq \vert \phi_{S_n}(t) - \phi_{S_n'}(t) \vert + \vert \phi_{S_n'}(t) - \exp(\lambda(e^{it} - 1)) \vert \\
      &\leq \underbrace{\vert \phi_{S_n}(t) - \phi_{S_n'}(t) \vert}_{\longrightarrow 0} + \vert \underbrace{\exp(s_n(e^{it} - 1)) - \exp(\lambda(e^{it} - 1)) \vert}_{\longrightarrow 0 \text{ car } s_n \longrightarrow \lambda}
      &\longrightarrow 0
    \end{align*}
    et le théorème de Lévy permet de conclure.
  \end{proof}
  %</content>
\end{document}
