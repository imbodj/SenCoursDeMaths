\documentclass[12pt, a4paper]{report}

% LuaLaTeX :

\RequirePackage{iftex}
\RequireLuaTeX

% Packages :

\usepackage[french]{babel}
%\usepackage[utf8]{inputenc}
%\usepackage[T1]{fontenc}
\usepackage[pdfencoding=auto, pdfauthor={Hugo Delaunay}, pdfsubject={Mathématiques}, pdfcreator={agreg.skyost.eu}]{hyperref}
\usepackage{amsmath}
\usepackage{amsthm}
%\usepackage{amssymb}
\usepackage{stmaryrd}
\usepackage{tikz}
\usepackage{tkz-euclide}
\usepackage{fontspec}
\defaultfontfeatures[Erewhon]{FontFace = {bx}{n}{Erewhon-Bold.otf}}
\usepackage{fourier-otf}
\usepackage[nobottomtitles*]{titlesec}
\usepackage{fancyhdr}
\usepackage{listings}
\usepackage{catchfilebetweentags}
\usepackage[french, capitalise, noabbrev]{cleveref}
\usepackage[fit, breakall]{truncate}
\usepackage[top=2.5cm, right=2cm, bottom=2.5cm, left=2cm]{geometry}
\usepackage{enumitem}
\usepackage{tocloft}
\usepackage{microtype}
%\usepackage{mdframed}
%\usepackage{thmtools}
\usepackage{xcolor}
\usepackage{tabularx}
\usepackage{xltabular}
\usepackage{aligned-overset}
\usepackage[subpreambles=true]{standalone}
\usepackage{environ}
\usepackage[normalem]{ulem}
\usepackage{etoolbox}
\usepackage{setspace}
\usepackage[bibstyle=reading, citestyle=draft]{biblatex}
\usepackage{xpatch}
\usepackage[many, breakable]{tcolorbox}
\usepackage[backgroundcolor=white, bordercolor=white, textsize=scriptsize]{todonotes}
\usepackage{luacode}
\usepackage{float}
\usepackage{needspace}
\everymath{\displaystyle}

% Police :

\setmathfont{Erewhon Math}

% Tikz :

\usetikzlibrary{calc}
\usetikzlibrary{3d}

% Longueurs :

\setlength{\parindent}{0pt}
\setlength{\headheight}{15pt}
\setlength{\fboxsep}{0pt}
\titlespacing*{\chapter}{0pt}{-20pt}{10pt}
\setlength{\marginparwidth}{1.5cm}
\setstretch{1.1}

% Métadonnées :

\author{agreg.skyost.eu}
\date{\today}

% Titres :

\setcounter{secnumdepth}{3}

\renewcommand{\thechapter}{\Roman{chapter}}
\renewcommand{\thesubsection}{\Roman{subsection}}
\renewcommand{\thesubsubsection}{\arabic{subsubsection}}
\renewcommand{\theparagraph}{\alph{paragraph}}

\titleformat{\chapter}{\huge\bfseries}{\thechapter}{20pt}{\huge\bfseries}
\titleformat*{\section}{\LARGE\bfseries}
\titleformat{\subsection}{\Large\bfseries}{\thesubsection \, - \,}{0pt}{\Large\bfseries}
\titleformat{\subsubsection}{\large\bfseries}{\thesubsubsection. \,}{0pt}{\large\bfseries}
\titleformat{\paragraph}{\bfseries}{\theparagraph. \,}{0pt}{\bfseries}

\setcounter{secnumdepth}{4}

% Table des matières :

\renewcommand{\cftsecleader}{\cftdotfill{\cftdotsep}}
\addtolength{\cftsecnumwidth}{10pt}

% Redéfinition des commandes :

\renewcommand*\thesection{\arabic{section}}
\renewcommand{\ker}{\mathrm{Ker}}

% Nouvelles commandes :

\newcommand{\website}{https://github.com/imbodj/SenCoursDeMaths}

\newcommand{\tr}[1]{\mathstrut ^t #1}
\newcommand{\im}{\mathrm{Im}}
\newcommand{\rang}{\operatorname{rang}}
\newcommand{\trace}{\operatorname{trace}}
\newcommand{\id}{\operatorname{id}}
\newcommand{\stab}{\operatorname{Stab}}
\newcommand{\paren}[1]{\left(#1\right)}
\newcommand{\croch}[1]{\left[ #1 \right]}
\newcommand{\Grdcroch}[1]{\Bigl[ #1 \Bigr]}
\newcommand{\grdcroch}[1]{\bigl[ #1 \bigr]}
\newcommand{\abs}[1]{\left\lvert #1 \right\rvert}
\newcommand{\limi}[3]{\lim_{#1\to #2}#3}
\newcommand{\pinf}{+\infty}
\newcommand{\minf}{-\infty}
%%%%%%%%%%%%%% ENSEMBLES %%%%%%%%%%%%%%%%%
\newcommand{\ensemblenombre}[1]{\mathbb{#1}}
\newcommand{\Nn}{\ensemblenombre{N}}
\newcommand{\Zz}{\ensemblenombre{Z}}
\newcommand{\Qq}{\ensemblenombre{Q}}
\newcommand{\Qqp}{\Qq^+}
\newcommand{\Rr}{\ensemblenombre{R}}
\newcommand{\Cc}{\ensemblenombre{C}}
\newcommand{\Nne}{\Nn^*}
\newcommand{\Zze}{\Zz^*}
\newcommand{\Zzn}{\Zz^-}
\newcommand{\Qqe}{\Qq^*}
\newcommand{\Rre}{\Rr^*}
\newcommand{\Rrp}{\Rr_+}
\newcommand{\Rrm}{\Rr_-}
\newcommand{\Rrep}{\Rr_+^*}
\newcommand{\Rrem}{\Rr_-^*}
\newcommand{\Cce}{\Cc^*}
%%%%%%%%%%%%%%  INTERVALLES %%%%%%%%%%%%%%%%%
\newcommand{\intff}[2]{\left[#1\;,\; #2\right]  }
\newcommand{\intof}[2]{\left]#1 \;, \;#2\right]  }
\newcommand{\intfo}[2]{\left[#1 \;,\; #2\right[  }
\newcommand{\intoo}[2]{\left]#1 \;,\; #2\right[  }

\providecommand{\newpar}{\\[\medskipamount]}

\newcommand{\annexessection}{%
  \newpage%
  \subsection*{Annexes}%
}

\providecommand{\lesson}[3]{%
  \title{#3}%
  \hypersetup{pdftitle={#2 : #3}}%
  \setcounter{section}{\numexpr #2 - 1}%
  \section{#3}%
  \fancyhead[R]{\truncate{0.73\textwidth}{#2 : #3}}%
}

\providecommand{\development}[3]{%
  \title{#3}%
  \hypersetup{pdftitle={#3}}%
  \section*{#3}%
  \fancyhead[R]{\truncate{0.73\textwidth}{#3}}%
}

\providecommand{\sheet}[3]{\development{#1}{#2}{#3}}

\providecommand{\ranking}[1]{%
  \title{Terminale #1}%
  \hypersetup{pdftitle={Terminale #1}}%
  \section*{Terminale #1}%
  \fancyhead[R]{\truncate{0.73\textwidth}{Terminale #1}}%
}

\providecommand{\summary}[1]{%
  \textit{#1}%
  \par%
  \medskip%
}

\tikzset{notestyleraw/.append style={inner sep=0pt, rounded corners=0pt, align=center}}

%\newcommand{\booklink}[1]{\website/bibliographie\##1}
\newcounter{reference}
\newcommand{\previousreference}{}
\providecommand{\reference}[2][]{%
  \needspace{20pt}%
  \notblank{#1}{
    \needspace{20pt}%
    \renewcommand{\previousreference}{#1}%
    \stepcounter{reference}%
    \label{reference-\previousreference-\thereference}%
  }{}%
  \todo[noline]{%
    \protect\vspace{20pt}%
    \protect\par%
    \protect\notblank{#1}{\cite{[\previousreference]}\\}{}%
    \protect\hyperref[reference-\previousreference-\thereference]{p. #2}%
  }%
}

\definecolor{devcolor}{HTML}{00695c}
\providecommand{\dev}[1]{%
  \reversemarginpar%
  \todo[noline]{
    \protect\vspace{20pt}%
    \protect\par%
    \bfseries\color{devcolor}\href{\website/developpements/#1}{[DEV]}
  }%
  \normalmarginpar%
}

% En-têtes :

\pagestyle{fancy}
\fancyhead[L]{\truncate{0.23\textwidth}{\thepage}}
\fancyfoot[C]{\scriptsize \href{\website}{\texttt{https://github.com/imbodj/SenCoursDeMaths}}}

% Couleurs :

\definecolor{property}{HTML}{ffeb3b}
\definecolor{proposition}{HTML}{ffc107}
\definecolor{lemma}{HTML}{ff9800}
\definecolor{theorem}{HTML}{f44336}
\definecolor{corollary}{HTML}{e91e63}
\definecolor{definition}{HTML}{673ab7}
\definecolor{notation}{HTML}{9c27b0}
\definecolor{example}{HTML}{00bcd4}
\definecolor{cexample}{HTML}{795548}
\definecolor{application}{HTML}{009688}
\definecolor{remark}{HTML}{3f51b5}
\definecolor{algorithm}{HTML}{607d8b}
%\definecolor{proof}{HTML}{e1f5fe}
\definecolor{exercice}{HTML}{e1f5fe}

% Théorèmes :

\theoremstyle{definition}
\newtheorem{theorem}{Théorème}

\newtheorem{property}[theorem]{Propriété}
\newtheorem{proposition}[theorem]{Proposition}
\newtheorem{lemma}[theorem]{Activité d'introduction}
\newtheorem{corollary}[theorem]{Conséquence}

\newtheorem{definition}[theorem]{Définition}
\newtheorem{notation}[theorem]{Notation}

\newtheorem{example}[theorem]{Exemple}
\newtheorem{cexample}[theorem]{Contre-exemple}
\newtheorem{application}[theorem]{Application}

\newtheorem{algorithm}[theorem]{Algorithme}
\newtheorem{exercice}[theorem]{Exercice}

\theoremstyle{remark}
\newtheorem{remark}[theorem]{Remarque}

\counterwithin*{theorem}{section}

\newcommand{\applystyletotheorem}[1]{
  \tcolorboxenvironment{#1}{
    enhanced,
    breakable,
    colback=#1!8!white,
    %right=0pt,
    %top=8pt,
    %bottom=8pt,
    boxrule=0pt,
    frame hidden,
    sharp corners,
    enhanced,borderline west={4pt}{0pt}{#1},
    %interior hidden,
    sharp corners,
    after=\par,
  }
}

\applystyletotheorem{property}
\applystyletotheorem{proposition}
\applystyletotheorem{lemma}
\applystyletotheorem{theorem}
\applystyletotheorem{corollary}
\applystyletotheorem{definition}
\applystyletotheorem{notation}
\applystyletotheorem{example}
\applystyletotheorem{cexample}
\applystyletotheorem{application}
\applystyletotheorem{remark}
%\applystyletotheorem{proof}
\applystyletotheorem{algorithm}
\applystyletotheorem{exercice}

% Environnements :

\NewEnviron{whitetabularx}[1]{%
  \renewcommand{\arraystretch}{2.5}
  \colorbox{white}{%
    \begin{tabularx}{\textwidth}{#1}%
      \BODY%
    \end{tabularx}%
  }%
}

% Maths :

\DeclareFontEncoding{FMS}{}{}
\DeclareFontSubstitution{FMS}{futm}{m}{n}
\DeclareFontEncoding{FMX}{}{}
\DeclareFontSubstitution{FMX}{futm}{m}{n}
\DeclareSymbolFont{fouriersymbols}{FMS}{futm}{m}{n}
\DeclareSymbolFont{fourierlargesymbols}{FMX}{futm}{m}{n}
\DeclareMathDelimiter{\VERT}{\mathord}{fouriersymbols}{152}{fourierlargesymbols}{147}

% Code :

\definecolor{greencode}{rgb}{0,0.6,0}
\definecolor{graycode}{rgb}{0.5,0.5,0.5}
\definecolor{mauvecode}{rgb}{0.58,0,0.82}
\definecolor{bluecode}{HTML}{1976d2}
\lstset{
  basicstyle=\footnotesize\ttfamily,
  breakatwhitespace=false,
  breaklines=true,
  %captionpos=b,
  commentstyle=\color{greencode},
  deletekeywords={...},
  escapeinside={\%*}{*)},
  extendedchars=true,
  frame=none,
  keepspaces=true,
  keywordstyle=\color{bluecode},
  language=Python,
  otherkeywords={*,...},
  numbers=left,
  numbersep=5pt,
  numberstyle=\tiny\color{graycode},
  rulecolor=\color{black},
  showspaces=false,
  showstringspaces=false,
  showtabs=false,
  stepnumber=2,
  stringstyle=\color{mauvecode},
  tabsize=2,
  %texcl=true,
  xleftmargin=10pt,
  %title=\lstname
}

\newcommand{\codedirectory}{}
\newcommand{\inputalgorithm}[1]{%
  \begin{algorithm}%
    \strut%
    \lstinputlisting{\codedirectory#1}%
  \end{algorithm}%
}



% Bibliographie :

%\addbibresource{\bibliographypath}%
\defbibheading{bibliography}[\bibname]{\section*{#1}}
\renewbibmacro*{entryhead:full}{\printfield{labeltitle}}%
\DeclareFieldFormat{url}{\newline\footnotesize\url{#1}}%

\AtEndDocument{%
  \newpage%
  \pagestyle{empty}%
  \printbibliography%
}


\begin{document}
  %<*content>
  \development{algebra, analysis}{suite-de-polygones}{Suite de polygones}

  \summary{Il s'agit ici d'étudier une suite de polygones à l'aide de déterminants classiques, et de montrer qu'elle converge vers l'isobarycentre du polygone de départ.}

  \reference[GOU21]{153}

  \begin{lemma}[Déterminant circulant]
    \label{suite-de-polygones-1}
    Soient $n \in \mathbb{N}^*$ et $a_1, \dots, a_n \in \mathbb{C}$. On pose $\omega = e^{\frac{2i\pi}{n}}$. Alors
    \[ \begin{vmatrix} a_0 & a_1 & \dots & a_{n-1} \\ a_{n-1} & a_0 & \dots & a_{n-2}\\ \vdots & \vdots & \ddots & \vdots \\ a_1 & a_2 & \dots & a_0 \end{vmatrix} = \prod_{j=0}^{n-1} P(\omega^j) \]
    où $P = \sum_{k=0}^{n-1} a_k X^k$.
  \end{lemma}

  \begin{proof}
    On définit
    \[ A = \begin{pmatrix} a_0 & a_1 & \dots & a_{n-1} \\ a_{n-1} & a_0 & \dots & a_{n-2}\\ \vdots & \vdots & \ddots & \vdots \\ a_1 & a_2 & \dots & a_0 \end{pmatrix} \in \mathcal{M}_n(\mathbb{C}) \text{ et } \Omega = (\omega^{(i-1)(j-1)})_{i, j \in \llbracket 1, n \rrbracket} \in \mathcal{M}_n(\mathbb{C}) \]
    Pour $i \geq 2$, la $i$-ième ligne de $A$ est
    \[ \begin{pmatrix} a_{n-i+1} & \dots & a_{n-1} & a_0 & \dots & a_{n-i-2} \end{pmatrix} \]
    Si on multiplie cette ligne par la $j$-ième colonne de $\Omega$, on obtient le coefficient
    \begin{align*}
      &a_{n-i+1} + a_{n-i+2} \omega^{j-1} + \dots + a_0 \omega^{(j-1)(i-1)} + a_1 \omega^{(j-1)i} + \dots + a_{n-i-2} \omega^{(j-1)(n-1)} \\
      =& \, \omega^{(j-1)(i-1)} (a_0 + a_1 \omega^{j-1} + \dots + a_{n-1} \omega^{(j-1)(n-1)}) \\
      =& \, \omega^{(j-1)(i-1)} P(\omega^{j-1})
    \end{align*}
    et c'est encore vrai pour $i = 1$ puisque $\omega^0 = 1$. Donc la $j$-ième colonne de $A \Omega$ est égale à la $j$-ième colonne de $\Omega$ multipliée par $P(\omega^{j-1})$. Ceci entraîne que
    \[ \det(A) \det(\Omega) = \det(A\Omega) = P(1) P(\omega) \dots P(\omega^{n-1}) \det(\Omega) \]
    et le déterminant $\det(\Omega)$ est non nul (en tant que déterminant de Vandermonde à paramètres deux-à-deux distincts). D'où :
    \[ \det(A) = P(1) P(\omega) \dots P(\omega^{n-1}) \]
  \end{proof}

  \reference[I-P]{389}

  \begin{theorem}[Suite de polygones]
    Soit $P_0$ un polygone dont les sommets sont $\{ z_{0,1}, \dots, z_{0,n} \}$. On définit la suite de polygones $(P_k)$ par récurrence en disant que, pour tout $k \in \mathbb{N}^*$, les sommets de $P_{k+1}$ sont les milieux des arêtes de $P_k$.
    \begin{center}
      \begin{tikzpicture}
        \coordinate (A) at (0:3);
        \coordinate (B) at (72:3);
        \coordinate (C) at (2*72:3);
        \coordinate (D) at (3*72:3);
        \coordinate (E) at (4*72:3);
        \coordinate (F) at (A);
        \foreach \i in {0,...,10} {
          \draw(A) node {$\bullet$};
          \draw(B) node {$\bullet$};
          \draw(C) node {$\bullet$};
          \draw(D) node {$\bullet$};
          \draw(E) node {$\bullet$};
          \draw[fill=cyan!60, fill opacity=0.2](A) -- (B) -- (C) -- (D) -- (E) -- (A);
          \coordinate (A) at ($(A)!0.5!(B)$);
          \coordinate (B) at ($(B)!0.5!(C)$);
          \coordinate (C) at ($(C)!0.5!(D)$);
          \coordinate (D) at ($(D)!0.5!(E)$);
          \coordinate (E) at ($(E)!0.5!(F)$);
          \coordinate (F) at (A);
        }
      \end{tikzpicture}
    \end{center}
    Alors la suite $(P_k)$ converge vers l'isobarycentre de $P_0$.
  \end{theorem}

  \begin{proof}
    On identifie $P_k$ au vecteur colonne $Z_k = \begin{pmatrix} z_{k,1} \\ \vdots \\ z_{k,n} \end{pmatrix} \in \mathbb{C}^n$. Il s'agit de montrer que la suite $(Z_k)$ converge vers $\begin{pmatrix} g \\ \vdots \\ g \end{pmatrix}$ où $g$ désigne l'isobarycentre de $P_0$.
    \newpar
    En utilisant la notation matricielle, la relation de récurrence s'écrit
    \[ \forall k \in \mathbb{N}, \, Z_{k+1} = \begin{pmatrix} \frac{z_{k,1} + z_{k,2}}{2} \\ \vdots \\ \frac{z_{k,n} + z_{k,1}}{2} \end{pmatrix} = AZ_k \text{ où } A = \begin{pmatrix} \frac{1}{2} & \frac{1}{2} & 0 & \dots & 0 \\ 0 & \frac{1}{2} & \frac{1}{2} & \ddots & \vdots \\ \vdots & \ddots & \ddots & \ddots & \vdots \\ 0 & \dots & 0 & \frac{1}{2} & \frac{1}{2} \\ \frac{1}{2} & 0 & \dots & 0 & \frac{1}{2} \end{pmatrix} \]
    Par une récurrence immédiate (c'est une suite géométrique), on a donc $\forall k \in \mathbb{N}$, $Z_k = A^k Z_0$. Il suffit donc de montrer que $(A^k)$ converge dans $\mathcal{M}_n(\mathbb{C})$ (muni d'une norme quelconque par équivalence des normes en dimension finie).
    \newpar
    Pour cela, étudions les valeurs propres de $A$ :
    \[ \chi_A = \det(A - X I_n) = \begin{vmatrix} a_0 & a_1 & \dots & a_{n-1} \\ a_{n-1} & a_0 & \dots & a_{n-2}\\ \vdots & \vdots & \ddots & \vdots \\ a_1 & a_2 & \dots & a_0 \end{vmatrix} \]
    avec $a_0 = \frac{1}{2} - X$, $a_1 = \frac{1}{2}$ et $\forall i > 2, \, a_i = 0$. On reconnaît le déterminant circulant du \cref{suite-de-polygones-1} et en posant $P(Y) = \sum_{k=0}^{n-1} a_k Y^k$ et $\omega = e^{\frac{2i\pi}{n}}$, la formule du déterminant circulant nous donne :
    \[ \chi_A = \prod_{j=1}^n P(\omega^j) = \prod_{j=1}^n \left( \sum_{k=0}^{n-1} a_k \omega^{kj} \right) = \prod_{j=1}^n \left( \frac{1}{2} - X + \frac{1}{2} \omega^j \right) = \prod_{j=1}^n (\lambda_j - X) \]
    où $\lambda_j = \frac{1+\omega^j}{2}$. Et comme $\lambda_i = \lambda_j \iff i = j$, le polynôme $\chi_A$ est scindé à racines simples. Donc $\exists Q \in \mathrm{GL}_n(\mathbb{C})$ telle que $A=QDQ^{-1}$ et $D = \operatorname{Diag}(\lambda_1, \dots, \lambda_n)$. Or pour $j \neq n$,
    \[ |\lambda_j| = \left| \frac{1 + \omega^j}{2} \right| = \left| e^{\frac{ij \pi}{n}} \frac{e^{\frac{ij \pi}{n}} + e^{-\frac{ij \pi}{n}}}{2} \right| = \left| \cos \left( \frac{\pi j}{n} \right) \right| < 1 \]
    Ainsi, $\lambda_j^k \longrightarrow 0$ si $j < n$, donc la suite $(A^k)$ converge dans $\mathcal{M}_n(\mathbb{C})$ vers la matrice $B = Q \operatorname{Diag}(0, \dots, 0, 1)Q^{-1}$ par continuité de l'application $M \mapsto QMQ^{-1}$.
    \newpar
    On pose donc $X = B Z_0$, de sorte que la suite $(Z_k)$ converge vers $X$. Par continuité de $M \mapsto AM$, la limite $X$ vérifie forcément $X = AX$ ie. $X$ est vecteur propre de $A$ associé à la valeur propre $1$. Or l'espace propre de $A$ associé à la valeur propre $1$ contient le vecteur $\begin{pmatrix} 1 \\ \vdots \\ 1 \end{pmatrix}$ et est de dimension $1$ (car $\chi_A$ possède $n$ racines distinctes), donc il est engendré par ce vecteur. Ainsi, il existe $a \in \mathbb{C}$ tel que $X = \begin{pmatrix} a \\ \vdots \\ a \end{pmatrix}$ ie. $(Z_k)$ converge vers le point d'affixe $a$.
    \newpar
    Enfin, on remarque que si $g$ est l'isobarycentre de $P_0$, il est aussi égal à celui de $P_k$ pour tout $k$ (que l'on note $g_k$) car pour tout $k \geq 1$ :
    \[ g_k = \frac{1}{n} \sum_{i=1}^n z_{k,i} = \frac{1}{n} \sum_{i=1}^n \frac{z_{k-1,i} + z_{k-1,i+1}}{2} = \frac{1}{n} \sum_{i=1}^n z_{k-1,i} = g_{k-1} \]
    (en considérant les indices $i$ modulo $n$). Or, la suite $(Z_k)$ converge vers $\begin{pmatrix} a \\ \vdots \\ a \end{pmatrix}$, et la fonction $\varphi$ qui à $n$ points du plan associe son isobarycentre est continue. Donc,
    \[ g_k = \varphi(Z_k) \longrightarrow \varphi(a, \dots, a) = a \]
    et comme pour tout $k$, $g_k = g$, on a bien $g = a$.
  \end{proof}
  %</content>
\end{document}
