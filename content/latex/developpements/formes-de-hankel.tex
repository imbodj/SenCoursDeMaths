\documentclass[12pt, a4paper]{report}

% LuaLaTeX :

\RequirePackage{iftex}
\RequireLuaTeX

% Packages :

\usepackage[french]{babel}
%\usepackage[utf8]{inputenc}
%\usepackage[T1]{fontenc}
\usepackage[pdfencoding=auto, pdfauthor={Hugo Delaunay}, pdfsubject={Mathématiques}, pdfcreator={agreg.skyost.eu}]{hyperref}
\usepackage{amsmath}
\usepackage{amsthm}
%\usepackage{amssymb}
\usepackage{stmaryrd}
\usepackage{tikz}
\usepackage{tkz-euclide}
\usepackage{fontspec}
\defaultfontfeatures[Erewhon]{FontFace = {bx}{n}{Erewhon-Bold.otf}}
\usepackage{fourier-otf}
\usepackage[nobottomtitles*]{titlesec}
\usepackage{fancyhdr}
\usepackage{listings}
\usepackage{catchfilebetweentags}
\usepackage[french, capitalise, noabbrev]{cleveref}
\usepackage[fit, breakall]{truncate}
\usepackage[top=2.5cm, right=2cm, bottom=2.5cm, left=2cm]{geometry}
\usepackage{enumitem}
\usepackage{tocloft}
\usepackage{microtype}
%\usepackage{mdframed}
%\usepackage{thmtools}
\usepackage{xcolor}
\usepackage{tabularx}
\usepackage{xltabular}
\usepackage{aligned-overset}
\usepackage[subpreambles=true]{standalone}
\usepackage{environ}
\usepackage[normalem]{ulem}
\usepackage{etoolbox}
\usepackage{setspace}
\usepackage[bibstyle=reading, citestyle=draft]{biblatex}
\usepackage{xpatch}
\usepackage[many, breakable]{tcolorbox}
\usepackage[backgroundcolor=white, bordercolor=white, textsize=scriptsize]{todonotes}
\usepackage{luacode}
\usepackage{float}
\usepackage{needspace}
\everymath{\displaystyle}

% Police :

\setmathfont{Erewhon Math}

% Tikz :

\usetikzlibrary{calc}
\usetikzlibrary{3d}

% Longueurs :

\setlength{\parindent}{0pt}
\setlength{\headheight}{15pt}
\setlength{\fboxsep}{0pt}
\titlespacing*{\chapter}{0pt}{-20pt}{10pt}
\setlength{\marginparwidth}{1.5cm}
\setstretch{1.1}

% Métadonnées :

\author{agreg.skyost.eu}
\date{\today}

% Titres :

\setcounter{secnumdepth}{3}

\renewcommand{\thechapter}{\Roman{chapter}}
\renewcommand{\thesubsection}{\Roman{subsection}}
\renewcommand{\thesubsubsection}{\arabic{subsubsection}}
\renewcommand{\theparagraph}{\alph{paragraph}}

\titleformat{\chapter}{\huge\bfseries}{\thechapter}{20pt}{\huge\bfseries}
\titleformat*{\section}{\LARGE\bfseries}
\titleformat{\subsection}{\Large\bfseries}{\thesubsection \, - \,}{0pt}{\Large\bfseries}
\titleformat{\subsubsection}{\large\bfseries}{\thesubsubsection. \,}{0pt}{\large\bfseries}
\titleformat{\paragraph}{\bfseries}{\theparagraph. \,}{0pt}{\bfseries}

\setcounter{secnumdepth}{4}

% Table des matières :

\renewcommand{\cftsecleader}{\cftdotfill{\cftdotsep}}
\addtolength{\cftsecnumwidth}{10pt}

% Redéfinition des commandes :

\renewcommand*\thesection{\arabic{section}}
\renewcommand{\ker}{\mathrm{Ker}}

% Nouvelles commandes :

\newcommand{\website}{https://github.com/imbodj/SenCoursDeMaths}

\newcommand{\tr}[1]{\mathstrut ^t #1}
\newcommand{\im}{\mathrm{Im}}
\newcommand{\rang}{\operatorname{rang}}
\newcommand{\trace}{\operatorname{trace}}
\newcommand{\id}{\operatorname{id}}
\newcommand{\stab}{\operatorname{Stab}}
\newcommand{\paren}[1]{\left(#1\right)}
\newcommand{\croch}[1]{\left[ #1 \right]}
\newcommand{\Grdcroch}[1]{\Bigl[ #1 \Bigr]}
\newcommand{\grdcroch}[1]{\bigl[ #1 \bigr]}
\newcommand{\abs}[1]{\left\lvert #1 \right\rvert}
\newcommand{\limi}[3]{\lim_{#1\to #2}#3}
\newcommand{\pinf}{+\infty}
\newcommand{\minf}{-\infty}
%%%%%%%%%%%%%% ENSEMBLES %%%%%%%%%%%%%%%%%
\newcommand{\ensemblenombre}[1]{\mathbb{#1}}
\newcommand{\Nn}{\ensemblenombre{N}}
\newcommand{\Zz}{\ensemblenombre{Z}}
\newcommand{\Qq}{\ensemblenombre{Q}}
\newcommand{\Qqp}{\Qq^+}
\newcommand{\Rr}{\ensemblenombre{R}}
\newcommand{\Cc}{\ensemblenombre{C}}
\newcommand{\Nne}{\Nn^*}
\newcommand{\Zze}{\Zz^*}
\newcommand{\Zzn}{\Zz^-}
\newcommand{\Qqe}{\Qq^*}
\newcommand{\Rre}{\Rr^*}
\newcommand{\Rrp}{\Rr_+}
\newcommand{\Rrm}{\Rr_-}
\newcommand{\Rrep}{\Rr_+^*}
\newcommand{\Rrem}{\Rr_-^*}
\newcommand{\Cce}{\Cc^*}
%%%%%%%%%%%%%%  INTERVALLES %%%%%%%%%%%%%%%%%
\newcommand{\intff}[2]{\left[#1\;,\; #2\right]  }
\newcommand{\intof}[2]{\left]#1 \;, \;#2\right]  }
\newcommand{\intfo}[2]{\left[#1 \;,\; #2\right[  }
\newcommand{\intoo}[2]{\left]#1 \;,\; #2\right[  }

\providecommand{\newpar}{\\[\medskipamount]}

\newcommand{\annexessection}{%
  \newpage%
  \subsection*{Annexes}%
}

\providecommand{\lesson}[3]{%
  \title{#3}%
  \hypersetup{pdftitle={#2 : #3}}%
  \setcounter{section}{\numexpr #2 - 1}%
  \section{#3}%
  \fancyhead[R]{\truncate{0.73\textwidth}{#2 : #3}}%
}

\providecommand{\development}[3]{%
  \title{#3}%
  \hypersetup{pdftitle={#3}}%
  \section*{#3}%
  \fancyhead[R]{\truncate{0.73\textwidth}{#3}}%
}

\providecommand{\sheet}[3]{\development{#1}{#2}{#3}}

\providecommand{\ranking}[1]{%
  \title{Terminale #1}%
  \hypersetup{pdftitle={Terminale #1}}%
  \section*{Terminale #1}%
  \fancyhead[R]{\truncate{0.73\textwidth}{Terminale #1}}%
}

\providecommand{\summary}[1]{%
  \textit{#1}%
  \par%
  \medskip%
}

\tikzset{notestyleraw/.append style={inner sep=0pt, rounded corners=0pt, align=center}}

%\newcommand{\booklink}[1]{\website/bibliographie\##1}
\newcounter{reference}
\newcommand{\previousreference}{}
\providecommand{\reference}[2][]{%
  \needspace{20pt}%
  \notblank{#1}{
    \needspace{20pt}%
    \renewcommand{\previousreference}{#1}%
    \stepcounter{reference}%
    \label{reference-\previousreference-\thereference}%
  }{}%
  \todo[noline]{%
    \protect\vspace{20pt}%
    \protect\par%
    \protect\notblank{#1}{\cite{[\previousreference]}\\}{}%
    \protect\hyperref[reference-\previousreference-\thereference]{p. #2}%
  }%
}

\definecolor{devcolor}{HTML}{00695c}
\providecommand{\dev}[1]{%
  \reversemarginpar%
  \todo[noline]{
    \protect\vspace{20pt}%
    \protect\par%
    \bfseries\color{devcolor}\href{\website/developpements/#1}{[DEV]}
  }%
  \normalmarginpar%
}

% En-têtes :

\pagestyle{fancy}
\fancyhead[L]{\truncate{0.23\textwidth}{\thepage}}
\fancyfoot[C]{\scriptsize \href{\website}{\texttt{https://github.com/imbodj/SenCoursDeMaths}}}

% Couleurs :

\definecolor{property}{HTML}{ffeb3b}
\definecolor{proposition}{HTML}{ffc107}
\definecolor{lemma}{HTML}{ff9800}
\definecolor{theorem}{HTML}{f44336}
\definecolor{corollary}{HTML}{e91e63}
\definecolor{definition}{HTML}{673ab7}
\definecolor{notation}{HTML}{9c27b0}
\definecolor{example}{HTML}{00bcd4}
\definecolor{cexample}{HTML}{795548}
\definecolor{application}{HTML}{009688}
\definecolor{remark}{HTML}{3f51b5}
\definecolor{algorithm}{HTML}{607d8b}
%\definecolor{proof}{HTML}{e1f5fe}
\definecolor{exercice}{HTML}{e1f5fe}

% Théorèmes :

\theoremstyle{definition}
\newtheorem{theorem}{Théorème}

\newtheorem{property}[theorem]{Propriété}
\newtheorem{proposition}[theorem]{Proposition}
\newtheorem{lemma}[theorem]{Activité d'introduction}
\newtheorem{corollary}[theorem]{Conséquence}

\newtheorem{definition}[theorem]{Définition}
\newtheorem{notation}[theorem]{Notation}

\newtheorem{example}[theorem]{Exemple}
\newtheorem{cexample}[theorem]{Contre-exemple}
\newtheorem{application}[theorem]{Application}

\newtheorem{algorithm}[theorem]{Algorithme}
\newtheorem{exercice}[theorem]{Exercice}

\theoremstyle{remark}
\newtheorem{remark}[theorem]{Remarque}

\counterwithin*{theorem}{section}

\newcommand{\applystyletotheorem}[1]{
  \tcolorboxenvironment{#1}{
    enhanced,
    breakable,
    colback=#1!8!white,
    %right=0pt,
    %top=8pt,
    %bottom=8pt,
    boxrule=0pt,
    frame hidden,
    sharp corners,
    enhanced,borderline west={4pt}{0pt}{#1},
    %interior hidden,
    sharp corners,
    after=\par,
  }
}

\applystyletotheorem{property}
\applystyletotheorem{proposition}
\applystyletotheorem{lemma}
\applystyletotheorem{theorem}
\applystyletotheorem{corollary}
\applystyletotheorem{definition}
\applystyletotheorem{notation}
\applystyletotheorem{example}
\applystyletotheorem{cexample}
\applystyletotheorem{application}
\applystyletotheorem{remark}
%\applystyletotheorem{proof}
\applystyletotheorem{algorithm}
\applystyletotheorem{exercice}

% Environnements :

\NewEnviron{whitetabularx}[1]{%
  \renewcommand{\arraystretch}{2.5}
  \colorbox{white}{%
    \begin{tabularx}{\textwidth}{#1}%
      \BODY%
    \end{tabularx}%
  }%
}

% Maths :

\DeclareFontEncoding{FMS}{}{}
\DeclareFontSubstitution{FMS}{futm}{m}{n}
\DeclareFontEncoding{FMX}{}{}
\DeclareFontSubstitution{FMX}{futm}{m}{n}
\DeclareSymbolFont{fouriersymbols}{FMS}{futm}{m}{n}
\DeclareSymbolFont{fourierlargesymbols}{FMX}{futm}{m}{n}
\DeclareMathDelimiter{\VERT}{\mathord}{fouriersymbols}{152}{fourierlargesymbols}{147}

% Code :

\definecolor{greencode}{rgb}{0,0.6,0}
\definecolor{graycode}{rgb}{0.5,0.5,0.5}
\definecolor{mauvecode}{rgb}{0.58,0,0.82}
\definecolor{bluecode}{HTML}{1976d2}
\lstset{
  basicstyle=\footnotesize\ttfamily,
  breakatwhitespace=false,
  breaklines=true,
  %captionpos=b,
  commentstyle=\color{greencode},
  deletekeywords={...},
  escapeinside={\%*}{*)},
  extendedchars=true,
  frame=none,
  keepspaces=true,
  keywordstyle=\color{bluecode},
  language=Python,
  otherkeywords={*,...},
  numbers=left,
  numbersep=5pt,
  numberstyle=\tiny\color{graycode},
  rulecolor=\color{black},
  showspaces=false,
  showstringspaces=false,
  showtabs=false,
  stepnumber=2,
  stringstyle=\color{mauvecode},
  tabsize=2,
  %texcl=true,
  xleftmargin=10pt,
  %title=\lstname
}

\newcommand{\codedirectory}{}
\newcommand{\inputalgorithm}[1]{%
  \begin{algorithm}%
    \strut%
    \lstinputlisting{\codedirectory#1}%
  \end{algorithm}%
}




\begin{document}
  %<*content>
  \development{algebra}{formes-de-hankel}{Formes de Hankel}

  \summary{Le but de ce développement est de construire une forme quadratique permettant de dénombrer les racines réelles distinctes d'un polynôme en fonction de ses racines complexes.}

  \reference[C-G]{356}

  Soit $P \in \mathbb{R}[X]$ un polynôme de degré $n$.

  \begin{theorem}[Formes de Hankel]
    On note $x_1, \dots, x_t$ les racines complexes de $P$ de multiplicités respectives $m_1, \dots, m_t$. On pose
    \[ s_0 = n \text{ et } \forall k \geq 1, \, s_k = \sum_{i=1}^t m_i x_i^k \]
    Alors :
    \begin{enumerate}[label=(\roman*)]
      \item $\sigma = \sum_{i, j \in \llbracket 0, n-1 \rrbracket} s_{i+j} X_i X_j$ définit une forme quadratique sur $\mathbb{C}^n$ ainsi qu'une forme quadratique $\sigma_{\mathbb{R}}$ sur $\mathbb{R}^n$.
      \item Si on note $(p,q)$ la signature de $\sigma_{\mathbb{R}}$, on a :
      \begin{itemize}
        \item $t = p + q$.
        \item Le nombre de racines réelles distinctes de $P$ est $p-q$.
      \end{itemize}
    \end{enumerate}
  \end{theorem}

  \begin{proof}
    $\sigma$ est un polynôme homogène de degré $2$ sur $\mathbb{C}$ (car la somme des exposants est $2$ pour chacun des monômes), qui définit donc une forme quadratique sur $\mathbb{C}^n$. De plus, on peut écrire :
    \[ \forall k \geq 1, \, s_k = \sum_{\substack{x_i \text{ racine de P} \\ x_i \in \mathbb{R}}} m_i x^i + \sum_{\substack{x \text{ racine de P} \\ x_i \in \mathbb{C}}} m_i (x^i + \overline{x}^k) \]
    donc $s_k = \overline{s_k}$ ie. $s_k \in \mathbb{R}$. Donc $\sigma$ définit une forme quadratique $\sigma_{\mathbb{R}}$ sur $\mathbb{R}^n$. D'où le premier point.
    \newpar
    Soit $\varphi_k$ la forme linéaire sur $\mathbb{C}^n$ définie par le polynôme homogène de degré $1$
    \[ P_k(X_0, \dots, X_{n-1}) = X_0 + x_k X_1 + \dots + x_k^{n-1} X_{n-1} \]
    pour $k \in \llbracket 0, t \rrbracket$. Dans la base duale $(e_i^*)_{i \in \llbracket 0, n-1 \rrbracket}$ de la base canonique $(e_i)_{i \in \llbracket 0, n-1 \rrbracket}$ de $\mathbb{C}^n$, on a
    \[ \varphi_k = e_0^* + x_k e_1^* + \dots + x_k^{n-1} e_{n-1}^* \]
    Et comme
    \[ \det((\varphi_k)_{k \in \llbracket 0, t \rrbracket}) = \begin{vmatrix} 1 & 1 & \dots & 1 \\ x_0 & x_1 & \dots & x_t \\ \vdots & \ddots & \ddots & \vdots \\ x_0^{n-1} & x_1^{n-1} & \dots & x_t^{n-1} \end{vmatrix} \overset{\text{Vandermonde}}{\neq} 0 \]
    la famille $(\varphi_k)_{k \in \llbracket 0, t \rrbracket}$ est de rang $t$ sur $\mathbb{C}$. Or, le coefficient de $X_i X_j$ dans $\sum_{k=1}^t m_k P_k^2$ vaut
    \[ \begin{cases} \sum_{k=1}^t m_k x_k^{2i} = s_{i+j} &\text{ si } i=j \\ \sum_{k=1}^t 2 m_k x_k^i x_k^j = \sum_{k=1}^t 2 m_k x_k^{i+j} = 2s_{i+j} &\text{ sinon} \end{cases} \]
    donc, $\sigma = \sum_{k=1}^t m_k \varphi_k^2$. En particulier, $\rang (\sigma) = t$ par indépendance des $\varphi_k$. On en déduit,
    \[ p+q = \rang(\sigma_{\mathbb{R}}) = \rang(\sigma) = t \]
    (le rang est invariant par extension de corps).
    \newpar
    Soit $k \in \llbracket 0, t \rrbracket$. Calculons la signature de la forme quadratique $\varphi_k^2 + \overline{\varphi_k}^2$ :
    \begin{itemize}
      \item Si $x_k \in \mathbb{R}$, on a $\varphi_k^2 + \overline{\varphi_k}^2 = 2 \varphi_k^2$, qui est de signature $(1, 0)$ car $\varphi_k \neq 0$.
      \item Si $x_k \notin \mathbb{R}$, on a $\varphi_k^2 + \overline{\varphi_k}^2 = 2 \operatorname{Re}(\varphi_k)^2 - 2 \operatorname{Im}(\varphi_k)^2$ qui est bien une forme quadratique réelle. Et $x_k \neq \overline{x_k}$, donc la matrice
      \[ \begin{pmatrix} 1 & 1 \\ x_k & \overline{x_k} \\ \vdots & \vdots \\ x_k^{n-1} & \overline{x_k}^{n-1} \end{pmatrix} \]
      est de rang $2$ (cf. le mineur correspondant aux deux premières lignes). Donc $\varphi_k$ et $\overline{\varphi_k}$ sont indépendantes. Ainsi, $\rang(\varphi_k^2 + \overline{\varphi_k}^2) = 2$ sur $\mathbb{C}$, donc sur $\mathbb{R}$ aussi (toujours par invariance du rang par extension de corps). Donc la signature de $\varphi_k^2 + \overline{\varphi_k}^2$ est $(1, 1)$.
    \end{itemize}
    Maintenant, regroupons les $\varphi_k$ conjuguées entre elles lorsqu'elles ne sont pas réelles :
    \[ \sigma = \sum_{\substack{k=1 \\ x_k \in \mathbb{R}}}^t m_k \varphi_k^2 + \sum_{\substack{k=1 \\ x_k \notin \mathbb{R}}}^t m_k (\varphi_k^2 + \overline{\varphi_k}^2) \]
    En passant à la signature, on obtient :
    \[ (p, q) = (r, 0) + \left( \frac{t-r}{2}, \frac{t-r}{2} \right) = \left( \frac{t+r}{2}, \frac{t-r}{2} \right) \]
    où $r$ désigne le nombre de racines réelles distinctes de $P$. Par unicité de la signature d'une forme quadratique réelle, on a bien $p-q=r$. D'où le point $(ii)$.
  \end{proof}

  \begin{remark}
    Tout l'intérêt de ces formes quadratiques est qu'on peut calculer les $s_k$ par récurrence en utilisant les polynômes symétriques élémentaires, sans avoir besoin des racines.
  \end{remark}

  \reference[GOU21]{86}

  \begin{proposition}[Sommes de Newton]
    On pose $P = \sum_{k=0}^n a_k X^k$. Les sommes de Newton vérifient les relations suivantes :
    \begin{enumerate}[label=(\roman*)]
      \item $s_0 = n$.
      \item $\forall k \in \llbracket 1, n-1 \rrbracket, \, s_k = -k a_{n-k} \sum_{i=1}^{k-1} s_i a_{n-k+i}$.
      \item $\forall p \in \mathbb{N}, \, s_{p+n} = \sum_{i=1}^{n} s_i a_{p+n-i}$.
    \end{enumerate}
  \end{proposition}
  %</content>
\end{document}
