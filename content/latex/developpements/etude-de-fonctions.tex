\documentclass[12pt, a4paper]{report}

% LuaLaTeX :

\RequirePackage{iftex}
\RequireLuaTeX

% Packages :

\usepackage[french]{babel}
%\usepackage[utf8]{inputenc}
%\usepackage[T1]{fontenc}
\usepackage[pdfencoding=auto, pdfauthor={Hugo Delaunay}, pdfsubject={Mathématiques}, pdfcreator={agreg.skyost.eu}]{hyperref}
\usepackage{amsmath}
\usepackage{amsthm}
%\usepackage{amssymb}
\usepackage{stmaryrd}
\usepackage{tikz}
\usepackage{tkz-euclide}
\usepackage{fontspec}
\defaultfontfeatures[Erewhon]{FontFace = {bx}{n}{Erewhon-Bold.otf}}
\usepackage{fourier-otf}
\usepackage[nobottomtitles*]{titlesec}
\usepackage{fancyhdr}
\usepackage{listings}
\usepackage{catchfilebetweentags}
\usepackage[french, capitalise, noabbrev]{cleveref}
\usepackage[fit, breakall]{truncate}
\usepackage[top=2.5cm, right=2cm, bottom=2.5cm, left=2cm]{geometry}
\usepackage{enumitem}
\usepackage{tocloft}
\usepackage{microtype}
%\usepackage{mdframed}
%\usepackage{thmtools}
\usepackage{xcolor}
\usepackage{tabularx}
\usepackage{xltabular}
\usepackage{aligned-overset}
\usepackage[subpreambles=true]{standalone}
\usepackage{environ}
\usepackage[normalem]{ulem}
\usepackage{etoolbox}
\usepackage{setspace}
\usepackage[bibstyle=reading, citestyle=draft]{biblatex}
\usepackage{xpatch}
\usepackage[many, breakable]{tcolorbox}
\usepackage[backgroundcolor=white, bordercolor=white, textsize=scriptsize]{todonotes}
\usepackage{luacode}
\usepackage{float}
\usepackage{needspace}
\everymath{\displaystyle}

% Police :

\setmathfont{Erewhon Math}

% Tikz :

\usetikzlibrary{calc}
\usetikzlibrary{3d}

% Longueurs :

\setlength{\parindent}{0pt}
\setlength{\headheight}{15pt}
\setlength{\fboxsep}{0pt}
\titlespacing*{\chapter}{0pt}{-20pt}{10pt}
\setlength{\marginparwidth}{1.5cm}
\setstretch{1.1}

% Métadonnées :

\author{agreg.skyost.eu}
\date{\today}

% Titres :

\setcounter{secnumdepth}{3}

\renewcommand{\thechapter}{\Roman{chapter}}
\renewcommand{\thesubsection}{\Roman{subsection}}
\renewcommand{\thesubsubsection}{\arabic{subsubsection}}
\renewcommand{\theparagraph}{\alph{paragraph}}

\titleformat{\chapter}{\huge\bfseries}{\thechapter}{20pt}{\huge\bfseries}
\titleformat*{\section}{\LARGE\bfseries}
\titleformat{\subsection}{\Large\bfseries}{\thesubsection \, - \,}{0pt}{\Large\bfseries}
\titleformat{\subsubsection}{\large\bfseries}{\thesubsubsection. \,}{0pt}{\large\bfseries}
\titleformat{\paragraph}{\bfseries}{\theparagraph. \,}{0pt}{\bfseries}

\setcounter{secnumdepth}{4}

% Table des matières :

\renewcommand{\cftsecleader}{\cftdotfill{\cftdotsep}}
\addtolength{\cftsecnumwidth}{10pt}

% Redéfinition des commandes :

\renewcommand*\thesection{\arabic{section}}
\renewcommand{\ker}{\mathrm{Ker}}

% Nouvelles commandes :

\newcommand{\website}{https://github.com/imbodj/SenCoursDeMaths}

\newcommand{\tr}[1]{\mathstrut ^t #1}
\newcommand{\im}{\mathrm{Im}}
\newcommand{\rang}{\operatorname{rang}}
\newcommand{\trace}{\operatorname{trace}}
\newcommand{\id}{\operatorname{id}}
\newcommand{\stab}{\operatorname{Stab}}
\newcommand{\paren}[1]{\left(#1\right)}
\newcommand{\croch}[1]{\left[ #1 \right]}
\newcommand{\Grdcroch}[1]{\Bigl[ #1 \Bigr]}
\newcommand{\grdcroch}[1]{\bigl[ #1 \bigr]}
\newcommand{\abs}[1]{\left\lvert #1 \right\rvert}
\newcommand{\limi}[3]{\lim_{#1\to #2}#3}
\newcommand{\pinf}{+\infty}
\newcommand{\minf}{-\infty}
%%%%%%%%%%%%%% ENSEMBLES %%%%%%%%%%%%%%%%%
\newcommand{\ensemblenombre}[1]{\mathbb{#1}}
\newcommand{\Nn}{\ensemblenombre{N}}
\newcommand{\Zz}{\ensemblenombre{Z}}
\newcommand{\Qq}{\ensemblenombre{Q}}
\newcommand{\Qqp}{\Qq^+}
\newcommand{\Rr}{\ensemblenombre{R}}
\newcommand{\Cc}{\ensemblenombre{C}}
\newcommand{\Nne}{\Nn^*}
\newcommand{\Zze}{\Zz^*}
\newcommand{\Zzn}{\Zz^-}
\newcommand{\Qqe}{\Qq^*}
\newcommand{\Rre}{\Rr^*}
\newcommand{\Rrp}{\Rr_+}
\newcommand{\Rrm}{\Rr_-}
\newcommand{\Rrep}{\Rr_+^*}
\newcommand{\Rrem}{\Rr_-^*}
\newcommand{\Cce}{\Cc^*}
%%%%%%%%%%%%%%  INTERVALLES %%%%%%%%%%%%%%%%%
\newcommand{\intff}[2]{\left[#1\;,\; #2\right]  }
\newcommand{\intof}[2]{\left]#1 \;, \;#2\right]  }
\newcommand{\intfo}[2]{\left[#1 \;,\; #2\right[  }
\newcommand{\intoo}[2]{\left]#1 \;,\; #2\right[  }

\providecommand{\newpar}{\\[\medskipamount]}

\newcommand{\annexessection}{%
  \newpage%
  \subsection*{Annexes}%
}

\providecommand{\lesson}[3]{%
  \title{#3}%
  \hypersetup{pdftitle={#2 : #3}}%
  \setcounter{section}{\numexpr #2 - 1}%
  \section{#3}%
  \fancyhead[R]{\truncate{0.73\textwidth}{#2 : #3}}%
}

\providecommand{\development}[3]{%
  \title{#3}%
  \hypersetup{pdftitle={#3}}%
  \section*{#3}%
  \fancyhead[R]{\truncate{0.73\textwidth}{#3}}%
}

\providecommand{\sheet}[3]{\development{#1}{#2}{#3}}

\providecommand{\ranking}[1]{%
  \title{Terminale #1}%
  \hypersetup{pdftitle={Terminale #1}}%
  \section*{Terminale #1}%
  \fancyhead[R]{\truncate{0.73\textwidth}{Terminale #1}}%
}

\providecommand{\summary}[1]{%
  \textit{#1}%
  \par%
  \medskip%
}

\tikzset{notestyleraw/.append style={inner sep=0pt, rounded corners=0pt, align=center}}

%\newcommand{\booklink}[1]{\website/bibliographie\##1}
\newcounter{reference}
\newcommand{\previousreference}{}
\providecommand{\reference}[2][]{%
  \needspace{20pt}%
  \notblank{#1}{
    \needspace{20pt}%
    \renewcommand{\previousreference}{#1}%
    \stepcounter{reference}%
    \label{reference-\previousreference-\thereference}%
  }{}%
  \todo[noline]{%
    \protect\vspace{20pt}%
    \protect\par%
    \protect\notblank{#1}{\cite{[\previousreference]}\\}{}%
    \protect\hyperref[reference-\previousreference-\thereference]{p. #2}%
  }%
}

\definecolor{devcolor}{HTML}{00695c}
\providecommand{\dev}[1]{%
  \reversemarginpar%
  \todo[noline]{
    \protect\vspace{20pt}%
    \protect\par%
    \bfseries\color{devcolor}\href{\website/developpements/#1}{[DEV]}
  }%
  \normalmarginpar%
}

% En-têtes :

\pagestyle{fancy}
\fancyhead[L]{\truncate{0.23\textwidth}{\thepage}}
\fancyfoot[C]{\scriptsize \href{\website}{\texttt{https://github.com/imbodj/SenCoursDeMaths}}}

% Couleurs :

\definecolor{property}{HTML}{ffeb3b}
\definecolor{proposition}{HTML}{ffc107}
\definecolor{lemma}{HTML}{ff9800}
\definecolor{theorem}{HTML}{f44336}
\definecolor{corollary}{HTML}{e91e63}
\definecolor{definition}{HTML}{673ab7}
\definecolor{notation}{HTML}{9c27b0}
\definecolor{example}{HTML}{00bcd4}
\definecolor{cexample}{HTML}{795548}
\definecolor{application}{HTML}{009688}
\definecolor{remark}{HTML}{3f51b5}
\definecolor{algorithm}{HTML}{607d8b}
%\definecolor{proof}{HTML}{e1f5fe}
\definecolor{exercice}{HTML}{e1f5fe}

% Théorèmes :

\theoremstyle{definition}
\newtheorem{theorem}{Théorème}

\newtheorem{property}[theorem]{Propriété}
\newtheorem{proposition}[theorem]{Proposition}
\newtheorem{lemma}[theorem]{Activité d'introduction}
\newtheorem{corollary}[theorem]{Conséquence}

\newtheorem{definition}[theorem]{Définition}
\newtheorem{notation}[theorem]{Notation}

\newtheorem{example}[theorem]{Exemple}
\newtheorem{cexample}[theorem]{Contre-exemple}
\newtheorem{application}[theorem]{Application}

\newtheorem{algorithm}[theorem]{Algorithme}
\newtheorem{exercice}[theorem]{Exercice}

\theoremstyle{remark}
\newtheorem{remark}[theorem]{Remarque}

\counterwithin*{theorem}{section}

\newcommand{\applystyletotheorem}[1]{
  \tcolorboxenvironment{#1}{
    enhanced,
    breakable,
    colback=#1!8!white,
    %right=0pt,
    %top=8pt,
    %bottom=8pt,
    boxrule=0pt,
    frame hidden,
    sharp corners,
    enhanced,borderline west={4pt}{0pt}{#1},
    %interior hidden,
    sharp corners,
    after=\par,
  }
}

\applystyletotheorem{property}
\applystyletotheorem{proposition}
\applystyletotheorem{lemma}
\applystyletotheorem{theorem}
\applystyletotheorem{corollary}
\applystyletotheorem{definition}
\applystyletotheorem{notation}
\applystyletotheorem{example}
\applystyletotheorem{cexample}
\applystyletotheorem{application}
\applystyletotheorem{remark}
%\applystyletotheorem{proof}
\applystyletotheorem{algorithm}
\applystyletotheorem{exercice}

% Environnements :

\NewEnviron{whitetabularx}[1]{%
  \renewcommand{\arraystretch}{2.5}
  \colorbox{white}{%
    \begin{tabularx}{\textwidth}{#1}%
      \BODY%
    \end{tabularx}%
  }%
}

% Maths :

\DeclareFontEncoding{FMS}{}{}
\DeclareFontSubstitution{FMS}{futm}{m}{n}
\DeclareFontEncoding{FMX}{}{}
\DeclareFontSubstitution{FMX}{futm}{m}{n}
\DeclareSymbolFont{fouriersymbols}{FMS}{futm}{m}{n}
\DeclareSymbolFont{fourierlargesymbols}{FMX}{futm}{m}{n}
\DeclareMathDelimiter{\VERT}{\mathord}{fouriersymbols}{152}{fourierlargesymbols}{147}

% Code :

\definecolor{greencode}{rgb}{0,0.6,0}
\definecolor{graycode}{rgb}{0.5,0.5,0.5}
\definecolor{mauvecode}{rgb}{0.58,0,0.82}
\definecolor{bluecode}{HTML}{1976d2}
\lstset{
  basicstyle=\footnotesize\ttfamily,
  breakatwhitespace=false,
  breaklines=true,
  %captionpos=b,
  commentstyle=\color{greencode},
  deletekeywords={...},
  escapeinside={\%*}{*)},
  extendedchars=true,
  frame=none,
  keepspaces=true,
  keywordstyle=\color{bluecode},
  language=Python,
  otherkeywords={*,...},
  numbers=left,
  numbersep=5pt,
  numberstyle=\tiny\color{graycode},
  rulecolor=\color{black},
  showspaces=false,
  showstringspaces=false,
  showtabs=false,
  stepnumber=2,
  stringstyle=\color{mauvecode},
  tabsize=2,
  %texcl=true,
  xleftmargin=10pt,
  %title=\lstname
}

\newcommand{\codedirectory}{}
\newcommand{\inputalgorithm}[1]{%
  \begin{algorithm}%
    \strut%
    \lstinputlisting{\codedirectory#1}%
  \end{algorithm}%
}




\begin{document}
	%<*content>
	\development{algebra}{etude-de-fonctions}{Fonctions numériques(TS2)}

 \summary{}
 
	\begin{exercice}
Soit la fonction $ f $ définie par :
$ f(x)=x^3-3x^2+4 $

On note $\mathscr{C}_{f}$ 
sa courbe dans un repère orthonormé.
\begin{enumerate}
\item Étudier le sens de variations de $ f $  puis établir son tableau de  variations.
\item Justifier que l'équation $ f(x)=2 $ admet au moins une solution dans $ \mathbb{R} $.
\item  Montrer que $ f $ admet sur $ \intfo{2}{\pinf} $ une bijection réciproque $ f^{-1} $ dont on précisera l'ensemble de définition.
\item Montre que  le point $(1, 2)$ est un point d'inflexion de $\mathscr{C}_{f}$.  
\item Tracer $\mathscr{C}_{f}$ et $\mathscr{C}_{f^{-1}}$.
\end{enumerate}
\end{exercice}

\begin{exercice}
Soit la fonction $ f $ définie par:
 $ f(x)=\dfrac{x^3+x^2-4}{x^2-4} $ 
\medskip

 On note par $\mathscr{C}_{f}$  sa courbe dans un repère orthonormé.
\begin{enumerate}
\item 
\begin{enumerate} 
\item Déterminer les asymptotes à la courbe $\mathscr{C}_{f}$.
\item Soit l'asymptote oblique $ \Delta $ de $ \mathcal{C}_f $.
Étudier la position relative de $ \mathcal{C}_f $ par rapport à $ \Delta $.
\end{enumerate}
\item Étudier le sens de variations de $ f $  puis établir son tableau de  variations.
\item Soit $ I $ le point d'intersection de $\Delta$ et $\mathscr{C}_{f}$.

 Montrer que $ I $  est un centre de symétrie de  $\mathscr{C}_{f}$.
\item Tracer $\mathscr{C}_{f}\: $ (unité 1cm).
\end{enumerate}
\end{exercice}

\begin{exercice}

On considère la fonction $f$ définie  par :
$\quad f(x)=\dfrac{2x-\sqrt{x}}{2+\sqrt{x}}.$

\begin{enumerate}
\item Déterminer Df puis  calculer $ \limi{x}{\pinf}{f(x)} $.
\item 
\begin{enumerate}
\item Montrer que sa dérivée est définie  par :
$\quad f'(x)=\dfrac{x+4\sqrt{x}-1}{\sqrt{x}\left(2+\sqrt{x}\right)^2}.$


\item Résoudre l'équation :
$ X^2+4X-1=0$
puis en déduire le signe de $f'(x)$ ainsi que les variations de $f$ sur Df.

\item Dresser alors le tableau de variations  de la fonction $f$ sur Df.

On veillera notamment à calculer la valeur exacte  de l'extremum de $f$.
\end{enumerate}
\item Déterminer la branche infinie de  la courbe  de $ f $ puis construire cette courbe (unité  8cm).
\end{enumerate}
\end{exercice}
\begin{exercice}
Soit la fonction $ f $ définie par:
 $ f(x)=(1-x)\sqrt{\abs{1-x^2}} $ 

$\mathscr{C}_{f}$  sa courbe dans un repère  orthonormé d'unité 2cm.

\begin{enumerate}
\item Déterminer l'ensemble de définition de $ f $.
\item Etudier la dérivabilité de $ f $  en $ 1$ et $ -1$.
\item Calculer $ f'(x) $ et étudier son signe. 

Dresser le tableau de variations de $ f $.
\item Déterminer les branches infinies de la courbe de $ f $.
\item Tracer $\mathscr{C}_{f}$ dans le  repère.
\end{enumerate}
\end{exercice}
\begin{exercice}

\textbf{Partie A}

Soit la fonction $ P $ définie par $ P(x)=4x^3-3x-8 $.
\begin{enumerate}
\item Etablir le tableau de variations de $ P $.
\item Montrer que l'équation $ P(x)=0 $  admet une unique  solution $ \alpha $ dans $ \Rr $.

Vérifier que $ \alpha \in\bigl[1,45\; ;  \; 1,46\bigr]$.
\item En déduire le signe de $ P(x) $ sur $ \Rr $.
\end{enumerate}

 \textbf{ Partie B}


Soit la fonction $ f $ définie sur $ \intfo{0}{\pinf} $ par : $\quad f(x)=\dfrac{x^3+1}{4x^2-1}$.
 $\; \mathcal{C}_f $  sa courbe représentative.

\begin{enumerate}
\item Etudier les limites  de $ f $ aux bornes de $Df $.
\item Calculer  $ f^{\prime}(x) $ en fonction de $ P(x) $.
\item En déduire   le signe de  $ f^{\prime}(x) $ puis  dresser le tableau de variations de $ f $.
\item Montrer que $ f(\alpha)=\dfrac{3}{8}\alpha $.
\item Montrer que la droite  d'équation $ \mathcal{D} :$ $ y=\dfrac{x}{4} $ est une asymptote à $ \mathcal{C}_f $.
\item Étudier les positions relatives de $ \mathcal{C}_f $ et $ \mathcal{D} $.
\item Tracer  $ \mathcal{C}_f $ dans un repère orthonormé.
\end{enumerate}
\end{exercice}

\begin{exercice}

\textbf{Partie A}

Soit la fonction $ g $ définie par :
$ g(x)=-x\sqrt{1+x^{2}}-1 $.
\begin{enumerate}
\item Etudier les variations de $ g $.
\item Montrer que l'équation $ g(x)=0 $  admet une unique  solution $ \alpha $ dans $ \Rr $.
\item Déterminer la valeur  exacte de $ \alpha  $.
\item En déduire que:\\ si $ \alpha<0$ alors  $ g(x) >0$  et si $ \alpha\geq 0$ alors  $ g(x) \leq0$ 
\end{enumerate}
\medskip

 \textbf{ Partie B}\\
Soit la fonction $ f $ définie par :

\medskip 

$ f(x)= -\dfrac{x^{3}}{3}- \sqrt{1+x^{2}}$.
\begin{enumerate}
\item Calculer les limites aux bornes de $Df $.
\item Déterminer la nature des branches infinies de $ \mathcal{C}$.
\item Calculer  $ f^{\prime}(x) $ en fonction de $ g(x) $.
\item Montrer que $ f(\alpha)= \dfrac{3-\alpha^{4}}{3\alpha}$.
\item  Dresser le tableau de variations de $ f $.
\item Déterminer l'équation de la tangente (T) à $ \mathcal{C}_f $ au point d'abscisse $ 0. $
\item Étudier la position relative  de $ \mathcal{C}$ par rapport à (T).
\item Tracer  $ \mathcal{C}_f $ dans un repère orthonormé d'unité 2cm.
\end{enumerate}
\end{exercice}

\begin{exercice}
  On considère  la fonction $ f $ définie par:
    $$f(x)=\begin{cases}  
-x+2-\dfrac{2x}{x^2+1} & \text{ si }  x \leq 1\\
x-1-3\sqrt{x^{2} -1 } & \text{ si }  x > 1  
\end{cases} $$ 
\begin{enumerate} 
\item Calculer les limites aux bornes de $ D_{f}$.
\item Etudier la continuité de $ f $ en 1.
\item Etudier la dérivabilité  de $ f $ en 1. Interpréter graphiquement le résultat.
\item
\begin{enumerate} 
\item Déterminer les  asymptotes de $ \mathcal{C}_f $.
\item Etudier la position de $ \mathcal{C}_f $ par rapport à ses asymptotes.
\end{enumerate}
\item Calculer $ f^{\prime}(x) $ sur les intervalles où $ f $ est dérivable en justifiant la dérivabilité de $ f $ sur chacun de ces intervalles  puis dresser son tableau de variations.
\item Construire la courbe $ \mathcal{C}_f $.
\item 
\begin{enumerate} 
\item Soit $ g $ la restriction de $ f $ à l'intervalle 
$ I = ]-\infty, \;\; 1] $.

 Montrer que $ g $ réalise une bijection de  $ I $ vers un intervalle  $ J $ à préciser.
 \item  Tracer $ \mathcal{C}_{g^{-1}}$ dans le repère.
\end{enumerate}
\end{enumerate}
 \end{exercice}

 \begin{exercice}

Soit  $\; f(x)=\cos 4x+ 2\sin 2x $. 
 \begin{enumerate}
 \item  Déterminer  $ D_f $ puis justifier le choix de $\; I= [0,\; \pi] $ comme intervalle  d'étude de $ f $.
 \item Montrer que: $ f^{\prime}(x)=4\paren{1-2\sin 2x}\cos 2x$ , $ x\in I$.
 \item Résoudre dans  $ I$ l'équation $ f'(x)=0. $
 \item En déduire le tableau de variations  de $ f$.
% \item Montrer que la droite $ x=\dfrac{\pi}{4} $ est un axe de symétrie de $ \mathcal{C} $.
 \item Construire $ \mathcal{C}_{f} $  sur \; $ [0,\;\; \pi] $.
 \end{enumerate}
    
\end{exercice}
 
 \begin{exercice}

Soit  $ \;f(x)=\dfrac{\cos^2 x}{\sin x} $.
 \begin{enumerate}
 \item Déterminer  l'ensemble de définition de $ f $.
 \item Démontrer que $ f $ est une fonction impaire et periodique de période $ 2\pi $.
 \item Démontrer  que  $ \mathcal{C}_{f} $  admet la droite $ x=\dfrac{\pi}{2} $ comme axe  de symétrie.
 \item Dresser  le tableau de variations  de $ f$ sur $\bigl[0\; ,\;\dfrac{\pi}{2}\bigr] $.
 \item Construire $ \mathcal{C}_{f} $  sur  $ \;[-\pi,\; \pi] $.
 \end{enumerate}
    
\end{exercice}

 

	%</content>
\end{document}
