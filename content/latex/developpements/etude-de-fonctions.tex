\input{../common}

\begin{document}
	%<*content>
	\development{algebra}{etude-de-fonctions}{Fonctions numériques(TS2)}

 \summary{}
 
	\begin{exercice}
Soit la fonction $ f $ définie par :
$ f(x)=x^3-3x^2+4 $

On note $\mathscr{C}_{f}$ 
sa courbe dans un repère orthonormé.
\begin{enumerate}
\item Étudier le sens de variations de $ f $  puis établir son tableau de  variations.
\item Justifier que l'équation $ f(x)=2 $ admet au moins une solution dans $ \mathbb{R} $.
\item  Montrer que $ f $ admet sur $ \intfo{2}{\pinf} $ une bijection réciproque $ f^{-1} $ dont on précisera l'ensemble de définition.
\item Montre que  le point $(1, 2)$ est un point d'inflexion de $\mathscr{C}_{f}$.  
\item Tracer $\mathscr{C}_{f}$ et $\mathscr{C}_{f^{-1}}$.
\end{enumerate}
\end{exercice}

\begin{exercice}
Soit la fonction $ f $ définie par:
 $ f(x)=\dfrac{x^3+x^2-4}{x^2-4} $ 
\medskip

 On note par $\mathscr{C}_{f}$  sa courbe dans un repère orthonormé.
\begin{enumerate}
\item 
\begin{enumerate} 
\item Déterminer les asymptotes à la courbe $\mathscr{C}_{f}$.
\item Soit l'asymptote oblique $ \Delta $ de $ \mathcal{C}_f $.
Étudier la position relative de $ \mathcal{C}_f $ par rapport à $ \Delta $.
\end{enumerate}
\item Étudier le sens de variations de $ f $  puis établir son tableau de  variations.
\item Soit $ I $ le point d'intersection de $\Delta$ et $\mathscr{C}_{f}$.

 Montrer que $ I $  est un centre de symétrie de  $\mathscr{C}_{f}$.
\item Tracer $\mathscr{C}_{f}\: $ (unité 1cm).
\end{enumerate}
\end{exercice}

\begin{exercice}

On considère la fonction $f$ définie  par :
$\quad f(x)=\dfrac{2x-\sqrt{x}}{2+\sqrt{x}}.$

\begin{enumerate}
\item Déterminer Df puis  calculer $ \limi{x}{\pinf}{f(x)} $.
\item 
\begin{enumerate}
\item Montrer que sa dérivée est définie  par :
$\quad f'(x)=\dfrac{x+4\sqrt{x}-1}{\sqrt{x}\left(2+\sqrt{x}\right)^2}.$


\item Résoudre l'équation :
$ X^2+4X-1=0$
puis en déduire le signe de $f'(x)$ ainsi que les variations de $f$ sur Df.

\item Dresser alors le tableau de variations  de la fonction $f$ sur Df.

On veillera notamment à calculer la valeur exacte  de l'extremum de $f$.
\end{enumerate}
\item Déterminer la branche infinie de  la courbe  de $ f $ puis construire cette courbe (unité  8cm).
\end{enumerate}
\end{exercice}
\begin{exercice}
Soit la fonction $ f $ définie par:
 $ f(x)=(1-x)\sqrt{\abs{1-x^2}} $ 

$\mathscr{C}_{f}$  sa courbe dans un repère  orthonormé d'unité 2cm.

\begin{enumerate}
\item Déterminer l'ensemble de définition de $ f $.
\item Etudier la dérivabilité de $ f $  en $ 1$ et $ -1$.
\item Calculer $ f'(x) $ et étudier son signe. 

Dresser le tableau de variations de $ f $.
\item Déterminer les branches infinies de la courbe de $ f $.
\item Tracer $\mathscr{C}_{f}$ dans le  repère.
\end{enumerate}
\end{exercice}
\begin{exercice}

\textbf{Partie A}

Soit la fonction $ P $ définie par $ P(x)=4x^3-3x-8 $.
\begin{enumerate}
\item Etablir le tableau de variations de $ P $.
\item Montrer que l'équation $ P(x)=0 $  admet une unique  solution $ \alpha $ dans $ \Rr $.

Vérifier que $ \alpha \in\bigl[1,45\; ;  \; 1,46\bigr]$.
\item En déduire le signe de $ P(x) $ sur $ \Rr $.
\end{enumerate}

 \textbf{ Partie B}


Soit la fonction $ f $ définie sur $ \intfo{0}{\pinf} $ par : $\quad f(x)=\dfrac{x^3+1}{4x^2-1}$.
 $\; \mathcal{C}_f $  sa courbe représentative.

\begin{enumerate}
\item Etudier les limites  de $ f $ aux bornes de $Df $.
\item Calculer  $ f^{\prime}(x) $ en fonction de $ P(x) $.
\item En déduire   le signe de  $ f^{\prime}(x) $ puis  dresser le tableau de variations de $ f $.
\item Montrer que $ f(\alpha)=\dfrac{3}{8}\alpha $.
\item Montrer que la droite  d'équation $ \mathcal{D} :$ $ y=\dfrac{x}{4} $ est une asymptote à $ \mathcal{C}_f $.
\item Étudier les positions relatives de $ \mathcal{C}_f $ et $ \mathcal{D} $.
\item Tracer  $ \mathcal{C}_f $ dans un repère orthonormé.
\end{enumerate}
\end{exercice}

\begin{exercice}

\textbf{Partie A}

Soit la fonction $ g $ définie par :
$ g(x)=-x\sqrt{1+x^{2}}-1 $.
\begin{enumerate}
\item Etudier les variations de $ g $.
\item Montrer que l'équation $ g(x)=0 $  admet une unique  solution $ \alpha $ dans $ \Rr $.
\item Déterminer la valeur  exacte de $ \alpha  $.
\item En déduire que:\\ si $ \alpha<0$ alors  $ g(x) >0$  et si $ \alpha\geq 0$ alors  $ g(x) \leq0$ 
\end{enumerate}
\medskip

 \textbf{ Partie B}\\
Soit la fonction $ f $ définie par :

\medskip 

$ f(x)= -\dfrac{x^{3}}{3}- \sqrt{1+x^{2}}$.
\begin{enumerate}
\item Calculer les limites aux bornes de $Df $.
\item Déterminer la nature des branches infinies de $ \mathcal{C}$.
\item Calculer  $ f^{\prime}(x) $ en fonction de $ g(x) $.
\item Montrer que $ f(\alpha)= \dfrac{3-\alpha^{4}}{3\alpha}$.
\item  Dresser le tableau de variations de $ f $.
\item Déterminer l'équation de la tangente (T) à $ \mathcal{C}_f $ au point d'abscisse $ 0. $
\item Étudier la position relative  de $ \mathcal{C}$ par rapport à (T).
\item Tracer  $ \mathcal{C}_f $ dans un repère orthonormé d'unité 2cm.
\end{enumerate}
\end{exercice}

\begin{exercice}
  On considère  la fonction $ f $ définie par:
    $$f(x)=\begin{cases}  
-x+2-\dfrac{2x}{x^2+1} & \text{ si }  x \leq 1\\
x-1-3\sqrt{x^{2} -1 } & \text{ si }  x > 1  
\end{cases} $$ 
\begin{enumerate} 
\item Calculer les limites aux bornes de $ D_{f}$.
\item Etudier la continuité de $ f $ en 1.
\item Etudier la dérivabilité  de $ f $ en 1. Interpréter graphiquement le résultat.
\item
\begin{enumerate} 
\item Déterminer les  asymptotes de $ \mathcal{C}_f $.
\item Etudier la position de $ \mathcal{C}_f $ par rapport à ses asymptotes.
\end{enumerate}
\item Calculer $ f^{\prime}(x) $ sur les intervalles où $ f $ est dérivable en justifiant la dérivabilité de $ f $ sur chacun de ces intervalles  puis dresser son tableau de variations.
\item Construire la courbe $ \mathcal{C}_f $.
\item 
\begin{enumerate} 
\item Soit $ g $ la restriction de $ f $ à l'intervalle 
$ I = ]-\infty, \;\; 1] $.

 Montrer que $ g $ réalise une bijection de  $ I $ vers un intervalle  $ J $ à préciser.
 \item  Tracer $ \mathcal{C}_{g^{-1}}$ dans le repère.
\end{enumerate}
\end{enumerate}
 \end{exercice}

 \begin{exercice}

Soit  $\; f(x)=\cos 4x+ 2\sin 2x $. 
 \begin{enumerate}
 \item  Déterminer  $ D_f $ puis justifier le choix de $\; I= [0,\; \pi] $ comme intervalle  d'étude de $ f $.
 \item Montrer que: $ f^{\prime}(x)=4\paren{1-2\sin 2x}\cos 2x$ , $ x\in I$.
 \item Résoudre dans  $ I$ l'équation $ f'(x)=0. $
 \item En déduire le tableau de variations  de $ f$.
% \item Montrer que la droite $ x=\dfrac{\pi}{4} $ est un axe de symétrie de $ \mathcal{C} $.
 \item Construire $ \mathcal{C}_{f} $  sur \; $ [0,\;\; \pi] $.
 \end{enumerate}
    
\end{exercice}
 
 \begin{exercice}

Soit  $ \;f(x)=\dfrac{\cos^2 x}{\sin x} $.
 \begin{enumerate}
 \item Déterminer  l'ensemble de définition de $ f $.
 \item Démontrer que $ f $ est une fonction impaire et periodique de période $ 2\pi $.
 \item Démontrer  que  $ \mathcal{C}_{f} $  admet la droite $ x=\dfrac{\pi}{2} $ comme axe  de symétrie.
 \item Dresser  le tableau de variations  de $ f$ sur $\bigl[0\; ,\;\dfrac{\pi}{2}\bigr] $.
 \item Construire $ \mathcal{C}_{f} $  sur  $ \;[-\pi,\; \pi] $.
 \end{enumerate}
    
\end{exercice}

 

	%</content>
\end{document}
