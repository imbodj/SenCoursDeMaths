\input{../common}
\everymath{\displaystyle}
\begin{document}
  %<*content>
  \development{algebra}{complexes-deux}{Nombres complexes et forme trigonométrique}

  \summary{Initiation sur les nombres complexes}

\begin{exercice}
\begin{enumerate}
\item Rappeler la forme trigonométrique d'un nombre complexe $ z $.
\item Mettre  sous la  forme trigonométrique les nombres complexes $ z $ suivants.

 $\textbf{a)} \;\;z=2\sqrt{3}-6\i
\hspace*{0.5cm}\textbf{b)} \;\;  z=-\dfrac{3}{2}+\i\dfrac{\sqrt{3}}{2} \hspace*{0.5cm}\textbf{c)} \;\; z=\paren{2+2\i}\paren{-\sqrt{3}+\i}^2
\hspace*{0.5cm}\textbf{d)} \;\; z=2\i\eexp{\i\frac{\pi}{6}} $
\medskip

 $\textbf{e)} \;\;z=\paren{-3+3\i}\eexp{\i\frac{\pi}{3}}
\hspace*{0.5cm}\textbf{f)} \;\;  z=1+\cos 2\theta +\i\sin 2\theta \hspace*{0.5cm}\textbf{g)} \;\; z=\sin \frac{\pi}{5} +\i\cos \frac{\pi}{5}$

$\textbf{h)} \;\; z=\frac{1+\sqrt{2}+\i}{1-\sqrt{2}+\i} $

\end{enumerate}
\end{exercice}

\begin{exercice}
Mettre sous forme algébrique les nombres complexes suivants.
\medskip

 $  z_1=\paren{1+\i}^{17}
\hspace*{0.5cm}  z_2=\paren{-\sqrt{3}+\i}^{2021}  \hspace*{0.5cm}  z_3=\dfrac{\paren{1+\i}^{3}}{\paren{\sqrt{3}+\i}^{4}} \hspace*{0.5cm} z_4=\eexp{\i\frac{\pi}{3}}+\eexp{-\i\frac{\pi}{6}} \hspace*{0.5cm} z_5=\dfrac{-\i\paren{\sqrt{3}-\i}^{2}}{2\paren{1-\i\sqrt{3}}^{7}}$

\end{exercice}


\begin{exercice}
 Le plan complexe  est rapporté  d'un repère orthonormé direct  $ (O ;\overrightarrow{u},\overrightarrow{v}) $. 
On considère les points  $ A$  , $B $ et $  C $  d'affixes respectives $ 1 +\i$  , $ \; 3+2\i  \; $  \; et  \; $ \; 3\i  $.
\begin{enumerate}
\item Donner une mesure de chacun des angles  orientés suivants :
  $ (\overrightarrow{u} , \overrightarrow{OA} )$ , $ (\overrightarrow{u} , \overrightarrow{OC} )$,  $ (\overrightarrow{v} , \overrightarrow{OA} )$ ,$ (\overrightarrow{v} , \overrightarrow{OC} )$, $ (\overrightarrow{CA} , \overrightarrow{CB} )$  et $ (\overrightarrow{AB}, \overrightarrow{AC} )$.
\item  Soit  $ Z = \dfrac{z_{C}-z_{A}}{z_{B}-z_{A}} $.
\begin{enumerate}
 \item Calculer  $  |Z| $ et un argument $ Z $. 
 \item Interpréter géométriquement $  |Z| $ et un argument $ Z $.
  En déduire la nature du triangle $ ABC $ . 
\end{enumerate} 
\end{enumerate} 
 \end{exercice}
\begin{exercice}
Soit  $\; z_1=\sqrt{2}+\i\sqrt{6} $, $\quad z_2=2-2\i\quad $  et   $\quad Z=\frac{z_1}{z_2} $.
    \begin{enumerate} 
   \item Ecrire $ Z $ sous forme algébrique.
  \item Ecrire  $z_1 $ et $ z_2 $  sous forme trigonométrique.
   \item En déduire  $ Z $ sous forme trigonométrique.
   \item Déterminer les valeurs de $\cos \frac{\pi}{12} $ et $\sin \frac{\pi}{12} $.
 \end{enumerate}
 
\end{exercice}

\begin{exercice}
 Soit $ \omega =\sqrt{3}+1 +\i\paren{\sqrt{3}-1}$
\begin{enumerate}
\item Ecrire $ \omega^{2} $ sous forme algébrique.
\item Déterminer le module et un argument de $ \omega ^{2} $. En déduire  le module et un argument de $ \omega  $.

\end{enumerate}
 \end{exercice}
\begin{exercice}
 Identifier la réponse juste et donner la justification.
\begin{enumerate}
\item  Si $\; \dfrac{\pi}{6}\; $\;  est un argument de \; $\; \dfrac{9}{z}$ \; alors un argument de $\; \dfrac{\i}{z^{2}} $  est:\\
  $\textbf{a/\;} \dfrac{\pi}{6} \hspace*{1cm} \textbf{b/\;\;}   -\dfrac{5\pi}{6} \hspace*{1cm}   \textbf{c/\;}   \dfrac{5\pi}{6} $
\item Soit $ z $ un nombre complexe non nul d’argument $ \theta $. Un argument de $ \dfrac{-1+\i\sqrt{3}}{\overline{z}} $  est:\\
  $ \textbf{a/\;} -\dfrac{\pi}{3}+\theta \hspace*{1cm} \textbf{b/\;\;}   \dfrac{2\pi}{3}+\theta \hspace*{1cm}   \textbf{c/\;}   \dfrac{2\pi}{3}-\theta  $
\item Un argument de   $ \;\sin(x)+\i\cos(x)\; $ est:\\  $\textbf{a/\;} -x\hspace*{1cm} \textbf{b/\;\;}   x  \hspace*{1cm}   \textbf{c/\;}   \dfrac{\pi}{2}-x    v \hspace*{1cm} \textbf{d/\;}   \dfrac{\pi}{2}+x  $
\item Le nombre complexe $\;  (\sqrt{3}+\i)^{1689} $  \\ \textbf{a/\;}  est un réel $ \hspace*{0.5cm} $ \textbf{b/\;}  est un imaginaire pur \textbf{c/\;} n'est ni  réel ni imaginaire pur. 
\item Le conjugué de $ \eexp{\i\theta}$ est :


$\textbf{a/\;\;}   -\eexp{\i\theta}  \hspace*{1.5cm }\textbf{b/\;\;}   \eexp{-\i\theta}    \hspace*{1.5cm }  \textbf{c/\;\;}   \eexp{\i\theta} $
\end{enumerate}

\end{exercice}

\begin{exercice}
On consid\`ere les trois nombres complexes suivants : $z_{1} =(1 - i)(1+2i)$, $\;z_{2}=\dfrac{2 +6i}{3-i}\;$  et $ \;z_{3}=\dfrac{4i}{i-1} $.
\medskip

 Soit $ M_{1} $ , $ M_{2} $ et  $ M_{3} $  leurs images respectives dans le plan.
 \begin{enumerate}
 \item  Donner leurs formes ag\'ebriques. 
 \item  Placer  $ M_{1} $ , $ M_{2} $  et  $ M_{3} $ dans le plan complexe.
 \item Calculer  $\dfrac{z_{3} -z_{1}}{z_{2} -z_{1}}$ .  En d\'eduire que le triangle   $ M_{1} $$ M_{2} $$ M_{3}$  est rectangle isoc\`ele  . 
 \item   D\'eterminer l'affixe du point $ M_{4} $ telle  que le quadrilat\`ere  $ M_{1}M_{2}  M_{4} M_{3}$  soit un carr\'e .
 \item Montrer que les points $ M_{1} $ ,$ M_{2} $ , $ M_{3}$   et  $ M_{4} $  appartiennent \`a  un m\^eme cercle dont on pr\'ecisera les  \'el\'ements .
 \end{enumerate}

\end{exercice}

\begin{exercice}
 $ x\in\mathbb{R} $. Soient les nombres complexes suivants:
\medskip

$ Z'= -2\paren{\cos \frac{\pi}{3}+\i\sin\frac{\pi}{3}} $ et $ Z= (1-x)\paren{\cos \frac{\pi}{3}+\i\sin\frac{\pi}{3}} $
\begin{enumerate}
\item Calculer le module et un argument de $ Z' $.
\item Calculer le module et un argument de $ Z $.

( On discutera selon les valeurs de $ x) $


Donner pour chaque cas la forme trigonométrique et la forme algébrique de  $ Z $.
\item Montrer que $ Z^{2004} $ est un nombre réel dont on précisera le signe.
\item Montrer que l'équation  $ |Z|=2 $ a deux solutions $ Z_1$ et $ Z_2$.

Ecrire  $ Z_1$ et $ Z_2$ forme algébrique.
\item Placer  les points A et B d'affixes respectives $ 2\text{e}^{\i\frac{\pi}{3}} $  et $ -2\text{e}^{\i\frac{\pi}{3}} $ dans le plan complexe muni d'un repère orthonormé $ \ouv. $

Vérifier que les points A, B et $ O $ sont alignés.
\end{enumerate}
\end{exercice}
  %</content>
\end{document}
