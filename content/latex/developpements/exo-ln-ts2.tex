\input{../common}

\begin{document}
	%<*content>
	\development{algebra}{exo-ln-ts2}{Fonction ln (TS2)}

 \summary{Initiation ln }
 
	\begin{exercice}
Ecrire plus simplement en un seul logarithme, chacune des expressions suivantes.

\medskip

 $ A=3\ln 2+\ln 5-2\ln 5\quad  $  $ B=2\ln 2+2\ln 5+1 $ 
 \bigskip
 
  $ C=\frac{1}{2}\ln3+ \ln \text{e}^2-\ln \frac{2}{\text{e}}+3\quad  $  $ D=\ln \bigl(3+\sqrt{5}\bigr)+\ln \bigl(3-\sqrt{5}\bigr) $
\end{exercice}

\begin{exercice}
Résoudre dans $ \mathbb{R} $ les équations suivantes.\medskip

\textbf{1)}\;  $ \ln(x-1)=\ln(2-x)  $  \medskip

\textbf{2)}\;$ \ln (x -2)-\ln (x+1)=2\ln 2 $ \medskip

 \textbf{3)}\; $ \ln ( x-2 )+\ln (x+3)=\ln(5x-9)$ \medskip
 
  \textbf{4)}\; $ \ln \left(\dfrac{x-1}{2x-1}\right) =0$ 
\end{exercice}

\begin{exercice}
Résoudre dans $ \mathbb{R} $ les inéquations suivantes.\medskip

\textbf{1)}\;  $ \ln(x-4)\leq \ln(10-x)  $  \medskip

 \textbf{3)}\; $ \ln ( x-1 )+\ln (x+2)\geq \ln (4x-8)$ \medskip
 
  \textbf{4)}\; $ \ln \left(\dfrac{x-1}{2x-1}\right) >0$ 
\end{exercice}

\begin{exercice}
On considère le polynôme $ P(x)=2x^{3}-9x^{2}+x+12$.
\begin{enumerate}
\item Résoudre dans $ \mathbb{R} $   l'inéquation $ P(x)\leq0 $.
\item En déduire les  solutions  de l'équation et l' inéquation suivantes.

\medskip

\textbf{a)}\;  $2\ln^{3}x-9\ln^{2} x+\ln(x)+12=0 $.

\medskip
\textbf{b)}\; $\ln(2x-3) +2\ln(x-2) \leq\ln(-2x^{2}+19x-24)$


\end{enumerate}
\end{exercice}
\begin{exercice}
 Résoudre dans $ \mathbb{R}^{2} $  les systèmes suivants:
\medskip

\textbf{1)} $\; \left\{\begin{array}{l}  \ln(x+2)+3 \ln(y-1)=4 \\[0.25cm]  2\ln(x+2)-\ln(y-1) =2 \end{array}\right. \hspace*{0.5cm}$
\textbf{2)} $\;  \left\{\begin{array}{l} x-y=\frac{3}{2} \\[0.25cm]  \ln x+\ln y=0 \end{array}\right.$

\end{exercice}

\begin{exercice}
Déterminer les limites  de $ f $ aux bornes de $ D_f $ puis  calculer sa fonction dérivée f$^{\prime} $.

\medskip
1)  $ f(x)=\dfrac{\ln(x)+1}{\ln(x)-1}\qquad $     2) $ f(x)= \dfrac{\ln(1+2x)}{x} $ 
\medskip

   3) $ f(x)=\dfrac{x+\ln(x)}{2x} \qquad$  4)  $ f(x)= x \ln\paren{1+\dfrac{1}{x}}  $ 
\end{exercice}
\begin{exercice}
Dresser le tableau de variations  des fonctions suivantes.\medskip

1) $ f(x)=\sqrt{3-\ln(x)} \quad$  2) $ f(x)= (\ln x)^{2}-2\ln x-3 $


\medskip 3) $ f(x)= \dfrac{\ln(x)}{1-\ln(x)}\hspace*{0.5cm}$   4)$ f(x)= \dfrac{\ln(x+1)}{x+1} $

\end{exercice}


\begin{exercice}
Etudier le signe des expressions suivantes: 

\medskip
$A(x)  = \ln x\bigl(\ln x+1\bigr)\quad  $  $ B(x) = 1-\ln(1-x) $

\medskip  
$C(x)  =2\ln^2(x) -\ln (x)-1  \quad $  $ D(x)=  \ln(x)-x+1$ 
\end{exercice}


\begin{exercice}
Soit $ f(x)= x-1+ \ln\abs{\dfrac{x+1}{x-1}} $
\begin{enumerate}
\item Déterminer  les limites aux bornes de  $D_{f}$.
\item Dresser le tableau de variations de $ f. $
\item Montrer que le point I$ (0;1)$ est à la fois centre de symétrie  et  point d'inflexion de la courbe de $ f$. 
\item Montrer que l'équation $ f(x)=0 $  admet une unique solution $ \alpha $ tel que $ 0<  \alpha<\frac{1}{2} $.
\item Représenter $ f. $
\end{enumerate}
\end{exercice}

\begin{exercice}
Soit  $ \; f(x)=\ln(-x^{2}+4x-3) $.
  \begin{enumerate}
\item 
\begin{enumerate}
 \item  Étudier les variations  de $ f$.
 \item En déduire le signe de $ f(x) $ sur $ D_{f} $.  
\end{enumerate}
\item Soit $ g $ la restriction de $ f  $ à  $ I=]2  ,3[ $.

 Montrer que $ g $ est une bijection de $ ]2  ,3[ $  sur un intervalle $ J $ à déterminer.
\end{enumerate}

 Soit $ F(x)=(x-1)\ln (x-1)-(3-x)\ln (3-x)-2x$  
\begin{enumerate}
\item Montrer que $ F $ est une primitive  sur $I $ de $ f $.
\item Etudier les variations de $ F $ et tracer $ C_{F} $.
\end{enumerate}
\end{exercice}



	%</content>
\end{document}
