\input{../common}

\begin{document}
	%<*content>
	\development{algebra}{primitives}{Primitives (TS2)}

 \summary{}
 
\begin{exercice}
Déterminer une primitive de chacune des fonctions suivantes :
\begin{enumerate}
  \item $f_1(x) = x^3+2$
  \item $f_2(x) = \dfrac{2}{x^2}$
  \item $f_3(x) = \cos(3x)$
  \item $f_4(x) = \dfrac{4}{\sqrt{x}}$
  \item $f_5(x) = 2\cos x \sin x$
\end{enumerate}
\end{exercice}

\begin{exercice}
Déterminer une primitive de chacune des fonctions suivantes, en posant une substitution adaptée si nécessaire :
\begin{enumerate}
  \item $f(x) = x(3x^2 - 1)^3$
  \item $f(x) = \dfrac{x}{(x^2 + 1)^2}$
  \item $f(x) = (x- 1)\sqrt{x-1}$
  \item $f(x) = \dfrac{\cos x}{1 + \sin x}$
\end{enumerate}
\end{exercice}
\begin{exercice}
 Soit la fonction $ f $  définie sur $  [0,\; +\infty[ $  par : \; $ f(x)=\dfrac{x^3}{(1+x^2)^3}+ \dfrac{x^2}{(1+x^3)^2\sqrt{1+x^3}} $.
\begin{enumerate}
\item Justifier que $ f $ admet des primitives sur $  [0,\; +\infty[ $.
\item Déterminer les réels $a $ et $ b$ tels que :\; $ \dfrac{x^3}{(1+x^2)^3}=\dfrac{ax}{(1+x^2)^2}+\dfrac{bx}{(1+x^2)^3} $.
\item En déduire la primitive $ F $ de $ f $  sur $  [0,\; +\infty[ $ qui s'annule en 0.
\end{enumerate}
\end{exercice}
\begin{exercice}
Soit la fonction $ g $  définie sur $ \mathbb{R} $  par : \\ $ g(x)=\paren{\cos 3x  +\cos x}\cos x $.
\begin{enumerate}
\item Déterminer les réels $a $ , $ b $, $ c $ et $ d$ tels que :\; $ g(x)=a+b\cos 2x +c\cos 4x +d\cos 6x$.
\item En déduire une primitive de $ g $  sur $\mathbb{R} $.  
\end{enumerate}
\end{exercice}

\begin{exercice}
Soit $f$ la fonction définie sur $]0,+\infty[$ par $f(x) = \ln x$.

\begin{enumerate}
  \item Justifier que $f$ admet une primitive sur $]0,+\infty[$.
  \item On considère la fonction $F(x) = x\ln x-x$. Montrer que $F$ est une primitive de $f$ sur $]0,+\infty[$.
  \item Déterminer une primitive de $g(x) = \ln (3x)$ sur $]0,+\infty[$.
\end{enumerate}
\end{exercice}

\begin{exercice}
Soit $f$ la fonction définie sur $\mathbb{R}$ par $f(x) = xe^x$.

\begin{enumerate}
  \item Montrer que $f$ est continue sur $\mathbb{R}$.
  \item Justifier que $f$ admet une primitive sur $\mathbb{R}$.
  \item En posant $F(x) = (x - 1)e^x$, montrer que $F$ est une primitive de $f$.
  \item Déterminer la primitive $G$ de $f$ dont la courbe passe  par le  point  $(0; 1)$ .
\end{enumerate}
\end{exercice}


\begin{exercice}
Une entreprise modélise la température (en °C) d’un four en fonction du temps $t$ (en minutes) par la dérivée $T'(t) = 4t - 20$, valable pour $t \in [0, 10]$.\\
On sait  qu’à l’instant $t = 0$, la température est de $300^\circ$C.\\
 À quel instant la température est-elle minimale ? Quelle est cette température ?
  
\end{exercice}
	%</content>
\end{document}