\input{../common}

\begin{document}
	%<*content>
	\development{algebra}{primitives}{Primitives (TS2)}

 \summary{}
 
\begin{exercice}
Pour chacune des fonctions définies ci-dessous, trouver :
\begin{enumerate}
    \item le(s) plus grand(s) intervalle(s) $I$ sur le(s)quel(s)  elle admet des primitives.
    \item l'expression d'une primitive sur chaque intervalle.
\end{enumerate}

1) $f(x)=3x^2 - 2x + 5$

\bigskip
 2)$ f(x)= (2x+1)(x^2+x - 5)$ 
 
 \bigskip
  3) $f(x) = \dfrac{x}{(1-x^2)^2}$

\bigskip
4) $f(x) = \dfrac{1}{4x^2 -4x + 1}$ 
\bigskip

 5) $f(x) = \dfrac{1}{x^3 - 3x^2+3x - 1}$ 
 
 \bigskip
   6) $f(x) = \cos 2x\cos 3x$
\bigskip

7) $f(x) = \sin^3 x \cos^4 x $

\bigskip
 8) $f(x) = \cos x\sin 3x$

\end{exercice}

\begin{exercice}
Pour chaque fonction $f$, déterminer une primitive sur $I$, prenant la valeur $b$ en  $a$ :
\begin{align*}
1)\quad & f(x) = \dfrac{2}{(3-x)^3}, \quad I = ]-\infty,3], \quad a=0, \ b=4 
\\[0.3cm]
2)\quad & f(x) = \dfrac{\sin x}{\cos^3 x}, \quad I = ]0,\tfrac{\pi}{2}[, \quad a=\tfrac{\pi}{3}, \ b=1
\\[0.3cm]
3)\quad & f(x) = \dfrac{x}{\sqrt{1+x^2}}, \quad I =\mathbb{R}, \quad a=0, \ b=1
\\[0.3cm]
4)\quad & f(x) = \dfrac{x\sin x+\cos x}{x^2}, \quad I = ]-\infty, 0[, \quad a=0, \ b=1
\\[0.3cm]
5)\quad & f(x) = \dfrac{1+\sin x}{\cos^2 x}, \quad I = [0,  \tfrac{\pi}{2}[ \quad a=\tfrac{\pi}{4}, \ b=1
\end{align*}
\end{exercice}
\pagebreak
\begin{exercice}
Déterminer une primitive $F$ dans chacun des cas suivants :
\begin{align*}
1)\quad & f(x) = (x^2+x+1)\sqrt{x} 
& 8)\quad & f(x)= \dfrac{12}{(x-1)^2}\left(1 + \dfrac{3}{x-1}\right)^3
\\[0.3cm]
2)\quad & f(x)= (x^2+x+1)\sqrt{x+1} 
& 9)\quad & f(x)= 1+ \dfrac{1}{\tan^2 x}
\\[0.3cm]
3)\quad & f(x)= x\sqrt{3-x}
& 10)\quad & f(x)= \dfrac{\cos(\sqrt{x})}{\sqrt{x}}
\\[0.3cm]
4)\quad & f(x)= \dfrac{x+1}{(x^2+2x+3)\sqrt{x^2+2x+3}}
& 11)\quad & f(x)= \dfrac{x^2-2x}{(x-1)^2} 
\\[0.3cm]
5)\quad & f(x)= \sin(x^2+2x) + (2x^2+2x)\cos (x^2+2x)
& 12)\quad & f(x)= \dfrac{x^3+3x}{(x^2-1)^3} \; \left(Écrire\; f(x)= \frac{a}{(x-1)^3}+\frac{b}{(x+1)^3}\right) 
\\[0.3cm]
6)\quad & f(x)= \dfrac{1}{1+\cos x} 
& 13)\quad & f(x)= \dfrac{x^2+1}{(x^2-1)^2} \; \left(Écrire\; x^2+1= \frac{1}{2}\left((x+1)^2+(x-1)^2\right)\right)
\\[0.3cm]
7)\quad & f(x)=\sqrt{x+1} +\frac{1}{2}\dfrac{x}{\sqrt{x+1}}& 14)\quad & f(x)=\dfrac{3x^2+4x+4}{(x^2+2x)^2}\; \left(Écrire\; f(x)= \frac{a}{x^2}+\frac{b}{(x+2)^2}\right) 
\end{align*}

\end{exercice}
\begin{exercice}
On se propose de déterminer une primitive sur $\mathbb{R}$ des fonctions :
\[
f(x) = x\cos^2 x, \quad g(x) = x\sin^2 x
\]

\begin{enumerate}
    \item Déterminer une primitive de $f+g$.
    \item Linéariser $\cos^2 x -\sin^2 x$. En déduire  qu'il existe deux réels $ a$ et $ b$ tels que  la fonction $x \mapsto a \sin 2x + b \cos 2x$ soit une primitive  sur $ \mathbb{R} $ de $f - g$.
    \item Conclure.
\end{enumerate}


\end{exercice}
\begin{exercice}
 Soit la fonction $ f $  définie sur $  [0,\; +\infty[ $  par : \; $ f(x)=\dfrac{x^3}{(1+x^2)^3}+ \dfrac{x^2}{(1+x^3)^2\sqrt{1+x^3}} $.
\begin{enumerate}
\item Justifier que $ f $ admet des primitives sur $  [0,\; +\infty[ $.
\item Déterminer les réels $a $ et $ b$ tels que :\; $ \dfrac{x^3}{(1+x^2)^3}=\dfrac{ax}{(1+x^2)^2}+\dfrac{bx}{(1+x^2)^3} $.
\item En déduire la primitive $ F $ de $ f $  sur $  [0,\; +\infty[ $ qui s'annule en 0.
\end{enumerate}
\end{exercice}
\begin{exercice}
Soit la fonction $ g $  définie sur $ \mathbb{R} $  par : \\ $ g(x)=\paren{\cos 3x  +\cos x}\cos x $.
\begin{enumerate}
\item Déterminer les réels $a $ , $ b $, $ c $ et $ d$ tels que :\; $ g(x)=a+b\cos 2x +c\cos 4x +d\cos 6x$.
\item En déduire une primitive de $ g $  sur $\mathbb{R} $.  
\end{enumerate}
\end{exercice}



\begin{exercice}
Une entreprise modélise la température (en °C) d’un four en fonction du temps $t$ (en minutes) par la dérivée $T'(t) = 4t - 20$, valable pour $t \in [0, 10]$.\\
On sait  qu’à l’instant $t = 0$, la température est de $300^\circ$C.\\
 À quel instant la température est-elle minimale ? Quelle est cette température ?
  
\end{exercice}
	%</content>
\end{document}