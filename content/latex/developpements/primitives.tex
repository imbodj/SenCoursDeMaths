\documentclass[12pt, a4paper]{report}

% LuaLaTeX :

\RequirePackage{iftex}
\RequireLuaTeX

% Packages :

\usepackage[french]{babel}
%\usepackage[utf8]{inputenc}
%\usepackage[T1]{fontenc}
\usepackage[pdfencoding=auto, pdfauthor={Hugo Delaunay}, pdfsubject={Mathématiques}, pdfcreator={agreg.skyost.eu}]{hyperref}
\usepackage{amsmath}
\usepackage{amsthm}
%\usepackage{amssymb}
\usepackage{stmaryrd}
\usepackage{tikz}
\usepackage{tkz-euclide}
\usepackage{fontspec}
\defaultfontfeatures[Erewhon]{FontFace = {bx}{n}{Erewhon-Bold.otf}}
\usepackage{fourier-otf}
\usepackage[nobottomtitles*]{titlesec}
\usepackage{fancyhdr}
\usepackage{listings}
\usepackage{catchfilebetweentags}
\usepackage[french, capitalise, noabbrev]{cleveref}
\usepackage[fit, breakall]{truncate}
\usepackage[top=2.5cm, right=2cm, bottom=2.5cm, left=2cm]{geometry}
\usepackage{enumitem}
\usepackage{tocloft}
\usepackage{microtype}
%\usepackage{mdframed}
%\usepackage{thmtools}
\usepackage{xcolor}
\usepackage{tabularx}
\usepackage{xltabular}
\usepackage{aligned-overset}
\usepackage[subpreambles=true]{standalone}
\usepackage{environ}
\usepackage[normalem]{ulem}
\usepackage{etoolbox}
\usepackage{setspace}
\usepackage[bibstyle=reading, citestyle=draft]{biblatex}
\usepackage{xpatch}
\usepackage[many, breakable]{tcolorbox}
\usepackage[backgroundcolor=white, bordercolor=white, textsize=scriptsize]{todonotes}
\usepackage{luacode}
\usepackage{float}
\usepackage{needspace}
\everymath{\displaystyle}

% Police :

\setmathfont{Erewhon Math}

% Tikz :

\usetikzlibrary{calc}
\usetikzlibrary{3d}

% Longueurs :

\setlength{\parindent}{0pt}
\setlength{\headheight}{15pt}
\setlength{\fboxsep}{0pt}
\titlespacing*{\chapter}{0pt}{-20pt}{10pt}
\setlength{\marginparwidth}{1.5cm}
\setstretch{1.1}

% Métadonnées :

\author{agreg.skyost.eu}
\date{\today}

% Titres :

\setcounter{secnumdepth}{3}

\renewcommand{\thechapter}{\Roman{chapter}}
\renewcommand{\thesubsection}{\Roman{subsection}}
\renewcommand{\thesubsubsection}{\arabic{subsubsection}}
\renewcommand{\theparagraph}{\alph{paragraph}}

\titleformat{\chapter}{\huge\bfseries}{\thechapter}{20pt}{\huge\bfseries}
\titleformat*{\section}{\LARGE\bfseries}
\titleformat{\subsection}{\Large\bfseries}{\thesubsection \, - \,}{0pt}{\Large\bfseries}
\titleformat{\subsubsection}{\large\bfseries}{\thesubsubsection. \,}{0pt}{\large\bfseries}
\titleformat{\paragraph}{\bfseries}{\theparagraph. \,}{0pt}{\bfseries}

\setcounter{secnumdepth}{4}

% Table des matières :

\renewcommand{\cftsecleader}{\cftdotfill{\cftdotsep}}
\addtolength{\cftsecnumwidth}{10pt}

% Redéfinition des commandes :

\renewcommand*\thesection{\arabic{section}}
\renewcommand{\ker}{\mathrm{Ker}}

% Nouvelles commandes :

\newcommand{\website}{https://github.com/imbodj/SenCoursDeMaths}

\newcommand{\tr}[1]{\mathstrut ^t #1}
\newcommand{\im}{\mathrm{Im}}
\newcommand{\rang}{\operatorname{rang}}
\newcommand{\trace}{\operatorname{trace}}
\newcommand{\id}{\operatorname{id}}
\newcommand{\stab}{\operatorname{Stab}}
\newcommand{\paren}[1]{\left(#1\right)}
\newcommand{\croch}[1]{\left[ #1 \right]}
\newcommand{\Grdcroch}[1]{\Bigl[ #1 \Bigr]}
\newcommand{\grdcroch}[1]{\bigl[ #1 \bigr]}
\newcommand{\abs}[1]{\left\lvert #1 \right\rvert}
\newcommand{\limi}[3]{\lim_{#1\to #2}#3}
\newcommand{\pinf}{+\infty}
\newcommand{\minf}{-\infty}
%%%%%%%%%%%%%% ENSEMBLES %%%%%%%%%%%%%%%%%
\newcommand{\ensemblenombre}[1]{\mathbb{#1}}
\newcommand{\Nn}{\ensemblenombre{N}}
\newcommand{\Zz}{\ensemblenombre{Z}}
\newcommand{\Qq}{\ensemblenombre{Q}}
\newcommand{\Qqp}{\Qq^+}
\newcommand{\Rr}{\ensemblenombre{R}}
\newcommand{\Cc}{\ensemblenombre{C}}
\newcommand{\Nne}{\Nn^*}
\newcommand{\Zze}{\Zz^*}
\newcommand{\Zzn}{\Zz^-}
\newcommand{\Qqe}{\Qq^*}
\newcommand{\Rre}{\Rr^*}
\newcommand{\Rrp}{\Rr_+}
\newcommand{\Rrm}{\Rr_-}
\newcommand{\Rrep}{\Rr_+^*}
\newcommand{\Rrem}{\Rr_-^*}
\newcommand{\Cce}{\Cc^*}
%%%%%%%%%%%%%%  INTERVALLES %%%%%%%%%%%%%%%%%
\newcommand{\intff}[2]{\left[#1\;,\; #2\right]  }
\newcommand{\intof}[2]{\left]#1 \;, \;#2\right]  }
\newcommand{\intfo}[2]{\left[#1 \;,\; #2\right[  }
\newcommand{\intoo}[2]{\left]#1 \;,\; #2\right[  }

\providecommand{\newpar}{\\[\medskipamount]}

\newcommand{\annexessection}{%
  \newpage%
  \subsection*{Annexes}%
}

\providecommand{\lesson}[3]{%
  \title{#3}%
  \hypersetup{pdftitle={#2 : #3}}%
  \setcounter{section}{\numexpr #2 - 1}%
  \section{#3}%
  \fancyhead[R]{\truncate{0.73\textwidth}{#2 : #3}}%
}

\providecommand{\development}[3]{%
  \title{#3}%
  \hypersetup{pdftitle={#3}}%
  \section*{#3}%
  \fancyhead[R]{\truncate{0.73\textwidth}{#3}}%
}

\providecommand{\sheet}[3]{\development{#1}{#2}{#3}}

\providecommand{\ranking}[1]{%
  \title{Terminale #1}%
  \hypersetup{pdftitle={Terminale #1}}%
  \section*{Terminale #1}%
  \fancyhead[R]{\truncate{0.73\textwidth}{Terminale #1}}%
}

\providecommand{\summary}[1]{%
  \textit{#1}%
  \par%
  \medskip%
}

\tikzset{notestyleraw/.append style={inner sep=0pt, rounded corners=0pt, align=center}}

%\newcommand{\booklink}[1]{\website/bibliographie\##1}
\newcounter{reference}
\newcommand{\previousreference}{}
\providecommand{\reference}[2][]{%
  \needspace{20pt}%
  \notblank{#1}{
    \needspace{20pt}%
    \renewcommand{\previousreference}{#1}%
    \stepcounter{reference}%
    \label{reference-\previousreference-\thereference}%
  }{}%
  \todo[noline]{%
    \protect\vspace{20pt}%
    \protect\par%
    \protect\notblank{#1}{\cite{[\previousreference]}\\}{}%
    \protect\hyperref[reference-\previousreference-\thereference]{p. #2}%
  }%
}

\definecolor{devcolor}{HTML}{00695c}
\providecommand{\dev}[1]{%
  \reversemarginpar%
  \todo[noline]{
    \protect\vspace{20pt}%
    \protect\par%
    \bfseries\color{devcolor}\href{\website/developpements/#1}{[DEV]}
  }%
  \normalmarginpar%
}

% En-têtes :

\pagestyle{fancy}
\fancyhead[L]{\truncate{0.23\textwidth}{\thepage}}
\fancyfoot[C]{\scriptsize \href{\website}{\texttt{https://github.com/imbodj/SenCoursDeMaths}}}

% Couleurs :

\definecolor{property}{HTML}{ffeb3b}
\definecolor{proposition}{HTML}{ffc107}
\definecolor{lemma}{HTML}{ff9800}
\definecolor{theorem}{HTML}{f44336}
\definecolor{corollary}{HTML}{e91e63}
\definecolor{definition}{HTML}{673ab7}
\definecolor{notation}{HTML}{9c27b0}
\definecolor{example}{HTML}{00bcd4}
\definecolor{cexample}{HTML}{795548}
\definecolor{application}{HTML}{009688}
\definecolor{remark}{HTML}{3f51b5}
\definecolor{algorithm}{HTML}{607d8b}
%\definecolor{proof}{HTML}{e1f5fe}
\definecolor{exercice}{HTML}{e1f5fe}

% Théorèmes :

\theoremstyle{definition}
\newtheorem{theorem}{Théorème}

\newtheorem{property}[theorem]{Propriété}
\newtheorem{proposition}[theorem]{Proposition}
\newtheorem{lemma}[theorem]{Activité d'introduction}
\newtheorem{corollary}[theorem]{Conséquence}

\newtheorem{definition}[theorem]{Définition}
\newtheorem{notation}[theorem]{Notation}

\newtheorem{example}[theorem]{Exemple}
\newtheorem{cexample}[theorem]{Contre-exemple}
\newtheorem{application}[theorem]{Application}

\newtheorem{algorithm}[theorem]{Algorithme}
\newtheorem{exercice}[theorem]{Exercice}

\theoremstyle{remark}
\newtheorem{remark}[theorem]{Remarque}

\counterwithin*{theorem}{section}

\newcommand{\applystyletotheorem}[1]{
  \tcolorboxenvironment{#1}{
    enhanced,
    breakable,
    colback=#1!8!white,
    %right=0pt,
    %top=8pt,
    %bottom=8pt,
    boxrule=0pt,
    frame hidden,
    sharp corners,
    enhanced,borderline west={4pt}{0pt}{#1},
    %interior hidden,
    sharp corners,
    after=\par,
  }
}

\applystyletotheorem{property}
\applystyletotheorem{proposition}
\applystyletotheorem{lemma}
\applystyletotheorem{theorem}
\applystyletotheorem{corollary}
\applystyletotheorem{definition}
\applystyletotheorem{notation}
\applystyletotheorem{example}
\applystyletotheorem{cexample}
\applystyletotheorem{application}
\applystyletotheorem{remark}
%\applystyletotheorem{proof}
\applystyletotheorem{algorithm}
\applystyletotheorem{exercice}

% Environnements :

\NewEnviron{whitetabularx}[1]{%
  \renewcommand{\arraystretch}{2.5}
  \colorbox{white}{%
    \begin{tabularx}{\textwidth}{#1}%
      \BODY%
    \end{tabularx}%
  }%
}

% Maths :

\DeclareFontEncoding{FMS}{}{}
\DeclareFontSubstitution{FMS}{futm}{m}{n}
\DeclareFontEncoding{FMX}{}{}
\DeclareFontSubstitution{FMX}{futm}{m}{n}
\DeclareSymbolFont{fouriersymbols}{FMS}{futm}{m}{n}
\DeclareSymbolFont{fourierlargesymbols}{FMX}{futm}{m}{n}
\DeclareMathDelimiter{\VERT}{\mathord}{fouriersymbols}{152}{fourierlargesymbols}{147}

% Code :

\definecolor{greencode}{rgb}{0,0.6,0}
\definecolor{graycode}{rgb}{0.5,0.5,0.5}
\definecolor{mauvecode}{rgb}{0.58,0,0.82}
\definecolor{bluecode}{HTML}{1976d2}
\lstset{
  basicstyle=\footnotesize\ttfamily,
  breakatwhitespace=false,
  breaklines=true,
  %captionpos=b,
  commentstyle=\color{greencode},
  deletekeywords={...},
  escapeinside={\%*}{*)},
  extendedchars=true,
  frame=none,
  keepspaces=true,
  keywordstyle=\color{bluecode},
  language=Python,
  otherkeywords={*,...},
  numbers=left,
  numbersep=5pt,
  numberstyle=\tiny\color{graycode},
  rulecolor=\color{black},
  showspaces=false,
  showstringspaces=false,
  showtabs=false,
  stepnumber=2,
  stringstyle=\color{mauvecode},
  tabsize=2,
  %texcl=true,
  xleftmargin=10pt,
  %title=\lstname
}

\newcommand{\codedirectory}{}
\newcommand{\inputalgorithm}[1]{%
  \begin{algorithm}%
    \strut%
    \lstinputlisting{\codedirectory#1}%
  \end{algorithm}%
}




\begin{document}
	%<*content>
	\development{algebra}{primitives}{Primitives (TS2)}

 \summary{}
 
\begin{exercice}
Pour chacune des fonctions définies ci-dessous, trouver :
\begin{enumerate}
    \item le(s) plus grand(s) intervalle(s) $I$ sur le(s)quel(s)  elle admet des primitives.
    \item l'expression d'une primitive sur chaque intervalle.
\end{enumerate}

1) $f(x)=3x^2 - 2x + 5$

\bigskip
 2)$ f(x)= (2x+1)(x^2+x - 5)$ 
 
 \bigskip
  3) $f(x) = \dfrac{x}{(1-x^2)^2}$

\bigskip
4) $f(x) = \dfrac{1}{4x^2 -4x + 1}$ 
\bigskip

 5) $f(x) = \dfrac{1}{x^3 - 3x^2+3x - 1}$ 
 
 \bigskip
   6) $f(x) = \cos 2x\cos 3x$
\bigskip

7) $f(x) = \sin^3 x \cos^4 x $

\bigskip
 8) $f(x) = \cos x\sin 3x$

\end{exercice}

\begin{exercice}
Pour chaque fonction $f$, déterminer une primitive sur $I$, prenant la valeur $b$ en  $a$ :
\begin{align*}
1)\quad & f(x) = \dfrac{2}{(3-x)^3}, \quad I = ]-\infty,3], \quad a=0, \ b=4 
\\[0.3cm]
2)\quad & f(x) = \dfrac{\sin x}{\cos^3 x}, \quad I = ]0,\tfrac{\pi}{2}[, \quad a=\tfrac{\pi}{3}, \ b=1
\\[0.3cm]
3)\quad & f(x) = \dfrac{x}{\sqrt{1+x^2}}, \quad I =\mathbb{R}, \quad a=0, \ b=1
\\[0.3cm]
4)\quad & f(x) = \dfrac{x\sin x+\cos x}{x^2}, \quad I = ]-\infty, 0[, \quad a=0, \ b=1
\\[0.3cm]
5)\quad & f(x) = \dfrac{1+\sin x}{\cos^2 x}, \quad I = [0,  \tfrac{\pi}{2}[ \quad a=\tfrac{\pi}{4}, \ b=1
\end{align*}
\end{exercice}
\pagebreak
\begin{exercice}
Déterminer une primitive $F$ dans chacun des cas suivants :
\begin{align*}
1)\quad & f(x) = (x^2+x+1)\sqrt{x} 
& 8)\quad & f(x)= \dfrac{12}{(x-1)^2}\left(1 + \dfrac{3}{x-1}\right)^3
\\[0.3cm]
2)\quad & f(x)= (x^2+x+1)\sqrt{x+1} 
& 9)\quad & f(x)= 1+ \dfrac{1}{\tan^2 x}
\\[0.3cm]
3)\quad & f(x)= x\sqrt{3-x}
& 10)\quad & f(x)= \dfrac{\cos(\sqrt{x})}{\sqrt{x}}
\\[0.3cm]
4)\quad & f(x)= \dfrac{x+1}{(x^2+2x+3)\sqrt{x^2+2x+3}}
& 11)\quad & f(x)= \dfrac{x^2-2x}{(x-1)^2} 
\\[0.3cm]
5)\quad & f(x)= \sin(x^2+2x) + (2x^2+2x)\cos (x^2+2x)
& 12)\quad & f(x)= \dfrac{x^3+3x}{(x^2-1)^3} \; \left(Écrire\; f(x)= \frac{a}{(x-1)^3}+\frac{b}{(x+1)^3}\right) 
\\[0.3cm]
6)\quad & f(x)= \dfrac{1}{1+\cos x} 
& 13)\quad & f(x)= \dfrac{x^2+1}{(x^2-1)^2} \; \left(Écrire\; x^2+1= \frac{1}{2}\left((x+1)^2+(x-1)^2\right)\right)
\\[0.3cm]
7)\quad & f(x)=\sqrt{x+1} +\frac{1}{2}\dfrac{x}{\sqrt{x+1}}& 14)\quad & f(x)=\dfrac{3x^2+4x+4}{(x^2+2x)^2}\; \left(Écrire\; f(x)= \frac{a}{x^2}+\frac{b}{(x+2)^2}\right) 
\end{align*}

\end{exercice}
\begin{exercice}
On se propose de déterminer une primitive sur $\mathbb{R}$ des fonctions :
\[
f(x) = x\cos^2 x, \quad g(x) = x\sin^2 x
\]

\begin{enumerate}
    \item Déterminer une primitive de $f+g$.
    \item Linéariser $\cos^2 x -\sin^2 x$. En déduire  qu'il existe deux réels $ a$ et $ b$ tels que  la fonction $x \mapsto a \sin 2x + b \cos 2x$ soit une primitive  sur $ \mathbb{R} $ de $f - g$.
    \item Conclure.
\end{enumerate}


\end{exercice}
\begin{exercice}
 Soit la fonction $ f $  définie sur $  [0,\; +\infty[ $  par : \; $ f(x)=\dfrac{x^3}{(1+x^2)^3}+ \dfrac{x^2}{(1+x^3)^2\sqrt{1+x^3}} $.
\begin{enumerate}
\item Justifier que $ f $ admet des primitives sur $  [0,\; +\infty[ $.
\item Déterminer les réels $a $ et $ b$ tels que :\; $ \dfrac{x^3}{(1+x^2)^3}=\dfrac{ax}{(1+x^2)^2}+\dfrac{bx}{(1+x^2)^3} $.
\item En déduire la primitive $ F $ de $ f $  sur $  [0,\; +\infty[ $ qui s'annule en 0.
\end{enumerate}
\end{exercice}
\begin{exercice}
Soit la fonction $ g $  définie sur $ \mathbb{R} $  par : \\ $ g(x)=\paren{\cos 3x  +\cos x}\cos x $.
\begin{enumerate}
\item Déterminer les réels $a $ , $ b $, $ c $ et $ d$ tels que :\; $ g(x)=a+b\cos 2x +c\cos 4x +d\cos 6x$.
\item En déduire une primitive de $ g $  sur $\mathbb{R} $.  
\end{enumerate}
\end{exercice}



\begin{exercice}
Une entreprise modélise la température (en °C) d’un four en fonction du temps $t$ (en minutes) par la dérivée $T'(t) = 4t - 20$, valable pour $t \in [0, 10]$.\\
On sait  qu’à l’instant $t = 0$, la température est de $300^\circ$C.\\
 À quel instant la température est-elle minimale ? Quelle est cette température ?
  
\end{exercice}
	%</content>
\end{document}