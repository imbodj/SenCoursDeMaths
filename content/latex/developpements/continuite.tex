\input{../common}

\begin{document}
	%<*content>
	\development{algebra}{continuite}{Fonctions  continues et T.V.I}

 \summary{}
 
	\begin{exercice}
	Recopier et compléter les pointillés.
\begin{enumerate}
 \item Si la fonction $ u $ est ................  alors la fonction  $ \sqrt{u} $ est continue  sur l'intervalle $ I $.
\item 
 Si une fonction  $ f  $ est ........ sur un intervalle $[a, b]$  et si  ......... alors l'équation $f(x) = 0$  admet une solution unique sur
$[a, b]$. 
\item Si une fonction  $ f  $ est ........ sur l'intervalle $[2, 3]$  et si  ......... alors l'équation $f(x) = m$ admet  au moins une solution  sur
$[2, 3]$.
 \item   Dans un repère orthonormal la courbe d'une fonction bijective et celle de sa réciproque sont symétriques ..............
 
\end{enumerate}
\end{exercice}
\begin{exercice}

Les affirmations suivantes sont-elles vraies ou fausses ? On justifiera la réponse.
\begin{enumerate}
\item La fonction définie par $ h(x)=\dfrac{\cos\sqrt{x}-1}{x} $  admet un prolongement par continuité en $ 0 $.

\item L'image d'un intervalle par une fonction est un intervalle.
\item La fonction $ x\longmapsto \sqrt{x^2-1} $ est continue sur $ ]-\infty\; ; \;-1] $.
\item  Si f est continue sur un intervalle I et si $ m\in $ f(I) alors l'équation f(x) $ =m $ admet au moins une
solution dans I.
 \item Soit la fonction $ u $  telle que $ x-2\leq u(x)\leq x+3 $  pour tout $  x>1 $.  Alors $\limi{x}{+\infty}{\dfrac{u(x)}{\sqrt{x}}}=\pinf $.



\item  Si $ f $ est continue et monotone   sur un intervalle  $I $  alors elle réalise une bijection de $ I $  vers $ f(I) $.
\end{enumerate}
\end{exercice}
\begin{exercice}

Montrer  que  la fonction $ f $  définie par :  $\; f(x)=\dfrac{x+2}{x} -(x-1)\sqrt{x+1}\;$ est continue sur  \; $  \; ]0,\; +\infty[ $ puis justifier que l'équation $ f(x)=0 $ admet au moins une solution dans   $ \; ]0,\; +\infty[ $.
\end{exercice}
\begin{exercice}
\begin{enumerate}
\item  Soit  la fonction $ h $ définie par \; $ \; h(x)=\dfrac{\sqrt{x+1}-2}{x^2-2x-3} $.
\begin{enumerate}
\item Déterminer l'ensemble de définition de $ h $.
\item Étudier la limite de $ h $ en 3. 
\item Définir  le prolongement par continuité  de $h $  en 3.
\end{enumerate}
\item Soit $ g $ la fonction définie  sur $]-1; +\infty[$ par  $ \; g(x)=\dfrac{1}{\paren{\sqrt{x+1}+2}\paren{x+1}} $.
\begin{enumerate}
\item Montrer que $ g $ est continue sur \; $ \; ]-1; +\infty[$.
\item Montrer que l'équation $ g(x)=1 $ admet au moins une solution dans \; $ \; ]-0,7\;  ; \; -0,6[ $. 
\end{enumerate}
\end{enumerate}
\end{exercice}
\begin{exercice}
On considère la fonction $ f $ de $ \mathbb{R} \longrightarrow  \mathbb{R}  $ définie par:

	$f(x)= \left\{\begin{array}{l c l}
 3x-1+\dfrac{3x}{x-2}  &  \text{si}\; x > 1\\ 	 
  -x+\sqrt{1-x} & \text{si} \; x\leq 1 
\end{array}\right. $
	\begin{enumerate}
	\item Justifier que  l'ensemble de définition de $ f $ est $ \text{D}_f=\mathbb{R}\setminus\left\{2\right\} $.
	\item Étudier les limites de $ f $ aux  bornes de D$ _{f} $.
\item Étudier la continuité de  $ f $ en 1 puis sur les intervalles $ ]-\infty, \;1[ $ ,  $ ]1, 2[$  et  $ ]2, \:+\infty[$. 
	\item Étudier la nature des branches infinies de  la courbe $ \mathcal{C}_{f} $. 
	\item Préciser la position relative de $ \mathcal{C}_{f} $  par rapport à son asymptote oblique.
	\end{enumerate}
\end{exercice}

	%</content>
\end{document}
