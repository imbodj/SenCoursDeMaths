\input{../common}

\begin{document}
	%<*content>
	\development{analysis}{expo}{Exponentielle}

 \summary{}
 
	
\begin{exercice}
Simplifier au maximum les expressions suivantes.

\begin{enumerate}
\item $ \dfrac{(\eexp{2})^{5}}{\eexp{5}}$ 
\item $ \sqrt{\eexp{2}} \times\dfrac{1}{\eexp{-2}}$
\item $ \dfrac{\eexp{-3}\times \eexp{2}}{\eexp{7} }$

\item  $ \dfrac{1}{1+\eexp{}}-\dfrac{\eexp{-1}}{1+\eexp{-1}}$                                               

\end{enumerate}

  \end{exercice}
  
  \begin{exercice}
Simplifier chacune des expressions.

\begin{enumerate}
\item $ \eexp{-2x}\times\eexp{2x} $
\item $( \eexp{x})^{3}\times\eexp{-2x}$ 
\item $  \eexp{2x+1}\times\eexp{1-x} $
\item  $ \dfrac{\eexp{2x+3}}{\eexp{2x-1}} $                                               
\item $\dfrac{\eexp{x+2}}{\eexp{-x+2}}  $
\item $ \dfrac{\eexp{x}+\eexp{-x}}{\eexp{x}} $
\item $ \dfrac{\eexp{x}+\eexp{-x}}{\eexp{-x}} $
\item $\paren{\eexp{x}+\eexp{-x}}^{2}-1-\eexp{-2x}$
\item  $ \dfrac{\eexp{3x}+\eexp{x}}{\eexp{2x}+1} $ 
\end{enumerate}


  \end{exercice}
  
  
  \begin{exercice}
Démontrer les égalités suivantes.
\begin{enumerate}
\item $\paren{\eexp{x}+\eexp{-x}}^{2}-~\paren{\eexp{x}-\eexp{-x}}^{2}=2$.
\item $ \eexp{2x}-5\eexp{x}+4= \paren{\eexp{x}-1}\paren{\eexp{x}-4}$
\item $ \eexp{-x}-\eexp{-2x}=\frac{\eexp{x}-1}{\eexp{2x}} $
\item $ \dfrac{\eexp{x}-1}{\eexp{x}+1}= \dfrac{1-\eexp{-x}}{1+\eexp{-x}}$
\item $  \dfrac{\eexp{x}-\eexp{-x}}{\eexp{x}+\eexp{-x}} =\dfrac{\eexp{2x}-1}{\eexp{2x}+1}$
\end{enumerate}


  \end{exercice}
  
 
 
   \begin{exercice}
    Résoudre dans $ \Rr $  les équations suivantes.
  
\begin{enumerate}
\item $ \eexp{3x}=1 $
\item $ \eexp{-3x+5}=\eexp{-2x+5} $
\item $ \eexp{-3x+5}\times \eexp{x+1}=\eexp{5} $
\item $ \eexp{-3x+5}=\eexp{2} $
\item $ \eexp{x}+4=0 $
\item  $  \eexp{2x-1}=16 $ 
\item $ \eexp{-x}-\frac{1}{2}=0 $
\item $\eexp{\paren{x+1}\paren{2x+1}}=1 $  
\item $  \eexp{x^{2}+3x+4}=\eexp{2}$  
 \item $ \eexp{x^{2}+x+1}=\eexp{3x+5}$ 
 \item $ \eexp{x^{2}+1}\times \eexp{2x}=\eexp{2x^{2}}$
  
\end{enumerate}

  \end{exercice}
  
  \begin{exercice}
Résoudre dans $ \Rr $  les équations suivantes.
   
\begin{enumerate}
\item   $\paren{\eexp{x}-1}\paren{\eexp{x}-2}=0 $ 
\item $ \eexp{2x}-4\eexp{x}+3=0$  
\item $ \eexp{2x}-5\eexp{x}=14$  
\item  $ \eexp{2x}+\eexp{x}=2 $  
\item $ \eexp{x}+\eexp{-x}=2$  
\item $ \eexp{-x}+2\eexp{x}=3$ 
\item $ \eexp{-2x}+2\eexp{-x}-3=0$
\item $ \eexp{3x}+3\eexp{2x}+2\eexp{x}=0$ 
\item $ \eexp{3x}-\eexp{2x}-2\eexp{x}=0$ 

\end{enumerate}
 \end{exercice}
 
   \begin{exercice}
    Résoudre dans $ \Rr $  les inéquations suivantes.
   
\begin{enumerate}
\item $\eexp{2x}>3 $  
\item $ \eexp{-2x}\leq 3$  
\item  $ \eexp{-2x}\leq\eexp{x+5} $  
\item $ \eexp{x^{2}}>\eexp{9}$ 
\item  $ \eexp{x^{2}}>\eexp{3x+2}  $ 
\item   $\paren{\eexp{x}-1}\paren{\eexp{x}-2}>0 $ 
\item   $\paren{2\eexp{x}-4}\paren{\eexp{x}-3}<0 $ 
\item $\paren{\eexp{x}+5}\paren{\eexp{2x}-2}>0 $
\item $ \dfrac{\eexp{x}-3}{\eexp{x}-2}\geq0 $ 
\item $ \dfrac{\eexp{x}-3}{\eexp{x}-2}\geq2 $ 
\end{enumerate}

  \end{exercice}
  
  \begin{exercice}
Résoudre dans $ \Rr $  les inéquations suivantes.
  
\begin{enumerate}
\item $ \eexp{2x}-\eexp{x}<0$   
\item $ \eexp{2x}-5\eexp{x}-14<0$    
\item $ \eexp{2x}-3\eexp{x}+2\geq0$ 
\item $ \eexp{2x}-12\eexp{x}+20>0$ 
\item $ -\eexp{2x}+11\eexp{x}-30>0$ 
\item $ \eexp{2x}-25<0$ 
\end{enumerate}

 \end{exercice}

   \begin{exercice}
  On considère le polynôme $ P(x)=2x^{3}-3x^{2}-11x+6$.
\begin{enumerate}
\item Calculer  $ P(2) $  puis montrer que :\\ $ P(x)=(x+2)(x-3)(2x-1) $.
\item Résoudre dans $ \Rr $  l'équation $ P(x)=0 $.
\item En déduire les  solutions des équations  suivantes.
\begin{enumerate}
\item $ 2(\ln(x))^{3}-3(\ln(x))^{2}-11\ln(x)+6=0 $.
\item $ 2\eexp{3x}-3\eexp{2x}-11\eexp{x}+6=0 $.

\end{enumerate}
\end{enumerate}
 \end{exercice}
 
  \begin{exercice}
  On considère le polynôme $ P(x)=x^{3}-9x^{2}-x+9$.
\begin{enumerate}
\item Calculer  $ P(9) $  en déduire une factorisation de  $ P(x)$.
\item Résoudre dans $ \Rr $  l'équation $ P(x)=0 $.
\item Résoudre dans $ \Rr $  l'équation $ \dfrac{\eexp{2x}+9\eexp{-x}}{9\eexp{x}+1}=1$.
\end{enumerate}
 \end{exercice}
  
 \begin{exercice}

\begin{enumerate} 
\item Développer, réduire et ordonner  $\; P(x)=(x+2)(x-3)(x+1) $ 
\item Résoudre $ \Rr $  , $ P(x)=0 $  et  $ P(x)\leq0 $.
\item En déduire  la résolution des équations suivantes.
\begin{enumerate} 
\item $\paren{\ln x}^{3}-7\paren{\ln x}^{2}-6=0  $                                                   
 \item $ \eexp{3x}-7\eexp{x}-6=0 $.

\end{enumerate}
\end{enumerate}

  \end{exercice}
  \begin{exercice}

\begin{enumerate}
\item Résoudre dans $ \Rr $   l'équation $x^{3}-3x^{2}+2x=0  $ 
\item En déduire  la résolution des équations suivantes.
\begin{enumerate} 
\item $\paren{\ln x}^{3}-3\paren{\ln x}^{2}+2\ln x=0  $                                                   
\item $ \eexp{3x}-3\eexp{2x}+2\eexp{x}=0 $.

\end{enumerate}
\end{enumerate}

  \end{exercice}
  

  \begin{exercice}
 
  On considère le polynôme $ P(x)=ax^{3}+bx^{2}+14x+c$\; où $ a$, $ b $ et $ c $  sont des réels.
\begin{enumerate}
\item Déterminer  $ a$, $ b $ et $ c $ sachant que   $ P(0)=-20 $,  $ \;P(-2)=0 $ et $ P(-1)=-24 $
\item On pose $ P(x)=-2x^{3}+8x^{2}+14x-20$
\begin{enumerate}
\item Factoriser $  P(x) $.
\item Résoudre dans $ \Rr $  l'équation $ P(x)=0 $.
\item En déduire les  solutions des équations  suivantes.
\item $ -2\eexp{3x}+8\eexp{2x}+14\eexp{x}-20=0 $.
\item $ -2(\ln(x))^{3}+8(\ln(x))^{2}+14\ln(x)-20=0 $.
\end{enumerate}
\end{enumerate}
 \end{exercice}
 
  
  \begin{exercice}
Résoudre les systèmes d'équations suivants.

\begin{enumerate}
\item $  \left\{\begin{array}{l}2\eexp{x}-\eexp{y}=15  \\ \eexp{x}+2\eexp{y} =40 \end{array}\right.$
\item $  \left\{\begin{array}{l}2\eexp{x}-3\eexp{y}=-5  \\ 3\eexp{x}+4\eexp{y} =18 \end{array}\right.$
\item $  \left\{\begin{array}{l}5\eexp{-x}-3\eexp{-y}=3  \\ 7\eexp{-x}+6\eexp{-y} =11 \end{array}\right.$
\item $  \left\{\begin{array}{l}\eexp{x}\times\eexp{2y+2}=0  \\ \eexp{x+7}\times\eexp{y} = \eexp{}\end{array}\right.$
\item $  \left\{\begin{array}{l}\eexp{x-y}=12 \\ \eexp{x+y} =\frac{4}{3} \end{array}\right.$
\end{enumerate}

  \end{exercice}
  
  
  \begin{exercice}


\begin{enumerate}
\item Résoudre dans $ \Rr^{2} $: \;$  \left\{\begin{array}{l}2X-Y=  7\\ 3X+4Y =5 \end{array}\right.$
\item En déduire la résolution  dans $ \Rr^{2} $  des systèmes :

 $  \left\{\begin{array}{l}2\eexp{x}-\eexp{y}=7  \\ 3\eexp{x}+4\eexp{y} =5 \end{array}\right.
\hspace*{0.5cm}  \left\{\begin{array}{l}2\ln x -\ln y=7  \\ 3\ln x+4\ln x =5 \end{array}\right.$

\end{enumerate}

  \end{exercice}
  
  
   \begin{exercice}
Résoudre les systèmes d'équations suivants.

\begin{enumerate}
\item $  \left\{\begin{array}{l}\eexp{x}\times\eexp{y}=\eexp{2} \\ xy =-15 \end{array}\right.$
\item $  \left\{\begin{array}{l}\eexp{x}\times\eexp{y}=\eexp{10} \\ \ln x  +\ln y =\ln 21 \end{array}\right.$
\item $  \left\{\begin{array}{l}\eexp{x}-3\ln y=11 \\ 2\eexp{x}+\ln y =1 \end{array}\right.$
\item $  \left\{\begin{array}{l}\ln x+\ln 4=\ln 3-\ln y \\ \eexp{x}=\eexp{2-y}  \end{array}\right.$
\end{enumerate}

  \end{exercice}

  \begin{exercice}
 Calculer la dérivée de  $ f $  dans chaque cas.
 
\begin{enumerate}
\item $ f(x)=\eexp{x}-x-1 $ 
\item $ f(x)= (x+5)\eexp{x}$  
\item $f(x)= x^{2}\eexp{x} $  
\item $ f(x)= xe^{2x} $ 
\item  $f(x)= \eexp{x}-\eexp{-x}  $  
 \item $ f(x)= \eexp{x^{2}-6x}$ 
  \item $ f(x)= x\eexp{x^{2}-6x}$
   \item $ f(x)= \eexp{\frac{2}{x}}$ 
   \item  $f(x)=\eexp{\frac{x}{x+2}} $  
   \item $ f(x)=\dfrac{\eexp{x}+1}{\eexp{x}-2} $
   \item $ f(x)=\dfrac{\eexp{2x}}{\eexp{x}-2} $
   \item $ f(x)=\dfrac{\eexp{x}+\eexp{-x}}{\eexp{x}-\eexp{-x}} $
   \item $ f(x)= (4e^{x}-1)(e^{2x}+1) $
   \item $ f(x)= \dfrac{1+e^{x}}{1-2e^{x}} $ 
   \item $ f(x)= \dfrac{2e^{x}}{e^{x}-1} $
\end{enumerate}

 \end{exercice}
 
  \begin{exercice}
 
 
  Déterminer les limites suivantes.
 
  \begin{enumerate}
   \item $ \displaystyle\lim_{x \to \pinf} (e^{-x}+1)  $ \item $ \displaystyle\lim_{x \to 0}( e^{-x}+1)  $ \item $ \displaystyle\lim_{x \to \minf}( e^{-x}+1)  $ \item $ \displaystyle\lim_{x \to \pinf} xe^{x}  $ \item $ \displaystyle\lim_{x \to \minf}x e^{-x}  $ \item $\displaystyle \lim_{x \to \minf}x e^{x}  $ \item $ \displaystyle \lim_{x \to \minf} \dfrac{e^{2x}-1}{e^{x}+1}  $ \item $\displaystyle \lim_{x \to \pinf} \dfrac{1}{1+e^{x}} $\item $\displaystyle \lim_{x \to \pinf}\paren{ e^{x} +\dfrac{1}{x}} $ \item $\displaystyle \lim_{x \to \pinf} \dfrac{1}{1+e^{-2x}}  $ 
 \end{enumerate}

 \end{exercice}
 
 
 \begin{exercice}
 
 Déterminer les limites suivantes.

 \begin{enumerate}
 \item   $ \displaystyle\lim_{x \to \pinf} (e^{2x}-e^{x} +1) $ \item   $\displaystyle \lim_{x \to \pinf} \dfrac{e^{x}+1}{2x} $  \item  $ \displaystyle\lim_{x \to \minf}( x^{2}-4x +1)e^{x} $  \item  $\displaystyle \lim_{x \to \pinf}(e^{x}  -x^{2}-x) $ \item  $\displaystyle \lim_{x \to 0} \dfrac{1-3e^{x}}{e^{x}-1} $
 \end{enumerate}

 \end{exercice}
  \subsection*{Etude de fonctions}
  \begin{exercice}

 Etudier les variations de $ f $  dans chaque cas.
 
\begin{enumerate}
\item $ f(x)=x\eexp{x} $ 
\item $ f(x)=\dfrac{x}{\eexp{x}} $
\item $ f(x)=x^{2}\eexp{x}$   
\item $f(x)= \dfrac{\eexp{x}}{x} $  
\item $f(x)= \eexp{x+2}$ 
\item $f(x)= \eexp{2x-2} $   
\item  $f(x)=  \eexp{-x^{2}} $
\item  $f(x)=  \eexp{x^{2}+2x+1} $   
 \item $ f(x)= \eexp{3-x}$ 
  \item $ f(x)= x-\eexp{x}$ 
  \item $ f(x)= x+\eexp{x}$ 
\end{enumerate}

 \end{exercice}
 
 \begin{exercice}
  Soit $ f(x)=x-\dfrac{\eexp{x}}{\eexp{x}+2} $,\; pour tout réel $ x. $
  \begin{enumerate}
  \item Calculer  les limites de $ f $  en $\pinf $ et $ \minf$.
  \item Montrer que $ f^{\prime} (x)=\dfrac{\eexp{2x}+1}{\paren{\eexp{x}+2}^{2}}$
  \item En déduire le tableau de variation de $ f. $
  \item Montrer que la droite (D) d'équation \; $ y=x $  est une asymptote à la courbe de $ f $ en $ \minf. $
  \item Montrer que la droite (D') d'équation \; $ y=x-1$  est une asymptote à la courbe de $ f $ en $ \pinf. $
  \end{enumerate}
 \end{exercice}
 
 \begin{exercice}

 On considère la fonction $ f $ définie par $ f(x)=\dfrac{2\eexp{2x}}{\eexp{2x}+1} $
 \begin{enumerate}
 \item Calculer  les limites de $ f $  en $\pinf $ et $ \minf$.
 \item Montrer que $ f^{\prime} (x)=\dfrac{4\eexp{2x}}{\paren{\eexp{2x}+1}^{2}}.\:$.\; En  déduire le tableau de variations de $ f. $
 \item Montrer que le point $I(0,1)$ est un centre de symétrie à la courbe de $ f. $
 \end{enumerate}
  \end{exercice}



 \begin{exercice}
Soit la fonction $ f $ définie par $ f(x)=\paren{x^{2}-5x+7}\eexp{x} $
 \begin{enumerate}
 \item Calculer  les limites de $ f $  en $\pinf $ et $ \minf$.
 \item Montrer que  le signe de  $ f^{\prime} (x)$  est celui de $ x^{2}-3x+2 $.
 \item Dresser le tableau de variations de $ f. $
 \item Que représente l'axe des abscisses pour la courbe de $ f $  ?
 \item Donner une équation de la tangente au point d'abscisse 0.
 \end{enumerate}
  \end{exercice}
  
  \begin{exercice}
Soit la fonction $ f $ définie par $ f(x)=\eexp{2x}-\eexp{-2x}-4x $
\begin{enumerate}
 \item Etudier la parité de $ f $
 \item
 Dans la suite on étudie $ f $  sur $ \intfo{0}{\pinf} $.
 \begin{enumerate}
 \item Calculer  la limite de $ f $  en $\pinf $.
 \item Montrer que   $ f^{\prime} (x)=2\paren{\eexp{x}-\eexp{-x}}^{2}$ .
 \item Dresser le tableau de variations de $ f. $
 \item Représenter la courbe de $ f $  
 
 \end{enumerate}
  \end{enumerate}
  \end{exercice}
  
  \begin{exercice}
 On considère la fonction $ f $ définie par $ f(x)=\dfrac{2}{\eexp{x}+2} $
 \begin{enumerate}
 \item Calculer  les limites de $ f $  en $\pinf $ et $ \minf$  et  préciser les asymptotes à la courbe de $ f. $
 \item Calculer $ f^{\prime} (x)$.\; En  déduire le tableau de variation de $ f. $
 \item Montrer que le point I$\paren{\ln 2,\frac{1}{2}}$ est un centre de symétrie à la courbe de $ f. $
 \item Donner une équation de la tangente au point d'abscisse 1.
 \item Déterminer les coordonnées du point A intersection de la courbe avec l'axe des ordonnées.
 \end{enumerate}
  \end{exercice}
\begin{exercice}
 On considère  la fonction $ f $  définie par : $ f(x)= xe^{-x} $
   \begin{enumerate}
   \item Déterminer le domaine de définition  E  de $ f $  puis les limites  aux bornes  de E.
   \item a. Montrer que   la fonction dérivée de $ f $ est telle que $ f'(x)=(1-x)e^{-x} $ .\\b. Dresser le tableau de variations de $ f $.
   
   \item  Déterminer la nature de la branche infinie de la courbe en $ \minf. $
   \item Tracer la courbe.
   \end{enumerate}
\end{exercice}

\begin{exercice}
Soit  la fonction $ f $  définie par : $ f(x)= (1-x)e^{x}+1 $
   \begin{enumerate}
   \item Déterminer le domaine de définition de $ f $  puis les limites  aux bornes  de ce domaine.\\
   Que peut-on en déduire pour la courbe de $ f $ ?
   \item Calculer  la fonction dérivée de $ f $, étudier son signe puis dresser son tableau de variations de $ f $.
   \item Déterminer  l'équation de la tangente au point où la courbe coupe l'axe des ordonnées.
   \item  Déterminer la nature de la branche infinie de la courbe en $ \minf. $
   \item Tracer la courbe.
   \end{enumerate}
\end{exercice}


\begin{exercice}

 Soit  la fonction $ f $  définie par : $ f(x)= e^{x^{2}-2x} $
   \begin{enumerate}
   \item Déterminer le domaine de définition de $ f $  puis les limites  aux bornes.
   \item Calculer  la fonction dérivée de $ f $, étudier son signe puis dresser le tableau de variations de $ f $.
   \item Déterminer  l'équation de la tangente aux points d'abscisse $0 $ et $2 $.
   \item Résoudre dans $ \Rr $ l'équation $ f(x)=1. $
   \end{enumerate}
\end{exercice}


\begin{exercice}
 Soit  la fonction $ f $  définie par :~~ $ f(x)= \dfrac{e^{x}-1}{e^{x}+1} $
   \begin{enumerate}
   \item Déterminer le domaine de définition de $ f $.\\ Calculer la limite de $ f $  en $ \minf $.\\
   En déduire l'équation d'une asymptote à la courbe en $ \minf $ \\ Calculer la limite de $ f $  en $ \pinf $.
   En déduire l'équation d'une  deuxième  asymptote à la courbe en $ \pinf $.
   \item Calculer  la fonction dérivée de $ f $, étudier son signe puis dresser le tableau de variations de $ f $.
   \item Déterminer  l'équation de la tangente au point où la courbe coupe l'axe des ordonnées.
   \item  Montrer que pour tout $ x $ réel ,  $ f(-x)=-f(x). $ 
   
   Que peut on en déduire pour la fonction $ f $  et pour sa courbe représentative ?

   \end{enumerate}
\end{exercice}


\begin{exercice}
 Soit la fonction $ f $ définie par : ~~ $ f(x)= \dfrac{2e^{x}-1}{e^{x}-1} $ 
   \begin{enumerate}
   \item Déterminer les réels $a $ et $b $ tels que $ f(x)= a+\dfrac{b}{e^{x}-1} $
   \item  Déterminer l'ensemble de définition de la fonction f et étudier les limites aux bornes de cet ensemble de définition. 
   \item
   \begin{enumerate}
   \item  Déterminer la dérivée $ f' $ de la fonction $ f $.
   \item Etudier le sens de variations de la fonction $ f $.
   \item Dresser le tableau de variations de la fonction $ f $.      
   \end{enumerate}
   \item  On appelle   $ \mathcal{C} $ la courbe représentative de $ f $ dans un repère orthonormé $ \oij $ (unité:2cm)
   \begin{enumerate}
   \item  Montrer que le point  ~~ A $\paren{0 ;\frac{3}{2} }$  ~~  est un centre de symétrie  pour $ \mathcal{C} $.
   \item Tracer $ \mathcal{C} $.
\end{enumerate}    
   \end{enumerate}
   
   
\end{exercice}


\begin{exercice}
 Soit la fonction $ f $ définie par : ~~ $ f(x)= x+\ln\paren{2-\eexp{x}} $ 
   \begin{enumerate}
   \item Résoudre l'inéquation $ 2-\eexp{x}> 0$.\\ En déduire le domaine de définition  $ Df $  de $ f. $.
   \item Etudier les limites $\minf $ et en  $ \ln 2$  de la fonction $ f $. 
   \item  Calculer   $ f' $   puis établir le tableau de variation de $ f. $
   \item
   \begin{enumerate}
   \item  Montrer que pour tout $ x\in Df $, \\ $ f(x)=x+\ln 2+\ln\paren{1-\dfrac{\eexp{x}}{2}}$.
   \item En déduire que la droite $ y= x+\ln 2 $  est une asymptote en $\minf $  à la courbe de $ f. $
   \item Préciser la position de la courbe par rapport à cette asymptote.    
   \end{enumerate}
   \item Tracer la courbe   $ \mathcal{C} $  représentative de $ f $ dans un repère orthonormé $ \oij $ (unité:1cm)
      
   \end{enumerate}
   
   
\end{exercice}

	%</content>
\end{document}
