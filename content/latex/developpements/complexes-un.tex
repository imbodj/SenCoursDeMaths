\input{../common}
\everymath{\displaystyle}
\begin{document}
  %<*content>
  \development{algebra}{complexes-un}{Nombres complexes}

  \summary{Initiation sur les nombres complexes}

\begin{exercice}
\begin{enumerate}
\item Rappeler la forme algébrique d'un nombre complexe $ z $.
\item Déterminer  la  forme algébrique les nombres complexes suivants.

\textbf{a)} $ \;\;z=(4+\i\sqrt{3})(1-\i)
\hspace*{0.5cm}\textbf{b)} \;\; z=\dfrac{\sqrt{2}+\i\sqrt{3}}{\sqrt{2}-\i\sqrt{3}} \hspace*{0.5cm}\textbf{c)} \;\; z=\dfrac{-1-2\i}{\paren{1+\i}^2} 
\hspace*{0.5cm}\textbf{d)} \;\; z=\dfrac{3-2\i}{2+\i}-\dfrac{\i+3}{1-\i} $
\end{enumerate}
\end{exercice}

\begin{exercice}
 Résoudre  dans  $ \mathbb{C} $ les équations suivantes. On donnera les solutions sous forme algébrique.
\begin{enumerate}
\item  $ 5\i z -3=-2z-5\i$
\item  $ \dfrac{\i z+1}{z-3\i}=2+\text{i}$

\item  $ (1-\text{i})\overline{z}=1 -3\text{i}$
\item  $ (z-\text{i})^2 +(z -3\text{i})^2=0$

\item $2z-\overline{z} =3-6\i$
\item $ (3+\i)z-(1-2\i)\overline{z} =1-2\i$
\end{enumerate}

\end{exercice}

\begin{exercice}
 On considère dans le plan complexe muni du repère orthonormal $ \ouv $, les points $A $ , $ B$ , $ C $  et $ D $ d'affixes respectives $ z_A=3+\i $,\;  $z_B=1+3\i $ \; , \;  $ z_C=1- \i $ \; et \; $ z_D=\overline{z_A} $.
 \begin{enumerate}
  \item Placer ces points dans le repère.
  \item À l'aide de nombres complexes calculer \; $ AB$, $AC  $ et $ BC$.

   \item En déduire la nature  du triangle $ ABC. $ 
   \item Le jardin potager de M. Mbaye  est formé du quadrilatère $ ABCD $ qu'il voudrait clôturer  par un fil barbelé en laissant une porte de $ 0.8 $ mètre. Le rouleau de 5m de
ce fil lui est vendu à 3500 FCFA. (\textit{On prendra dans cette question 1 m pour unité})


 Combien va-t-il dépenser pour clôturer son jardin. 

  
\end{enumerate}
\end{exercice}
\begin{exercice}
Soit $ z=x+\i y $ où $x $ et $y $ sont des réels et $ M $ son image.

Soit  $ Z=\dfrac{z+2-\i}{z-\i} $
 
 \begin{enumerate}
\item Écrire \textit{Re}$ (Z) $  et \textit{Im}$(Z) $ en fonction de $x $ et $y $ .
\item Déterminer l'ensemble des points $ M(z) $ tel que  $ Z $ soit réel.
\item Déterminer l'ensemble des points $ M(z) $ tel que  $ Z $ soit imaginaire pur.
\item Déterminer l'ensemble des points $ M(z) $ tel que  $ |Z|=1$.
\item Déterminer l'ensemble des points $ M(z) $ tel que  $ |Z|=2$.
\end{enumerate}
\end{exercice}
\begin{exercice}
Identifier la réponse juste et donner la justification.
\begin{enumerate}
\item Pour tout nombre complexe $ z $ et tout réel $ y $, le conjugué de $z + \i y$ est égale à :


\textbf{a)} $ z -\i y \hspace*{1cm}\textbf{b)} \overline{z} - \i y \hspace*{1cm} \textbf{c)} z - \i \overline{y} $
\item La partie imaginaire du  complexe $ z $ est égale à :


\textbf{a)} $ \dfrac{z+ \overline{z}}{2} \hspace*{1cm} \textbf{b)} \dfrac{z- \overline{z}}{2\i} \hspace*{1cm} \textbf{c)} \dfrac{z- \overline{z}}{2}$

\item   Le module  du  nombre complexe  $ z+\i$  est  égal :


\textbf{a/\;\;} $ \abs{z} +1 \hspace*{1cm} \textbf{b)} \sqrt{z^{2}+1} \hspace*{1cm} \textbf{c)} \abs{\i z-1} $

\medskip 
\item Le système $ \left\{\begin{array}{l c l}
(1-\i)z +\i z'=2-3\i \\[0.25cm] 	 
 (1+\i)z -(2+3\i) z'=3\i
\end{array}\right. $


\medskip 
 a pour ensemble solution dans $ \mathbb{C}^2 $
 
 
 \medskip 
 \textbf{a/\;\;} $ \paren{2+\i, -\i} \hspace*{1cm} \textbf{b)} \paren{2+\i, \;\i} \hspace*{1cm} \textbf{c)} \paren{2-\i, -\i} $ 

\end{enumerate}
\end{exercice}

\begin{exercice}
Le plan complexe est rapporté à un repère  orthonormé $ \ouv $. L'unité graphique est le mètre.
\medskip

$ \bullet $  \textbf{M.DIOP} a un terrain de forme rectangulaire  dont les dimensions $x$ et $y$ sont tels que:
\medskip

 $\hspace*{3cm} (2+3\i)z+(1-3\i)\overline{z} =6+3\i \qquad z=x+\i y $ 

\medskip

 Il voudrait construire sur ce terrain une école, et pour cela il a besoin de recouvrir  toute la superficie de ce terrain avec des carreaux. Le carton de carreaux coûte 14 000 FCFA et peut recouvrir une superficie de 5$m^2$.


\medskip


$ \bullet $ Le terrain que  \textbf{M.NDIAYE}  possède est situé en plein  quartier administratif  dont la forme est celle des points $ M $ d'affixes $ z\neq -1+2\i y $ tel que \; $ \dfrac{z-7+4\i}{z+1-2\i}$ soit un imaginaire pur. Il souhaite l'hypothéquer  avec une voiture dont la valeur est estimée à 1 170 000 FCFA. Sachant que son terrain a une valeur de $ 15000$  F CFA le mètre carré.

\medskip

\textbf{Votre travail en tant qu'élève de TS2, consiste à résoudre les tâches suivantes en justifiant votre démarche par un raisonnement bien détaillé}.

\medskip

\textbf{Tâches:}
\begin{enumerate}
\item Déterminer une estimation du montant nécessaire pour l'achat des carreaux devant recouvrir entièrement le terrain de  \textbf{M.DIOP}.
\item \textbf{M.NDIAYE}  réussira- t-il à être propriétaire de cette voiture? 
\end{enumerate}
\end{exercice}

  %</content>
\end{document}
