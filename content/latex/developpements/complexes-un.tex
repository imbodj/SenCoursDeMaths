\documentclass[12pt, a4paper]{report}

% LuaLaTeX :

\RequirePackage{iftex}
\RequireLuaTeX

% Packages :

\usepackage[french]{babel}
%\usepackage[utf8]{inputenc}
%\usepackage[T1]{fontenc}
\usepackage[pdfencoding=auto, pdfauthor={Hugo Delaunay}, pdfsubject={Mathématiques}, pdfcreator={agreg.skyost.eu}]{hyperref}
\usepackage{amsmath}
\usepackage{amsthm}
%\usepackage{amssymb}
\usepackage{stmaryrd}
\usepackage{tikz}
\usepackage{tkz-euclide}
\usepackage{fontspec}
\defaultfontfeatures[Erewhon]{FontFace = {bx}{n}{Erewhon-Bold.otf}}
\usepackage{fourier-otf}
\usepackage[nobottomtitles*]{titlesec}
\usepackage{fancyhdr}
\usepackage{listings}
\usepackage{catchfilebetweentags}
\usepackage[french, capitalise, noabbrev]{cleveref}
\usepackage[fit, breakall]{truncate}
\usepackage[top=2.5cm, right=2cm, bottom=2.5cm, left=2cm]{geometry}
\usepackage{enumitem}
\usepackage{tocloft}
\usepackage{microtype}
%\usepackage{mdframed}
%\usepackage{thmtools}
\usepackage{xcolor}
\usepackage{tabularx}
\usepackage{xltabular}
\usepackage{aligned-overset}
\usepackage[subpreambles=true]{standalone}
\usepackage{environ}
\usepackage[normalem]{ulem}
\usepackage{etoolbox}
\usepackage{setspace}
\usepackage[bibstyle=reading, citestyle=draft]{biblatex}
\usepackage{xpatch}
\usepackage[many, breakable]{tcolorbox}
\usepackage[backgroundcolor=white, bordercolor=white, textsize=scriptsize]{todonotes}
\usepackage{luacode}
\usepackage{float}
\usepackage{needspace}
\everymath{\displaystyle}

% Police :

\setmathfont{Erewhon Math}

% Tikz :

\usetikzlibrary{calc}
\usetikzlibrary{3d}

% Longueurs :

\setlength{\parindent}{0pt}
\setlength{\headheight}{15pt}
\setlength{\fboxsep}{0pt}
\titlespacing*{\chapter}{0pt}{-20pt}{10pt}
\setlength{\marginparwidth}{1.5cm}
\setstretch{1.1}

% Métadonnées :

\author{agreg.skyost.eu}
\date{\today}

% Titres :

\setcounter{secnumdepth}{3}

\renewcommand{\thechapter}{\Roman{chapter}}
\renewcommand{\thesubsection}{\Roman{subsection}}
\renewcommand{\thesubsubsection}{\arabic{subsubsection}}
\renewcommand{\theparagraph}{\alph{paragraph}}

\titleformat{\chapter}{\huge\bfseries}{\thechapter}{20pt}{\huge\bfseries}
\titleformat*{\section}{\LARGE\bfseries}
\titleformat{\subsection}{\Large\bfseries}{\thesubsection \, - \,}{0pt}{\Large\bfseries}
\titleformat{\subsubsection}{\large\bfseries}{\thesubsubsection. \,}{0pt}{\large\bfseries}
\titleformat{\paragraph}{\bfseries}{\theparagraph. \,}{0pt}{\bfseries}

\setcounter{secnumdepth}{4}

% Table des matières :

\renewcommand{\cftsecleader}{\cftdotfill{\cftdotsep}}
\addtolength{\cftsecnumwidth}{10pt}

% Redéfinition des commandes :

\renewcommand*\thesection{\arabic{section}}
\renewcommand{\ker}{\mathrm{Ker}}

% Nouvelles commandes :

\newcommand{\website}{https://github.com/imbodj/SenCoursDeMaths}

\newcommand{\tr}[1]{\mathstrut ^t #1}
\newcommand{\im}{\mathrm{Im}}
\newcommand{\rang}{\operatorname{rang}}
\newcommand{\trace}{\operatorname{trace}}
\newcommand{\id}{\operatorname{id}}
\newcommand{\stab}{\operatorname{Stab}}
\newcommand{\paren}[1]{\left(#1\right)}
\newcommand{\croch}[1]{\left[ #1 \right]}
\newcommand{\Grdcroch}[1]{\Bigl[ #1 \Bigr]}
\newcommand{\grdcroch}[1]{\bigl[ #1 \bigr]}
\newcommand{\abs}[1]{\left\lvert #1 \right\rvert}
\newcommand{\limi}[3]{\lim_{#1\to #2}#3}
\newcommand{\pinf}{+\infty}
\newcommand{\minf}{-\infty}
%%%%%%%%%%%%%% ENSEMBLES %%%%%%%%%%%%%%%%%
\newcommand{\ensemblenombre}[1]{\mathbb{#1}}
\newcommand{\Nn}{\ensemblenombre{N}}
\newcommand{\Zz}{\ensemblenombre{Z}}
\newcommand{\Qq}{\ensemblenombre{Q}}
\newcommand{\Qqp}{\Qq^+}
\newcommand{\Rr}{\ensemblenombre{R}}
\newcommand{\Cc}{\ensemblenombre{C}}
\newcommand{\Nne}{\Nn^*}
\newcommand{\Zze}{\Zz^*}
\newcommand{\Zzn}{\Zz^-}
\newcommand{\Qqe}{\Qq^*}
\newcommand{\Rre}{\Rr^*}
\newcommand{\Rrp}{\Rr_+}
\newcommand{\Rrm}{\Rr_-}
\newcommand{\Rrep}{\Rr_+^*}
\newcommand{\Rrem}{\Rr_-^*}
\newcommand{\Cce}{\Cc^*}
%%%%%%%%%%%%%%  INTERVALLES %%%%%%%%%%%%%%%%%
\newcommand{\intff}[2]{\left[#1\;,\; #2\right]  }
\newcommand{\intof}[2]{\left]#1 \;, \;#2\right]  }
\newcommand{\intfo}[2]{\left[#1 \;,\; #2\right[  }
\newcommand{\intoo}[2]{\left]#1 \;,\; #2\right[  }

\providecommand{\newpar}{\\[\medskipamount]}

\newcommand{\annexessection}{%
  \newpage%
  \subsection*{Annexes}%
}

\providecommand{\lesson}[3]{%
  \title{#3}%
  \hypersetup{pdftitle={#2 : #3}}%
  \setcounter{section}{\numexpr #2 - 1}%
  \section{#3}%
  \fancyhead[R]{\truncate{0.73\textwidth}{#2 : #3}}%
}

\providecommand{\development}[3]{%
  \title{#3}%
  \hypersetup{pdftitle={#3}}%
  \section*{#3}%
  \fancyhead[R]{\truncate{0.73\textwidth}{#3}}%
}

\providecommand{\sheet}[3]{\development{#1}{#2}{#3}}

\providecommand{\ranking}[1]{%
  \title{Terminale #1}%
  \hypersetup{pdftitle={Terminale #1}}%
  \section*{Terminale #1}%
  \fancyhead[R]{\truncate{0.73\textwidth}{Terminale #1}}%
}

\providecommand{\summary}[1]{%
  \textit{#1}%
  \par%
  \medskip%
}

\tikzset{notestyleraw/.append style={inner sep=0pt, rounded corners=0pt, align=center}}

%\newcommand{\booklink}[1]{\website/bibliographie\##1}
\newcounter{reference}
\newcommand{\previousreference}{}
\providecommand{\reference}[2][]{%
  \needspace{20pt}%
  \notblank{#1}{
    \needspace{20pt}%
    \renewcommand{\previousreference}{#1}%
    \stepcounter{reference}%
    \label{reference-\previousreference-\thereference}%
  }{}%
  \todo[noline]{%
    \protect\vspace{20pt}%
    \protect\par%
    \protect\notblank{#1}{\cite{[\previousreference]}\\}{}%
    \protect\hyperref[reference-\previousreference-\thereference]{p. #2}%
  }%
}

\definecolor{devcolor}{HTML}{00695c}
\providecommand{\dev}[1]{%
  \reversemarginpar%
  \todo[noline]{
    \protect\vspace{20pt}%
    \protect\par%
    \bfseries\color{devcolor}\href{\website/developpements/#1}{[DEV]}
  }%
  \normalmarginpar%
}

% En-têtes :

\pagestyle{fancy}
\fancyhead[L]{\truncate{0.23\textwidth}{\thepage}}
\fancyfoot[C]{\scriptsize \href{\website}{\texttt{https://github.com/imbodj/SenCoursDeMaths}}}

% Couleurs :

\definecolor{property}{HTML}{ffeb3b}
\definecolor{proposition}{HTML}{ffc107}
\definecolor{lemma}{HTML}{ff9800}
\definecolor{theorem}{HTML}{f44336}
\definecolor{corollary}{HTML}{e91e63}
\definecolor{definition}{HTML}{673ab7}
\definecolor{notation}{HTML}{9c27b0}
\definecolor{example}{HTML}{00bcd4}
\definecolor{cexample}{HTML}{795548}
\definecolor{application}{HTML}{009688}
\definecolor{remark}{HTML}{3f51b5}
\definecolor{algorithm}{HTML}{607d8b}
%\definecolor{proof}{HTML}{e1f5fe}
\definecolor{exercice}{HTML}{e1f5fe}

% Théorèmes :

\theoremstyle{definition}
\newtheorem{theorem}{Théorème}

\newtheorem{property}[theorem]{Propriété}
\newtheorem{proposition}[theorem]{Proposition}
\newtheorem{lemma}[theorem]{Activité d'introduction}
\newtheorem{corollary}[theorem]{Conséquence}

\newtheorem{definition}[theorem]{Définition}
\newtheorem{notation}[theorem]{Notation}

\newtheorem{example}[theorem]{Exemple}
\newtheorem{cexample}[theorem]{Contre-exemple}
\newtheorem{application}[theorem]{Application}

\newtheorem{algorithm}[theorem]{Algorithme}
\newtheorem{exercice}[theorem]{Exercice}

\theoremstyle{remark}
\newtheorem{remark}[theorem]{Remarque}

\counterwithin*{theorem}{section}

\newcommand{\applystyletotheorem}[1]{
  \tcolorboxenvironment{#1}{
    enhanced,
    breakable,
    colback=#1!8!white,
    %right=0pt,
    %top=8pt,
    %bottom=8pt,
    boxrule=0pt,
    frame hidden,
    sharp corners,
    enhanced,borderline west={4pt}{0pt}{#1},
    %interior hidden,
    sharp corners,
    after=\par,
  }
}

\applystyletotheorem{property}
\applystyletotheorem{proposition}
\applystyletotheorem{lemma}
\applystyletotheorem{theorem}
\applystyletotheorem{corollary}
\applystyletotheorem{definition}
\applystyletotheorem{notation}
\applystyletotheorem{example}
\applystyletotheorem{cexample}
\applystyletotheorem{application}
\applystyletotheorem{remark}
%\applystyletotheorem{proof}
\applystyletotheorem{algorithm}
\applystyletotheorem{exercice}

% Environnements :

\NewEnviron{whitetabularx}[1]{%
  \renewcommand{\arraystretch}{2.5}
  \colorbox{white}{%
    \begin{tabularx}{\textwidth}{#1}%
      \BODY%
    \end{tabularx}%
  }%
}

% Maths :

\DeclareFontEncoding{FMS}{}{}
\DeclareFontSubstitution{FMS}{futm}{m}{n}
\DeclareFontEncoding{FMX}{}{}
\DeclareFontSubstitution{FMX}{futm}{m}{n}
\DeclareSymbolFont{fouriersymbols}{FMS}{futm}{m}{n}
\DeclareSymbolFont{fourierlargesymbols}{FMX}{futm}{m}{n}
\DeclareMathDelimiter{\VERT}{\mathord}{fouriersymbols}{152}{fourierlargesymbols}{147}

% Code :

\definecolor{greencode}{rgb}{0,0.6,0}
\definecolor{graycode}{rgb}{0.5,0.5,0.5}
\definecolor{mauvecode}{rgb}{0.58,0,0.82}
\definecolor{bluecode}{HTML}{1976d2}
\lstset{
  basicstyle=\footnotesize\ttfamily,
  breakatwhitespace=false,
  breaklines=true,
  %captionpos=b,
  commentstyle=\color{greencode},
  deletekeywords={...},
  escapeinside={\%*}{*)},
  extendedchars=true,
  frame=none,
  keepspaces=true,
  keywordstyle=\color{bluecode},
  language=Python,
  otherkeywords={*,...},
  numbers=left,
  numbersep=5pt,
  numberstyle=\tiny\color{graycode},
  rulecolor=\color{black},
  showspaces=false,
  showstringspaces=false,
  showtabs=false,
  stepnumber=2,
  stringstyle=\color{mauvecode},
  tabsize=2,
  %texcl=true,
  xleftmargin=10pt,
  %title=\lstname
}

\newcommand{\codedirectory}{}
\newcommand{\inputalgorithm}[1]{%
  \begin{algorithm}%
    \strut%
    \lstinputlisting{\codedirectory#1}%
  \end{algorithm}%
}



\everymath{\displaystyle}
\begin{document}
  %<*content>
  \development{algebra}{complexes-un}{Nombres complexes}

  \summary{Initiation sur les nombres complexes}

\begin{exercice}
\begin{enumerate}
\item Rappeler la forme algébrique d'un nombre complexe $ z $.
\item Déterminer  la  forme algébrique les nombres complexes suivants.

\textbf{a)} $ \;\;z=(4+\i\sqrt{3})(1-\i)
\hspace*{0.5cm}\textbf{b)} \;\; z=\dfrac{\sqrt{2}+\i\sqrt{3}}{\sqrt{2}-\i\sqrt{3}} \hspace*{0.5cm}\textbf{c)} \;\; z=\dfrac{-1-2\i}{\paren{1+\i}^2} 
\hspace*{0.5cm}\textbf{d)} \;\; z=\dfrac{3-2\i}{2+\i}-\dfrac{\i+3}{1-\i} $
\end{enumerate}
\end{exercice}

\begin{exercice}
 Résoudre  dans  $ \mathbb{C} $ les équations suivantes. On donnera les solutions sous forme algébrique.
\begin{enumerate}
\item  $ 5\i z -3=-2z-5\i$
\item  $ \dfrac{\i z+1}{z-3\i}=2+\text{i}$

\item  $ (1-\text{i})\overline{z}=1 -3\text{i}$
\item  $ (z-\text{i})^2 +(z -3\text{i})^2=0$

\item $2z-\overline{z} =3-6\i$
\item $ (3+\i)z-(1-2\i)\overline{z} =1-2\i$
\end{enumerate}

\end{exercice}

\begin{exercice}
 On considère dans le plan complexe muni du repère orthonormal $ \ouv $, les points $A $ , $ B$ , $ C $  et $ D $ d'affixes respectives $ z_A=3+\i $,\;  $z_B=1+3\i $ \; , \;  $ z_C=1- \i $ \; et \; $ z_D=\overline{z_A} $.
 \begin{enumerate}
  \item Placer ces points dans le repère.
  \item À l'aide de nombres complexes calculer \; $ AB$, $AC  $ et $ BC$.

   \item En déduire la nature  du triangle $ ABC. $ 
   \item Le jardin potager de M. Mbaye  est formé du quadrilatère $ ABCD $ qu'il voudrait clôturer  par un fil barbelé en laissant une porte de $ 0.8 $ mètre. Le rouleau de 5m de
ce fil lui est vendu à 3500 FCFA. (\textit{On prendra dans cette question 1 m pour unité})


 Combien va-t-il dépenser pour clôturer son jardin. 

  
\end{enumerate}
\end{exercice}
\begin{exercice}
Soit $ z=x+\i y $ où $x $ et $y $ sont des réels et $ M $ son image.

Soit  $ Z=\dfrac{z+2-\i}{z-\i} $
 
 \begin{enumerate}
\item Écrire \textit{Re}$ (Z) $  et \textit{Im}$(Z) $ en fonction de $x $ et $y $ .
\item Déterminer l'ensemble des points $ M(z) $ tel que  $ Z $ soit réel.
\item Déterminer l'ensemble des points $ M(z) $ tel que  $ Z $ soit imaginaire pur.
\item Déterminer l'ensemble des points $ M(z) $ tel que  $ |Z|=1$.
\item Déterminer l'ensemble des points $ M(z) $ tel que  $ |Z|=2$.
\end{enumerate}
\end{exercice}
\begin{exercice}
Identifier la réponse juste et donner la justification.
\begin{enumerate}
\item Pour tout nombre complexe $ z $ et tout réel $ y $, le conjugué de $z + \i y$ est égale à :


\textbf{a)} $ z -\i y \hspace*{1cm}\textbf{b)} \overline{z} - \i y \hspace*{1cm} \textbf{c)} z - \i \overline{y} $
\item La partie imaginaire du  complexe $ z $ est égale à :


\textbf{a)} $ \dfrac{z+ \overline{z}}{2} \hspace*{1cm} \textbf{b)} \dfrac{z- \overline{z}}{2\i} \hspace*{1cm} \textbf{c)} \dfrac{z- \overline{z}}{2}$

\item   Le module  du  nombre complexe  $ z+\i$  est  égal :


\textbf{a/\;\;} $ \abs{z} +1 \hspace*{1cm} \textbf{b)} \sqrt{z^{2}+1} \hspace*{1cm} \textbf{c)} \abs{\i z-1} $

\medskip 
\item Le système $ \left\{\begin{array}{l c l}
(1-\i)z +\i z'=2-3\i \\[0.25cm] 	 
 (1+\i)z -(2+3\i) z'=3\i
\end{array}\right. $


\medskip 
 a pour ensemble solution dans $ \mathbb{C}^2 $
 
 
 \medskip 
 \textbf{a/\;\;} $ \paren{2+\i, -\i} \hspace*{1cm} \textbf{b)} \paren{2+\i, \;\i} \hspace*{1cm} \textbf{c)} \paren{2-\i, -\i} $ 

\end{enumerate}
\end{exercice}

\begin{exercice}
Le plan complexe est rapporté à un repère  orthonormé $ \ouv $. L'unité graphique est le mètre.
\medskip

$ \bullet $  \textbf{M.DIOP} a un terrain de forme rectangulaire  dont les dimensions $x$ et $y$ sont tels que:
\medskip

 $\hspace*{3cm} (2+3\i)z+(1-3\i)\overline{z} =6+3\i \qquad z=x+\i y $ 

\medskip

 Il voudrait construire sur ce terrain une école, et pour cela il a besoin de recouvrir  toute la superficie de ce terrain avec des carreaux. Le carton de carreaux coûte 14 000 FCFA et peut recouvrir une superficie de 5$m^2$.


\medskip


$ \bullet $ Le terrain que  \textbf{M.NDIAYE}  possède est situé en plein  quartier administratif  dont la forme est celle des points $ M $ d'affixes $ z\neq -1+2\i y $ tel que \; $ \dfrac{z-7+4\i}{z+1-2\i}$ soit un imaginaire pur. Il souhaite l'hypothéquer  avec une voiture dont la valeur est estimée à 1 170 000 FCFA. Sachant que son terrain a une valeur de $ 15000$  F CFA le mètre carré.

\medskip

\textbf{Votre travail en tant qu'élève de TS2, consiste à résoudre les tâches suivantes en justifiant votre démarche par un raisonnement bien détaillé}.

\medskip

\textbf{Tâches:}
\begin{enumerate}
\item Déterminer une estimation du montant nécessaire pour l'achat des carreaux devant recouvrir entièrement le terrain de  \textbf{M.DIOP}.
\item \textbf{M.NDIAYE}  réussira- t-il à être propriétaire de cette voiture? 
\end{enumerate}
\end{exercice}

  %</content>
\end{document}
