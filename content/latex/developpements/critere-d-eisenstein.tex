\documentclass[12pt, a4paper]{report}

% LuaLaTeX :

\RequirePackage{iftex}
\RequireLuaTeX

% Packages :

\usepackage[french]{babel}
%\usepackage[utf8]{inputenc}
%\usepackage[T1]{fontenc}
\usepackage[pdfencoding=auto, pdfauthor={Hugo Delaunay}, pdfsubject={Mathématiques}, pdfcreator={agreg.skyost.eu}]{hyperref}
\usepackage{amsmath}
\usepackage{amsthm}
%\usepackage{amssymb}
\usepackage{stmaryrd}
\usepackage{tikz}
\usepackage{tkz-euclide}
\usepackage{fontspec}
\defaultfontfeatures[Erewhon]{FontFace = {bx}{n}{Erewhon-Bold.otf}}
\usepackage{fourier-otf}
\usepackage[nobottomtitles*]{titlesec}
\usepackage{fancyhdr}
\usepackage{listings}
\usepackage{catchfilebetweentags}
\usepackage[french, capitalise, noabbrev]{cleveref}
\usepackage[fit, breakall]{truncate}
\usepackage[top=2.5cm, right=2cm, bottom=2.5cm, left=2cm]{geometry}
\usepackage{enumitem}
\usepackage{tocloft}
\usepackage{microtype}
%\usepackage{mdframed}
%\usepackage{thmtools}
\usepackage{xcolor}
\usepackage{tabularx}
\usepackage{xltabular}
\usepackage{aligned-overset}
\usepackage[subpreambles=true]{standalone}
\usepackage{environ}
\usepackage[normalem]{ulem}
\usepackage{etoolbox}
\usepackage{setspace}
\usepackage[bibstyle=reading, citestyle=draft]{biblatex}
\usepackage{xpatch}
\usepackage[many, breakable]{tcolorbox}
\usepackage[backgroundcolor=white, bordercolor=white, textsize=scriptsize]{todonotes}
\usepackage{luacode}
\usepackage{float}
\usepackage{needspace}
\everymath{\displaystyle}

% Police :

\setmathfont{Erewhon Math}

% Tikz :

\usetikzlibrary{calc}
\usetikzlibrary{3d}

% Longueurs :

\setlength{\parindent}{0pt}
\setlength{\headheight}{15pt}
\setlength{\fboxsep}{0pt}
\titlespacing*{\chapter}{0pt}{-20pt}{10pt}
\setlength{\marginparwidth}{1.5cm}
\setstretch{1.1}

% Métadonnées :

\author{agreg.skyost.eu}
\date{\today}

% Titres :

\setcounter{secnumdepth}{3}

\renewcommand{\thechapter}{\Roman{chapter}}
\renewcommand{\thesubsection}{\Roman{subsection}}
\renewcommand{\thesubsubsection}{\arabic{subsubsection}}
\renewcommand{\theparagraph}{\alph{paragraph}}

\titleformat{\chapter}{\huge\bfseries}{\thechapter}{20pt}{\huge\bfseries}
\titleformat*{\section}{\LARGE\bfseries}
\titleformat{\subsection}{\Large\bfseries}{\thesubsection \, - \,}{0pt}{\Large\bfseries}
\titleformat{\subsubsection}{\large\bfseries}{\thesubsubsection. \,}{0pt}{\large\bfseries}
\titleformat{\paragraph}{\bfseries}{\theparagraph. \,}{0pt}{\bfseries}

\setcounter{secnumdepth}{4}

% Table des matières :

\renewcommand{\cftsecleader}{\cftdotfill{\cftdotsep}}
\addtolength{\cftsecnumwidth}{10pt}

% Redéfinition des commandes :

\renewcommand*\thesection{\arabic{section}}
\renewcommand{\ker}{\mathrm{Ker}}

% Nouvelles commandes :

\newcommand{\website}{https://github.com/imbodj/SenCoursDeMaths}

\newcommand{\tr}[1]{\mathstrut ^t #1}
\newcommand{\im}{\mathrm{Im}}
\newcommand{\rang}{\operatorname{rang}}
\newcommand{\trace}{\operatorname{trace}}
\newcommand{\id}{\operatorname{id}}
\newcommand{\stab}{\operatorname{Stab}}
\newcommand{\paren}[1]{\left(#1\right)}
\newcommand{\croch}[1]{\left[ #1 \right]}
\newcommand{\Grdcroch}[1]{\Bigl[ #1 \Bigr]}
\newcommand{\grdcroch}[1]{\bigl[ #1 \bigr]}
\newcommand{\abs}[1]{\left\lvert #1 \right\rvert}
\newcommand{\limi}[3]{\lim_{#1\to #2}#3}
\newcommand{\pinf}{+\infty}
\newcommand{\minf}{-\infty}
%%%%%%%%%%%%%% ENSEMBLES %%%%%%%%%%%%%%%%%
\newcommand{\ensemblenombre}[1]{\mathbb{#1}}
\newcommand{\Nn}{\ensemblenombre{N}}
\newcommand{\Zz}{\ensemblenombre{Z}}
\newcommand{\Qq}{\ensemblenombre{Q}}
\newcommand{\Qqp}{\Qq^+}
\newcommand{\Rr}{\ensemblenombre{R}}
\newcommand{\Cc}{\ensemblenombre{C}}
\newcommand{\Nne}{\Nn^*}
\newcommand{\Zze}{\Zz^*}
\newcommand{\Zzn}{\Zz^-}
\newcommand{\Qqe}{\Qq^*}
\newcommand{\Rre}{\Rr^*}
\newcommand{\Rrp}{\Rr_+}
\newcommand{\Rrm}{\Rr_-}
\newcommand{\Rrep}{\Rr_+^*}
\newcommand{\Rrem}{\Rr_-^*}
\newcommand{\Cce}{\Cc^*}
%%%%%%%%%%%%%%  INTERVALLES %%%%%%%%%%%%%%%%%
\newcommand{\intff}[2]{\left[#1\;,\; #2\right]  }
\newcommand{\intof}[2]{\left]#1 \;, \;#2\right]  }
\newcommand{\intfo}[2]{\left[#1 \;,\; #2\right[  }
\newcommand{\intoo}[2]{\left]#1 \;,\; #2\right[  }

\providecommand{\newpar}{\\[\medskipamount]}

\newcommand{\annexessection}{%
  \newpage%
  \subsection*{Annexes}%
}

\providecommand{\lesson}[3]{%
  \title{#3}%
  \hypersetup{pdftitle={#2 : #3}}%
  \setcounter{section}{\numexpr #2 - 1}%
  \section{#3}%
  \fancyhead[R]{\truncate{0.73\textwidth}{#2 : #3}}%
}

\providecommand{\development}[3]{%
  \title{#3}%
  \hypersetup{pdftitle={#3}}%
  \section*{#3}%
  \fancyhead[R]{\truncate{0.73\textwidth}{#3}}%
}

\providecommand{\sheet}[3]{\development{#1}{#2}{#3}}

\providecommand{\ranking}[1]{%
  \title{Terminale #1}%
  \hypersetup{pdftitle={Terminale #1}}%
  \section*{Terminale #1}%
  \fancyhead[R]{\truncate{0.73\textwidth}{Terminale #1}}%
}

\providecommand{\summary}[1]{%
  \textit{#1}%
  \par%
  \medskip%
}

\tikzset{notestyleraw/.append style={inner sep=0pt, rounded corners=0pt, align=center}}

%\newcommand{\booklink}[1]{\website/bibliographie\##1}
\newcounter{reference}
\newcommand{\previousreference}{}
\providecommand{\reference}[2][]{%
  \needspace{20pt}%
  \notblank{#1}{
    \needspace{20pt}%
    \renewcommand{\previousreference}{#1}%
    \stepcounter{reference}%
    \label{reference-\previousreference-\thereference}%
  }{}%
  \todo[noline]{%
    \protect\vspace{20pt}%
    \protect\par%
    \protect\notblank{#1}{\cite{[\previousreference]}\\}{}%
    \protect\hyperref[reference-\previousreference-\thereference]{p. #2}%
  }%
}

\definecolor{devcolor}{HTML}{00695c}
\providecommand{\dev}[1]{%
  \reversemarginpar%
  \todo[noline]{
    \protect\vspace{20pt}%
    \protect\par%
    \bfseries\color{devcolor}\href{\website/developpements/#1}{[DEV]}
  }%
  \normalmarginpar%
}

% En-têtes :

\pagestyle{fancy}
\fancyhead[L]{\truncate{0.23\textwidth}{\thepage}}
\fancyfoot[C]{\scriptsize \href{\website}{\texttt{https://github.com/imbodj/SenCoursDeMaths}}}

% Couleurs :

\definecolor{property}{HTML}{ffeb3b}
\definecolor{proposition}{HTML}{ffc107}
\definecolor{lemma}{HTML}{ff9800}
\definecolor{theorem}{HTML}{f44336}
\definecolor{corollary}{HTML}{e91e63}
\definecolor{definition}{HTML}{673ab7}
\definecolor{notation}{HTML}{9c27b0}
\definecolor{example}{HTML}{00bcd4}
\definecolor{cexample}{HTML}{795548}
\definecolor{application}{HTML}{009688}
\definecolor{remark}{HTML}{3f51b5}
\definecolor{algorithm}{HTML}{607d8b}
%\definecolor{proof}{HTML}{e1f5fe}
\definecolor{exercice}{HTML}{e1f5fe}

% Théorèmes :

\theoremstyle{definition}
\newtheorem{theorem}{Théorème}

\newtheorem{property}[theorem]{Propriété}
\newtheorem{proposition}[theorem]{Proposition}
\newtheorem{lemma}[theorem]{Activité d'introduction}
\newtheorem{corollary}[theorem]{Conséquence}

\newtheorem{definition}[theorem]{Définition}
\newtheorem{notation}[theorem]{Notation}

\newtheorem{example}[theorem]{Exemple}
\newtheorem{cexample}[theorem]{Contre-exemple}
\newtheorem{application}[theorem]{Application}

\newtheorem{algorithm}[theorem]{Algorithme}
\newtheorem{exercice}[theorem]{Exercice}

\theoremstyle{remark}
\newtheorem{remark}[theorem]{Remarque}

\counterwithin*{theorem}{section}

\newcommand{\applystyletotheorem}[1]{
  \tcolorboxenvironment{#1}{
    enhanced,
    breakable,
    colback=#1!8!white,
    %right=0pt,
    %top=8pt,
    %bottom=8pt,
    boxrule=0pt,
    frame hidden,
    sharp corners,
    enhanced,borderline west={4pt}{0pt}{#1},
    %interior hidden,
    sharp corners,
    after=\par,
  }
}

\applystyletotheorem{property}
\applystyletotheorem{proposition}
\applystyletotheorem{lemma}
\applystyletotheorem{theorem}
\applystyletotheorem{corollary}
\applystyletotheorem{definition}
\applystyletotheorem{notation}
\applystyletotheorem{example}
\applystyletotheorem{cexample}
\applystyletotheorem{application}
\applystyletotheorem{remark}
%\applystyletotheorem{proof}
\applystyletotheorem{algorithm}
\applystyletotheorem{exercice}

% Environnements :

\NewEnviron{whitetabularx}[1]{%
  \renewcommand{\arraystretch}{2.5}
  \colorbox{white}{%
    \begin{tabularx}{\textwidth}{#1}%
      \BODY%
    \end{tabularx}%
  }%
}

% Maths :

\DeclareFontEncoding{FMS}{}{}
\DeclareFontSubstitution{FMS}{futm}{m}{n}
\DeclareFontEncoding{FMX}{}{}
\DeclareFontSubstitution{FMX}{futm}{m}{n}
\DeclareSymbolFont{fouriersymbols}{FMS}{futm}{m}{n}
\DeclareSymbolFont{fourierlargesymbols}{FMX}{futm}{m}{n}
\DeclareMathDelimiter{\VERT}{\mathord}{fouriersymbols}{152}{fourierlargesymbols}{147}

% Code :

\definecolor{greencode}{rgb}{0,0.6,0}
\definecolor{graycode}{rgb}{0.5,0.5,0.5}
\definecolor{mauvecode}{rgb}{0.58,0,0.82}
\definecolor{bluecode}{HTML}{1976d2}
\lstset{
  basicstyle=\footnotesize\ttfamily,
  breakatwhitespace=false,
  breaklines=true,
  %captionpos=b,
  commentstyle=\color{greencode},
  deletekeywords={...},
  escapeinside={\%*}{*)},
  extendedchars=true,
  frame=none,
  keepspaces=true,
  keywordstyle=\color{bluecode},
  language=Python,
  otherkeywords={*,...},
  numbers=left,
  numbersep=5pt,
  numberstyle=\tiny\color{graycode},
  rulecolor=\color{black},
  showspaces=false,
  showstringspaces=false,
  showtabs=false,
  stepnumber=2,
  stringstyle=\color{mauvecode},
  tabsize=2,
  %texcl=true,
  xleftmargin=10pt,
  %title=\lstname
}

\newcommand{\codedirectory}{}
\newcommand{\inputalgorithm}[1]{%
  \begin{algorithm}%
    \strut%
    \lstinputlisting{\codedirectory#1}%
  \end{algorithm}%
}




\begin{document}
  %<*content>
  \development{algebra}{critere-d-eisenstein}{Critère d'Eisenstein}

  \summary{Ici, nous démontrons le célèbre critère d'Eisenstein que l'on utilise énormément en pratique pour montrer qu'un polynôme est irréductible.}

  Soit $A$ un anneau commutatif et unitaire.

  \begin{notation}
    Soit $P \in A[X]$. On note $\gamma(P)$ le contenu du polynôme $P$.
  \end{notation}

  \reference[ULM18]{32}

  \begin{lemma}
    \label{critere-d-eisenstein-1}
    Soit $p \in A$ tel que $(p)$ est premier. Alors $A/(p)$ est intègre.
  \end{lemma}

  \begin{proof}
    Soient $\overline{a}, \overline{b} \in A/(p)$. On suppose $\overline{a} \overline{b} = 0$. Comme $\overline{a} \overline{b} = \overline{ab}$, on a $ab \in (p)$. Donc par hypothèse,
    \begin{align*}
      &a \in (p) \text{ ou } b \in (p) \\
      \implies& \overline{a} = 0 \text{ ou } \overline{b} = 0
    \end{align*}
    et ainsi $A/(p)$ est bien intègre.
  \end{proof}

  \reference{22}

  \begin{lemma}
    \label{critere-d-eisenstein-2}
    Si $A$ est intègre, alors $A[X]$ l'est aussi.
  \end{lemma}

  \begin{proof}
    Soient $P, Q \in A[X]$ non nuls, de degrés respectifs $n \geq 1$ et $m \geq 1$ que l'on écrit $P = \sum_{i=0}^n a_i X^i$ et $Q = \sum_{j=0}^m b_j X^j$. Alors, le coefficient de $X^{n+m}$ dans le produit $PQ$ est $a_n b_m$. Comme $a_n \neq 0$, $b_m \neq 0$ et $A$ est intègre, ce coefficient est non nul. Donc en particulier, le produit $PQ$ est non nul.
  \end{proof}

  \reference{64}

  \begin{lemma}
    \label{critere-d-eisenstein-3}
    On suppose $A$ factoriel. Soit $a \in A$ irréductible. Alors $(a)$ est premier.
  \end{lemma}

  \begin{proof}
    On suppose que $a \mid bc$ avec $b, c \in A$. Alors, il existe $d \in A$ tel que
    \[ ad = bc \tag{$*$} \]
    Si $b$ est inversible, alors $a \mid c$. De même, si $c$ est inversible, alors $a \mid b$. Supposons donc que $b$ et $c$ ne sont pas inversibles. Comme $a$ est irréductible, on en déduit que $d$ est un élément non nul et non inversible de $A$. Il existe donc des décompositions en irréductibles
    \[ b = \beta_1 \dots \beta_n, \, c = \gamma_1 \dots \gamma_m \text{ et } d = \delta_1 \dots \delta_k \]
    avec $n, m, k \in \mathbb{N}^*$. Par conséquent, en injectant dans $(*)$ :
    \[ a \delta_1 \dots \delta_k = \beta_1 \dots \beta_n \gamma_1 \dots \gamma_m \]
    Comme la factorisation en irréductibles est unique à l'ordre près, il existe $\beta_i$ ou $\gamma_j$ qui est associé à $a$. Si bien que $a$ divise $b$ ou $c$ ; c'est ce que l'on voulait démontrer.
  \end{proof}

  \reference[GOZ]{10}

  \begin{lemma}[Gauss]
    \label{critere-d-eisenstein-4}
    On suppose $A$ factoriel. Alors :
    \begin{enumerate}[label=(\roman*)]
      \item \label{critere-d-eisenstein-5} Le produit de deux polynômes primitifs est primitif.
      \item $\forall P, Q \in A[X] \setminus \{ 0 \}$, $\gamma(PQ) = \gamma(P) \gamma(Q)$.
    \end{enumerate}
  \end{lemma}

  \begin{proof}
    \begin{enumerate}[label=(\roman*)]
      \item Soient $P, Q \in A[X]$ tels que $\gamma(P) = \gamma(Q) = 1$. Supposons $\gamma(PQ) \neq 1$. Alors, il existe $p \in A$ irréductible tel que $p$ divise tous les coefficients de $PQ$. Donc, dans $A/(p)$, $\overline{PQ} = \overline{P} \, \overline{Q} = 0$.
      \newpar
      Mais, par le \cref{critere-d-eisenstein-3}, $(p)$ est premier. Donc par le \cref{critere-d-eisenstein-1} $A/(p)$ est intègre, et en particulier, $A/(p)[X]$ l'est aussi par le \cref{critere-d-eisenstein-2}. Ainsi, $\overline{P} = 0$ ou $\overline{Q} = 0$ : absurde.
      \item En factorisant, on écrit $P = \gamma(P)R$ et $Q = \gamma(Q)S$ où $R, S \in A[X]$ avec $\gamma(R) = \gamma(S) = 1$. D'où $PQ = \gamma(P)\gamma(Q)RS$ avec $\gamma(RS) = 1$ par le \cref{critere-d-eisenstein-5}. Ainsi, $\gamma(PQ) = \gamma(P) \gamma(Q).$
    \end{enumerate}
  \end{proof}

  \begin{theorem}[Critère d'Eisenstein]
    \label{critere-d-eisenstein-6}
    Soient $\mathbb{K}$ le corps des fractions de $A$ et $P = \sum_{i=0}^n a_i X^i \in A[X]$ de degré $n \geq 1$. On suppose que $A$ est factoriel et qu'il existe $p \in A$ irréductible tel que :
    \begin{enumerate}[label=(\roman*)]
      \item \label{critere-d-eisenstein-7} $p \mid a_i$, $\forall i \in \llbracket 0, n-1 \rrbracket$.
      \item \label{critere-d-eisenstein-8} $p \nmid a_n$.
      \item \label{critere-d-eisenstein-9} $p^2 \nmid a_0$.
    \end{enumerate}
    Alors $P$ est irréductible dans $\mathbb{K}[X]$.
  \end{theorem}

  \begin{proof}
    Par l'absurde, on suppose $P = UV$ avec $U, V \in \mathbb{K}[X]$ de degré supérieur ou égal à $1$. Soit $a$ un multiple commun à tous les dénominateurs des coefficients non nuls de $U$ et $V$. On a
    \[ a^2 P = \underbrace{a U}_{\substack{= U_1 \\ \in A[X]}} \underbrace{a V}_{\substack{= V_1 \\ \in A[X]}} \]
    On applique le \cref{critere-d-eisenstein-4} pour obtenir :
    \[ a^2 \gamma(P) = \gamma(U_1) \gamma(V_1) \tag{$*$} \]
    En factorisant, on écrit $U_1 = \gamma(U_1) U_2$ et $V_1 = \gamma(V_1) V_2$ avec $U_2, V_2 \in A[X]$. Il vient :
    \[ a^2 P = \gamma(U_1) \gamma(V_1) U_2 V_2 \overset{(*)}{=} a^2 \gamma(P) U_2 V_2 \]
    Et comme $a \in A \setminus \{ 0 \}$ et que $A$ est intègre, on a $P = \gamma(P) U_2 V_2 = U_3 V_3$ avec $U_3 = \gamma(P) U_2 \in A[X]$ et $V_3 = V_2 \in A[X]$ (dans un souci de symétrie des notations) qui sont de degré supérieur ou égal à $1$.
    \newpar
    On pose $U_3 = \sum_{i=0}^r b_i X^i$ et $V_3 = \sum_{j=0}^s c_j X^j$ avec $b_r c_s = a_n \neq 0$ par définition de $P$. Dans $A/(p)$, on a
    \[ \underbrace{\overline{P}}_{= \overline{a_n} X^n} = \overline{U_3 V_3} = \overline{U_3} \, \overline{V_3} \]
    et en particulier, le terme de degré $0$, $\overline{b_0 c_0} = \overline{b_0} \overline{c_0}$ est nul. Mais, $p$ est irréductible et $A$ est factoriel, donc au vu du \cref{critere-d-eisenstein-3}, $(p)$ est premier et $A/(p)$ est intègre par le \cref{critere-d-eisenstein-1}. Donc par le \cref{critere-d-eisenstein-2}, $A/(p)[X]$ est aussi intègre. D'où $\overline{b_0} = 0$ ou $\overline{c_0} = 0$ (mais pas les deux car sinon $p^2 \mid b_0 c_0 = a_0$, ce qui serait en contradiction avec le \cref{critere-d-eisenstein-9}).
    \newpar
    On suppose donc $\overline{b_0} = 0$ et $\overline{c_0} \neq 0$. Si on avait $\forall i \in \llbracket 0, r \rrbracket$, $\overline{b_i} = 0$, on aurait en particulier $\overline{b_r} = 0$, et donc $\overline{b_r} \overline{c_s} = \overline{a_n} = 0$ (exclu par le \cref{critere-d-eisenstein-8}). Donc,
    \[ \exists i \in \llbracket 0, r-1 \rrbracket \text{ tel que } \overline{b_0} = \dots = \overline{b_i} = 0 \text{ et } b_{i+1} \neq 0 \]
    Ainsi,
    \[ \overline{a_{i+1}} = \sum_{k=0}^{i+1} \overline{b_k} \overline{c_{i+1-k}} = \underbrace{\overline{b_{i+1}}}_{\neq 0} \underbrace{\overline{c_0}}_{\neq 0} \neq 0 \]
    ce qui est absurde au vu du \cref{critere-d-eisenstein-7} car $i \in \llbracket 0, r-1 \rrbracket$ avec $r-1 \leq n-1$.
  \end{proof}

  \reference[PER]{67}

  \begin{application}
    Soit $n \in \mathbb{N}^*$. Il existe des polynômes irréductibles de degré $n$ sur $\mathbb{Z}$.
  \end{application}

  \begin{proof}
    On applique le \cref{critere-d-eisenstein-6} au polynôme $P = X^n - 2$ avec le premier $p = 2$ qui nous donne l'irréductibilité du polynôme sur $\mathbb{Q}$. Reste à montrer qu'il est irréductible sur $\mathbb{Z}$.
    \newpar
    Or, en supposant $P$ réductible sur $\mathbb{Z}$, on peut écrire $P = QR$ avec $Q, R \in \mathbb{Z}[X]$ de degré supérieur ou égal à $1$ car $P$ est primitif. Mais à fortiori, $Q, R \in \mathbb{Q}[X]$ et ne sont pas inversibles donc $P$ est réductible sur $\mathbb{Q}$ : absurde.
  \end{proof}
  %</content>
\end{document}
