\documentclass[12pt, a4paper]{report}

% LuaLaTeX :

\RequirePackage{iftex}
\RequireLuaTeX

% Packages :

\usepackage[french]{babel}
%\usepackage[utf8]{inputenc}
%\usepackage[T1]{fontenc}
\usepackage[pdfencoding=auto, pdfauthor={Hugo Delaunay}, pdfsubject={Mathématiques}, pdfcreator={agreg.skyost.eu}]{hyperref}
\usepackage{amsmath}
\usepackage{amsthm}
%\usepackage{amssymb}
\usepackage{stmaryrd}
\usepackage{tikz}
\usepackage{tkz-euclide}
\usepackage{fontspec}
\defaultfontfeatures[Erewhon]{FontFace = {bx}{n}{Erewhon-Bold.otf}}
\usepackage{fourier-otf}
\usepackage[nobottomtitles*]{titlesec}
\usepackage{fancyhdr}
\usepackage{listings}
\usepackage{catchfilebetweentags}
\usepackage[french, capitalise, noabbrev]{cleveref}
\usepackage[fit, breakall]{truncate}
\usepackage[top=2.5cm, right=2cm, bottom=2.5cm, left=2cm]{geometry}
\usepackage{enumitem}
\usepackage{tocloft}
\usepackage{microtype}
%\usepackage{mdframed}
%\usepackage{thmtools}
\usepackage{xcolor}
\usepackage{tabularx}
\usepackage{xltabular}
\usepackage{aligned-overset}
\usepackage[subpreambles=true]{standalone}
\usepackage{environ}
\usepackage[normalem]{ulem}
\usepackage{etoolbox}
\usepackage{setspace}
\usepackage[bibstyle=reading, citestyle=draft]{biblatex}
\usepackage{xpatch}
\usepackage[many, breakable]{tcolorbox}
\usepackage[backgroundcolor=white, bordercolor=white, textsize=scriptsize]{todonotes}
\usepackage{luacode}
\usepackage{float}
\usepackage{needspace}
\everymath{\displaystyle}

% Police :

\setmathfont{Erewhon Math}

% Tikz :

\usetikzlibrary{calc}
\usetikzlibrary{3d}

% Longueurs :

\setlength{\parindent}{0pt}
\setlength{\headheight}{15pt}
\setlength{\fboxsep}{0pt}
\titlespacing*{\chapter}{0pt}{-20pt}{10pt}
\setlength{\marginparwidth}{1.5cm}
\setstretch{1.1}

% Métadonnées :

\author{agreg.skyost.eu}
\date{\today}

% Titres :

\setcounter{secnumdepth}{3}

\renewcommand{\thechapter}{\Roman{chapter}}
\renewcommand{\thesubsection}{\Roman{subsection}}
\renewcommand{\thesubsubsection}{\arabic{subsubsection}}
\renewcommand{\theparagraph}{\alph{paragraph}}

\titleformat{\chapter}{\huge\bfseries}{\thechapter}{20pt}{\huge\bfseries}
\titleformat*{\section}{\LARGE\bfseries}
\titleformat{\subsection}{\Large\bfseries}{\thesubsection \, - \,}{0pt}{\Large\bfseries}
\titleformat{\subsubsection}{\large\bfseries}{\thesubsubsection. \,}{0pt}{\large\bfseries}
\titleformat{\paragraph}{\bfseries}{\theparagraph. \,}{0pt}{\bfseries}

\setcounter{secnumdepth}{4}

% Table des matières :

\renewcommand{\cftsecleader}{\cftdotfill{\cftdotsep}}
\addtolength{\cftsecnumwidth}{10pt}

% Redéfinition des commandes :

\renewcommand*\thesection{\arabic{section}}
\renewcommand{\ker}{\mathrm{Ker}}

% Nouvelles commandes :

\newcommand{\website}{https://github.com/imbodj/SenCoursDeMaths}

\newcommand{\tr}[1]{\mathstrut ^t #1}
\newcommand{\im}{\mathrm{Im}}
\newcommand{\rang}{\operatorname{rang}}
\newcommand{\trace}{\operatorname{trace}}
\newcommand{\id}{\operatorname{id}}
\newcommand{\stab}{\operatorname{Stab}}
\newcommand{\paren}[1]{\left(#1\right)}
\newcommand{\croch}[1]{\left[ #1 \right]}
\newcommand{\Grdcroch}[1]{\Bigl[ #1 \Bigr]}
\newcommand{\grdcroch}[1]{\bigl[ #1 \bigr]}
\newcommand{\abs}[1]{\left\lvert #1 \right\rvert}
\newcommand{\limi}[3]{\lim_{#1\to #2}#3}
\newcommand{\pinf}{+\infty}
\newcommand{\minf}{-\infty}
%%%%%%%%%%%%%% ENSEMBLES %%%%%%%%%%%%%%%%%
\newcommand{\ensemblenombre}[1]{\mathbb{#1}}
\newcommand{\Nn}{\ensemblenombre{N}}
\newcommand{\Zz}{\ensemblenombre{Z}}
\newcommand{\Qq}{\ensemblenombre{Q}}
\newcommand{\Qqp}{\Qq^+}
\newcommand{\Rr}{\ensemblenombre{R}}
\newcommand{\Cc}{\ensemblenombre{C}}
\newcommand{\Nne}{\Nn^*}
\newcommand{\Zze}{\Zz^*}
\newcommand{\Zzn}{\Zz^-}
\newcommand{\Qqe}{\Qq^*}
\newcommand{\Rre}{\Rr^*}
\newcommand{\Rrp}{\Rr_+}
\newcommand{\Rrm}{\Rr_-}
\newcommand{\Rrep}{\Rr_+^*}
\newcommand{\Rrem}{\Rr_-^*}
\newcommand{\Cce}{\Cc^*}
%%%%%%%%%%%%%%  INTERVALLES %%%%%%%%%%%%%%%%%
\newcommand{\intff}[2]{\left[#1\;,\; #2\right]  }
\newcommand{\intof}[2]{\left]#1 \;, \;#2\right]  }
\newcommand{\intfo}[2]{\left[#1 \;,\; #2\right[  }
\newcommand{\intoo}[2]{\left]#1 \;,\; #2\right[  }

\providecommand{\newpar}{\\[\medskipamount]}

\newcommand{\annexessection}{%
  \newpage%
  \subsection*{Annexes}%
}

\providecommand{\lesson}[3]{%
  \title{#3}%
  \hypersetup{pdftitle={#2 : #3}}%
  \setcounter{section}{\numexpr #2 - 1}%
  \section{#3}%
  \fancyhead[R]{\truncate{0.73\textwidth}{#2 : #3}}%
}

\providecommand{\development}[3]{%
  \title{#3}%
  \hypersetup{pdftitle={#3}}%
  \section*{#3}%
  \fancyhead[R]{\truncate{0.73\textwidth}{#3}}%
}

\providecommand{\sheet}[3]{\development{#1}{#2}{#3}}

\providecommand{\ranking}[1]{%
  \title{Terminale #1}%
  \hypersetup{pdftitle={Terminale #1}}%
  \section*{Terminale #1}%
  \fancyhead[R]{\truncate{0.73\textwidth}{Terminale #1}}%
}

\providecommand{\summary}[1]{%
  \textit{#1}%
  \par%
  \medskip%
}

\tikzset{notestyleraw/.append style={inner sep=0pt, rounded corners=0pt, align=center}}

%\newcommand{\booklink}[1]{\website/bibliographie\##1}
\newcounter{reference}
\newcommand{\previousreference}{}
\providecommand{\reference}[2][]{%
  \needspace{20pt}%
  \notblank{#1}{
    \needspace{20pt}%
    \renewcommand{\previousreference}{#1}%
    \stepcounter{reference}%
    \label{reference-\previousreference-\thereference}%
  }{}%
  \todo[noline]{%
    \protect\vspace{20pt}%
    \protect\par%
    \protect\notblank{#1}{\cite{[\previousreference]}\\}{}%
    \protect\hyperref[reference-\previousreference-\thereference]{p. #2}%
  }%
}

\definecolor{devcolor}{HTML}{00695c}
\providecommand{\dev}[1]{%
  \reversemarginpar%
  \todo[noline]{
    \protect\vspace{20pt}%
    \protect\par%
    \bfseries\color{devcolor}\href{\website/developpements/#1}{[DEV]}
  }%
  \normalmarginpar%
}

% En-têtes :

\pagestyle{fancy}
\fancyhead[L]{\truncate{0.23\textwidth}{\thepage}}
\fancyfoot[C]{\scriptsize \href{\website}{\texttt{https://github.com/imbodj/SenCoursDeMaths}}}

% Couleurs :

\definecolor{property}{HTML}{ffeb3b}
\definecolor{proposition}{HTML}{ffc107}
\definecolor{lemma}{HTML}{ff9800}
\definecolor{theorem}{HTML}{f44336}
\definecolor{corollary}{HTML}{e91e63}
\definecolor{definition}{HTML}{673ab7}
\definecolor{notation}{HTML}{9c27b0}
\definecolor{example}{HTML}{00bcd4}
\definecolor{cexample}{HTML}{795548}
\definecolor{application}{HTML}{009688}
\definecolor{remark}{HTML}{3f51b5}
\definecolor{algorithm}{HTML}{607d8b}
%\definecolor{proof}{HTML}{e1f5fe}
\definecolor{exercice}{HTML}{e1f5fe}

% Théorèmes :

\theoremstyle{definition}
\newtheorem{theorem}{Théorème}

\newtheorem{property}[theorem]{Propriété}
\newtheorem{proposition}[theorem]{Proposition}
\newtheorem{lemma}[theorem]{Activité d'introduction}
\newtheorem{corollary}[theorem]{Conséquence}

\newtheorem{definition}[theorem]{Définition}
\newtheorem{notation}[theorem]{Notation}

\newtheorem{example}[theorem]{Exemple}
\newtheorem{cexample}[theorem]{Contre-exemple}
\newtheorem{application}[theorem]{Application}

\newtheorem{algorithm}[theorem]{Algorithme}
\newtheorem{exercice}[theorem]{Exercice}

\theoremstyle{remark}
\newtheorem{remark}[theorem]{Remarque}

\counterwithin*{theorem}{section}

\newcommand{\applystyletotheorem}[1]{
  \tcolorboxenvironment{#1}{
    enhanced,
    breakable,
    colback=#1!8!white,
    %right=0pt,
    %top=8pt,
    %bottom=8pt,
    boxrule=0pt,
    frame hidden,
    sharp corners,
    enhanced,borderline west={4pt}{0pt}{#1},
    %interior hidden,
    sharp corners,
    after=\par,
  }
}

\applystyletotheorem{property}
\applystyletotheorem{proposition}
\applystyletotheorem{lemma}
\applystyletotheorem{theorem}
\applystyletotheorem{corollary}
\applystyletotheorem{definition}
\applystyletotheorem{notation}
\applystyletotheorem{example}
\applystyletotheorem{cexample}
\applystyletotheorem{application}
\applystyletotheorem{remark}
%\applystyletotheorem{proof}
\applystyletotheorem{algorithm}
\applystyletotheorem{exercice}

% Environnements :

\NewEnviron{whitetabularx}[1]{%
  \renewcommand{\arraystretch}{2.5}
  \colorbox{white}{%
    \begin{tabularx}{\textwidth}{#1}%
      \BODY%
    \end{tabularx}%
  }%
}

% Maths :

\DeclareFontEncoding{FMS}{}{}
\DeclareFontSubstitution{FMS}{futm}{m}{n}
\DeclareFontEncoding{FMX}{}{}
\DeclareFontSubstitution{FMX}{futm}{m}{n}
\DeclareSymbolFont{fouriersymbols}{FMS}{futm}{m}{n}
\DeclareSymbolFont{fourierlargesymbols}{FMX}{futm}{m}{n}
\DeclareMathDelimiter{\VERT}{\mathord}{fouriersymbols}{152}{fourierlargesymbols}{147}

% Code :

\definecolor{greencode}{rgb}{0,0.6,0}
\definecolor{graycode}{rgb}{0.5,0.5,0.5}
\definecolor{mauvecode}{rgb}{0.58,0,0.82}
\definecolor{bluecode}{HTML}{1976d2}
\lstset{
  basicstyle=\footnotesize\ttfamily,
  breakatwhitespace=false,
  breaklines=true,
  %captionpos=b,
  commentstyle=\color{greencode},
  deletekeywords={...},
  escapeinside={\%*}{*)},
  extendedchars=true,
  frame=none,
  keepspaces=true,
  keywordstyle=\color{bluecode},
  language=Python,
  otherkeywords={*,...},
  numbers=left,
  numbersep=5pt,
  numberstyle=\tiny\color{graycode},
  rulecolor=\color{black},
  showspaces=false,
  showstringspaces=false,
  showtabs=false,
  stepnumber=2,
  stringstyle=\color{mauvecode},
  tabsize=2,
  %texcl=true,
  xleftmargin=10pt,
  %title=\lstname
}

\newcommand{\codedirectory}{}
\newcommand{\inputalgorithm}[1]{%
  \begin{algorithm}%
    \strut%
    \lstinputlisting{\codedirectory#1}%
  \end{algorithm}%
}




\begin{document}
	%<*content>
	\development{algebra}{exo-expo-ts2}{Fonction expo (TS2)}

 \summary{Initiation expo}
 
	
\begin{exercice}
\begin{enumerate}
\item Simplifier  au maximum les expressions suivantes:\medskip

 $A= \dfrac{\text{e}^{3+\ln x^{2}}}{2x} \quad $  $ B=\dfrac{\text{e}^{3x}+\text{e}^{x}}{\text{e}^{2x}+1}$ 
\item Prouver que pour tout réel $ x $:\medskip

  \textbf{a)} $\; \dfrac{\text{e}^{x}-\text{e}^{-x}}{\text{e}^{x}+\text{e}^{-x}}=\dfrac{\text{e}^{2x}-1}{\text{e}^{2x}+1}=\dfrac{1-\text{e}^{-2x}}{1+\text{e}^{-2x}}$ 

\medskip

 \textbf{b)} $\; \ln\Bigl(1+ \text{e}^{x}\Bigr)= x+\ln\Bigl(1+ \text{e}^{-x}\Bigr)$
\end{enumerate}
\end{exercice}

\begin{exercice}
On considère le polynôme $ P(x)=2x^{3}-9x^{2}+x+12$.
\begin{enumerate}
\item Résoudre dans $ \mathbb{R}:\quad $    $ P(x)\leq0 $.
\item En déduire les  solutions de l'équation et l' inéquation suivantes.

\medskip

\textbf{a)} $ \;2\text{e}^{3x}-9\text{e}^{2x}+\text{e}^{x}+12=0\hspace*{1cm} $ 
\textbf{b)} $\;\dfrac{\text{e}^{2x}\bigl(2\text{e}^{x}-9\bigr)}{\text{e}^{x}+12} \leq -1 $.
\end{enumerate}
\end{exercice}
\begin{exercice}
 Résoudre dans $ \mathbb{R}^{2} $  les systèmes suivants:

\medskip

\textbf{a)} $\; \left\{\begin{array}{l}  2\text{e}^{x} +3\text{e}^{1+y}=13 \\[0.25cm] \text{e}^{x}+\text{e}^{1+y}=5 \end{array}\right.\hspace*{0.5cm}$
\textbf{b)} $ \;  \left\{\begin{array}{l} xy=-15 \\[0.25cm]  \text{e}^{x}\times\text{e}^{y}=\text{e}^{2} \end{array}\right.$

\end{exercice}

\begin{exercice}
Calculer la dérivée de chacune des fonctions suivantes.

\medskip

1) $\; f(x)=(x^2-5x+1)\text{e}^{3x-1}\quad  $  2) $ f(x)=\sqrt{x}\text{e}^{-x^2} \quad
3) \; f(x)=\ln \bigl(1+\text{e}^{x}\bigr)\quad $

$   4) \; f(x)= \dfrac{\text{e}^{x}+\text{e}^{-x}} {\text{e}^{x}-\text{e}^{-x}} \quad
  5) \; f(x)= \ln\Bigl(\dfrac{\text{e}^{x}-1} {x+\text{e}^{x}}\Bigr)\qquad $  6) $ f(x)=\text{exp}\Bigr ( \dfrac{1}{x^2-x}\Bigl) $ 
\end{exercice}
\begin{exercice}
Etudier les limites suivantes:
\medskip

  1) $\displaystyle \lim_{x \to +\infty} {\bigl(\text{e}^{x}-x^2-x \bigr)}\qquad $ 2) $\displaystyle \lim_{x \to 0} \dfrac{1-\text{e}^{-2x}}{3x}   
   \hspace*{0.5cm}
    3) \; \displaystyle \lim_{x \to 0} \dfrac{\text{e}^{2x}-\text{e}^{-x}}{x}  \hspace*{0.5cm} 
   4) \; \displaystyle\lim_{x \to-\infty }\text{e}^{-2x}+3x  $ \medskip
 
 
  5) $ \displaystyle\lim_{x \to 0^{+} }\dfrac{\ln(2-\text{e}^{-x})}{x} \qquad $ 
   6) $ \displaystyle\lim_{|x| \to +\infty }(x+2)\text{e}^{-x}  $ 
\end{exercice}

\begin{exercice}
Dresser le tableau de variations  des fonctions suivantes.\medskip

1) $ \;f(x)=\text{e}^{2x}-\text{e}^{-2x}-4x \quad $  2) $\; f(x)= (x^2-5x+7)\text{e}^{2x} \quad
 3)  \;f(x)= x-\dfrac{\text{e}^{x}}{\text{e}^{x}+2}$   
 
4) $ \;f(x)=x- \text{e}^{\frac{x-2}{2} }\quad $    5) $\; f(x)= (x-1)\paren{2-\text{e}^{-x}}$
\end{exercice}



\begin{exercice} 
Soit  f$(x)=1-\dfrac{4\text{e}^{x}}{1+\text{e}^{2x}} \;$ et $\; \mathcal{C}\; $\ sa courbe. 
\begin{enumerate}
\item Démontrer que pour tout   $ x $, f$ (-x)=\text{f}(x) $. Que peut-on en déduire pour la courbe  $ \mathcal{C} $?
\item Déterminer $\displaystyle \lim_{x \to +\infty }{\text{f}(x)} $. 
Interpréter.
\item Vérifier que $ \forall x \in \mathbb{R}\;$\; f$ ^{\prime}(x)=\dfrac{4\text{e}^{x}\paren{\text{e}^{2x}-1}}{\paren{\text{e}^{2x}+1}^{2}} $
\item En déduire le sens de variation de la fonction f sur l'intervalle \;$ [0,\;+\infty[ $.
\item Montrer que la courbe \;$ \mathcal{C} $\; coupe  l'axe des abscisses en un unique point A \;d'abscisse \; $ a $\; positive.\ Montrer que \; $1,31 <  a < 1,32$. Donner une allure de   $ \mathcal{C} $ dans le repère.
\item Donner le signe de f$ (x) $  pour   $ x  \in \mathbb{R}$.
\end{enumerate}
\end{exercice}

\begin{exercice} 
 Soit   
 $\; f(x)= \left\{\begin{array}{l}  -x+7-4\text{e}^{x}\;\;\;\text{si}\;x \leq 0  \\[0.25cm] 
 x+3-x\ln x\;\;\;\text{si} \; x >0
\end{array}\right.$
 \begin{enumerate}
 \item
\begin{enumerate} 
\item Etudier la continuité de $ f $  en $ 0 $.
\item Etudier la dérivabilité de $ f $  en $ 0 \;$. Interpréter le résultat graphiquement. 
\item Ecrire l'équation de la tangente à $ \mathcal{C}_{f} $ au point d'abscisse $ \text{e} $.
\end{enumerate}
\item Déterminer les limites aux bornes de D$ f $.
\item Etudier les  branches infinies de $ \mathcal{C}_{f} $.
\item Établir le tableau de variations de $ f $. 
\item Démontrer que l'équation $ f(x)=0 $  admet une solution unique $ \alpha >0$.
\item Construire la courbe $ \mathcal{C}_{f} $. 
\end{enumerate}
\end{exercice}

\begin{exercice}
On considère la suite $\left(U_n\right)$ de nombres réels  définie pour tout entier naturel $n$ par : 
$\quad\left\{\begin{array}{l c l}
U_0		& =& 0\\
U_{n+1}& =& \ln\left(1+\text{e}^{U_n}\right)
\end{array}\right.$
 \begin{enumerate}
\item Calculer  $ U_3$.
\item Exprimer $ U_{n} $ en fonction de  $ U_{n+1} $.
\item Soit la suite  $\left(V_n\right)$ définie par: \; $\; V_n= \text{e}^{U_n}$, $\quad n\in\mathbb{N} $.
\begin{enumerate}
\item  Montrer que  $ ( V_n )$ est une suite arithmétique  dont on précisera la  raison et le premier terme.
\item Exprimer $  V_n $  puis  $  U_n $  en fonction de $ n. $ 
\item Etudier la convergence de la suite $ ( U_n) $ et préciser sa limite.
 \end{enumerate} 

  \end{enumerate}
\end{exercice}

	%</content>
\end{document}
