\documentclass[12pt, a4paper]{report}

% LuaLaTeX :

\RequirePackage{iftex}
\RequireLuaTeX

% Packages :

\usepackage[french]{babel}
%\usepackage[utf8]{inputenc}
%\usepackage[T1]{fontenc}
\usepackage[pdfencoding=auto, pdfauthor={Hugo Delaunay}, pdfsubject={Mathématiques}, pdfcreator={agreg.skyost.eu}]{hyperref}
\usepackage{amsmath}
\usepackage{amsthm}
%\usepackage{amssymb}
\usepackage{stmaryrd}
\usepackage{tikz}
\usepackage{tkz-euclide}
\usepackage{fontspec}
\defaultfontfeatures[Erewhon]{FontFace = {bx}{n}{Erewhon-Bold.otf}}
\usepackage{fourier-otf}
\usepackage[nobottomtitles*]{titlesec}
\usepackage{fancyhdr}
\usepackage{listings}
\usepackage{catchfilebetweentags}
\usepackage[french, capitalise, noabbrev]{cleveref}
\usepackage[fit, breakall]{truncate}
\usepackage[top=2.5cm, right=2cm, bottom=2.5cm, left=2cm]{geometry}
\usepackage{enumitem}
\usepackage{tocloft}
\usepackage{microtype}
%\usepackage{mdframed}
%\usepackage{thmtools}
\usepackage{xcolor}
\usepackage{tabularx}
\usepackage{xltabular}
\usepackage{aligned-overset}
\usepackage[subpreambles=true]{standalone}
\usepackage{environ}
\usepackage[normalem]{ulem}
\usepackage{etoolbox}
\usepackage{setspace}
\usepackage[bibstyle=reading, citestyle=draft]{biblatex}
\usepackage{xpatch}
\usepackage[many, breakable]{tcolorbox}
\usepackage[backgroundcolor=white, bordercolor=white, textsize=scriptsize]{todonotes}
\usepackage{luacode}
\usepackage{float}
\usepackage{needspace}
\everymath{\displaystyle}

% Police :

\setmathfont{Erewhon Math}

% Tikz :

\usetikzlibrary{calc}
\usetikzlibrary{3d}

% Longueurs :

\setlength{\parindent}{0pt}
\setlength{\headheight}{15pt}
\setlength{\fboxsep}{0pt}
\titlespacing*{\chapter}{0pt}{-20pt}{10pt}
\setlength{\marginparwidth}{1.5cm}
\setstretch{1.1}

% Métadonnées :

\author{agreg.skyost.eu}
\date{\today}

% Titres :

\setcounter{secnumdepth}{3}

\renewcommand{\thechapter}{\Roman{chapter}}
\renewcommand{\thesubsection}{\Roman{subsection}}
\renewcommand{\thesubsubsection}{\arabic{subsubsection}}
\renewcommand{\theparagraph}{\alph{paragraph}}

\titleformat{\chapter}{\huge\bfseries}{\thechapter}{20pt}{\huge\bfseries}
\titleformat*{\section}{\LARGE\bfseries}
\titleformat{\subsection}{\Large\bfseries}{\thesubsection \, - \,}{0pt}{\Large\bfseries}
\titleformat{\subsubsection}{\large\bfseries}{\thesubsubsection. \,}{0pt}{\large\bfseries}
\titleformat{\paragraph}{\bfseries}{\theparagraph. \,}{0pt}{\bfseries}

\setcounter{secnumdepth}{4}

% Table des matières :

\renewcommand{\cftsecleader}{\cftdotfill{\cftdotsep}}
\addtolength{\cftsecnumwidth}{10pt}

% Redéfinition des commandes :

\renewcommand*\thesection{\arabic{section}}
\renewcommand{\ker}{\mathrm{Ker}}

% Nouvelles commandes :

\newcommand{\website}{https://github.com/imbodj/SenCoursDeMaths}

\newcommand{\tr}[1]{\mathstrut ^t #1}
\newcommand{\im}{\mathrm{Im}}
\newcommand{\rang}{\operatorname{rang}}
\newcommand{\trace}{\operatorname{trace}}
\newcommand{\id}{\operatorname{id}}
\newcommand{\stab}{\operatorname{Stab}}
\newcommand{\paren}[1]{\left(#1\right)}
\newcommand{\croch}[1]{\left[ #1 \right]}
\newcommand{\Grdcroch}[1]{\Bigl[ #1 \Bigr]}
\newcommand{\grdcroch}[1]{\bigl[ #1 \bigr]}
\newcommand{\abs}[1]{\left\lvert #1 \right\rvert}
\newcommand{\limi}[3]{\lim_{#1\to #2}#3}
\newcommand{\pinf}{+\infty}
\newcommand{\minf}{-\infty}
%%%%%%%%%%%%%% ENSEMBLES %%%%%%%%%%%%%%%%%
\newcommand{\ensemblenombre}[1]{\mathbb{#1}}
\newcommand{\Nn}{\ensemblenombre{N}}
\newcommand{\Zz}{\ensemblenombre{Z}}
\newcommand{\Qq}{\ensemblenombre{Q}}
\newcommand{\Qqp}{\Qq^+}
\newcommand{\Rr}{\ensemblenombre{R}}
\newcommand{\Cc}{\ensemblenombre{C}}
\newcommand{\Nne}{\Nn^*}
\newcommand{\Zze}{\Zz^*}
\newcommand{\Zzn}{\Zz^-}
\newcommand{\Qqe}{\Qq^*}
\newcommand{\Rre}{\Rr^*}
\newcommand{\Rrp}{\Rr_+}
\newcommand{\Rrm}{\Rr_-}
\newcommand{\Rrep}{\Rr_+^*}
\newcommand{\Rrem}{\Rr_-^*}
\newcommand{\Cce}{\Cc^*}
%%%%%%%%%%%%%%  INTERVALLES %%%%%%%%%%%%%%%%%
\newcommand{\intff}[2]{\left[#1\;,\; #2\right]  }
\newcommand{\intof}[2]{\left]#1 \;, \;#2\right]  }
\newcommand{\intfo}[2]{\left[#1 \;,\; #2\right[  }
\newcommand{\intoo}[2]{\left]#1 \;,\; #2\right[  }

\providecommand{\newpar}{\\[\medskipamount]}

\newcommand{\annexessection}{%
  \newpage%
  \subsection*{Annexes}%
}

\providecommand{\lesson}[3]{%
  \title{#3}%
  \hypersetup{pdftitle={#2 : #3}}%
  \setcounter{section}{\numexpr #2 - 1}%
  \section{#3}%
  \fancyhead[R]{\truncate{0.73\textwidth}{#2 : #3}}%
}

\providecommand{\development}[3]{%
  \title{#3}%
  \hypersetup{pdftitle={#3}}%
  \section*{#3}%
  \fancyhead[R]{\truncate{0.73\textwidth}{#3}}%
}

\providecommand{\sheet}[3]{\development{#1}{#2}{#3}}

\providecommand{\ranking}[1]{%
  \title{Terminale #1}%
  \hypersetup{pdftitle={Terminale #1}}%
  \section*{Terminale #1}%
  \fancyhead[R]{\truncate{0.73\textwidth}{Terminale #1}}%
}

\providecommand{\summary}[1]{%
  \textit{#1}%
  \par%
  \medskip%
}

\tikzset{notestyleraw/.append style={inner sep=0pt, rounded corners=0pt, align=center}}

%\newcommand{\booklink}[1]{\website/bibliographie\##1}
\newcounter{reference}
\newcommand{\previousreference}{}
\providecommand{\reference}[2][]{%
  \needspace{20pt}%
  \notblank{#1}{
    \needspace{20pt}%
    \renewcommand{\previousreference}{#1}%
    \stepcounter{reference}%
    \label{reference-\previousreference-\thereference}%
  }{}%
  \todo[noline]{%
    \protect\vspace{20pt}%
    \protect\par%
    \protect\notblank{#1}{\cite{[\previousreference]}\\}{}%
    \protect\hyperref[reference-\previousreference-\thereference]{p. #2}%
  }%
}

\definecolor{devcolor}{HTML}{00695c}
\providecommand{\dev}[1]{%
  \reversemarginpar%
  \todo[noline]{
    \protect\vspace{20pt}%
    \protect\par%
    \bfseries\color{devcolor}\href{\website/developpements/#1}{[DEV]}
  }%
  \normalmarginpar%
}

% En-têtes :

\pagestyle{fancy}
\fancyhead[L]{\truncate{0.23\textwidth}{\thepage}}
\fancyfoot[C]{\scriptsize \href{\website}{\texttt{https://github.com/imbodj/SenCoursDeMaths}}}

% Couleurs :

\definecolor{property}{HTML}{ffeb3b}
\definecolor{proposition}{HTML}{ffc107}
\definecolor{lemma}{HTML}{ff9800}
\definecolor{theorem}{HTML}{f44336}
\definecolor{corollary}{HTML}{e91e63}
\definecolor{definition}{HTML}{673ab7}
\definecolor{notation}{HTML}{9c27b0}
\definecolor{example}{HTML}{00bcd4}
\definecolor{cexample}{HTML}{795548}
\definecolor{application}{HTML}{009688}
\definecolor{remark}{HTML}{3f51b5}
\definecolor{algorithm}{HTML}{607d8b}
%\definecolor{proof}{HTML}{e1f5fe}
\definecolor{exercice}{HTML}{e1f5fe}

% Théorèmes :

\theoremstyle{definition}
\newtheorem{theorem}{Théorème}

\newtheorem{property}[theorem]{Propriété}
\newtheorem{proposition}[theorem]{Proposition}
\newtheorem{lemma}[theorem]{Activité d'introduction}
\newtheorem{corollary}[theorem]{Conséquence}

\newtheorem{definition}[theorem]{Définition}
\newtheorem{notation}[theorem]{Notation}

\newtheorem{example}[theorem]{Exemple}
\newtheorem{cexample}[theorem]{Contre-exemple}
\newtheorem{application}[theorem]{Application}

\newtheorem{algorithm}[theorem]{Algorithme}
\newtheorem{exercice}[theorem]{Exercice}

\theoremstyle{remark}
\newtheorem{remark}[theorem]{Remarque}

\counterwithin*{theorem}{section}

\newcommand{\applystyletotheorem}[1]{
  \tcolorboxenvironment{#1}{
    enhanced,
    breakable,
    colback=#1!8!white,
    %right=0pt,
    %top=8pt,
    %bottom=8pt,
    boxrule=0pt,
    frame hidden,
    sharp corners,
    enhanced,borderline west={4pt}{0pt}{#1},
    %interior hidden,
    sharp corners,
    after=\par,
  }
}

\applystyletotheorem{property}
\applystyletotheorem{proposition}
\applystyletotheorem{lemma}
\applystyletotheorem{theorem}
\applystyletotheorem{corollary}
\applystyletotheorem{definition}
\applystyletotheorem{notation}
\applystyletotheorem{example}
\applystyletotheorem{cexample}
\applystyletotheorem{application}
\applystyletotheorem{remark}
%\applystyletotheorem{proof}
\applystyletotheorem{algorithm}
\applystyletotheorem{exercice}

% Environnements :

\NewEnviron{whitetabularx}[1]{%
  \renewcommand{\arraystretch}{2.5}
  \colorbox{white}{%
    \begin{tabularx}{\textwidth}{#1}%
      \BODY%
    \end{tabularx}%
  }%
}

% Maths :

\DeclareFontEncoding{FMS}{}{}
\DeclareFontSubstitution{FMS}{futm}{m}{n}
\DeclareFontEncoding{FMX}{}{}
\DeclareFontSubstitution{FMX}{futm}{m}{n}
\DeclareSymbolFont{fouriersymbols}{FMS}{futm}{m}{n}
\DeclareSymbolFont{fourierlargesymbols}{FMX}{futm}{m}{n}
\DeclareMathDelimiter{\VERT}{\mathord}{fouriersymbols}{152}{fourierlargesymbols}{147}

% Code :

\definecolor{greencode}{rgb}{0,0.6,0}
\definecolor{graycode}{rgb}{0.5,0.5,0.5}
\definecolor{mauvecode}{rgb}{0.58,0,0.82}
\definecolor{bluecode}{HTML}{1976d2}
\lstset{
  basicstyle=\footnotesize\ttfamily,
  breakatwhitespace=false,
  breaklines=true,
  %captionpos=b,
  commentstyle=\color{greencode},
  deletekeywords={...},
  escapeinside={\%*}{*)},
  extendedchars=true,
  frame=none,
  keepspaces=true,
  keywordstyle=\color{bluecode},
  language=Python,
  otherkeywords={*,...},
  numbers=left,
  numbersep=5pt,
  numberstyle=\tiny\color{graycode},
  rulecolor=\color{black},
  showspaces=false,
  showstringspaces=false,
  showtabs=false,
  stepnumber=2,
  stringstyle=\color{mauvecode},
  tabsize=2,
  %texcl=true,
  xleftmargin=10pt,
  %title=\lstname
}

\newcommand{\codedirectory}{}
\newcommand{\inputalgorithm}[1]{%
  \begin{algorithm}%
    \strut%
    \lstinputlisting{\codedirectory#1}%
  \end{algorithm}%
}




\begin{document}
	%<*content>
	\development{analysis}{polynomes}{Exercices sur les polynômes}

 \summary{}
 
	\begin{exercice}
Résoudre dans $ \Rr $ les équations suivantes:
\begin{enumerate}
\item $ -x^{2}-2x+3=0 $
\item $ x^{2}-2x=15 $
\item $ x(x+3)=x+1$
\item $ 4x^{2}-3x =0$
\item $ \paren{2x-1}\paren{-3x^{2}+12x -8}=0 $
\item $\paren{x^{2}-2}\paren{x^{2}+1}=0 $
\item $\dfrac{30}{x}+\dfrac{18}{x+3}=7 $
\end{enumerate}

\end{exercice}


\begin{exercice}
Résoudre dans $ \Rr $  les inéquations suivantes:
\begin{enumerate}
\item  $x^{2}-13x-48\leq 0 $
\item  $-x^{2}+13x+48\leq 0 $
\item  $-x^{2}+13x+48> 0 $
\item  $2x^{2}+2\sqrt{2}x+1> 0 $
\item  $x^{2}+2\sqrt{3}x+2< 0 $
\item  $x^{2}+x-2\geq 1 $
\item  $(x+1)(-x^{2}+x+6)> 0 $
\item  $(1-4x)(x^{2}+5x+4)>0 $
\item $ \dfrac{x^{2}-3x-4}{x-3}\geq0 $
\item  $\dfrac{x-1}{x+1}> 2x $
\end{enumerate}
\end{exercice}

\begin{exercice}
On considère le polynôme  suivant:\; $ P(x)=-2x^3+9x^2-7x-6 $.
\begin{enumerate}
\item 
\begin{enumerate}
\item Montrer que $ 2 $ est une racine de  $ P(x)$. 

\item En déduire que $ P(x) $ peut s'écrire sous la forme  \; $  P(x)=(x-2)(ax^2+bx+c)$ où $a $ , $b $ et $ c $ sont des réels à préciser.
\item Factoriser  $ P(x)$ en produit de facteurs de polynômes de premier degré.
\end{enumerate}

\item On suppose maintenant que:\; \; $ P(x)=(-2x-1)(x-2)(x-3) $.
\begin{enumerate}
\item Résoudre dans $ \mathbb{R} $ l'équation \; $ P(x)=0$. 
\item Résoudre dans $ \mathbb{R} $ l'inéquation\; $ P(x) <0 $. 
\end{enumerate}
\end{enumerate}
\end{exercice}

\begin{exercice}
On considère le  polynôme  suivant :  $ P(x)=3x^3+17x^2+9x-5 $.
\begin{enumerate}
\item Montrer que $P(x)$ est factorisable par $ x+1 $ puis l'écrire sous la forme   :\; $ P(x)=(x+1)Q(x) $ où  $Q(x)$ est un  polynôme à préciser.
\item Factoriser $Q(x)$.
\item En déduire que $P(x)=(3x-1)(x+5)(x+1)$.
\item Résoudre dans $ \mathbb{R} $ l'inéquation \; $ P(x)\leq 0 $.
\item Résoudre dans $ \mathbb{R} $ l'équation\; $ P(x)=0 $\;  puis\;  $ 3(2x-5)^3+17(2x-5)^2+9(2x-5)-5=0$. 
\end{enumerate}
\end{exercice}

\begin{exercice}
Soit le polynôme   $P(x) = 2x^4 + 5x^3 -5x^2 -5x + 3 $.
\begin{enumerate}
\item  Calculer  $ P(1) $ et $ P(-3) $. Que peut-on en déduire?
\item Montrer que : $ P(x)=(x-1)(2x^3+7x^2+2x-3)$.
\item On pose Q$(x)= 2x^3+7x^2+2x-3 $.
\begin{enumerate}
 \item Trouver trois réels $ a $, $ b$ et $c$  tels que : Q$(x)=(x+3)(ax^2+bx+c)$. 
\item En déduire une  factorisation  de $ P(x) $.
\end{enumerate} 

\item  Étudier dans $ \mathbb{R} $, le signe de    $P(x)  $.
\end{enumerate}

\end{exercice}
\begin{exercice}
On considère le polynôme  suivant:\; $ h(x)=4x^3+x^2-4x-1 $.

\begin{enumerate}
\item Vérifier que  $ 1 $ est une racine de  $ h(x)$. 
\item En déduire  une factorisation de  $ h(x)$  par la méthode de HORNER.
\item Soit\;  $ R(x)=\dfrac{(4x+1)(x+1)(x-1)}{4x^2-7x-2} $
\begin{enumerate}
\item Montrer que \; $(4x+1)(x-2)=4x^2-7x-2$. 
\item Simplifier \; $  R(x) $. 
\item  Étudier, suivant les valeurs du réel $ x $ ,le signe de $ R(x)$. 
\end{enumerate}
\end{enumerate}

\end{exercice}
\begin{exercice}
On considère le polynôme  suivant:\; $ P(x)=3x^4+14x^3-8x^2-14x+5 $.
\begin{enumerate}
\item Vérifier que  $ 1 $  et $ -5 $ sont des racines de  $ P(x)$. 
\item En utilisant  la méthode de HORNER, trouver le quotient $ Q(x)$ de la division de $ P(x)$ par $ (x-1) $.


\item Puis en utilisant  de nouveau  la méthode de HORNER, trouver le quotient $ Q'(x)$ de la division de $ Q(x)$ par $ (x+5) $.

\item Factoriser    $ Q'(x)$ puis $ P(x)$.
\item Résoudre dans $ \mathbb{R} $ l'inéquation \; $ P(x)\geq 0 $.

\item Soit\;  $ F(x)=\dfrac{(3x-1)(x^2-1)(x+5)}{x^2+x-2} $.


\begin{enumerate}
\item Montrer que \; $x^2+x-2=(x-1)(x+2)$. 


\item Simplifier \; $  F(x) $. 


\item   Étudier, suivant les valeurs du réel $ x $ ,le signe de $ F(x)$.  
\end{enumerate}
\end{enumerate}

\end{exercice}
\begin{exercice}

\begin{enumerate}
\item  
 Soit $ P(x)= x^3+bx^2+c x+d $ où $ b $, $c $ et $ d$ sont des réels.
 
  Sachant que $ P(1)=4 $,\;  $ P(-1)=-16 $ \;et \;  $ P(3)=0 $, déterminer les réels $ b $, $c $ et $ d$. 
\item On pose \; $ P(x)=x^3-6x^2+9x $
\begin{enumerate}
\item Factoriser $ P(x) $.
\item Etudier suivant les valeurs de $ x $ , le signe de $ P(x) $.
\end{enumerate}
\item Une entreprise vend un produit, et le profit réalisé en fonction du nombre de produits vendus 
$ x $ est donné par la fonction de profit suivante :
\; $P(x)=x^3-6x^2+9x$

Combien de produits l"entreprise doit-elle produire au moins pour réaliser un bénéfice ?
\end{enumerate}
\end{exercice}


	%</content>
\end{document}
