\documentclass[12pt, a4paper]{report}

% LuaLaTeX :

\RequirePackage{iftex}
\RequireLuaTeX

% Packages :

\usepackage[french]{babel}
%\usepackage[utf8]{inputenc}
%\usepackage[T1]{fontenc}
\usepackage[pdfencoding=auto, pdfauthor={Hugo Delaunay}, pdfsubject={Mathématiques}, pdfcreator={agreg.skyost.eu}]{hyperref}
\usepackage{amsmath}
\usepackage{amsthm}
%\usepackage{amssymb}
\usepackage{stmaryrd}
\usepackage{tikz}
\usepackage{tkz-euclide}
\usepackage{fontspec}
\defaultfontfeatures[Erewhon]{FontFace = {bx}{n}{Erewhon-Bold.otf}}
\usepackage{fourier-otf}
\usepackage[nobottomtitles*]{titlesec}
\usepackage{fancyhdr}
\usepackage{listings}
\usepackage{catchfilebetweentags}
\usepackage[french, capitalise, noabbrev]{cleveref}
\usepackage[fit, breakall]{truncate}
\usepackage[top=2.5cm, right=2cm, bottom=2.5cm, left=2cm]{geometry}
\usepackage{enumitem}
\usepackage{tocloft}
\usepackage{microtype}
%\usepackage{mdframed}
%\usepackage{thmtools}
\usepackage{xcolor}
\usepackage{tabularx}
\usepackage{xltabular}
\usepackage{aligned-overset}
\usepackage[subpreambles=true]{standalone}
\usepackage{environ}
\usepackage[normalem]{ulem}
\usepackage{etoolbox}
\usepackage{setspace}
\usepackage[bibstyle=reading, citestyle=draft]{biblatex}
\usepackage{xpatch}
\usepackage[many, breakable]{tcolorbox}
\usepackage[backgroundcolor=white, bordercolor=white, textsize=scriptsize]{todonotes}
\usepackage{luacode}
\usepackage{float}
\usepackage{needspace}
\everymath{\displaystyle}

% Police :

\setmathfont{Erewhon Math}

% Tikz :

\usetikzlibrary{calc}
\usetikzlibrary{3d}

% Longueurs :

\setlength{\parindent}{0pt}
\setlength{\headheight}{15pt}
\setlength{\fboxsep}{0pt}
\titlespacing*{\chapter}{0pt}{-20pt}{10pt}
\setlength{\marginparwidth}{1.5cm}
\setstretch{1.1}

% Métadonnées :

\author{agreg.skyost.eu}
\date{\today}

% Titres :

\setcounter{secnumdepth}{3}

\renewcommand{\thechapter}{\Roman{chapter}}
\renewcommand{\thesubsection}{\Roman{subsection}}
\renewcommand{\thesubsubsection}{\arabic{subsubsection}}
\renewcommand{\theparagraph}{\alph{paragraph}}

\titleformat{\chapter}{\huge\bfseries}{\thechapter}{20pt}{\huge\bfseries}
\titleformat*{\section}{\LARGE\bfseries}
\titleformat{\subsection}{\Large\bfseries}{\thesubsection \, - \,}{0pt}{\Large\bfseries}
\titleformat{\subsubsection}{\large\bfseries}{\thesubsubsection. \,}{0pt}{\large\bfseries}
\titleformat{\paragraph}{\bfseries}{\theparagraph. \,}{0pt}{\bfseries}

\setcounter{secnumdepth}{4}

% Table des matières :

\renewcommand{\cftsecleader}{\cftdotfill{\cftdotsep}}
\addtolength{\cftsecnumwidth}{10pt}

% Redéfinition des commandes :

\renewcommand*\thesection{\arabic{section}}
\renewcommand{\ker}{\mathrm{Ker}}

% Nouvelles commandes :

\newcommand{\website}{https://github.com/imbodj/SenCoursDeMaths}

\newcommand{\tr}[1]{\mathstrut ^t #1}
\newcommand{\im}{\mathrm{Im}}
\newcommand{\rang}{\operatorname{rang}}
\newcommand{\trace}{\operatorname{trace}}
\newcommand{\id}{\operatorname{id}}
\newcommand{\stab}{\operatorname{Stab}}
\newcommand{\paren}[1]{\left(#1\right)}
\newcommand{\croch}[1]{\left[ #1 \right]}
\newcommand{\Grdcroch}[1]{\Bigl[ #1 \Bigr]}
\newcommand{\grdcroch}[1]{\bigl[ #1 \bigr]}
\newcommand{\abs}[1]{\left\lvert #1 \right\rvert}
\newcommand{\limi}[3]{\lim_{#1\to #2}#3}
\newcommand{\pinf}{+\infty}
\newcommand{\minf}{-\infty}
%%%%%%%%%%%%%% ENSEMBLES %%%%%%%%%%%%%%%%%
\newcommand{\ensemblenombre}[1]{\mathbb{#1}}
\newcommand{\Nn}{\ensemblenombre{N}}
\newcommand{\Zz}{\ensemblenombre{Z}}
\newcommand{\Qq}{\ensemblenombre{Q}}
\newcommand{\Qqp}{\Qq^+}
\newcommand{\Rr}{\ensemblenombre{R}}
\newcommand{\Cc}{\ensemblenombre{C}}
\newcommand{\Nne}{\Nn^*}
\newcommand{\Zze}{\Zz^*}
\newcommand{\Zzn}{\Zz^-}
\newcommand{\Qqe}{\Qq^*}
\newcommand{\Rre}{\Rr^*}
\newcommand{\Rrp}{\Rr_+}
\newcommand{\Rrm}{\Rr_-}
\newcommand{\Rrep}{\Rr_+^*}
\newcommand{\Rrem}{\Rr_-^*}
\newcommand{\Cce}{\Cc^*}
%%%%%%%%%%%%%%  INTERVALLES %%%%%%%%%%%%%%%%%
\newcommand{\intff}[2]{\left[#1\;,\; #2\right]  }
\newcommand{\intof}[2]{\left]#1 \;, \;#2\right]  }
\newcommand{\intfo}[2]{\left[#1 \;,\; #2\right[  }
\newcommand{\intoo}[2]{\left]#1 \;,\; #2\right[  }

\providecommand{\newpar}{\\[\medskipamount]}

\newcommand{\annexessection}{%
  \newpage%
  \subsection*{Annexes}%
}

\providecommand{\lesson}[3]{%
  \title{#3}%
  \hypersetup{pdftitle={#2 : #3}}%
  \setcounter{section}{\numexpr #2 - 1}%
  \section{#3}%
  \fancyhead[R]{\truncate{0.73\textwidth}{#2 : #3}}%
}

\providecommand{\development}[3]{%
  \title{#3}%
  \hypersetup{pdftitle={#3}}%
  \section*{#3}%
  \fancyhead[R]{\truncate{0.73\textwidth}{#3}}%
}

\providecommand{\sheet}[3]{\development{#1}{#2}{#3}}

\providecommand{\ranking}[1]{%
  \title{Terminale #1}%
  \hypersetup{pdftitle={Terminale #1}}%
  \section*{Terminale #1}%
  \fancyhead[R]{\truncate{0.73\textwidth}{Terminale #1}}%
}

\providecommand{\summary}[1]{%
  \textit{#1}%
  \par%
  \medskip%
}

\tikzset{notestyleraw/.append style={inner sep=0pt, rounded corners=0pt, align=center}}

%\newcommand{\booklink}[1]{\website/bibliographie\##1}
\newcounter{reference}
\newcommand{\previousreference}{}
\providecommand{\reference}[2][]{%
  \needspace{20pt}%
  \notblank{#1}{
    \needspace{20pt}%
    \renewcommand{\previousreference}{#1}%
    \stepcounter{reference}%
    \label{reference-\previousreference-\thereference}%
  }{}%
  \todo[noline]{%
    \protect\vspace{20pt}%
    \protect\par%
    \protect\notblank{#1}{\cite{[\previousreference]}\\}{}%
    \protect\hyperref[reference-\previousreference-\thereference]{p. #2}%
  }%
}

\definecolor{devcolor}{HTML}{00695c}
\providecommand{\dev}[1]{%
  \reversemarginpar%
  \todo[noline]{
    \protect\vspace{20pt}%
    \protect\par%
    \bfseries\color{devcolor}\href{\website/developpements/#1}{[DEV]}
  }%
  \normalmarginpar%
}

% En-têtes :

\pagestyle{fancy}
\fancyhead[L]{\truncate{0.23\textwidth}{\thepage}}
\fancyfoot[C]{\scriptsize \href{\website}{\texttt{https://github.com/imbodj/SenCoursDeMaths}}}

% Couleurs :

\definecolor{property}{HTML}{ffeb3b}
\definecolor{proposition}{HTML}{ffc107}
\definecolor{lemma}{HTML}{ff9800}
\definecolor{theorem}{HTML}{f44336}
\definecolor{corollary}{HTML}{e91e63}
\definecolor{definition}{HTML}{673ab7}
\definecolor{notation}{HTML}{9c27b0}
\definecolor{example}{HTML}{00bcd4}
\definecolor{cexample}{HTML}{795548}
\definecolor{application}{HTML}{009688}
\definecolor{remark}{HTML}{3f51b5}
\definecolor{algorithm}{HTML}{607d8b}
%\definecolor{proof}{HTML}{e1f5fe}
\definecolor{exercice}{HTML}{e1f5fe}

% Théorèmes :

\theoremstyle{definition}
\newtheorem{theorem}{Théorème}

\newtheorem{property}[theorem]{Propriété}
\newtheorem{proposition}[theorem]{Proposition}
\newtheorem{lemma}[theorem]{Activité d'introduction}
\newtheorem{corollary}[theorem]{Conséquence}

\newtheorem{definition}[theorem]{Définition}
\newtheorem{notation}[theorem]{Notation}

\newtheorem{example}[theorem]{Exemple}
\newtheorem{cexample}[theorem]{Contre-exemple}
\newtheorem{application}[theorem]{Application}

\newtheorem{algorithm}[theorem]{Algorithme}
\newtheorem{exercice}[theorem]{Exercice}

\theoremstyle{remark}
\newtheorem{remark}[theorem]{Remarque}

\counterwithin*{theorem}{section}

\newcommand{\applystyletotheorem}[1]{
  \tcolorboxenvironment{#1}{
    enhanced,
    breakable,
    colback=#1!8!white,
    %right=0pt,
    %top=8pt,
    %bottom=8pt,
    boxrule=0pt,
    frame hidden,
    sharp corners,
    enhanced,borderline west={4pt}{0pt}{#1},
    %interior hidden,
    sharp corners,
    after=\par,
  }
}

\applystyletotheorem{property}
\applystyletotheorem{proposition}
\applystyletotheorem{lemma}
\applystyletotheorem{theorem}
\applystyletotheorem{corollary}
\applystyletotheorem{definition}
\applystyletotheorem{notation}
\applystyletotheorem{example}
\applystyletotheorem{cexample}
\applystyletotheorem{application}
\applystyletotheorem{remark}
%\applystyletotheorem{proof}
\applystyletotheorem{algorithm}
\applystyletotheorem{exercice}

% Environnements :

\NewEnviron{whitetabularx}[1]{%
  \renewcommand{\arraystretch}{2.5}
  \colorbox{white}{%
    \begin{tabularx}{\textwidth}{#1}%
      \BODY%
    \end{tabularx}%
  }%
}

% Maths :

\DeclareFontEncoding{FMS}{}{}
\DeclareFontSubstitution{FMS}{futm}{m}{n}
\DeclareFontEncoding{FMX}{}{}
\DeclareFontSubstitution{FMX}{futm}{m}{n}
\DeclareSymbolFont{fouriersymbols}{FMS}{futm}{m}{n}
\DeclareSymbolFont{fourierlargesymbols}{FMX}{futm}{m}{n}
\DeclareMathDelimiter{\VERT}{\mathord}{fouriersymbols}{152}{fourierlargesymbols}{147}

% Code :

\definecolor{greencode}{rgb}{0,0.6,0}
\definecolor{graycode}{rgb}{0.5,0.5,0.5}
\definecolor{mauvecode}{rgb}{0.58,0,0.82}
\definecolor{bluecode}{HTML}{1976d2}
\lstset{
  basicstyle=\footnotesize\ttfamily,
  breakatwhitespace=false,
  breaklines=true,
  %captionpos=b,
  commentstyle=\color{greencode},
  deletekeywords={...},
  escapeinside={\%*}{*)},
  extendedchars=true,
  frame=none,
  keepspaces=true,
  keywordstyle=\color{bluecode},
  language=Python,
  otherkeywords={*,...},
  numbers=left,
  numbersep=5pt,
  numberstyle=\tiny\color{graycode},
  rulecolor=\color{black},
  showspaces=false,
  showstringspaces=false,
  showtabs=false,
  stepnumber=2,
  stringstyle=\color{mauvecode},
  tabsize=2,
  %texcl=true,
  xleftmargin=10pt,
  %title=\lstname
}

\newcommand{\codedirectory}{}
\newcommand{\inputalgorithm}[1]{%
  \begin{algorithm}%
    \strut%
    \lstinputlisting{\codedirectory#1}%
  \end{algorithm}%
}




\begin{document}
  %<*content>
  \development{analysis}{theoreme-d-abel-angulaire}{Théorème d'Abel angulaire}

  \summary{On montre le théorème d'Abel ``angulaire'', qui permet d'intervertir certaines sommes et limites, et on l'applique justement au calcul de deux sommes.}

  \reference[GOU20]{263}

  \begin{theorem}[Abel angulaire]
    \label{theoreme-d-abel-angulaire-1}
    Soit $\sum a_n z^n$ une série entière de rayon de convergence supérieur ou égal à $1$ telle que $\sum a_n$ converge. On note $f$ la somme de cette série sur le disque unité $D$ de $\mathbb{C}$. On fixe $\theta_0 \in \left[ 0, \frac{\pi}{2} \right[$ et on pose $\Delta_{\theta_0} = \{ z \in D \mid \exists \rho > 0 \text{ et } \exists \theta \in [-\theta_0, \theta_0] \text{ tels que } z = 1 - \rho e^{i\theta} \}$.
    \begin{center}
      \begin{tikzpicture}
        \draw[->] (-3, 0) -- (3, 0) node[right] {$x$};
        \draw[->] (0, -3) -- (0, 3) node[above] {$y$};
        \draw (0,2) node {$\bullet$} node[above right]{$1$};
        \draw (2,0) node {$\bullet$} node[below right]{$1$};
        \draw (0,0) circle (2);
        \coordinate (A) at (130:3.5);
        \coordinate (B) at (230:3.5);
        \coordinate (C) at (2,0);
        \begin{scope}
          \path[clip] circle (2);
          \path[clip] (A) -- (B) -- (C) -- cycle;
          \draw [transparent, fill=blue!30, fill opacity=0.3] (C) circle (9);
        \end{scope}
        \begin{scope}
          \path[clip] (A) -- (180:3.5) -- (C) -- cycle;
          \draw (C) circle (1);
        \end{scope}
        \draw (0.7,0.35) node {$\theta_0$};
        \draw (C) -- (A);
        \draw (C) -- (B);
        \coordinate (S) at (210:3.5);
        \coordinate (E) at (-0.5,-0.5);
        \draw [->] (S) to [out=50] (E);
        \draw (212:3.7) node {$\Delta_{\theta_0}$};
      \end{tikzpicture}
    \end{center}
    Alors $\lim_{\substack{z \rightarrow 1 \\ z \in \Delta_{\theta_0}}} f(z) = \sum_{n=0}^{+\infty} a_n$.
  \end{theorem}

  \begin{proof}
    On note $\forall n \in \mathbb{N}$, $S = \sum_{n=0}^{+\infty} a_n$, $S_n = \sum_{k=0}^n a_k$ et $R_n = S - S_n$. On chercher à majorer $|f(z) - S|$ ; on va effectuer une transformation d'Abel en écrivant $\forall n \geq 1$, $a_n = R_{n-1} - R_n$. Soit $z \in D \setminus \{ 0 \}$. $\forall N \in \mathbb{N}^*$, on a
    \begin{align*}
      \sum_{n=0}^N a_n z^n - S_N &= \sum_{n=0}^N a_n(z^n - 1) \\
      &= \sum_{n=1}^N (R_{n-1} - R_n)(z^n - 1) \\
      &= \sum_{n=0}^{N-1} R_n(z^{n+1} - 1) - \sum_{n=1}^N R_n(z^n - 1) \\
      &= \sum_{n=0}^{N-1} R_n(z^{n+1} - z^n) - R_N(z^N - 1) \\
      &= (z-1) \sum_{n=0}^{N-1} R_nz^n - R_N(z^N - 1)
    \end{align*}
    Donc en faisant $N \rightarrow +\infty$ :
    \[ f(z) - S = (z-1) \sum_{n=0}^{+\infty} R_nz^n \tag{$*$} \]
    Soit $\epsilon > 0$. $\exists N \in \mathbb{N}$ tel que $\forall n \geq N$, $|R_n| < \epsilon$. D'après $(*)$, $\forall z \in D$,
    \begin{align*}
      |f(z)-S| &\leq |z-1| \left| \sum_{n=0}^N R_n z^n \right| + \epsilon |z-1| \left( \sum_{n=N+1}^{+\infty} |z|^n \right) \\
      &\leq |z-1| \left( \sum_{n=0}^N |R_n| \right) + \epsilon \frac{|z-1|}{1-|z|} \tag{$**$}
    \end{align*}
    Soit $z \in \Delta_{\theta_0}$ de sorte que $z = 1-\rho e^{i\theta}$ avec $\rho > 0$ et $|\theta| \leq \theta_0$. Notons avant toute chose que $|z-1| = \rho$. Cherchons maintenant des conditions sur $z$ pour majorer les deux termes :
    \begin{itemize}
      \item On a :
      \begin{align*}
        |z|^2 &= (1 - \rho \cos(\theta))^2 + (\rho \sin(\theta))^2 \\
        &= 1 - 2 \rho \cos(\theta) + \rho^2 (\cos(\theta)^2 + \sin(\theta)^2) \\
        &= 1 - 2 \rho \cos(\theta) + \rho^2
      \end{align*}
      En supposant $\rho \leq \cos(\theta_0)$, cela permet de majorer le deuxième terme de $(**)$ :
      \begin{align*}
        \frac{|z-1|}{1-|z|} &= \frac{|z-1|}{1-|z|^2}(1+|z|) \\
        &= \frac{\rho}{2 \rho \cos(\theta) - \rho^2}(1+|z|) \\
        &\leq \frac{2}{2\cos(\theta) - \rho} \\
        &\leq \frac{2}{2\cos(\theta_0) - \cos(\theta_0)} \\
        &= \frac{2}{\cos(\theta_0)}
      \end{align*}
      \item Soit $\alpha > 0$ suffisamment petit pour que $\alpha \sum_{n=0}^N |R_n| < \epsilon$. Si $z \in \Delta_{\theta_0}$ tel que $|z-1| \leq \alpha$, alors on peut majorer le premier terme de $(**)$ :
      \[ |z-1| \left( \sum_{n=0}^N |R_n| \right) \leq \alpha \left( \sum_{n=0}^N |R_n| \right) < \epsilon \]
    \end{itemize}
    Donc, en faisant $z \longrightarrow 1$ tel que $z \in \Delta_{\theta_0}$ (on aura bien $\rho = |z-1| \leq \inf \{ \alpha, \cos(\theta_0) \}$), et en injectant les deux majorations trouvées dans $(**)$ :
    \[ |f(z)-S| \leq \epsilon + \epsilon \frac{2}{\cos(\theta_0)} = \epsilon \left(1 + \frac{2}{\cos(\theta_0)} \right) \]
    d'où le résultat.
  \end{proof}

  \begin{application}
    \[ \sum_{n=0}^{+\infty} \frac{(-1)^n}{(2n+1)} = \frac{\pi}{4} \]
  \end{application}

  \begin{proof}
    En appliquant le \cref{theoreme-d-abel-angulaire-1} :
    \begin{align*}
      \sum_{n=0}^{+\infty} \frac{(-1)^n}{(2n+1)} &= \lim_{\substack{x \rightarrow 1 \\ x < 1}} \sum_{n=0}^{+\infty} \frac{(-1)^n}{(2n+1)} x^n \\
      &\overset{x > 0}{=} \lim_{\substack{x \rightarrow 1 \\ x < 1}} \frac{1}{\sqrt{x}} \sum_{n=0}^{+\infty} \frac{(-1)^n}{(2n+1)} \sqrt{x}^{2n+1} \\
      &= \lim_{\substack{x \rightarrow 1 \\ x < 1}} \frac{1}{\sqrt{x}} \arctan(\sqrt{x}) \\
      &= \arctan(1) \\
      &= \frac{\pi}{4}
    \end{align*}
  \end{proof}

  La preuve de l'application précédente écrite dans \cite{[GOU20]} est un peu lacunaire. Merci aux personnes qui l'ont signalée et corrigée.

  \begin{application}
    \[ \sum_{n=0}^{+\infty} \frac{(-1)^{n-1}}{n} = \ln(2) \]
  \end{application}

  \begin{proof}
    Toujours en appliquant le \cref{theoreme-d-abel-angulaire-1} :
    \begin{align*}
      \sum_{n=0}^{+\infty} \frac{(-1)^{n-1}}{n} &= \lim_{\substack{x \rightarrow 1 \\ x < 1}} \sum_{n=0}^{+\infty} \frac{(-1)^{n-1}}{n} x^n \\
      &= \lim_{\substack{x \rightarrow 1 \\ x < 1}} \ln(1 + x) \\
      &= \ln(2)
    \end{align*}
  \end{proof}
  %</content>
\end{document}
