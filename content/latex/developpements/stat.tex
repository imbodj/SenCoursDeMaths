\documentclass[12pt, a4paper]{report}

% LuaLaTeX :

\RequirePackage{iftex}
\RequireLuaTeX

% Packages :

\usepackage[french]{babel}
%\usepackage[utf8]{inputenc}
%\usepackage[T1]{fontenc}
\usepackage[pdfencoding=auto, pdfauthor={Hugo Delaunay}, pdfsubject={Mathématiques}, pdfcreator={agreg.skyost.eu}]{hyperref}
\usepackage{amsmath}
\usepackage{amsthm}
%\usepackage{amssymb}
\usepackage{stmaryrd}
\usepackage{tikz}
\usepackage{tkz-euclide}
\usepackage{fontspec}
\defaultfontfeatures[Erewhon]{FontFace = {bx}{n}{Erewhon-Bold.otf}}
\usepackage{fourier-otf}
\usepackage[nobottomtitles*]{titlesec}
\usepackage{fancyhdr}
\usepackage{listings}
\usepackage{catchfilebetweentags}
\usepackage[french, capitalise, noabbrev]{cleveref}
\usepackage[fit, breakall]{truncate}
\usepackage[top=2.5cm, right=2cm, bottom=2.5cm, left=2cm]{geometry}
\usepackage{enumitem}
\usepackage{tocloft}
\usepackage{microtype}
%\usepackage{mdframed}
%\usepackage{thmtools}
\usepackage{xcolor}
\usepackage{tabularx}
\usepackage{xltabular}
\usepackage{aligned-overset}
\usepackage[subpreambles=true]{standalone}
\usepackage{environ}
\usepackage[normalem]{ulem}
\usepackage{etoolbox}
\usepackage{setspace}
\usepackage[bibstyle=reading, citestyle=draft]{biblatex}
\usepackage{xpatch}
\usepackage[many, breakable]{tcolorbox}
\usepackage[backgroundcolor=white, bordercolor=white, textsize=scriptsize]{todonotes}
\usepackage{luacode}
\usepackage{float}
\usepackage{needspace}
\everymath{\displaystyle}

% Police :

\setmathfont{Erewhon Math}

% Tikz :

\usetikzlibrary{calc}
\usetikzlibrary{3d}

% Longueurs :

\setlength{\parindent}{0pt}
\setlength{\headheight}{15pt}
\setlength{\fboxsep}{0pt}
\titlespacing*{\chapter}{0pt}{-20pt}{10pt}
\setlength{\marginparwidth}{1.5cm}
\setstretch{1.1}

% Métadonnées :

\author{agreg.skyost.eu}
\date{\today}

% Titres :

\setcounter{secnumdepth}{3}

\renewcommand{\thechapter}{\Roman{chapter}}
\renewcommand{\thesubsection}{\Roman{subsection}}
\renewcommand{\thesubsubsection}{\arabic{subsubsection}}
\renewcommand{\theparagraph}{\alph{paragraph}}

\titleformat{\chapter}{\huge\bfseries}{\thechapter}{20pt}{\huge\bfseries}
\titleformat*{\section}{\LARGE\bfseries}
\titleformat{\subsection}{\Large\bfseries}{\thesubsection \, - \,}{0pt}{\Large\bfseries}
\titleformat{\subsubsection}{\large\bfseries}{\thesubsubsection. \,}{0pt}{\large\bfseries}
\titleformat{\paragraph}{\bfseries}{\theparagraph. \,}{0pt}{\bfseries}

\setcounter{secnumdepth}{4}

% Table des matières :

\renewcommand{\cftsecleader}{\cftdotfill{\cftdotsep}}
\addtolength{\cftsecnumwidth}{10pt}

% Redéfinition des commandes :

\renewcommand*\thesection{\arabic{section}}
\renewcommand{\ker}{\mathrm{Ker}}

% Nouvelles commandes :

\newcommand{\website}{https://github.com/imbodj/SenCoursDeMaths}

\newcommand{\tr}[1]{\mathstrut ^t #1}
\newcommand{\im}{\mathrm{Im}}
\newcommand{\rang}{\operatorname{rang}}
\newcommand{\trace}{\operatorname{trace}}
\newcommand{\id}{\operatorname{id}}
\newcommand{\stab}{\operatorname{Stab}}
\newcommand{\paren}[1]{\left(#1\right)}
\newcommand{\croch}[1]{\left[ #1 \right]}
\newcommand{\Grdcroch}[1]{\Bigl[ #1 \Bigr]}
\newcommand{\grdcroch}[1]{\bigl[ #1 \bigr]}
\newcommand{\abs}[1]{\left\lvert #1 \right\rvert}
\newcommand{\limi}[3]{\lim_{#1\to #2}#3}
\newcommand{\pinf}{+\infty}
\newcommand{\minf}{-\infty}
%%%%%%%%%%%%%% ENSEMBLES %%%%%%%%%%%%%%%%%
\newcommand{\ensemblenombre}[1]{\mathbb{#1}}
\newcommand{\Nn}{\ensemblenombre{N}}
\newcommand{\Zz}{\ensemblenombre{Z}}
\newcommand{\Qq}{\ensemblenombre{Q}}
\newcommand{\Qqp}{\Qq^+}
\newcommand{\Rr}{\ensemblenombre{R}}
\newcommand{\Cc}{\ensemblenombre{C}}
\newcommand{\Nne}{\Nn^*}
\newcommand{\Zze}{\Zz^*}
\newcommand{\Zzn}{\Zz^-}
\newcommand{\Qqe}{\Qq^*}
\newcommand{\Rre}{\Rr^*}
\newcommand{\Rrp}{\Rr_+}
\newcommand{\Rrm}{\Rr_-}
\newcommand{\Rrep}{\Rr_+^*}
\newcommand{\Rrem}{\Rr_-^*}
\newcommand{\Cce}{\Cc^*}
%%%%%%%%%%%%%%  INTERVALLES %%%%%%%%%%%%%%%%%
\newcommand{\intff}[2]{\left[#1\;,\; #2\right]  }
\newcommand{\intof}[2]{\left]#1 \;, \;#2\right]  }
\newcommand{\intfo}[2]{\left[#1 \;,\; #2\right[  }
\newcommand{\intoo}[2]{\left]#1 \;,\; #2\right[  }

\providecommand{\newpar}{\\[\medskipamount]}

\newcommand{\annexessection}{%
  \newpage%
  \subsection*{Annexes}%
}

\providecommand{\lesson}[3]{%
  \title{#3}%
  \hypersetup{pdftitle={#2 : #3}}%
  \setcounter{section}{\numexpr #2 - 1}%
  \section{#3}%
  \fancyhead[R]{\truncate{0.73\textwidth}{#2 : #3}}%
}

\providecommand{\development}[3]{%
  \title{#3}%
  \hypersetup{pdftitle={#3}}%
  \section*{#3}%
  \fancyhead[R]{\truncate{0.73\textwidth}{#3}}%
}

\providecommand{\sheet}[3]{\development{#1}{#2}{#3}}

\providecommand{\ranking}[1]{%
  \title{Terminale #1}%
  \hypersetup{pdftitle={Terminale #1}}%
  \section*{Terminale #1}%
  \fancyhead[R]{\truncate{0.73\textwidth}{Terminale #1}}%
}

\providecommand{\summary}[1]{%
  \textit{#1}%
  \par%
  \medskip%
}

\tikzset{notestyleraw/.append style={inner sep=0pt, rounded corners=0pt, align=center}}

%\newcommand{\booklink}[1]{\website/bibliographie\##1}
\newcounter{reference}
\newcommand{\previousreference}{}
\providecommand{\reference}[2][]{%
  \needspace{20pt}%
  \notblank{#1}{
    \needspace{20pt}%
    \renewcommand{\previousreference}{#1}%
    \stepcounter{reference}%
    \label{reference-\previousreference-\thereference}%
  }{}%
  \todo[noline]{%
    \protect\vspace{20pt}%
    \protect\par%
    \protect\notblank{#1}{\cite{[\previousreference]}\\}{}%
    \protect\hyperref[reference-\previousreference-\thereference]{p. #2}%
  }%
}

\definecolor{devcolor}{HTML}{00695c}
\providecommand{\dev}[1]{%
  \reversemarginpar%
  \todo[noline]{
    \protect\vspace{20pt}%
    \protect\par%
    \bfseries\color{devcolor}\href{\website/developpements/#1}{[DEV]}
  }%
  \normalmarginpar%
}

% En-têtes :

\pagestyle{fancy}
\fancyhead[L]{\truncate{0.23\textwidth}{\thepage}}
\fancyfoot[C]{\scriptsize \href{\website}{\texttt{https://github.com/imbodj/SenCoursDeMaths}}}

% Couleurs :

\definecolor{property}{HTML}{ffeb3b}
\definecolor{proposition}{HTML}{ffc107}
\definecolor{lemma}{HTML}{ff9800}
\definecolor{theorem}{HTML}{f44336}
\definecolor{corollary}{HTML}{e91e63}
\definecolor{definition}{HTML}{673ab7}
\definecolor{notation}{HTML}{9c27b0}
\definecolor{example}{HTML}{00bcd4}
\definecolor{cexample}{HTML}{795548}
\definecolor{application}{HTML}{009688}
\definecolor{remark}{HTML}{3f51b5}
\definecolor{algorithm}{HTML}{607d8b}
%\definecolor{proof}{HTML}{e1f5fe}
\definecolor{exercice}{HTML}{e1f5fe}

% Théorèmes :

\theoremstyle{definition}
\newtheorem{theorem}{Théorème}

\newtheorem{property}[theorem]{Propriété}
\newtheorem{proposition}[theorem]{Proposition}
\newtheorem{lemma}[theorem]{Activité d'introduction}
\newtheorem{corollary}[theorem]{Conséquence}

\newtheorem{definition}[theorem]{Définition}
\newtheorem{notation}[theorem]{Notation}

\newtheorem{example}[theorem]{Exemple}
\newtheorem{cexample}[theorem]{Contre-exemple}
\newtheorem{application}[theorem]{Application}

\newtheorem{algorithm}[theorem]{Algorithme}
\newtheorem{exercice}[theorem]{Exercice}

\theoremstyle{remark}
\newtheorem{remark}[theorem]{Remarque}

\counterwithin*{theorem}{section}

\newcommand{\applystyletotheorem}[1]{
  \tcolorboxenvironment{#1}{
    enhanced,
    breakable,
    colback=#1!8!white,
    %right=0pt,
    %top=8pt,
    %bottom=8pt,
    boxrule=0pt,
    frame hidden,
    sharp corners,
    enhanced,borderline west={4pt}{0pt}{#1},
    %interior hidden,
    sharp corners,
    after=\par,
  }
}

\applystyletotheorem{property}
\applystyletotheorem{proposition}
\applystyletotheorem{lemma}
\applystyletotheorem{theorem}
\applystyletotheorem{corollary}
\applystyletotheorem{definition}
\applystyletotheorem{notation}
\applystyletotheorem{example}
\applystyletotheorem{cexample}
\applystyletotheorem{application}
\applystyletotheorem{remark}
%\applystyletotheorem{proof}
\applystyletotheorem{algorithm}
\applystyletotheorem{exercice}

% Environnements :

\NewEnviron{whitetabularx}[1]{%
  \renewcommand{\arraystretch}{2.5}
  \colorbox{white}{%
    \begin{tabularx}{\textwidth}{#1}%
      \BODY%
    \end{tabularx}%
  }%
}

% Maths :

\DeclareFontEncoding{FMS}{}{}
\DeclareFontSubstitution{FMS}{futm}{m}{n}
\DeclareFontEncoding{FMX}{}{}
\DeclareFontSubstitution{FMX}{futm}{m}{n}
\DeclareSymbolFont{fouriersymbols}{FMS}{futm}{m}{n}
\DeclareSymbolFont{fourierlargesymbols}{FMX}{futm}{m}{n}
\DeclareMathDelimiter{\VERT}{\mathord}{fouriersymbols}{152}{fourierlargesymbols}{147}

% Code :

\definecolor{greencode}{rgb}{0,0.6,0}
\definecolor{graycode}{rgb}{0.5,0.5,0.5}
\definecolor{mauvecode}{rgb}{0.58,0,0.82}
\definecolor{bluecode}{HTML}{1976d2}
\lstset{
  basicstyle=\footnotesize\ttfamily,
  breakatwhitespace=false,
  breaklines=true,
  %captionpos=b,
  commentstyle=\color{greencode},
  deletekeywords={...},
  escapeinside={\%*}{*)},
  extendedchars=true,
  frame=none,
  keepspaces=true,
  keywordstyle=\color{bluecode},
  language=Python,
  otherkeywords={*,...},
  numbers=left,
  numbersep=5pt,
  numberstyle=\tiny\color{graycode},
  rulecolor=\color{black},
  showspaces=false,
  showstringspaces=false,
  showtabs=false,
  stepnumber=2,
  stringstyle=\color{mauvecode},
  tabsize=2,
  %texcl=true,
  xleftmargin=10pt,
  %title=\lstname
}

\newcommand{\codedirectory}{}
\newcommand{\inputalgorithm}[1]{%
  \begin{algorithm}%
    \strut%
    \lstinputlisting{\codedirectory#1}%
  \end{algorithm}%
}




\begin{document}
	%<*content>
	\development{analysis}{stat}{Statistique}

 \summary{}
 
	
	\begin{exercice}	
 Soit $(x_i , y_i)$  une série statistique double d'effectif total $ N $.

\medskip

$$
\begin{array}{|c|c|c|c|c|}
 \hline
  x_i   & x_1 & x_2 &  \dots & x_N \\ 
 \hline
   y_j   & y_1 & y_2 & \dots & y_N \\ \hline
\end{array}$$

\medskip

Reproduire et compléter chacune des formules ci-dessous :


\begin{enumerate}
\item  $ \overline{x}=\dfrac{\dots}{N}  \hspace*{0.5cm}   \overline{y}=\dfrac{\dots}{N} $.
 \item   $ V(x)=\dfrac{\dots}{N}-\dots   \hspace*{0.5cm}   V(y)=\dfrac{\dots}{N}\dots-\dots $
 
 \item   $ \text{cov}(x,y)=\dfrac{\dots}{N}- \dots \times \dots \hspace*{1cm}$. le coefficient de corrélation $ r= \dfrac{\dots}{\dots \times \dots}$.
 \item  La droite de régression de  $ y $ en $ x $ notée D$_{ y/x }$  a pour équation $ \dots $

\end{enumerate}
\end{exercice}
\begin{exercice}

Répondre par vrai (V) ou faux (F) à chacune des affirmations ci-dessous.


\begin{enumerate}
\item Dans une série statistique double ($x$, $y$) le point moyen G a pour coordonnées $ ( \overline{x}\;;\; \overline{y}) $.
\item Dans une série statistique double le coefficient de corrélation linéaire vérifie  $\; 0.87\leq |r|\leq 1 $. 
\item La droite de régression  D$_{ y/x }$  passe par  le point moyen.
\item Si un nuage de points semble être allongé alors, un ajustement linéaire est suggéré.
\end{enumerate}
\end{exercice}
\begin{exercice}

Le tableau ci-dessous donne le poids moyen (y) d'un enfant en fonction de son âge (x).
$$
\begin{array}{|c|c|c|c|c|c|c|c|c|}
 \hline
 x\; \text{(années)} &0&1&2&4&7&11&12\\ \hline
 y \;\text{(kg) } &3,5&6,5&9,5&14&21&32,5&34\\\hline
\end{array}
$$
\begin{enumerate}
\item Représenter le nuage de points de cette série statistique dans le plan muni du repère orthogonal. 
  
Unité graphique : en abscisse 1cm pour 1 année et en ordonnée 1cm pour 2 kg. 
\item Calculer:
\begin{enumerate}
\item les moyennes $ \overline{x} $ et $ \overline{y} $ puis placer le point moyen G.
\item  les variances $ V(x) $  et $ V(y) $.
\item  les écart-types $ \sigma(x) $   et $ \sigma(y) $.
\end{enumerate}
\item Calculer le coefficient de corrélation linéaire $ r $. Interpréter le résultat .
\item Déterminer une équation  de la droite   de régression de $y$ en $x$.
\item Si l'évolution se poursuit dans les mêmes conditions,
\begin{enumerate}
 \item déterminer l'âge  à partir duquel le poids est égal à 40 kg.
\item   quel sera le poids de l'enfant au bout de 15 années ?
\end{enumerate}
\end{enumerate}
\end{exercice}

\begin{exercice}
 Le tableau suivant donne la production d'arachide d'une certaine région depuis l'année 2000.\\
Les années ont été numérotées $ x_{i} $ et la production exprimée en centaine de tonnes,
 est notée $y_{i}$.
$$
\begin{array}{|c|c|c|c|c|c|c|}
\hline
\text{Années :}   & 2000 & 2001 & 2002 & 2003 & 2004 & 2005 \\
\hline
\text{ numéro de l'année} \;  x_{i}   & 1 & 2 & 3 & 4 & 5 & 6 \\
\hline
\text{Production} \; y_{i} & 5 & 9 & 7 & 10 & 12 & 10\\
\hline
\end{array}
$$
\begin{enumerate}
\item Représenter le nuage de points associé à cette série statistique.
\item Par la méthode des moindres carrés, donner une équation de la droite de\\
 régression $y$ en  $x $ .Tracer cette droite sur le graphique de la question  précédente  et indiquer les coordonnées du point moyen $ G $.
 \item En supposant que l'évolution est la même au cours des années suivantes quel tonnage pourrait-on prévoir en $ 2014 $ ?
\end{enumerate}
\end{exercice}
\begin{exercice}
On donne la série statistique double  : 
$$
\begin{array}{|c|c|c|c|c|c|c|c|c|}
\hline
    x & 35 & 40 &  35 &  65 & 65 &  85 & 90 &  k    \\
\hline
  y  & 3 & 4 & 5 & 10 & 8 &  13 &  14 & 15  \\
\hline
\end{array}
$$
\begin{enumerate}
\item Déterminer l'entier naturel $ k $  sachant que la droite de régression de $y$ par rapport à $x$ passe par le point moyen $ G $ d'abscisse 65 .
\item Calculer le coefficient de corrélation linéaire entre les caractères $x$ et $y $.
\item Déterminer une équation  de la droite  de régression de $y$ par rapport à $x$.
\item Estimer $ x $ sachant que $  y= 20 $.
 \end{enumerate}

\end{exercice}
\begin{exercice}
Une entreprise sénégalaise effectue un don d'engrais (en milliers de kilogrammes) à la culture d'arachide dans cinq
régions du pays. Son intention est de tester l'efficacité de son engrais par rapport à la production (en milliers de
tonnes) obtenue. Le tableau ci-dessous représente la production d'arachide ($y_i $) en fonction de la quantité
d'engrais ($x_i $) utilisée.
$$
 \begin{array}{|c|c|c|c|c|c|}
 \hline
 x_i  &6&8&9&10&12\\ \hline
 y_i    &10&14&15&18&20\\\hline
  \end{array}
 $$
 
 \medskip
A l'aide des informations ci-dessus et des outils mathématiques au programme :

\begin{enumerate}
\item la production d'arachide obtenue est-elle fortement corrélée à la quantité d'engrais utilisée? Justifier la
réponse.
\item donner une estimation de la production si le don d'engrais s'élève à 20 (en milliers de kilogrammes).
\end{enumerate}

\end{exercice}

\begin{exercice}
Le tableau ci-dessous donne le nombre total d'adhérents au club de mathématiques pour les dix premiers mois de l'année 2025( de janvier à octobre).
$$
\begin{array}{|c|c|c|c|c|c|c|c|c|c|c|}
\hline
\text{ Mois de l'année 2025 }\;  (x)  &1 & 2 &  3 &  4 & 5 & 6 & 7 &  8&9&10    \\
\hline
\text{Nombre d'adhérents}\;   (y)  & 10 & 30 & 50 & 40 & 60 &  55 & 70 & 80 & 70&80 \\
\hline
\end{array}
$$
Une fondation veut octroyer une aide financière au club si le nombre d'adhérents dépasse 200 élèves. Les élèves veulent déterminer quand ils pourront recevoir ce don.
Faisant partie de ce club, vous êtes sollicité par tes camarades pour  répondre à leur préoccupation.
Exploitez le tableau ci-dessus pour répondre aux questions suivantes.

\medskip
 En supposant que les adhésions suivent cette évolution, déterminer   la période (mois et année) à laquelle le club pourrait recevoir ce don.
  
\end{exercice}
	%</content>
\end{document}
