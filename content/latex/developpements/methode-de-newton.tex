\documentclass[12pt, a4paper]{report}

% LuaLaTeX :

\RequirePackage{iftex}
\RequireLuaTeX

% Packages :

\usepackage[french]{babel}
%\usepackage[utf8]{inputenc}
%\usepackage[T1]{fontenc}
\usepackage[pdfencoding=auto, pdfauthor={Hugo Delaunay}, pdfsubject={Mathématiques}, pdfcreator={agreg.skyost.eu}]{hyperref}
\usepackage{amsmath}
\usepackage{amsthm}
%\usepackage{amssymb}
\usepackage{stmaryrd}
\usepackage{tikz}
\usepackage{tkz-euclide}
\usepackage{fontspec}
\defaultfontfeatures[Erewhon]{FontFace = {bx}{n}{Erewhon-Bold.otf}}
\usepackage{fourier-otf}
\usepackage[nobottomtitles*]{titlesec}
\usepackage{fancyhdr}
\usepackage{listings}
\usepackage{catchfilebetweentags}
\usepackage[french, capitalise, noabbrev]{cleveref}
\usepackage[fit, breakall]{truncate}
\usepackage[top=2.5cm, right=2cm, bottom=2.5cm, left=2cm]{geometry}
\usepackage{enumitem}
\usepackage{tocloft}
\usepackage{microtype}
%\usepackage{mdframed}
%\usepackage{thmtools}
\usepackage{xcolor}
\usepackage{tabularx}
\usepackage{xltabular}
\usepackage{aligned-overset}
\usepackage[subpreambles=true]{standalone}
\usepackage{environ}
\usepackage[normalem]{ulem}
\usepackage{etoolbox}
\usepackage{setspace}
\usepackage[bibstyle=reading, citestyle=draft]{biblatex}
\usepackage{xpatch}
\usepackage[many, breakable]{tcolorbox}
\usepackage[backgroundcolor=white, bordercolor=white, textsize=scriptsize]{todonotes}
\usepackage{luacode}
\usepackage{float}
\usepackage{needspace}
\everymath{\displaystyle}

% Police :

\setmathfont{Erewhon Math}

% Tikz :

\usetikzlibrary{calc}
\usetikzlibrary{3d}

% Longueurs :

\setlength{\parindent}{0pt}
\setlength{\headheight}{15pt}
\setlength{\fboxsep}{0pt}
\titlespacing*{\chapter}{0pt}{-20pt}{10pt}
\setlength{\marginparwidth}{1.5cm}
\setstretch{1.1}

% Métadonnées :

\author{agreg.skyost.eu}
\date{\today}

% Titres :

\setcounter{secnumdepth}{3}

\renewcommand{\thechapter}{\Roman{chapter}}
\renewcommand{\thesubsection}{\Roman{subsection}}
\renewcommand{\thesubsubsection}{\arabic{subsubsection}}
\renewcommand{\theparagraph}{\alph{paragraph}}

\titleformat{\chapter}{\huge\bfseries}{\thechapter}{20pt}{\huge\bfseries}
\titleformat*{\section}{\LARGE\bfseries}
\titleformat{\subsection}{\Large\bfseries}{\thesubsection \, - \,}{0pt}{\Large\bfseries}
\titleformat{\subsubsection}{\large\bfseries}{\thesubsubsection. \,}{0pt}{\large\bfseries}
\titleformat{\paragraph}{\bfseries}{\theparagraph. \,}{0pt}{\bfseries}

\setcounter{secnumdepth}{4}

% Table des matières :

\renewcommand{\cftsecleader}{\cftdotfill{\cftdotsep}}
\addtolength{\cftsecnumwidth}{10pt}

% Redéfinition des commandes :

\renewcommand*\thesection{\arabic{section}}
\renewcommand{\ker}{\mathrm{Ker}}

% Nouvelles commandes :

\newcommand{\website}{https://github.com/imbodj/SenCoursDeMaths}

\newcommand{\tr}[1]{\mathstrut ^t #1}
\newcommand{\im}{\mathrm{Im}}
\newcommand{\rang}{\operatorname{rang}}
\newcommand{\trace}{\operatorname{trace}}
\newcommand{\id}{\operatorname{id}}
\newcommand{\stab}{\operatorname{Stab}}
\newcommand{\paren}[1]{\left(#1\right)}
\newcommand{\croch}[1]{\left[ #1 \right]}
\newcommand{\Grdcroch}[1]{\Bigl[ #1 \Bigr]}
\newcommand{\grdcroch}[1]{\bigl[ #1 \bigr]}
\newcommand{\abs}[1]{\left\lvert #1 \right\rvert}
\newcommand{\limi}[3]{\lim_{#1\to #2}#3}
\newcommand{\pinf}{+\infty}
\newcommand{\minf}{-\infty}
%%%%%%%%%%%%%% ENSEMBLES %%%%%%%%%%%%%%%%%
\newcommand{\ensemblenombre}[1]{\mathbb{#1}}
\newcommand{\Nn}{\ensemblenombre{N}}
\newcommand{\Zz}{\ensemblenombre{Z}}
\newcommand{\Qq}{\ensemblenombre{Q}}
\newcommand{\Qqp}{\Qq^+}
\newcommand{\Rr}{\ensemblenombre{R}}
\newcommand{\Cc}{\ensemblenombre{C}}
\newcommand{\Nne}{\Nn^*}
\newcommand{\Zze}{\Zz^*}
\newcommand{\Zzn}{\Zz^-}
\newcommand{\Qqe}{\Qq^*}
\newcommand{\Rre}{\Rr^*}
\newcommand{\Rrp}{\Rr_+}
\newcommand{\Rrm}{\Rr_-}
\newcommand{\Rrep}{\Rr_+^*}
\newcommand{\Rrem}{\Rr_-^*}
\newcommand{\Cce}{\Cc^*}
%%%%%%%%%%%%%%  INTERVALLES %%%%%%%%%%%%%%%%%
\newcommand{\intff}[2]{\left[#1\;,\; #2\right]  }
\newcommand{\intof}[2]{\left]#1 \;, \;#2\right]  }
\newcommand{\intfo}[2]{\left[#1 \;,\; #2\right[  }
\newcommand{\intoo}[2]{\left]#1 \;,\; #2\right[  }

\providecommand{\newpar}{\\[\medskipamount]}

\newcommand{\annexessection}{%
  \newpage%
  \subsection*{Annexes}%
}

\providecommand{\lesson}[3]{%
  \title{#3}%
  \hypersetup{pdftitle={#2 : #3}}%
  \setcounter{section}{\numexpr #2 - 1}%
  \section{#3}%
  \fancyhead[R]{\truncate{0.73\textwidth}{#2 : #3}}%
}

\providecommand{\development}[3]{%
  \title{#3}%
  \hypersetup{pdftitle={#3}}%
  \section*{#3}%
  \fancyhead[R]{\truncate{0.73\textwidth}{#3}}%
}

\providecommand{\sheet}[3]{\development{#1}{#2}{#3}}

\providecommand{\ranking}[1]{%
  \title{Terminale #1}%
  \hypersetup{pdftitle={Terminale #1}}%
  \section*{Terminale #1}%
  \fancyhead[R]{\truncate{0.73\textwidth}{Terminale #1}}%
}

\providecommand{\summary}[1]{%
  \textit{#1}%
  \par%
  \medskip%
}

\tikzset{notestyleraw/.append style={inner sep=0pt, rounded corners=0pt, align=center}}

%\newcommand{\booklink}[1]{\website/bibliographie\##1}
\newcounter{reference}
\newcommand{\previousreference}{}
\providecommand{\reference}[2][]{%
  \needspace{20pt}%
  \notblank{#1}{
    \needspace{20pt}%
    \renewcommand{\previousreference}{#1}%
    \stepcounter{reference}%
    \label{reference-\previousreference-\thereference}%
  }{}%
  \todo[noline]{%
    \protect\vspace{20pt}%
    \protect\par%
    \protect\notblank{#1}{\cite{[\previousreference]}\\}{}%
    \protect\hyperref[reference-\previousreference-\thereference]{p. #2}%
  }%
}

\definecolor{devcolor}{HTML}{00695c}
\providecommand{\dev}[1]{%
  \reversemarginpar%
  \todo[noline]{
    \protect\vspace{20pt}%
    \protect\par%
    \bfseries\color{devcolor}\href{\website/developpements/#1}{[DEV]}
  }%
  \normalmarginpar%
}

% En-têtes :

\pagestyle{fancy}
\fancyhead[L]{\truncate{0.23\textwidth}{\thepage}}
\fancyfoot[C]{\scriptsize \href{\website}{\texttt{https://github.com/imbodj/SenCoursDeMaths}}}

% Couleurs :

\definecolor{property}{HTML}{ffeb3b}
\definecolor{proposition}{HTML}{ffc107}
\definecolor{lemma}{HTML}{ff9800}
\definecolor{theorem}{HTML}{f44336}
\definecolor{corollary}{HTML}{e91e63}
\definecolor{definition}{HTML}{673ab7}
\definecolor{notation}{HTML}{9c27b0}
\definecolor{example}{HTML}{00bcd4}
\definecolor{cexample}{HTML}{795548}
\definecolor{application}{HTML}{009688}
\definecolor{remark}{HTML}{3f51b5}
\definecolor{algorithm}{HTML}{607d8b}
%\definecolor{proof}{HTML}{e1f5fe}
\definecolor{exercice}{HTML}{e1f5fe}

% Théorèmes :

\theoremstyle{definition}
\newtheorem{theorem}{Théorème}

\newtheorem{property}[theorem]{Propriété}
\newtheorem{proposition}[theorem]{Proposition}
\newtheorem{lemma}[theorem]{Activité d'introduction}
\newtheorem{corollary}[theorem]{Conséquence}

\newtheorem{definition}[theorem]{Définition}
\newtheorem{notation}[theorem]{Notation}

\newtheorem{example}[theorem]{Exemple}
\newtheorem{cexample}[theorem]{Contre-exemple}
\newtheorem{application}[theorem]{Application}

\newtheorem{algorithm}[theorem]{Algorithme}
\newtheorem{exercice}[theorem]{Exercice}

\theoremstyle{remark}
\newtheorem{remark}[theorem]{Remarque}

\counterwithin*{theorem}{section}

\newcommand{\applystyletotheorem}[1]{
  \tcolorboxenvironment{#1}{
    enhanced,
    breakable,
    colback=#1!8!white,
    %right=0pt,
    %top=8pt,
    %bottom=8pt,
    boxrule=0pt,
    frame hidden,
    sharp corners,
    enhanced,borderline west={4pt}{0pt}{#1},
    %interior hidden,
    sharp corners,
    after=\par,
  }
}

\applystyletotheorem{property}
\applystyletotheorem{proposition}
\applystyletotheorem{lemma}
\applystyletotheorem{theorem}
\applystyletotheorem{corollary}
\applystyletotheorem{definition}
\applystyletotheorem{notation}
\applystyletotheorem{example}
\applystyletotheorem{cexample}
\applystyletotheorem{application}
\applystyletotheorem{remark}
%\applystyletotheorem{proof}
\applystyletotheorem{algorithm}
\applystyletotheorem{exercice}

% Environnements :

\NewEnviron{whitetabularx}[1]{%
  \renewcommand{\arraystretch}{2.5}
  \colorbox{white}{%
    \begin{tabularx}{\textwidth}{#1}%
      \BODY%
    \end{tabularx}%
  }%
}

% Maths :

\DeclareFontEncoding{FMS}{}{}
\DeclareFontSubstitution{FMS}{futm}{m}{n}
\DeclareFontEncoding{FMX}{}{}
\DeclareFontSubstitution{FMX}{futm}{m}{n}
\DeclareSymbolFont{fouriersymbols}{FMS}{futm}{m}{n}
\DeclareSymbolFont{fourierlargesymbols}{FMX}{futm}{m}{n}
\DeclareMathDelimiter{\VERT}{\mathord}{fouriersymbols}{152}{fourierlargesymbols}{147}

% Code :

\definecolor{greencode}{rgb}{0,0.6,0}
\definecolor{graycode}{rgb}{0.5,0.5,0.5}
\definecolor{mauvecode}{rgb}{0.58,0,0.82}
\definecolor{bluecode}{HTML}{1976d2}
\lstset{
  basicstyle=\footnotesize\ttfamily,
  breakatwhitespace=false,
  breaklines=true,
  %captionpos=b,
  commentstyle=\color{greencode},
  deletekeywords={...},
  escapeinside={\%*}{*)},
  extendedchars=true,
  frame=none,
  keepspaces=true,
  keywordstyle=\color{bluecode},
  language=Python,
  otherkeywords={*,...},
  numbers=left,
  numbersep=5pt,
  numberstyle=\tiny\color{graycode},
  rulecolor=\color{black},
  showspaces=false,
  showstringspaces=false,
  showtabs=false,
  stepnumber=2,
  stringstyle=\color{mauvecode},
  tabsize=2,
  %texcl=true,
  xleftmargin=10pt,
  %title=\lstname
}

\newcommand{\codedirectory}{}
\newcommand{\inputalgorithm}[1]{%
  \begin{algorithm}%
    \strut%
    \lstinputlisting{\codedirectory#1}%
  \end{algorithm}%
}




\begin{document}
  %<*content>
  \development{analysis}{methode-de-newton}{Méthode de Newton}

  \summary{On démontre ici la méthode de Newton qui permet de trouver numériquement une approximation précise d'un zéro d'une fonction réelle d'une variable réelle.}

  \reference[ROU]{152}

  \begin{theorem}[Méthode de Newton]
    Soit $f : [c, d] \rightarrow \mathbb{R}$ une fonction de classe $\mathcal{C}^2$ strictement croissante sur $[c, d]$. On considère la fonction
    \[ \varphi :
    \begin{array}{ccc}
      [c, d] &\rightarrow& \mathbb{R} \\
      x &\mapsto& x - \frac{f(x)}{f'(x)}
    \end{array}
    \]
    (qui est bien définie car $f' > 0$). Alors :
    \begin{enumerate}[label=(\roman*)]
      \item $\exists! a \in [c, d]$ tel que $f(a) = 0$.
      \item $\exists \alpha > 0$ tel que $I = [a - \alpha, a + \alpha]$ est stable par $\varphi$.
      \item La suite $(x_n)$ des itérés (définie par récurrence par $x_{n+1} = \varphi(x_n)$ pour tout $n \geq 0$) converge quadratiquement vers $a$ pour tout $x_0 \in I$.
    \end{enumerate}
  \end{theorem}

  \begin{proof}
    Soit $x \in [c, d]$. Comme $f(a) = 0$, on peut écrire :
    \begin{align*}
      \varphi(x) - a &= x - a - \frac{f(x) - f(a)}{f'(x)} \\
      &= \frac{f(a) - f(x) - (a-x)f'(x)}{f'(x)}
    \end{align*}
    Or, la formule de Taylor-Lagrange à l'ordre $2$ donne l'existence d'un $z \in ]a, x[$ tel que
    \begin{align*}
      &f(a) = f(x) + f'(x)(a-x) + \frac{1}{2} f''(z)(a-x)^2 \\
      \iff& f(a) - f(x) - f'(x)(a-x)  = \frac{1}{2} f''(z)(a-x)^2
    \end{align*}
    D'où :
    \[ \varphi(x) - a = \frac{f''(z)}{2f'(x)}(x-a)^2 \tag{$*$} \]
    Soit $C = \frac{\max_{x \in [c, d]} |f''(x)|}{2\min_{x \in [c, d]} |f'(x)|}$. Par $(*)$, on a :
    \[ \forall x \in [c, d], \, |\varphi(x)-a| \leq C |x-a|^2 \]
    Soit maintenant $\alpha > 0$ suffisamment petit pour que $C\alpha < 1$ et que $I = [a - \alpha, a + \alpha] \subseteq [c, d]$. Alors :
    \[ x \in I \implies |\varphi(x) - a| \leq C\alpha^2 < \alpha \]
    (la première inégalité se voit en faisant un dessin, et la seconde vient du fait que $C\alpha < 1$). D'où $\varphi(I) \subseteq I$. Et si $x_0 \in I$, on a donc $\forall n \in \mathbb{N}$, $x_n \in I$ et
    \begin{align*}
      |x_{n+1} - a| &= |\varphi(x_n) - a| \\
      &\leq C |x_n - a|^2
    \end{align*}
    D'où $C |x_n - a| \leq (C |x_0 - a|)^{2^n} \leq (C \alpha)^{2^n}$ où $C \alpha < 1$. On a donc bien convergence quadratique de la suite $(x_n)$ vers le réel $a$.
  \end{proof}

  \reference[DEM]{100}

  \begin{remark}
    On suppose que l'on connaisse une approximation grossière du point que l'on nomme $x_0$.
    \begin{center}
      \begin{tikzpicture}
        \draw[->] (-1, 0) -- (5, 0) node[right] {$x$};
        \draw[->] (0, -1) -- (0, 5) node[above] {$y$};
        \draw[thick, domain=-1:4, color=teal] plot ({\x}, {0.3*(\x+0.5)^2-1});
        \draw[domain=1.3:4] plot ({\x}, {2.675+2.1*(\x-3)});
        \draw(3, 2.675) node{$\bullet$};
        \draw(3, 0.1) -- (3, -0.1) node[below] {$x_0$};
        \draw(1.72619048, 0.1) -- (1.72619048, -0.1) node[below, shift={(0.1,0)}] {$x_1$};
        \draw(1.32574, -0.1) -- (1.32574, 0.1) node[above] {$a$};
        \draw[dashed] (3, 0) -- (3, 2.675) node[above left] {$f(x_0)$};
      \end{tikzpicture}
    \end{center}
    L'idée de la méthode est de remplacer la courbe représentative de $f$ par sa tangente au point $x_0$ :
    \[ y = f'(x_0)(x-x_0) + f(x_0) \]
    L'abscisse $x_1$ du point d'intersection de cette tangente avec l'axe des abscisses est donnée par
    \[ x_1 = x_0 - \frac{f(x_0)}{f'(x_0)} \]
    d'où le fait d'itérer la fonction $\varphi : x \mapsto x - \frac{f(x)}{f'(x)}$.
  \end{remark}

  \reference[ROU]{152}

  \begin{corollary}
    En reprenant les hypothèses et notations du théorème précédent, et en supposant de plus $f$ strictement convexe sur $[c, d]$, le résultat du théorème est vrai sur $I = [a, d]$. De plus :
    \begin{enumerate}[label=(\roman*)]
      \item $(x_n)$ est strictement décroissante (ou constante).
      \item $x_{n+1} - a \sim \frac{f''(a)}{2f'(a)} (x_n - a)^2$ pour $x_0 > a$.
    \end{enumerate}
  \end{corollary}

  \begin{proof}
    La dérivée $f'$ est strictement croissante (car $f$ est strictement convexe) sur $]c, d[$. Ainsi, soit $x \in [a, d]$. Si $x = a$, on a $\varphi(x) = x$, et la suite $(x_n)$ est alors constante. Supposons maintenant $x > a$. On a :
    \[ \varphi(x) = x - \frac{\overbrace{f(x)}^{> 0}}{\underbrace{f'(x)}_{> 0}} < x \]
    Et par $(*)$ (de la démonstration précédente), $\exists z \in ]a, x[$ :
    \[ \varphi(x) - a = \frac{f''(z)}{2f'(z)} (x-a)^2 > 0 \iff \varphi(x) < a \]
    Ainsi, $I = [a, d]$ est stable par $\varphi$ et pour $x_0 \in ]a, d]$, on a $x_n \in ]a, d]$ pour tout $n \in \mathbb{N}$ et la suite $(x_n)$ est strictement décroissante. La suite $(x_n)$ admet donc une limite $\ell$ vérifiant $\varphi(\ell) = \ell \iff f(\ell) = 0$ ie. $\ell = a$ par unicité. Comme dans le théorème précédent, la convergence est quadratique :
    \[ 0 \leq x_{n+1} - a \leq C (x_n - a)^2 \]
    Enfin, si $x_0 \in ]a, d]$, on a comme dans $(*)$ :
    \[ \forall n \in \mathbb{N}, \, x_n > a \text{ et } \frac{x_{n+1} - a}{(x_n - a)^2} = \frac{f''(z_n)}{2f'(x_n)} \]
    où $z_n \in ]a, x_n[$ (d'après la démarche effectuée pour obtenir $(*)$). On fait tendre $n$ vers l'infini et la fraction de droite tend vers $\frac{f''(a)}{2f'(a)}$; d'où le résultat.
  \end{proof}

  \begin{remark}
    L'ajout de l'hypothèse de convexité à la méthode de Newton, nous permet de nous affranchir de l'intervalle $I$ tout en gardant la même vitesse de convergence.
  \end{remark}
  %</content>
\end{document}
