\documentclass[12pt, a4paper]{report}

% LuaLaTeX :

\RequirePackage{iftex}
\RequireLuaTeX

% Packages :

\usepackage[french]{babel}
%\usepackage[utf8]{inputenc}
%\usepackage[T1]{fontenc}
\usepackage[pdfencoding=auto, pdfauthor={Hugo Delaunay}, pdfsubject={Mathématiques}, pdfcreator={agreg.skyost.eu}]{hyperref}
\usepackage{amsmath}
\usepackage{amsthm}
%\usepackage{amssymb}
\usepackage{stmaryrd}
\usepackage{tikz}
\usepackage{tkz-euclide}
\usepackage{fontspec}
\defaultfontfeatures[Erewhon]{FontFace = {bx}{n}{Erewhon-Bold.otf}}
\usepackage{fourier-otf}
\usepackage[nobottomtitles*]{titlesec}
\usepackage{fancyhdr}
\usepackage{listings}
\usepackage{catchfilebetweentags}
\usepackage[french, capitalise, noabbrev]{cleveref}
\usepackage[fit, breakall]{truncate}
\usepackage[top=2.5cm, right=2cm, bottom=2.5cm, left=2cm]{geometry}
\usepackage{enumitem}
\usepackage{tocloft}
\usepackage{microtype}
%\usepackage{mdframed}
%\usepackage{thmtools}
\usepackage{xcolor}
\usepackage{tabularx}
\usepackage{xltabular}
\usepackage{aligned-overset}
\usepackage[subpreambles=true]{standalone}
\usepackage{environ}
\usepackage[normalem]{ulem}
\usepackage{etoolbox}
\usepackage{setspace}
\usepackage[bibstyle=reading, citestyle=draft]{biblatex}
\usepackage{xpatch}
\usepackage[many, breakable]{tcolorbox}
\usepackage[backgroundcolor=white, bordercolor=white, textsize=scriptsize]{todonotes}
\usepackage{luacode}
\usepackage{float}
\usepackage{needspace}
\everymath{\displaystyle}

% Police :

\setmathfont{Erewhon Math}

% Tikz :

\usetikzlibrary{calc}
\usetikzlibrary{3d}

% Longueurs :

\setlength{\parindent}{0pt}
\setlength{\headheight}{15pt}
\setlength{\fboxsep}{0pt}
\titlespacing*{\chapter}{0pt}{-20pt}{10pt}
\setlength{\marginparwidth}{1.5cm}
\setstretch{1.1}

% Métadonnées :

\author{agreg.skyost.eu}
\date{\today}

% Titres :

\setcounter{secnumdepth}{3}

\renewcommand{\thechapter}{\Roman{chapter}}
\renewcommand{\thesubsection}{\Roman{subsection}}
\renewcommand{\thesubsubsection}{\arabic{subsubsection}}
\renewcommand{\theparagraph}{\alph{paragraph}}

\titleformat{\chapter}{\huge\bfseries}{\thechapter}{20pt}{\huge\bfseries}
\titleformat*{\section}{\LARGE\bfseries}
\titleformat{\subsection}{\Large\bfseries}{\thesubsection \, - \,}{0pt}{\Large\bfseries}
\titleformat{\subsubsection}{\large\bfseries}{\thesubsubsection. \,}{0pt}{\large\bfseries}
\titleformat{\paragraph}{\bfseries}{\theparagraph. \,}{0pt}{\bfseries}

\setcounter{secnumdepth}{4}

% Table des matières :

\renewcommand{\cftsecleader}{\cftdotfill{\cftdotsep}}
\addtolength{\cftsecnumwidth}{10pt}

% Redéfinition des commandes :

\renewcommand*\thesection{\arabic{section}}
\renewcommand{\ker}{\mathrm{Ker}}

% Nouvelles commandes :

\newcommand{\website}{https://github.com/imbodj/SenCoursDeMaths}

\newcommand{\tr}[1]{\mathstrut ^t #1}
\newcommand{\im}{\mathrm{Im}}
\newcommand{\rang}{\operatorname{rang}}
\newcommand{\trace}{\operatorname{trace}}
\newcommand{\id}{\operatorname{id}}
\newcommand{\stab}{\operatorname{Stab}}
\newcommand{\paren}[1]{\left(#1\right)}
\newcommand{\croch}[1]{\left[ #1 \right]}
\newcommand{\Grdcroch}[1]{\Bigl[ #1 \Bigr]}
\newcommand{\grdcroch}[1]{\bigl[ #1 \bigr]}
\newcommand{\abs}[1]{\left\lvert #1 \right\rvert}
\newcommand{\limi}[3]{\lim_{#1\to #2}#3}
\newcommand{\pinf}{+\infty}
\newcommand{\minf}{-\infty}
%%%%%%%%%%%%%% ENSEMBLES %%%%%%%%%%%%%%%%%
\newcommand{\ensemblenombre}[1]{\mathbb{#1}}
\newcommand{\Nn}{\ensemblenombre{N}}
\newcommand{\Zz}{\ensemblenombre{Z}}
\newcommand{\Qq}{\ensemblenombre{Q}}
\newcommand{\Qqp}{\Qq^+}
\newcommand{\Rr}{\ensemblenombre{R}}
\newcommand{\Cc}{\ensemblenombre{C}}
\newcommand{\Nne}{\Nn^*}
\newcommand{\Zze}{\Zz^*}
\newcommand{\Zzn}{\Zz^-}
\newcommand{\Qqe}{\Qq^*}
\newcommand{\Rre}{\Rr^*}
\newcommand{\Rrp}{\Rr_+}
\newcommand{\Rrm}{\Rr_-}
\newcommand{\Rrep}{\Rr_+^*}
\newcommand{\Rrem}{\Rr_-^*}
\newcommand{\Cce}{\Cc^*}
%%%%%%%%%%%%%%  INTERVALLES %%%%%%%%%%%%%%%%%
\newcommand{\intff}[2]{\left[#1\;,\; #2\right]  }
\newcommand{\intof}[2]{\left]#1 \;, \;#2\right]  }
\newcommand{\intfo}[2]{\left[#1 \;,\; #2\right[  }
\newcommand{\intoo}[2]{\left]#1 \;,\; #2\right[  }

\providecommand{\newpar}{\\[\medskipamount]}

\newcommand{\annexessection}{%
  \newpage%
  \subsection*{Annexes}%
}

\providecommand{\lesson}[3]{%
  \title{#3}%
  \hypersetup{pdftitle={#2 : #3}}%
  \setcounter{section}{\numexpr #2 - 1}%
  \section{#3}%
  \fancyhead[R]{\truncate{0.73\textwidth}{#2 : #3}}%
}

\providecommand{\development}[3]{%
  \title{#3}%
  \hypersetup{pdftitle={#3}}%
  \section*{#3}%
  \fancyhead[R]{\truncate{0.73\textwidth}{#3}}%
}

\providecommand{\sheet}[3]{\development{#1}{#2}{#3}}

\providecommand{\ranking}[1]{%
  \title{Terminale #1}%
  \hypersetup{pdftitle={Terminale #1}}%
  \section*{Terminale #1}%
  \fancyhead[R]{\truncate{0.73\textwidth}{Terminale #1}}%
}

\providecommand{\summary}[1]{%
  \textit{#1}%
  \par%
  \medskip%
}

\tikzset{notestyleraw/.append style={inner sep=0pt, rounded corners=0pt, align=center}}

%\newcommand{\booklink}[1]{\website/bibliographie\##1}
\newcounter{reference}
\newcommand{\previousreference}{}
\providecommand{\reference}[2][]{%
  \needspace{20pt}%
  \notblank{#1}{
    \needspace{20pt}%
    \renewcommand{\previousreference}{#1}%
    \stepcounter{reference}%
    \label{reference-\previousreference-\thereference}%
  }{}%
  \todo[noline]{%
    \protect\vspace{20pt}%
    \protect\par%
    \protect\notblank{#1}{\cite{[\previousreference]}\\}{}%
    \protect\hyperref[reference-\previousreference-\thereference]{p. #2}%
  }%
}

\definecolor{devcolor}{HTML}{00695c}
\providecommand{\dev}[1]{%
  \reversemarginpar%
  \todo[noline]{
    \protect\vspace{20pt}%
    \protect\par%
    \bfseries\color{devcolor}\href{\website/developpements/#1}{[DEV]}
  }%
  \normalmarginpar%
}

% En-têtes :

\pagestyle{fancy}
\fancyhead[L]{\truncate{0.23\textwidth}{\thepage}}
\fancyfoot[C]{\scriptsize \href{\website}{\texttt{https://github.com/imbodj/SenCoursDeMaths}}}

% Couleurs :

\definecolor{property}{HTML}{ffeb3b}
\definecolor{proposition}{HTML}{ffc107}
\definecolor{lemma}{HTML}{ff9800}
\definecolor{theorem}{HTML}{f44336}
\definecolor{corollary}{HTML}{e91e63}
\definecolor{definition}{HTML}{673ab7}
\definecolor{notation}{HTML}{9c27b0}
\definecolor{example}{HTML}{00bcd4}
\definecolor{cexample}{HTML}{795548}
\definecolor{application}{HTML}{009688}
\definecolor{remark}{HTML}{3f51b5}
\definecolor{algorithm}{HTML}{607d8b}
%\definecolor{proof}{HTML}{e1f5fe}
\definecolor{exercice}{HTML}{e1f5fe}

% Théorèmes :

\theoremstyle{definition}
\newtheorem{theorem}{Théorème}

\newtheorem{property}[theorem]{Propriété}
\newtheorem{proposition}[theorem]{Proposition}
\newtheorem{lemma}[theorem]{Activité d'introduction}
\newtheorem{corollary}[theorem]{Conséquence}

\newtheorem{definition}[theorem]{Définition}
\newtheorem{notation}[theorem]{Notation}

\newtheorem{example}[theorem]{Exemple}
\newtheorem{cexample}[theorem]{Contre-exemple}
\newtheorem{application}[theorem]{Application}

\newtheorem{algorithm}[theorem]{Algorithme}
\newtheorem{exercice}[theorem]{Exercice}

\theoremstyle{remark}
\newtheorem{remark}[theorem]{Remarque}

\counterwithin*{theorem}{section}

\newcommand{\applystyletotheorem}[1]{
  \tcolorboxenvironment{#1}{
    enhanced,
    breakable,
    colback=#1!8!white,
    %right=0pt,
    %top=8pt,
    %bottom=8pt,
    boxrule=0pt,
    frame hidden,
    sharp corners,
    enhanced,borderline west={4pt}{0pt}{#1},
    %interior hidden,
    sharp corners,
    after=\par,
  }
}

\applystyletotheorem{property}
\applystyletotheorem{proposition}
\applystyletotheorem{lemma}
\applystyletotheorem{theorem}
\applystyletotheorem{corollary}
\applystyletotheorem{definition}
\applystyletotheorem{notation}
\applystyletotheorem{example}
\applystyletotheorem{cexample}
\applystyletotheorem{application}
\applystyletotheorem{remark}
%\applystyletotheorem{proof}
\applystyletotheorem{algorithm}
\applystyletotheorem{exercice}

% Environnements :

\NewEnviron{whitetabularx}[1]{%
  \renewcommand{\arraystretch}{2.5}
  \colorbox{white}{%
    \begin{tabularx}{\textwidth}{#1}%
      \BODY%
    \end{tabularx}%
  }%
}

% Maths :

\DeclareFontEncoding{FMS}{}{}
\DeclareFontSubstitution{FMS}{futm}{m}{n}
\DeclareFontEncoding{FMX}{}{}
\DeclareFontSubstitution{FMX}{futm}{m}{n}
\DeclareSymbolFont{fouriersymbols}{FMS}{futm}{m}{n}
\DeclareSymbolFont{fourierlargesymbols}{FMX}{futm}{m}{n}
\DeclareMathDelimiter{\VERT}{\mathord}{fouriersymbols}{152}{fourierlargesymbols}{147}

% Code :

\definecolor{greencode}{rgb}{0,0.6,0}
\definecolor{graycode}{rgb}{0.5,0.5,0.5}
\definecolor{mauvecode}{rgb}{0.58,0,0.82}
\definecolor{bluecode}{HTML}{1976d2}
\lstset{
  basicstyle=\footnotesize\ttfamily,
  breakatwhitespace=false,
  breaklines=true,
  %captionpos=b,
  commentstyle=\color{greencode},
  deletekeywords={...},
  escapeinside={\%*}{*)},
  extendedchars=true,
  frame=none,
  keepspaces=true,
  keywordstyle=\color{bluecode},
  language=Python,
  otherkeywords={*,...},
  numbers=left,
  numbersep=5pt,
  numberstyle=\tiny\color{graycode},
  rulecolor=\color{black},
  showspaces=false,
  showstringspaces=false,
  showtabs=false,
  stepnumber=2,
  stringstyle=\color{mauvecode},
  tabsize=2,
  %texcl=true,
  xleftmargin=10pt,
  %title=\lstname
}

\newcommand{\codedirectory}{}
\newcommand{\inputalgorithm}[1]{%
  \begin{algorithm}%
    \strut%
    \lstinputlisting{\codedirectory#1}%
  \end{algorithm}%
}




\begin{document}
	%<*content>
	\development{algebra}{similitudes-directes}{Similitudes directes plan complexe}

 \summary{}
 
\begin{exercice}
D\'eterminer en justifiant la nature des transformations du plan suivantes :
\begin{enumerate}
  \item $z' = z + 1 - 2\i$
  \item $z' = \i z + 1$
  \item $z' = -3z - 1 + \i$
  \item $z' = (1 + \i)z - 1 + \i$
\end{enumerate}
\end{exercice}

\begin{exercice}
D\'eterminer l'\'ecriture complexe des transformations suivantes :
\begin{enumerate}
  \item Translation de vecteur $\vec{u}$ d'affixe $-2 + 3\i$.
  \item Rotation de centre $A$ d'affixe $-\i$ et d'angle $\frac{3\pi}{4}$.
  \item Homoth\'etie de centre $A$ d'affixe $-1 + \i$ et de rapport $\frac{1}{2}$.
  \item Similitude directe de centre $A$ d'affixe $-2$, de rapport $2$ et d'angle $-\frac{2\pi}{3}$.
\end{enumerate}
\end{exercice}

\begin{exercice}
Dans chacun des cas suivants, d\'eterminer l'\'ecriture complexe, puis la nature et les \'el\'ements caract\'eristiques des transformations $ s_2\circ s_1 $ et $ s_1\circ s_2 $.
\begin{enumerate}
  \item $s_1 : z' = 2\i z + 1 - 2\i \quad \text{et} \quad s_2 : z' = \frac{1}{2}\i z + 1 - \frac{1}{2}\i$
  \item $s_1 : z' = (1 - \i)z + 1 + \i \quad \text{et} \quad s_2 : z' = -2z$
\end{enumerate}
\end{exercice}

\begin{exercice}
Soit $f$ l'application du plan dans lui-m\^eme d'expression analytique : 
\[\begin{cases} x' = x + y + 2 \\ y' = -x + y + 1 \end{cases}\]
\begin{enumerate}
  \item D\'eterminer l'\'ecriture complexe de $f$.
  \item D\'eterminer la nature et les \'el\'ements caract\'eristiques de $f$.
\end{enumerate}
\end{exercice}

\begin{exercice}
Dans le plan complexe, on consid\`ere les points $A$ et $C$ d'affixes respectives $1 + \i\sqrt{3}$ et $1 - \i\sqrt{3}$, et l'application $f$ définie par $z' = e^{2\i\frac{\pi}{3}}z$.
\begin{enumerate}
  \item D\'eterminer les images des points $A$ et $C$ par $f$.
  \item En d\'eduire l'\'equation de l'image de la droite (AC).
\end{enumerate}
\end{exercice}

\begin{exercice}
Soit le plan muni d'un rep\`ere orthonorm\'e $(o, \vec{u}, \vec{v})$, $\Omega$ est le point de coordonn\'ees $(2,1)$, $S$ est la similitude de centre $\Omega$, de rapport $\sqrt{2}$ et d'angle $\frac{\pi}{4}$, $(D)$ est la droite $3x + 3y - 4 = 0$, $(C)$ le cercle de centre $O$ et de rayon 3.
\begin{enumerate}
  \item Soit $M(x,y)$ un point et $M'(x',y')$ son image par $S$. Exprimer $x'$ et $y'$ en fonction de $x$ et $y$.
  \item En d\'eduire l'\'equation de $(D')$ image de $(D)$ par $S$.
  \item En d\'eduire l'\'equation de $C'$ image de $C$ par $S$.
\end{enumerate}
\end{exercice}

\begin{exercice}
Soit $S$ la similitude directe d'\'ecriture complexe : $z' = 3\i z - 9 - 3\i$. D\'eterminer l'image par $S$ :
\begin{enumerate}
  \item Du cercle de centre $K(1 - 3\i)$ et de rayon 1.
  \item De la droite $(D)$ d'\'equation $x = 1$.
\end{enumerate}
\end{exercice}

\begin{exercice}
Le plan complexe est rapport\'e \`a un rep\`ere orthonormal $(o, \vec{u}, \vec{v})$. On donne $Z_A = -1 - \i$, $Z_B = 2 - \i$, $Z_C = -1 + 2\i$. On consid\`ere la similitude $S$ de centre $B$ qui transforme $A$ en $C$.
\begin{enumerate}
  \item D\'eterminer le rapport et l'angle de la similitude $S$.
  \item Donner l'\'ecriture complexe de $S$.
  \item $\mathcal{C}$ est le cercle circonscrit au triangle $ABC$, d\'eterminer les caract\'eristiques de $\mathcal{C}'$ image par $S$.
\end{enumerate}
\end{exercice}

\begin{exercice}
On consid\`ere les points $Z_A = 2\i$, $Z_B = -\sqrt{3} + 2\i$, $Z_C = -2\sqrt{3} - \i$.
\begin{enumerate}
  \item Donner l'\'ecriture complexe de la similitude directe qui transforme $A$ en $B$ et $B$ en $C$.
  \item D\'eterminer l'affixe du centre, le rapport et l'angle de cette similitude.
\end{enumerate}
\end{exercice}

\begin{exercice}
A et B ont pour affixes $Z_A = 3\i$ et $Z_B = 4 + \i$. $r$ est la rotation de centre $A$ et d'angle $\frac{\pi}{2}$.
\begin{enumerate}
  \item Montrer que $z' = \i z + 3 + 3\i$.
  \item Montrer que $Z_C = 2 + 7\i$ si $r(B) = C$.
  \item Montrer que $ABC$ est rectangle isoc\`ele en $A$.
  \item Soit $D$ le milieu de $[AC]$, $h(z) = \frac{1}{2}z + \frac{3}{2}\i$.
    \begin{enumerate}
      \item Montrer que $h(C) = D$.
      \item Exprimer $z' - 3\i$ en fonction de $z - 3\i$ et en d\'eduire la nature de $h$.
    \end{enumerate}
  \item $S$ est la similitude de centre $A$, de rapport $\frac{1}{2}$ et d'angle $\frac{\pi}{2}$. Donner son \'ecriture complexe.
\end{enumerate}
\end{exercice}

\begin{exercice}
Soit $a = -1 - \i$ et $(z_n)$ la suite d\'efinie par :
\[\begin{cases} z_0 = 0, \quad z_1 = \i \\ z_{n+1} = (1-a)z_n + az_{n-1} \end{cases}\]
\begin{enumerate}
  \item D\'eterminer $z_2$ et $z_3$.
  \item Soit $u_n = z_{n+1} - z_n$
    \begin{enumerate}
      \item D\'eterminer $u_0$ et $u_1$.
      \item Montrer que $(u_n)$ est g\'eom\'etrique de raison $-a$.
      \item Exprimer $u_n$ en fonction de $n$ et $a$.
    \end{enumerate}
  \item Montrer que $S_n = z_n$ avec $S_n = u_0 + \cdots + u_{n-1}$, puis que $z_n = -1 + (1+\i)^n$.
  \item
    \begin{enumerate}
      \item D\'eterminer le module et un argument de $a$.
      \item Donner la forme alg\'ebrique de $z_{19}$.
    \end{enumerate}
\end{enumerate}
\end{exercice}

\begin{exercice}
Soit $T(z) = (1+\i\sqrt{3})z + \sqrt{3}$.
\begin{enumerate}
  \item
    \begin{enumerate}
      \item Montrer que $T$ admet un point invariant $J$.
      \item D\'eterminer la nature et les \'el\'ements caract\'eristiques de $T$.
      \item Exprimer $\overrightarrow{JM'}$ en fonction de $\overrightarrow{JM}$ et donner l'angle $(\overrightarrow{JM}, \overrightarrow{JM'})$.
    \end{enumerate}
  \item Soit $I$ d'affixe $-1$.
    \begin{enumerate}
      \item D\'eterminer $I' = T(I)$.
      \item D\'eterminer l'ensemble $\mathcal{D}$ des $z$ tels que $|z + 1| = |\i z + 1|$.
      \item D\'eterminer $\mathcal{C}$ des $z$ tels que $|(1+\i)z -1+\i| = |1 - \i|$.
      \item D\'eterminer $\mathcal{D}' = T(\mathcal{D})$, $\mathcal{C}' = T(\mathcal{C})$.
    \end{enumerate}
  \item Soit $S(z) = \frac{\i}{2}z - \i$.
    \begin{enumerate}
      \item D\'eterminer le rapport et l'angle de $T \circ S$.
      \item En d\'eduire la nature de $T \circ S$.
      \item Donner l'\'ecriture complexe de $T \circ S$.
    \end{enumerate}
  \item
    \begin{enumerate}
      \item Montrer que $T^3$ est une homoth\'etie.
      \item D\'eterminer les $n \in \mathbb{N}$ pour lesquels $T^n$ est une homoth\'etie.
    \end{enumerate}
\end{enumerate}
\end{exercice}

\begin{exercice}
Soit $T(z) = a^2z + b$ et le point $I \paren{ 2\i}$.
\begin{enumerate}
  \item $T$ est une translation de vecteur $\vec{U}(0, 2)$.
  \item $T$ est une homoth\'etie de rapport $-4$ et de centre $I$.
  \item $T$ est une rotation de centre $I$ et d'angle $-\frac{\pi}{2}$.
  \item $T$ est une similitude directe transformant $A(-1, 0)$ en $I$ et de centre $B(-1,1)$.
\end{enumerate}
\end{exercice}

	%</content>
\end{document}
