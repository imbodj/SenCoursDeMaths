\input{../common}
\everymath{\displaystyle}
\begin{document}
  %<*content>
  \development{algebra}{exos-limites}{Limites (TS2)}

  \summary{}
  \begin{exercice}
     Vrai ou faux?
      \begin{enumerate}
      \item Si   $ \;\displaystyle\lim_{x \to a}\; g(x)=a $ \; et \;  $\; \displaystyle\lim_{x \to b}\; f(x)=a \;$\;  alors  \;  $\; \lim_{x \to b}\; \paren{g \circ f}(x)=a $.
       \item Soit la fonction $ u $  telle que $ x-2\leq u(x)\leq x+3 $  pour tout $  x>1 $.  Alors $\limi{x}{+\infty}{\dfrac{u(x)}{\sqrt{x}}}=\pinf $.
       \item  Si une fonction  $ f $ définie et  strictement croissante  sur $ \mathbb{R} $ telle que: $ \;\displaystyle\lim_{x \to \minf}f(x)=0$  et $\;\displaystyle \lim_{x \to \pinf}f(x)=1$ alors :

\medskip

\textbf{a)} $ \;\displaystyle\lim_{x \to \pinf}f(-x+1)=0$   $\quad \textbf{b)} \;\displaystyle\lim_{x \to 0^{+}}f\paren{x+\frac{1}{x}}=1$  $ \quad \textbf{c)}\;\displaystyle\lim_{x \to \minf}\frac{f(x)+x}{f(|x|)-1}=+\infty$.


       \end{enumerate}
\end{exercice}
  \begin{exercice}
      Étudier les limites suivantes.
      \begin{enumerate}
\item  $ \displaystyle \lim_{x \to +\infty}{\sin \dfrac{\pi}{x}}$
 \item $\displaystyle \lim_{x \to \infty}{x\sin \dfrac{\pi}{x}}$
\item $ \displaystyle \lim_{x \to -\infty}{\sqrt{\dfrac{2x^2}{1-x}}}$
\item $ \displaystyle \lim_{x \to+\infty}\left(x-\sqrt{x}+\dfrac{1}{x}\right)^3$
\item $ \displaystyle \lim_{x \to +\infty} \; x\left(\sqrt{\dfrac{x}{x+1}-1}\right)$
 \end{enumerate}
\end{exercice}
  \begin{exercice}
        Calculer la limite suivante.
     $ \displaystyle \lim_{x \to 0} \dfrac{\sqrt{x+4}-2}{x}$ 
     
       En déduire :
        $$ \displaystyle \lim_{x \to \dfrac{\pi}{2}} \dfrac{\sqrt{4+\cos x}-2}{\cos x}\hspace*{0.5cm}\text{et}\hspace*{0.5cm}
       \displaystyle \lim_{x \to \dfrac{\pi}{2}} \dfrac{\sqrt{4+\sin x}-2}{ x}$$
        \end{exercice}
        \begin{exercice}
        On considère la fonction $ f $ définie sur $ [2,\;+\infty[ $ par $ f(x)=\dfrac{3x+\sin x}{x-1} $
        
Montrer que , pour tout $ x\geq 2 $ ,  $ \;\left |f(x) -3\right | \leq \dfrac{4}{x-1}$.

 En déduire la limite de  $ f $ en $ +\infty?.$
\end{exercice}
  \begin{exercice}
 Soit  la fonction $ f $ définie par :  $ f : x\mapsto x^2 \sin\left(\dfrac{1}{x}\right)+1\,\;$ $ \forall x\in\mathbb{R}^{\ast} $
 \begin{enumerate}
\item   Montrer que  $ \forall x\in\mathbb{R}^{\ast} \;\;$  $ 1 -x^2\leq  f(x) \leq  1  +x^2$
\item  En déduire :
\begin{enumerate}
\item $\displaystyle \lim_{x \to 0}{f(x)}$
\item $\displaystyle \lim_{x \to +\infty}{\dfrac{f(x)}{x^3}}$
\item  $ \displaystyle\lim_{x \to 0}{\dfrac{f(x)-1}{x}}$.
 \end{enumerate}
  \end{enumerate}
\end{exercice}
  \begin{exercice}
   Soi  $ f $  une fonction  définie sur $\mathbb{R}\setminus\{-1\} $ telle que: $ \displaystyle\lim_{x \to -\infty}f(x)=0 $,  $ \displaystyle\lim_{x \to +\infty}f(x)=1$,  $ \displaystyle\lim_{x \to -1^{-}}f(x)=+\infty $  et  $\displaystyle \lim_{x \to -1^{+}}f(x)=-\infty $. 
\begin{enumerate}
\item Interpréter graphiquement ces  limites.
\item  En déduire   les limites suivantes.
\begin{enumerate}
\item  $ \displaystyle\lim_{x \to+\infty}\; f\left(\sqrt{x}\right)$
\item $ \displaystyle\lim_{x \to   +\infty}\;\;f\left(-1+\frac{1}{x}\right)$
\item $\displaystyle \lim_{x \to 0^{-}}\;\;f\left(\dfrac{1}{x}\right)$
\item  $ \displaystyle\lim_{x \to -\infty}\;\;\left(\dfrac{f(x)-1}{2f(x)+1}\right)^2$
\end{enumerate}
\end{enumerate}
\end{exercice}
  \begin{exercice}
Étudier les limites suivantes.
\begin{enumerate}
\item $\displaystyle \lim_{x \to 1}\;\dfrac{x^{10}-1}{x-1} $
\item $\displaystyle  \lim_{x \to -1} \; \dfrac{x\sqrt{x+2}+1}{x+1}$
\item $ \displaystyle \lim_{x \to  \dfrac{\pi}{6}} \; \dfrac{2\sin x-1}{6x-\pi}$
\item  $ \displaystyle  \lim_{x \to 0} \; \dfrac{\cos^5 x+\sin 2x-1}{x} $ 
\end{enumerate}
\end{exercice}
 \begin{exercice}

 Soit $ f $ une  fonction  définie et dérivable sur  $\;\mathbb{R} \;$   tel que $ f(1)=0 \;$   et   $\; f'(1)=-1 $.
 
 \medskip
 
  $ \mathcal{C}_{f} $ admet  une asymptote d'équation $ y=3 $  en $ \minf $  et une asymptote d'équation $ y=x+4 $  en $ \pinf $.

\medskip
\begin{enumerate}
\item Calculer les limites suivantes.

\begin{enumerate}
\item $ \displaystyle\lim_{x \to 0}\; f\paren{\dfrac{x-1}{x^2}}$
\item $ \displaystyle\lim_{x \to +\infty}\;\;\dfrac{f(x)}{ x+f(x)}$
\item $\displaystyle \lim_{x \to +\infty}\; \dfrac{1}{f(x)-x+3}$
\end{enumerate}

\medskip
\item On considère  la limite  suivante  $\displaystyle \lim_{x \to +\infty}\; xf\paren{1+\dfrac{1}{x}}$.

\medskip 
\begin{enumerate}
\item Justifier qu'il y a une présence de forme indéterminée.
\item En posant $ X=1+\frac{1}{x} $,   calculer cette limite.
\end{enumerate}


\end{enumerate}

 
\end{exercice}
  %</content>
\end{document}
