\documentclass[12pt, a4paper]{report}

% LuaLaTeX :

\RequirePackage{iftex}
\RequireLuaTeX

% Packages :

\usepackage[french]{babel}
%\usepackage[utf8]{inputenc}
%\usepackage[T1]{fontenc}
\usepackage[pdfencoding=auto, pdfauthor={Hugo Delaunay}, pdfsubject={Mathématiques}, pdfcreator={agreg.skyost.eu}]{hyperref}
\usepackage{amsmath}
\usepackage{amsthm}
%\usepackage{amssymb}
\usepackage{stmaryrd}
\usepackage{tikz}
\usepackage{tkz-euclide}
\usepackage{fontspec}
\defaultfontfeatures[Erewhon]{FontFace = {bx}{n}{Erewhon-Bold.otf}}
\usepackage{fourier-otf}
\usepackage[nobottomtitles*]{titlesec}
\usepackage{fancyhdr}
\usepackage{listings}
\usepackage{catchfilebetweentags}
\usepackage[french, capitalise, noabbrev]{cleveref}
\usepackage[fit, breakall]{truncate}
\usepackage[top=2.5cm, right=2cm, bottom=2.5cm, left=2cm]{geometry}
\usepackage{enumitem}
\usepackage{tocloft}
\usepackage{microtype}
%\usepackage{mdframed}
%\usepackage{thmtools}
\usepackage{xcolor}
\usepackage{tabularx}
\usepackage{xltabular}
\usepackage{aligned-overset}
\usepackage[subpreambles=true]{standalone}
\usepackage{environ}
\usepackage[normalem]{ulem}
\usepackage{etoolbox}
\usepackage{setspace}
\usepackage[bibstyle=reading, citestyle=draft]{biblatex}
\usepackage{xpatch}
\usepackage[many, breakable]{tcolorbox}
\usepackage[backgroundcolor=white, bordercolor=white, textsize=scriptsize]{todonotes}
\usepackage{luacode}
\usepackage{float}
\usepackage{needspace}
\everymath{\displaystyle}

% Police :

\setmathfont{Erewhon Math}

% Tikz :

\usetikzlibrary{calc}
\usetikzlibrary{3d}

% Longueurs :

\setlength{\parindent}{0pt}
\setlength{\headheight}{15pt}
\setlength{\fboxsep}{0pt}
\titlespacing*{\chapter}{0pt}{-20pt}{10pt}
\setlength{\marginparwidth}{1.5cm}
\setstretch{1.1}

% Métadonnées :

\author{agreg.skyost.eu}
\date{\today}

% Titres :

\setcounter{secnumdepth}{3}

\renewcommand{\thechapter}{\Roman{chapter}}
\renewcommand{\thesubsection}{\Roman{subsection}}
\renewcommand{\thesubsubsection}{\arabic{subsubsection}}
\renewcommand{\theparagraph}{\alph{paragraph}}

\titleformat{\chapter}{\huge\bfseries}{\thechapter}{20pt}{\huge\bfseries}
\titleformat*{\section}{\LARGE\bfseries}
\titleformat{\subsection}{\Large\bfseries}{\thesubsection \, - \,}{0pt}{\Large\bfseries}
\titleformat{\subsubsection}{\large\bfseries}{\thesubsubsection. \,}{0pt}{\large\bfseries}
\titleformat{\paragraph}{\bfseries}{\theparagraph. \,}{0pt}{\bfseries}

\setcounter{secnumdepth}{4}

% Table des matières :

\renewcommand{\cftsecleader}{\cftdotfill{\cftdotsep}}
\addtolength{\cftsecnumwidth}{10pt}

% Redéfinition des commandes :

\renewcommand*\thesection{\arabic{section}}
\renewcommand{\ker}{\mathrm{Ker}}

% Nouvelles commandes :

\newcommand{\website}{https://github.com/imbodj/SenCoursDeMaths}

\newcommand{\tr}[1]{\mathstrut ^t #1}
\newcommand{\im}{\mathrm{Im}}
\newcommand{\rang}{\operatorname{rang}}
\newcommand{\trace}{\operatorname{trace}}
\newcommand{\id}{\operatorname{id}}
\newcommand{\stab}{\operatorname{Stab}}
\newcommand{\paren}[1]{\left(#1\right)}
\newcommand{\croch}[1]{\left[ #1 \right]}
\newcommand{\Grdcroch}[1]{\Bigl[ #1 \Bigr]}
\newcommand{\grdcroch}[1]{\bigl[ #1 \bigr]}
\newcommand{\abs}[1]{\left\lvert #1 \right\rvert}
\newcommand{\limi}[3]{\lim_{#1\to #2}#3}
\newcommand{\pinf}{+\infty}
\newcommand{\minf}{-\infty}
%%%%%%%%%%%%%% ENSEMBLES %%%%%%%%%%%%%%%%%
\newcommand{\ensemblenombre}[1]{\mathbb{#1}}
\newcommand{\Nn}{\ensemblenombre{N}}
\newcommand{\Zz}{\ensemblenombre{Z}}
\newcommand{\Qq}{\ensemblenombre{Q}}
\newcommand{\Qqp}{\Qq^+}
\newcommand{\Rr}{\ensemblenombre{R}}
\newcommand{\Cc}{\ensemblenombre{C}}
\newcommand{\Nne}{\Nn^*}
\newcommand{\Zze}{\Zz^*}
\newcommand{\Zzn}{\Zz^-}
\newcommand{\Qqe}{\Qq^*}
\newcommand{\Rre}{\Rr^*}
\newcommand{\Rrp}{\Rr_+}
\newcommand{\Rrm}{\Rr_-}
\newcommand{\Rrep}{\Rr_+^*}
\newcommand{\Rrem}{\Rr_-^*}
\newcommand{\Cce}{\Cc^*}
%%%%%%%%%%%%%%  INTERVALLES %%%%%%%%%%%%%%%%%
\newcommand{\intff}[2]{\left[#1\;,\; #2\right]  }
\newcommand{\intof}[2]{\left]#1 \;, \;#2\right]  }
\newcommand{\intfo}[2]{\left[#1 \;,\; #2\right[  }
\newcommand{\intoo}[2]{\left]#1 \;,\; #2\right[  }

\providecommand{\newpar}{\\[\medskipamount]}

\newcommand{\annexessection}{%
  \newpage%
  \subsection*{Annexes}%
}

\providecommand{\lesson}[3]{%
  \title{#3}%
  \hypersetup{pdftitle={#2 : #3}}%
  \setcounter{section}{\numexpr #2 - 1}%
  \section{#3}%
  \fancyhead[R]{\truncate{0.73\textwidth}{#2 : #3}}%
}

\providecommand{\development}[3]{%
  \title{#3}%
  \hypersetup{pdftitle={#3}}%
  \section*{#3}%
  \fancyhead[R]{\truncate{0.73\textwidth}{#3}}%
}

\providecommand{\sheet}[3]{\development{#1}{#2}{#3}}

\providecommand{\ranking}[1]{%
  \title{Terminale #1}%
  \hypersetup{pdftitle={Terminale #1}}%
  \section*{Terminale #1}%
  \fancyhead[R]{\truncate{0.73\textwidth}{Terminale #1}}%
}

\providecommand{\summary}[1]{%
  \textit{#1}%
  \par%
  \medskip%
}

\tikzset{notestyleraw/.append style={inner sep=0pt, rounded corners=0pt, align=center}}

%\newcommand{\booklink}[1]{\website/bibliographie\##1}
\newcounter{reference}
\newcommand{\previousreference}{}
\providecommand{\reference}[2][]{%
  \needspace{20pt}%
  \notblank{#1}{
    \needspace{20pt}%
    \renewcommand{\previousreference}{#1}%
    \stepcounter{reference}%
    \label{reference-\previousreference-\thereference}%
  }{}%
  \todo[noline]{%
    \protect\vspace{20pt}%
    \protect\par%
    \protect\notblank{#1}{\cite{[\previousreference]}\\}{}%
    \protect\hyperref[reference-\previousreference-\thereference]{p. #2}%
  }%
}

\definecolor{devcolor}{HTML}{00695c}
\providecommand{\dev}[1]{%
  \reversemarginpar%
  \todo[noline]{
    \protect\vspace{20pt}%
    \protect\par%
    \bfseries\color{devcolor}\href{\website/developpements/#1}{[DEV]}
  }%
  \normalmarginpar%
}

% En-têtes :

\pagestyle{fancy}
\fancyhead[L]{\truncate{0.23\textwidth}{\thepage}}
\fancyfoot[C]{\scriptsize \href{\website}{\texttt{https://github.com/imbodj/SenCoursDeMaths}}}

% Couleurs :

\definecolor{property}{HTML}{ffeb3b}
\definecolor{proposition}{HTML}{ffc107}
\definecolor{lemma}{HTML}{ff9800}
\definecolor{theorem}{HTML}{f44336}
\definecolor{corollary}{HTML}{e91e63}
\definecolor{definition}{HTML}{673ab7}
\definecolor{notation}{HTML}{9c27b0}
\definecolor{example}{HTML}{00bcd4}
\definecolor{cexample}{HTML}{795548}
\definecolor{application}{HTML}{009688}
\definecolor{remark}{HTML}{3f51b5}
\definecolor{algorithm}{HTML}{607d8b}
%\definecolor{proof}{HTML}{e1f5fe}
\definecolor{exercice}{HTML}{e1f5fe}

% Théorèmes :

\theoremstyle{definition}
\newtheorem{theorem}{Théorème}

\newtheorem{property}[theorem]{Propriété}
\newtheorem{proposition}[theorem]{Proposition}
\newtheorem{lemma}[theorem]{Activité d'introduction}
\newtheorem{corollary}[theorem]{Conséquence}

\newtheorem{definition}[theorem]{Définition}
\newtheorem{notation}[theorem]{Notation}

\newtheorem{example}[theorem]{Exemple}
\newtheorem{cexample}[theorem]{Contre-exemple}
\newtheorem{application}[theorem]{Application}

\newtheorem{algorithm}[theorem]{Algorithme}
\newtheorem{exercice}[theorem]{Exercice}

\theoremstyle{remark}
\newtheorem{remark}[theorem]{Remarque}

\counterwithin*{theorem}{section}

\newcommand{\applystyletotheorem}[1]{
  \tcolorboxenvironment{#1}{
    enhanced,
    breakable,
    colback=#1!8!white,
    %right=0pt,
    %top=8pt,
    %bottom=8pt,
    boxrule=0pt,
    frame hidden,
    sharp corners,
    enhanced,borderline west={4pt}{0pt}{#1},
    %interior hidden,
    sharp corners,
    after=\par,
  }
}

\applystyletotheorem{property}
\applystyletotheorem{proposition}
\applystyletotheorem{lemma}
\applystyletotheorem{theorem}
\applystyletotheorem{corollary}
\applystyletotheorem{definition}
\applystyletotheorem{notation}
\applystyletotheorem{example}
\applystyletotheorem{cexample}
\applystyletotheorem{application}
\applystyletotheorem{remark}
%\applystyletotheorem{proof}
\applystyletotheorem{algorithm}
\applystyletotheorem{exercice}

% Environnements :

\NewEnviron{whitetabularx}[1]{%
  \renewcommand{\arraystretch}{2.5}
  \colorbox{white}{%
    \begin{tabularx}{\textwidth}{#1}%
      \BODY%
    \end{tabularx}%
  }%
}

% Maths :

\DeclareFontEncoding{FMS}{}{}
\DeclareFontSubstitution{FMS}{futm}{m}{n}
\DeclareFontEncoding{FMX}{}{}
\DeclareFontSubstitution{FMX}{futm}{m}{n}
\DeclareSymbolFont{fouriersymbols}{FMS}{futm}{m}{n}
\DeclareSymbolFont{fourierlargesymbols}{FMX}{futm}{m}{n}
\DeclareMathDelimiter{\VERT}{\mathord}{fouriersymbols}{152}{fourierlargesymbols}{147}

% Code :

\definecolor{greencode}{rgb}{0,0.6,0}
\definecolor{graycode}{rgb}{0.5,0.5,0.5}
\definecolor{mauvecode}{rgb}{0.58,0,0.82}
\definecolor{bluecode}{HTML}{1976d2}
\lstset{
  basicstyle=\footnotesize\ttfamily,
  breakatwhitespace=false,
  breaklines=true,
  %captionpos=b,
  commentstyle=\color{greencode},
  deletekeywords={...},
  escapeinside={\%*}{*)},
  extendedchars=true,
  frame=none,
  keepspaces=true,
  keywordstyle=\color{bluecode},
  language=Python,
  otherkeywords={*,...},
  numbers=left,
  numbersep=5pt,
  numberstyle=\tiny\color{graycode},
  rulecolor=\color{black},
  showspaces=false,
  showstringspaces=false,
  showtabs=false,
  stepnumber=2,
  stringstyle=\color{mauvecode},
  tabsize=2,
  %texcl=true,
  xleftmargin=10pt,
  %title=\lstname
}

\newcommand{\codedirectory}{}
\newcommand{\inputalgorithm}[1]{%
  \begin{algorithm}%
    \strut%
    \lstinputlisting{\codedirectory#1}%
  \end{algorithm}%
}




\begin{document}
  %<*content>
  \development{analysis}{dual-de-lp}{Dual de \texorpdfstring{$L_p$}{Lp}}

  \summary{Avec les propriétés hilbertiennes de $L_2$ couplées à certaines propriétés des espaces $L_p$, on montre que le dual d'un espace $L_p$ est $L_q$ pour $\frac{1}{p} + \frac{1}{q} = 1$, dans le cas où $p \in ]1, 2[$ et où l'espace est de mesure finie.}

  Soit $(X, \mathcal{A}, \mu)$ un espace mesuré de mesure finie.

  \begin{notation}
    On note $\forall p \in ]1, 2[$, $L_p = L_p(X, \mathcal{A}, \mu)$.
  \end{notation}

  \begin{lemma}
    \label{dual-de-lp-1}
    Soient $p \in ]1, 2[$ et $f \in L_2$. Alors $f \in L_p$ telle que $\Vert f \Vert_p \leq M \Vert f \Vert_2$ où $M \geq 0$.
  \end{lemma}

  \begin{proof}
    Comme $p \in ]1, 2[$, on a $\frac{2}{p} > 1$. Soit $r$ tel que $\frac{p}{2} + \frac{1}{r} = 1$. On applique l'inégalité de Hölder à $g = \vert f \vert^p \mathbb{1}_X$ de sorte que
    \[ \int_X \vert f \vert^p \, \mathrm{d}\mu = \Vert \vert f \vert^p \mathbb{1}_X \Vert_1 \leq \Vert \vert f \vert^p \Vert_{\frac{2}{p}} \Vert \mathbb{1}_X \Vert_r \leq \mu(X)^{\frac{1}{r}} \Vert f \Vert_2^p \]
    d'où le résultat.
  \end{proof}

  \begin{lemma}
    \label{dual-de-lp-2}
    Soit $p \in ]1, 2[$. Alors $L_2$ est dense dans $L_p$ pour la norme $\Vert . \Vert_p$.
  \end{lemma}

  \begin{proof}
    Soit $f \in L_p$. On considère la suite de fonction $(f_n)$ définie par
    \[ \forall n \in \mathbb{N}, \, f_n = f \mathbb{1}_{|f| \leq n} \]
    Clairement, $(f_n)$ est une suite de $L_2$. On va chercher à appliquer le théorème de convergence dominée à la suite de fonctions $(g_n)$ définie pour tout $n \in \mathbb{N}$ par $g_n = |f_n - f|^p$ :
    \begin{itemize}
      \item $\forall n \in \mathbb{N}$, $g_n$ est mesurable.
      \item $(g_n)$ converge presque partout vers la fonction nulle.
      \item Par convexité de la fonction $x \mapsto x^p$, on a
      \[ |f_n - f|^p = 2^p \left| \frac{f_n}{2} - \frac{f}{2} \right|^p \leq 2^{p-1} (|f|^p + |f_n|^p) \leq 2^p |f|^p \in L_1 \]
    \end{itemize}
    On peut donc conclure
    \[ \Vert f - f_n \Vert^p_p = \int_X |f - f_n|^p \, \mathrm{d}\mu \longrightarrow 0 \]
    ce qu'il fallait démontrer.
  \end{proof}

  \reference[Z-Q]{222}

  \begin{theorem}
    L'application
    \[
    \varphi :
    \begin{array}{ll}
      L_q &\rightarrow (L_p)' \\
      g &\mapsto \left( \varphi_g : f \mapsto \int_X f \overline{g} \, \mathrm{d}\mu \right)
    \end{array}
    \qquad \text{ où } \frac{1}{p} + \frac{1}{q} = 1
    \]
    est une isométrie linéaire surjective. C'est donc un isomorphisme isométrique.
  \end{theorem}

  \begin{proof}
    Soient $g \in L_q$ et $f \in L_p$. L'inégalité de Hölder donne
    \[ \vert \varphi_g(f) \vert \leq \Vert g \Vert_q \Vert f \Vert_p \]
    donc $\varphi_g \in (L_p)'$ et $\VERT \varphi_g \VERT \leq \Vert g \Vert_q$. De plus, si $g = 0$, alors $\VERT \varphi_g \VERT = \Vert g \Vert_q = 0$. On peut donc supposer $g \neq 0$.
    \newpar
     Soit $u$ une fonction mesurable de module 1, telle que $g = u \vert g \vert$. On pose $h = \overline{u} \vert g \vert^{q-1}$. Comme $q = p(q-1)$, on a
    \[ \int_X \vert h \vert^p \, \mathrm{d}\mu = \int_X \vert g \vert^{(q-1)p} \, \mathrm{d}\mu = \int_X \vert g \vert^{q} \, \mathrm{d}\mu < + \infty \]
    d'où $h \in L_p$ et $\Vert h \Vert_p^p = \Vert g \Vert_q^q = \vert \varphi_g(h) \vert$. Comme, $\frac{\vert \varphi_g(h) \vert}{\Vert h \Vert_p} \leq \VERT \varphi_g \VERT$, on a en particulier,
    \[ \underbrace{\int_X \vert g \vert^{q} \, \mathrm{d}\mu}_{= \vert \varphi_g(h) \vert} \leq \VERT \varphi_g \VERT \underbrace{\left ( \int_X \vert g \vert^{q} \, \mathrm{d}\mu \right )^{\frac{1}{p}}}_{= \Vert h \Vert_p} \]
    et ainsi,
    \[ \VERT \varphi_g \VERT \geq \left ( \int_X \vert g \vert^{q} \, \mathrm{d}\mu \right )^{1 - \frac{1}{p}} = \left ( \int_X \vert g \vert^{q} \, \mathrm{d}\mu \right )^{\frac{1}{q}} = \Vert g \Vert_q \]
    donc $\VERT \varphi_g \VERT = \Vert g \Vert_q$ et $\varphi$ est une isométrie.
    \newpar
    Montrons qu'elle est surjective. Soit $\ell \in (L_p)'$. D'après le \cref{dual-de-lp-1}, on a $L_2 \subseteq L_p$, donc on peut considérer la restriction $\widetilde{\ell} = \ell_{| L_2}$.
    \[ \forall f \in L_2, \quad \vert \widetilde{\ell}(f) \vert \leq \VERT \ell \VERT \Vert f \Vert_p \leq M \Vert \ell \Vert \Vert f \Vert_2 \implies \widetilde{\ell} \in (L_2)' \]
    Comme $L_2$ est un espace de Hilbert, on peut appliquer le théorème de représentation de Riesz à $\widetilde{\ell}$. Il existe $g \in L_2$ telle que
    \[ \forall f \in L_2, \quad \widetilde{\ell}(f) = \int_X f \overline{g} \, \mathrm{d}\mu \]
    Pour conclure, il reste à montrer que $g \in L_q$ et que l'égalité précédente est vérifiée sur $L_p$. Comme précédemment, on considère $u$ de module $1$ telle que $g = u \vert g \vert$ et on pose $f_n = \overline{u} \vert g \vert^{q-1} \mathbb{1}_{\vert g \vert \leq n} \in L_\infty \subseteq L_2$. On a
    \[ \int_X \vert g \vert^q \mathbb{1}_{\vert g \vert \leq n} \, \mathrm{d}\mu = \vert \ell(f_n) \vert \leq \Vert \ell \Vert \Vert f_n \Vert_p = \Vert \ell \Vert \left ( \int_X \vert g \vert^q \mathbb{1}_{\vert g \vert \leq n} \, \mathrm{d}\mu \right )^{\frac{1}{p}} \]
    D'où
    \[ \left ( \int_X \vert g \vert^q \mathbb{1}_{\vert g \vert \leq n} \, \mathrm{d}\mu \right )^{\frac{1}{q}} = \left ( \int_X \vert g \vert^q \mathbb{1}_{\vert g \vert \leq n} \, \mathrm{d}\mu \right )^{1 - \frac{1}{p}} \leq \VERT \ell \VERT \]
    D'après le théorème de convergence monotone, on a
    \[ \lim_{n \rightarrow +\infty} \left ( \int_X \vert g \vert^q \mathbb{1}_{\vert g \vert \leq n} \, \mathrm{d}\mu \right )^{\frac{1}{q}} = \left ( \int_X \vert g \vert^q \, \mathrm{d}\mu \right )^{\frac{1}{q}} \leq \VERT \ell \VERT \]
    Et en particulier, $g \in L_q$ de norme inférieure ou égale à $\VERT \ell \VERT$. Ainsi, on a $\forall f \in L_2$, $\ell(f) = \varphi_g(f)$. Les applications $\ell$ et $\varphi_g$ sont continues sur $L_p$ et $L_2$ est dense dans $L_p$ (par le \cref{dual-de-lp-2}), donc on a bien $\ell = \varphi_g = \varphi(g)$.
  \end{proof}

  \reference[LI]{140}

  \begin{remark}
    Plus généralement, si l'on identifie $g$ et $\varphi_g$ :
    \begin{itemize}
      \item $L_q$ est le dual topologique de $L_p$ pour $p \in ]1, +\infty[$.
      \item $L_\infty$ est le dual topologique de $L_1$ si $\mu$ est $\sigma$-finie.
    \end{itemize}
  \end{remark}
  %</content>
\end{document}
