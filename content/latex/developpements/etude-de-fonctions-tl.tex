\documentclass[12pt, a4paper]{report}

% LuaLaTeX :

\RequirePackage{iftex}
\RequireLuaTeX

% Packages :

\usepackage[french]{babel}
%\usepackage[utf8]{inputenc}
%\usepackage[T1]{fontenc}
\usepackage[pdfencoding=auto, pdfauthor={Hugo Delaunay}, pdfsubject={Mathématiques}, pdfcreator={agreg.skyost.eu}]{hyperref}
\usepackage{amsmath}
\usepackage{amsthm}
%\usepackage{amssymb}
\usepackage{stmaryrd}
\usepackage{tikz}
\usepackage{tkz-euclide}
\usepackage{fontspec}
\defaultfontfeatures[Erewhon]{FontFace = {bx}{n}{Erewhon-Bold.otf}}
\usepackage{fourier-otf}
\usepackage[nobottomtitles*]{titlesec}
\usepackage{fancyhdr}
\usepackage{listings}
\usepackage{catchfilebetweentags}
\usepackage[french, capitalise, noabbrev]{cleveref}
\usepackage[fit, breakall]{truncate}
\usepackage[top=2.5cm, right=2cm, bottom=2.5cm, left=2cm]{geometry}
\usepackage{enumitem}
\usepackage{tocloft}
\usepackage{microtype}
%\usepackage{mdframed}
%\usepackage{thmtools}
\usepackage{xcolor}
\usepackage{tabularx}
\usepackage{xltabular}
\usepackage{aligned-overset}
\usepackage[subpreambles=true]{standalone}
\usepackage{environ}
\usepackage[normalem]{ulem}
\usepackage{etoolbox}
\usepackage{setspace}
\usepackage[bibstyle=reading, citestyle=draft]{biblatex}
\usepackage{xpatch}
\usepackage[many, breakable]{tcolorbox}
\usepackage[backgroundcolor=white, bordercolor=white, textsize=scriptsize]{todonotes}
\usepackage{luacode}
\usepackage{float}
\usepackage{needspace}
\everymath{\displaystyle}

% Police :

\setmathfont{Erewhon Math}

% Tikz :

\usetikzlibrary{calc}
\usetikzlibrary{3d}

% Longueurs :

\setlength{\parindent}{0pt}
\setlength{\headheight}{15pt}
\setlength{\fboxsep}{0pt}
\titlespacing*{\chapter}{0pt}{-20pt}{10pt}
\setlength{\marginparwidth}{1.5cm}
\setstretch{1.1}

% Métadonnées :

\author{agreg.skyost.eu}
\date{\today}

% Titres :

\setcounter{secnumdepth}{3}

\renewcommand{\thechapter}{\Roman{chapter}}
\renewcommand{\thesubsection}{\Roman{subsection}}
\renewcommand{\thesubsubsection}{\arabic{subsubsection}}
\renewcommand{\theparagraph}{\alph{paragraph}}

\titleformat{\chapter}{\huge\bfseries}{\thechapter}{20pt}{\huge\bfseries}
\titleformat*{\section}{\LARGE\bfseries}
\titleformat{\subsection}{\Large\bfseries}{\thesubsection \, - \,}{0pt}{\Large\bfseries}
\titleformat{\subsubsection}{\large\bfseries}{\thesubsubsection. \,}{0pt}{\large\bfseries}
\titleformat{\paragraph}{\bfseries}{\theparagraph. \,}{0pt}{\bfseries}

\setcounter{secnumdepth}{4}

% Table des matières :

\renewcommand{\cftsecleader}{\cftdotfill{\cftdotsep}}
\addtolength{\cftsecnumwidth}{10pt}

% Redéfinition des commandes :

\renewcommand*\thesection{\arabic{section}}
\renewcommand{\ker}{\mathrm{Ker}}

% Nouvelles commandes :

\newcommand{\website}{https://github.com/imbodj/SenCoursDeMaths}

\newcommand{\tr}[1]{\mathstrut ^t #1}
\newcommand{\im}{\mathrm{Im}}
\newcommand{\rang}{\operatorname{rang}}
\newcommand{\trace}{\operatorname{trace}}
\newcommand{\id}{\operatorname{id}}
\newcommand{\stab}{\operatorname{Stab}}
\newcommand{\paren}[1]{\left(#1\right)}
\newcommand{\croch}[1]{\left[ #1 \right]}
\newcommand{\Grdcroch}[1]{\Bigl[ #1 \Bigr]}
\newcommand{\grdcroch}[1]{\bigl[ #1 \bigr]}
\newcommand{\abs}[1]{\left\lvert #1 \right\rvert}
\newcommand{\limi}[3]{\lim_{#1\to #2}#3}
\newcommand{\pinf}{+\infty}
\newcommand{\minf}{-\infty}
%%%%%%%%%%%%%% ENSEMBLES %%%%%%%%%%%%%%%%%
\newcommand{\ensemblenombre}[1]{\mathbb{#1}}
\newcommand{\Nn}{\ensemblenombre{N}}
\newcommand{\Zz}{\ensemblenombre{Z}}
\newcommand{\Qq}{\ensemblenombre{Q}}
\newcommand{\Qqp}{\Qq^+}
\newcommand{\Rr}{\ensemblenombre{R}}
\newcommand{\Cc}{\ensemblenombre{C}}
\newcommand{\Nne}{\Nn^*}
\newcommand{\Zze}{\Zz^*}
\newcommand{\Zzn}{\Zz^-}
\newcommand{\Qqe}{\Qq^*}
\newcommand{\Rre}{\Rr^*}
\newcommand{\Rrp}{\Rr_+}
\newcommand{\Rrm}{\Rr_-}
\newcommand{\Rrep}{\Rr_+^*}
\newcommand{\Rrem}{\Rr_-^*}
\newcommand{\Cce}{\Cc^*}
%%%%%%%%%%%%%%  INTERVALLES %%%%%%%%%%%%%%%%%
\newcommand{\intff}[2]{\left[#1\;,\; #2\right]  }
\newcommand{\intof}[2]{\left]#1 \;, \;#2\right]  }
\newcommand{\intfo}[2]{\left[#1 \;,\; #2\right[  }
\newcommand{\intoo}[2]{\left]#1 \;,\; #2\right[  }

\providecommand{\newpar}{\\[\medskipamount]}

\newcommand{\annexessection}{%
  \newpage%
  \subsection*{Annexes}%
}

\providecommand{\lesson}[3]{%
  \title{#3}%
  \hypersetup{pdftitle={#2 : #3}}%
  \setcounter{section}{\numexpr #2 - 1}%
  \section{#3}%
  \fancyhead[R]{\truncate{0.73\textwidth}{#2 : #3}}%
}

\providecommand{\development}[3]{%
  \title{#3}%
  \hypersetup{pdftitle={#3}}%
  \section*{#3}%
  \fancyhead[R]{\truncate{0.73\textwidth}{#3}}%
}

\providecommand{\sheet}[3]{\development{#1}{#2}{#3}}

\providecommand{\ranking}[1]{%
  \title{Terminale #1}%
  \hypersetup{pdftitle={Terminale #1}}%
  \section*{Terminale #1}%
  \fancyhead[R]{\truncate{0.73\textwidth}{Terminale #1}}%
}

\providecommand{\summary}[1]{%
  \textit{#1}%
  \par%
  \medskip%
}

\tikzset{notestyleraw/.append style={inner sep=0pt, rounded corners=0pt, align=center}}

%\newcommand{\booklink}[1]{\website/bibliographie\##1}
\newcounter{reference}
\newcommand{\previousreference}{}
\providecommand{\reference}[2][]{%
  \needspace{20pt}%
  \notblank{#1}{
    \needspace{20pt}%
    \renewcommand{\previousreference}{#1}%
    \stepcounter{reference}%
    \label{reference-\previousreference-\thereference}%
  }{}%
  \todo[noline]{%
    \protect\vspace{20pt}%
    \protect\par%
    \protect\notblank{#1}{\cite{[\previousreference]}\\}{}%
    \protect\hyperref[reference-\previousreference-\thereference]{p. #2}%
  }%
}

\definecolor{devcolor}{HTML}{00695c}
\providecommand{\dev}[1]{%
  \reversemarginpar%
  \todo[noline]{
    \protect\vspace{20pt}%
    \protect\par%
    \bfseries\color{devcolor}\href{\website/developpements/#1}{[DEV]}
  }%
  \normalmarginpar%
}

% En-têtes :

\pagestyle{fancy}
\fancyhead[L]{\truncate{0.23\textwidth}{\thepage}}
\fancyfoot[C]{\scriptsize \href{\website}{\texttt{https://github.com/imbodj/SenCoursDeMaths}}}

% Couleurs :

\definecolor{property}{HTML}{ffeb3b}
\definecolor{proposition}{HTML}{ffc107}
\definecolor{lemma}{HTML}{ff9800}
\definecolor{theorem}{HTML}{f44336}
\definecolor{corollary}{HTML}{e91e63}
\definecolor{definition}{HTML}{673ab7}
\definecolor{notation}{HTML}{9c27b0}
\definecolor{example}{HTML}{00bcd4}
\definecolor{cexample}{HTML}{795548}
\definecolor{application}{HTML}{009688}
\definecolor{remark}{HTML}{3f51b5}
\definecolor{algorithm}{HTML}{607d8b}
%\definecolor{proof}{HTML}{e1f5fe}
\definecolor{exercice}{HTML}{e1f5fe}

% Théorèmes :

\theoremstyle{definition}
\newtheorem{theorem}{Théorème}

\newtheorem{property}[theorem]{Propriété}
\newtheorem{proposition}[theorem]{Proposition}
\newtheorem{lemma}[theorem]{Activité d'introduction}
\newtheorem{corollary}[theorem]{Conséquence}

\newtheorem{definition}[theorem]{Définition}
\newtheorem{notation}[theorem]{Notation}

\newtheorem{example}[theorem]{Exemple}
\newtheorem{cexample}[theorem]{Contre-exemple}
\newtheorem{application}[theorem]{Application}

\newtheorem{algorithm}[theorem]{Algorithme}
\newtheorem{exercice}[theorem]{Exercice}

\theoremstyle{remark}
\newtheorem{remark}[theorem]{Remarque}

\counterwithin*{theorem}{section}

\newcommand{\applystyletotheorem}[1]{
  \tcolorboxenvironment{#1}{
    enhanced,
    breakable,
    colback=#1!8!white,
    %right=0pt,
    %top=8pt,
    %bottom=8pt,
    boxrule=0pt,
    frame hidden,
    sharp corners,
    enhanced,borderline west={4pt}{0pt}{#1},
    %interior hidden,
    sharp corners,
    after=\par,
  }
}

\applystyletotheorem{property}
\applystyletotheorem{proposition}
\applystyletotheorem{lemma}
\applystyletotheorem{theorem}
\applystyletotheorem{corollary}
\applystyletotheorem{definition}
\applystyletotheorem{notation}
\applystyletotheorem{example}
\applystyletotheorem{cexample}
\applystyletotheorem{application}
\applystyletotheorem{remark}
%\applystyletotheorem{proof}
\applystyletotheorem{algorithm}
\applystyletotheorem{exercice}

% Environnements :

\NewEnviron{whitetabularx}[1]{%
  \renewcommand{\arraystretch}{2.5}
  \colorbox{white}{%
    \begin{tabularx}{\textwidth}{#1}%
      \BODY%
    \end{tabularx}%
  }%
}

% Maths :

\DeclareFontEncoding{FMS}{}{}
\DeclareFontSubstitution{FMS}{futm}{m}{n}
\DeclareFontEncoding{FMX}{}{}
\DeclareFontSubstitution{FMX}{futm}{m}{n}
\DeclareSymbolFont{fouriersymbols}{FMS}{futm}{m}{n}
\DeclareSymbolFont{fourierlargesymbols}{FMX}{futm}{m}{n}
\DeclareMathDelimiter{\VERT}{\mathord}{fouriersymbols}{152}{fourierlargesymbols}{147}

% Code :

\definecolor{greencode}{rgb}{0,0.6,0}
\definecolor{graycode}{rgb}{0.5,0.5,0.5}
\definecolor{mauvecode}{rgb}{0.58,0,0.82}
\definecolor{bluecode}{HTML}{1976d2}
\lstset{
  basicstyle=\footnotesize\ttfamily,
  breakatwhitespace=false,
  breaklines=true,
  %captionpos=b,
  commentstyle=\color{greencode},
  deletekeywords={...},
  escapeinside={\%*}{*)},
  extendedchars=true,
  frame=none,
  keepspaces=true,
  keywordstyle=\color{bluecode},
  language=Python,
  otherkeywords={*,...},
  numbers=left,
  numbersep=5pt,
  numberstyle=\tiny\color{graycode},
  rulecolor=\color{black},
  showspaces=false,
  showstringspaces=false,
  showtabs=false,
  stepnumber=2,
  stringstyle=\color{mauvecode},
  tabsize=2,
  %texcl=true,
  xleftmargin=10pt,
  %title=\lstname
}

\newcommand{\codedirectory}{}
\newcommand{\inputalgorithm}[1]{%
  \begin{algorithm}%
    \strut%
    \lstinputlisting{\codedirectory#1}%
  \end{algorithm}%
}




\begin{document}
	%<*content>
	\development{analysis}{etude-de-fonction-tl}{Etude de fonctions  (TL)}

 \summary{}
 
	\subsection*{ÉLÉMENTS DE SYMÉTRIE}
\begin{exercice}
On considère les fonctions suivantes définies sur leur ensemble
de définition. Montrer que  la courbe de ces fonctions admet l'élément de symétrie indiqué.
\begin{enumerate}
\item $f(x)= \dfrac{3x-5}{1-x}\hspace*{2cm}$  centre de symétrie $ I(1, -3) $

\item $f(x)= \dfrac{x^2+2x+1}{x^2+1}\hspace*{1cm}$ centre de symétrie $ I(0, 1) $

\item $f(x)=3x^2 - 6x + 7\hspace*{1cm}$ axe de symétrie $ x=1 $

\item $f(x)=\dfrac{x^2 - x -2}{x^2 - x + 1}\hspace*{1cm}$ axe de symétrie $ x=\dfrac{1}{2} $
\end{enumerate}
\end{exercice}
\subsection*{ETUDE DE FONCTION}
\begin{exercice}
 Soit la fonction  $f$ d\'efinie par : $f(x)=x^3-3x-2 $ 
\begin{enumerate}
 \item D\'eterminer $\mathscr{D}_f$.
 \item Calculer les limites aux bornes de $\mathscr{D}_f$
 \item Calculer $f'(x)$ et dresser le tableau de variation de $f$.
 \item Montrer que le point $S(0;-2)$ est un centre de sym\'etrie \`a la courbe  $\mathscr{C}_f$ de $f$ .
\item D\'eterminer les abscisses des points d'intersection de la courbe $\mathscr{C}_f$ avec l'axe des abscisses.
  \item D\'eterminer une \'equation de la tangente \`a $\mathscr{C}_f$ en chacun de ces points .
  \item Construire les tangentes , puis la courbe $\mathscr{C}_f$ dans le rep\`ere
\end{enumerate}
\end{exercice}
\begin{exercice}
 Soit la fonction $f$ d\'efinie par $f(x)=\dfrac{x^2-3x+6}{x-1}$
\begin{enumerate}
 \item Déterminer le domaine de d\'efinition $\mathscr{D}_f$ de $f$ .
 \item Calculer les limites aux bornes de $\mathscr{D}_f$ et pr\'eciser les asymptotes \'eventuelles. 
 \item Déterminer 3 r\'eels $a,b,~$ et $c$ tels que $f(x)=ax+b+\dfrac{c}{x-1}$. 
 
 En d\'eduire que la droite 
 $y=ax+b$ est une asymptotes \`a la courbe de $f$ .
 \item Calculer $f'(x)$ et dresser son tableau de variation .
 \item Montrer que le point $I(1;~-1)$ est un centre de sym\'etrie \`a la courbe de $f$ .
 \item Etudier la position de la courbe de $f$, par rapport \`a l'asymptote $y=ax+b$ .
 \item Tracer la courbe $(\mathscr{C}_f)$ de $f$ et les asymptotes dans un rep\`ere orthonorm\'e. 
\end{enumerate}
\end{exercice}
\begin{exercice}
Soit la fonction d\'efinie par :$f(x)=\sqrt{2x+5}$
\begin{enumerate}
 \item D\'eterminer le domaine de d\'efinition $\mathscr{D}_f$ de $f$, et calculer les limites aux bornes de 
 $\mathscr{D}_f$. 
 \item Calculer $f'(x)$ pour $x\in \big]-\frac{5}{2};~+\infty\big[$
 \item Repr\'esenter graphiquement $f$ dans un rep\`ere $\oij$.
\end{enumerate}
\end{exercice}
\begin{exercice} 
Soit la fonction num\'erique $ f $ de la variable r\'eelle $x$ d\'efinie 

$f(x)=\dfrac{x^2+x-2}{x+1}$.\medskip

On appelle $\mathscr{C}_f$, la repr\'esentation graphique de $f$ dans un rep\`ere orthonorm\'e; unit\'e graphique: $1cm$
\begin{enumerate}
 \item D\'eterminer l'ensemble de d\'efinition $\mathscr{D}_f$ de $f$ ; puis \'etudier les limites aux bornes de 
 $\mathscr{D}_f$ .
 \item Montrer que la droite  $(\mathcal{D})$ d'\'equation $y=x$ est une asymptote oblique \`a $\mathscr{C}_f$, et 
 pr\'eciser l'autre asymptote.
 \item Etudier la position de $\mathscr{C}_f$ par rapport  \`a $(\mathcal{D})$ .
 \item Montrer que le point $S(-1;~-1)$ est un centre de sym\'etrie de $(\mathscr{C}_f)$.
 \item D\'eterminer pour tout $x\in \mathscr{D}_f$, ~$f'(x)$, puis \'etablir le tableau de variation de $f$ .
 \item Montrer que $(\mathscr{C}_f)$ rencontre l'axe des abscisses aux points $A$ et $B$ d'abscisses respectives\;
 $-2$~et~ $1$.
 \item Donner une \'equation de la tangente \`a $\mathscr{C}_f$ en $A$, puis une \'equation de la tangente \`a 
 $(\mathscr{C}_f)$ .
 \item Construire $\mathscr{C}_f$ , les asymptotes et les tangentes.
\end{enumerate}
\end{exercice}
\begin{exercice}
 Le plan est muni du rep\`ere orthonorm\'e $\oij$. 
 
 Soit $f$ la fonctions d\'efinies par : 
 $f(x)=\dfrac{2x-1}{x-1}$.\\
On d\'esigne par $(\mathscr{C})$  la repr\'esentation graphique  de $f$.
\begin{enumerate}
 \item Étudier les limites aux bornes du domaine $ D $ de  $f$.
 \item En déduire les asymptotes de $(\mathscr{C})$.
 \item Calculer $ f'(x) $ puis donner son signe sur $ D $.
 \item Etablir le tableau de variation de  $f$ puis construire $(\mathscr{C})$ dans le repère.
 \item R\'esoudre graphiquement:
 \begin{enumerate}
\item $f(x)=-1$
\item $f(x)=4$
\item $f(x)\leq  3$
\item $f(x)>-3$
\end{enumerate}
\end{enumerate}
\end{exercice}

\begin{exercice}
On donne $f(x)=\dfrac{2x^2+3}{x^2-1}$
\begin{enumerate}
\item Déterminer l'ensemble de définition de $ f $
\item Etudier la parité de $ f$.
\item Dresser le tableau de $ f $ sur $ [0,\; + \infty[ $.
\item Tracer sa courbe repr\'esentative.
\end{enumerate}
\end{exercice}
\begin{exercice}
Monsieur Ahmadou est le gestionnaire de l'entreprise où vous avez postulé pour un emploi. M. Ahmadou vous
explique, lors de l'interview, que le bénéfice en fonction du nombre (en milliers) de chaussures est défini par :

\medskip
 $ b(x)=x^3-x^2-5x+5 $\; avec \; $ 1<x<3 $

\medskip
M. Ahmadou souhaite maîtriser l'évolution de ce bénéfice, pour cela il vous propose de l'aider à:

\medskip
\begin{enumerate}
\item  déterminer le nombre de chaussures dont le bénéfice est nul.

\item  déterminer  l'intervalle de valeurs du nombre de chaussures menant à une perte.

\item déterminer  l'intervalle de valeurs du nombre de chaussures menant à un gain positif.
\end{enumerate}
\end{exercice}

	%</content>
\end{document}
