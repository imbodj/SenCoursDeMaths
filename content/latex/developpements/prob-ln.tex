\input{../common}

\begin{document}
	%<*content>
	\development{algebra}{prob-ln}{Etude de fonctions ln}

 \summary{Problèmes sur le logarithme népérien}
 
	
\textbf{PROBLÈME 1} 


\textbf{Partie A}

\medskip
Soit $ g $ la fonction définie par:
$\quad  g(x)=2x-2-x\ln(x) $
\begin{enumerate}
\item Étudier  les variations de $ g $ puis dresser son tableau de variations.
\item Montrer que l'équation $ g(x)=0 $ admet deux solutions $ \alpha $ et $ \beta $  telles que: $ \alpha\in\intoo{0}{\mathrm{e}} $ et  $ \beta \in~\intoo{\mathrm{e}}{\pinf} $.

 Préciser la valeur exacte de $ \alpha $  et établir que $ 4,5<\beta <5 $.
\item  En déduire le signe de $ g(x)$ suivant les valeurs de $ x $.
\end{enumerate}

\textbf{Partie B}


\medskip
On considère la fonction $ f $ définie   sur $ \intoo{0}{\pinf} $ par:    $$\left \{\begin{array}{l} f(x)=\dfrac{\paren{\ln(x)}^{2}}{x-1}~~~ \textrm{si}~~ x\neq 1 ~\\ f(1)=0\end{array}\right.$$
 \begin{enumerate}
\item 
\begin{enumerate}
\item Étudier la continuité de $ f $  en $ 1 $.
\item Étudier la dérivabilité de $ f $  en $ 1 $.
\end{enumerate}
\item
\begin{enumerate}
\item Montrer que pour $ x\neq1 $ et $ x>0 $: $\; f'(x)=\dfrac{\ln(x)}{x(x-1)^{2}}\times g(x) $.
\item  Dresser le tableau de variations de $ f $.
\end{enumerate}
\item Montrer que $ f(\beta)=\dfrac{4(\beta-1)}{\beta^{2}} $.
\item Donner une équation de la tangente à la courbe $\mathscr{C}$  de $ f $ au point d'abscisse $ 1 $.
\item Soit $ h $ la restriction de $ f $ à l'intervalle $ \intof{0}{1} $. 
\begin{enumerate}
\item Montrer que $ h $ admet une bijection réciproque $ h^{-1} $ puis établir le tableau de variation de  $ h^{-1} $.

\end{enumerate}
\item Tracer $\mathscr{C}$ et celle de $ h^{-1} $  dans le même repère.
\end{enumerate}


\vspace{1cm}

\textbf{PROBLÈME 2}

On considère la fonction $ f $ définie par: \\   $f(x)=\left \{\begin{array}{l} \dfrac{x\ln(x)}{x+1}~~~ \textrm{si}~~ x> 0 ~\\[0.5cm] \ln(1-x)~~~ \textrm{si}~~ x\leq 0 \end{array}\right.$


\medskip
\underline{\textbf{Partie A}}

\medskip
 \begin{enumerate}
\item Montrer que $ \text{D}_{f} =\Rr$ et calculer les limites  de $ f $ aux bornes  de $ \text{D}_{f} $ .
\begin{enumerate}
\item Étudier la continuité de $ f $  en $ 0 $.
\item Étudier la dérivabilité de $ f $  en $ 0 $.\\ Interpréter graphiquement les résultats obtenus.
\end{enumerate}
 \item Étudier les branches infinies  de  $\mathscr{C}_{f}$ 

\end{enumerate}


\medskip
\underline{\textbf{Partie B}}


\medskip


\begin{enumerate}
\item Soit $ h(x)=\ln(x)+x+1 $ 
\begin{enumerate}
\item Dresser le tableau de variations de $ h $.
\item  Montrer que l'équation $ h(x)=0 $ admet une unique solution $ \alpha $   et montrer  que\\ $ 0,27<\alpha <0,28 $.
\item  En déduire le signe de $ h(x)$.
\end{enumerate}
\item 
\begin{enumerate}
\item Montrer que  $ f'(x)=\dfrac{h(x)}{(x+1)^{2}} $ pour $ x>0 $ en déduire le signe de $ f'(x) $ .
\item Calculer $ f'(x)$ pour $ x<0 $
\item  Montrer  que $ f(\alpha)=-\alpha $.\\ Établir le tableau de variations de $ f $.
\item Tracer la courbe  $\mathscr{C}_{f}$   dans un RON.
\end{enumerate}
\end{enumerate}

\textbf{Partie C}

Soit $ g $  la restriction de la fonction  $ f $ à l'intervalle $ I=\intfo{\alpha}{\pinf}$.
\begin{enumerate}
\item Montrer que $ g $  réalise une bijection de $ I $ vers un intervalle $ J $ à déterminer.
\item Déterminer une équation de la tangente à $\mathscr{C}_{g^{-1}}$ au point d'abscisse $ 0 $.
\item Tracer $\mathscr{C}_{g^{-1}}$  dans le repère précèdent.
\end{enumerate}








	%</content>
\end{document}
