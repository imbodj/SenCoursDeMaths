\documentclass[12pt, a4paper]{report}

% LuaLaTeX :

\RequirePackage{iftex}
\RequireLuaTeX

% Packages :

\usepackage[french]{babel}
%\usepackage[utf8]{inputenc}
%\usepackage[T1]{fontenc}
\usepackage[pdfencoding=auto, pdfauthor={Hugo Delaunay}, pdfsubject={Mathématiques}, pdfcreator={agreg.skyost.eu}]{hyperref}
\usepackage{amsmath}
\usepackage{amsthm}
%\usepackage{amssymb}
\usepackage{stmaryrd}
\usepackage{tikz}
\usepackage{tkz-euclide}
\usepackage{fontspec}
\defaultfontfeatures[Erewhon]{FontFace = {bx}{n}{Erewhon-Bold.otf}}
\usepackage{fourier-otf}
\usepackage[nobottomtitles*]{titlesec}
\usepackage{fancyhdr}
\usepackage{listings}
\usepackage{catchfilebetweentags}
\usepackage[french, capitalise, noabbrev]{cleveref}
\usepackage[fit, breakall]{truncate}
\usepackage[top=2.5cm, right=2cm, bottom=2.5cm, left=2cm]{geometry}
\usepackage{enumitem}
\usepackage{tocloft}
\usepackage{microtype}
%\usepackage{mdframed}
%\usepackage{thmtools}
\usepackage{xcolor}
\usepackage{tabularx}
\usepackage{xltabular}
\usepackage{aligned-overset}
\usepackage[subpreambles=true]{standalone}
\usepackage{environ}
\usepackage[normalem]{ulem}
\usepackage{etoolbox}
\usepackage{setspace}
\usepackage[bibstyle=reading, citestyle=draft]{biblatex}
\usepackage{xpatch}
\usepackage[many, breakable]{tcolorbox}
\usepackage[backgroundcolor=white, bordercolor=white, textsize=scriptsize]{todonotes}
\usepackage{luacode}
\usepackage{float}
\usepackage{needspace}
\everymath{\displaystyle}

% Police :

\setmathfont{Erewhon Math}

% Tikz :

\usetikzlibrary{calc}
\usetikzlibrary{3d}

% Longueurs :

\setlength{\parindent}{0pt}
\setlength{\headheight}{15pt}
\setlength{\fboxsep}{0pt}
\titlespacing*{\chapter}{0pt}{-20pt}{10pt}
\setlength{\marginparwidth}{1.5cm}
\setstretch{1.1}

% Métadonnées :

\author{agreg.skyost.eu}
\date{\today}

% Titres :

\setcounter{secnumdepth}{3}

\renewcommand{\thechapter}{\Roman{chapter}}
\renewcommand{\thesubsection}{\Roman{subsection}}
\renewcommand{\thesubsubsection}{\arabic{subsubsection}}
\renewcommand{\theparagraph}{\alph{paragraph}}

\titleformat{\chapter}{\huge\bfseries}{\thechapter}{20pt}{\huge\bfseries}
\titleformat*{\section}{\LARGE\bfseries}
\titleformat{\subsection}{\Large\bfseries}{\thesubsection \, - \,}{0pt}{\Large\bfseries}
\titleformat{\subsubsection}{\large\bfseries}{\thesubsubsection. \,}{0pt}{\large\bfseries}
\titleformat{\paragraph}{\bfseries}{\theparagraph. \,}{0pt}{\bfseries}

\setcounter{secnumdepth}{4}

% Table des matières :

\renewcommand{\cftsecleader}{\cftdotfill{\cftdotsep}}
\addtolength{\cftsecnumwidth}{10pt}

% Redéfinition des commandes :

\renewcommand*\thesection{\arabic{section}}
\renewcommand{\ker}{\mathrm{Ker}}

% Nouvelles commandes :

\newcommand{\website}{https://github.com/imbodj/SenCoursDeMaths}

\newcommand{\tr}[1]{\mathstrut ^t #1}
\newcommand{\im}{\mathrm{Im}}
\newcommand{\rang}{\operatorname{rang}}
\newcommand{\trace}{\operatorname{trace}}
\newcommand{\id}{\operatorname{id}}
\newcommand{\stab}{\operatorname{Stab}}
\newcommand{\paren}[1]{\left(#1\right)}
\newcommand{\croch}[1]{\left[ #1 \right]}
\newcommand{\Grdcroch}[1]{\Bigl[ #1 \Bigr]}
\newcommand{\grdcroch}[1]{\bigl[ #1 \bigr]}
\newcommand{\abs}[1]{\left\lvert #1 \right\rvert}
\newcommand{\limi}[3]{\lim_{#1\to #2}#3}
\newcommand{\pinf}{+\infty}
\newcommand{\minf}{-\infty}
%%%%%%%%%%%%%% ENSEMBLES %%%%%%%%%%%%%%%%%
\newcommand{\ensemblenombre}[1]{\mathbb{#1}}
\newcommand{\Nn}{\ensemblenombre{N}}
\newcommand{\Zz}{\ensemblenombre{Z}}
\newcommand{\Qq}{\ensemblenombre{Q}}
\newcommand{\Qqp}{\Qq^+}
\newcommand{\Rr}{\ensemblenombre{R}}
\newcommand{\Cc}{\ensemblenombre{C}}
\newcommand{\Nne}{\Nn^*}
\newcommand{\Zze}{\Zz^*}
\newcommand{\Zzn}{\Zz^-}
\newcommand{\Qqe}{\Qq^*}
\newcommand{\Rre}{\Rr^*}
\newcommand{\Rrp}{\Rr_+}
\newcommand{\Rrm}{\Rr_-}
\newcommand{\Rrep}{\Rr_+^*}
\newcommand{\Rrem}{\Rr_-^*}
\newcommand{\Cce}{\Cc^*}
%%%%%%%%%%%%%%  INTERVALLES %%%%%%%%%%%%%%%%%
\newcommand{\intff}[2]{\left[#1\;,\; #2\right]  }
\newcommand{\intof}[2]{\left]#1 \;, \;#2\right]  }
\newcommand{\intfo}[2]{\left[#1 \;,\; #2\right[  }
\newcommand{\intoo}[2]{\left]#1 \;,\; #2\right[  }

\providecommand{\newpar}{\\[\medskipamount]}

\newcommand{\annexessection}{%
  \newpage%
  \subsection*{Annexes}%
}

\providecommand{\lesson}[3]{%
  \title{#3}%
  \hypersetup{pdftitle={#2 : #3}}%
  \setcounter{section}{\numexpr #2 - 1}%
  \section{#3}%
  \fancyhead[R]{\truncate{0.73\textwidth}{#2 : #3}}%
}

\providecommand{\development}[3]{%
  \title{#3}%
  \hypersetup{pdftitle={#3}}%
  \section*{#3}%
  \fancyhead[R]{\truncate{0.73\textwidth}{#3}}%
}

\providecommand{\sheet}[3]{\development{#1}{#2}{#3}}

\providecommand{\ranking}[1]{%
  \title{Terminale #1}%
  \hypersetup{pdftitle={Terminale #1}}%
  \section*{Terminale #1}%
  \fancyhead[R]{\truncate{0.73\textwidth}{Terminale #1}}%
}

\providecommand{\summary}[1]{%
  \textit{#1}%
  \par%
  \medskip%
}

\tikzset{notestyleraw/.append style={inner sep=0pt, rounded corners=0pt, align=center}}

%\newcommand{\booklink}[1]{\website/bibliographie\##1}
\newcounter{reference}
\newcommand{\previousreference}{}
\providecommand{\reference}[2][]{%
  \needspace{20pt}%
  \notblank{#1}{
    \needspace{20pt}%
    \renewcommand{\previousreference}{#1}%
    \stepcounter{reference}%
    \label{reference-\previousreference-\thereference}%
  }{}%
  \todo[noline]{%
    \protect\vspace{20pt}%
    \protect\par%
    \protect\notblank{#1}{\cite{[\previousreference]}\\}{}%
    \protect\hyperref[reference-\previousreference-\thereference]{p. #2}%
  }%
}

\definecolor{devcolor}{HTML}{00695c}
\providecommand{\dev}[1]{%
  \reversemarginpar%
  \todo[noline]{
    \protect\vspace{20pt}%
    \protect\par%
    \bfseries\color{devcolor}\href{\website/developpements/#1}{[DEV]}
  }%
  \normalmarginpar%
}

% En-têtes :

\pagestyle{fancy}
\fancyhead[L]{\truncate{0.23\textwidth}{\thepage}}
\fancyfoot[C]{\scriptsize \href{\website}{\texttt{https://github.com/imbodj/SenCoursDeMaths}}}

% Couleurs :

\definecolor{property}{HTML}{ffeb3b}
\definecolor{proposition}{HTML}{ffc107}
\definecolor{lemma}{HTML}{ff9800}
\definecolor{theorem}{HTML}{f44336}
\definecolor{corollary}{HTML}{e91e63}
\definecolor{definition}{HTML}{673ab7}
\definecolor{notation}{HTML}{9c27b0}
\definecolor{example}{HTML}{00bcd4}
\definecolor{cexample}{HTML}{795548}
\definecolor{application}{HTML}{009688}
\definecolor{remark}{HTML}{3f51b5}
\definecolor{algorithm}{HTML}{607d8b}
%\definecolor{proof}{HTML}{e1f5fe}
\definecolor{exercice}{HTML}{e1f5fe}

% Théorèmes :

\theoremstyle{definition}
\newtheorem{theorem}{Théorème}

\newtheorem{property}[theorem]{Propriété}
\newtheorem{proposition}[theorem]{Proposition}
\newtheorem{lemma}[theorem]{Activité d'introduction}
\newtheorem{corollary}[theorem]{Conséquence}

\newtheorem{definition}[theorem]{Définition}
\newtheorem{notation}[theorem]{Notation}

\newtheorem{example}[theorem]{Exemple}
\newtheorem{cexample}[theorem]{Contre-exemple}
\newtheorem{application}[theorem]{Application}

\newtheorem{algorithm}[theorem]{Algorithme}
\newtheorem{exercice}[theorem]{Exercice}

\theoremstyle{remark}
\newtheorem{remark}[theorem]{Remarque}

\counterwithin*{theorem}{section}

\newcommand{\applystyletotheorem}[1]{
  \tcolorboxenvironment{#1}{
    enhanced,
    breakable,
    colback=#1!8!white,
    %right=0pt,
    %top=8pt,
    %bottom=8pt,
    boxrule=0pt,
    frame hidden,
    sharp corners,
    enhanced,borderline west={4pt}{0pt}{#1},
    %interior hidden,
    sharp corners,
    after=\par,
  }
}

\applystyletotheorem{property}
\applystyletotheorem{proposition}
\applystyletotheorem{lemma}
\applystyletotheorem{theorem}
\applystyletotheorem{corollary}
\applystyletotheorem{definition}
\applystyletotheorem{notation}
\applystyletotheorem{example}
\applystyletotheorem{cexample}
\applystyletotheorem{application}
\applystyletotheorem{remark}
%\applystyletotheorem{proof}
\applystyletotheorem{algorithm}
\applystyletotheorem{exercice}

% Environnements :

\NewEnviron{whitetabularx}[1]{%
  \renewcommand{\arraystretch}{2.5}
  \colorbox{white}{%
    \begin{tabularx}{\textwidth}{#1}%
      \BODY%
    \end{tabularx}%
  }%
}

% Maths :

\DeclareFontEncoding{FMS}{}{}
\DeclareFontSubstitution{FMS}{futm}{m}{n}
\DeclareFontEncoding{FMX}{}{}
\DeclareFontSubstitution{FMX}{futm}{m}{n}
\DeclareSymbolFont{fouriersymbols}{FMS}{futm}{m}{n}
\DeclareSymbolFont{fourierlargesymbols}{FMX}{futm}{m}{n}
\DeclareMathDelimiter{\VERT}{\mathord}{fouriersymbols}{152}{fourierlargesymbols}{147}

% Code :

\definecolor{greencode}{rgb}{0,0.6,0}
\definecolor{graycode}{rgb}{0.5,0.5,0.5}
\definecolor{mauvecode}{rgb}{0.58,0,0.82}
\definecolor{bluecode}{HTML}{1976d2}
\lstset{
  basicstyle=\footnotesize\ttfamily,
  breakatwhitespace=false,
  breaklines=true,
  %captionpos=b,
  commentstyle=\color{greencode},
  deletekeywords={...},
  escapeinside={\%*}{*)},
  extendedchars=true,
  frame=none,
  keepspaces=true,
  keywordstyle=\color{bluecode},
  language=Python,
  otherkeywords={*,...},
  numbers=left,
  numbersep=5pt,
  numberstyle=\tiny\color{graycode},
  rulecolor=\color{black},
  showspaces=false,
  showstringspaces=false,
  showtabs=false,
  stepnumber=2,
  stringstyle=\color{mauvecode},
  tabsize=2,
  %texcl=true,
  xleftmargin=10pt,
  %title=\lstname
}

\newcommand{\codedirectory}{}
\newcommand{\inputalgorithm}[1]{%
  \begin{algorithm}%
    \strut%
    \lstinputlisting{\codedirectory#1}%
  \end{algorithm}%
}




\begin{document}
  %<*content>
  \development{algebra}{decomposition-polaire}{Décomposition polaire}

  \summary{On montre que toute matrice $M \in \mathrm{GL}_n(\mathbb{R})$ peut s'écrire de manière unique $M = OS$ avec $O \in \mathcal{O}_n(\mathbb{R})$ et $S \in \mathcal{S}_n^{++}(\mathbb{R})$, et que l'application $(O, S) \mapsto M$ est un homéomorphisme.}

  \begin{lemma}
    \label{decomposition-polaire-1}
    Soit $S \in \mathcal{S}_n(\mathbb{R})$. Alors $S \in \mathcal{S}_n^{++}(\mathbb{R})$ si et seulement si toutes ses valeurs propres sont strictement positives.
  \end{lemma}

  \begin{proof}
    Par le théorème spectral, on peut écrire $S = \tr P \operatorname{Diag}(\lambda_1, \dots, \lambda_n) P$ avec $P \in \mathcal{O}_n(\mathbb{R})$. Si on suppose $\lambda_1, \dots, \lambda_n > 0$, on a $\forall x \neq 0$,
    \[ \tr x S x = \tr (Px) \operatorname{Diag}(\lambda_1, \dots, \lambda_n) (Px) > 0 \text{ car } \operatorname{Diag}(\lambda_1, \dots, \lambda_n) \in \mathcal{S}_n^{++}(\mathbb{R}) \]
    d'où le résultat.
    \newpar
    Réciproquement, on suppose $\forall x \neq 0$, $\tr x S x > 0$. Avec $x = \tr P e_1$ (où $e_1$ désigne le vecteur dont la première coordonnée vaut $1$ et les autres sont nulles),
    \[ \tr x S x = \tr (Px) \operatorname{Diag}(\lambda_1, \dots, \lambda_n) (Px) = \tr e_1 D e_1 = \lambda_1 > 0 \]
    Et on peut faire de même pour montrer que $\forall i \in \llbracket 1, n \rrbracket$, $\lambda_i > 0$.
  \end{proof}

  \begin{lemma}
    \label{decomposition-polaire-2}
    $\mathcal{S}_n^+(\mathbb{R})$ est un fermé de $\mathcal{M}_n(\mathbb{R})$ et $\mathrm{GL}_n(\mathbb{R}) \, \cap \, \mathcal{S}_n^{+}(\mathbb{R}) \subseteq \mathcal{S}_n^{++}(\mathbb{R})$.
  \end{lemma}

  \begin{proof}
    Pour la première assertion, il suffit de constater que
    \[ \mathcal{S}_n^+(\mathbb{R}) = \{ M \in \mathcal{M}_n(\mathbb{R}) \mid \tr M = M \} \, \cap \, \left( \bigcap_{x \in \mathbb{R}^n} \{ M \in \mathcal{M}_n(\mathbb{R}) \mid \tr x M x \geq 0 \} \right) \]
    qui est une intersection de fermés (par image réciproque). Maintenant, si $M \in \mathrm{GL}_n(\mathbb{R}) \, \cap \, \mathcal{S}_n^{+}(\mathbb{R})$, alors $M$ est diagonalisable avec des valeurs propres positives ou nulles (par le théorème spectral). Mais comme $\det(M) \neq 0$, toutes les valeurs propres de $M$ sont strictement positives. Donc par le \cref{decomposition-polaire-1}, $M \in \mathcal{S}_n^{++}(\mathbb{R})$.
  \end{proof}

  \reference[C-G]{376}

  \begin{theorem}[Décomposition polaire]
    L'application
    \[ \mu :
    \begin{array}{ccc}
      \mathcal{O}_n(\mathbb{R}) \times \mathcal{S}_n^{++}(\mathbb{R}) &\rightarrow& \mathrm{GL}_n(\mathbb{R}) \\
      (O, S) &\mapsto& OS
    \end{array}
    \]
    est un homéomorphisme.
  \end{theorem}

  \begin{proof}
    Montrer qu'une application est un homéomorphisme se fait en $4$ étapes : on montre qu'elle est continue, injective, surjective, et que la réciproque est elle aussi continue.
    \begin{itemize}
      \item \uline{L'application est bien définie et continue :} Si $O \in \mathcal{O}_n(\mathbb{R})$ et $S \in \mathcal{S}_n^{++}(\mathbb{R})$, alors $OS \in \mathrm{GL}_n(\mathbb{R})$. De plus, $\mu$ est continue en tant que restriction de la multiplication matricielle.
      \item \uline{L'application est surjective :} Soit $M \in \mathrm{GL}_n(\mathbb{R})$. Si $x \neq 0$, on a
      \[ \tr x (\tr M M) x = \tr (Mx) (Mx) = \Vert Mx \Vert_2^2 > 0 \]
      En particulier, $\tr M M \in \mathcal{S}_n^{++}(\mathbb{R})$. Par le théorème spectral, il existe $P \in \mathcal{O}_n(\mathbb{R})$ et $\lambda_1, \dots, \lambda_n > 0$ tels que $\tr M M = P \operatorname{Diag}(\lambda_1, \dots, \lambda_n) P^{-1}$. On pose alors
      \[ D = \operatorname{Diag} \left(\sqrt{\lambda_1}, \dots, \sqrt{\lambda_n} \right) \text{ et } S = P D P^{-1} \]
      de sorte que $S^2 = \tr M M$. Mais de plus,
      \[ \tr S = \tr P^{-1} \tr D \tr P = S \implies S \in \mathcal{S}_n(\mathbb{R}) \]
      et par le \cref{decomposition-polaire-1},
      \[ \forall i \in \llbracket 1, n \rrbracket, \, \sqrt{\lambda_i} > 0 \implies S \in \mathcal{S}_n^{++}(\mathbb{R}) \]
      On pose donc $O = MS^{-1}$ (ie. $M = OS$), et on a
      \[ \tr O O = \tr (MS^{-1}) MS^{-1} = \tr S^{-1} \tr M M S^{-1} = S^{-1} S^2 S^{-1} = I_n \implies O \in \mathcal{O}_n(\mathbb{R}) \]
      Donc $\mu(O, S) = M$ et $\mu$ est surjective.
      \item \uline{L'application est injective :} Soit $M = OS \in \mathrm{GL}_n(\mathbb{R})$ (avec $O$ et $S$ comme précédemment). Soit $M = O'S'$ une autre décomposition polaire de $M$. Alors il vient,
      \[ S^2 = \tr M M = \tr (O'S') O'S' = \tr S' \tr O' O' S' = S'^{2} \]
      Soit $Q$ un polynôme tel que $\forall i \in \llbracket 1, n \rrbracket$, $Q(\lambda_i) = \sqrt{\lambda_i}$ (les polynômes d'interpolation de Lagrange conviennent parfaitement). Alors,
      \[\ S = PD \tr P = PQ \left(D^2 \right) \tr P = Q \left(PD^2 \tr P \right) = Q \left(\tr M M \right) = Q \left(S^2 \right) = Q \left(S'^2 \right) \]
      Mais $S'$ commute avec $S'^2$, donc avec $S = Q \left(S'^2 \right)$. En particulier, $S$ et $S'$ sont codiagonalisables, il existe $P_0 \in \mathrm{GL}_n(\mathbb{R})$ et $\mu_1, \dots, \mu_n, \mu'_1, \dots, \mu'_n \in \mathbb{R}$ tels que
      \[ S = P_0 \operatorname{Diag}(\mu_1, \dots, \mu_n) P_0^{-1} \text{ et } S' = P_0 \operatorname{Diag} \left (\mu'_1, \dots, \mu'_n \right) P_0^{-1} \]
      d'où :
      \begin{align*}
        S^2 = S'^2 & \implies P_0 \operatorname{Diag} \left(\mu^2_1, \dots, \mu^2_n \right) P_0^{-1} = P_0 \operatorname{Diag} \left (\mu'^2_1, \dots, \mu'^2_n \right) P_0^{-1} \\
        & \implies \mu^2_i = \mu'^2_i \qquad \forall i \in \llbracket 1, n \rrbracket \\
        & \implies \mu_i = \mu'_i \qquad \forall i \in \llbracket 1, n \rrbracket \text{ car } \forall i \in \llbracket 1, n \rrbracket, \, \mu_i > 0 \\
        & \implies S = S'
      \end{align*}
      Ainsi, $O = MS^{-1} = MS'^{-1} = O'$. Donc $\mu$ est injective.
      \item \uline{L'application inverse est continue :} Soit $(M_p) \in \mathrm{GL}_n(\mathbb{R})^{\mathbb{N}}$ qui converge vers $M \in \mathrm{GL}_n(\mathbb{R})$. Il s'agit de montrer que la suite $\left (\mu^{-1} \left (M_p \right) \right) = (O_p, S_p)$ converge vers $\mu^{-1}(M) = (O, S)$. Comme $\mathcal{O}_n(\mathbb{R})$ est compact, il existe $\varphi : \mathbb{N} \rightarrow \mathbb{N}$ strictement croissante telle que la suite extraite $(O_{\varphi(p)})$ converge vers une valeur d'adhérence $\overline{O} \in \mathcal{O}_n(\mathbb{R})$. Ainsi, la suite $(S_{\varphi(p)})$ converge vers $\overline{S} = \overline{O}^{-1} M$.
      \newpar
      Mais, $\overline{S} = \overline{O}^{-1} M \in \mathrm{GL}_n(\mathbb{R}) \, \cap \, \overline{\mathcal{S}_n^{++}(\mathbb{R})}$. Donc par le \cref{decomposition-polaire-1},
      \[ \overline{S} \in \mathrm{GL}_n(\mathbb{R}) \, \cap \, \mathcal{S}_n^{+}(\mathbb{R}) \]
      et par le \cref{decomposition-polaire-2},
      \[ \overline{S} \in \mathcal{S}_n^{++}(\mathbb{R}) \]
      On a $M = \overline{O} \overline{S}$, d'où, par unicité de la décomposition polaire, $\overline{O} = O$ et $\overline{S} = S$.
    \end{itemize}
  \end{proof}

  \begin{remark}
    La preuve vaut encore dans le cas complexe (pour le groupe unitaire et les matrices hermitiennes).
  \end{remark}
  %</content>
\end{document}
